
\documentclass{createspace}


\newcommand{\N}{\mathbb N}
\newcommand{\Z}{\mathbb Z}
\newcommand{\Q}{\mathbb Q}
\newcommand{\R}{\mathbb R}
\newcommand{\C}{\mathbb C}

\newcommand{\shrap}{\mathbin{\#}}
\DeclareMathOperator*{\bigshrap}{\#}

\DeclareMathOperator{\ColoringGroup}{Col}

\newcommand{\bracket}[1]{\left\langle{#1}\right\rangle}

\newcommand{\alexander}{\Delta}
\newcommand{\conway}{\nabla}
\newcommand{\jones}{V}
% span?

\DeclareMathOperator{\braid}{b}
\DeclareMathOperator{\bridge}{br}
\DeclareMathOperator{\crossing}{cr}
\DeclareMathOperator{\genus}{g}
\DeclareMathOperator{\linking}{lk}
\DeclareMathOperator{\ropelength}{len}
\DeclareMathOperator{\sign}{sgn}
\DeclareMathOperator{\stick}{s}
\DeclareMathOperator{\trace}{tr}
\DeclareMathOperator{\unknotting}{u}
\DeclareMathOperator{\volume}{vol}
\DeclareMathOperator{\writhe}{wr}

\newcommand{\inversedcurvearrowright}{\rotatebox[origin=c]{180}{$\curvearrowleft$}}
\newcommand{\inversedcurvearrowleft}{\rotatebox[origin=c]{180}{$\curvearrowright$}}



\usepackage{comment}
\includecomment{comment}

% forced by -output-directory option in makefile
% see https://tex.stackexchange.com/a/564296
\usepackage{etoolbox}
\makeatletter
\patchcmd\imki@putindex
  {\imki@exec{\imki@program \imki@options #1.idx}}
  {\imki@exec{cd ../build;\imki@program\imki@options#1.idx}}
  {\message{Patch succeeded in imki@putindex}}
  {\errmessage{Patch failed in imki@putindex}}
\makeatother
\makeindex[title=Skorowidz]
\makeindex[name=persons,title=Indeks osób]

\usepackage{enumitem}
\usepackage{booktabs}
\usepackage{longtable}
\usepackage[table]{xcolor}
\usepackage[colorinlistoftodos,prependcaption]{todonotes}
\usepackage{tikz}
\usetikzlibrary{arrows.meta}
\usetikzlibrary{decorations.markings}
\usetikzlibrary{decorations.pathreplacing}
\usetikzlibrary{knots}
\colorlet{darkblue}{blue!80!black}
\definecolor{diagramfiller}{HTML}{5784BA}
\definecolor{first_colour}{HTML}{A15D98}

\tikzset{
    ->-/.style={decoration={markings, mark=at position .5 with {\arrow{>}}},postaction={decorate}},
    -<-/.style={decoration={markings, mark=at position .5 with {\arrow{<}}},postaction={decorate}},
    TIKZ_ARCH/.style ={
        draw=black,
        line join=miter,
        line cap=butt,
        miter limit=4.00,
        line width=0.2 mm
    },
}

\newcommand{\SmallUnknot} {\begin{tikzpicture}[baseline=-0.65ex, scale=0.03001]
    \begin{knot}[clip width=5, end tolerance=1pt])
        \useasboundingbox (-5, -5) rectangle (5, 5); % REMOVE ME
        \draw[thick] (0, 0) circle (5);
\draw[thick,red,fill=red] (-0, 0) circle (3);
    \end{knot}
\end{tikzpicture}}

\newcommand{\MediumUnknot} {\begin{tikzpicture}[baseline=-0.65ex, scale=0.06001]
    \begin{knot}[clip width=7, end tolerance=1pt])
        \useasboundingbox (-7, -5) rectangle (7, 5); % REMOVE ME
        \draw[thick] (0, 0) circle (5);
\draw[thick,red,fill=red] (-0, 0) circle (3);
    \end{knot}
\end{tikzpicture}}

\newcommand{\LargeUnknot} {\begin{tikzpicture}[baseline=-0.65ex, scale=0.15001]
    \begin{knot}[clip width=15, end tolerance=1pt])
        \useasboundingbox (-7, -5) rectangle (7, 5); % REMOVE ME
        \draw[thick] (0, 0) circle (5);
\draw[thick,red,fill=red] (-0, 0) circle (3);
    \end{knot}
\end{tikzpicture}}

\newcommand{\SmallPlusCrossing} {\begin{tikzpicture}[baseline=-0.65ex, scale=0.03001]
    \begin{knot}[clip width=5, end tolerance=1pt])
        \useasboundingbox (-5, -5) rectangle (5, 5); % REMOVE ME
        \strand[thick] (-5, -5) to (5, 5);
        \strand[thick] (5, -5) to (-5, 5);
\draw[thick,red,fill=red] (-0, 0) circle (3);
    \end{knot}
\end{tikzpicture}}

\newcommand{\MediumPlusCrossing} {\begin{tikzpicture}[baseline=-0.65ex, scale=0.06001]
    \begin{knot}[clip width=7, end tolerance=1pt])
        \useasboundingbox (-7, -5) rectangle (7, 5); % REMOVE ME
        \strand[thick] (-5, -5) to (5, 5);
        \strand[thick] (5, -5) to (-5, 5);
\draw[thick,red,fill=red] (-0, 0) circle (3);
    \end{knot}
\end{tikzpicture}}

\newcommand{\LargePlusCrossing} {\begin{tikzpicture}[baseline=-0.65ex, scale=0.15001]
    \begin{knot}[clip width=15, end tolerance=1pt])
        \useasboundingbox (-7, -5) rectangle (7, 5); % REMOVE ME
        \strand[thick] (-5, -5) to (5, 5);
        \strand[thick] (5, -5) to (-5, 5);
\draw[thick,red,fill=red] (-0, 0) circle (3);
    \end{knot}
\end{tikzpicture}}

\newcommand{\LargePlusCrossingColouring} {\begin{tikzpicture}[baseline=-0.65ex, scale=0.15001]
    \begin{knot}[clip width=15, end tolerance=1pt])
        \useasboundingbox (-7, -5) rectangle (7, 5); % REMOVE ME
        \strand[thick] (-5, -5) to (5, 5);
        \strand[thick] (5, -5) to (-5, 5);
        \node[first_colour] at (5, 5)[below right] {$c$};
        \node[first_colour] at (5, -5)[above right] {$b$};
        \node[first_colour] at (-5, 5)[below left] {$a$};
\draw[thick,red,fill=red] (-0, 0) circle (3);
    \end{knot}
\end{tikzpicture}}

\newcommand{\SmallPlusCrossingLabel} {\begin{tikzpicture}[baseline=-0.65ex, scale=0.03001]
    \begin{knot}[clip width=5, end tolerance=1pt])
        \useasboundingbox (-5, -5) rectangle (5, 5); % REMOVE ME
        \strand[thick,-latex] (-5, -5) to (5, 5);
        \strand[thick,] (5, -5) to (-5, 5);
        \node[first_colour] at (5, 5)[below right] {$g$};
        \node[first_colour] at (5, -5)[above right] {$h$};
        \node[first_colour] at (-5, 5)[below left] {$k$};
\draw[thick,red,fill=red] (-0, 0) circle (3);
    \end{knot}
\end{tikzpicture}}

\newcommand{\MediumPlusCrossingLabel} {\begin{tikzpicture}[baseline=-0.65ex, scale=0.06001]
    \begin{knot}[clip width=7, end tolerance=1pt])
        \useasboundingbox (-7, -5) rectangle (7, 5); % REMOVE ME
        \strand[thick,-latex] (-5, -5) to (5, 5);
        \strand[thick,] (5, -5) to (-5, 5);
        \node[first_colour] at (5, 5)[below right] {$g$};
        \node[first_colour] at (5, -5)[above right] {$h$};
        \node[first_colour] at (-5, 5)[below left] {$k$};
\draw[thick,red,fill=red] (-0, 0) circle (3);
    \end{knot}
\end{tikzpicture}}

\newcommand{\LargePlusCrossingLabel} {\begin{tikzpicture}[baseline=-0.65ex, scale=0.15001]
    \begin{knot}[clip width=15, end tolerance=1pt])
        \useasboundingbox (-7, -5) rectangle (7, 5); % REMOVE ME
        \strand[thick,-latex] (-5, -5) to (5, 5);
        \strand[thick,] (5, -5) to (-5, 5);
        \node[first_colour] at (5, 5)[below right] {$g$};
        \node[first_colour] at (5, -5)[above right] {$h$};
        \node[first_colour] at (-5, 5)[below left] {$k$};
\draw[thick,red,fill=red] (-0, 0) circle (3);
    \end{knot}
\end{tikzpicture}}

\newcommand{\SmallPlusCrossingMatrix} {\begin{tikzpicture}[baseline=-0.65ex, scale=0.03001]
    \begin{knot}[clip width=5, end tolerance=1pt])
        \useasboundingbox (-5, -5) rectangle (5, 5); % REMOVE ME
        \strand[thick,] (-5, -5) to (5, 5);
        \strand[thick,] (5, -5) to (-5, 5);
        \node[first_colour] at (5, 5)[below right] {$x_i$};
        \node[first_colour] at (5, -5)[above right] {$x_j$};
        \node[first_colour] at (-5, 5)[below left] {$x_k$};
\draw[thick,red,fill=red] (-0, 0) circle (3);
    \end{knot}
\end{tikzpicture}}

\newcommand{\MediumPlusCrossingMatrix} {\begin{tikzpicture}[baseline=-0.65ex, scale=0.06001]
    \begin{knot}[clip width=7, end tolerance=1pt])
        \useasboundingbox (-7, -5) rectangle (7, 5); % REMOVE ME
        \strand[thick,] (-5, -5) to (5, 5);
        \strand[thick,] (5, -5) to (-5, 5);
        \node[first_colour] at (5, 5)[below right] {$x_i$};
        \node[first_colour] at (5, -5)[above right] {$x_j$};
        \node[first_colour] at (-5, 5)[below left] {$x_k$};
\draw[thick,red,fill=red] (-0, 0) circle (3);
    \end{knot}
\end{tikzpicture}}

\newcommand{\LargePlusCrossingMatrix} {\begin{tikzpicture}[baseline=-0.65ex, scale=0.15001]
    \begin{knot}[clip width=15, end tolerance=1pt])
        \useasboundingbox (-7, -5) rectangle (7, 5); % REMOVE ME
        \strand[thick,] (-5, -5) to (5, 5);
        \strand[thick,] (5, -5) to (-5, 5);
        \node[first_colour] at (5, 5)[below right] {$x_i$};
        \node[first_colour] at (5, -5)[above right] {$x_j$};
        \node[first_colour] at (-5, 5)[below left] {$x_k$};
\draw[thick,red,fill=red] (-0, 0) circle (3);
    \end{knot}
\end{tikzpicture}}

\newcommand{\SmallPlusCrossingArrows} {\begin{tikzpicture}[baseline=-0.65ex, scale=0.03001]
    \begin{knot}[clip width=5, end tolerance=1pt])
        \useasboundingbox (-5, -5) rectangle (5, 5); % REMOVE ME
        \strand[thick,-latex] (-5, -5) to (5, 5);
        \strand[thick,-latex] (5, -5) to (-5, 5);
\draw[thick,red,fill=red] (-0, 0) circle (3);
    \end{knot}
\end{tikzpicture}}

\newcommand{\MediumPlusCrossingArrows} {\begin{tikzpicture}[baseline=-0.65ex, scale=0.06001]
    \begin{knot}[clip width=7, end tolerance=1pt])
        \useasboundingbox (-7, -5) rectangle (7, 5); % REMOVE ME
        \strand[thick,-latex] (-5, -5) to (5, 5);
        \strand[thick,-latex] (5, -5) to (-5, 5);
\draw[thick,red,fill=red] (-0, 0) circle (3);
    \end{knot}
\end{tikzpicture}}

\newcommand{\LargePlusCrossingArrows} {\begin{tikzpicture}[baseline=-0.65ex, scale=0.15001]
    \begin{knot}[clip width=15, end tolerance=1pt])
        \useasboundingbox (-7, -5) rectangle (7, 5); % REMOVE ME
        \strand[thick,-latex] (-5, -5) to (5, 5);
        \strand[thick,-latex] (5, -5) to (-5, 5);
\draw[thick,red,fill=red] (-0, 0) circle (3);
    \end{knot}
\end{tikzpicture}}

\newcommand{\SmallMinusCrossing} {\begin{tikzpicture}[baseline=-0.65ex, scale=0.03001]
    \begin{knot}[clip width=5, end tolerance=1pt, flip crossing/.list={1}])
        \useasboundingbox (-5, -5) rectangle (5, 5); % REMOVE ME
        \strand[thick] (-5, -5) to (5, 5);
        \strand[thick] (-5, 5) to (5, -5);
\draw[thick,red,fill=red] (-0, 0) circle (3);
    \end{knot}
\end{tikzpicture}}

\newcommand{\MediumMinusCrossing} {\begin{tikzpicture}[baseline=-0.65ex, scale=0.06001]
    \begin{knot}[clip width=7, end tolerance=1pt, flip crossing/.list={1}])
        \useasboundingbox (-7, -5) rectangle (7, 5); % REMOVE ME
        \strand[thick] (-5, -5) to (5, 5);
        \strand[thick] (-5, 5) to (5, -5);
\draw[thick,red,fill=red] (-0, 0) circle (3);
    \end{knot}
\end{tikzpicture}}

\newcommand{\LargeMinusCrossing} {\begin{tikzpicture}[baseline=-0.65ex, scale=0.15001]
    \begin{knot}[clip width=15, end tolerance=1pt, flip crossing/.list={1}])
        \useasboundingbox (-7, -5) rectangle (7, 5); % REMOVE ME
        \strand[thick] (-5, -5) to (5, 5);
        \strand[thick] (-5, 5) to (5, -5);
\draw[thick,red,fill=red] (-0, 0) circle (3);
    \end{knot}
\end{tikzpicture}}

\newcommand{\SmallMinusCrossingArrows} {\begin{tikzpicture}[baseline=-0.65ex, scale=0.03001]
    \begin{knot}[clip width=5, end tolerance=1pt, flip crossing/.list={1}])
        \useasboundingbox (-5, -5) rectangle (5, 5); % REMOVE ME
        \strand[thick,-latex] (-5, -5) to (5, 5);
        \strand[thick,-latex] (5, -5) to (-5, 5);
\draw[thick,red,fill=red] (-0, 0) circle (3);
    \end{knot}
\end{tikzpicture}}

\newcommand{\MediumMinusCrossingArrows} {\begin{tikzpicture}[baseline=-0.65ex, scale=0.06001]
    \begin{knot}[clip width=7, end tolerance=1pt, flip crossing/.list={1}])
        \useasboundingbox (-7, -5) rectangle (7, 5); % REMOVE ME
        \strand[thick,-latex] (-5, -5) to (5, 5);
        \strand[thick,-latex] (5, -5) to (-5, 5);
\draw[thick,red,fill=red] (-0, 0) circle (3);
    \end{knot}
\end{tikzpicture}}

\newcommand{\LargeMinusCrossingArrows} {\begin{tikzpicture}[baseline=-0.65ex, scale=0.15001]
    \begin{knot}[clip width=15, end tolerance=1pt, flip crossing/.list={1}])
        \useasboundingbox (-7, -5) rectangle (7, 5); % REMOVE ME
        \strand[thick,-latex] (-5, -5) to (5, 5);
        \strand[thick,-latex] (5, -5) to (-5, 5);
\draw[thick,red,fill=red] (-0, 0) circle (3);
    \end{knot}
\end{tikzpicture}}

\newcommand{\SmallMinusCrossingChessboard} {\begin{tikzpicture}[baseline=-0.65ex, scale=0.03001]
    \begin{knot}[clip width=5, end tolerance=1pt, flip crossing/.list={1}])
        \useasboundingbox (-5, -5) rectangle (5, 5); % REMOVE ME
        \strand[thick] (-5, -5) to (5, 5);
        \strand[thick] (-5, 5) to (5, -5);\fill[diagramfiller] (-4, 5) to (0, 1) to (4, 5);
        \fill[diagramfiller] (-4, -5) to (0, -1) to (4, -5);
        \node[first_colour] at (-5, 0) {$-1$};
\draw[thick,red,fill=red] (-0, 0) circle (3);
    \end{knot}
\end{tikzpicture}}

\newcommand{\MediumMinusCrossingChessboard} {\begin{tikzpicture}[baseline=-0.65ex, scale=0.06001]
    \begin{knot}[clip width=7, end tolerance=1pt, flip crossing/.list={1}])
        \useasboundingbox (-7, -5) rectangle (7, 5); % REMOVE ME
        \strand[thick] (-5, -5) to (5, 5);
        \strand[thick] (-5, 5) to (5, -5);\fill[diagramfiller] (-4, 5) to (0, 1) to (4, 5);
        \fill[diagramfiller] (-4, -5) to (0, -1) to (4, -5);
        \node[first_colour] at (-5, 0) {$-1$};
\draw[thick,red,fill=red] (-0, 0) circle (3);
    \end{knot}
\end{tikzpicture}}

\newcommand{\LargeMinusCrossingChessboard} {\begin{tikzpicture}[baseline=-0.65ex, scale=0.15001]
    \begin{knot}[clip width=15, end tolerance=1pt, flip crossing/.list={1}])
        \useasboundingbox (-7, -5) rectangle (7, 5); % REMOVE ME
        \strand[thick] (-5, -5) to (5, 5);
        \strand[thick] (-5, 5) to (5, -5);\fill[diagramfiller] (-4, 5) to (0, 1) to (4, 5);
        \fill[diagramfiller] (-4, -5) to (0, -1) to (4, -5);
        \node[first_colour] at (-5, 0) {$-1$};
\draw[thick,red,fill=red] (-0, 0) circle (3);
    \end{knot}
\end{tikzpicture}}

\newcommand{\SmallCrossingChessboardA} {\begin{tikzpicture}[baseline=-0.65ex, scale=0.03001]
    \begin{knot}[clip width=5, end tolerance=1pt, flip crossing/.list={1}])
        \useasboundingbox (-5, -5) rectangle (5, 5); % REMOVE ME
        \strand[thick] (-5,5) to (5,-5);
        \strand[thick] (-5,-5) to (5,5);
        \fill[diagramfiller] (-4, 5) to (0, 1) to (4, 5);
        \fill[diagramfiller] (-4, -5) to (0, -1) to (4, -5);
        \node[first_colour] at (-5, -5)[left] {$a$};
        \node[first_colour] at (-5, +5)[left] {$b$};
        \node[first_colour] at (+5, -5)[right] {$c$};
        \node[first_colour] at (+5, +5)[right] {$a$};
\draw[thick,red,fill=red] (-0, 0) circle (3);
    \end{knot}
\end{tikzpicture}}

\newcommand{\MediumCrossingChessboardA} {\begin{tikzpicture}[baseline=-0.65ex, scale=0.06001]
    \begin{knot}[clip width=7, end tolerance=1pt, flip crossing/.list={1}])
        \useasboundingbox (-7, -5) rectangle (7, 5); % REMOVE ME
        \strand[thick] (-5,5) to (5,-5);
        \strand[thick] (-5,-5) to (5,5);
        \fill[diagramfiller] (-4, 5) to (0, 1) to (4, 5);
        \fill[diagramfiller] (-4, -5) to (0, -1) to (4, -5);
        \node[first_colour] at (-5, -5)[left] {$a$};
        \node[first_colour] at (-5, +5)[left] {$b$};
        \node[first_colour] at (+5, -5)[right] {$c$};
        \node[first_colour] at (+5, +5)[right] {$a$};
\draw[thick,red,fill=red] (-0, 0) circle (3);
    \end{knot}
\end{tikzpicture}}

\newcommand{\LargeCrossingChessboardA} {\begin{tikzpicture}[baseline=-0.65ex, scale=0.15001]
    \begin{knot}[clip width=15, end tolerance=1pt, flip crossing/.list={1}])
        \useasboundingbox (-7, -5) rectangle (7, 5); % REMOVE ME
        \strand[thick] (-5,5) to (5,-5);
        \strand[thick] (-5,-5) to (5,5);
        \fill[diagramfiller] (-4, 5) to (0, 1) to (4, 5);
        \fill[diagramfiller] (-4, -5) to (0, -1) to (4, -5);
        \node[first_colour] at (-5, -5)[left] {$a$};
        \node[first_colour] at (-5, +5)[left] {$b$};
        \node[first_colour] at (+5, -5)[right] {$c$};
        \node[first_colour] at (+5, +5)[right] {$a$};
\draw[thick,red,fill=red] (-0, 0) circle (3);
    \end{knot}
\end{tikzpicture}}

\newcommand{\SmallCrossingChessboardB} {\begin{tikzpicture}[baseline=-0.65ex, scale=0.03001]
    \begin{knot}[clip width=5, end tolerance=1pt, flip crossing/.list={1}])
        \useasboundingbox (-5, -5) rectangle (5, 5); % REMOVE ME
        \strand[thick] (-5,5) to (5,-5);
        \strand[thick] (-5,-5) to (5,5);
        \fill[diagramfiller] (5, -4) to (1, 0) to (5, 4);
        \fill[diagramfiller] (-5, -4) to (-1, 0) to (-5, 4);
        \node[first_colour] at (-5, -5)[left] {$a$};
        \node[first_colour] at (-5, +5)[left] {$b$};
        \node[first_colour] at (+5, -5)[right] {$c$};
        \node[first_colour] at (+5, +5)[right] {$a$};
\draw[thick,red,fill=red] (-0, 0) circle (3);
    \end{knot}
\end{tikzpicture}}

\newcommand{\MediumCrossingChessboardB} {\begin{tikzpicture}[baseline=-0.65ex, scale=0.06001]
    \begin{knot}[clip width=7, end tolerance=1pt, flip crossing/.list={1}])
        \useasboundingbox (-7, -5) rectangle (7, 5); % REMOVE ME
        \strand[thick] (-5,5) to (5,-5);
        \strand[thick] (-5,-5) to (5,5);
        \fill[diagramfiller] (5, -4) to (1, 0) to (5, 4);
        \fill[diagramfiller] (-5, -4) to (-1, 0) to (-5, 4);
        \node[first_colour] at (-5, -5)[left] {$a$};
        \node[first_colour] at (-5, +5)[left] {$b$};
        \node[first_colour] at (+5, -5)[right] {$c$};
        \node[first_colour] at (+5, +5)[right] {$a$};
\draw[thick,red,fill=red] (-0, 0) circle (3);
    \end{knot}
\end{tikzpicture}}

\newcommand{\LargeCrossingChessboardB} {\begin{tikzpicture}[baseline=-0.65ex, scale=0.15001]
    \begin{knot}[clip width=15, end tolerance=1pt, flip crossing/.list={1}])
        \useasboundingbox (-7, -5) rectangle (7, 5); % REMOVE ME
        \strand[thick] (-5,5) to (5,-5);
        \strand[thick] (-5,-5) to (5,5);
        \fill[diagramfiller] (5, -4) to (1, 0) to (5, 4);
        \fill[diagramfiller] (-5, -4) to (-1, 0) to (-5, 4);
        \node[first_colour] at (-5, -5)[left] {$a$};
        \node[first_colour] at (-5, +5)[left] {$b$};
        \node[first_colour] at (+5, -5)[right] {$c$};
        \node[first_colour] at (+5, +5)[right] {$a$};
\draw[thick,red,fill=red] (-0, 0) circle (3);
    \end{knot}
\end{tikzpicture}}

\newcommand{\SmallPlusCrossingChessboard} {\begin{tikzpicture}[baseline=-0.65ex, scale=0.03001]
    \begin{knot}[clip width=5, end tolerance=1pt])
        \useasboundingbox (-5, -5) rectangle (5, 5); % REMOVE ME
        \strand[thick] (-5, -5) to (5, 5);
        \strand[thick] (-5, 5) to (5, -5);\fill[diagramfiller] (-4, 5) to (0, 1) to (4, 5);
        \fill[diagramfiller] (-4, -5) to (0, -1) to (4, -5);
        \node[first_colour] at (-5, 0) {$+1$};
\draw[thick,red,fill=red] (-0, 0) circle (3);
    \end{knot}
\end{tikzpicture}}

\newcommand{\MediumPlusCrossingChessboard} {\begin{tikzpicture}[baseline=-0.65ex, scale=0.06001]
    \begin{knot}[clip width=7, end tolerance=1pt])
        \useasboundingbox (-7, -5) rectangle (7, 5); % REMOVE ME
        \strand[thick] (-5, -5) to (5, 5);
        \strand[thick] (-5, 5) to (5, -5);\fill[diagramfiller] (-4, 5) to (0, 1) to (4, 5);
        \fill[diagramfiller] (-4, -5) to (0, -1) to (4, -5);
        \node[first_colour] at (-5, 0) {$+1$};
\draw[thick,red,fill=red] (-0, 0) circle (3);
    \end{knot}
\end{tikzpicture}}

\newcommand{\LargePlusCrossingChessboard} {\begin{tikzpicture}[baseline=-0.65ex, scale=0.15001]
    \begin{knot}[clip width=15, end tolerance=1pt])
        \useasboundingbox (-7, -5) rectangle (7, 5); % REMOVE ME
        \strand[thick] (-5, -5) to (5, 5);
        \strand[thick] (-5, 5) to (5, -5);\fill[diagramfiller] (-4, 5) to (0, 1) to (4, 5);
        \fill[diagramfiller] (-4, -5) to (0, -1) to (4, -5);
        \node[first_colour] at (-5, 0) {$+1$};
\draw[thick,red,fill=red] (-0, 0) circle (3);
    \end{knot}
\end{tikzpicture}}

\newcommand{\LargeMinusCrossingQuandle} {\begin{tikzpicture}[baseline=-0.65ex, scale=0.15001]
    \begin{knot}[clip width=15, end tolerance=1pt, flip crossing/.list={1}])
        \useasboundingbox (-7, -5) rectangle (7, 5); % REMOVE ME
        \strand[thick] (-5, -5) to (5, 5);
        \strand[thick,-latex] (5, -5) to (-5, 5);\node[first_colour] at (5, 5)[below right] {$x$};
        \node[first_colour] at (-5, -5)[above left] {$x \triangleright y$};
        \node[first_colour] at (-5, 5)[below left] {$y$};
\draw[thick,red,fill=red] (-0, 0) circle (3);
    \end{knot}
\end{tikzpicture}}

\newcommand{\SmallAlphaSmoothing} {\begin{tikzpicture}[baseline=-0.65ex, scale=0.03001]
    \begin{knot}[clip width=5, end tolerance=1pt])
        \useasboundingbox (-5, -5) rectangle (5, 5); % REMOVE ME
        \draw[thick] (-5, -5) to [out=45, in=-45] (-5, 5);
        \draw[thick] (5, -5) to [out=135, in=-135] (5, 5);
\draw[thick,red,fill=red] (-0, 0) circle (3);
    \end{knot}
\end{tikzpicture}}

\newcommand{\MediumAlphaSmoothing} {\begin{tikzpicture}[baseline=-0.65ex, scale=0.06001]
    \begin{knot}[clip width=7, end tolerance=1pt])
        \useasboundingbox (-7, -5) rectangle (7, 5); % REMOVE ME
        \draw[thick] (-5, -5) to [out=45, in=-45] (-5, 5);
        \draw[thick] (5, -5) to [out=135, in=-135] (5, 5);
\draw[thick,red,fill=red] (-0, 0) circle (3);
    \end{knot}
\end{tikzpicture}}

\newcommand{\LargeAlphaSmoothing} {\begin{tikzpicture}[baseline=-0.65ex, scale=0.15001]
    \begin{knot}[clip width=15, end tolerance=1pt])
        \useasboundingbox (-7, -5) rectangle (7, 5); % REMOVE ME
        \draw[thick] (-5, -5) to [out=45, in=-45] (-5, 5);
        \draw[thick] (5, -5) to [out=135, in=-135] (5, 5);
\draw[thick,red,fill=red] (-0, 0) circle (3);
    \end{knot}
\end{tikzpicture}}

\newcommand{\SmallBetaSmoothing} {\begin{tikzpicture}[baseline=-0.65ex, scale=0.03001]
    \begin{knot}[clip width=5, end tolerance=1pt])
        \useasboundingbox (-5, -5) rectangle (5, 5); % REMOVE ME
        \draw[thick] (-5, -5) [in=135, out=45] to (5, -5);
        \draw[thick] (-5, 5) [in=-135, out=-45] to (5, 5);
\draw[thick,red,fill=red] (-0, 0) circle (3);
    \end{knot}
\end{tikzpicture}}

\newcommand{\MediumBetaSmoothing} {\begin{tikzpicture}[baseline=-0.65ex, scale=0.06001]
    \begin{knot}[clip width=7, end tolerance=1pt])
        \useasboundingbox (-7, -5) rectangle (7, 5); % REMOVE ME
        \draw[thick] (-5, -5) [in=135, out=45] to (5, -5);
        \draw[thick] (-5, 5) [in=-135, out=-45] to (5, 5);
\draw[thick,red,fill=red] (-0, 0) circle (3);
    \end{knot}
\end{tikzpicture}}

\newcommand{\LargeBetaSmoothing} {\begin{tikzpicture}[baseline=-0.65ex, scale=0.15001]
    \begin{knot}[clip width=15, end tolerance=1pt])
        \useasboundingbox (-7, -5) rectangle (7, 5); % REMOVE ME
        \draw[thick] (-5, -5) [in=135, out=45] to (5, -5);
        \draw[thick] (-5, 5) [in=-135, out=-45] to (5, 5);
\draw[thick,red,fill=red] (-0, 0) circle (3);
    \end{knot}
\end{tikzpicture}}

\newcommand{\SmallJustSmoothing} {\begin{tikzpicture}[baseline=-0.65ex, scale=0.03001]
    \begin{knot}[clip width=5, end tolerance=1pt])
        \useasboundingbox (-5, -5) rectangle (5, 5); % REMOVE ME
        \draw[thick,-latex] (-5, -5) to [out=45, in=-45] (-5, 5);
        \draw[thick,-latex] (5, -5) to [out=135, in=-135] (5, 5);
\draw[thick,red,fill=red] (-0, 0) circle (3);
    \end{knot}
\end{tikzpicture}}

\newcommand{\MediumJustSmoothing} {\begin{tikzpicture}[baseline=-0.65ex, scale=0.06001]
    \begin{knot}[clip width=7, end tolerance=1pt])
        \useasboundingbox (-7, -5) rectangle (7, 5); % REMOVE ME
        \draw[thick,-latex] (-5, -5) to [out=45, in=-45] (-5, 5);
        \draw[thick,-latex] (5, -5) to [out=135, in=-135] (5, 5);
\draw[thick,red,fill=red] (-0, 0) circle (3);
    \end{knot}
\end{tikzpicture}}

\newcommand{\LargeJustSmoothing} {\begin{tikzpicture}[baseline=-0.65ex, scale=0.15001]
    \begin{knot}[clip width=15, end tolerance=1pt])
        \useasboundingbox (-7, -5) rectangle (7, 5); % REMOVE ME
        \draw[thick,-latex] (-5, -5) to [out=45, in=-45] (-5, 5);
        \draw[thick,-latex] (5, -5) to [out=135, in=-135] (5, 5);
\draw[thick,red,fill=red] (-0, 0) circle (3);
    \end{knot}
\end{tikzpicture}}

\newcommand{\MediumReidemeisterOneLeft} {\begin{tikzpicture}[baseline=-0.65ex, scale=0.06001]
    \begin{knot}[clip width=7, end tolerance=1pt])
        \useasboundingbox (-7, -5) rectangle (7, 5); % REMOVE ME
        \strand[thick] (-5, 5)  [in=left, out=-60] to (3, -5) [in=down, out=right] to (5, 0);
        \strand[thick] (-5, -5) [in=left, out=60]  to (3, 5)  [in=up, out=right]   to (5, 0);
\draw[thick,red,fill=red] (-0, 0) circle (3);
    \end{knot}
\end{tikzpicture}}

\newcommand{\LargeReidemeisterOneLeft} {\begin{tikzpicture}[baseline=-0.65ex, scale=0.15001]
    \begin{knot}[clip width=15, end tolerance=1pt])
        \useasboundingbox (-7, -5) rectangle (7, 5); % REMOVE ME
        \strand[thick] (-5, 5)  [in=left, out=-60] to (3, -5) [in=down, out=right] to (5, 0);
        \strand[thick] (-5, -5) [in=left, out=60]  to (3, 5)  [in=up, out=right]   to (5, 0);
\draw[thick,red,fill=red] (-0, 0) circle (3);
    \end{knot}
\end{tikzpicture}}

\newcommand{\MedLarReidemeisterOneLeft} {\begin{tikzpicture}[baseline=-0.65ex, scale=0.1101]
    \begin{knot}[clip width=7, end tolerance=1pt])
        \useasboundingbox (-7, -5) rectangle (7, 5); % REMOVE ME
        \strand[thick] (-5, 5)  [in=left, out=-60] to (3, -5) [in=down, out=right] to (5, 0);
        \strand[thick] (-5, -5) [in=left, out=60]  to (3, 5)  [in=up, out=right]   to (5, 0);
\draw[thick,red,fill=red] (-0, 0) circle (3);
    \end{knot}
\end{tikzpicture}}

\newcommand{\MediumReidemeisterOneLeftProof} {\begin{tikzpicture}[baseline=-0.65ex, scale=0.06001]
    \begin{knot}[clip width=7, end tolerance=1pt])
        \useasboundingbox (-7, -5) rectangle (7, 5); % REMOVE ME
        \strand[thick] (-5, 5)  [in=left, out=-60] to (3, -5) [in=down, out=right] to (5, 0);
        \strand[thick] (-5, -5) [in=left, out=60]  to (3, 5)  [in=up, out=right]   to (5, 0);\node[first_colour] at (-5, -5)[below] {$b \equiv a$};
        \node[first_colour] at (-5,  5)[above] {$a$};
\draw[thick,red,fill=red] (-0, 0) circle (3);
    \end{knot}
\end{tikzpicture}}

\newcommand{\LargeReidemeisterOneLeftProof} {\begin{tikzpicture}[baseline=-0.65ex, scale=0.15001]
    \begin{knot}[clip width=15, end tolerance=1pt])
        \useasboundingbox (-7, -5) rectangle (7, 5); % REMOVE ME
        \strand[thick] (-5, 5)  [in=left, out=-60] to (3, -5) [in=down, out=right] to (5, 0);
        \strand[thick] (-5, -5) [in=left, out=60]  to (3, 5)  [in=up, out=right]   to (5, 0);\node[first_colour] at (-5, -5)[below] {$b \equiv a$};
        \node[first_colour] at (-5,  5)[above] {$a$};
\draw[thick,red,fill=red] (-0, 0) circle (3);
    \end{knot}
\end{tikzpicture}}

\newcommand{\MediumReidemeisterOneLeftRightQuandleProof} {\begin{tikzpicture}[baseline=-0.65ex, scale=0.06001]
    \begin{knot}[clip width=7, end tolerance=1pt, flip crossing/.list={1}])
        \useasboundingbox (-12, -5) rectangle (12, 5); % REMOVE ME
        \strand[thick] (0, 2) [in=up, out=left] to (-10, -3) [in=left, out=down] to (-5, -5);
        \strand[thick] (-10, 5) [in=up, out=right]  to (-4, -3)  [in=right, out=down]   to (-5, -5);\strand[thick,latex-] (10, 5) [in=up, out=left]  to (4, -3)  [in=left, out=down]   to (5, -5);\strand[thick] (0, 2) [in=up, out=right] to (10, -3) [in=right, out=down] to (5, -5);
        \node[first_colour] at (-10, 5) [left]  {$x$};
        \node[first_colour] at (10, 5)  [right] {$x$};
        \node[first_colour] at (0, 2)   [above] {$x \triangleright x$};
\draw[thick,red,fill=red] (-0, 0) circle (3);
    \end{knot}
\end{tikzpicture}}

\newcommand{\LargeReidemeisterOneLeftRightQuandleProof} {\begin{tikzpicture}[baseline=-0.65ex, scale=0.15001]
    \begin{knot}[clip width=15, end tolerance=1pt, flip crossing/.list={1}])
        \useasboundingbox (-12, -5) rectangle (12, 5); % REMOVE ME
        \strand[thick] (0, 2) [in=up, out=left] to (-10, -3) [in=left, out=down] to (-5, -5);
        \strand[thick] (-10, 5) [in=up, out=right]  to (-4, -3)  [in=right, out=down]   to (-5, -5);\strand[thick,latex-] (10, 5) [in=up, out=left]  to (4, -3)  [in=left, out=down]   to (5, -5);\strand[thick] (0, 2) [in=up, out=right] to (10, -3) [in=right, out=down] to (5, -5);
        \node[first_colour] at (-10, 5) [left]  {$x$};
        \node[first_colour] at (10, 5)  [right] {$x$};
        \node[first_colour] at (0, 2)   [above] {$x \triangleright x$};
\draw[thick,red,fill=red] (-0, 0) circle (3);
    \end{knot}
\end{tikzpicture}}

\newcommand{\MediumReidemeisterOneRight} {\begin{tikzpicture}[baseline=-0.65ex, scale=0.06001]
    \begin{knot}[clip width=7, end tolerance=1pt])
        \useasboundingbox (-7, -5) rectangle (7, 5); % REMOVE ME
        \strand[thick] (-5, -5) [in=left, out=60] to  (3, 5)  [in=up, out=right]   to (5, 0);
        \strand[thick] (-5, 5)  [in=left, out=-60] to (3, -5) [in=down, out=right] to (5, 0);
\draw[thick,red,fill=red] (-0, 0) circle (3);
    \end{knot}
\end{tikzpicture}}

\newcommand{\MediumReidemeisterOneRightQuandleProof} {\begin{tikzpicture}[baseline=-0.65ex, scale=0.06001]
    \begin{knot}[clip width=7, end tolerance=1pt])
        \useasboundingbox (-7, -5) rectangle (7, 5); % REMOVE ME
        \strand[thick] (-5, -5) [in=left, out=60] to  (3, 5)  [in=up, out=right]   to (5, 0);
        \strand[thick,latex-] (-5, 5)  [in=left, out=-60] to (3, -5) [in=down, out=right] to (5, 0);
        \node[first_colour] at (-5, -4)[left] {$x$};
        \node[first_colour] at (-5,  4)[left] {$x \triangleright x$};
\draw[thick,red,fill=red] (-0, 0) circle (3);
    \end{knot}
\end{tikzpicture}}

\newcommand{\LargeReidemeisterOneRightQuandleProof} {\begin{tikzpicture}[baseline=-0.65ex, scale=0.15001]
    \begin{knot}[clip width=15, end tolerance=1pt])
        \useasboundingbox (-7, -5) rectangle (7, 5); % REMOVE ME
        \strand[thick] (-5, -5) [in=left, out=60] to  (3, 5)  [in=up, out=right]   to (5, 0);
        \strand[thick,latex-] (-5, 5)  [in=left, out=-60] to (3, -5) [in=down, out=right] to (5, 0);
        \node[first_colour] at (-5, -4)[left] {$x$};
        \node[first_colour] at (-5,  4)[left] {$x \triangleright x$};
\draw[thick,red,fill=red] (-0, 0) circle (3);
    \end{knot}
\end{tikzpicture}}

\newcommand{\MediumReidemeisterOneStraight} {\begin{tikzpicture}[baseline=-0.65ex, scale=0.06001]
    \begin{knot}[clip width=7, end tolerance=1pt])
        \useasboundingbox (-2, -5) rectangle (2, 5); % REMOVE ME
        \strand[thick] (0, -5) to (0, 5);
\draw[thick,red,fill=red] (-0, 0) circle (3);
    \end{knot}
\end{tikzpicture}}

\newcommand{\LargeReidemeisterOneStraight} {\begin{tikzpicture}[baseline=-0.65ex, scale=0.15001]
    \begin{knot}[clip width=15, end tolerance=1pt])
        \useasboundingbox (-2, -5) rectangle (2, 5); % REMOVE ME
        \strand[thick] (0, -5) to (0, 5);
\draw[thick,red,fill=red] (-0, 0) circle (3);
    \end{knot}
\end{tikzpicture}}

\newcommand{\MedLarReidemeisterOneStraight} {\begin{tikzpicture}[baseline=-0.65ex, scale=0.1101]
    \begin{knot}[clip width=7, end tolerance=1pt])
        \useasboundingbox (-2, -5) rectangle (2, 5); % REMOVE ME
        \strand[thick] (0, -5) to (0, 5);
\draw[thick,red,fill=red] (-0, 0) circle (3);
    \end{knot}
\end{tikzpicture}}

\newcommand{\MediumReidemeisterOneStraightProof} {\begin{tikzpicture}[baseline=-0.65ex, scale=0.06001]
    \begin{knot}[clip width=7, end tolerance=1pt])
        \useasboundingbox (-2, -5) rectangle (2, 5); % REMOVE ME
        \strand[thick] (0, -5) to (0, 5);
        \node[first_colour] at (0,  0)[left] {$a$};
\draw[thick,red,fill=red] (-0, 0) circle (3);
    \end{knot}
\end{tikzpicture}}

\newcommand{\LargeReidemeisterOneStraightProof} {\begin{tikzpicture}[baseline=-0.65ex, scale=0.15001]
    \begin{knot}[clip width=15, end tolerance=1pt])
        \useasboundingbox (-2, -5) rectangle (2, 5); % REMOVE ME
        \strand[thick] (0, -5) to (0, 5);
        \node[first_colour] at (0,  0)[left] {$a$};
\draw[thick,red,fill=red] (-0, 0) circle (3);
    \end{knot}
\end{tikzpicture}}

\newcommand{\MediumReidemeisterOneStraightQuandleProof} {\begin{tikzpicture}[baseline=-0.65ex, scale=0.06001]
    \begin{knot}[clip width=7, end tolerance=1pt])
        \useasboundingbox (-2, -5) rectangle (2, 5); % REMOVE ME
        \strand[thick] (0, -6.65) to (0, 6);
        \node[first_colour] at (0,  0)[left] {$x$};
\draw[thick,red,fill=red] (-0, 0) circle (3);
    \end{knot}
\end{tikzpicture}}

\newcommand{\LargeReidemeisterOneStraightQuandleProof} {\begin{tikzpicture}[baseline=-0.65ex, scale=0.15001]
    \begin{knot}[clip width=15, end tolerance=1pt])
        \useasboundingbox (-2, -5) rectangle (2, 5); % REMOVE ME
        \strand[thick] (0, -6.65) to (0, 6);
        \node[first_colour] at (0,  0)[left] {$x$};
\draw[thick,red,fill=red] (-0, 0) circle (3);
    \end{knot}
\end{tikzpicture}}

\newcommand{\MediumReidemeisterOneStraightQuandleProofRotated} {\begin{tikzpicture}[baseline=-0.65ex, scale=0.06001]
    \begin{knot}[clip width=7, end tolerance=1pt])
        \useasboundingbox (-7, -2) rectangle (7, 2); % REMOVE ME
        \strand[thick, -latex] (-5, 0) to (5, 0);
        \node[first_colour] at (0,  0)[above] {$x$};
\draw[thick,red,fill=red] (-0, 0) circle (3);
    \end{knot}
\end{tikzpicture}}

\newcommand{\LargeReidemeisterOneStraightQuandleProofRotated} {\begin{tikzpicture}[baseline=-0.65ex, scale=0.15001]
    \begin{knot}[clip width=15, end tolerance=1pt])
        \useasboundingbox (-7, -2) rectangle (7, 2); % REMOVE ME
        \strand[thick, -latex] (-5, 0) to (5, 0);
        \node[first_colour] at (0,  0)[above] {$x$};
\draw[thick,red,fill=red] (-0, 0) circle (3);
    \end{knot}
\end{tikzpicture}}

\newcommand{\MediumReidemeisterOneSmoothA} {\begin{tikzpicture}[baseline=-0.65ex, scale=0.06001]
    \begin{knot}[clip width=7, end tolerance=1pt])
        \useasboundingbox (-7, -5) rectangle (7, 5); % REMOVE ME
        \strand[thick] (-5, 5)  [in=left, out=-60] to (-2, 1.5)  [in=left, out=right] to (3, 5)  [in=up]   to (5, 0);
        \strand[thick] (-5, -5) [in=left, out=60] to  (-2, -1.5) [in=left, out=right] to (3, -5) [in=down] to (5, 0);
\draw[thick,red,fill=red] (-0, 0) circle (3);
    \end{knot}
\end{tikzpicture}}

\newcommand{\LargeReidemeisterOneSmoothA} {\begin{tikzpicture}[baseline=-0.65ex, scale=0.15001]
    \begin{knot}[clip width=15, end tolerance=1pt])
        \useasboundingbox (-7, -5) rectangle (7, 5); % REMOVE ME
        \strand[thick] (-5, 5)  [in=left, out=-60] to (-2, 1.5)  [in=left, out=right] to (3, 5)  [in=up]   to (5, 0);
        \strand[thick] (-5, -5) [in=left, out=60] to  (-2, -1.5) [in=left, out=right] to (3, -5) [in=down] to (5, 0);
\draw[thick,red,fill=red] (-0, 0) circle (3);
    \end{knot}
\end{tikzpicture}}

\newcommand{\MediumReidemeisterOneSmoothB} {\begin{tikzpicture}[baseline=-0.65ex, scale=0.06001]
    \begin{knot}[clip width=7, end tolerance=1pt])
        \useasboundingbox (-7, -5) rectangle (7, 5); % REMOVE ME
        \strand[thick] (-5, -5) [in=down, out=up] to (-3.5, 0) to (-5, 5);
        \strand[thick] (-1, 0) [in=left, out=up] to (3, 5) [in=up, out=right] to (5, 0);
        \strand[thick] (-1, 0) [in=left, out=down] to (3, -5) [in=down, out=right] to (5, 0);
\draw[thick,red,fill=red] (-0, 0) circle (3);
    \end{knot}
\end{tikzpicture}}

\newcommand{\LargeReidemeisterOneSmoothB} {\begin{tikzpicture}[baseline=-0.65ex, scale=0.15001]
    \begin{knot}[clip width=15, end tolerance=1pt])
        \useasboundingbox (-7, -5) rectangle (7, 5); % REMOVE ME
        \strand[thick] (-5, -5) [in=down, out=up] to (-3.5, 0) to (-5, 5);
        \strand[thick] (-1, 0) [in=left, out=up] to (3, 5) [in=up, out=right] to (5, 0);
        \strand[thick] (-1, 0) [in=left, out=down] to (3, -5) [in=down, out=right] to (5, 0);
\draw[thick,red,fill=red] (-0, 0) circle (3);
    \end{knot}
\end{tikzpicture}}

\newcommand{\MediumReidemeisterTwoA} {\begin{tikzpicture}[baseline=-0.65ex, scale=0.06001]
    \begin{knot}[clip width=7, end tolerance=1pt])
        \useasboundingbox (-3.5, -5) rectangle (3.5, 5); % REMOVE ME
        \strand[thick] (-2.5, 5) to [in=up, out=down] (2.5, 0);
        \strand[thick] (-2.5, -5) to [in=down, out=up] (2.5, 0);
        \strand[thick] (2.5, 5) to [in=up, out=down] (-2.5, 0);
        \strand[thick] (2.5, -5) to [in=down, out=up] (-2.5, 0);
\draw[thick,red,fill=red] (-0, 0) circle (3);
    \end{knot}
\end{tikzpicture}}

\newcommand{\LargeReidemeisterTwoA} {\begin{tikzpicture}[baseline=-0.65ex, scale=0.15001]
    \begin{knot}[clip width=15, end tolerance=1pt])
        \useasboundingbox (-3.5, -5) rectangle (3.5, 5); % REMOVE ME
        \strand[thick] (-2.5, 5) to [in=up, out=down] (2.5, 0);
        \strand[thick] (-2.5, -5) to [in=down, out=up] (2.5, 0);
        \strand[thick] (2.5, 5) to [in=up, out=down] (-2.5, 0);
        \strand[thick] (2.5, -5) to [in=down, out=up] (-2.5, 0);
\draw[thick,red,fill=red] (-0, 0) circle (3);
    \end{knot}
\end{tikzpicture}}

\newcommand{\MediumReidemeisterTwoQuandleA} {\begin{tikzpicture}[baseline=-0.65ex, scale=0.06001]
    \begin{knot}[clip width=7, end tolerance=1pt])
        \useasboundingbox (-3.5, -5) rectangle (12.5, 5); % REMOVE ME
        \strand[thick] (-2.5, 5) to [in=up, out=down] (2.5, 0);
        \strand[thick] (-2.5, -5) to [in=down, out=up] (2.5, 0);
        \strand[thick] (2.5, 5) to [in=up, out=down] (-2.5, 0);
        \strand[thick] (2.5, -5) to [in=down, out=up] (-2.5, 0);
        \node[first_colour] at (-2.5, -5) [left] {$x$};
        \node[first_colour] at (2.5, -5) [right] {$y$};
        \node[first_colour] at (-2.5, 0) [left] {$y \triangleright x$};
        \node[first_colour] at (2.5, 5) [right] {$x \triangleleft (y \triangleright x)$};
\draw[thick,red,fill=red] (-0, 0) circle (3);
    \end{knot}
\end{tikzpicture}}

\newcommand{\LargeReidemeisterTwoQuandleA} {\begin{tikzpicture}[baseline=-0.65ex, scale=0.15001]
    \begin{knot}[clip width=15, end tolerance=1pt])
        \useasboundingbox (-3.5, -5) rectangle (12.5, 5); % REMOVE ME
        \strand[thick] (-2.5, 5) to [in=up, out=down] (2.5, 0);
        \strand[thick] (-2.5, -5) to [in=down, out=up] (2.5, 0);
        \strand[thick] (2.5, 5) to [in=up, out=down] (-2.5, 0);
        \strand[thick] (2.5, -5) to [in=down, out=up] (-2.5, 0);
        \node[first_colour] at (-2.5, -5) [left] {$x$};
        \node[first_colour] at (2.5, -5) [right] {$y$};
        \node[first_colour] at (-2.5, 0) [left] {$y \triangleright x$};
        \node[first_colour] at (2.5, 5) [right] {$x \triangleleft (y \triangleright x)$};
\draw[thick,red,fill=red] (-0, 0) circle (3);
    \end{knot}
\end{tikzpicture}}

\newcommand{\MediumReidemeisterTwoColouringA} {\begin{tikzpicture}[baseline=-0.65ex, scale=0.06001]
    \begin{knot}[clip width=7, end tolerance=1pt])
        \useasboundingbox (-3.5, -5) rectangle (3.5, 5); % REMOVE ME
        \strand[thick] (-2.5, 5) to [in=up, out=down] (2.5, 0);
        \strand[thick] (-2.5, -5) to [in=down, out=up] (2.5, 0);
        \strand[thick] (2.5, 5) to [in=up, out=down] (-2.5, 0);
        \strand[thick] (2.5, -5) to [in=down, out=up] (-2.5, 0);
        \node[first_colour] at (-4, -2.5)[left] {$d \equiv b$};
        \node[first_colour] at (4, 2.5)[right] {$a$};
        \node[first_colour] at (4, 0) [right] {$c \equiv 2a-b$};
        \node[first_colour] at (-4, 2.5) [left] {$b$};
\draw[thick,red,fill=red] (-0, 0) circle (3);
    \end{knot}
\end{tikzpicture}}

\newcommand{\LargeReidemeisterTwoColouringA} {\begin{tikzpicture}[baseline=-0.65ex, scale=0.15001]
    \begin{knot}[clip width=15, end tolerance=1pt])
        \useasboundingbox (-3.5, -5) rectangle (3.5, 5); % REMOVE ME
        \strand[thick] (-2.5, 5) to [in=up, out=down] (2.5, 0);
        \strand[thick] (-2.5, -5) to [in=down, out=up] (2.5, 0);
        \strand[thick] (2.5, 5) to [in=up, out=down] (-2.5, 0);
        \strand[thick] (2.5, -5) to [in=down, out=up] (-2.5, 0);
        \node[first_colour] at (-4, -2.5)[left] {$d \equiv b$};
        \node[first_colour] at (4, 2.5)[right] {$a$};
        \node[first_colour] at (4, 0) [right] {$c \equiv 2a-b$};
        \node[first_colour] at (-4, 2.5) [left] {$b$};
\draw[thick,red,fill=red] (-0, 0) circle (3);
    \end{knot}
\end{tikzpicture}}

\newcommand{\MediumReidemeisterTwoLinkingA} {\begin{tikzpicture}[baseline=-0.65ex, scale=0.06001]
    \begin{knot}[clip width=7, end tolerance=1pt])
        \useasboundingbox (-7, -5) rectangle (7, 5); % REMOVE ME
        \strand[thick] (-2.5, 5) to [in=up, out=down] (2.5, 0);
        \strand[thick] (-2.5, -5) to [in=down, out=up] (2.5, 0);
        \strand[thick] (2.5, 5) to [in=up, out=down] (-2.5, 0);
        \strand[thick] (2.5, -5) to [in=down, out=up] (-2.5, 0);
        \node[blue] at (-4,2.5)[left] {$a$};
        \node[blue] at (-4,-2.5)[left] {$-a$};
\draw[thick,red,fill=red] (-0, 0) circle (3);
    \end{knot}
\end{tikzpicture}}

\newcommand{\LargeReidemeisterTwoLinkingA} {\begin{tikzpicture}[baseline=-0.65ex, scale=0.15001]
    \begin{knot}[clip width=15, end tolerance=1pt])
        \useasboundingbox (-7, -5) rectangle (7, 5); % REMOVE ME
        \strand[thick] (-2.5, 5) to [in=up, out=down] (2.5, 0);
        \strand[thick] (-2.5, -5) to [in=down, out=up] (2.5, 0);
        \strand[thick] (2.5, 5) to [in=up, out=down] (-2.5, 0);
        \strand[thick] (2.5, -5) to [in=down, out=up] (-2.5, 0);
        \node[blue] at (-4,2.5)[left] {$a$};
        \node[blue] at (-4,-2.5)[left] {$-a$};
\draw[thick,red,fill=red] (-0, 0) circle (3);
    \end{knot}
\end{tikzpicture}}

\newcommand{\MedLarReidemeisterTwoLinkingA} {\begin{tikzpicture}[baseline=-0.65ex, scale=0.1101]
    \begin{knot}[clip width=7, end tolerance=1pt])
        \useasboundingbox (-7, -5) rectangle (7, 5); % REMOVE ME
        \strand[thick] (-2.5, 5) to [in=up, out=down] (2.5, 0);
        \strand[thick] (-2.5, -5) to [in=down, out=up] (2.5, 0);
        \strand[thick] (2.5, 5) to [in=up, out=down] (-2.5, 0);
        \strand[thick] (2.5, -5) to [in=down, out=up] (-2.5, 0);
        \node[blue] at (-4,2.5)[left] {$a$};
        \node[blue] at (-4,-2.5)[left] {$-a$};
\draw[thick,red,fill=red] (-0, 0) circle (3);
    \end{knot}
\end{tikzpicture}}

\newcommand{\MediumReidemeisterTwoB} {\begin{tikzpicture}[baseline=-0.65ex, scale=0.06001]
    \begin{knot}[clip width=7, end tolerance=1pt])
        \useasboundingbox (-3.5, -5) rectangle (3.5, 5); % REMOVE ME
        \strand[thick] (-2.5, 5) to [in=up, out=down] (-1, 0);
        \strand[thick] (-2.5, -5) to [in=down, out=up] (-1, 0);
        \strand[thick] (2.5, 5) to [in=up, out=down] (1, 0);
        \strand[thick] (2.5, -5) to [in=down, out=up] (1, 0);
\draw[thick,red,fill=red] (-0, 0) circle (3);
    \end{knot}
\end{tikzpicture}}

\newcommand{\LargeReidemeisterTwoB} {\begin{tikzpicture}[baseline=-0.65ex, scale=0.15001]
    \begin{knot}[clip width=15, end tolerance=1pt])
        \useasboundingbox (-3.5, -5) rectangle (3.5, 5); % REMOVE ME
        \strand[thick] (-2.5, 5) to [in=up, out=down] (-1, 0);
        \strand[thick] (-2.5, -5) to [in=down, out=up] (-1, 0);
        \strand[thick] (2.5, 5) to [in=up, out=down] (1, 0);
        \strand[thick] (2.5, -5) to [in=down, out=up] (1, 0);
\draw[thick,red,fill=red] (-0, 0) circle (3);
    \end{knot}
\end{tikzpicture}}

\newcommand{\MedLarReidemeisterTwoB} {\begin{tikzpicture}[baseline=-0.65ex, scale=0.1101]
    \begin{knot}[clip width=7, end tolerance=1pt])
        \useasboundingbox (-3.5, -5) rectangle (3.5, 5); % REMOVE ME
        \strand[thick] (-2.5, 5) to [in=up, out=down] (-1, 0);
        \strand[thick] (-2.5, -5) to [in=down, out=up] (-1, 0);
        \strand[thick] (2.5, 5) to [in=up, out=down] (1, 0);
        \strand[thick] (2.5, -5) to [in=down, out=up] (1, 0);
\draw[thick,red,fill=red] (-0, 0) circle (3);
    \end{knot}
\end{tikzpicture}}

\newcommand{\MediumReidemeisterTwoQuandleB} {\begin{tikzpicture}[baseline=-0.65ex, scale=0.06001]
    \begin{knot}[clip width=7, end tolerance=1pt])
        \useasboundingbox (-5, -5) rectangle (5, 5); % REMOVE ME
        \strand[thick] (-2.5, 5) to [in=up, out=down] (-1, 0);
        \strand[thick] (-2.5, -5) to [in=down, out=up] (-1, 0);
        \strand[thick] (2.5, 5) to [in=up, out=down] (1, 0);
        \strand[thick] (2.5, -5) to [in=down, out=up] (1, 0);
        \node[first_colour] at (2, 0) [right] {$y$};
        \node[first_colour] at (-2, 0) [left] {$x$};
\draw[thick,red,fill=red] (-0, 0) circle (3);
    \end{knot}
\end{tikzpicture}}

\newcommand{\LargeReidemeisterTwoQuandleB} {\begin{tikzpicture}[baseline=-0.65ex, scale=0.15001]
    \begin{knot}[clip width=15, end tolerance=1pt])
        \useasboundingbox (-5, -5) rectangle (5, 5); % REMOVE ME
        \strand[thick] (-2.5, 5) to [in=up, out=down] (-1, 0);
        \strand[thick] (-2.5, -5) to [in=down, out=up] (-1, 0);
        \strand[thick] (2.5, 5) to [in=up, out=down] (1, 0);
        \strand[thick] (2.5, -5) to [in=down, out=up] (1, 0);
        \node[first_colour] at (2, 0) [right] {$y$};
        \node[first_colour] at (-2, 0) [left] {$x$};
\draw[thick,red,fill=red] (-0, 0) circle (3);
    \end{knot}
\end{tikzpicture}}

\newcommand{\MedLarReidemeisterTwoQuandleB} {\begin{tikzpicture}[baseline=-0.65ex, scale=0.1101]
    \begin{knot}[clip width=7, end tolerance=1pt])
        \useasboundingbox (-5, -5) rectangle (5, 5); % REMOVE ME
        \strand[thick] (-2.5, 5) to [in=up, out=down] (-1, 0);
        \strand[thick] (-2.5, -5) to [in=down, out=up] (-1, 0);
        \strand[thick] (2.5, 5) to [in=up, out=down] (1, 0);
        \strand[thick] (2.5, -5) to [in=down, out=up] (1, 0);
        \node[first_colour] at (2, 0) [right] {$y$};
        \node[first_colour] at (-2, 0) [left] {$x$};
\draw[thick,red,fill=red] (-0, 0) circle (3);
    \end{knot}
\end{tikzpicture}}

\newcommand{\MediumReidemeisterThreeA} {\begin{tikzpicture}[baseline=-0.65ex, scale=0.06001]
    \begin{knot}[clip width=7, end tolerance=1pt, flip crossing/.list={1,2,3}])
        \useasboundingbox (-6, -5) rectangle (6, 5); % REMOVE ME
        \strand[thick] (-5, -5) -- (5, 5);
        \strand[thick] (-5, 5) -- (5, -5);
        \strand[thick] (-5, 0) to [in=left, out=right] (0, 5);
        \strand[thick] (5, 0) to [in=right, out=left] (0, 5);
\draw[thick,red,fill=red] (-0, 0) circle (3);
    \end{knot}
\end{tikzpicture}}

\newcommand{\LargeReidemeisterThreeA} {\begin{tikzpicture}[baseline=-0.65ex, scale=0.15001]
    \begin{knot}[clip width=15, end tolerance=1pt, flip crossing/.list={1,2,3}])
        \useasboundingbox (-6, -5) rectangle (6, 5); % REMOVE ME
        \strand[thick] (-5, -5) -- (5, 5);
        \strand[thick] (-5, 5) -- (5, -5);
        \strand[thick] (-5, 0) to [in=left, out=right] (0, 5);
        \strand[thick] (5, 0) to [in=right, out=left] (0, 5);
\draw[thick,red,fill=red] (-0, 0) circle (3);
    \end{knot}
\end{tikzpicture}}

\newcommand{\MediumReidemeisterThreeB} {\begin{tikzpicture}[baseline=-0.65ex, scale=0.06001]
    \begin{knot}[clip width=7, end tolerance=1pt, flip crossing/.list={1,2,3}])
        \useasboundingbox (-6, -5) rectangle (6, 5); % REMOVE ME
        \strand[thick] (-5, -5) -- (5, 5);
        \strand[thick] (-5, 5) -- (5, -5);
        \strand[thick] (-5, 0) to [in=left, out=right] (0, -5);
        \strand[thick] (5, 0) to [in=right, out=left] (0, -5);
\draw[thick,red,fill=red] (-0, 0) circle (3);
    \end{knot}
\end{tikzpicture}}

\newcommand{\LargeReidemeisterThreeB} {\begin{tikzpicture}[baseline=-0.65ex, scale=0.15001]
    \begin{knot}[clip width=15, end tolerance=1pt, flip crossing/.list={1,2,3}])
        \useasboundingbox (-6, -5) rectangle (6, 5); % REMOVE ME
        \strand[thick] (-5, -5) -- (5, 5);
        \strand[thick] (-5, 5) -- (5, -5);
        \strand[thick] (-5, 0) to [in=left, out=right] (0, -5);
        \strand[thick] (5, 0) to [in=right, out=left] (0, -5);
\draw[thick,red,fill=red] (-0, 0) circle (3);
    \end{knot}
\end{tikzpicture}}

\newcommand{\MediumReidemeisterThreeLinkingA} {\begin{tikzpicture}[baseline=-0.65ex, scale=0.06001]
    \begin{knot}[clip width=7, end tolerance=1pt, flip crossing/.list={1,2,3}])
        \useasboundingbox (-7, -5) rectangle (7, 5); % REMOVE ME
        \strand[thick] (-5, -5) -- (5, 5);
        \strand[thick] (-5, 5) -- (5, -5);
        \strand[thick] (-5, 0) to [in=left, out=right] (0, 5);
        \strand[thick] (5, 0) to [in=right, out=left] (0, 5);
        \node[blue] at (-4,2.5)[left] {$a$};
        \node[blue] at (4,2.5)[right] {$b$};
        \node[blue] at (0,-2)[below] {$c$};
\draw[thick,red,fill=red] (-0, 0) circle (3);
    \end{knot}
\end{tikzpicture}}

\newcommand{\LargeReidemeisterThreeLinkingA} {\begin{tikzpicture}[baseline=-0.65ex, scale=0.15001]
    \begin{knot}[clip width=15, end tolerance=1pt, flip crossing/.list={1,2,3}])
        \useasboundingbox (-7, -5) rectangle (7, 5); % REMOVE ME
        \strand[thick] (-5, -5) -- (5, 5);
        \strand[thick] (-5, 5) -- (5, -5);
        \strand[thick] (-5, 0) to [in=left, out=right] (0, 5);
        \strand[thick] (5, 0) to [in=right, out=left] (0, 5);
        \node[blue] at (-4,2.5)[left] {$a$};
        \node[blue] at (4,2.5)[right] {$b$};
        \node[blue] at (0,-2)[below] {$c$};
\draw[thick,red,fill=red] (-0, 0) circle (3);
    \end{knot}
\end{tikzpicture}}

\newcommand{\MedLarReidemeisterThreeLinkingA} {\begin{tikzpicture}[baseline=-0.65ex, scale=0.1101]
    \begin{knot}[clip width=7, end tolerance=1pt, flip crossing/.list={1,2,3}])
        \useasboundingbox (-7, -5) rectangle (7, 5); % REMOVE ME
        \strand[thick] (-5, -5) -- (5, 5);
        \strand[thick] (-5, 5) -- (5, -5);
        \strand[thick] (-5, 0) to [in=left, out=right] (0, 5);
        \strand[thick] (5, 0) to [in=right, out=left] (0, 5);
        \node[blue] at (-4,2.5)[left] {$a$};
        \node[blue] at (4,2.5)[right] {$b$};
        \node[blue] at (0,-2)[below] {$c$};
\draw[thick,red,fill=red] (-0, 0) circle (3);
    \end{knot}
\end{tikzpicture}}

\newcommand{\MediumReidemeisterThreeLinkingB} {\begin{tikzpicture}[baseline=-0.65ex, scale=0.06001]
    \begin{knot}[clip width=7, end tolerance=1pt, flip crossing/.list={1,2,3}])
        \useasboundingbox (-7, -5) rectangle (7, 5); % REMOVE ME
        \strand[thick] (-5, -5) -- (5, 5);
        \strand[thick] (-5, 5) -- (5, -5);
        \strand[thick] (-5, 0) to [in=left, out=right] (0, -5);
        \strand[thick] (5, 0) to [in=right, out=left] (0, -5);
        \node[blue] at (-4,-2.5)[left] {$a$};
        \node[blue] at (4,-2.5)[right] {$b$};
        \node[blue] at (0,2)[above] {$c$};
\draw[thick,red,fill=red] (-0, 0) circle (3);
    \end{knot}
\end{tikzpicture}}

\newcommand{\LargeReidemeisterThreeLinkingB} {\begin{tikzpicture}[baseline=-0.65ex, scale=0.15001]
    \begin{knot}[clip width=15, end tolerance=1pt, flip crossing/.list={1,2,3}])
        \useasboundingbox (-7, -5) rectangle (7, 5); % REMOVE ME
        \strand[thick] (-5, -5) -- (5, 5);
        \strand[thick] (-5, 5) -- (5, -5);
        \strand[thick] (-5, 0) to [in=left, out=right] (0, -5);
        \strand[thick] (5, 0) to [in=right, out=left] (0, -5);
        \node[blue] at (-4,-2.5)[left] {$a$};
        \node[blue] at (4,-2.5)[right] {$b$};
        \node[blue] at (0,2)[above] {$c$};
\draw[thick,red,fill=red] (-0, 0) circle (3);
    \end{knot}
\end{tikzpicture}}

\newcommand{\MedLarReidemeisterThreeLinkingB} {\begin{tikzpicture}[baseline=-0.65ex, scale=0.1101]
    \begin{knot}[clip width=7, end tolerance=1pt, flip crossing/.list={1,2,3}])
        \useasboundingbox (-7, -5) rectangle (7, 5); % REMOVE ME
        \strand[thick] (-5, -5) -- (5, 5);
        \strand[thick] (-5, 5) -- (5, -5);
        \strand[thick] (-5, 0) to [in=left, out=right] (0, -5);
        \strand[thick] (5, 0) to [in=right, out=left] (0, -5);
        \node[blue] at (-4,-2.5)[left] {$a$};
        \node[blue] at (4,-2.5)[right] {$b$};
        \node[blue] at (0,2)[above] {$c$};
\draw[thick,red,fill=red] (-0, 0) circle (3);
    \end{knot}
\end{tikzpicture}}

\newcommand{\MediumReidemeisterThreeColouringA} {\begin{tikzpicture}[baseline=-0.65ex, scale=0.06001]
    \begin{knot}[clip width=7, end tolerance=1pt, flip crossing/.list={1,2,3}])
        \useasboundingbox (-10, -7) rectangle (10, 7); % REMOVE ME
        \strand[thick] (-5, -5) -- (5, 5);
        \strand[thick] (-5, 5) -- (5, -5);
        \strand[thick] (-5, 0) to [in=left, out=right] (0, 5);
        \strand[thick] (5, 0) to [in=right, out=left] (0, 5);
        \node[first_colour] at (-5, 4) [left] {$b$};
        \node[first_colour] at (5, 4) [right] {$c$};
        \node[first_colour] at (-5, 0) [left] {$a$};
        \node[first_colour] at (-5, -5) [below] {$2a-2b+c$};
        \node[first_colour] at (5, -5) [below] {$2a-b$};
\draw[thick,red,fill=red] (-0, 0) circle (3);
    \end{knot}
\end{tikzpicture}}

\newcommand{\LargeReidemeisterThreeColouringA} {\begin{tikzpicture}[baseline=-0.65ex, scale=0.15001]
    \begin{knot}[clip width=15, end tolerance=1pt, flip crossing/.list={1,2,3}])
        \useasboundingbox (-10, -7) rectangle (10, 7); % REMOVE ME
        \strand[thick] (-5, -5) -- (5, 5);
        \strand[thick] (-5, 5) -- (5, -5);
        \strand[thick] (-5, 0) to [in=left, out=right] (0, 5);
        \strand[thick] (5, 0) to [in=right, out=left] (0, 5);
        \node[first_colour] at (-5, 4) [left] {$b$};
        \node[first_colour] at (5, 4) [right] {$c$};
        \node[first_colour] at (-5, 0) [left] {$a$};
        \node[first_colour] at (-5, -5) [below] {$2a-2b+c$};
        \node[first_colour] at (5, -5) [below] {$2a-b$};
\draw[thick,red,fill=red] (-0, 0) circle (3);
    \end{knot}
\end{tikzpicture}}

\newcommand{\MedLarReidemeisterThreeColouringA} {\begin{tikzpicture}[baseline=-0.65ex, scale=0.1101]
    \begin{knot}[clip width=7, end tolerance=1pt, flip crossing/.list={1,2,3}])
        \useasboundingbox (-10, -7) rectangle (10, 7); % REMOVE ME
        \strand[thick] (-5, -5) -- (5, 5);
        \strand[thick] (-5, 5) -- (5, -5);
        \strand[thick] (-5, 0) to [in=left, out=right] (0, 5);
        \strand[thick] (5, 0) to [in=right, out=left] (0, 5);
        \node[first_colour] at (-5, 4) [left] {$b$};
        \node[first_colour] at (5, 4) [right] {$c$};
        \node[first_colour] at (-5, 0) [left] {$a$};
        \node[first_colour] at (-5, -5) [below] {$2a-2b+c$};
        \node[first_colour] at (5, -5) [below] {$2a-b$};
\draw[thick,red,fill=red] (-0, 0) circle (3);
    \end{knot}
\end{tikzpicture}}

\newcommand{\MediumReidemeisterThreeQuandleA} {\begin{tikzpicture}[baseline=-0.65ex, scale=0.06001]
    \begin{knot}[clip width=7, end tolerance=1pt, flip crossing/.list={1}])
        \useasboundingbox (-7, -5) rectangle (20, 7); % REMOVE ME
        \strand[thick] (-5, -5)  -- (4, 4);
        \strand[thick,-latex] (4, 4) to (5, 5);\strand[thick] (-5, 5) -- (4, -4);
        \strand[thick,-latex] (-4, 4) to (5, -5);
        \strand[thick] (-5, 0) to [in=left, out=right] (0, 4);
        \strand[thick] (5, 0) to [in=right, out=left] (0, 4);
        \strand[thick,-latex] (5, 0) to (6, 0);\node[first_colour] at (-5, 4) [ left] {$z$};
        \node[first_colour] at (-5, 0) [left] {$y$};
        \node[first_colour] at (-5, -4) [ left] {$x$};
        \node[first_colour] at (0, 4) [above] {$y \triangleright z$};
        \node[first_colour] at (5, 4) [right] {$(x \triangleright z) \triangleright (y \triangleright z)$};
\draw[thick,red,fill=red] (-0, 0) circle (3);
    \end{knot}
\end{tikzpicture}}

\newcommand{\LargeReidemeisterThreeQuandleA} {\begin{tikzpicture}[baseline=-0.65ex, scale=0.15001]
    \begin{knot}[clip width=7, end tolerance=1pt, flip crossing/.list={1}])
        \useasboundingbox (-7, -5) rectangle (20, 7); % REMOVE ME
        \strand[thick] (-5, -5)  -- (4, 4);
        \strand[thick,-latex] (4, 4) to (5, 5);\strand[thick] (-5, 5) -- (4, -4);
        \strand[thick,-latex] (-4, 4) to (5, -5);
        \strand[thick] (-5, 0) to [in=left, out=right] (0, 4);
        \strand[thick] (5, 0) to [in=right, out=left] (0, 4);
        \strand[thick,-latex] (5, 0) to (6, 0);\node[first_colour] at (-5, 4) [ left] {$z$};
        \node[first_colour] at (-5, 0) [left] {$y$};
        \node[first_colour] at (-5, -4) [ left] {$x$};
        \node[first_colour] at (0, 4) [above] {$y \triangleright z$};
        \node[first_colour] at (5, 4) [right] {$(x \triangleright z) \triangleright (y \triangleright z)$};
\draw[thick,red,fill=red] (-0, 0) circle (3);
    \end{knot}
\end{tikzpicture}}

\newcommand{\MedLarReidemeisterThreeQuandleA} {\begin{tikzpicture}[baseline=-0.65ex, scale=0.1101]
    \begin{knot}[clip width=7, end tolerance=1pt, flip crossing/.list={1}])
        \useasboundingbox (-7, -5) rectangle (20, 7); % REMOVE ME
        \strand[thick] (-5, -5)  -- (4, 4);
        \strand[thick,-latex] (4, 4) to (5, 5);\strand[thick] (-5, 5) -- (4, -4);
        \strand[thick,-latex] (-4, 4) to (5, -5);
        \strand[thick] (-5, 0) to [in=left, out=right] (0, 4);
        \strand[thick] (5, 0) to [in=right, out=left] (0, 4);
        \strand[thick,-latex] (5, 0) to (6, 0);\node[first_colour] at (-5, 4) [ left] {$z$};
        \node[first_colour] at (-5, 0) [left] {$y$};
        \node[first_colour] at (-5, -4) [ left] {$x$};
        \node[first_colour] at (0, 4) [above] {$y \triangleright z$};
        \node[first_colour] at (5, 4) [right] {$(x \triangleright z) \triangleright (y \triangleright z)$};
\draw[thick,red,fill=red] (-0, 0) circle (3);
    \end{knot}
\end{tikzpicture}}

\newcommand{\MediumReidemeisterThreeQuandleB} {\begin{tikzpicture}[baseline=-0.65ex, scale=0.06001]
    \begin{knot}[clip width=7, end tolerance=1pt, flip crossing/.list={1}])
        \useasboundingbox (-7, -5) rectangle (15, 7); % REMOVE ME
        \strand[thick] (-5, -5)  -- (4, 4);
        \strand[thick,-latex] (4, 4) to (5, 5);\strand[thick] (-5, 5) -- (4, -4);
        \strand[thick,-latex] (-4, 4) to (5, -5);
        \strand[thick] (-5, 0) to [in=left, out=right] (0, -4);
        \strand[thick] (5, 0) to [in=right, out=left] (0, -4);
        \strand[thick,-latex] (5, 0) to (6, 0);\node[first_colour] at (-5, 4) [ left] {$z$};
        \node[first_colour] at (-5, 0) [left] {$y$};
        \node[first_colour] at (-5, -4) [ left] {$x$};
        \node[first_colour] at (5, 4) [right] {$(x \triangleright y) \triangleright z$};
\draw[thick,red,fill=red] (-0, 0) circle (3);
    \end{knot}
\end{tikzpicture}}

\newcommand{\LargeReidemeisterThreeQuandleB} {\begin{tikzpicture}[baseline=-0.65ex, scale=0.15001]
    \begin{knot}[clip width=7, end tolerance=1pt, flip crossing/.list={1}])
        \useasboundingbox (-7, -5) rectangle (15, 7); % REMOVE ME
        \strand[thick] (-5, -5)  -- (4, 4);
        \strand[thick,-latex] (4, 4) to (5, 5);\strand[thick] (-5, 5) -- (4, -4);
        \strand[thick,-latex] (-4, 4) to (5, -5);
        \strand[thick] (-5, 0) to [in=left, out=right] (0, -4);
        \strand[thick] (5, 0) to [in=right, out=left] (0, -4);
        \strand[thick,-latex] (5, 0) to (6, 0);\node[first_colour] at (-5, 4) [ left] {$z$};
        \node[first_colour] at (-5, 0) [left] {$y$};
        \node[first_colour] at (-5, -4) [ left] {$x$};
        \node[first_colour] at (5, 4) [right] {$(x \triangleright y) \triangleright z$};
\draw[thick,red,fill=red] (-0, 0) circle (3);
    \end{knot}
\end{tikzpicture}}

\newcommand{\MedLarReidemeisterThreeQuandleB} {\begin{tikzpicture}[baseline=-0.65ex, scale=0.1101]
    \begin{knot}[clip width=7, end tolerance=1pt, flip crossing/.list={1}])
        \useasboundingbox (-7, -5) rectangle (15, 7); % REMOVE ME
        \strand[thick] (-5, -5)  -- (4, 4);
        \strand[thick,-latex] (4, 4) to (5, 5);\strand[thick] (-5, 5) -- (4, -4);
        \strand[thick,-latex] (-4, 4) to (5, -5);
        \strand[thick] (-5, 0) to [in=left, out=right] (0, -4);
        \strand[thick] (5, 0) to [in=right, out=left] (0, -4);
        \strand[thick,-latex] (5, 0) to (6, 0);\node[first_colour] at (-5, 4) [ left] {$z$};
        \node[first_colour] at (-5, 0) [left] {$y$};
        \node[first_colour] at (-5, -4) [ left] {$x$};
        \node[first_colour] at (5, 4) [right] {$(x \triangleright y) \triangleright z$};
\draw[thick,red,fill=red] (-0, 0) circle (3);
    \end{knot}
\end{tikzpicture}}

\newcommand{\MediumReidemeisterThreeColouringB} {\begin{tikzpicture}[baseline=-0.65ex, scale=0.06001]
    \begin{knot}[clip width=7, end tolerance=1pt, flip crossing/.list={1,2,3}])
        \useasboundingbox (-10, -7) rectangle (10, 7); % REMOVE ME
        \strand[thick] (-5, -5) -- (5, 5);
        \strand[thick] (-5, 5) -- (5, -5);
        \strand[thick] (-5, 0) to [in=left, out=right] (0, -5);
        \strand[thick] (5, 0) to [in=right, out=left] (0, -5);
        \node[first_colour] at (-5, 4) [left] {$b$};
        \node[first_colour] at (5, 4) [right] {$c$};
        \node[first_colour] at (5, 0) [right] {$a$};
        \node[first_colour] at (-5, -5) [below] {$2a-2b+c$};
        \node[first_colour] at (5, -5) [below] {$2a-b$};
\draw[thick,red,fill=red] (-0, 0) circle (3);
    \end{knot}
\end{tikzpicture}}

\newcommand{\LargeReidemeisterThreeColouringB} {\begin{tikzpicture}[baseline=-0.65ex, scale=0.15001]
    \begin{knot}[clip width=7, end tolerance=1pt, flip crossing/.list={1,2,3}])
        \useasboundingbox (-10, -7) rectangle (10, 7); % REMOVE ME
        \strand[thick] (-5, -5) -- (5, 5);
        \strand[thick] (-5, 5) -- (5, -5);
        \strand[thick] (-5, 0) to [in=left, out=right] (0, -5);
        \strand[thick] (5, 0) to [in=right, out=left] (0, -5);
        \node[first_colour] at (-5, 4) [left] {$b$};
        \node[first_colour] at (5, 4) [right] {$c$};
        \node[first_colour] at (5, 0) [right] {$a$};
        \node[first_colour] at (-5, -5) [below] {$2a-2b+c$};
        \node[first_colour] at (5, -5) [below] {$2a-b$};
\draw[thick,red,fill=red] (-0, 0) circle (3);
    \end{knot}
\end{tikzpicture}}

\newcommand{\MedLarReidemeisterThreeColouringB} {\begin{tikzpicture}[baseline=-0.65ex, scale=0.1101]
    \begin{knot}[clip width=7, end tolerance=1pt, flip crossing/.list={1,2,3}])
        \useasboundingbox (-10, -7) rectangle (10, 7); % REMOVE ME
        \strand[thick] (-5, -5) -- (5, 5);
        \strand[thick] (-5, 5) -- (5, -5);
        \strand[thick] (-5, 0) to [in=left, out=right] (0, -5);
        \strand[thick] (5, 0) to [in=right, out=left] (0, -5);
        \node[first_colour] at (-5, 4) [left] {$b$};
        \node[first_colour] at (5, 4) [right] {$c$};
        \node[first_colour] at (5, 0) [right] {$a$};
        \node[first_colour] at (-5, -5) [below] {$2a-2b+c$};
        \node[first_colour] at (5, -5) [below] {$2a-b$};
\draw[thick,red,fill=red] (-0, 0) circle (3);
    \end{knot}
\end{tikzpicture}}

\newcommand{\MediumTemperleyA} {\begin{tikzpicture}[baseline=-0.65ex, scale=0.06001]
    \begin{knot}[clip width=7, end tolerance=1pt])
        \useasboundingbox (-5, -5) rectangle (5, 5); % REMOVE ME
        \draw[thick] (-5, -5) to (5, -5);
        \draw[thick] (-5, -0) to (5, +0);
        \draw[thick] (-5, +5) to (5, +5);
\draw[thick,red,fill=red] (-0, 0) circle (3);
    \end{knot}
\end{tikzpicture}}

\newcommand{\LargeTemperleyA} {\begin{tikzpicture}[baseline=-0.65ex, scale=0.15001]
    \begin{knot}[clip width=15, end tolerance=1pt])
        \useasboundingbox (-5, -5) rectangle (5, 5); % REMOVE ME
        \draw[thick] (-5, -5) to (5, -5);
        \draw[thick] (-5, -0) to (5, +0);
        \draw[thick] (-5, +5) to (5, +5);
\draw[thick,red,fill=red] (-0, 0) circle (3);
    \end{knot}
\end{tikzpicture}}

\newcommand{\MedLarTemperleyA} {\begin{tikzpicture}[baseline=-0.65ex, scale=0.1101]
    \begin{knot}[clip width=7, end tolerance=1pt])
        \useasboundingbox (-5, -5) rectangle (5, 5); % REMOVE ME
        \draw[thick] (-5, -5) to (5, -5);
        \draw[thick] (-5, -0) to (5, +0);
        \draw[thick] (-5, +5) to (5, +5);
\draw[thick,red,fill=red] (-0, 0) circle (3);
    \end{knot}
\end{tikzpicture}}

\newcommand{\MediumTemperleyB} {\begin{tikzpicture}[baseline=-0.65ex, scale=0.06001]
    \begin{knot}[clip width=7, end tolerance=1pt])
        \useasboundingbox (-5, -5) rectangle (5, 5); % REMOVE ME
        \draw[thick] (-5, -5) [in=down, out=right] to (-1, -2.5) [in=right, out=up] to (-5, -0);
        \draw[thick] (5, -5) [in=down, out=left] to (1, -2.5) [in=left, out=up] to (5, -0);
        \draw[thick] (-5, +5) to (5, +5);
\draw[thick,red,fill=red] (-0, 0) circle (3);
    \end{knot}
\end{tikzpicture}}

\newcommand{\LargeTemperleyB} {\begin{tikzpicture}[baseline=-0.65ex, scale=0.15001]
    \begin{knot}[clip width=15, end tolerance=1pt])
        \useasboundingbox (-5, -5) rectangle (5, 5); % REMOVE ME
        \draw[thick] (-5, -5) [in=down, out=right] to (-1, -2.5) [in=right, out=up] to (-5, -0);
        \draw[thick] (5, -5) [in=down, out=left] to (1, -2.5) [in=left, out=up] to (5, -0);
        \draw[thick] (-5, +5) to (5, +5);
\draw[thick,red,fill=red] (-0, 0) circle (3);
    \end{knot}
\end{tikzpicture}}

\newcommand{\MedLarTemperleyB} {\begin{tikzpicture}[baseline=-0.65ex, scale=0.1101]
    \begin{knot}[clip width=7, end tolerance=1pt])
        \useasboundingbox (-5, -5) rectangle (5, 5); % REMOVE ME
        \draw[thick] (-5, -5) [in=down, out=right] to (-1, -2.5) [in=right, out=up] to (-5, -0);
        \draw[thick] (5, -5) [in=down, out=left] to (1, -2.5) [in=left, out=up] to (5, -0);
        \draw[thick] (-5, +5) to (5, +5);
\draw[thick,red,fill=red] (-0, 0) circle (3);
    \end{knot}
\end{tikzpicture}}

\newcommand{\MediumTemperleyC} {\begin{tikzpicture}[baseline=-0.65ex, scale=0.06001]
    \begin{knot}[clip width=7, end tolerance=1pt])
        \useasboundingbox (-5, -5) rectangle (5, 5); % REMOVE ME
        \draw[thick] (-5, -5) to (5, -5);
        \draw[thick] (-5, 0) [in=down, out=right] to (-1, 2.5) [in=right, out=up] to (-5, 5);
        \draw[thick] (5, 0) [in=down, out=left] to (1, 2.5) [in=left, out=up] to (5, 5);
\draw[thick,red,fill=red] (-0, 0) circle (3);
    \end{knot}
\end{tikzpicture}}

\newcommand{\LargeTemperleyC} {\begin{tikzpicture}[baseline=-0.65ex, scale=0.15001]
    \begin{knot}[clip width=15, end tolerance=1pt])
        \useasboundingbox (-5, -5) rectangle (5, 5); % REMOVE ME
        \draw[thick] (-5, -5) to (5, -5);
        \draw[thick] (-5, 0) [in=down, out=right] to (-1, 2.5) [in=right, out=up] to (-5, 5);
        \draw[thick] (5, 0) [in=down, out=left] to (1, 2.5) [in=left, out=up] to (5, 5);
\draw[thick,red,fill=red] (-0, 0) circle (3);
    \end{knot}
\end{tikzpicture}}

\newcommand{\MedLarTemperleyC} {\begin{tikzpicture}[baseline=-0.65ex, scale=0.1101]
    \begin{knot}[clip width=7, end tolerance=1pt])
        \useasboundingbox (-5, -5) rectangle (5, 5); % REMOVE ME
        \draw[thick] (-5, -5) to (5, -5);
        \draw[thick] (-5, 0) [in=down, out=right] to (-1, 2.5) [in=right, out=up] to (-5, 5);
        \draw[thick] (5, 0) [in=down, out=left] to (1, 2.5) [in=left, out=up] to (5, 5);
\draw[thick,red,fill=red] (-0, 0) circle (3);
    \end{knot}
\end{tikzpicture}}

\newcommand{\MediumTemperleyD} {\begin{tikzpicture}[baseline=-0.65ex, scale=0.06001]
    \begin{knot}[clip width=7, end tolerance=1pt])
        \useasboundingbox (-5, -5) rectangle (5, 5); % REMOVE ME
        \draw[thick] (-5, -5) [in=left, out=right] to (5, 5);
        \draw[thick] (-5, 0) [in=down, out=right] to (-1, 2.5) [in=right, out=up] to (-5, 5);
        \draw[thick] (5, -5) [in=down, out=left] to (1, -2.5) [in=left, out=up] to (5, 0);
\draw[thick,red,fill=red] (-0, 0) circle (3);
    \end{knot}
\end{tikzpicture}}

\newcommand{\LargeTemperleyD} {\begin{tikzpicture}[baseline=-0.65ex, scale=0.15001]
    \begin{knot}[clip width=15, end tolerance=1pt])
        \useasboundingbox (-5, -5) rectangle (5, 5); % REMOVE ME
        \draw[thick] (-5, -5) [in=left, out=right] to (5, 5);
        \draw[thick] (-5, 0) [in=down, out=right] to (-1, 2.5) [in=right, out=up] to (-5, 5);
        \draw[thick] (5, -5) [in=down, out=left] to (1, -2.5) [in=left, out=up] to (5, 0);
\draw[thick,red,fill=red] (-0, 0) circle (3);
    \end{knot}
\end{tikzpicture}}

\newcommand{\MedLarTemperleyD} {\begin{tikzpicture}[baseline=-0.65ex, scale=0.1101]
    \begin{knot}[clip width=7, end tolerance=1pt])
        \useasboundingbox (-5, -5) rectangle (5, 5); % REMOVE ME
        \draw[thick] (-5, -5) [in=left, out=right] to (5, 5);
        \draw[thick] (-5, 0) [in=down, out=right] to (-1, 2.5) [in=right, out=up] to (-5, 5);
        \draw[thick] (5, -5) [in=down, out=left] to (1, -2.5) [in=left, out=up] to (5, 0);
\draw[thick,red,fill=red] (-0, 0) circle (3);
    \end{knot}
\end{tikzpicture}}

\newcommand{\MediumTemperleyE} {\begin{tikzpicture}[baseline=-0.65ex, scale=0.06001]
    \begin{knot}[clip width=7, end tolerance=1pt])
        \useasboundingbox (-5, -5) rectangle (5, 5); % REMOVE ME
        \draw[thick] (-5, 5) [in=left, out=right] to (5, -5);
        \draw[thick] (-5, 0) [in=up, out=right] to (-1, -2.5) [in=right, out=down] to (-5, -5);
        \draw[thick] (5, 5) [in=up, out=left] to (1, 2.5) [in=left, out=down] to (5, 0);
\draw[thick,red,fill=red] (-0, 0) circle (3);
    \end{knot}
\end{tikzpicture}}

\newcommand{\LargeTemperleyE} {\begin{tikzpicture}[baseline=-0.65ex, scale=0.15001]
    \begin{knot}[clip width=15, end tolerance=1pt])
        \useasboundingbox (-5, -5) rectangle (5, 5); % REMOVE ME
        \draw[thick] (-5, 5) [in=left, out=right] to (5, -5);
        \draw[thick] (-5, 0) [in=up, out=right] to (-1, -2.5) [in=right, out=down] to (-5, -5);
        \draw[thick] (5, 5) [in=up, out=left] to (1, 2.5) [in=left, out=down] to (5, 0);
\draw[thick,red,fill=red] (-0, 0) circle (3);
    \end{knot}
\end{tikzpicture}}

\newcommand{\MedLarTemperleyE} {\begin{tikzpicture}[baseline=-0.65ex, scale=0.1101]
    \begin{knot}[clip width=7, end tolerance=1pt])
        \useasboundingbox (-5, -5) rectangle (5, 5); % REMOVE ME
        \draw[thick] (-5, 5) [in=left, out=right] to (5, -5);
        \draw[thick] (-5, 0) [in=up, out=right] to (-1, -2.5) [in=right, out=down] to (-5, -5);
        \draw[thick] (5, 5) [in=up, out=left] to (1, 2.5) [in=left, out=down] to (5, 0);
\draw[thick,red,fill=red] (-0, 0) circle (3);
    \end{knot}
\end{tikzpicture}}

\newcommand{\MediumIsthmus} {\begin{tikzpicture}[baseline=-0.65ex, scale=0.06001]
    \begin{knot}[clip width=7, end tolerance=1pt])
        \useasboundingbox (-25, -5) rectangle (5, 5); % REMOVE ME
        \strand[semithick] (-5,-5) rectangle (5,5);
        \strand[semithick] (-5, -3) [in=right, out=left] to (-15, 3);
        \strand[semithick] (-5, 3) [in=right, out=left] to (-15, -3);
        \node at (-20, -3) {$\ldots$};
        \node at (-20,  3) {$\ldots$};
\draw[thick,red,fill=red] (-0, 0) circle (3);
    \end{knot}
\end{tikzpicture}}

\newcommand{\LargeIsthmus} {\begin{tikzpicture}[baseline=-0.65ex, scale=0.15001]
    \begin{knot}[clip width=15, end tolerance=1pt])
        \useasboundingbox (-25, -5) rectangle (5, 5); % REMOVE ME
        \strand[semithick] (-5,-5) rectangle (5,5);
        \strand[semithick] (-5, -3) [in=right, out=left] to (-15, 3);
        \strand[semithick] (-5, 3) [in=right, out=left] to (-15, -3);
        \node at (-20, -3) {$\ldots$};
        \node at (-20,  3) {$\ldots$};
\draw[thick,red,fill=red] (-0, 0) circle (3);
    \end{knot}
\end{tikzpicture}}

\newcommand{\MedLarIsthmus} {\begin{tikzpicture}[baseline=-0.65ex, scale=0.1101]
    \begin{knot}[clip width=7, end tolerance=1pt])
        \useasboundingbox (-25, -5) rectangle (5, 5); % REMOVE ME
        \strand[semithick] (-5,-5) rectangle (5,5);
        \strand[semithick] (-5, -3) [in=right, out=left] to (-15, 3);
        \strand[semithick] (-5, 3) [in=right, out=left] to (-15, -3);
        \node at (-20, -3) {$\ldots$};
        \node at (-20,  3) {$\ldots$};
\draw[thick,red,fill=red] (-0, 0) circle (3);
    \end{knot}
\end{tikzpicture}}

\newcommand{\MediumJonesShrapA} {\begin{tikzpicture}[baseline=-0.65ex, scale=0.06001]
    \begin{knot}[clip width=7, end tolerance=1pt, flip crossing/.list={1}])
        \useasboundingbox (-23, -10) rectangle (23, 10); % REMOVE ME
        \strand[thick] (-22, -10) rectangle (-12, 10);
        \strand[thick] (22, -10) rectangle (12, 10);
        \strand[thick,-latex] (12, -6) [in=right, out=left] to (6, -6) to (-6, 6) to (-12, 6);
        \strand[thick,-latex] (-12, -6) [in=left, out=right] to (-6, -6) to (6, 6) to (12, 6);
        \node at (-17, 0) {$K_1$};
        \node at (17, 0) {$K_2$};
\draw[thick,red,fill=red] (-0, 0) circle (3);
    \end{knot}
\end{tikzpicture}}

\newcommand{\LargeJonesShrapA} {\begin{tikzpicture}[baseline=-0.65ex, scale=0.15001]
    \begin{knot}[clip width=15, end tolerance=1pt, flip crossing/.list={1}])
        \useasboundingbox (-23, -10) rectangle (23, 10); % REMOVE ME
        \strand[thick] (-22, -10) rectangle (-12, 10);
        \strand[thick] (22, -10) rectangle (12, 10);
        \strand[thick,-latex] (12, -6) [in=right, out=left] to (6, -6) to (-6, 6) to (-12, 6);
        \strand[thick,-latex] (-12, -6) [in=left, out=right] to (-6, -6) to (6, 6) to (12, 6);
        \node at (-17, 0) {$K_1$};
        \node at (17, 0) {$K_2$};
\draw[thick,red,fill=red] (-0, 0) circle (3);
    \end{knot}
\end{tikzpicture}}

\newcommand{\MedLarJonesShrapA} {\begin{tikzpicture}[baseline=-0.65ex, scale=0.1101]
    \begin{knot}[clip width=7, end tolerance=1pt, flip crossing/.list={1}])
        \useasboundingbox (-23, -10) rectangle (23, 10); % REMOVE ME
        \strand[thick] (-22, -10) rectangle (-12, 10);
        \strand[thick] (22, -10) rectangle (12, 10);
        \strand[thick,-latex] (12, -6) [in=right, out=left] to (6, -6) to (-6, 6) to (-12, 6);
        \strand[thick,-latex] (-12, -6) [in=left, out=right] to (-6, -6) to (6, 6) to (12, 6);
        \node at (-17, 0) {$K_1$};
        \node at (17, 0) {$K_2$};
\draw[thick,red,fill=red] (-0, 0) circle (3);
    \end{knot}
\end{tikzpicture}}

\newcommand{\MediumJonesShrapB} {\begin{tikzpicture}[baseline=-0.65ex, scale=0.06001]
    \begin{knot}[clip width=7, end tolerance=1pt])
        \useasboundingbox (-23, -10) rectangle (23, 10); % REMOVE ME
        \strand[thick] (-22, -10) rectangle (-12, 10);
        \strand[thick] (22, -10) rectangle (12, 10);
        \strand[thick,-latex] (12, -6) [in=right, out=left] to (6, -6) to (-6, 6) to (-12, 6);
        \strand[thick,-latex] (-12, -6) [in=left, out=right] to (-6, -6) to (6, 6) to (12, 6);
        \node at (-17, 0) {$K_1$};
        \node at (17, 0) {$K_2$};
\draw[thick,red,fill=red] (-0, 0) circle (3);
    \end{knot}
\end{tikzpicture}}

\newcommand{\LargeJonesShrapB} {\begin{tikzpicture}[baseline=-0.65ex, scale=0.15001]
    \begin{knot}[clip width=15, end tolerance=1pt])
        \useasboundingbox (-23, -10) rectangle (23, 10); % REMOVE ME
        \strand[thick] (-22, -10) rectangle (-12, 10);
        \strand[thick] (22, -10) rectangle (12, 10);
        \strand[thick,-latex] (12, -6) [in=right, out=left] to (6, -6) to (-6, 6) to (-12, 6);
        \strand[thick,-latex] (-12, -6) [in=left, out=right] to (-6, -6) to (6, 6) to (12, 6);
        \node at (-17, 0) {$K_1$};
        \node at (17, 0) {$K_2$};
\draw[thick,red,fill=red] (-0, 0) circle (3);
    \end{knot}
\end{tikzpicture}}

\newcommand{\MedLarJonesShrapB} {\begin{tikzpicture}[baseline=-0.65ex, scale=0.1101]
    \begin{knot}[clip width=7, end tolerance=1pt])
        \useasboundingbox (-23, -10) rectangle (23, 10); % REMOVE ME
        \strand[thick] (-22, -10) rectangle (-12, 10);
        \strand[thick] (22, -10) rectangle (12, 10);
        \strand[thick,-latex] (12, -6) [in=right, out=left] to (6, -6) to (-6, 6) to (-12, 6);
        \strand[thick,-latex] (-12, -6) [in=left, out=right] to (-6, -6) to (6, 6) to (12, 6);
        \node at (-17, 0) {$K_1$};
        \node at (17, 0) {$K_2$};
\draw[thick,red,fill=red] (-0, 0) circle (3);
    \end{knot}
\end{tikzpicture}}

\newcommand{\MediumJonesShrapAB} {\begin{tikzpicture}[baseline=-0.65ex, scale=0.06001]
    \begin{knot}[clip width=7, end tolerance=1pt])
        \useasboundingbox (-23, -10) rectangle (23, 10); % REMOVE ME
        \strand[thick] (-22, -10) rectangle (-12, 10);
        \strand[thick] (-12, -6) [in=down, out=right] to (-2, 0);
        \strand[thick,Latex-] (-12, 6) [in=up, out=right] to (-2, 0);
        \strand[thick] (22, -10) rectangle (12, 10);
        \strand[thick] (12, -6) [in=down, out=left] to (2, 0);
        \strand[thick,Latex-] (12, 6) [in=up, out=left] to (2, 0);
        \node at (-17, 0) {$K_1$};
        \node at (17, 0) {$K_2$};
\draw[thick,red,fill=red] (-0, 0) circle (3);
    \end{knot}
\end{tikzpicture}}

\newcommand{\LargeJonesShrapAB} {\begin{tikzpicture}[baseline=-0.65ex, scale=0.15001]
    \begin{knot}[clip width=15, end tolerance=1pt])
        \useasboundingbox (-23, -10) rectangle (23, 10); % REMOVE ME
        \strand[thick] (-22, -10) rectangle (-12, 10);
        \strand[thick] (-12, -6) [in=down, out=right] to (-2, 0);
        \strand[thick,Latex-] (-12, 6) [in=up, out=right] to (-2, 0);
        \strand[thick] (22, -10) rectangle (12, 10);
        \strand[thick] (12, -6) [in=down, out=left] to (2, 0);
        \strand[thick,Latex-] (12, 6) [in=up, out=left] to (2, 0);
        \node at (-17, 0) {$K_1$};
        \node at (17, 0) {$K_2$};
\draw[thick,red,fill=red] (-0, 0) circle (3);
    \end{knot}
\end{tikzpicture}}

\newcommand{\MedLarJonesShrapAB} {\begin{tikzpicture}[baseline=-0.65ex, scale=0.1101]
    \begin{knot}[clip width=7, end tolerance=1pt])
        \useasboundingbox (-23, -10) rectangle (23, 10); % REMOVE ME
        \strand[thick] (-22, -10) rectangle (-12, 10);
        \strand[thick] (-12, -6) [in=down, out=right] to (-2, 0);
        \strand[thick,Latex-] (-12, 6) [in=up, out=right] to (-2, 0);
        \strand[thick] (22, -10) rectangle (12, 10);
        \strand[thick] (12, -6) [in=down, out=left] to (2, 0);
        \strand[thick,Latex-] (12, 6) [in=up, out=left] to (2, 0);
        \node at (-17, 0) {$K_1$};
        \node at (17, 0) {$K_2$};
\draw[thick,red,fill=red] (-0, 0) circle (3);
    \end{knot}
\end{tikzpicture}}

\newcommand{\MediumKauffmanReidemeisterTwoA} {\begin{tikzpicture}[baseline=-0.65ex, scale=0.06001]
    \begin{knot}[clip width=7, end tolerance=1pt])
        \useasboundingbox (-7, -5) rectangle (7, 5); % REMOVE ME
        \strand[thick] (5, -5) .. controls (5, -2) and (-5, -2) .. (-5, 0);
        \strand[thick] (5, 5) .. controls (5, 2) and (-5, 2) .. (-5, 0);
        \strand[thick] (-5, -5) .. controls (-5, -2) and (5, -2) .. (5, 0);
        \strand[thick] (-5, 5) .. controls (-5, 2) and (5, 2) .. (5, 0);
\draw[thick,red,fill=red] (-0, 0) circle (3);
    \end{knot}
\end{tikzpicture}}

\newcommand{\LargeKauffmanReidemeisterTwoA} {\begin{tikzpicture}[baseline=-0.65ex, scale=0.15001]
    \begin{knot}[clip width=15, end tolerance=1pt])
        \useasboundingbox (-7, -5) rectangle (7, 5); % REMOVE ME
        \strand[thick] (5, -5) .. controls (5, -2) and (-5, -2) .. (-5, 0);
        \strand[thick] (5, 5) .. controls (5, 2) and (-5, 2) .. (-5, 0);
        \strand[thick] (-5, -5) .. controls (-5, -2) and (5, -2) .. (5, 0);
        \strand[thick] (-5, 5) .. controls (-5, 2) and (5, 2) .. (5, 0);
\draw[thick,red,fill=red] (-0, 0) circle (3);
    \end{knot}
\end{tikzpicture}}

\newcommand{\MedLarKauffmanReidemeisterTwoA} {\begin{tikzpicture}[baseline=-0.65ex, scale=0.1101]
    \begin{knot}[clip width=7, end tolerance=1pt])
        \useasboundingbox (-7, -5) rectangle (7, 5); % REMOVE ME
        \strand[thick] (5, -5) .. controls (5, -2) and (-5, -2) .. (-5, 0);
        \strand[thick] (5, 5) .. controls (5, 2) and (-5, 2) .. (-5, 0);
        \strand[thick] (-5, -5) .. controls (-5, -2) and (5, -2) .. (5, 0);
        \strand[thick] (-5, 5) .. controls (-5, 2) and (5, 2) .. (5, 0);
\draw[thick,red,fill=red] (-0, 0) circle (3);
    \end{knot}
\end{tikzpicture}}

\newcommand{\MediumKauffmanReidemeisterTwoB} {\begin{tikzpicture}[baseline=-0.65ex, scale=0.06001]
    \begin{knot}[clip width=7, end tolerance=1pt])
        \useasboundingbox (-7, -5) rectangle (7, 5); % REMOVE ME
        \strand[thick] (5, -5) .. controls (5, -3) and (-5, -3) .. (-5, -1);
        \strand[thick] (-5, -5) .. controls (-5, -3) and (5, -3) .. (5, -1);
        \strand[thick] (-5, -1) [in=left, out=up] to (0, 1) to [in=up, out=right] (5, -1);
        \strand[thick] (-5, 5) [in=left, out=down] to (0, 3) to [in=down, out=right] (5, 5);
\draw[thick,red,fill=red] (-0, 0) circle (3);
    \end{knot}
\end{tikzpicture}}

\newcommand{\LargeKauffmanReidemeisterTwoB} {\begin{tikzpicture}[baseline=-0.65ex, scale=0.15001]
    \begin{knot}[clip width=15, end tolerance=1pt])
        \useasboundingbox (-7, -5) rectangle (7, 5); % REMOVE ME
        \strand[thick] (5, -5) .. controls (5, -3) and (-5, -3) .. (-5, -1);
        \strand[thick] (-5, -5) .. controls (-5, -3) and (5, -3) .. (5, -1);
        \strand[thick] (-5, -1) [in=left, out=up] to (0, 1) to [in=up, out=right] (5, -1);
        \strand[thick] (-5, 5) [in=left, out=down] to (0, 3) to [in=down, out=right] (5, 5);
\draw[thick,red,fill=red] (-0, 0) circle (3);
    \end{knot}
\end{tikzpicture}}

\newcommand{\MedLarKauffmanReidemeisterTwoB} {\begin{tikzpicture}[baseline=-0.65ex, scale=0.1101]
    \begin{knot}[clip width=7, end tolerance=1pt])
        \useasboundingbox (-7, -5) rectangle (7, 5); % REMOVE ME
        \strand[thick] (5, -5) .. controls (5, -3) and (-5, -3) .. (-5, -1);
        \strand[thick] (-5, -5) .. controls (-5, -3) and (5, -3) .. (5, -1);
        \strand[thick] (-5, -1) [in=left, out=up] to (0, 1) to [in=up, out=right] (5, -1);
        \strand[thick] (-5, 5) [in=left, out=down] to (0, 3) to [in=down, out=right] (5, 5);
\draw[thick,red,fill=red] (-0, 0) circle (3);
    \end{knot}
\end{tikzpicture}}

\newcommand{\MediumKauffmanReidemeisterTwoC} {\begin{tikzpicture}[baseline=-0.65ex, scale=0.06001]
    \begin{knot}[clip width=7, end tolerance=1pt])
        \useasboundingbox (-7, -5) rectangle (7, 5); % REMOVE ME
        \strand[thick] (5, -5) .. controls (5, -2) and (-5, -2) .. (-5, 0);
        \strand[thick] (5, 5) to (5, 0);
        \strand[thick] (-5, -5) .. controls (-5, -2) and (5, -2) .. (5, 0);
        \strand[thick] (-5, 5) to (-5, 0);
\draw[thick,red,fill=red] (-0, 0) circle (3);
    \end{knot}
\end{tikzpicture}}

\newcommand{\LargeKauffmanReidemeisterTwoC} {\begin{tikzpicture}[baseline=-0.65ex, scale=0.15001]
    \begin{knot}[clip width=15, end tolerance=1pt])
        \useasboundingbox (-7, -5) rectangle (7, 5); % REMOVE ME
        \strand[thick] (5, -5) .. controls (5, -2) and (-5, -2) .. (-5, 0);
        \strand[thick] (5, 5) to (5, 0);
        \strand[thick] (-5, -5) .. controls (-5, -2) and (5, -2) .. (5, 0);
        \strand[thick] (-5, 5) to (-5, 0);
\draw[thick,red,fill=red] (-0, 0) circle (3);
    \end{knot}
\end{tikzpicture}}

\newcommand{\MedLarKauffmanReidemeisterTwoC} {\begin{tikzpicture}[baseline=-0.65ex, scale=0.1101]
    \begin{knot}[clip width=7, end tolerance=1pt])
        \useasboundingbox (-7, -5) rectangle (7, 5); % REMOVE ME
        \strand[thick] (5, -5) .. controls (5, -2) and (-5, -2) .. (-5, 0);
        \strand[thick] (5, 5) to (5, 0);
        \strand[thick] (-5, -5) .. controls (-5, -2) and (5, -2) .. (5, 0);
        \strand[thick] (-5, 5) to (-5, 0);
\draw[thick,red,fill=red] (-0, 0) circle (3);
    \end{knot}
\end{tikzpicture}}

\newcommand{\MediumKauffmanReidemeisterThreeA} {\begin{tikzpicture}[baseline=-0.65ex, scale=0.06001]
    \begin{knot}[clip width=5, end tolerance=1pt, flip crossing/.list={1,2,3}])
        \useasboundingbox (-7, -5) rectangle (7, 5); % REMOVE ME
        \strand[thick] (-5, -5) -- (5, 5);
        \strand[thick] (-5, 5) -- (5, -5);
        \strand[thick] (-5, 0) .. controls (-3, 0) and (-3, 5) .. (0, 5) .. controls (3, 5) and (3, 0) .. (5, 0);
\draw[thick,red,fill=red] (-0, 0) circle (3);
    \end{knot}
\end{tikzpicture}}

\newcommand{\LargeKauffmanReidemeisterThreeA} {\begin{tikzpicture}[baseline=-0.65ex, scale=0.15001]
    \begin{knot}[clip width=5, end tolerance=1pt, flip crossing/.list={1,2,3}])
        \useasboundingbox (-7, -5) rectangle (7, 5); % REMOVE ME
        \strand[thick] (-5, -5) -- (5, 5);
        \strand[thick] (-5, 5) -- (5, -5);
        \strand[thick] (-5, 0) .. controls (-3, 0) and (-3, 5) .. (0, 5) .. controls (3, 5) and (3, 0) .. (5, 0);
\draw[thick,red,fill=red] (-0, 0) circle (3);
    \end{knot}
\end{tikzpicture}}

\newcommand{\MedLarKauffmanReidemeisterThreeA} {\begin{tikzpicture}[baseline=-0.65ex, scale=0.1101]
    \begin{knot}[clip width=5, end tolerance=1pt, flip crossing/.list={1,2,3}])
        \useasboundingbox (-7, -5) rectangle (7, 5); % REMOVE ME
        \strand[thick] (-5, -5) -- (5, 5);
        \strand[thick] (-5, 5) -- (5, -5);
        \strand[thick] (-5, 0) .. controls (-3, 0) and (-3, 5) .. (0, 5) .. controls (3, 5) and (3, 0) .. (5, 0);
\draw[thick,red,fill=red] (-0, 0) circle (3);
    \end{knot}
\end{tikzpicture}}

\newcommand{\MediumKauffmanReidemeisterThreeFlippedA} {\begin{tikzpicture}[baseline=-0.65ex, scale=0.06001]
    \begin{knot}[clip width=5, end tolerance=1pt, flip crossing/.list={1,2,3}])
        \useasboundingbox (-7, -5) rectangle (7, 5); % REMOVE ME
        \strand[thick] (-5, -5) -- (5, 5);
        \strand[thick] (-5, 5) -- (5, -5);
        \strand[thick] (-5, 0) .. controls (-3, 0) and (-3, -5) .. (0, -5) .. controls (3, -5) and (3, 0) .. (5, 0);
\draw[thick,red,fill=red] (-0, 0) circle (3);
    \end{knot}
\end{tikzpicture}}

\newcommand{\LargeKauffmanReidemeisterThreeFlippedA} {\begin{tikzpicture}[baseline=-0.65ex, scale=0.15001]
    \begin{knot}[clip width=5, end tolerance=1pt, flip crossing/.list={1,2,3}])
        \useasboundingbox (-7, -5) rectangle (7, 5); % REMOVE ME
        \strand[thick] (-5, -5) -- (5, 5);
        \strand[thick] (-5, 5) -- (5, -5);
        \strand[thick] (-5, 0) .. controls (-3, 0) and (-3, -5) .. (0, -5) .. controls (3, -5) and (3, 0) .. (5, 0);
\draw[thick,red,fill=red] (-0, 0) circle (3);
    \end{knot}
\end{tikzpicture}}

\newcommand{\MedLarKauffmanReidemeisterThreeFlippedA} {\begin{tikzpicture}[baseline=-0.65ex, scale=0.1101]
    \begin{knot}[clip width=5, end tolerance=1pt, flip crossing/.list={1,2,3}])
        \useasboundingbox (-7, -5) rectangle (7, 5); % REMOVE ME
        \strand[thick] (-5, -5) -- (5, 5);
        \strand[thick] (-5, 5) -- (5, -5);
        \strand[thick] (-5, 0) .. controls (-3, 0) and (-3, -5) .. (0, -5) .. controls (3, -5) and (3, 0) .. (5, 0);
\draw[thick,red,fill=red] (-0, 0) circle (3);
    \end{knot}
\end{tikzpicture}}

\newcommand{\MediumKauffmanReidemeisterThreeB} {\begin{tikzpicture}[baseline=-0.65ex, scale=0.06001]
    \begin{knot}[clip width=5, end tolerance=1pt, flip crossing/.list={1,2,3}])
        \useasboundingbox (-7, -5) rectangle (7, 5); % REMOVE ME
        \strand[thick] (-5, 5) [in=-120, out=-60] to (5, 5);
        \strand[thick] (-5, -5) [in=120, out=60] to (5, -5);
        \strand[thick] (-5, 0) .. controls (-3, 0) and (-3, 5) .. (0, 5) .. controls (3, 5) and (3, 0) .. (5, 0);
\draw[thick,red,fill=red] (-0, 0) circle (3);
    \end{knot}
\end{tikzpicture}}

\newcommand{\LargeKauffmanReidemeisterThreeB} {\begin{tikzpicture}[baseline=-0.65ex, scale=0.15001]
    \begin{knot}[clip width=5, end tolerance=1pt, flip crossing/.list={1,2,3}])
        \useasboundingbox (-7, -5) rectangle (7, 5); % REMOVE ME
        \strand[thick] (-5, 5) [in=-120, out=-60] to (5, 5);
        \strand[thick] (-5, -5) [in=120, out=60] to (5, -5);
        \strand[thick] (-5, 0) .. controls (-3, 0) and (-3, 5) .. (0, 5) .. controls (3, 5) and (3, 0) .. (5, 0);
\draw[thick,red,fill=red] (-0, 0) circle (3);
    \end{knot}
\end{tikzpicture}}

\newcommand{\MedLarKauffmanReidemeisterThreeB} {\begin{tikzpicture}[baseline=-0.65ex, scale=0.1101]
    \begin{knot}[clip width=5, end tolerance=1pt, flip crossing/.list={1,2,3}])
        \useasboundingbox (-7, -5) rectangle (7, 5); % REMOVE ME
        \strand[thick] (-5, 5) [in=-120, out=-60] to (5, 5);
        \strand[thick] (-5, -5) [in=120, out=60] to (5, -5);
        \strand[thick] (-5, 0) .. controls (-3, 0) and (-3, 5) .. (0, 5) .. controls (3, 5) and (3, 0) .. (5, 0);
\draw[thick,red,fill=red] (-0, 0) circle (3);
    \end{knot}
\end{tikzpicture}}

\newcommand{\MediumKauffmanReidemeisterThreeFlippedB} {\begin{tikzpicture}[baseline=-0.65ex, scale=0.06001]
    \begin{knot}[clip width=5, end tolerance=1pt, flip crossing/.list={1,2,3}])
        \useasboundingbox (-7, -5) rectangle (7, 5); % REMOVE ME
        \strand[thick] (-5, 5) [in=-120, out=-60] to (5, 5);
        \strand[thick] (-5, -5) [in=120, out=60] to (5, -5);
        \strand[thick] (-5, 0) .. controls (-3, 0) and (-3, -5) .. (0, -5) .. controls (3, -5) and (3, 0) .. (5, 0);
\draw[thick,red,fill=red] (-0, 0) circle (3);
    \end{knot}
\end{tikzpicture}}

\newcommand{\LargeKauffmanReidemeisterThreeFlippedB} {\begin{tikzpicture}[baseline=-0.65ex, scale=0.15001]
    \begin{knot}[clip width=5, end tolerance=1pt, flip crossing/.list={1,2,3}])
        \useasboundingbox (-7, -5) rectangle (7, 5); % REMOVE ME
        \strand[thick] (-5, 5) [in=-120, out=-60] to (5, 5);
        \strand[thick] (-5, -5) [in=120, out=60] to (5, -5);
        \strand[thick] (-5, 0) .. controls (-3, 0) and (-3, -5) .. (0, -5) .. controls (3, -5) and (3, 0) .. (5, 0);
\draw[thick,red,fill=red] (-0, 0) circle (3);
    \end{knot}
\end{tikzpicture}}

\newcommand{\MedLarKauffmanReidemeisterThreeFlippedB} {\begin{tikzpicture}[baseline=-0.65ex, scale=0.1101]
    \begin{knot}[clip width=5, end tolerance=1pt, flip crossing/.list={1,2,3}])
        \useasboundingbox (-7, -5) rectangle (7, 5); % REMOVE ME
        \strand[thick] (-5, 5) [in=-120, out=-60] to (5, 5);
        \strand[thick] (-5, -5) [in=120, out=60] to (5, -5);
        \strand[thick] (-5, 0) .. controls (-3, 0) and (-3, -5) .. (0, -5) .. controls (3, -5) and (3, 0) .. (5, 0);
\draw[thick,red,fill=red] (-0, 0) circle (3);
    \end{knot}
\end{tikzpicture}}

\newcommand{\MediumKauffmanReidemeisterThreeC} {\begin{tikzpicture}[baseline=-0.65ex, scale=0.06001]
    \begin{knot}[clip width=5, end tolerance=1pt, flip crossing/.list={1,2,3}])
        \useasboundingbox (-7, -5) rectangle (7, 5); % REMOVE ME
        \strand[thick] (-5, -5) to [out=30, in=-30] (-5, 5);
        \strand[thick] (5, -5) to [out=150, in=-150] (5, 5);
        \strand[thick] (-6, 0) .. controls (-3, 0) and (-3, 5) .. (0, 5) .. controls (3, 5) and (3, 0) .. (6, 0);  
\draw[thick,red,fill=red] (-0, 0) circle (3);
    \end{knot}
\end{tikzpicture}}

\newcommand{\LargeKauffmanReidemeisterThreeC} {\begin{tikzpicture}[baseline=-0.65ex, scale=0.15001]
    \begin{knot}[clip width=5, end tolerance=1pt, flip crossing/.list={1,2,3}])
        \useasboundingbox (-7, -5) rectangle (7, 5); % REMOVE ME
        \strand[thick] (-5, -5) to [out=30, in=-30] (-5, 5);
        \strand[thick] (5, -5) to [out=150, in=-150] (5, 5);
        \strand[thick] (-6, 0) .. controls (-3, 0) and (-3, 5) .. (0, 5) .. controls (3, 5) and (3, 0) .. (6, 0);  
\draw[thick,red,fill=red] (-0, 0) circle (3);
    \end{knot}
\end{tikzpicture}}

\newcommand{\MedLarKauffmanReidemeisterThreeC} {\begin{tikzpicture}[baseline=-0.65ex, scale=0.1101]
    \begin{knot}[clip width=5, end tolerance=1pt, flip crossing/.list={1,2,3}])
        \useasboundingbox (-7, -5) rectangle (7, 5); % REMOVE ME
        \strand[thick] (-5, -5) to [out=30, in=-30] (-5, 5);
        \strand[thick] (5, -5) to [out=150, in=-150] (5, 5);
        \strand[thick] (-6, 0) .. controls (-3, 0) and (-3, 5) .. (0, 5) .. controls (3, 5) and (3, 0) .. (6, 0);  
\draw[thick,red,fill=red] (-0, 0) circle (3);
    \end{knot}
\end{tikzpicture}}

\newcommand{\MediumKauffmanReidemeisterThreeFlippedC} {\begin{tikzpicture}[baseline=-0.65ex, scale=0.06001]
    \begin{knot}[clip width=5, end tolerance=1pt, flip crossing/.list={1,2,3}])
        \useasboundingbox (-7, -5) rectangle (7, 5); % REMOVE ME
        \strand[thick] (-5, -5) to [out=30, in=-30] (-5, 5);
        \strand[thick] (5, -5) to [out=150, in=-150] (5, 5);
        \strand[thick] (-6, 0) .. controls (-3, 0) and (-3, -5) .. (0, -5) .. controls (3, -5) and (3, 0) .. (6, 0); 
\draw[thick,red,fill=red] (-0, 0) circle (3);
    \end{knot}
\end{tikzpicture}}

\newcommand{\LargeKauffmanReidemeisterThreeFlippedC} {\begin{tikzpicture}[baseline=-0.65ex, scale=0.15001]
    \begin{knot}[clip width=5, end tolerance=1pt, flip crossing/.list={1,2,3}])
        \useasboundingbox (-7, -5) rectangle (7, 5); % REMOVE ME
        \strand[thick] (-5, -5) to [out=30, in=-30] (-5, 5);
        \strand[thick] (5, -5) to [out=150, in=-150] (5, 5);
        \strand[thick] (-6, 0) .. controls (-3, 0) and (-3, -5) .. (0, -5) .. controls (3, -5) and (3, 0) .. (6, 0); 
\draw[thick,red,fill=red] (-0, 0) circle (3);
    \end{knot}
\end{tikzpicture}}

\newcommand{\MedLarKauffmanReidemeisterThreeFlippedC} {\begin{tikzpicture}[baseline=-0.65ex, scale=0.1101]
    \begin{knot}[clip width=5, end tolerance=1pt, flip crossing/.list={1,2,3}])
        \useasboundingbox (-7, -5) rectangle (7, 5); % REMOVE ME
        \strand[thick] (-5, -5) to [out=30, in=-30] (-5, 5);
        \strand[thick] (5, -5) to [out=150, in=-150] (5, 5);
        \strand[thick] (-6, 0) .. controls (-3, 0) and (-3, -5) .. (0, -5) .. controls (3, -5) and (3, 0) .. (6, 0); 
\draw[thick,red,fill=red] (-0, 0) circle (3);
    \end{knot}
\end{tikzpicture}}

\newcommand{\MediumKauffmanReidemeisterThreeD} {\begin{tikzpicture}[baseline=-0.65ex, scale=0.06001]
    \begin{knot}[clip width=5, end tolerance=1pt, flip crossing/.list={1,2,3}])
        \useasboundingbox (-7, -5) rectangle (7, 5); % REMOVE ME
        \strand[thick] (-5, 5) [in=-120, out=-60] to (5, 5);
        \strand[thick] (-5, -5) [in=120, out=60] to (5, -5);
        \strand[thick] (-5, 0) to (5, 0);
\draw[thick,red,fill=red] (-0, 0) circle (3);
    \end{knot}
\end{tikzpicture}}

\newcommand{\LargeKauffmanReidemeisterThreeD} {\begin{tikzpicture}[baseline=-0.65ex, scale=0.15001]
    \begin{knot}[clip width=5, end tolerance=1pt, flip crossing/.list={1,2,3}])
        \useasboundingbox (-7, -5) rectangle (7, 5); % REMOVE ME
        \strand[thick] (-5, 5) [in=-120, out=-60] to (5, 5);
        \strand[thick] (-5, -5) [in=120, out=60] to (5, -5);
        \strand[thick] (-5, 0) to (5, 0);
\draw[thick,red,fill=red] (-0, 0) circle (3);
    \end{knot}
\end{tikzpicture}}

\newcommand{\MedLarKauffmanReidemeisterThreeD} {\begin{tikzpicture}[baseline=-0.65ex, scale=0.1101]
    \begin{knot}[clip width=5, end tolerance=1pt, flip crossing/.list={1,2,3}])
        \useasboundingbox (-7, -5) rectangle (7, 5); % REMOVE ME
        \strand[thick] (-5, 5) [in=-120, out=-60] to (5, 5);
        \strand[thick] (-5, -5) [in=120, out=60] to (5, -5);
        \strand[thick] (-5, 0) to (5, 0);
\draw[thick,red,fill=red] (-0, 0) circle (3);
    \end{knot}
\end{tikzpicture}}

\newcommand{\MediumKauffmanReidemeisterThreeE} {\begin{tikzpicture}[baseline=-0.65ex, scale=0.06001]
    \begin{knot}[clip width=5, end tolerance=1pt, flip crossing/.list={1,2,3}])
        \useasboundingbox (-7, -5) rectangle (7, 5); % REMOVE ME
        \strand[thick] (-5, -5) to [out=30, in=-30] (-5, 5);
        \strand[thick] (5, -5) to [out=150, in=-150] (5, 5);
        \strand[thick] (-6, 0) to (6, 0);
\draw[thick,red,fill=red] (-0, 0) circle (3);
    \end{knot}
\end{tikzpicture}}

\newcommand{\LargeKauffmanReidemeisterThreeE} {\begin{tikzpicture}[baseline=-0.65ex, scale=0.15001]
    \begin{knot}[clip width=5, end tolerance=1pt, flip crossing/.list={1,2,3}])
        \useasboundingbox (-7, -5) rectangle (7, 5); % REMOVE ME
        \strand[thick] (-5, -5) to [out=30, in=-30] (-5, 5);
        \strand[thick] (5, -5) to [out=150, in=-150] (5, 5);
        \strand[thick] (-6, 0) to (6, 0);
\draw[thick,red,fill=red] (-0, 0) circle (3);
    \end{knot}
\end{tikzpicture}}

\newcommand{\MedLarKauffmanReidemeisterThreeE} {\begin{tikzpicture}[baseline=-0.65ex, scale=0.1101]
    \begin{knot}[clip width=5, end tolerance=1pt, flip crossing/.list={1,2,3}])
        \useasboundingbox (-7, -5) rectangle (7, 5); % REMOVE ME
        \strand[thick] (-5, -5) to [out=30, in=-30] (-5, 5);
        \strand[thick] (5, -5) to [out=150, in=-150] (5, 5);
        \strand[thick] (-6, 0) to (6, 0);
\draw[thick,red,fill=red] (-0, 0) circle (3);
    \end{knot}
\end{tikzpicture}}



%%%%%%%%%% UNKNOT START %%%%%%%%%%

%%%%%%%%%% UNKNOT END %%%%%%%%%%

%%%%%%%%%% LEFT/RIGHT, BIG/SMALL START %%%%%%%%%%

% \LeftCrossing - for example in crossing sign definition, where nothing besides the crossing is shown
% \SREDNIAWYLeftCrossing - for skein relations
% \LittleLeftCrossing - for inline usage

\newcommand{\LeftCrossing} {\begin{tikzpicture}[baseline=-0.65ex, scale=0.1]
    \begin{knot}[clip width=5, end tolerance=1pt]
        \strand[red,thick,] (-5, -5) to (5, 5);
        \strand[red,thick,] (-5, 5) to (5, -5);
    \end{knot}
\end{tikzpicture}}

\newcommand{\RightCrossing} {\begin{tikzpicture}[baseline=-0.65ex, scale=0.1]
    \begin{knot}[clip width=5, end tolerance=1pt, flip crossing/.list={1}]
        \strand[red,thick,] (-5, -5) to (5, 5);
        \strand[red,thick,] (-5, 5) to (5, -5);
    \end{knot}
\end{tikzpicture}}

\newcommand{\LittleLeftCrossing} {\begin{tikzpicture}[baseline=-0.65ex, scale=0.03]
    \useasboundingbox (-5, -5) rectangle (5, 5);
    \begin{knot}[clip width=5, end tolerance=1pt]
        \strand[red,thick,] (-5, -5) to (5, 5);
        \strand[red,thick,] (-5, 5) to (5, -5);
    \end{knot}
    \end{tikzpicture}
}

\newcommand{\LittleRightCrossing} {\begin{tikzpicture}[baseline=-0.65ex, scale=0.03]
    \useasboundingbox (-5, -5) rectangle (5, 5);
    \begin{knot}[clip width=5, end tolerance=1pt, flip crossing/.list={1}]
        \strand[red,thick,] (-5, -5) to (5, 5);
        \strand[red,thick,] (-5, 5) to (5, -5);
    \end{knot}
\end{tikzpicture}}

%%%%%%%%%% LEFT/RIGHT, BIG/SMALL  END %%%%%%%%%%

%%%%%%%%%% LEFT/RIGHT SMOOTHING, BIG/SMALL START %%%%%%%%%%

\newcommand{\LittleLeftSmoothing} {\begin{tikzpicture}[baseline=-0.65ex,scale=0.03]
    \useasboundingbox (-5, -5) rectangle (5, 5);
    \begin{knot}[clip width=5, end tolerance=1pt]
        \strand[red,thick,] (-5, -5) [in=135, out=45] to (5, -5);
        \strand[red,thick,] (-5, 5) [in=-135, out=-45] to (5, 5);
    \end{knot}
\end{tikzpicture}}

\newcommand{\LittleRightSmoothing} {\begin{tikzpicture}[baseline=-0.65ex,scale=0.03]
    \useasboundingbox (-5, -5) rectangle (5, 5);
    \begin{knot}[clip width=5, end tolerance=1pt]
        \strand[red,thick,] (-4, -5) to [out=45, in=-45] (-4, 5);
        \strand[red,thick,] (4, -5) to [out=135, in=-135] (4, 5);
    \end{knot}
\end{tikzpicture}}

%%%%%%%%%% LEFT/RIGHT SMOOTHING, BIG/SMALL END %%%%%%%%%%

%%%%%%%%%% REIDEMEISTER-1 START %%%%%%%%%%

%%%%%%%%%% REIDEMEISTER-1 END %%%%%%%%%%

%%%%%%%%%% REIDEMEISTER-2 START %%%%%%%%%%

\newcommand{\VirtualOmegaVa} {\begin{tikzpicture}[baseline=-0.65ex,scale=0.15]
    \useasboundingbox (-7, -7) rectangle (7, 7);
    \begin{knot}[clip width=4, end tolerance=1pt]
        \strand[red,thick,] (5, 5) .. controls (5, 2) and (-5, 2) .. (-5, 0);
        \strand[red,thick,] (-5, 5) .. controls (-5, 2) and (5, 2) .. (5, 0);
        \draw[thick,-latex] (-5, 0) [in=135, out=down] to (0, -2) [in=up, out=-45] to (5, -5);
        \draw[thick,-latex] (5, 0) [in=45, out=down] to (0, -2) [in=up, out=225] to (-5, -5);
        \draw[thick,fill=black] (0, -2) circle (0.5);
    \end{knot}
\end{tikzpicture}}

\newcommand{\VirtualOmegaVb} {\begin{tikzpicture}[baseline=-0.65ex,scale=0.15]
    \useasboundingbox (-7, -7) rectangle (7, 7);
    \begin{knot}[clip width=5, end tolerance=1pt]
        \strand[red,thick,] (5, 0) .. controls (5, -2) and (-5, -3) .. (-5, -5);
        \strand[red,thick,] (-5, 0) .. controls (-5, -2) and (5, -3) .. (5, -5);
        \draw[thick,] (-5, 5) [in=135, out=down] to (0, 3);
        \draw[thick,-latex] (0, 3) [in=up, out=-45] to (5, 0);
        \draw[thick,] (5, 5) [in=45, out=down] to (0, 3);
        \draw[thick,-latex] (0, 3) [in=up, out=225] to (-5, 0);
        \draw[thick,fill=black] (0, 3) circle (0.5);
    \end{knot}
\end{tikzpicture}}

%%%%%%%%%% REIDEMEISTER-2 END %%%%%%%%%%

%%%%%%%%%% REIDEMEISTER-3 START %%%%%%%%%%

% a, b - third arc passes over crossing
% c, d - third arc passes under crossing

\newcommand{\VirtualReidemeisterIIIa} {\begin{tikzpicture}[baseline=-0.65ex,scale=0.15]
    \useasboundingbox (-7, -7) rectangle (7, 7);
    \begin{knot}[clip width=15, flip crossing/.list={1, 2, 3}, end tolerance=1pt]
        \strand[red,thick,-latex] (-0.25, -0.25) -- (-5, -5);
        \strand[red,thick,] (5, 5) -- (0.25, 0.25);
        %
        \strand[red,thick,-latex] (-0.25, 0.25) -- (-5, 5);
        \strand[red,thick,] (5, -5) -- (0.25, -0.25);
        %
        \strand[red,thick,-latex] (-5, 0) .. controls (-3, 0) and (-3, 5) .. (0, 5) .. controls (3, 5) and (3, 0) .. (5, 0);
        \draw[thick,fill=black] (0, 0) circle (0.5);
    \end{knot}
\end{tikzpicture}}

\newcommand{\WUVirtualReidemeisterIIIa} {\begin{tikzpicture}[baseline=-0.65ex,scale=0.15]
    \useasboundingbox (-7, -7) rectangle (7, 7);
    \begin{knot}[clip width=7, end tolerance=1pt]
    \strand[thick,latex-] (-5, -5) to (-4, -4);
    \strand[thick] (-4, -4) to (-0.25, -0.25);
    \strand[thick] (0.25, 0.25) to (5, 5);
    %
    \strand[thick] (5, -5) to (0.25, -0.25);
    \strand[thick] (-0.25, 0.25) to (-4, 4);
    \strand[thick,-latex] (-4, 4) to (-5, 5);
    %
    \strand[thick,-latex] (-5, 0) [in=left, out=right] to (0, 5) to (5, 0); 
    %
    \draw[thick,fill=black] (0, 0) circle (0.75);
    \end{knot}
\end{tikzpicture}}

\newcommand{\VirtualReidemeisterIIIb} {\begin{tikzpicture}[baseline=-0.65ex,scale=0.15]
    \useasboundingbox (-7, -7) rectangle (7, 7);
    \begin{knot}[clip width=4, flip crossing/.list={1, 2, 3}, end tolerance=1pt]
        \strand[red,thick,-latex] (-0.25, -0.25) -- (-5, -5);
        \strand[red,thick,] (5, 5) -- (0.25, 0.25);
        %
        \strand[red,thick,-latex] (-0.25, 0.25) -- (-5, 5);
        \strand[red,thick,] (5, -5) -- (0.25, -0.25);
        %
        \strand[red,thick,-latex] (-5, 0) .. controls (-3, 0) and (-3, -5) .. (0, -5) .. controls (3, -5) and (3, 0) .. (5, 0);
        \draw[thick,fill=black] (0, 0) circle (0.5);
    \end{knot}
\end{tikzpicture}}

\newcommand{\VirtualReidemeisterIIIc} {\begin{tikzpicture}[baseline=-0.65ex,scale=0.15]
    \useasboundingbox (-7, -7) rectangle (7, 7);
    \begin{knot}[clip width=4, end tolerance=1pt]
        \strand[red,thick,-latex] (-0.25, -0.25) -- (-5, -5);
        \strand[red,thick,] (5, 5) -- (0.25, 0.25);
        %
        \strand[red,thick,-latex] (-0.25, 0.25) -- (-5, 5);
        \strand[red,thick,] (5, -5) -- (0.25, -0.25);
        %
        \strand[red,thick,-latex] (-5, 0) .. controls (-3, 0) and (-3, 5) .. (0, 5) .. controls (3, 5) and (3, 0) .. (5, 0);
        \draw[thick,fill=black] (0, 0) circle (0.5);
    \end{knot}
\end{tikzpicture}}

\newcommand{\VirtualReidemeisterIIId} {\begin{tikzpicture}[baseline=-0.65ex,scale=0.15]
    \useasboundingbox (-7, -7) rectangle (7, 7);
    \begin{knot}[clip width=4, end tolerance=1pt]
        \strand[red,thick,-latex] (-0.25, -0.25) -- (-5, -5);
        \strand[red,thick,] (5, 5) -- (0.25, 0.25);
        %
        \strand[red,thick,-latex] (-0.25, 0.25) -- (-5, 5);
        \strand[red,thick,] (5, -5) -- (0.25, -0.25);
        %
        \strand[red,thick,-latex] (-5, 0) .. controls (-3, 0) and (-3, -5) .. (0, -5) .. controls (3, -5) and (3, 0) .. (5, 0);
        \draw[thick,fill=black] (0, 0) circle (0.5);
    \end{knot}
\end{tikzpicture}}

%%%%%%%%%% REIDEMEISTER-3 END %%%%%%%%%%

%%%%%%%%%% SKEIN START %%%%%%%%%%

\newcommand{\SREDNIAWYSkeinPlus} {\begin{tikzpicture}[baseline=-0.65ex,scale=0.06]
    \useasboundingbox (-7, -5) rectangle (7, 5);
    \begin{knot}[clip width=4, end tolerance=1pt]
        \strand[red,thick, -Latex] (-5, -5) to (5, 5);
        \strand[red,thick,Latex-] (-5, 5) to (5, -5);
    \end{knot}
\end{tikzpicture}}

\newcommand{\SREDNIAWYSkeinMinus} {\begin{tikzpicture}[baseline=-0.65ex,scale=0.06]
    \useasboundingbox (-7, -5) rectangle (7, 5);
    \begin{knot}[clip width=4, end tolerance=1pt, flip crossing/.list={1}]
        \strand[red,thick, -Latex] (-5, -5) to (5, 5);
        \strand[red,thick,Latex-] (-5, 5) to (5, -5);
    \end{knot}
\end{tikzpicture}}

\newcommand{\SREDNIAWYSkeinZero} {\begin{tikzpicture}[baseline=-0.65ex,scale=0.06]
    \useasboundingbox (-7, -5) rectangle (7, 5);
    \begin{knot}[clip width=5, end tolerance=1pt, flip crossing/.list={1}]
        \strand[red,thick, -latex] (-5, -5) to [out=45, in=-45] (-5, 5);
        \strand[red,thick, -latex] (5, -5) to [out=135, in=-135] (5, 5);
    \end{knot}
\end{tikzpicture}}

%%%%%%%%%% SKEIN END %%%%%%%%%%

%%% VASSILIEV START %%%
\newcommand{\SingularCrossing} {\begin{tikzpicture}[baseline=-0.65ex,scale=0.06]
    \useasboundingbox (-7, -5) rectangle (7, 5);
    \begin{knot}[clip width=5, end tolerance=1pt]
        \strand[red,thick,] (-0.5, -0.5) to (-5, -5);
        \strand[red,thick,-Latex] (-0.5, 0.5) to (-5, 5);
        \strand[red,thick,] (0.5, -0.5) to (5, -5);
        \strand[red,thick,-Latex] (0.5, 0.5) to (5, 5);
        \draw[black,fill=black] (0,0) circle (1);
    \end{knot}
\end{tikzpicture}}
      
%%% VASSILIEV END



\definecolor{lightgray}{gray}{0.9} % define lightgray
\let\oldtabular\tabular % alternate rowcolors for all tables
\let\endoldtabular\endtabular
\renewenvironment{tabular}
{\rowcolors{2}{white}{lightgray}\oldtabular}
{\endoldtabular}
\let\oldlongtable\longtable % alternate rowcolors for all long-tables
\let\endoldlongtable\endlongtable
\renewenvironment{longtable}
{\rowcolors{2}{white}{lightgray}\oldlongtable}
{\endoldlongtable}

\author{Casimir Allard}
\title{Kombinatoryczna teoria węzłów}

\begin{document}


% strona pierwsza

\thispagestyle{empty}
{\noindent\fontsize{18pt}{18pt}\selectfont Księgozbiór matemagiczny, tom 61}

\noindent\makebox[\linewidth]{\rule{\textwidth}{1pt}}

\newpage

% koniec strony pierwszej

% strona druga

\thispagestyle{empty}
\phantom{nothing}
\newpage

% koniec strony drugiej


% strona trzecia

\thispagestyle{empty}
{\noindent\fontsize{18pt}{18pt}\selectfont Casimir Allard, Adélaïde Gauthier}

\noindent\makebox[\linewidth]{\rule{\textwidth}{1pt}}

\vspace{10mm}

{\noindent\fontsize{24pt}{24pt}\selectfont \textbf{Kombinatoryczna\\teoria węzłów}}
\vspace{10mm}

{\noindent\fontsize{14pt}{14pt}\selectfont Wydanie trzecie poprawione}

\newpage

% koniec strony trzeciej

% strona czwarta

\thispagestyle{empty}
\begin{figure}[H]
\begin{minipage}[b]{.48\linewidth}
{\noindent Prof. Casimir Allard\\
Université Bordeaux I\\
351 Cours de la Libération\\
33400 Talence, Francja}
\end{minipage}
\begin{minipage}[b]{.48\linewidth}
{\noindent Adélaïde Gauthier\\
École polytechnique\\
Route de Saclay\\
91128 Palaiseau, Francja}
\end{minipage}
\end{figure}

{\noindent \textbf{Kategorie MSC 2020}\\57K10 (niskowymiarowa teoria węzłów),\\57K30 (topologia 3-rozmaitości)} \vspace{5mm}

{\noindent \textbf{Tytuł oryginału}\\La théorie combinatoire des næuds}
\vspace{5mm}

{\noindent \textbf{Z francuskiego tłumaczyła}\\Juliette Buis} 
\vspace{5mm}

{\noindent \textbf{Okładkę zaprojektował}\\Wulfgang Kot}
\vspace{5mm}

{\noindent \textbf{Zredagował}\\Radosław Jagodowy}
\vspace{5mm}

{\noindent \textbf{Zredagowała technicznie}\\Klara Chmiel}
\vspace{5mm}

{\noindent \textbf{Złożyli i połamali}\\Porte de Versailles, Paryż}
\vspace{5mm}

{\noindent \textbf{Korekty dokonali}\\Jerzy Maślanka, Zuzanna Szpinak}

\vfill

{\noindent Copyleft by Antykwariat Czarnoksięski, Gorzów Wielkopolski 2022.
Książka, a także każda jej część, mogą być przedrukowywane oraz w jakikolwiek inny sposób reprodukowane czy powielane mechanicznie, fotooptycznie, zapisywane elektronicznie lub magnetycznie, oraz odczytywane w środkach publicznego przekazu bez pisemnej zgody wydawcy.}

\vspace{5mm}

{\noindent Przygotowano w systemie \TeX, wydrukowano na siarczystym papierze.}

% koniec strony czwartej


\tableofcontents

\chapter{Preludium}
\input{10-introduction/100-intro}
\input{10-introduction/101-knots}

\section{Diagramy. Ruchy Reidemeistera}

Chociaż w~świetle definicji \ref{def:knot} węzły są pewnymi regularnymi podzbiorami przestrzeni $\R^3$, z~kombinatorycznego punktu widzenia wygodniej jest rysować je na płaszczyźnie.

% DICTIONARY;oriented;zorientowany;węzeł
\begin{definition}[orientacja]
\index{węzeł!zorientowany}%
\index{orientacja|see {węzeł zorientowany}}%
    Węzeł, w~którym wybrano kierunek, w~którym należy się po nim poruszać, nazywamy zorientowanym.
    Splot nazywamy zorientowanym, jeśli wszystkie jego ogniwa traktowane jako węzły są zorientowane.
\end{definition}

Orientację na diagramie zaznaczamy małą strzałką wskazującą kierunek poruszania się.

% DICTIONARY;shadow;cień;-
\begin{definition}
\index{cień}%
    Rzut węzła $K \subseteq \R^3$ na płaszczyznę nazywamy cieniem.
\end{definition}

% DICTIONARY;crossing;skrzyżowanie;-
\begin{definition}[skrzyżowanie]
\index{skrzyżowanie}%
    Podwójny punkt w cieniu nazywamy skrzyżowaniem.
\end{definition}

% DICTIONARY;diagram;diagram;-
\begin{definition}[diagram]
\index{diagram}%
    Cień razem z~informacją o~tym, jak przebiegają skrzyżowania i pozbawiony katastrof: potrójnych przecięć, stycznych czy dziobów nazywamy diagramem.
    % TODO: Narysować katastrofy
\end{definition}

\begin{definition}[włókno]
\index{włókno}%
    Fragment diagramu, który biegnie między dwoma kolejnymi tunelami, czyli podskrzyżowaniami, nazywamy włóknem.
\end{definition}

\begin{definition}[nić]
\index{nić}%
    Fragment diagramu, który biegnie między dwoma kolejnymi skrzyżowaniami, nazywamy nicią.
\end{definition}

Nici powstają z włókien przez rozcięcie ich przy każdym nadskrzyżowaniu.

\begin{proposition}
    Niech $L$ będzie splotem.
    Jego diagramy tworzą otwarty i~gęsty podzbiór wszystkich rzutów.
\end{proposition}

Kawauchi \cite[s. 7]{kawauchi96} wspomina w tym miejscu podręcznik Crowella, Foxa \cite[s. 7]{crowell63}.
To samo jest na przykład w~\cite[s. 10]{burde14}.

\begin{proof}
    Rzut splotu na równoległe płaszczyzny jest taki sam, a te można sparametryzować prostymi przechodzącymi przez początek układu współrzędnych, które tworzą przestrzeń rzutową $\R \mathbb P^2$.
    Niech $S$ będzie zbiorem prostych, które dają złe rzuty.
    Wystarczy pokazać jego nigdziegęstość.
    Okazuje się, że $S$ jest też jednowymiarowy.
\end{proof}

\begin{corollary}
    Każdy splot posiada diagram.
\end{corollary}

Każdy węzeł ma zatem wiele diagramów; mając dane dwa różne chcielibyśmy wiedzieć, czy przedstawiają ten sam węzeł.
Kurt Reidemeister podał proste kryterium rozstrzygające ten problem w~latach dwudziestych ubiegłego wieku.
Najpierw zdefiniujmy trzy lokalne operacje na diagramach.

% DICTIONARY;Reidemeister/Turaev/... move;ruch Reidemeistera/Turajewa/...;-
\begin{definition}[ruchy Reidemeistera]
\index{ruchy Reidemeistera}%
    Trzy gatunki lokalnych deformacji diagramu splotu:
    \begin{figure}[H]
    \centering
    %
    \begin{minipage}[b]{.3\linewidth}
        \[
            \LargeReidemeisterOneLeft \stackrel{R_1}{\cong} \LargeReidemeisterOneStraight
        \]
        \subcaption{ruch $R_1$}
    \end{minipage}
    %
    \begin{minipage}[b]{.3\linewidth}
        \[
            \LargeReidemeisterTwoA \stackrel{R_2}{\cong} \LargeReidemeisterTwoB
        \]
        \subcaption{ruch $R_2$}
    \end{minipage}
    %
    \begin{minipage}[b]{.35\linewidth}
        \[
            \LargeReidemeisterThreeA \stackrel{R_3}{\cong} \LargeReidemeisterThreeB
        \]
        \subcaption{ruch $R_3$}
    \end{minipage}
    \end{figure}
    skręcenie lub rozkręcenie ($R_1$), wsunięcie lub rozsunięcie ($R_2$) oraz przesunięcie łuku przez skrzyżowanie ($R_3$) nazywamy ruchami Reidemeistera.
\end{definition}

Ruch $R_i$ operuje więc na $i$ łukach diagramu.
Reidemeister w~swojej pierwszej pracy przyjął inną kolejność, jego drugi ruch jest naszym pierwszym.

\begin{theorem}[Reidemeister, 1927]
\label{thm:reidemeister}%
\index{twierdzenie!Reidemeistera}%
    Niech $D_1, D_2$ będą diagramami dwóch splotów $L_1, L_2$.
    Sploty $L_1, L_2$ są takie same wtedy i tylko wtedy, gdy diagram $D_2$ można otrzymać z $D_1$ wykonując skończony ciąg ruchów Reidemeistera oraz gładkich deformacji łuków, bez zmiany biegu skrzyżowań.
\end{theorem}

Dowód podali niezależnie Reidemeister \cite{reidemeister27} oraz Alexander, Briggs \cite{briggs27}.
Twierdzenie Reidemeistera jest prawdziwe także dla splotów zorientowanych, wtedy jednak w każdym ruchu trzeba uwzględnić wszystkie możliwe orientacje łuków.

\begin{proof}
    Szkielet dowodu można znaleźć w~książce Burdego i~Zieschanga \cite[s. 9-11]{burde14}.
    Kluczowe pomysły zawiera też \cite[s. 11-12]{prasolov97} Prasołowa i~Sosińskiego.
    Innym przystępnym źródłem jest podręcznik \cite[s. 50-56]{murasugi96} Murasugiego.
\end{proof}

Trace \cite{trace83} zauważył, że dwa diagramy jednego węzła są związane tylko II i III ruchem (ale nie I) wtedy i tylko wtedy, gdy mają ten sam spin oraz indeks punktu względem krzywej (,,winding number'').
Z prac Östlunda \cite{ostlund01}, Manturowa\footnote{Niestety nie wiem, o które strony tej książki chodzi.} \cite{manturov04} oraz Haggego \cite{hagge06} wynika, że dla każdego węzła istnieje para diagramów, do przejścia między którymi trzeba wykorzystać wszystkie trzy ruchy.
% praca Haggego nazywa się "Every Reidemeister move is needed for each knot type" ale nawet w MathSciNecie wspomnieni są Ostlund i Manturow, więc zostawiam. Tekst skopiowany z Wiki
Coward \cite{coward06} zademonstrował, że nawet jeśli wszystkie trzy ruchy są potrzebne, można je wykonywać w specjalnej kolejności: najpierw tylko I ruchy, potem tylko II ruchy, następnie tylko III ruchy i znowu II ruchy.

W praktyce twierdzenia \ref{thm:reidemeister} nie stosuje się bezpośrednio do diagramów splotów.
Mając dane dwa spójne diagramy tego samego splotu trudno znaleźć jest ciąg ruchów przekształcający jeden z nich w drugi.
Załóżmy, że widać na nich odpowiednio $n_1, n_2$ skrzyżowań.
Jak piszą Coward, Lackenby w \cite{coward11}, istnieje funkcja $f \colon \N \times \N \to \N$ taka, że między dwoma diagramami można przejść wykonując co najwyżej $f(n_1, n_2)$ ruchów.
Wynika to z oczywistego faktu, że istnieje skończenie wiele spójnych diagramów o danej liczbie skrzyżowań oraz twierdzenia Reidemeistera.
Okazuje się jednak, że od funkcji $f$ można żądać, by była obliczalna i faktycznie, główny wynik \cite{coward11} orzeka, że
\begin{equation}
    f(n_1, n_2) = 2^{2^{\ldots^{2^{n_1 + n_2}}}}
\end{equation}
jest taką funkcją.
Piętrowa potęga liczy sobie aż $10^{1000000 (n_1 + n_2)}$ warstw, ale przynajmniej jest jawnie zdefiniowana.
Natomiast jeżeli $n_2 = 0$, czyli drugi diagram przedstawia niewęzeł, wystarcza $(236n_1)^{11}$ ruchów, to świeższy wynik samego Lackenby'a \cite{lackenby15}.

\begin{tobedone}
    Przedstawić rozumowanie (piramidka z węzłami), dlaczego to nie jest takie oczywiste.
\end{tobedone}

Hayashi \cite{hayashi05} dowiódł, że liczbę ruchów Reidemeistera potrzebnych, by rozszczepić splot można ograniczyć z góry na podstawie indeksu skrzyżowaniowego.

% DICTIONARY;invariant;niezmiennik;-
Zamiast tego definiuje się niezmienniki, czyli funkcje ze zbioru wszystkich diagramów, które nie zmieniają swojej wartości podczas wykonywania ruchów Reidemeistera.
Kiedy pewien niezmiennik przyjmuje różne wartości na dwóch diagramach, te przedstawiają dwa istotnie różne sploty.
Gdy wartości są te same, nie dostajemy żadnej informacji.
Sploty mogą być równoważne albo nie.
Niezmienniki będą nam stale towarzyszyć w~wędrówce po krainie węzłów.

% koniec sekcji Ruchy Reidemeistera

Wolfgang Haken \cite{haken61} podał przepis na wykrycie diagramu niewęzła i rozwiązał częściowo jeden z~ważniejszych problemów teorii węzłów.
Długo nikt nie podjął się implementacji tego algorytmu\footnote{Moritz Epple pisze ,,this algorithm was extremely impractical'', w recenzji z MathSciNet proponuje, żeby przed przeczytaniem pełnej niepotrzebnych dygresji pracy Hakena poznać artykuł \cite{schubert61} Schuberta.}, udało się to Burtonowi, Budneyowi oraz Petterssonowi w~komputerowym programie Regina\footnote{Dostępny pod adresem \url{https://regina-normal.github.io/}.} na przełomie tysiącleci.
%=% https://mathscinet.ams.org/mathscinet-getitem?mr=141107
% DICTIONARY;incompressible;nieściśliwy;-
Burton, Rubinstein i~Tillman pokazali w~pracy \cite{burton12}, jak sprawdzać, czy powierzchnia normalna na striangulowanej 3-rozmaitości jest (nie)ściśliwa w~czasie wykładniczym.
To okazało się wystarczyć do udzielenia negatywnej odpowiedzi na pytanie Thurstona: ,,czy przestrzeń Seiferta-Webera jest rozmaitością Hakena?'', a zatem wykraczającego poza poziom tej pracy.
\index{przestrzeń Seiferta-Webera}%
\index{rozmaitość Hakena}%

SnapPea\footnote{Dostępny pod adresem \url{http://geometrygames.org/SnapPea/index.html}.} to inny popularny wśród niskowymiarowych topologów program pozwalający badać hiperboliczne 3-rozmaitości, patrz sekcja \ref{sec:hyperbolic}.

Wiadomo, że genus oraz zredukowana kohomologia Chowanowa wykrywa niewęzły (fakty \ref{prp:genus_detects_unknot}, \ref{khovanov_detects_unknot}) i nie wiadomo, czy wielomian Jonesa to robi (hipoteza \ref{con:jones}).
% TODO: wiadomo, że Alexandera nie wykrywa
% i nie zrobił tego w THE EFFICIENT CERTIFICATION OF KNOTTEDNESS AND THURSTON NORM, bo to wyszło na arxiv w 2016
\index{genus}%
\index{wielomian!Jonesa}%
W lutym 2021 Lackenby ogłosił nowy algorytm rozpoznający niewęzły w~quasiwielomianowym czasie.

Przykładami trudnych w~rozpoznaniu niewęzłów są: niewęzeł Goritza, Freedmana.
Więcej trudnych niewęzłów zawiera praca \cite{zanellati16} autorstwa Petronio oraz Zanellatiego.

\index{niewęzeł}
\index{niewęzeł!Goritza}
\index{niewęzeł!Freedmana}
\begin{comment}
\begin{figure}[H]
    \begin{minipage}[b]{.32\linewidth}
        \centering
        \includegraphics[width=\linewidth]{../data/missing.jpg}
        \subcaption{normalny}
    \end{minipage}
    \begin{minipage}[b]{.32\linewidth}
        \centering
        \includegraphics[width=\linewidth]{../data/missing.jpg}
        \subcaption{Goritza}
    \end{minipage}
    \begin{minipage}[b]{.32\linewidth}
        \centering
        \includegraphics[width=\linewidth]{../data/missing.jpg}
        \subcaption{Freedmana}
    \end{minipage}
\end{figure}
\end{comment}

Zanim opowiemy, jak dotąd przebiegała klasyfikacja węzłów o małej liczbie skrzyżowań, zdefiniujemy klasę splotów ze specjalnymi diagramami.

% DICTIONARY;alternating;alternujący;węzeł
\begin{definition}[alternacja]
\index{węzeł!alternujący}%
    Diagram splotu, gdzie podczas poruszania się wzdłuż każdego ogniwa nad- oraz podskrzyżowania mijane są naprzemiennie, nazywamy alternującym.
    Splot jest alternujący, jeśli posiada alternujący diagram.
\end{definition}

Około 1961 roku Fox zapytał ,,What is an alternating knot?''.
Szukano takiej definicji węzła alternującego, która nie odnosi się bezpośrednio do diagramów, aż w~2015 roku Joshua Greene podał geometryczną charakteryzację: nierozszczepialny splot w $S^3$ jest alternujący wtedy i tylko wtedy, gdy ogranicza dodatnią oraz ujemną określoną powierzchnię rozpinającą \cite{greene17}.
% definite spanning surface

Sundberg oraz Thistlethwaite pokazali w 1998 roku, że liczba splotów alternujących rośnie wykładniczo (\cite{sundberg98}):

\begin{proposition}
    Niech $a_n$ oznacza liczbę pierwszych, alternujących supłów o~$n$ skrzyżowaniach.
\index{supeł}%
    Wtedy
    \begin{equation}
        a_n \sim (3c_1/4\sqrt{\pi})n^{-5/2}\lambda^{n-3/2},
    \end{equation}
    gdzie zarówno $c_1$, pierwszy współczynnik rozwinięcia Taylora funkcji $\Phi(\eta)$ zdefiniowanej w \cite{sundberg98}, jak i $\lambda$ są jawnie znanymi stałymi:
    \begin{align}
        c_1 & = \sqrt{\frac{5^7 \cdot (21001 + 371 \sqrt{21001})^3}{2 \cdot 3^{10} \cdot (17 + 3\sqrt{21001})^5}} \\
        \lambda & = \frac {1}{40} (101 + \sqrt{21001})
    \end{align}
    Niech $A_n$ oznacza liczbę pierwszych, alternujących splotów o $n$ skrzyżowaniach.
    Wtedy $A_n \approx \lambda^n$, dokładniej: jeśli $n \ge 3$, to
    \begin{equation}
        \frac{a_{n-1}}{16n - 24} \le A \le \frac{a_n - 1}{2}.
    \end{equation}
\end{proposition}

Czasami będziemy używać słów przed ich zdefiniowaniem, tak jak uczyniliśmy tutaj: węzły pierwsze i~supły pojawiają się odpowiednio w definicjach \ref{def:prime_knot}, \ref{def:tangle}.
Książkę trzeba więc przeczytać co~najmniej dwa razy.

\begin{proposition}
    Niech $a_n$ oznacza liczbę pierwszych, alternujących supłów o~$n$ skrzyżowaniach.
    Wtedy funkcja tworząca $f(z) = \sum_n a_n z^n$ spełnia równanie
    \begin{equation}
    f(1+z) - f(z)^2 - (1+f(z))q(f(z)) -z - \frac{2z^2}{1-z} = 0,
    \end{equation}
    gdzie $q(z)$ jest pomocniczą funkcją
    \begin{equation}
        q(z) = \frac{2z^2 - 10z - 1 + \sqrt{(1-4z)^3}} {2(z+2)^3} - \frac{2}{1+z} -z + 2.
    \end{equation}
\end{proposition}

Powyższa ciekawostka także pochodzi z cytowanej wcześniej pracy \cite{sundberg98}.



\subsection{Historia tablic węzłów}
% DICTIONARY;knot table;tablica węzłów;-
Pierwszą osobą, która podjęła się szukania węzłów, był Peter Guthrie Tait, szkocki fizyk.
\index[persons]{Tait, Peter}%
Razem z~Thomsonem (lordem Kelvinem) wierzyli, że węzły są kluczem do zrozumienia widma spektroskopowego różnych pierwiastków: na przykład atom sodu mógł być splotem Hopfa ze względu na dwie linie emisyjne.
\index[persons]{Thomson, William (lord Kelvin)}%
Eksperyment Michelsona-Morleya z~1887 roku zabił ich ,,wirową teorię atomu'', ale nie miało to znaczenia dla teorii węzłów jako działu matematyki.

Używana po dziś dzień strategia, którą przyjął Tait, jest stosunkowa prosta: narysować wszystkie możliwe diagramy o~zadanym indeksie skrzyżowaniowym, po czym połączyć ze sobą te, które przedstawiają jeden węzeł.
Na potrzeby pierwszego etapu Tait wymyślił schemat kodowania diagramów.
Opiszemy później jego ulepszenie, kod Dowkera-Thistlethwaite'a.

\subsubsection{Siedem i mniej skrzyżowań}
Tait wykorzystując swoją notację podał w~1876 pierwszą tablicę piętnastu węzłów o~mniej niż ośmiu skrzyżowaniach.
Nie należy traktować tego jako skromny wynik: nie miał on do dyspozycji żadnych twierdzeń topologicznych do odróżniania węzłów.
Onieśmielony przez liczbę możliwych kodów dla kolejnych indeksów skrzyżowaniowych, powstrzymał się przed rozszerzaniem swojej tablicy.
To właśnie grupowanie diagramów przedstawiających ten sam węzeł, a~nie samo szukanie wszystkich możliwych diagramów, sprawia trudność.

Aby sobie pomóc, Tait znalazł lokalną modyfikację diagramu, która nie zmienia indeksu skrzyżowaniowego, znaną obecnie\footnote{Dla Taita ,,flype'' było innym ruchem, prostą transformacją związaną ze zmianą wyboru nieskończonego obszaru, ale mało kto teraz o tym pamięta. Dowiedzieliśmy się o tym z pracy \cite{menasco93}; Menasco i~Thistlethwaite dowiedzieli się o~tym od Claude'a Webera.} jako flype.
\index[persons]{Menasco, William}%
\index[persons]{Thistlethwaite, Morwen}%
\index[persons]{Weber, Claude}
\index{flype}
Flype to stary szkocki czasownik oznaczający ,,wykręcać na drugą stronę''.

\begin{comment}
\[
\begin{tikzpicture}[baseline=-0.65ex, scale=0.1]
\begin{knot}[clip width=5, end tolerance=1pt, flip crossing/.list={1}]
    \strand[thick] (-21, -5) [in=180, out=0] to (-7, 5);
    \strand[thick] (-21, 5) [in=180, out=0] to (-7, -5);
    \draw (-7, -7) rectangle (7, 7);
    \node at (0, 0) {\Huge {$T$}};
    \draw[thick] (7, -5) to (21, -5);
    \draw[thick] (7, 5) to (21, 5);
\end{knot}
\end{tikzpicture}
\quad \cong_{\mathrm{flype}} \quad
\begin{tikzpicture}[baseline=-0.65ex, scale=0.1]
\begin{knot}[clip width=5, end tolerance=1pt]
    \strand[thick] (21, -5) [in=0, out=180] to (7, 5);
    \strand[thick] (21, 5) [in=0, out=180] to (7, -5);
    \draw (-7, -7) rectangle (7, 7);
    \node at (0, 0) {\rotatebox[origin=c]{-180}{\Huge $T$}};
    \draw[thick] (-7, -5) to (-21, -5);
    \draw[thick] (-7, 5) to (-21, 5);
\end{knot}
\end{tikzpicture}
\]
\end{comment}

Inną taktykę szukania węzłów przyjał wielebny Thomas Kirkman\footnote{Oto jak Kirkman definiował węzeł w stu słowach: ,,By a Knot of $n$ crossings, I understand a reticulation of any number of meshes of two or more edges, whose summits, all tessaraces, are each a single crossing, as when you cross your forefingers straight or slightly curved, so as not to link them, and such meshes that every thread is either seen, when the projection of the Knot with its $n$ crossings and no more is drawn in double lines, or conceived by the reader of its course when drawn in single line, to pass alternately under and over the threads to which it comes at successive crossings.''}: zaczynał od małego zbioru "nieredukowalnych" rzutów, do których systematycznie dokładał skrzyżowania.
\index[persons]{Kirkman, Thomas}%
% wielebny => Adams, s. 31
Tait przeczytał pracę Kirkmana, po czym w~latach 1884/1885 opracował listę węzłów alternujących o~mniej niż 11 skrzyżowaniach.
% Kirkman miał wtedy 78 lat!
Tuż przed oddaniem jej do druku odkrył inny spis węzłów stworzony przez amerykańskiego naukowca Charlesa Little'a.
\index[persons]{Little, Charles}%
Znalazł wtedy jeden duplikat u~siebie, natomiast u Little'a jeden duplikat i~jedno pominięcie.

\subsubsection{Dziesięć skrzyżowań}
Zachęcon przez Taita, Little zabrał się za alternujące węzły o~11 skrzyżowaniach i~za trudniejsze zadanie, stablicowanie węzłów niealternujących, czyli takich, które nie posiadają alternującego diagramu.
Jak wynika z~pierwszej pracy Taita, początkowo nie wierzono, że takie w~ogóle istnieją.
Dowód znaleziono wiele lat później, niealternujące są $8_{19}$, $8_{20}$, $8_{21}$, ale nie pierwsze węzły o mniejszej liczbie skrzyżowań.
Patrz twierdzenie \ref{prp:bankwitz}.
Little pracował przez sześć lat (1893 -- 1899) i~znalazł 43 niealternujące węzły o~10 skrzyżowaniach.
Żadnego nie pominął, ale trafił mu się jeden duplikat.
\index[persons]{Little, Charles}%

W kolejnych dziesięcioleciach nie nastąpił znaczący postęp, zarówno w~rozszerzaniu tablic jak i~sprawdzaniu tych już istniejących.
Haseman \cite{haseman18} w~1918 roku znalazła achiralne węzły o~12 (takich jest 54, praca Haseman podaje 61, ponieważ zawiera 7 duplikatów) i~14 skrzyżowaniach.
\index[persons]{Haseman, Mary}%
% AMPHICHEIRALS ACCORDING TO TAIT AND HASEMAN
W 1927 roku Alexander z~Briggsem przy użyciu pierwszej grupy homologii rozgałęzionego nakrycia cyklicznego (!) potrafili odróżnić od siebie dowolne dwa węzły (z~pominięciem 3 par) o~co najwyżej 9 skrzyżowaniach \cite{briggs27}.
\index[persons]{Briggs, Garland}%
\index[persons]{Alexander, James}%
Reidemeister poradził sobie z~tymi wyjątkami w~1932 roku, korzystając z~indeksu zaczepienia i~homomorfizmów z~grupy węzła na grupy diedralne \cite{reidemeister32}.
\index[persons]{Reidemeister, Kurt}%
% branch curves in irregular covers associated to homomorphisms of the knot group onto dihedral groups

\subsubsection{Jedenaście skrzyżowań}
Dopiero John Conway w~latach sześćdziesiątych minionego wieku znalazł pierwsze węzły o~mniej niż 12 skrzyżowaniach oraz wszystkie sploty o~mniej niż 11 skrzyżowaniach w~oparciu o~pomysły Kirkmana.
\index[persons]{Conway, John}%
% An enumeration of knots and links, 1970.
Zajęło mu to jedynie kilka godzin!
Metoda Conwaya jest tak dobra, że używamy jej po dziś dzień, na przykład Tuzun, Sikora zweryfikowali dzięki niej hipotezę \ref{con:jones} do 24 skrzyżowań.
\index[persons]{Tuzun, Robert}%
\index[persons]{Sikora, Adam}%

Conway znalazł 1 duplikat oraz 11 pominięć w~starych tablicach Little'a, ale sam popełnił 4 pominięcia.
Przeoczył między innymi słynny duplikat w~niealternującej tablicy, parę Perko.
% 1974?
\index{para Perko}%
\index{spin}%
Przyczyną było prawdopodobnie to, że dwa diagramy miały różny spin:
% DICTIONARY;2-pass move;2-przejście;-
Little błędnie twierdził, że spin minimalnego diagramu jest niezmiennikiem, gdyż błędnie założył, że 2-przejścia oraz flype wystarczają do zmiany dowolnego minimalnego diagramu w~inny.
\index[persons]{Little, Charles}%

Naprawienie błędu tego błędu zajęło chwilę: pominęcia w~tablicy Conwaya znalazł Caudron około 1980 roku \cite{caudron82}.
\index[persons]{Caudron, Alain}%
Rękopis \cite{bonahon89} Bonahona, Siebenmanna klasyfikuje węzły algebraiczne.
\index[persons]{Bonahon, Francis}%
\index[persons]{Siebenmann, Laurent}%
Z~nielicznymi niealgebraicznymi węzłami do 11 skrzyżowań poradził sobie Perko w \cite{perko80} oraz \cite{perko82}, co było kresem ery ręcznych obliczeń.
\index[persons]{Perko, Kenneth}%

% MAKOTO SAKUMA - A SURVEY OF THE IMPACT OF THURSTON’S WORK ON KNOT THEORY
% through hand calculation of homological invariants (in particular linking invariants) of finite branched coverings for those knots that are not covered by Bonahon and Siebenmann’s result described in Subsection 4.1. See [268] for an interesting historical note.

\subsubsection{Trzynaście skrzyżowań}
Na początku lat osiemdziesiątych ubiegłego wieku Dowker i~Thistlethwaite \cite{dowker83} z~pomocą komputera stablicowali węzły do 13 skrzyżowań.
\index[persons]{Dowker, Clifford}%
\index[persons]{Thistlethwaite, Morwen}%
Przez blisko dekadę nic się nie działo, aż wreszcie grupa studentów (Arnold, Au, Candy, Erdener, Fan, Flynn, Muir, Wu \cite{cray94}) wygrała dostęp do superkomputera Cray.
Razem z~Hoste znaleźli alternujące węzły do 14 skrzyżowań, jednocześnie sprawdzając istniejące tabele Thistlethwaite'a.
\index[persons]{Arnold, Brian}%
\index[persons]{Au, Michael}%
\index[persons]{Candy, Christoper}%
\index[persons]{Erdener, Kaan}%
\index[persons]{Fan, James}%
\index[persons]{Flynn, Richard}%
\index[persons]{Hoste, Jim}%
\index[persons]{Muir, Robs}%
\index[persons]{Wu, Danny}%

\subsubsection{Szesnaście skrzyżowań}
Około roku 1998 Hoste z~Weeksem (oraz niezależnie Thistlethwaite) znaleźli w~\cite{thistlethwaite98} 1 701 936 pierwszych węzłów do 16 skrzyżowań.
\index[persons]{Hoste, Jim}%
\index[persons]{Thistlethwaite, Morwen}%
\index[persons]{Weeks, Jeff}%
Spośród nich, tylko 32 nie jest węzłami hiperbolicznymi, wszystkie pozostałe poddają się maszynerii geometrii hiperbolicznej.

\subsubsection{Dziewiętnaście skrzyżowań}
Artykuł \cite{thistlethwaite98} zawiera informację, że jego autorzy szukają węzłów o~17 skrzyżowaniach, ale ja nie doszukałem się żadnej późniejszej publikacji na ten temat.
\index[persons]{Hoste, Jim}%
\index[persons]{Thistlethwaite, Morwen}%
\index[persons]{Weeks, Jeff}%
W 2004 Flint, Rankin oraz Schermann \cite{rankin04} znaleźli alternujące węzły do 22 skrzyżowań (obliczenia na stacji roboczej z procesorem Xeon oraz 3 gigabajtami pamięci zajęły około 45 godzin), po czym długo nie działo się nic.
\index[persons]{Flint, Ortho}%
\index[persons]{Rankin, Stuart}%
\index[persons]{Schermann, John}%
Dopiero w 2020 Burton \cite{burton20} stablicował węzły pierwsze do 19 skrzyżowań: \emph{,,Here we extend the tables from 16 to 19 crossings, with a total of 352 152 252 distinct non-trivial prime knots.''}
\index[persons]{Burton, Benjamin}%

\subsubsection{Sploty}
Cerf \cite{cerf98} pisze, że Conway znalazł wcześniej sploty do 10 skrzyżowań \cite{conway70}, zaś Caudron \cite{caudron82} poprawił wynik do 11 skrzyżowań, ale wszystkie te sploty są niezorientowane, a~naukowcy potrzebują zorientowanych.
\index[persons]{Cerf, Corinne}%
\index[persons]{Conway, John}%
\index[persons]{Caudron, Alain}%
Problem został zaadresowany najpierw przez Dolla i Hoste'a \cite{doll91}, którzy wydali na mikrofilmie tablicę splotów zorientowanych do 9 skrzyżowań, ale ich diagramy nie zawsze pasowały do tych narysowanych w~książce Rolfsena.
\index[persons]{Doll, Helmut}%
\index[persons]{Hoste, Jim}%

Cerf obiecuje pogodzić punkty widzenia Rolfsena oraz Dolla/Hoste'a i tworzy własną tablicę zorientowanych splotów do 11 skrzyżowań.
Sprawdziła jednocześnie poprawność starszych tablic Conwaya -- i nie znalazła żadnych błędów.




\subsection{Hipotezy Taita}
\index{hipoteza!Taita|(}

\begin{conjecture}[I hipoteza Taita]
\index{indeks skrzyżowaniowy}%
\label{con:tait_1}%
    Zredukowany alternujący diagram splotu ma minimalny indeks skrzyżowaniowy.
\end{conjecture}

Najpierw znaleziono dowód korzystający z wielomianu Jonesa: dokonali tego w 1987 roku równocześnie Kauffman \cite{kauffman87}, Murasugi \cite{murasugi87} oraz Thistlethwaite \cite{thistlethwaite87}.
\index[persons]{Kauffman, Louis}%
\index[persons]{Murasugi, Kunio}%
\index[persons]{Thistlethwaite, Morwen}%
Trzydzieści lat później Greene zaprezentował geometryczne podejście do problemu w \cite{greene17}.
\index[persons]{Greene, Joshua}%

\begin{conjecture}[II hipoteza Taita]
\index{spin}%
    Dwa zredukowane diagramy alternujące jednego węzła mają ten sam spin.
\end{conjecture}

Pierwsze dowody pochodzą znowu od Kauffmana \cite{kauffman87} oraz Thistlethwaite'a \cite{thistlethwaite87}.
\index[persons]{Kauffman, Louis}%
\index[persons]{Thistlethwaite, Morwen}%
Dla niektórych II hipoteza brzmi inaczej (,,achiralny splot alternujący ma zerowy spin''), dla innych jest prostym wnioskiem z naszego sformułowania.

\begin{conjecture}[III hipoteza Taita]
\index{flype}%
    Niech $D_1, D_2$ będą zredukowanymi alternującymi diagramami zorientowanego pierwszego splotu.
    Wtedy diagram $D_2$ można otrzymać z~$D_1$ korzystając jedynie z~ruchu \emph{flype}.
\end{conjecture}

Trzecią hipotezę udowodnił Menasco wspólnie z~Thistlethwaitem, \cite{menasco93}.
\index[persons]{Menasco, ?}%
\index[persons]{Thistlethwaite, Morwen}%
Wynika z~niej, że dwa zredukowane diagramy alternujące tego samego węzła mają ten sam spin.
Nie jest prawdziwa dla niealternujących splotów, przez co w~tablicach węzłów tak długo mieliśmy duplikat -- parę Perko.
\index{para Perko}%

Czasami mówi się jeszcze o IV hipotezie: że zwierciadlane węzły mają parzysty indeks skrzyżowań.
\index{węzeł!zwierciadlany}
Ta okazała się fałszywa.

Przedstawimy ze szczegółami dowód pierwszej hipotezy w~sekcji \ref{sub:tait_conjectures} oraz wspomnimy krótko o technikach użytych w dowodach pozostałych trzech.

\index{hipoteza!Taita|)}

% koniec podsekcji Hipotezy Taita



\subsection{Metody kodowania}
\subsubsection{Notacja Gaußa}
\index{notacja!Gaußa}
Pierwszymi osobami, które zajmowały się węzłami, był prawdopodobnie Gauß oraz jego uczeń, Listing.
Gauß wprowadził indeks zaczepienia dwóch węzłów jako pewna całka oraz notację dla węzłów.
Wybierzmy punkt na diagramie, który nie jest skrzyżowaniem i przemierzajmy go zgodnie z~orientacją.
Gdy mijamy nowe skrzyżowania, przypisujemy im kolejne liczby $1, 2, \ldots$, zaś dla starych skrzyżowań przepisujemy numer.
Jeżeli mijamy skrzyżowanie dołem, kodujemy je liczbą z minusem.

\begin{comment}
\begin{figure}[H]
    \centering
    \begin{minipage}[b]{.45\linewidth}
        \centering
        \includegraphics[width=0.90\linewidth]{../data/mixed/gauss_code.png}
        \subcaption{Węzeł o kodzie Gaußa 1 -2 3 -4 5 6 -7 -8 4 -9 2 -10 8 11 -6 -1 10 -3 9 -5 -11 7. Źródło:\\ \url{https://knotinfo.math.indiana.edu/descriptions/gauss_notation.html}.}
    \end{minipage}
    \quad
    \begin{minipage}[b]{.45\linewidth}
        \centering
        \includegraphics[width=0.90\linewidth]{../data/mixed/dowker_code.png}
        \subcaption{o kodzie (Dowkera-Thistlethwaite'a) 16 18 20 -22 4 2 8 -6 12 10 -14. Źródło:\\ \url{https://knotinfo.math.indiana.edu/descriptions/dt_notation.html}.}
    \end{minipage}
\end{figure}
\end{comment}

W ogólnym przypadku nie można odtworzyć węzła z jego kodu, ale można delikatnie zmienić notację, by było to możliwe.
Kiedy mijamy skrzyżowanie drugi raz, stawiamy minus przed liczbą, jeżeli skrzyżowanie jest lewoskrętne i plus w przeciwnym wypadku.
Nazywa się to rozszerzonym kodem Gaußa.
W naszym przykładzie, rozszerzony kod to 1 -2 3 -4 5 6 -7 -8 \textbf{-4} -9 \textbf{-2} -10 8 11 \textbf{6 -1 -10 3 9 -5 -11 -7}, pogrubione liczby odpowiadają drugim przejściom.

\subsubsection{Notacja Dowkera-Thistlethwaite'a}
\index{notacja!Taita}
\index{notacja!Dowkera-Thistlethwaite'a}
Poprawia nieopisaną tutaj notację Taita, opisana po raz pierwszy w~pracy \cite{dowker83}.

Tak jak w~notacji Gaußa, przemierzamy węzeł zaczynając poza skrzyżowaniem.
Tym razem jednak stare skrzyżowania dostają drugi, nowy numer.
Jak można zauważyć, każde skrzyżowanie ma parzystą oraz nieparzystą etykietę.

\begin{definition}
    Ciąg parzystych liczb występujących na diagramie kolejno przy $1, 3, \ldots$ nazywamy kodem Dowkera-Thistlethwaite'a.
    Jeżeli nieparzysta etykieta odpowiadała podskrzyżowaniu, zapisujemy liczbę z~minusem.
\end{definition}

Opisany powyżej kod nie jest idealny, ponieważ odtworzony z niego węzeł może być lustrzanym odbiciem wyjściowego.
Ogólniej, odbicie dowolnego składnika sumy spójnej nie zmienia kodu całego węzła.
Nie stanowi to jednak dużego problemu, ponieważ notacja została stworzona na potrzeby tablicowania węzłów pierwszych, a~te są niezorientowane.

Zaczynając od zredukowanego diagramu o $n$ skrzyżowaniach nie można doprowadzić do sytuacji, gdzie do pewnego skrzyżowania przypisane są dwie kolejne liczby całkowite.
Dzięki temu problem można przetłumaczyć na język teorii grafów.
Rozpatrzmy graf $G$, którego wierzchołkami są liczby $1, 2, \ldots, 2n$.
Połączmy niesąsiadujące modulo $2n$ wierzchołki o różnej parzystości krawędziami.
Graf ten powstaje przez usunięcie cyklu Hamiltona (łączącego kolejne liczby) z pełnego grafu dwudzielnego.
Zbiór par etykiet przy skrzyżowaniach węzła to skojarzenie doskonałe w grafie $G$.
Liczba skojarzeń prawie pokrywa się z rozwiązaniem zadania znanego w literaturze jako ,,problème des ménages'': na ile sposobów $n$ małżeństw można posadzić przy okrągłym stole tak, by żadne małżeństwo nie siedziało obok siebie i~każdy mężczyzna znalazł się obok dwóch kobiet?
Ustawienia, które powstają przez cykliczne permutowanie należy uznać za tożsame.
Gilbert znalazł w \cite{gilbert56} wzór na $a_n$, liczbę różnych kodów:
\begin{align}
u(m, t) & = 2m \sum_{k=0}^m {2m-k \choose k} \cdot (m-k)! \cdot \frac{(t-1)^k}{2m - k}  \\
a(n) & = \frac{1}{n} \sum_{d\mid n} \left(\frac{n}{d}\right)^d \cdot u \left(d, 1 - \frac{d}{n}\right) \cdot \varphi \left(\frac{n}{d}\right)
\end{align}

Kilka początkowych wartości to $a_3 = 1, 2, 5, 20, 87, 616, 4843, 44128, 444621, \ldots$ (ciąg A002484 w OEIS).

\subsubsection{Notacja Alexandera-Briggsa}
\index{notacja!Alexandera-Briggsa}
W~1927 roku Alexander, Briggs wprowadzili zupełnie inny sposób oznaczania węzłów -- wtedy do 9 skrzyżowań, ale przedłużoną do 10 skrzyżowań przez Rolfsena i używaną po dziś dzień.
Do opisu węzła używa się dwóch liczb: jego indeksu skrzyżowaniowanego z dolnym indeksem, oznaczającym miejsce w tablicy~węzłów.
I~tak węzły o~ośmiu skrzyżowaniach to $8_1, 8_2, \ldots,$ $8_{21}$.
Porządek jest umowny i jego wybór należy do osoby, która jako pierwsza znajdzie wszystkie węzły o danej liczbie skrzyżowań (ale węzeł skręcony występuje zawsze po torusowym).
\index{węzeł!skręcony}%
\index{węzeł!torusowy}%

Od jedenastu skrzyżowań pojawia się mała zmiana: węzły alternujące i niealternujące kataloguje się osobno.
I tak $11n_{185}$ to sto osiemdziesiąty piąty węzeł niealternujący o 11 skrzyżowaniach, zaś $11a_{367}$ to trzysta sześćdziesiąty siódmy alternujący.

Rolfsen w 1976 stworzył z kilkoma błędami tablicę diagramów pierwszych węzłów do 10 skrzyżowań.
Para Perko $10_{161}, 10_{162}$ przedstawia ten sam węzeł, zaś górne skrzyżowanie w~$10_{144}$ powinno być zmienione.
Ostatnie cztery węzły dostały nowe numery, by uniknąć duplikatu.
Kolejną usterką tablicy jest to, że notacja Conwaya oraz wielomian Alexandera dla węzłów $10_{83}$ oraz $10_{86}$ są zamienione miejscami.
Tu czyha pułapka:\footnote{Wiemy o niej dzięki stronie \url{http://stoimenov.net/stoimeno/homepage/ptab/}.} Stojmenow, nowe wydanie książki Rolfsena, atlas węzłów Bar-Natana oraz tablica niezmienników węzłowych Livingstona naprawiają to przez wymianę podpisów.
Podręcznik Kawauchiego wymienia diagramy.

Ze strony internetowej Stojmenowa dowiedzieliśmy się jeszcze czegoś.
Kolejność Rolfsena dla węzłów o 10 skrzyżowaniach obala nomenklaturę Little'a niealternujących oraz nadpisuje numerowanie Taita dla alternujących węzłów.
Alexander, Briggs zrobili wcześniej to samo dla 9 lub mniej skrzyżowań.

\subsubsection{Notacja Conwaya}
\index{notacja!Conwaya}
Wprowadzona przez Conwaya w~pracy \cite{conway70}.
Wymaga znajomości supłów, więc przedstawiamy ją w sekcji supłów: definicji \ref{conway_notation}.

\subsubsection{Nazwy zwyczajowe}
Niektóre węzły i sploty, w szczególności te o niskim indeksie skrzyżowaniowym, występują tak często w teorii węzłów, że doczekały się nazw zwyczajowych.
Oto ich lista:
\begin{compactitem}
% DICTIONARY;unknot;niewęzeł;-
% TODO: nigdzie w książce nie ma definicji niesplotu?
    \item węzeł $0_1$ to niewęzeł;
% DICTIONARY;trefoil knot;trójlistnik;-
    \item węzeł $3_1$ to trójlistnik,
% DICTIONARY;figure-eight;ósemka;-
    \item węzeł $4_1$ to ósemka albo węzeł Listinga,
% DICTIONARY;cinquefoil knot;pięciolistnik;-
    \item węzeł $5_1$ to pięciolistnik albo węzeł Solomona (!),
% DICTIONARY;stevedore knot;węzeł dokerski;-
    \item węzeł $6_1$ to węzeł dokerski,
    \item węzeł 11n34 to węzeł Conwaya,
    \item węzeł 11n42 to węzeł Kinoshity-Terasakiego,
    \item węzeł 12n242, czyli $(-2, 3, 7)$-precel, to węzeł Fintushela-Sterna,
% DICTIONARY;granny knot;węzeł babski;-
    \item suma spójna takich samych trójlistników to węzeł babski,
% DICTIONARY;square knot;węzeł prosty/płaski
    \item suma spójna lustrzanych trójlistników to węzeł prosty albo płaski (dość niefortunna nazwa),
    \item splot $2_1^2$ (L2a1) to splot Hopfa,
    \item splot $4_1^2$ (L4a1) to węzeł Solomona (!),
    \item splot $5_1^2$ (L5a1) to splot Whiteheada,
    \item splot $6_2^3$ (L6a4) to pierścienie Boromeuszy.
\end{compactitem}



\section{Operacje na węzłach}
Mając dany diagram splotu zorientowanego, można odwrócić jego wszystkie ogniwa albo wszystkie skrzyżowania.
Działania te nazywamy odpowiednio rewersem i lustrem, opisujemy je w~pierwszej podsekcji.
Dalej pojawi się suma niespójna oraz spójna, odpowiednik mnożenia liczb naturalnych zbadany dokładniej przez Schuberta około 1954 roku. 
Znacznie później (bo dopiero w~sekcji \ref{sec:tangle}) wprowadzimy jeszcze sumę i~iloczyn supłów.

\subsection{Lustro i~rewers} % (fold)
\begin{definition}[lustro]
\index{lustro}%
\index{węzeł!lustrzany}%
    Niech $L$ będzie zorientowanym splotem.
    Splot $mL$ powstały przez odbicie splotu $L$ względem dowolnej płaszczyzny nazywamy lustrem.
\end{definition}

\begin{definition}[rewers]
\index{rewers}%
\index{węzeł!odwrotny}%
    Niech $L$ będzie zorientowanym splotem.
    Splot $rL$ powstały przez odwrócenie orientacji wszystkich ogniw splotu $L$ nazywamy rewersem.
\end{definition}

\begin{comment}
\begin{figure}[H]
    \begin{minipage}[b]{.32\linewidth}
        \centering
        \includegraphics[width=\linewidth]{../data/link_mirror.png}
        \subcaption{lustro $mL$}
    \end{minipage}
    \begin{minipage}[b]{.32\linewidth}
        \centering
        \includegraphics[width=\linewidth]{../data/link.png}
        \subcaption{przykładowy splot $L$}
    \end{minipage}
    \begin{minipage}[b]{.32\linewidth}
        \centering
        \includegraphics[width=\linewidth]{../data/link_reverse.png}
        \subcaption{rewers $rL$}
    \end{minipage}
\end{figure}
\end{comment}

Na lewym obrazku odbiliśmy diagram względem poziomej prostej, innym sposobem na otrzymanie lustra jest odwrócenie wszystkich skrzyżowań, co odpowiada odbijaniu względem płaszczyzny papieru.
Zauważmy, że wykonując powyższe operacje na węźle możemy otrzymać mniej niż czterech różne obiekty ($L$, $mL$, $rL$, $mrL$) -- na przykład trójlistnik jest własnym rewersem, ale nie lustrem.

Wyróżniamy pięć typów symetrii węzłów:

\begin{definition}[całkowicie chiralny albo skrętny]
\index{węzeł!chiralny}%
\index{węzeł!skrętny|see {węzeł lustrzany}}%
    Węzły $K$, $rK$, $mK$ są parami nierównoważne. % chiral 9_32
\end{definition}

\begin{definition}[odwracalny]
    \index{węzeł!odwracalny}%
    Węzły $K \cong rK$ są równoważne. % reversible 3_1
\end{definition}

\begin{definition}[zwierciadlany ujemnie]
    \index{węzeł!zwierciadlany}%
    Węzły $K \cong mrK$ są równoważne. % negative amphicheiral 8_17
\end{definition}

\begin{definition}[zwierciadlany dodatnio]
    Węzły $K \cong mK$ są równoważne. % positive amphicheiral 12a_427
\end{definition}

\begin{definition}[całkowicie zwierciadlany]
    Węzły $K, rK, mK$ są parami równoważne. % fully amphicheiral 4_1
\end{definition}

\begin{example}
    Węzeł $9_{32}$ jest całkowicie skrętny.
\end{example}

% Całkowicie skrętne są też między innymi wszystkie węzły torusowe.
% TODO: wiki pisze Each nontrivial torus knot is prime[4] and chiral.[2]

\begin{example}
    \label{exm:trefoil_is_chiral}
    Trójlistnik jest odwracalny, ale nie zwierciadlany.
\end{example}

Po raz pierwszy odkrył to M. Dehn w 1914 roku \cite{dehn14}.
Oto, jak tego dokonał.
Iloraz grafu Cayleya dla grupy podstawowej trójlistnika, $G = \pi_1(S^3 - K)$, zanurza się w~produkt $\mathbb H^2 \times \R$, co pozwala wyznaczyć grupę zewnętrznych automorfizmów grupy $G$, $\Z/2\Z$.
\index{grupa!podstawowa}
Korzystając z południków i równoleżników pokazał następnie, że nietrywialny automorfizm odwraca orientację przestrzeni otaczającej.
My przekonamy się o~tym przez wyznaczenie wielomianu Jonesa trójlistnika, patrz wniosek \ref{cor:joines_of_amphicheiral}.

\begin{example}
    Węzeł $8_{17}$ jest zwierciadlany ujemnie, ale nie odwracalny.
\end{example}

Sześćdziesiąt lat temu matematycy nie byli pewni, czy węzły nieodwracalne w~ogóle istnieją \cite[problem 10]{fox62};
obecnie wiadomo, że nieodwracalne są prawie wszystkie węzły (\cite[s.~46]{murasugi96}).
W~roku 1962 Ralph Fox wskazał kilku kandydatów do tego tytułu.
Hale Trotter odkrył rok później nieskończoną rodzinę nieodwracalnych precli, patrz \ref{prp:pretzel_not_invertible}.

% MAKOTO SAKUMA - A SURVEY OF THE IMPACT OF THURSTON’S WORK ON KNOT THEORY
% Hartley [129] realized that one can apply this method to the problem of identifying noninvertible knots, as follows. Suppose no automorphism of Γ maps γ to γ−1. Then the set R(G(K), Γ, γ) is possibly different from the set R(G(K), Γ, γ−1), and there is a chance to show noninvertibility of K by comparing the homology invariants associated with φ ∈ R(G(K), Γ, γ) with those associated with φ′ ∈ R(G(K), Γ, γ−1). Hartley showed that this method is quite effective: he completely determined the 36 non-invertible knots up to 10 crossings claimed by Conway to be noninvertible.

\begin{example}
    Węzeł $12a427$ jest zwierciadlany dodatnio, ale nie odwracalny.
\end{example}

Żaden inny węzeł pierwszy o mniej niż 13 skrzyżowaniach nie ma tej cechy.

\begin{example}
    \label{property_of_eight_knot}
    Ósemka $4_1$ jest całkowicie zwierciadlana.
\end{example}

To najprostszy typ symetrii, wystarczy jawnie wskazać przekształcenie między diagramem węzła, jego lustra oraz odwrotności.

Tait odnosił wrażenie, że zwierciadlane węzły mają parzysty indeks skrzyżowań,
ale Hoste (Thistlethwaite?) znalazł w~1998 kontrprzykład o~piętnastu skrzyżowaniach.
Jest on jedynym znanym nam dzisiaj.
Jego gipoteza jest prawdziwa dla węzłów pierwszych, alternujących.

\begin{proposition}[10.4.4 w \cite{kawauchi96}]
    Niech $K$ będzie węzłem zwierciadlanym.
    Wtedy
    \begin{align}
        V(t) & = V(1/t) \\
        P(a, z) & = P(1/a, z) \\
        F(a, z) & = F(1/a, z),
    \end{align}
    gdzie $\jones, P, F$ oznacza kolejno wielomian Jonesa, HOMFLY oraz Kauffmana.
    Równość $\conway(z) = \conway(-z)$ zachodzi dla wszystkich węzłów, zwierciadlanych lub nie.

    Patrz fakt \ref{cor:joines_of_amphicheiral}.
\end{proposition}

Poniższa tabela oparta jest (kolejno) o~ciągi
\href{https://oeis.org/A051766}{51766},
\href{https://oeis.org/A051769}{51769},
\href{https://oeis.org/A051768}{51768},
\href{https://oeis.org/A051767}{51767},
\href{https://oeis.org/A052400}{52400},
z bazy danych ``The On-Line Encyclopedia of Integer Sequences'' (OEIS).

\begin{table}[h]
    \centering
    \begin{tabular}{@{}*{20}l@{}} \toprule
        skrzyżowania & 3 & 4 & 5 & 6 & 7 & 8 & 9 & 10 & 11 & 12 & 13 & 14 \\ \midrule
        całkowicie skrętne & 0 & 0 & 0 & 0 & 0 & 0 & 2 & 27 & 187 & 1103 & 6919 & 37885 \\
        odwracalne & 1 & 0 & 2 & 2 & 7 & 16 & 47 & 125 & 365 & 1015 & 3069 & 8813 \\
        $-$ zwierciadlane & 0 & 0 & 0 & 0 & 0 & 1 & 0 & 6 & 0 & 40 & 0 & 227 \\
        $+$ zwierciadlane & 0 & 0 & 0 & 0 & 0 & 0 & 0 & 0 & 0 & 1 & 0 & 6 \\
        zwierciadlane & 0 & 1 & 0 & 1 & 0 & 4 & 0 & 7 & 0 & 17 & 0 & 41 \\
        \bottomrule
        \hline
    \end{tabular}
    \caption{Liczba węzłów o~poszczególnych typach symetrii}
\end{table}

\begin{tobedone}[Kawauchi, definicja 10.3.2]
    Węzeł $K \subseteq S^3$ jest silnie odwracalny, jeśli istnieje inwolucja pary $(S^3, K)$ która zachowuje orientację sfery, ale odwraca orientację węzła.
\end{tobedone}

Węzeł silnie odwracalny jest odwracalny, ale nie vice versa (Hartley 1980, Whitten 1981).
Chyba, że ograniczymy się do węzłów hiperbolicznych (Kawauchi proposition 10.3.3, cf. 3.2.11).
Bycie silnie odwracalnym nie narzuca żadnych ograniczeń na wielomian Alexandera (odniesienie do $u = 1$?), patrz Sakai 1983.

\subsection{Węzły okresowe}
Można wyróżnić jeszcze jeden rodzaj symetrii.

\begin{definition}
    \label{def:period}
    \index{węzeł!okresowy}
    Węzeł $K$ nazywamy $n$-okresowym, jeśli istnieje obrót $f \colon \R^3 \to \R^3$ o~kąt $2\pi/n$ wokół pewnej prostej $l$, rozłącznej z~węzłem, taki że $f(K) = K$.
\end{definition}

Zamiast obrotów można rozpatrywać dowolne odwzorowania okresowe $f \colon S^3 \to S^3$, których zbiór punktów stałych jest rozłączny z węzłem $K$, homeomorficzny z $S^1$ oraz które trzymają węzeł $K$ w miejscu, ale dostaje się wtedy dokładnie taką samą klasę węzłów.
\index{hipoteza!Smitha}
Wynika to z hipotezy Smitha, otrzymanej z połączenia głębokich teorii dotyczących geometrii i topologii 3-rozmaitości.
% Kawauchi, ćwiczenie 10.1.10
% Morgan-Bass 1984

\begin{proposition}
    Zbiór wszystkich okresów jest niezmiennikiem węzłów.
\end{proposition}

Nieodwracalny węzeł $8_{17}$ nie posiada żadnych okresów.
% ćwiczenie 10.1.5 w Kawauchi
Węzeł $5_1$ 5-okresowy, co widać na standardowym diagramie, oraz 2-okresowy, tę drugą symetrię można dostrzec na diagramie realizującym indeks mostowy.
Trójlistnik ma dokładnie dwa okresy, $2$ i~$3$.
Ogólniej, jak głosi Kawauchi \cite[ćwiczenie 10.1.9]{kawauchi96}:

\begin{proposition}
    Jedynymi okresami węzła $(p, q)$-torusowego są dzielniki liczb $p$ oraz $q$.
\end{proposition}

Z~każdym węzłem okresowym związany jest inny, prostszy węzeł.
Niech $f$ będzie obrotem z definicji \ref{def:period}, zaś $p \colon \R^3 \to \R^3/f \simeq \R^3$ rzutem na przestrzeń ilorazową.
\index{węzeł!ilorazowy}
Wtedy $p(K)$ nazywamy \emph{węzłem ilorazowym}, zaś $K$ to jego $n$-krotne nakrycie.

Murasugi podał dwa warunki, które musi spełniać węzeł o~okresie $n = p^r$, gdzie $r$ jest liczbą pierwszą.
Do ich zrozumienia potrzebujemy prostej definicji.
Ustalmy półprostą, która nie jest styczna do węzła $K$, po czym zorientujmy ją oraz węzeł.
Indeksem zaczepienia $\lambda$ węzła $p(K)$ jest różnica między liczbą skrzyżowań dodatnich oraz ujemnych wzdłuż półprostej (bez znaku).

\begin{proposition}[warunek Murasugiego]
    \index{warunek!Murasugiego}
    \label{prp:murasugi_periodic}
    Niech $K$ będzie węzłem o~okresie $n = p^r$, gdzie $p$ jest liczbą pierwszą.
    Niech $J$ będzie jego węzłem ilorazowym, z~indeksem zaczepienia $\lambda$.
    Wtedy wielomian $\alexander_J$ jest dzielnikiem wielomianu $\alexander_K$ oraz istnieje pewna całkowita liczba $k$, taka że
    \begin{equation}
        \alexander_K(t) \equiv \pm t^k \alexander_J(n)^n \left(1 + t + t^2 + \ldots + t^{\lambda - 1}\right)^{n-1} \mod p.
    \end{equation}
\end{proposition}

\begin{proof}
    Mozolne operacje na macierzach, których wyznacznikiem jest wielomian Alexandera, patrz \cite{murasugi71}.
    Kawauchi przedstawia inny dowód: najpierw dowodzi tego dla węzła torusowego $T_{n, d}$, którego węzłem ilorazowym jest niewęzeł.
    W ogólnym przypadku, korzysta z relacji kłębiastej dla wielomianu Conwaya.
    Szczegóły oraz odsyłacze do dalszych prac znaleźć można w jego przeglądowej publikacji \cite[s. 122-124]{kawauchi96}.
\end{proof}

% Koniec podsekcji Lustro i rewers


\subsection{Węzły okresowe}
\index{węzeł!okresowy|(}%
Można wyróżnić jeszcze jeden rodzaj symetrii.

% DICTIONARY;period;okres;-
% DICTIONARY;periodic;okresowy;węzeł
\begin{definition}
\label{def:period}%
    Węzeł $K$ nazywamy $n$-okresowym, jeśli istnieje obrót $f \colon \R^3 \to \R^3$ o~kąt $2\pi/n$ wokół pewnej prostej $l$, rozłącznej z~węzłem, taki że $f(K) = K$.
\end{definition}

Zamiast obrotów można rozpatrywać dowolne odwzorowania okresowe $f \colon S^3 \to S^3$, których zbiór punktów stałych jest rozłączny z węzłem $K$, homeomorficzny z $S^1$ oraz które trzymają węzeł $K$ w miejscu, ale dostaje się wtedy dokładnie taką samą klasę węzłów.
Czemu?
Wynika to z hipotezy Smitha, otrzymanej z połączenia głębokich teorii dotyczących geometrii i topologii 3-rozmaitości.
\index{hipoteza!Smitha}%
Kawauchi \cite[s. 125]{kawauchi96} odsyła tu do książki Morgana, Bassa \cite{morgan84}, gdzie znajdziemy problem, jego historię i rozwiązanie.
\index[persons]{Morgan, John}%
\index[persnos]{Bass, Hyman}%

\begin{proposition}
    Zbiór wszystkich okresów jest niezmiennikiem węzłów.
\end{proposition}

Nieodwracalny węzeł $8_{17}$ nie posiada żadnych okresów.
% ćwiczenie 10.1.5 w Kawauchi
Węzeł $5_1$ jest 5-okresowy, co widać na standardowym diagramie, oraz 2-okresowy, tę drugą symetrię można dostrzec na diagramie realizującym liczbę mostową.
Trójlistnik ma dokładnie dwa okresy, $2$ i~$3$.
Ogólniej, jak głosi Kawauchi \cite[ćwiczenie 10.1.9]{kawauchi96}:

\begin{proposition}
    Jedynymi okresami węzła $(p, q)$-torusowego są dzielniki liczb $p$ oraz $q$.
\end{proposition}

Z~każdym węzłem okresowym związany jest inny, prostszy węzeł.
Niech $f$ będzie obrotem z definicji \ref{def:period}, zaś $p \colon \R^3 \to \R^3/f \simeq \R^3$ rzutem na przestrzeń ilorazową.
% DICTIONARY;quotient;ilorazowy;węzeł
\index{węzeł!ilorazowy}%
Wtedy $p(K)$ nazywamy \emph{węzłem ilorazowym}, zaś $K$ to jego $n$-krotne nakrycie.

Murasugi podał dwa warunki, które musi spełniać węzeł o~okresie $n = p^r$, gdzie $r$ jest liczbą pierwszą.
Do ich zrozumienia potrzebujemy prostej definicji.
Ustalmy półprostą, która nie jest styczna do węzła $K$, po czym zorientujmy ją oraz węzeł.
Indeksem zaczepienia $\lambda$ węzła $p(K)$ jest różnica między liczbą skrzyżowań dodatnich oraz ujemnych wzdłuż półprostej (bez znaku).

\begin{proposition}[warunek Murasugiego]
\index{warunek!Murasugiego}%
\label{prp:murasugi_periodic}%
    Niech $K$ będzie węzłem o~okresie $n = p^r$, gdzie $p$ jest liczbą pierwszą.
    Niech $J$ będzie jego węzłem ilorazowym, z~indeksem zaczepienia $\lambda$.
    Wtedy wielomian $\alexander_J$ jest dzielnikiem wielomianu $\alexander_K$ oraz istnieje pewna całkowita liczba $k$, taka że
    \begin{equation}
        \alexander_K(t) \equiv \pm t^k \alexander_J(n)^n \left(1 + t + t^2 + \ldots + t^{\lambda - 1}\right)^{n-1} \mod p.
    \end{equation}
\end{proposition}

\begin{proof}
    Mozolne operacje na macierzach, których wyznacznikiem jest wielomian Alexandera, patrz \cite{murasugi71}.
    Kawauchi przedstawia inny dowód: najpierw dowodzi tego dla węzła torusowego $T_{n, d}$, którego węzłem ilorazowym jest niewęzeł.
    W ogólnym przypadku, korzysta z relacji kłębiastej dla wielomianu Conwaya.
    Szczegóły oraz odsyłacze do dalszych prac znaleźć można w jego przeglądowej publikacji \cite[s. 122-124]{kawauchi96}.
\end{proof}

Z prac Borodzika dowiedzieliśmy się, że użytecznym narzędziem do badania okresowości węzłów jest kryterium Naika z pracy \cite{naik97} oraz że można je wzmocnić.
% Na przykład z https://arxiv.org/pdf/1810.03881.pdf się dowiedzieliśmy
\index{kryterium Naika (okresowości)}%

\index{węzeł!okresowy|)}%

% koniec podsekcji Węzły okresowe




\subsection{Suma niespójna i~suma spójna}

\begin{definition}[suma niespójna]
% DICTIONARY;distant union;suma niespójna;-
\index{suma niespójna}%
    Niech $L_1$ oraz $L_2$ będą splotami, które leżą po różnych stronach ustalonej płaszczyzny w przestrzeni $\R^3$.
    Teoriomnogościową sumę $L_1 \sqcup L_2$ nazywamy sumą niespójną splotów.
\end{definition}

\begin{definition}[suma spójna]
% DICTIONARY;connected sum;suma spójna;-
\index{suma spójna}%
    Niech $K_1, K_2$ będą zorientowanymi węzłami.
    Natnijmy każdy z nich w dwóch punktach tego samego krótkiego łuku, a następnie zszyjmy dwoma łukami, które nie przecinają już istniejących, jak na obrazku.
    Otrzymany węzeł nazywamy sumą spójną węzłów $K_1$ oraz $K_2$ i oznaczamy przez $K_1 \shrap K_2$.
\begin{comment}
    \[
        \begin{tikzpicture}[baseline=-0.65ex,scale=0.09]
        \useasboundingbox (-12, -15) rectangle (12, 10);
        \begin{knot}[clip width=5, flip crossing/.list={5}, ignore endpoint intersections=false,]
            \strand[thick] (-3.5, -3.5) [in=down, out=up] to (3.5, 3.5);
            \strand[thick] (3.5, 3.5) [in=right, out=up] to (-4.5, 10);
            \strand[thick] (-4.5, 10) [in=up, out=left] to (-10, 3.5);
            \strand[thick] (-10, 3.5) to (-10, -3.5);
            \strand[thick] (-10, -3.5) [in=left, out=down] to (-4.5, -10);
            \strand[thick] (-4.5, -10) [in=down, out=right] to (3.5, -3.5);
            \strand[thick] (3.5, -3.5) [in=down, out=up] to (-3.5, 3.5);
            \strand[thick] (-3.5, 3.5) [in=left, out=up] to (4.5, 10);
            \strand[thick] (4.5, 10) [in=up, out=right] to (10, 3.5);
            \strand[thick, -Latex] (10, 3.5) to (10, -3.5);
            \strand[thick] (10, -3.5) [in=right, out=down] to (4.5, -10);
            \strand[thick] (4.5, -10) [in=down, out=left] to (-3.5, -3.5);
            \node at (0, -15) {$K_1$};
        \end{knot}
        \end{tikzpicture}
        \shrap
        \begin{tikzpicture}[baseline=-0.65ex,scale=0.09]
        \useasboundingbox (-12, -15) rectangle (12, 10);
        \begin{knot}[clip width=5, flip crossing/.list={6}, ignore endpoint intersections=false,]
            \strand[thick] (-3.5, -3.5) [in=down, out=up] to (3.5, 3.5);
            %\strand[thick] (3.5, 3.5) [in=right, out=up] to (-4.5, 10);
            %\strand[thick] (-4.5, 10) [in=up, out=left] to (-10, 3.5);
            \strand[thick] (-10, -3.5) [in=left, out=up] to (0, 6.5);
            \strand[thick, Latex-] (-10, -3.5) [in=left, out=down] to (-4.5, -10);
            \strand[thick] (-4.5, -10) [in=down, out=right] to (3.5, -3.5);
            \strand[thick] (3.5, -3.5) [in=down, out=up] to (-3.5, 3.5);
            %\strand[thick] (-3.5, 3.5) [in=left, out=up] to (4.5, 10);
            %\strand[thick] (4.5, 10) [in=up, out=right] to (10, 3.5);
            \strand[thick] (10, -3.5) [in=right, out=up] to (0, 6.5);
            \strand[thick] (10, -3.5) [in=right, out=down] to (4.5, -10);
            \strand[thick] (4.5, -10) [in=down, out=left] to (-3.5, -3.5);
            %
            \strand[thick] (-3.5, 3.5) [in=left, out=up] to (0, 10);
            \strand[thick] (3.5, 3.5) [in=right, out=up] to (0, 10);
            \node at (0, -15) {$K_2$};
        \end{knot}
        \end{tikzpicture}
        =
        \begin{tikzpicture}[baseline=-0.65ex,scale=0.09]
        \useasboundingbox (-27, -15) rectangle (27, 10);
        \begin{knot}[clip width=5, flip crossing/.list={5, 22, 23}, ignore endpoint intersections=false,]
            \strand[thick] (-18.5, -3.5) [in=down, out=up] to (-11.5, 3.5);
            \strand[thick] (-11.5, 3.5) [in=right, out=up] to (-19.5, 10);
            \strand[thick] (-19.5, 10) [in=up, out=left] to (-25, 3.5);
            \strand[thick] (-25, 3.5) to (-25, -3.5);
            \strand[thick] (-25, -3.5) [in=left, out=down] to (-19.5, -10);
            \strand[thick] (-19.5, -10) [in=down, out=right] to (-11.5, -3.5);
            \strand[thick] (-11.5, -3.5) [in=down, out=up] to (-18.5, 3.5);
            \strand[thick] (-18.5, 3.5) [in=left, out=up] to (-10.5, 10);
            \strand[thick] (-10.5, 10) [in=left, out=right] to (-5, 2);
            \strand[thick, -Latex] (-5, 2) to (-5+6, 2);
            \strand[thick] (5, 2) to (-5+6, 2);
            \strand[thick] (3, -2) to [in=left, out=right] (10.5, -10);
            \strand[thick, -Latex] (3, -2) to (-3, -2);
            \strand[thick] (-5, -2) to (-3, -2);
            \strand[thick] (-5, -2) [in=right, out=left] to (-10.5, -10);
            \strand[thick] (-10.5, -10) [in=down, out=left] to (-18.5, -3.5);
            %%%
            \strand[thick] (11.5, -3.5) [in=down, out=up] to (18.5, 3.5);
            \strand[thick] (-10 +15, 2) [in=left, out=right] to (15, 6.5);
            \strand[thick] (10.5, -10) [in=down, out=right] to (18.5, -3.5);
            \strand[thick] (18.5, -3.5) [in=down, out=up] to (11.5, 3.5);
            \strand[thick] (25, -3.5) [in=right, out=up] to (15, 6.5);
            \strand[thick] (25, -3.5) [in=right, out=down] to (19.5, -10);
            \strand[thick] (19.5, -10) [in=down, out=left] to (11.5, -3.5);
            \strand[thick] (11.5, 3.5) [in=left, out=up] to (15, 10);
            \strand[thick] (18.5, 3.5) [in=right, out=up] to (15, 10);
            %%%
            \node at (0, -15) {$K_1 \shrap K_2$};
        \end{knot}
        \end{tikzpicture}
    \]
\end{comment}
\end{definition}

Pojęcie sumy spójnej węzłów (oraz satelity, opisane później) wprowadził do matematyki Schubert w \cite{schubert49}.
\index[persons]{Schubert, Horst}%

Ważna jest orientacja składników: suma dwóch trójlistników może być węzłem babskim lub prostym\footnote{To jedno z niewielu miejsc, gdzie nomenklatura pochodzi od żeglarzy.}.
\label{two_sums_of_two_trefoils}%
Uzasadnienie, że te węzły są różne, nie jest łatwym zadaniem.
Fox twierdzi, że Seifert wiedział to już w~1933 roku.
% Seifert, Herbert - Verschlingungsinvarianten. (German) Zbl 0008.18101 Sitzungsber. Preuß. Akad. Wiss., Phys.-Math. Kl. 1933, No. 26-29, 811-828 (1933).
Pokazał też w~króciutkim artykule \cite{fox52}, że

\begin{proposition}
    Dopełnienia węzła babskiego oraz prostego nie są homeomorficzne.
\end{proposition}

Suma tak samo skręconych trójlistników ma niezerową sygnaturę, więc nie może być plastrowa.
Natomiast suma przeciwnie skręconych jest plastrowa.\footnote{Nie wiem, skąd to wiem.}
\index{węzeł!plastrowy}

Warunku, by zszywające łuki nie przecinały diagramów, nie można pominąć: Cromwell w~\cite[s. 90]{cromwell04} pokazuje przykład dwóch niewęzłów, z~których otrzymano niepoprawnie dwie różne sumy, $6_1$ oraz $8_{20}$.

W topologii rozważa się podobną operację dla $n$-rozmaitości: z~każdej z~nich wycina się kulę, po czym skleja wzdłuż brzegowej sfery w~jedną rozmaitość.
Ale kiedy zajmujemy się węzłami, nie interesuje nas struktura rozmaitości (gdyż każdy węzeł jest homeomorficzny z~okręgiem), tylko zanurzenie w otaczającą przestrzeń.

\begin{proposition}
    Suma spójna węzłów jest dobrze określonym działaniem.
\end{proposition}

Suma spójna nie jest dobrze określona dla splotów: nie istnieje kanoniczny wybór, które ogniwa łączyć ze sobą.

\begin{proof}
    Niech dane będą węzły $K_1$ oraz $K_2$ oraz dwa różne łuki $\gamma_1$, $\gamma_2$, których można użyć do konstrukcji sumy spójnej.
    Skurczmy $K_1$ tak, by był bardzo mały, przeciągnijmy najpierw przez łuk $\gamma_1$, a~następnie wzdłuż węzła $K_2$ do miejsca, gdzie zaczyna się łuk $\gamma_2$.
    Na koniec odwróćmy proces, z łukiem~$\gamma_2$ w~miejscu łuku $\gamma_1$.
\end{proof}

\begin{proposition}
    Suma spójna jest działaniem łącznym oraz przemiennym.
    Niewęzeł stanowi jej element neutralny.
\end{proposition}

Prosty dowód tego faktu pozostawiamy Czytelnikowi.
W języku algebry mówimy, że węzły z~sumą spójną tworzą półgrupę (tak jak liczby naturalne z działaniem dodawania).
Dużo później pokażemy, że działaniu $\shrap$ brakuje elementów przeciwnych, więc ta struktura algebraiczna nie jest grupą.

\begin{proposition}
\label{first_time_sum_is_trivial}%
    Niech $K_1, K_2$ będą takimi węzłami, że $K_1 \shrap K_2 = \SmallUnknot$. Wtedy $K_1 = K_2 = \SmallUnknot$.
\end{proposition}

\begin{proof}[Niedowód]
% DICTIONARY;Mazur swindle;szwindel Mazura;-
    Technika ta zwana jest szwindlem Mazura.
\index{szwindel Mazura}%
    Załóżmy, że $K \shrap L = \SmallUnknot$ i~dopuśćmy wyjątkowo węzły dzikie.
    Skonstruujmy sumę $K \shrap L \shrap K \shrap \ldots$,
    przy czym kolejne składniki powinny zmniejszać się,
    aby ich suma nadal była węzłem.
    Wtedy
    \begin{align*}
        K & \simeq K \shrap [(L \shrap K) \shrap (L \shrap K) \ldots] \\
         & \simeq (K \shrap L) \shrap (K \shrap L) \shrap \ldots
         \simeq \SmallUnknot \shrap \SmallUnknot \shrap \ldots
         \simeq \SmallUnknot.
    \end{align*}
    Analogicznie pokazujemy, że $L \simeq \SmallUnknot$.
    (To jedyne miejsce w~całej książce, gdzie użyte zostają węzły dzikie.)
\end{proof}

W poprzednim wydaniu książki znajdowała się informacja, że dla dowodu tego faktu trzeba przytoczyć narzędzia topologii algebraicznej: powierzchnie Seiferta i~genus; wtedy jest to bezpośredni wniosek z~faktów \ref{prp:genus_detects_unknot} oraz \ref{prp:genus_of_sum}.
O~tym samym dowodzie wspomina Kawauchi \cite[s. 33]{kawauchi96}, a~fakt nazywa twierdzeniem o~nieanulowaniu.
Okazuje się jednak, że elementarny dowód istnieje!
% odkryte w https://aperiodical.com/2018/07/the-big-internet-math-off-round-1-jim-propp-v-zoe-griffiths/

Trzeba zajrzeć do \cite[s. 18-20]{kauffman95} dla dwóch rysunków tamże.

\begin{proof}
    (Na podstawie \cite[s. 18-20]{kauffman95}).
    Wyobraźmy sobie, że węzeł oraz torus połykająco-podążający $T$ został zawieszony między dwiema ścianami pokoju i~załóżmy nie wprost, że suma $K = K_1 \shrap K_2$ jest trywialna.
    Wtedy pewien homeomorfizm pokoju, który nie rusza ścian, prostuje sumę (zamienia pozornie zaplątany węzeł $K$ w odcinek $L$). 

    Niech $\pi$ będzie dowolną płaszczyzną zawierającą wyprostowaną sumę $K$.
    Tnie część wspólną torusa $T$ oraz ścian w czterech punktach, oznaczmy je $A, B$ (na lewej ścianie) oraz $C, D$ (na prawej).
    Zauważmy, że $\pi$ tnie $T$ w łukach, które wychodzą z $A, B, C, D$ oraz pewnych zamkniętych krzywych.
    Łuk wychodzący z~punktu $A$ nie może łączyć go z punktami $B$ lub $D$, ponieważ te leżą po drugiej stronie odcinka $L$ na płaszczyźnie $\pi$.
    
    Łuk $AC$ przedstawia niewęzeł.
    Jednocześnie jest on obrazem pewnego łuku, który łączył końce torusa $T$, zatem musi być równoważny z~węzłem-towarzyszem.
\end{proof}

Półgrupę węzłów z~operacją sumy spójnej można uczynić grupą na dwa sposoby: albo zmieniając działanie, albo osłabiając równoważność węzłów.
Drugi pomysł jest lepszy niż pierwszy.
Na początku lat pięćdziesiątych Milnor wprowadził pojęcie zgodności.
\index{węzeł!plastrowy}%
\index{węzeł!zgodny}%
Element neutralny nowej grupy to węzły plastrowe, ich opis leży w~sekcji \ref{sec:slice}.
Zgodność i plastrowe węzły to zagadnienia zakorzenione w~czterowymiarowej topologii.

Kawauchi \cite[s. 50-53]{kawauchi96} opisuje $2n$-sumę Murasugiego, narzędzie będące uogólnieniem sumy spójnej (która odpowiada wartości $n = 1$) z komentarzem, że jest bardzo przydatna do badania powierzchni Seiferta czy genusu.
\index{suma Murasugiego}%
Została wprowadzona dawno temu w~\cite{murasugi58}, by szacować stopień wielomianu Alexandera alternujących węzłów.
\index[persons]{Murasugi, Kunio}%

% DICTIONARY;suma paskowa;band sum;-
Innym uogólnieniem jest suma paskowa, patrz \cite[s. 31-32, 43]{kawauchi96}, specjalny przypadek hiperbolicznej transformacji splotu oraz fuzji splotu.
% TODO: sprawdzić, czy fuzja splotu ma trafić do indeksu

% Koniec podsekcji Suma niespójna i suma spójna




\input{10-introduction/104-primality}

\section{Niezmienniki liczbowe}
Jak wspomnieliśmy na początku rozdziału, sprawdzenie,
czy dwa diagramy przedstawiają sploty równoważne,
jest uciążliwym i~czasochłonnym zadaniem.
Aby je uprościć, podamy opis kilku prostych niezmienników o~naturalnych wartościach.
Zachodzą implikacje:
sploty równoważne $\Rightarrow$ ta sama wartość niezmiennika
oraz różne wartości niezmiennika $\Rightarrow$ różne sploty.

Tutaj przedstawiamy jedynie te niezmieniki, które nie wymagają mocnej znajomości reszty książki.
Są one miarą złożoności splotów zgodnie z następującym przepisem: niech $f$ będzie pewną funkcją określoną dla dowolnego diagramu splotu.
Wtedy odwzorowanie
\begin{equation}
    f(L) = \min \{f(D) : D \text{ jest diagramem splotu } L\}
\end{equation}
stanowi niezmiennik splotów.
Dowód jest trywialny i pozostawiamy go jako ćwiczenie dla Czytelnika.
Im większa wartość funkcji $f$, tym bardziej skomplikowany splot.

Później poznamy inne niezmienniki, oprócz opisanych powyżej miarą złożoności jest też liczba warkoczowa (definicja \ref{def:braid_number}), ale nie wyznacznik (definicja \ref{def:determinant}) i sygnatura (definicja \ref{def:signature}).
Przekonamy się też, że istnieją użyteczne niezmienniki, które są wielomianami albo innymi obiektami algebraicznymi.


% DICTIONARY;crossing number;indeks skrzyżowaniowy;-
\subsection{Indeks skrzyżowaniowy}
\index{indeks skrzyżowaniowy|(}%
\begin{definition}
    Niech $L$ będzie splotem.
    Minimalną liczbę skrzyżowań występujących na diagramach przedstawiających splot $L$ nazywamy indeksem skrzyżowaniowym i~oznaczamy $\crossing L$.
\end{definition}

Pytanie, czy indeks skrzyżowaniowy jest addytywny, to jeden z najstarszych problemów teorii węzłów.

\begin{conjecture}
\index{hipoteza!o indeksie skrzyżowaniowym}%
\index{suma spójna}%
\label{con:crossing_additive}%
    Niech $K_1$ oraz $K_2$ będą węzłami.
    Wtedy $\crossing K_1 + \crossing K_2 = \crossing K_1 \shrap K_2$.
\end{conjecture}

Oto częściowe odpowiedzi.
Jeśli $K_1, K_2$ są alternującymi węzłami o~odpowiednio $c_1, c_2$ skrzyżowaniach, to istnieje alternujący diagram ich sumy $K_1 \shrap K_2$ o~$c_1 + c_2$ skrzyżowaniach.
\index{węzeł!alternujący}%
Kauffman \cite[twierdzenie 2.10]{kauffman87}, Murasugi \cite[wniosek 6]{murasugi87} oraz Thistlethwaite \cite[wniosek 1]{thistlethwaite87} pokazali niezależnie, że diagram ten jest minimalny.
\index[persons]{Kauffman, Louis}%
\index[persons]{Murasugi, Kunio}%
\index[persons]{Thistlethwaite, Morwen}%

% DICTIONARY;adequate;adekwatny;węzeł
Thistlethwaite rozszerzył wynik do tak zwanych węzłów adekwatnych: sam \cite{thistlethwaite88} albo razem z Lickorishem \cite{lickorish88}.
\index[persons]{Lickorish, William}%
\index{węzeł!adekwatny}%
Mając diagram węzła $K$, można wygładzić wszystkie skrzyżowania dodatnio i dostać splot złożony z rozłącznych okręgów.
Jeżeli zmiana dowolnego pojedynczego wygładzenia na ujemne sprawia, że liczba ogniw splotu zmniejsza się, diagram nazywamy dodatnio adekwatnym.
Węzeł posiadający taki diagram też nazywamy dodatnio adekwatnym.
Analogicznie definiuje się ujemną adekwatność.
Węzeł, który jest dodatnio oraz ujemnie adekwatny, nazywamy krótko adekwatnym.

Na początku XX wieku Diao \cite{diao04} oraz Gruber \cite{gruber03} niezależnie udowodnili hipotezę \ref{con:crossing_additive} dla pewnej szerokiej klasy węzłów, obejmującej wszystkie węzły torusowe, wiele węzłów alternujących oraz pewne inne obiekty, których nie chcemy opisywać.
% diao04 -> tw. 3.8
\index[persons]{Diao, Yuanan}%
\index[persons]{Gruber, Hermann}%
\index{węzeł!torusowy}%

Lackenby w~pracy \cite{lackenby09} pokazał, że dla pewnej stałej $N \le 152$ zachodzi
\index[persons]{Lackenby, Marc}
\begin{equation}
    \frac 1 N \sum_{i=1}^n \crossing{K_i} \le \crossing \left(\bigshrap_{i=1}^n K_i\right) \le \sum_{i=1}^n \crossing{K_i}.
\end{equation}
(Tylko pierwsza nierówność jest ciekawa).
Jego argumentu wykorzystującego powierzchnie normalne nie można poprawić tak, by otrzymać stałą $N = 1$.
Jednocześnie od 2009 roku nie widać postępu nad hipotezą.

\index{indeks skrzyżowaniowy|)}%

% Koniec podsekcji Indeks skrzyżowaniowy




% DICTIONARY;unknotting number;liczba gordyjska;-
\subsection{Liczba gordyjska}
\index{liczba gordyjska|(}%

\begin{definition}
    Niech $L$ będzie splotem.
    Minimalną liczbę skrzyżowań, które trzeba odwrócić na pewnym jego diagramie, by dostać niewęzeł, nazywamy liczbą gordyjską i~oznaczamy $\unknotting L$.
\end{definition}

Zgodnie z ,,klasyczną'' definicją, między odwracaniem kolejnych skrzyżowań mamy prawo wykonać izotopie otaczające; natomiast zgodnie ze ,,standardową'' definicją, takie izotopie są zabronione.
Obie definicje są równoważne: tłumaczy to książka Adamsa \cite[s. 58]{adams94}.

\begin{lemma}
\label{lem:unknotting_well_defined}%
    W dowolnym rzucie splotu można odwrócić pewne skrzyżowania tak, by uzyskać diagram niesplotu.
\end{lemma}

\begin{proof}
    Bez straty ogólności załóźmy, że diagram przedstawia węzeł.
    Ustalmy zatem diagram węzła i~wybierzmy jakiś początkowy punkt na nim, różny od skrzyżowania wraz z~kierunkiem, wzdłuż którego będziemy przemierzać węzeł.
    Za każdym razem, kiedy odwiedzamy nowe skrzyżowanie, zmieniamy je w~razie potrzeby na takie, przez które przemieszczamy się wzdłuż górnego łuku.
    Skrzyżowań już odwiedzonych nie zmieniamy wcale.

    Teraz wyobraźmy sobie nasz nowy węzeł w~trójwymiarowej przestrzeni $\mathbb R^3$, przy czym oś $z$ skierowana jest z~płaszczyzny, w~której leży diagram, w~naszą stronę.
    Umieśćmy początkowy punkt tak, by jego trzecią współrzędną była $z = 1$.

    Przemierzając węzeł, zmniejszamy stopniowo tę współrzędną, aż osiągniemy wartość $0$ tuż przed punktem, z~którego wyruszyliśmy.
    Połączmy obydwa punkty (początkowy oraz ten, w~którym osiągamy współrzędną $z = 0$) pionowym odcinkiem.
    Zauważmy, że kiedy patrzymy na węzeł w~kierunku osi $z$, nie widzimy żadnych skrzyżowań.

    Oznacza to, że nasza procedura przekształciła początkowy diagram w~diagram niewęzła, co należało okazać.
\end{proof}

Shimizu w pracy \cite{shimizu14} rozpatruje różne operacje, które rozwiązują węzły lub sploty.
\index[persons]{Shimizu, Ayaka}%
Nie będziemy się nimi zajmować, podamy tylko przykład: zamiana pod- i nadskrzyżowań wokół obszaru na diagramie rozwiązuje węzły, ale nie sploty; kontrprzykładem jest splot Hopfa.
\index{splot!Hopfa}%
Patrz też, co pisze Kawauchi w \cite[s. 141-154]{kawauchi96}.

% znalezione przypadkiem w MR3143585
% 1979 Nakanishi: hipoteza że 4-ruch jest rozwiązujący
% dowody: 2-mostowe i 3-mostowe węzły, wszystkie do 12 skrzyżowań

Dla każdego nietrywialnego splotu istnieje diagram wymagający odwrócenia dowolnie wielu skrzyżowań.
Wcześniej Nakanishi \cite{nakanishi83} znalazł 2-gordyjski diagram 1-gordyjskiego węzła $6_2$ oraz udowodnił, że każdy nietrywialny węzeł ma diagram, który nie jest 1-gordyjski (trzynaście lat później, w \cite{nakanishi96}).
\index[persons]{Nakanishi, Yasutaka}%
Jego wyniki uogólnia praca Taniyamy \cite{taniyama09}:
\index[persons]{Taniyama, Kouki}%

\begin{proposition}
    Dla każdego $n \in \N$ istnieje diagram $D$ nietrywialnego splotu $L$ taki, że $\unknotting D \ge n$.
\end{proposition}

Pokazany jest tam jeszcze jeden godny uwagi fakt.

\begin{proposition}
    Jeśli liczba gordyjska diagramu $D$ wynosi $\frac 12 (\crossing D - 1)$, co jest maksymalną możliwą wartością zgodnie z~naszym prostym ograniczeniem, to węzeł jest $(2,p)$-torusowy albo wygląda jak diagram niewęzła po pierwszym ruchu Reidemeistera.
\end{proposition}


\subsubsection{Sploty 1-gordyjskie}
Sploty o liczbie gordyjskiej 1 zasługują na szczególną uwagę.

\begin{proposition}
\index{węzeł!wymierny}%
    Niech $L$ będzie wymiernym splotem 1-gordyjskim.
    Wtedy na minimalnym diagramie $L$ jedno ze skrzyżowań jest rozwiązujące.
\end{proposition}

\begin{proof}
\index{człowiek!Kanenobu, ?}%
\index{człowiek!Murakami, ?}%
\index{człowiek!Kohn, ?}%
    Kanenobu, Murakami dla węzłów \cite{kanenobumurakami86}, wkrótce po tym Kohn dla splotów \cite{kohn91}.
\end{proof}

Coward, Lackenby dowiedli w~\cite{coward11}, że jeśli $K$ jest 1-gordyjski i~o genusie 1, to z~dokładnością do pewnej relacji równoważności, tylko jedna zmiana skrzyżowania rozwiązuje go; chyba że $K$ jest ósemką -- wtedy takie zmiany są dwie.
\index{człowiek!Coward, ?}%
\index{człowiek!Lackenby, Marc}%

\begin{tobedone}
    Kawauchi, s. 151: $\unknotting = 1, g = 1$ to duble.
\end{tobedone}

\begin{proposition}
\label{prp:unknotting_one_prime}%
    Węzły $1$-gordyjskie są pierwsze.
\end{proposition}

Podejrzewał to Hilmar Wendt w~1937 roku, kiedy policzył liczbę gordyjską węzła babskiego używając homologii rozgałęzionego nakrycia cyklicznego \cite{wendt37}.
\index{człowiek!Wendt, Hilmar}%

\begin{proof}[Niedowód]
\index{człowiek!Scharlemann, Martin}%
\index{człowiek!Lackenby, Marc}%
\index{człowiek!Zhang, Xingru}%
    W pracy \cite{scharlemann85} z~1985 roku Scharlemann podał dość zawiłe uzasadnienie, w~które zamieszane były grafy planarne.
    Obecnie znamy prostsze dowody, patrz \cite{lackenby97} (Lackenby) albo \cite{zhang91} (Xingru).
\end{proof}

Scharlemann pokazał w \cite[wniosek 1.6]{scharlemann98}, że liczba gordyjska jest podaddytywna, to znaczy zachodzi $\unknotting(K_1 \shrap K_2) \le \unknotting(K_1) + \unknotting(K_2)$.
Stąd oraz z faktu \ref{prp:unknotting_one_prime} wynika, że suma dwóch $1$-gordyjskich węzłów jest $2$-gordyjska, ale od początku teorii węzłów podejrzewano dużo więcej:

\begin{conjecture}
\index{hipoteza!o liczbie gordyjskiej}%
    Niech $K_1, K_2$ będą węzłami.
    Wtedy $\unknotting (K_1 \shrap K_2) = \unknotting(K_1) + \unknotting(K_2)$, czyli liczba gordyjska jest addytywna.
\end{conjecture}




\subsubsection{Dolne ograniczenia liczby gordyjskiej}
Dokładna wartość liczby gordyjskiej jest znana tylko dla niektórych klas węzłów, na przykład torusowych (fakt \ref{prp:torus_unknotting_number}) albo skręconych.
\index{węzeł!torusowy}%
\index{węzeł!skręcony}%

Jeśli odwrócenie pewnych skrzyżowań daje niewęzeł, to odwrócenie pozostałych także.
To daje proste liczby gordyjskiej: $2 \unknotting K \le \crossing K$.
Nie jest zbyt pomocne, daje rozstrzygnięcie pięć razy dla pierwszych węzłów do 12 skrzyżowań: $3_{1}$, $5_{1}$, $7_{1}$, $9_{1}$, $11a_{367}$.

Borodzik oraz Friedl podali niedawno całkiem mocne ograniczenia na liczbę gordyjską w~pracach \cite{borodzik14} i~\cite{borodzik15}.
\index[persons]{Borodzik, Maciej}%
\index[persons]{Friedl, Stefan}%
Ich narzędziem jest parowanie Blanchfielda.
\index{parowanie Blanchfielda}%
Poprawiają tam starsze estymaty wynikające z~sygnatury Levine'a-Tristrama, indeksu Nakanishiego oraz przeszkody Lickorisha.
\index{indeks Nakanishiego}%
\index{przeszkoda Lickorisha}%
\index{sygnatura!Levine'a-Tristrama}%
Wśród węzłów o~co najwyżej dwunastu skrzyżowaniach 25 ma liczbę gordyjską równą co najmniej trzy, trudno było uzasadnić to innymi metodami.




\subsubsection{Znane wartości}
Dotychczas wyznaczono liczbę gordyjską prawie wszystkich węzłów pierwszych o~co najwyżej dziesięciu skrzyżowaniach.
Cha, Livingston \cite{cha18} podają następującą listę wyjątków:
$10_{11}$, $10_{47}$, $10_{51}$, $10_{54}$, $10_{61}$, $10_{76}$, $10_{77}$, $10_{79}$, $10_{100}$ (stan na rok 2018).
Poniżej podajemy za stroną internetową KnotInfo\footnote{Patrz \url{https://knotinfo.math.indiana.edu/descriptions/unknotting_number.html}. Pomijając węzły torusowe, skopiowane przez nas na liście oraz 1-gordyjskie, do dziesięciu skrzyżowań zostawia to: 2 węzły o~siedmiu skrzyżowaniach, 3 o~ośmiu, 15 o~dziewięciu i~68 o~dziesięciu.} listę odkrywców liczb gordyjskich węzłów do 10 skrzyżowań.
KnotInfo wymienia więcej, bo węzły do 12 skrzyżowań.

\begin{compactitem}
\item Lickorish \cite{lickorish85}: $7_{4}$.
\item Kanenobu, Murakami \cite{kanenobumurakami86}: $8_{3}$, $8_{4}$, $8_{6}$, $8_{8}$, $8_{12}$, $9_{5}$, $9_{8}$, $9_{15}$, $9_{17}$, $9_{31}$.
\item Szabó \cite{szabo05}: $8_{10}$, $10_{48}$, $10_{52}$, $10_{54}$ ($\unknotting \neq 1$), $10_{57}$, $10_{58}$, $10_{64}$, $10_{68}$, $10_{70}$, $10_{77}$ ($\unknotting \neq 1$), $10_{110}$, $10_{112}$, $10_{116}$, $10_{117}$, $10_{125}$, $10_{126}$, $10_{130}$, $10_{135}$, $10_{138}$, $10_{158}$, $10_{162}$.
\item Murakami, Yasuhara \cite{yasuhara00}, Stojmenow \cite{stoimenow04}: $8_{16}$.
\item Stojmenow \cite{stoimenow04}: $8_{18}$, $9_{37}$, $9_{40}$, $9_{46}$, $9_{48}$, $9_{49}$, $10_{103}$.
\item Owens \cite{owens08}: $9_{10}$, $9_{13}$, $9_{35}$, $9_{38}$, $10_{53}$, $10_{101}$, $10_{120}$.
\item Kanenobu, Murakami \cite{kanenobumurakami86}, Stojmenow \cite{stoimenow04}: $9_{15}$, $9_{17}$.
\item Kobayashi \cite{kobayashi89}: $9_{25}$.
\item Gordon, Luecke \cite{gordon06}, Szabó \cite{szabo05}: $9_{29}$, $10_{81}$, $10_{87}$, $10_{90}$, $10_{93}$, $10_{94}$, $10_{96}$.
\item Adams? \cite[s. 62]{adams94}: $10_{8}$.
\item Miyazawa \cite{miyazawa98}: $10_{65}$, $10_{69}$, $10_{89}$, $10_{108}$, $10_{163}$, $10_{165}$.
\item Traczyk \cite{traczyk99}, Szabó \cite{szabo05}: $10_{67}$.
\item Szabó \cite{szabo05}, ($\unknotting \neq 1$ Gordon, Luecke \cite{gordon06}): $10_{79}$.
\item Gordon, Luecke \cite{gordon06}, Szabó \cite{szabo05}, Nakanishi \cite{nakanishi05}: $10_{83}$.
\item Stojmenow \cite{stoimenow04}, Szabó \cite{szabo05}, Gordon, Luecke \cite{gordon06}: $10_{86}$.
\item Miyazawa \cite{miyazawa98}, Nakanishi \cite{nakanishi05}: $10_{97}$.
\item Szabó \cite{szabo05}, Stojmenow \cite{stoimenow04}, Nakanishi \cite{nakanishi05}: $10_{105}$, $10_{106}$, $10_{109}$, $10_{121}$.
\item Stojmenow \cite{stoimenow04}, Szabó \cite{szabo05}: $10_{131}$ (jedyny 1-gordyjski na tej liście!).
\item Gibson, Ishikawa \cite{ishikawa02}: $10_{139}$, $10_{145}$, $10_{152}$.
\item Gordon, Luecke \cite{gordon06}, Szabó \cite{szabo05}: $10_{148}$, $10_{151}$.
\item Gordon, Luecke \cite{gordon06}: $10_{153}$.
\item Stojmenow \cite{stoimenow03}, Gibson, Ishikawa \cite{ishikawa02}: $10_{154}$.
\item Gibson, Ishikawa \cite{ishikawa02}: $10_{161}$.
\end{compactitem}




\subsubsection{Przykład Nakanishiego-Bleilera. Hipoteza Bernharda-Jablana}
Najpierw Nakanishi \cite{nakanishi83}, a potem Bleiler \cite{bleiler84} odkryli fascynujący przykład wymiernego węzła $10_8$, który jest $2$-gordyjski, ale świadkiem tego nie może być żaden diagram mininalny, ponieważ, co jeszcze bardziej fascynujące, węzeł ten posiada tylko jeden diagram o~dziesięciu skrzyżowaniach oraz liczbie gordyjskiej 3.
\index[persons]{Bleiler, Steven}%
\index[persons]{Nakanishi, Yasutaka}%
\index{węzeł!10-8}%
Wynika stąd, że liczba $\unknotting$ nie musi być osiągana przez diagram minimalny, wbrew powszechnym przypuszczeniom obecnym jeszcze w latach 70.
Praca \cite{bernhard94} opisuje nieskończoną rodzinę węzłów zaczynającą się od węzła Bleilera.

Przykład Bleilera pokazuje, że do szukania liczby gordyjskiej potrzeba wyrafinowanego algorytmu.
Ponieważ odwrócenie jednego ze skrzyżowań na minimalnym diagramie węzła $10_8$ daje $1$-gordyjski węzeł $4_1, 5_1, 6_1$ lub $6_2$, możemy liczyć, że każdy diagram minimalny ma skrzyżowanie, którego odwrócenie zmniejsza liczbę gordyjską.
Dlatego jeszcze w~latach 90. Bernhard \cite{bernhard94} i Jablan \cite{jablan98} postawili hipotezę:

\begin{conjecture}[Bernharda-Jablana]
\index[persons]{Bernhard, James}%
\index[persons]{Jablan, Slavik}%
\index{hipoteza!Bernharda-Jablana}%
\label{con:bernhard_jablan}%
    Niech $K$ będzie węzłem z diagramem $D$, który realizuje liczbę gordyjską $\unknotting K$.
    Istnieje wtedy skrzyżowanie, którego odwrócenie daje nowy diagram $D'$ nowego węzła $K'$ o~mniejszej liczbie gordyjskiej: $1 + \unknotting D' = \unknotting D$.
\end{conjecture}

Zakładając prawdziwość hipotezy~\ref{con:bernhard_jablan}, mamy prosty sposób na wyznaczenie liczby $\unknotting K$: weźmy skończenie wiele diagramów minimalnych dla węzła $K$, na każdym z~nich odwracajmy skrzyżowania i rekursywnie szukajmy liczb gordyjskich prostszych węzłów.
Najmniejsza spośród nich różni się wtedy o~jeden od liczby $\unknotting K$.

Brittenham, Hermiller w artykule \cite{brittenham21} twierdzą, że hipoteza jest fałszywa.
Kontrprzykład został znaleziony komputerowo, z pomocą programu SnapPy.
\index{program SnapPy}%
\index[persons]{Brittenham, Mark}%
\index[persons]{Hermiller, Susan}%
Prawdziwość sprawdzono natomiast dla węzłów do jedenastu skrzyżowań oraz splotów o dwóch ogniwach do dziewięciu skrzyżowań (Kohn w \cite{kohn93}?).
\index[persons]{Kohn, Peter}%

\begin{example}[Brittenham, Hermiller]
\index{węzeł!12n-288}%
\index{węzeł!12n-491}%
\index{węzeł!12n-501}%
\index{węzeł!13n-3370}%
    Hipoteza Bernharda-Jablana jest fałszywa dla co najmniej jednego spośród czterech węzłów: $12n_{288}$, $12n_{491}$, $12n_{501}$, $13n_{3370}$.
\end{example}

Bleiler postawił w~\cite{bleiler84} problem: czy jeden węzeł może mieć kilka diagramów minimalnych, z~których tylko niektóre są świadkiem $1$-gordyjskości?
Rozwiązanie przyszło z Japonii: według Kanenobu, Murakamiego \cite{kanenobumurakami86} dzieje się tak m.in. dla węzła $8_{14}$.
\index{węzeł!8-14}%
\index[persons]{Kanenobu, Taizo}%
\index[persons]{Murakami, Hitoshi}%
Stojmenow w~pracy \cite{stoimenow01} pełnej różnych przykładów wskazał dodatkowo węzły $14_{36750}$ oraz $14_{36760}$.
\index{węzeł!14-36750}%
\index{węzeł!14-36760}%
\index[persons]{Stojmenow, Aleksander}%




\subsubsection{Liczba gordyjska jako metryka}
Mając dane dwa węzły $K_0, K_1$, rozpatrzmy wszystkie homotopie
\begin{equation}
    f : [0,1] \times S^1 \to \R^3
\end{equation}
takie, że wszystkie funkcje $f_t$ są zanurzeniami z co najwyżej jednym punktem podwójnym.
Zażądajmy dodatkowo, by styczne do krótkich łuków, które przecinają się w tym punkcie, były od siebie różne.
Odległością gordyjską między węzłami $K_0, K_1$ jest minimalna liczba podwójnych punktów, jakie posiada homotopia $f$.
Twierdzenie C~z~pracy\footnote{To nie jest główne twierdzenie tamże. Autorzy definiuję $\omega$-sygnaturę domknięcia warkocza, a~że sklejenie dwóch 4-rozmaitości z narożnikami (corners) nie odpowiada dodaniu ich sygnatur, to ich funkcja nie jest homomorfizmem. Wspomniany jest wzór Novikowa-Walla, który wyraża różnicę pewnych defektów jako indeks Masłowa i (to jest główne twierdzenie) różnica ta pokrywa się z kocyklem Meyera reprezentacji Burau-Squiera, cokolwiek to znaczy. Pojawia się również jakaś funkcja Rademachera.} Gambaudo, Ghysa \cite{gambaudo05} głosi, że przestrzeń wszystkich węzłów wyposażona w taką metrykę zawiera prawie idealną kopię przestrzeni euklidesowej dowolnego wymiaru.
\index[persons]{Gambaudo, Jean-Marc}%
\index[persons]{Ghys, Étienne}%
Dokładniej:

\begin{proposition}
    Dla każdej liczby całkowitej $n \ge 1$ istnieje funkcja $\xi: \Z^n \to \mathcal{K}$, dodatnie stałe $A, B, C$ oraz norma $\|\cdot\|$ na przestrzeni $\R^n$ takie, że spełniona jest podwójna nierówność
    \begin{equation}
        A\|x-y\|  - B \le d(\xi(x), \xi(y)) \le C\|x-y\|.
    \end{equation}
\end{proposition}

\begin{proof}
    Dowód korzysta z grup warkoczowych, które poznamy w sekcji \ref{sec:braid}.
\end{proof}



\index{liczba gordyjska|)}%

% Koniec podsekcji Liczba gordyjska



\input{10-introduction/105c-bridge}


% DICTIONARY;linking number;indeks zaczepienia;-
\subsection{Indeks zaczepienia}
\index{indeks zaczepienia|(}
Gauß wprowadził indeks zaczepienia dwóch węzłów jako pewna całka, ale żyjemy w~XXI wieku i wystarczy nam definicja odwołująca się do diagramów.

% DICTIONARY;sign;znak;skrzyżowanie
\begin{definition}[znak]
\index{znak skrzyżowania}%
    Liczbę $\pm 1$ przypisaną do skrzyżowania zgodnie z diagramem:
\begin{comment}
    \[
        \sign \left( \MediumPlusCrossingArrows \right) = +1 \quad
        \sign \left( \MediumMinusCrossingArrows \right) = -1
    \]
\end{comment}
    nazywamy znakiem skrzyżowania.
\end{definition}

Skrzyżowania dodatnie to takie, w których obrócenie dolnego łuku w prawo daje górny łuk, dlatego czasem nazywa się je także praworęcznymi.
Oczywiście skrzyżowania ujemne nazywamy wtedy leworęcznymi.
\index{skrzyżowanie!dodatnie i ujemne}%
\index{skrzyżowanie!lewo- i prawoskrętne}%

% DICTIONARY;smoothing;wygładzenie;-
\begin{definition}[wygładzenie]
\index{wygładzenie (skrzyżowania)}%
    Niech dany będzie diagram z wyróżnionym skrzyżowaniem.
    Wtedy diagramy
\begin{comment}
    \begin{figure}[H]
        \begin{minipage}[b]{.48\linewidth}
            \[
                \LargeAlphaSmoothing
            \]
            \subcaption{wygładzenie dodatnie}
        \end{minipage}
        \begin{minipage}[b]{.48\linewidth}
            \[
                \LargeBetaSmoothing
            \]
            \subcaption{wygładzenie ujemne}
        \end{minipage}
    \end{figure}
\end{comment}
    powstałe przez zmianę małego otoczenia tego skrzyżowania nazywamy wygładzeniami.
    Jeżeli nie zaznaczono inaczej, wygładzamy zgodnie ze znakiem skrzyżowania.
\end{definition}

\begin{definition}[indeks zaczepienia]
    Niech $L = K_1 \sqcup K_2$ będzie splotem o dwóch ogniwach.
    Wielkość
    \begin{equation}
        \linking(K_1, K_2) = \frac 12 \sum_i \sign c_i,
    \end{equation}
    gdzie sumowanie rozciąga się na wszystkie skrzyżowania, na których spotykają się łuki z różnych ogniw, nazywamy indeksem zaczepienia węzłów $K_1, K_2$.
    Ogólniej, jeśli dany jest splot $L = K_1 \sqcup \ldots \sqcup K_n$ posiadający $n$ ogniw, to jego indeks zaczepienia wyznacza wzór
    \begin{equation}
        \linking(L) = \sum_{i < j} \linking(K_i, K_j).
    \end{equation}
\end{definition}

Zauważmy, że indeks zaczepienia splotu Hopfa wynosi $1$, natomiast splotu Whiteheada $0$.
\index{splot Hopfa}%
\index{splot Whiteheada}%
Są zatem istotnie różne.
W obydwu przypadkach indeks zaczepienia jest liczbą całkowitą.
Istotnie, na mocy twierdzenia Jordana $\linking$ jest funkcją o całkowitych wartościach.

\begin{proposition}
    Indeks zaczepienia jest dobrze określonym niezmiennikiem zorientowanych splotów.
\end{proposition}

\begin{proof}
    Sprawdźmy wpływ ruchów Reidemeistera na wartość $\linking L$:
\begin{comment}
    \begin{figure}[H]
    \centering
    %
    \begin{minipage}[b]{.3\linewidth}
        \[
            \MedLarReidemeisterOneLeft \cong \MedLarReidemeisterOneStraight
        \]
        \subcaption{ruch $R_1$}
    \end{minipage}
    %
    \begin{minipage}[b]{.3\linewidth}
        \[
            \MedLarReidemeisterTwoLinkingA \cong \MedLarReidemeisterTwoB
        \]
        \subcaption{ruch $R_2$}
    \end{minipage}
    %
    \begin{minipage}[b]{.35\linewidth}
        \[
            \MedLarReidemeisterThreeLinkingA \cong \MedLarReidemeisterThreeLinkingB
        \]
        \subcaption{ruch $R_3$}
    \end{minipage}
\end{figure}
\end{comment}
    Na mocy twierdzenia Reidemeistera dowód został zakończony.
\end{proof}

\index{indeks zaczepienia|)}

% koniec podsekcji Indeks zaczepienia



\subsection{Spin}
\index{spin|(}

% DICTIONARY;writhe;spin;-
\begin{definition}[spin]
    Niech $D$ będzie diagramem zorientowanego splotu.
    Wielkość
    \begin{equation}
        \writhe D = \sum_c \operatorname{sign} c,
    \end{equation}
    gdzie sumowanie przebiega po wszystkich skrzyżowaniach diagramu $D$, nazywamy spinem.
\end{definition}

Co ważne, spin nie jest niezmiennikiem splotów ani węzłów.
Para Perko przedstawia ten sam węzeł z~minimalną liczbą skrzyżowań i~spinem równym siedem lub dziewięć.
\index{para Perko}%
Dzięki temu przez wiele lat nie została dostrzeżona.
Spin jest za to niezmiennikiem węzłów alternujących, mówi o~tym druga hipoteza Taita.
\index{hipoteza Taita}%

\begin{lemma}
    \label{lem:writhe_reidemeister}
    Spin nie zależy od orientacji.
    Tylko I ruch Reidemeistera zmienia spin:
\begin{comment}
    \begin{equation}
        \writhe \left(\MediumReidemeisterOneLeft\right) =
        \writhe \left(\MediumReidemeisterOneStraight\right) - 1.
    \end{equation}
\end{comment}
    Pozostałe ruchy nie mają na niego wpływu.
\end{lemma}

\index{spin|)}

% Koniec sekcji Spin



\subsection{Liczba patykowa}
\index{liczba patykowa|(}

% DICTIONARY;stick number;liczba patykowa;-

\begin{definition}
    Minimalną liczbę odcinków w~łamanej, która przedstawia węzeł $K$, nazywamy jego liczbą patykową i~oznaczamy $\stick(K)$.
\end{definition}

Wielkość tę wprowadził do matematyki Randell w~\cite{randell98} i~znalazł dokładną jej wartość dla niewęzła (3), trójlistnika (6) oraz ósemki (7).
\index[persons]{Randell, Richard}%
Negami trzy lata później w~\cite{negami91} pokazał przy użyciu teorii grafów, że dla nietrywialnych węzłów prawdziwe są nierówności
\index[persons]{Negami, Seiya}%
\begin{equation}
    \frac{5+\sqrt{9 + 8 \crossing K}}{2} \le \stick K \le 2 \crossing K.
\end{equation}

Trójlistnik to jedyny węzeł realizujący górne ograniczenie.
% Huh, Oh 2011 z trójlistnikiem?
Z~pracy Elrifaia \cite{elrifai06} (a wiemy o~niej z~\cite[s. 1]{huh11}) wynika, że dla węzłów o~co najwyżej 26 skrzyżowaniach, dolne ograniczenie jest ostre: można pisać $<$ w~miejsce $\le$.
\index[persons]{Elrifai, Elsayed}%

Jin oraz Kim w 1993 ograniczyli liczby patykowe dla węzłów torusowych korzystając z~liczby supermostowej.
\index[persons]{Jin, Gyo}%
\index[persons]{Kim, Hyoung-Seok}%
Wkrótce wynik został poprawiony przez samego Jina, w pracy \cite{jin97} znalazł dokładne wartości dla niektórych węzłów.
I~tak, jeśli $2 \le p < q < 2p$, to $\stick T_{p,q} = 2q$ oraz $\stick T_{p, 2p} = 4p-1$.
Ten sam wynik, choć dla węższego zakresu parametrów, odkryli Adams, Brennan, Greilsheimer, Woo \cite{greilsheimer97}.
\index[persons]{Adams, Colin}%
\index[persons]{Brennan, Bevin}%
\index[persons]{Greilsheimer, Deborah}%
\index[persons]{Woo, Alexander}%
\index{suma spójna}%
Autorzy niezależnie od siebie znaleźli proste oszacowanie z~góry dla liczby patykowej sumy spójnej:
\begin{equation}
    \stick(K_1 \shrap K_2) \le \stick(K_1) + \stick(K_2) - 3.
\end{equation}

Koniec dekady przyniósł jeszcze jedną pracę McCabe z~nierównością $\stick(K) \le 3 + \crossing (K)$ dla węzłów dwumostowych (\cite{mccabe98}) oraz odkrycie Calvo \cite{calvo01}: jeśli ograniczymy się do łamanych o co najwyżej siedmiu odcinkach, ósemka przestaje być odwracalna.
\index[persons]{McCabe, Cynthia}%
\index[persons]{Calvo, Jorge}%

Na początku XXI wieku nierówności Negamiego poprawiono, z dołu dokonał tego Calvo w~\cite{calvo01}, z góry natomiast Huh, Oh w \cite{huh11}.
\index[persons]{Calvo, Jorge}%
\index[persons]{Huh, Youngsik}%
\index[persons]{Oh, Seungsang}%
% huh11: simple and self-contained
Górne ograniczenie można zmniejszyć o $3/2$, jeżeli $K$ jest niealternującym węzłem pierwszym.
\begin{equation}
    \frac{7+\sqrt{1 + 8 \crossing K}}{2} \le \stick K \le \frac{3}{2} (1 + \crossing K).
\end{equation}

Liczba patykowa nie pojawia się już nigdzie w następnych rozdziałach.

\index{liczba patykowa|)}

% Koniec podsekcji Liczba patykowa




\subsection{Długość sznurowa}
\index{długość sznurowa|(}%
% DICTIONARY;ropelength;długość sznurowa;-
Matematyczne węzły nie mają grubości i można je dowolnie rozciągać.
Długość sznurowa, najsłabiej poznany niezmiennik numeryczny, pochodzi z~fizycznej teorii węzłów, która bierze pod uwagę obiekty wykonane z~nieelastycznych materiałów.

\begin{definition}
    Niech $L$ będzie splotem o długości $l$ oraz grubości $\tau$: posiada rurowe otoczenie bez samoprzecięć z~przekrojem poprzecznym o~promieniu $\tau$.
    Iloraz
    \begin{equation}
        \ropelength L = \frac l \tau
    \end{equation}
    nazywamy długością sznurową splotu.
\end{definition}

Przez wiele lat zastanawiano się: czy można zawiązać węzeł ze sznura o~długości jednej stopy i~promieniu jednego cala?
Lub równoważnie, czy $\ropelength K \le 12$ dla pewnego węzła $K$?
Na początku XXI wieku wiedzieliśmy z \cite{cantarella02}, że najkrótszy węzeł ma długość co najmniej $(2 + \sqrt 2)\pi \approx 10.726$, potem Diao udzielił negatywnej odpowiedzi na to pytanie w~\cite[s. 14]{diao03}.

Rozumowanie Denne, Diao, Sullivana \cite{denne06} oparte o~czterosieczne pokazuje, że długość sznurowa nietrywialnego węzła wynosi co najmniej $15.66$.
Ale eksperymenty komputerowe pokazują, że długość trójlistnika nie przekracza $16.372$, więc oszacowanie jest dość ostre.

Prowadzono obszerne poszukiwania na temat zależności między długością sznurową i~innymi niezmiennikami.
Mamy na przykład:

\begin{proposition}
    Istnieją stałe $c_1, c_2$ takie, że $c_1 \crossing^{3/4} K \le \ropelength K \le c_2 \crossing^{3/2} K$.
\end{proposition}

Udowodniono, że dolnym ograniczeniem na czynnik $c_1$ jest $(4\pi/11)^{3/4} \approx 1.105$ (gdyż tak pisze Cantarella i~inni w~\cite[tw. 23]{cantarella02}).
Wiemy też, że stała $c_1$ nie może przekraczać $12.64$ ze względu na węzeł torusowy $T(3, 5)$, przeczytaliśmy o~tym w~\cite{klotz21}.

Klotz, Maldonado \cite{klotz21} piszą, że Diao dostał lepsze dolne ograniczenie (lepsze niż ,,3/4'') dla węzłów do 1850 skrzyżowań (!!!):
\begin{equation}
    \frac 12 \left(17.334 + \sqrt{17.334^2 + 64 \pi \crossing K}\right) \le \ropelength K.
\end{equation}

Dowód górnego ograniczenia opiera się na cyklach Hamiltona w~grafach zanurzonych w~kratach liczbowych \cite{yu04} i zostało później poprawione:

% Ograniczenie to realizowane jest przez pewne węzły torusowe oraz sploty Hopfa. - wikipedia o c_1

\begin{proposition}
    $\ropelength K = O(\crossing K \cdot \log^5(\crossing K)).$
\end{proposition}

\begin{proof}
    Świeży wynik z \cite{diao19}, którego dowód wykorzystuje kraty liczbowe.
\end{proof}

Jakościowy wynik znaleźliśmy znowu w~\cite{klotz21}: zachodzi $\ropelength L \le a_u \crossing K \log^5 \crossing K$ (nie tylko dla węzłów, ale też splotów), przy czym stała $a_u$ musi być większa od $8\pi/\log^5 2 \approx 78.5$, by dobrze ograniczała splot Hopfa.

Długość sznurowa nie pojawia się w~dalszych rozdziałach.

\index{długość sznurowa|)}%

% Koniec podsekcji Długość sznurowa



\subsection{Podsumowanie}
% Livingston - Knot theory, 141 (zależności) i 144 (niezależności)
(Podana dalej lista może być niezrozumiała przy pierwszym czytaniu).
Między niektórymi niezmiennikami nie ma bezpośredniego związku:
\begin{enumerate}
    \item między liczbą gordyjską i mostową (fakt \ref{no_relation_bridge_unknotting})
    \item między liczbą gordyjską i wielomianem Alexandera (dowód faktu \ref{balanced_iff_four_conditions}),
    \item między liczbą mostową i genusem (wzmianka po fakcie \ref{no_relation_bridge_unknotting})
    \item między liczbą mostową i sygnaturą (wniosek \ref{no_relation_signature_bridge}).
    \item między defektami modulo różne liczby pierwsze (fakt \ref{no_relation_defects})
    \item między wielomianem Jonesa i Alexandera (paragraf przed faktem \ref{homfly_stronger})
\end{enumerate}

% Koniec sekcji Niezmienniki liczbowe



\chapter{Niezmienniki kolorowe}

Opisane w pierwszym rozdziale niezmienniki, takie jak liczba gordyjska czy liczba mostowa, pozwalają na odróżnienie od siebie niektórych węzłów, jednak wyznaczanie ich wartości nie jest łatwym zadaniem.
Dlatego nie potrafimy jeszcze uzasadnić, że istnieje jakikolwiek nietrywialny węzeł.
Zmieni się to teraz: poznamy kolorowania, niezmienniki splotów powstałe z diagramów, gdzie każde włókno występuje w jednym z trzech kolorów.

Następnie rozszerzymy paletę z trzech do skończenie wielu kolorów, by później zastąpić ją dowolną nieprzemienną grupą skończoną.
Nawet ten ostatni wariant kolorowania nie stanowi idealnego narzędzia do klasyfikacji węzłów.
Mówimy, że nie jest zupeły: istnieją różne węzły, którym przypisuje te same wartości, czyli ich nie odróżnia.
Problem ten będzie powtarzać się dla prawie wszystkich późniejszych niezmienników, z wyjątkiem dopiero całki Koncewicza (patrz sekcja \ref{sec:vassiliev}).

\section{Kolorowanie splotów}
\index{kolorowalność|(}

Przygodę z kolorowaniami rozpoczyna się zazwyczaj od trójkolorowalności.
Jest to pewna cecha diagramów, którą można posiadać albo nie.

\begin{definition}[trójkolorowalność]
\index{trójkolorowalność}%
    Niech $D$ będzie diagramem splotu $L$, którego łuki występują w~trzech kolorach.
    Jeżeli spełnione są następujące warunki:
    \begin{itemize}[leftmargin=*]
        \item nie wszystkie łuki są tego samego koloru,
        \item przy każdym skrzyżowaniu spotykają się albo trzy łuki w trzech różnych kolorach, albo wszystkie tego samego koloru,
    \end{itemize}
    to mówimy, że diagram $D$ jest trójkolorowalny.
\end{definition}

 Splot posiadający trójkolorowalny diagram nazywamy krótko trójkolorowalnym.

\begin{example}
    Trójlistnik jest trójkolorowalny, niewęzeł nie jest.
    Węzły te są zatem od siebie różne.
\end{example}

Dla wygody jako kolorów używać będziemy kolejnych liczb naturalnych $0, 1, \ldots, n-1$.
Pozwala to zapisać warunek kolorowalności równaniem algebraicznym, niezależnie od ilości użytych kolorów.

\begin{definition}[kolorowanie]
\index{równanie kolorujące}%
    \label{def:colouring_equation}
    Niech $L$ będzie splotem, zaś $n$ liczbą naturalną.
    Mówimy, że splot $L$ jest kolorowalny modulo $n$, jeśli posiada diagram, którego włóknom można przypisać liczby całkowite $0, \ldots, n - 1$ tak, by
    \begin{enumerate}
        \item istniały dwa włókna różnych kolorów,
        \item równanie $a + b \equiv 2c$ modulo $n$ było spełnione przy każdym skrzyżowaniu:
    \end{enumerate}
\begin{comment}
    \[
        \HugePlusCrossingColouring
    \]
\end{comment}
    Takie przyporządkowanie nazywamy kolorowaniem.
\end{definition}

Metoda ta została odkryta razem z~uogólnieniem do $n$ kolorów przez Ralpha Foxa w~1956, kiedy próbował uczynić teorię węzłów bardziej przystępną dla studentów.
Opierając się na definicji oraz ruchach Reidemeistera możemy wykazać pierwsze własności kolorowań.

Kolorowanie nazywamy trywialnym, jeśli używa tylko jednego koloru.

\begin{proposition}
    \label{prp:colouring_invariance}
    Własność ,,być $n$-kolorowalnym'' jest niezmiennikiem węzłów.
\end{proposition}

\begin{proof}
    Wystarczy sprawdzić, jak ruchy Reidemeistera zmieniają kolory.
    Pierwszy i~drugi:
\begin{comment}
    \begin{figure}[H]
    \centering
    %
    \begin{minipage}[b]{.45\linewidth}
        \[
            \LargeReidemeisterOneLeftProof \stackrel{R_1}{\cong} \LargeReidemeisterOneStraightProof
        \]
    \end{minipage}
    %
    \begin{minipage}[b]{.45\linewidth}
        \[
            \LargeReidemeisterTwoColouringA \stackrel{R_2}{\cong} \LargeReidemeisterTwoB
        \]
    \end{minipage}
    \end{figure}
\end{comment}
    Trzeci ruch także nie wymaga skomplikowanych rachunków.
    Najkrótszy łuk na diagramach ma kolor $2a-c$ po lewej oraz $2b-c$ po prawej stronie.
\begin{comment}
    \[
        \LargeReidemeisterThreeColouringA \cong \LargeReidemeisterThreeColouringB
        \qedhere
    \]
\end{comment}
\end{proof}

Trójlistnik koloruje się dokładnie modulo krotności trójki, ósemka zaś -- piątki.
Sama kolorowalność nie mówi wiele, splot jest kolorowalny lub nie.
Dowód faktu \ref{prp:colouring_invariance} pokazuje coś więcej: liczba kolorowań, być może trywialnych, jest mocniejszym niezmiennikiem.

\begin{lemma}
    \label{lem:colouring_arc}
    Ustalmy diagram $D$ dla węzła z~wybranym łukiem, oraz kolor $k \in \{0, \ldots, n - 1\}$.
    Bez straty ogólności możemy założyć, że krótki łuk jest koloru $k$.
\end{lemma}

Kolorem tym zazwyczaj jest $0$.

\begin{proof}
    Dodanie tej samej wartości do wszystkich łuków na dobrze pokolorowanym diagramie daje nowy, także dobrze pokolorowany diagram.
\end{proof}

\begin{proposition}
    \label{prp:no_colourings_mod_2}
    Żaden węzeł nie koloruje się modulo dwa.
\end{proposition}

\begin{proof}
    Załóżmy nie wprost, że istnieje nietrywialne kolorowanie.
    Analiza czterech możliwych skrzyżowań pokazuje, że włókna wychodzące z~tunelu muszą mieć ten sam kolor.
    Przechodząc wzdłuż węzła widzimy jeden kolor, wbrew założeniu nie wprost.
\end{proof}

\begin{proposition}
    Każdy splot o co najmniej dwóch ogniwach koloruje się modula dwa.
\end{proposition}

\begin{proof}
    Wystarczy pomalować jedną składową zerem, a~pozostałe jedynkami.
\end{proof}

Sploty rozszczepialne są $n$-kolorowalne dla każdego $n \ge 2$, można skorzystać z~tego samego schematu kolorowania.
\index{splot!rozszczepialny}%
Pierścienie Boromeuszy nie kolorują się modulo trzy, nie są zatem rozszczepialne.
% TODO: pierścienie Boromeuszy nie są nigdzie zdefiniowane
\index{pierścienie Boromeuszy}%
Sploty, które nie są kolorowalne modulo $n$ dla żadnej liczby $n \in \N$ nazywa się czasem niewidzialnymi, dwa węzły do dziesięciu skrzyżowań mają tę własność: $10_{124}$ oraz $10_{153}$.
\index{węzeł!niewidzialny}%

Pokażemy teraz, że suma równań kolorujących z dobrze wybranymi znakami jest postaci $0 \equiv 0 \mod n$.
Jest to składnik w dowodzie na to, że wyznacznik determinuje kolorowalność splotu.
Będziemy potrzebować pomocniczej definicji.

\begin{definition}[uszachowienie]
\index{uszachowienie}%
    Diagram rozcina płaszczyznę na obszary.
    Przyporządkowanie im jednego z~dwóch kolorów tak, by sąsiadujące ze sobą obszary były zawsze różnych kolorów, nazywamy uszachowieniem diagramu.
\end{definition}

Ustalmy węzeł $K$ oraz dowolne uszachowienie dla jego diagramu.
Skojarzmy z~każdym skrzyżowaniem równanie kolorujące, zgodnie z~poniższym schematem:
\begin{comment}
\begin{figure}[H]
    \begin{minipage}[b]{.48\linewidth}
    \[
        \HugeCrossingChessboardA
    \]
    \subcaption{$+a-b+a-c=0 \mod n$}
    \end{minipage}
    \begin{minipage}[b]{.48\linewidth}
    \[
        \HugeCrossingChessboardB
    \]
    \subcaption{$-a+b-a+c=0 \mod n$}
    \end{minipage}
\end{figure}
\end{comment}

\begin{proposition}
    \label{prp:colouring_sum_zero}
    Sumą równań kolorujących o dobrze wybranych znakach jest $0 \equiv 0 \mod n$.
\end{proposition}

\begin{proof}
    Każde równanie kolorujące składa się z~czterech wyrazów, po jednym od każdej nici, która spotyka się w~danym skrzyżowaniu.
    Nić biegnie między dwoma skrzyżowaniami, więc suma wszystkich równań kolorujących składa się z~par składników, po jednej parze na nić.
    Składniki te są przeciwnych znaków, zatem wzajemnie się znoszą.
    Suma równań kolorujących jest sumą zer, a~to należało udowodnić.
\end{proof}

Liczbę kolorowań splotu $L$ modulo $n$, trywialnych lub nie, oznaczamy przez $\tau_n(L)$.

\begin{proposition}
    Jeśli $K, L$ są węzłami, to $3\tau_3(K \shrap L) = \tau_3(K)\tau_3(L)$.
\end{proposition}

\begin{corollary}
    Istnieje nieskończenie wiele węzłów.
\end{corollary}

\begin{proof}
    Suma spójna $n$ trójlistników ma $3^{n+1}$, trywialnych lub nie, $3$-kolorowań.
\end{proof}

Dotychczas kolorowaliśmy diagramy węzłów liczbami $0, 1, 2, \ldots, n-1$, czyli elementami grupy $\Z/n\Z$, ale nic nie stoi na przeszkodzie, żeby próbować użyć dowolnej innej skończonej grupy.

\index{etykietowanie|(}

\begin{definition}[etykietowanie]
    Mówimy, że zorientowany węzeł $K$ jest etykietowalny grupą $G$ generowaną przez elementy $g_1, \ldots, g_n$, jeśli posiada diagram, którego włóknom przypisano elementy $g_1, \ldots, g_n$ tak, by równanie $gk=hg$ było spełnione przy każdym skrzyżowaniu ($g$: włókno biegnące górą, $k$: bo jego lewej stronie, $h$: po prawej).
\begin{comment}
    \[
        \LargePlusCrossingLabel
    \]
\end{comment}
\end{definition}

Równanie $gkg^{-1}=h$ mówi, że etykiety włókien wchodzących oraz wychodzących są sprzężone.
Wynika stąd, że wszystkie etykiety pochodzą z~jednej klasy sprzężoności.
Muszą jednocześnie generować całą grupę, dlatego $G$ musi być grupą nieprzemienną lub trywialną.
Etykietowalność jest niezmiennikiem węzłów i~nie zależy od orientacji węzła:
jeżeli elementy $g_1, \ldots, g_n$ generują grupę, to ich odwrotności także.

Rozpatrzmy węzły $6_1$ oraz $9_{46}$ i~spróbujmy etykietować je transpozycjami z~grupy $S_4$.
Wybranie dwóch etykiet przy jednym skrzyżowaniu $6_1$ wymusza etykiety dla wszystkich włókien.
Dwie transpozycje nie mogą generować grupy $S_4$, natomiast włókna węzła $9_{46}$ dają się etykietować samymi transpozycjami.
Węzły te są więc różne, choć mają te same własności kolorujące.

Etykietowanie jest mocnym narzędziem odróżniającym węzły.
Thistlethwaite w 1985 roku korzystając z niego klasyfikował węzły o~co najwyżej 13 skrzyżowaniach (jest ich, jak ostatecznie się okazało, 12965).
Mają one tylko 5639 różnych wielomianów Alexandera, ale etykietowania trzynastoma różnymi grupami pozwoliły zmniejszyć liczbę nierozpoznanych węzłów do około tysiąca.
Wśród nich 30 posiada wielomian Conwaya $1 + 2z^2 + 2z^4$, ale pary rozróżniane wielomianem HOMFLY mają też różne wielomiany Jonesa.
Wielomiany opisujemy w~rozdziale trzecim.

Niech $p \ge 3$ będzie liczbą pierwszą, natomiast $D_p = \langle r, s \mid r^p = s^2 = e, rsr = s \rangle$ grupą diedralną rzędu $2p$.
Elementy tej grupy to $1, r, r^2, \ldots, r^{p-1}, s, sr, \ldots, sr^{p-1}$.
,,Obrót'' $r^k$ jest sprzężony tylko ze swoją odwrotnością, ale ,,symetrie osiowe'' $sr^k$ tworzą jedną klasę sprzężoności.
Łatwo widać, że dowolne dwie z~nich generują całą grupę $D_p$.

\begin{proposition}
    Węzeł $K$ jest $p$-kolorowalny wtedy i~tylko wtedy, gdy jest $D_p$-etykietowalny.
\end{proposition}

\begin{proof}
    Załóżmy, że $K$ ma $n$ włókien.
    Wiemy już, że każde $D_p$-etykietowanie wykorzystuje tylko elementy $sr^{a_1}, \ldots, sr^{a_n}$ dla $1 \le a_i \le p$.
    Jest ono prawidłowe dokładnie wtedy, gdy analogiczne kolorowanie liczbami $a_1, \ldots, a_n$ jest prawidłowe.
\end{proof}

Kolorowania definiowano kiedyś jako surjekcje $\rho \colon \pi \to D_{2n}$ z~grupy podstawowej.
Jak mówi prezentacja Wirtingera, grupa splotu generowana jest przez ścieżki z~punktu bazowego w~$S^3$ do brzegu rurowego otoczenia splotu, wokół południka i~znowu do bazowego punktu.
\index{prezentacja Wirtingera}%
Fox zauważył, że z~surjektywności $\rho$ wynika, iż generatory mapują się na symetrie osiowe $sr^k$.
Ponieważ istnieje wzajemnie jednoznaczna odpowiedniość między generatorami grupy splotu oraz łukami diagramu, każdemu możemy przypisać liczbę całkowitą $k$.
Etykietowania są więc uogólnieniem kolorowań.
Rozumowanie, które przedstawiliśmy, prowadzi do prostej klasyfikacji grup, których można użyć do etykietowania.

\begin{proposition}
    Niech $K$ będzie węzłem, $\pi$ grupą podstawową jego dopełnienia, zaś $G$ dowolną grupą.
    Następujące warunki są równoważne: $K$ jest $G$-etykietowalny; istnieje surjekcja $\pi_1 \to G$.
\end{proposition}

Historycznie, prezentacja Wirtingera była pierwsza, zaś etykietowania odkryto później.

\begin{proposition}[Perko]
    Niech $K$ będzie węzłem.
    Jeżeli jest etykietowalny grupą $S_3$, to jest etykietowalny także grupą $S_4$.
\end{proposition}

Nie znam innych nietrywialnych faktów dotyczących etykietowań.

\index{etykietowanie|)}

\index{kolorowalność|)}

% Koniec sekcji Kolorowanie splotów



\section{Macierz kolorująca i~wyznacznik}
\index{macierz!kolorująca|(}%
Zajmiemy się teraz wyznacznikiem, pierwszym nieoczywistym niezmiennikiem splotów, który przypisuje każdemu pewną liczbę całkowitą.
Jest on blisko związany z~kolorowaniem.
Zauważmy, że pierwszy ruch Reidemeistera usuwa zamknięte krzywe, czyli pojedyncze łuki bez skrzyżowań.
Diagram bez takich krzywych ma tyle samo skrzyżowań, co łuków.

\begin{definition}[macierz kolorująca]
    Ustalmy diagram bez zamkniętych krzywych dla splotu $L$ z~łukami $x_0, \ldots, x_m$ oraz skrzyżowaniami $0, \ldots, m$.
    Definiujemy macierz $A_+$, której wyraz $a_{lj}$ jest współczynnikiem przy $x_j$ w~$l$-tym równaniu kolorującym:
\begin{comment}
    \[
        \LargePlusCrossingMatrix
    \]
\end{comment}
    Macierz kolorująca $A$ powstaje z~macierzy $A_+$ przez skreślenie dowolnego wiersza i~kolumny.
\end{definition}

Taka macierz jest kwadratowa, ponieważ z~każdego skrzyżowania wychodzą (tunelem) dwa włókna mające dwa końce.
Wykreślenie wiersza i~kolumny jest konieczne.
Gdybyśmy tego zaniechali, otrzymana macierz nie byłaby odwracalna, bowiem wiersze sumują się do zera (patrz fakt \ref{prp:colouring_sum_zero}).
Dla alternujących diagramów możemy żądać, by górą $i$-tego skrzyżowania biegło $i$-te włókno, wtedy na diagonali macierzy $A$ znajdą się same minus dwójki.

\index{wyznacznik|(}
\begin{definition}[wyznacznik]
\label{def:determinant}%
    Wyznacznikiem splotu nazywamy wyznacznik macierzy kolorującej $A$ bez znaku: $\det K := |\det A_K|$.
    Za wyznacznik niewęzła przyjmujemy liczbę $1$.
\end{definition}

\begin{definition}[defekt]
\index{defekt}%
    Wymiar jądra macierzy kolorującej modulo $p$, defekt, nazywamy defektem węzła.
\end{definition}

\begin{proposition}
\label{no_relation_defects}%
    Defekty modulo różne liczby pierwsze są niezależne od siebie.
\end{proposition}

\begin{proof}
    Na przykład suma spójna $k$ trójlistników i~$j$ węzłów $T_{2,5}$ posiada defekt $k$ modulo $3$ oraz $j$ modulo $5$.
    Podobne przykłady istnieją dla innych zbiorów liczb pierwszych.
\end{proof}

Pokażemy później (po poznaniu grupy kolorującej, czyli we wniosku \ref{cor:determinant_invariant}) lub jeszcze później (po wprowadzeniu wielomianu Alexandera, w~dowodzie faktu \ref{alexander_invariance}), że wyznacznik splotu jest dobrze określony: nie zależy on od wyboru etykietowania, minora macierzy oraz diagramu i~że jest niezmiennikiem.
Teraz ograniczymy się do jego kilku własności.

Defekt także jest niezmiennikiem, choć rzadziej używanym.
Węzeł o~defekcie $n$ modulo $p$ posiada $p(p^n-1)$ kolorowań $p$ kolorami.
Węzły $8_{18}$ oraz $9_{24}$ mają ten sam wyznacznik, $45$.
Ich defekty modulo $3$ to $1$ i~$2$, zatem są różne.

% Każde $p$-kolorowanie węzła $n$-mostowego jest wyznaczone przez kolory $n$ mostów
% % To brzmi podejrzanie: więc przestrzeń kolorowań ma wymiar co najwyżej $n$ i~defekt modulo $p$ nie przekracza $\operatorname{br} - 1$.
% Wnioskujemy stąd, że $(3,3,3)$-precel nie jest dwumostowy.

% \begin{proof}
%     \emph{Krok pierwszy}.
%     Pokażemy, że żaden ruch Reidemeistera nie zmienia wyznacznika.
%     \begin{enumerate}
%         \item \emph{Ruch $R_1$}. Diagram przed lub po ruchu zawiera co najmniej jedno włókno, które łączy tunel z~mostem pewnego skrzyżowania.
%         \item \emph{Ruch $R_2$}.
%         \item \emph{Ruch $R_3$}.
%     \end{enumerate}

%     \emph{Krok drugi}.
%     Niech $A_{i,j}$ oznacza minor powstały przez skreślenie $i$-tego wiersza oraz $j$-tej kolumny.
%     Pokażemy, że wartość wyznacznika nie zależy od wyboru $i$ oraz $j$.

%     Niech $X$ będzie macierzą $k \times k$ złożoną z~samych jedynek.
%     Suma elementów w~każdej kolumnie oraz każdym wierszu macierzy $A + X$ wynosi $k$, ponieważ znaki równań zostały dobrze wybrane.
%     Wykonujemy kolejno operacje:
%     \begin{enumerate}
%         \item Dodajemy do $i$-tego wiersza sumę pozostałych.
%         \item Dodajemy do $j$-tej kolumny sumę pozostałych.
%         Teraz $i$-ty wiersz oraz $j$-ta kolumna zawierają wyrazy $k$ z~wyjątkiem $a_{ij}$, który wynosi $k^2$.
%         \item Wyciągamy $k$ z~$i$-tego wiersza przed wyznacznik.
%         \item Odejmujemy $i$-ty wiersz od pozostałych.
%     \end{enumerate}
%     Rozwinięcie Laplace'a względem $j$-tej kolumny mówi, że $|\det (A+X)| = k^2 |(-1)^{i+j} \det A_{i,j}|$, co kończy dowód drugiego kroku.

%     \emph{Krok trzeci}.
%     Pokażemy, że zmiana etykietowania nie zmienia wyznacznika.
% \end{proof}

\begin{proposition}
\label{prp:bankwitz}%
\index{indeks skrzyżowaniowy}
    Niech $L$ będzie splotem alternującym.
    Wtedy $\det L \ge \crossing L$.
\end{proposition}

Pierwszy niepoprawny dowód pochodzi od Bankwitza \cite{bankwitz30}, który w 1930 roku ograniczył się do węzłów.
\index{człowiek!Bankwitz, ?}%
Poprawny dowód pojawił się dopiero trzy dekady później, kiedy Crowell \cite{crowell59} oparł się o~teoriografową pracę Whittneya.
\index{człowiek!Crowell, Richard}%
\index{człowiek!Whittney, ?}%
Spośród 84 diagramów pierwszych węzłów na końcu podręcznika Reidemeistera, 11 ma niealternujący diagram.
Crowell pokazał, że 7 z~tych 11 przedstawia niealternujące węzły.

\begin{example}
    $\det 8_{19} = 3 < 8 = \crossing 8_{19}$, więc węzeł $8_{19}$ jest niealternujący.
    Podobnie dowodzi się, że $9_{42}$, $10_{124}$, $10_{132}$, $10_{139}$, $10_{140}$, $10_{145}$, $10_{153}$, $10_{161}$, 21 węzłów pierwszych o~11 skrzyżowaniach oraz 75 węzłów pierwszych o~12 skrzyżowaniach nie są alternujące.
\end{example}

Niealternujących węzłów o 11 (odpowiednio: 12) skrzyżowaniach jest 185 (888), zatem ta prosta nierówność nie daje niesamowitch efektów.
Ale Crowell znalazł też mocniejsze ograniczenie:

\begin{proposition}
    Niech $L$ będzie pierwszym, alternującym splotem, który nie jest $(2, n)$-torusowy.
    Wtedy $\det L + 3 \ge 2 \crossing L$
\end{proposition}

Wyznacznik jest blisko związany z kolorowaniami.

\begin{lemma}
    Niech $A$ będzie macierzą $r \times r$ o całkowitych wyrazach.
    Istnieje niezerowy wektor $x \in (\Z/n\Z)^r$ taki, że $Ax \equiv 0 \mod n$ wtedy i tylko wtedy, gdy liczby $\det A$ oraz $n$ nie są względnie pierwsze.
\end{lemma}

\begin{proof}
    Z~algebry liniowej wiemy, że dla pewnych odwracalnych całkowitoliczbowych macierzy $C, R$ macierz $RAC = \operatorname{diag}(a_1, \ldots, a_m)$ jest diagonalna: to postać normalna Smitha.
    Istnieje odpowiedniość między niezerowymi rozwiązaniami równania $Ax \equiv 0$ oraz $Dy \equiv 0$, mamy bowiem $x \equiv Cy$, zatem bez straty ogólności możemy przyjąć, że macierz $A$ jest diagonalna.

    Istnieje niezerowy wektor $x$ taki, że $Ax \equiv 0 \mod n$ wtedy i tylko wtedy, jeśli istnieją $x_1, \ldots, x_m \in \Z/n\Z$, nie wszystkie zerowe, że dla każdego $i$ mamy $a_ix_i \equiv 0 \mod n$.
    Oznacza to, że dla pewnego $i$ liczby $a_i, n$ nie są względnie pierwsze.
    Macierz $A$ jest diagonalna, więc jej wyznacznik ma postać $\det A = \pm |a_1| \cdot \ldots \cdot |a_m|$.
    Wnioskujemy stąd, że liczby $\det A, n$ także nie są względnie pierwsze.
\end{proof}

\begin{proposition}
\label{prp:colour_determinant}%
    Splot $L$ koloruje się modulo $n$ wtedy i~tylko wtedy, gdy liczby $\det L$ oraz $n$ nie są względnie pierwsze.
\end{proposition}

\begin{proof}
    Wybierzmy diagram dla splotu $L$ z uporządkowanymi łukami i skrzyżowaniami.
    Bez straty ogólności ograniczmy się do tych kolorowań, gdzie $x_0 = 0$.
    Kolorowanie modulo $n$ istnieje dokładnie wtedy, gdy istnieje niezerowy wektor $(x_1, x_2, \ldots, x_m)$ taki, że $Ax \equiv 0 \mod n$.
    Na mocy lematu oraz definicji $\det L = |\det A|$, dowód zostaje zakończony.
\end{proof}

\begin{corollary}
    Jeżeli wyznacznik splotu $L$ wynosi $\det L = 1$, to splot $L$ jest niewidzialny.
\index{węzeł!niewidzialny}%
\end{corollary}

\begin{corollary}
    Jeżeli wyznacznik splotu $L$ wynosi $\det L = 0$, to splot $L$ koloruje się modulo wszystkie liczby naturalne.
\end{corollary}

\begin{corollary}
\index{splot!rozszczepialny}%
    Jeśli splot $L$ jest rozszczepialny, to jego wyznacznik wynosi $0$.
\end{corollary}

Poniższy problem pochodzi od Stojmenowa.

\begin{conjecture}[problem 12.25 w \cite{ohtsuki02}]
    Niech $n$ będzie nieparzystą sumą dwóch kwadratów.
    Czy istnieje pierwszy, alternujący, achiralny węzeł o~wyznaczniku $n$?
\end{conjecture}

Oto, co już wiemy.
Dla $n = 1, 9, 49$ oraz być może pewnej liczby $n > 2000$ niebędącej kwadratem, taki węzeł nie istnieje.
Jeśli istnieje achiralny węzeł o wyznaczniku $n$, to $n$ jest nieparzystą sumą dwóch kwadratów (\cite{hartley79}).
\index{człowiek!Hartley, Richard}%
Implikacja odwrotna także jest prawdziwa, od węzła można dodatkowo żądać bycia pierwszym lub alternującym, ale nie zawsze obydwu warunków jednocześnie.
Patrz też \cite{stoimenow05}.
\index{człowiek!Stoimenow, Alexander}%

\index{wyznacznik|)}

\index{macierz!Goeritza|(}
Istnieje jeszcze jedna kombinatoryczna metoda badania węzłów, która prowadzi między innymi do pojęcia wyznacznika.
Tuż przed wojną Lebrecht Goeritz pokazał, jak diagram węzła wyznacza specjalną formę kwadratową.
\index{człowiek!Goeritz, Lebrecht}%
Nieco później Trotter zmodyfikował jego pomysł, by sygnatura formy stanowiła niezmiennik splotów.
\index{człowiek!Trotter, Hale}%
Gordon, Litherland ujednolicili dwa wyżej wymienione podejścia w~pracy \cite{litherland81}.
\index{człowiek!Litherland, ?}%
\index{człowiek!Gordon, Cameron}%
My opiszemy krótko macierz Goeritza.

Ustalmy uszachowiony diagram $D$ dla splotu $L$.
\index{uszachowienie}%
Oznaczmy białe obszary $0, 1, \ldots, m$, przy czym $0$ jest obszarem nieograniczonym.
Przydzielmy skrzyżowaniom znaki:
\begin{comment}
\begin{figure}[H]
    \begin{minipage}[b]{.48\linewidth}
    \[
        \LargePlusCrossingChessboard
    \]
    \end{minipage}
    \begin{minipage}[b]{.48\linewidth}
    \[
        \LargeMinusCrossingChessboard
    \]
    \end{minipage}
\end{figure}
\end{comment}

\begin{definition}
    Macierz Goeritza powstaje przez skreślenie z~macierzy $G_+$ jednego wiersza oraz jednej kolumny:
    \begin{equation}
        G_+=\begin{pmatrix}
        G_{00} & \cdots & G_{0m} \\
        \vdots & \ddots & \vdots \\
        G_{m0} & \cdots & G_{mm}
        \end{pmatrix},
    \end{equation}
    gdzie jeśli $i\neq j$, to $G_{ij}$ jest sumą znaków skrzyżowań przyległych do $i$ oraz $j$.
    Dla $i = j$, $G_{ii}$ jest minus sumą znaków skrzyżowań wokół $j$-tego obszaru.
\end{definition}
% M. Kneser, D. Puppe, Quadratische Formen und Verschlingungsinvarianten von Knoten, Math. Z., 58, 1953, 376-384.
% R. H. Kyle, Branched covering spaces and the quadratic forms of links, Ann. of Math., 59(2), 1954, 539-548.

Macierz $G_+$ posiada dwie własności pozwalające wykryć proste błędy rachunkowe: jest symetryczna, a~jej kolumny i~wiersze sumują się do zera.
Jest przy tym zazwyczaj mniejsza od macierzy kolorującej.

\begin{proposition}
    Niech $K$ będzie ustalonym węzłem, $G$ jego macierzą Goeritza, zaś $A$ macierzą kolorującą.
    Z~dokładnością do znaku, obie macierze mają ten sam wyznacznik: $\det G = \pm \det A$.
\end{proposition}

Nie możemy niestety podać dowodu tego faktu, wymaga bowiem znajomości topologii algebraicznej, której wolelibyśmy nie zakładać.
Macierz Goeritza nie jest niezmiennikiem splotów.
Mamy jednak:

\begin{proposition}
    Niech $D_1, D_2$ będą dwoma diagramami ustalonego splotu.
    Wtedy macierz Goeritza $G_2$ można otrzymać z macierzy $G_1$ w skończonej liczbie kroków:
    \begin{enumerate}%[leftmargin=*]
    %\itemsep0em
        \item zamiany macierzy $G$ na $PGP^{-1}$, gdzie $P$ i~$P^{-1}$ mają całkowite wyrazy
        \item dopisania lub skreślenia $\pm 1$ na końcu przekątnej (dla węzłów) albo $-1, 0, 1$ (dla splotów).
    \end{enumerate}
\end{proposition}

\begin{proof}
\index{człowiek!Goeritz, Lebrecht}%
\index{człowiek!Kneser, ?}%
\index{człowiek!Puppe, ?}%
\index{człowiek!Kyle, ?}%
    Patrz prace: Goeritza \cite{goeritz33} i późniejsze, Knesera, Puppego \cite{kneser53} oraz Kyle'a \cite{kyle54}.
    % znalazłem te dowody w https://arxiv.org/pdf/0909.1118.pdf
\end{proof}

\index{macierz!Goeritza|)}

\index{macierz!kolorująca|)}

% Koniec sekcji Macierz i~wyznacznik


\input{20-colours/202-group}

\section{Kwandle i wraki}
\index{kwandel|(}%

Sekcja ta powstała częściowo w~oparciu o~notatki autorstwa Bergera, Geriga\footnote{dostępne pod adresem \url{https://scholar.harvard.edu/files/gerig/files/knotnotes.pdf}} oraz Bergera, Flannery'ego i~Sumnichta\footnote{dostępne pod adresem \url{https://github.com/thyrgle/191_Final_Project/blob/master/paper.pdf}}; ale najlepiej zacząć od długiej na trzy strony odpowiedzi Nelsona na pytanie ,,czym jest kwandel?'' w \cite{nelson16}.
\index[persons]{Berger, Andrew}%
\index[persons]{Gerig, Chris}%
\index[persons]{Flannery, Brandon}%
\index[persons]{Sumnicht, Christopher}%


Aksjomaty grupy stanowią uogólnienie symetrii, ponieważ składanie symetrii jest łączne, identyczność jest symetrią, funkcja odwrotna do symetrii jest symetrią.
Kwandle to struktura algebraiczna podobna do grupy: ich aksjomaty odzwierciedlają ruchy Reidemeistera.
\index{ruch!Reidemeistera}%

% DICTIONARY;quandle;kwandel;-
\begin{definition}[kwandel]
\index{kwandel}%
    Zbiór $X$ wyposażony w dwuargumentowe działanie $\triangleright$, które dla wszystkich elementów $x, y, z \in X$ spełnia trzy warunki:
    \begin{enumerate}
        \item $x \triangleright x = x$,
        \item odwzorowanie $\beta_y \colon X \to X$ dane wzorem $\beta_y(x) = x \triangleright y$ jest odwracalne,
        \item $(x \triangleright y) \triangleright z = (x \triangleright z) \triangleright (y \triangleright z)$,
    \end{enumerate}
    nazywamy kwandlem.
\end{definition}

David Joyce zapytany o znaczenie słowa \emph{quandle} odpowiedział: \emph{,,I needed a usable word. Distributive algebra had too many syllables. Piffle was already taken. I tried trindle and quagle, but they didn’t seem right, so I went with quandle''}.
\index[persons]{Joyce, David}%

Kwandle można rozpatrywać jako samodzielne konstrukcje algebraiczne.
My pokażemy, że są naturalnym niezmiennikiem węzłów.

Niech $X$ będzie skończonym kwandlem, zaś $K$ węzłem.
Elementy $x \in X$ będą dla nas kolorami, którymi oznaczymy długie łuki na diagramie węzła $K$.
Gdy trzy kolory spotykają się przy jednym skrzyżowaniu, definiujemy funkcję $\triangleright \colon X \times X \to X$, jak na rysunku.
To znaczy: kiedy łuk o kolorze $x$ przechodzi pod łukiem koloru $y$, staje się łukiem w kolorze $x \triangleright y$.
\begin{comment}
\[
    \LargeMinusCrossingQuandle
\]
\end{comment}

Ta definicja pochodzi z~nieopublikowanej korespondencji między Johnem Conwayem i~Gavenem Wraithem, którzy w 1959 byli studentami I stopnia na uniwersytecie w Cambridge.
\index[persons]{Conway, John}%
\index[persons]{Wraith, Gaven}%
% index: uniwersytet w Cambridge? a są inne?
Ponownie odkryto ją w latach 80. XX wieku: Joyce w 1982 po raz pierwszy nazwał te obiekty kwandlami, Matwiejow w tym samym roku jako grupoidy rozdzielne, Brieskorn w 1986 jako zbiory automorficzne.
\index[persons]{Joyce, David}%
\index[persons]{Matwiejow, Siergiej}%
% https://ru.wikipedia.org/wiki/Матвеев,_Сергей_Владимирович
\index[persons]{Brieskorn, Egbert}%
\index{grupoid rozdzielny|see {kwandel}}%
\index{zbiór automorficzny|see {kwandel}}%

Drugi aksjomat nazywa się czasem odwracalnością z prawej strony: znając $x \triangleright y$ oraz $y$ możemy odtworzyć element $x$, jednak znając $x$ być może nie jesteśmy w stanie odtworzyć elementu $y$.
Jedyny element $x$ taki, że $x \triangleright y = z$ nazwijmy $y \triangleleft z$.
To pozwala podać trochę inną definicję kwandli, my nie będziemy jej nigdy używać:

\begin{definition}
    Zbiór $X$ z dwuargumentowymi działaniami $\triangleright, \triangleleft$ taki, że dla wszystkich $x, y, z \in X$ zachodzi:
    \begin{align}
    x \triangleleft x & = x \\
    x \triangleright x & = x \\
    (x \triangleleft y) \triangleright x & = y \\
    x \triangleleft (y \triangleright x) & = y \\
     (x \triangleright z) \triangleright (y \triangleright z) & = (x \triangleright y) \triangleright z \\
    (x \triangleleft y) \triangleleft (x \triangleleft z) & = x \triangleleft (y \triangleleft z)
    \end{align}
    nazywamy kwandlem.
\end{definition}

Teraz możemy przetłumaczyć ruchy Reidemeistera w aksjomaty kwandli.
\index{ruch!Reidemeistera}%

\begin{proposition}
\index{niezmiennik!zliczający}%
    Niech $X$ będzie skończonym kwandlem.
    Liczba etykietowań diagramu elementami kwandla $X$ jest niezmiennikiem węzłów, zwanym niezmiennikiem zliczającym.
\end{proposition}

\begin{proof}
    Musimy pokazać, że etykiety na diagramiem przed każdym ruchem Reidemeistera wyznaczają jednoznacznie układ etykiet po tym ruchu.
    Pierwszy i drugi ruch:
\begin{comment}
    \begin{figure}[H]
        \begin{minipage}[b]{.48\linewidth}
        \[
            \LargeReidemeisterOneRightQuandleProof
            \stackrel{R_1}{\cong}
            \LargeReidemeisterOneStraightQuandleProof
        \]
        \end{minipage}
        \begin{minipage}[b]{.48\linewidth}
        \[
            \LargeReidemeisterTwoQuandleA \cong \LargeReidemeisterTwoQuandleB
        \]
        \end{minipage}
    \end{figure}
\end{comment}
    Trzeci ruch:
\begin{comment}
    \[
        \LargeReidemeisterThreeQuandleA \cong \LargeReidemeisterThreeQuandleB \qedhere
    \]
\end{comment}
\end{proof}

Homomorfizmy definiujemy standardowo, przez analogię do grup:

\begin{definition}
    Niech $Q_1, Q_2$ będą kwandlami.
    Odwzorowanie $f \colon Q_1 \to Q_2$ spełniające warunek
    \begin{equation}
        \forall x, y \in Q_1 : f(x \triangleright y) = f(x) \triangleright f(y),
    \end{equation}
    nazywamy homomorfizmem.
\end{definition}

Wiele znanych struktur algebraicznych okazuje się być źródłem kwandli.

\begin{example}[kwandel cykliczny/diedralny]
\index{kwandel!cykliczny}%
\index{kwandel!diedralny}%
    Grupa abelowa z działaniem $x \triangleright y = 2y - x$.
\end{example}

\begin{example}[kwandle sprzężone]
\index{kwandel!sprzężony}%
    Grupa z działaniem $x \triangleright y = y^{-n} x y^n$ dla każdego $n \in \N$.
\end{example}

\begin{example}[kwandel Alexandera]
\index{kwandel!Alexandera}%
    Moduł nad pierścieniem $\Z[t, 1/t]$ wielomianów Laurenta z~działaniem $x \triangleright y = tx + (1-t) y$.
\end{example}

\begin{example}[kwandel symplektyczny]
\index{kwandel!symplektyczny}%
    Przestrzeń liniowa i antysymetryczna forma dwuliniowa $\langle \cdot | \cdot \rangle$ z działaniem $x \triangleright y = x + \langle x | y \rangle y$.
\end{example}

Joyce w swojej rozprawie doktorskiej przypisał każdemu węzłowi $K$ pewien szczególny kwandel $Q(K)$, kwandel podstawowy.
\index[persons]{Joyce, David}%
\index{kwandel!podstawowy}%
Definicja tego obiektu jest dość zawiła: łuki diagramu są generatorami, zaś skrzyżowania odpowiadają za relacje.
Joyce pokazał, że kwandel $Q(K)$ wyznacza węzeł $K$ jednoznacznie z dokładnością do orientacji.
Nie czyni to jednak nowego niezmiennika użytecznym, gdyż wyznaczenie go nawet w najprostszych przypadkach stanowi trudność.
Niebrzydowski, Przytycki \cite{niebrzydowski09} pokazali na przykład w 2009, że
\index[persons]{Niebrzydowski, Maciej}%
\index[persons]{Przytycki, Józef}%

\begin{example}
    Kwandel podstawowy trójlistnika  jest izomorficzny z~pewnym rzutowym pierwotnym podkwandlem\footnote{Cokolwiek to znaczy!} odwzorowań liniowych przestrzeni symplektycznej $\Z \oplus \Z$
\end{example}

Aksjomaty grupy można wzmacniać (grupy abelowe) lub osłabiać (monoidy).
Podobnie czyni się z aksjomatami kwandli.
Kwandle inwolutywne odpowiadają węzłom bez orientacji, wraki dobrze opisują węzły obramowane, i tak dalej.
\index{węzeł!niezorientowany}%
% TODO: w zorientowanym dać patrz też do niezorientowanego?
\index{węzeł!obramowany}%

\begin{definition}[kwandel inwolutywny]
\index{kwandel!inwolutywny}%
\index{kei|see {kwandel inwolutywny}}%
    Kwandel $Q$, w którym dla wszystkich $x, y \in Q$ zachodzi $x \triangleleft (x \triangleleft y) = y$, nazywamy inwolutywnym albo kei.
\end{definition}

Kwandle inwolutywne badał jako pierwszy Mituhisa Takasaki (1943).
\index[persons]{Takasaki, Mituhisa}%
Szukał niełącznej struktury, która dobrze opisywałaby odbicia w skończonej geometrii.

% DICTIONARY;shelf;półka;-
\begin{definition}[półka]
\index{półka}%
    Zbiór $X$ wyposażony w dwuargumentowe działanie $\triangleright$ taki, że dla wszystkich elementów $x, y, z \in X$ zachodzi $(x \triangleright y) \triangleright z = (x \triangleright z) \triangleright (y \triangleright z)$, nazywamy półką.
\end{definition}

\begin{example}
\index{warkocz}%
% TODO: warkocz czy grupa warkoczy?
    Niech\footnote{Patrz \url{http://nlab-pages.s3.us-east-2.amazonaws.com/nlab/show/shelf\#infinite_braid_group}.} $B_\infty$ oznacza grupę wszystkich warkoczy, zaś $\phi$ będzie jej endomorfizmem posyłającym generator $\sigma_k$ na $\sigma_{k+1}$.
    Zbiór $B_\infty$ z działaniem $a \triangleleft b = a\phi(b)\sigma_1 \phi{a} ^{-1}$ jest półką.
\end{example}

To nieprzetłumaczalna gra słów: dwie półki (\emph{shelves}), lewa i prawa, które dobrze do siebie pasują, dają stojak (\emph{rack}, czyli dla nas wrak).
Jak napisała Crans w pracy \cite[s. 86]{crans04}: \emph{,,Just as a rack is comprised of two shelves which fit together nicely, a quandle is made up of two spindles''}.
\index[persons]{Crans, Alissa}%

Półka stanowi uogólnienie dwóch obiektów -- wraków i wrzecion.

% DICTIONARY;wrack;wrak
\begin{definition}[wrak]
\index{wrak}%
    Zbiór $X$ z dwuargumentowym działaniem $\triangleright$ takim, że dla każdej trójki elementów $x, y, z \in X$ zachodzi:
    \begin{enumerate}
        \item odwzorowanie $\beta_y \colon X \to X$ dane wzorem $\beta_y(x) = x \triangleright y$ jest odwracalne,
        \item $(x \triangleright y) \triangleright z = (x \triangleright z) \triangleright (y \triangleright z)$
    \end{enumerate}
    nazywamy wrakiem.
\end{definition}

\begin{example}
    % https://www1.cmc.edu/pages/faculty/VNelson/quandles.html
    Zbiór $X = \{1, 2, \ldots, n\}$ i permutacja $\sigma \in S_n$ z działaniem $x \triangleright y = \sigma(x)$.
\end{example}

\begin{example}
    % https://www1.cmc.edu/pages/faculty/VNelson/quandles.html
    Moduł nad pierścieniem $\Z[t^{\pm 1}, s]/(s^2 - (1-t)s)$ z działaniem $x \triangleright y = tx+sy$.
\end{example}

\index{węzeł!obramowany}% framed?
Wraki są naturalnym niezmiennikiem węzłów obramowanych, bo dobrze współgrają z II, III oraz podwójnym I ruchem Reidemeistera, który to nie zmienia spinu diagramu:
\begin{comment}
\[
    \LargeReidemeisterOneLeftRightQuandleProof
    \cong
    \LargeReidemeisterOneStraightQuandleProofRotated
\]
\end{comment}

% DICTIONARY;spindle;wrzeciono
\begin{definition}[wrzeciono]
\index{wrzeciono}%
    Zbiór $X$ z dwuargumentowym działaniem $\triangleright$ taki, że dla wszystkich elementów $x, y, z \in X$ zachodzi:
    \begin{enumerate}
        \item $x \triangleright x = x$,
        \item $(x \triangleright y) \triangleright z = (x \triangleright z) \triangleright (y \triangleright z)$
    \end{enumerate}
    nazywamy wrzecionem.
\end{definition}

Zatem kwandle to wraki, które są też wrzecionami.
Muszę w~tym miejscu wtrącić uwagę językową.
Conway nazwał wraki wrakami (\emph{wracks}), by częściowo zażartować z~nazwiska jego kolegi Gavina Wraitha, a częściowo by zaznaczyć, że są one tym, co zostaje z~grupy, w~której zapomniano o~mnożeniu, ale nie sprzęganiu (w~języku angielskim co najmniej od XVI wieku funkcjonuje zwrot ,,wrack and ruin'' oznaczający zniszczenie).
\index[persons]{Conway, John}%
\index[persons]{Wraith, Gaven}%
Obecnie dominuje określenie \emph{racks}.

Jak wyglądały poszukiwania małych kwandli?
Dionísio, Lopes \cite{lopes03} znaleźli 10 kwandli Alexandera, które odróżniają 249 węzłów pierwszych do 10 skrzyżowań.
\index[persons]{Dionísio, Miguel}%
\index[persons]{Lopes, Pedro}%
Po upływie prawie dekady Vendramin \cite{vendramin12} wytropił wszystkie 431 kwandli spójnych rzędu 35 lub mniejszego.
\index[persons]{Vendramin, Leandro}%
Clark, Elhamdadi, Saito oraz Yeatman \cite{clark13} pokazali niedawno zbiór 26 kwandli, które razem odróżniają od siebie wszystkie 2977 zorientowanych węzłów pierwszych o~co najwyżej 12 skrzyżowaniach.
\index[persons]{Clark, William}%
\index[persons]{Elhamdadi, Mohamed}%
\index[persons]{Saito, Masahico}%
\index[persons]{Yeatman, Timothy}%
Największy z~nich jest rzędu 182.

\begin{definition}[kwandel spójny]
\index{kwandel!spójny}%
    Niech $Q$ będzie kwandlem.
    W grupie wszystkich automorfizmów $Q$ można wyróżnić grupę generowaną przez automorfizmy wewnętrzne $\beta_y(x) = x \triangleright y$.
    Jeżeli działanie tej podgrupy na $Q$ jest przechodnie, kwandel nazywamy spójnym.
\end{definition}

Dużo otwartych problemów dotyczących kwandli można znaleźć w \cite[s. 455-465]{ohtsuki02}.

\index{kwandel|)}

% Koniec sekcji Kwandle i wraki



\chapter{Niezmienniki wielomianowe}

Wszystkie poznane dotąd niezmienniki (poza długością sznurową) przyjmowały całkowite wartości.
Teraz poszerzymy skrzynkę z~narzędziami o~klasyczne wielomiany Alexandera, Jonesa, HOMFLY; ale też późniejsze: BLM/Ho, Kauffmana oraz niezmienniki skończonego typu.
Co ciekawe, wielomiany te wywodzą się z~różnych działów matematyki: wielomian $\alexander$ Alexandera z~homologii pewnej przestrzeni nakrywającej, $\jones$ Jonesa: z~algebr von Neumanna.
HOMFLY (albo raczej HOMFLY-PT) to ich naturalne uogólnienie.

Atrakcyjnym wprowadzeniem była\footnote{Nielegalnie udostępniona w internecie, obecnie nie do przeczytania.} przygotowana przez matematyków niemieckich (a~przez to dostępna tylko w~ich języku) praca \cite{gellert09}.
Pierwotnymi artykułami były \cite{alexander28}, \cite{jones85} oraz \cite{homfly85}, wszystkie należą do przełomowych w~kombinatorycznej teorii węzłów.


\section{Wielomian Alexandera}
\index{wielomian!Alexandera|(}%

Wielomian Alexandera to najstarszy niezmiennik tego typu, odkryty w 1923 roku \cite{alexander23}.
Jego pierwsza definicja była czysto topologiczna algebraicznie: niech $K$ będzie węzłem w~3-sferze, zaś $X$ nieskończonym nakryciem cyklicznym jego dopełnienia (otrzymanym przez rozcięcie dopełnienia wzdłuż powierzchni Seiferta).
Na przestrzeni $X$ oraz grupie homologii $H_1(X)$, działa automorfizm $t$, który czyni z~niej moduł nad pierścieniem $\Z[t, t^{-1}]$, i~to skończenie prezentowalny.
Jeśli posiada przedstawienie z~$r$ generatorami i~$s$ relacjami, gdzie $r \le s$, rozpatrzmy ideał generowany przez minory $r \times r$ macierzy prezentacji (jeśli nie, weźmy ideał zerowy).
Alexander pokazał, że ideał ten zawsze jest niezerowy i~główny.

Jak widać, nie jest to definicja, do pracy z którą bylibyśmy przygotowani.
Dlatego podamy prostszy opis, oparty o równania kolorujące.
Później sprawdzimy, jaki wpływ na wielomian mają poznane wcześniej operacje na węzłach oraz co łączy go ze zdefiniowanymi wcześniej i~później numerycznymi niezmiennikami.
Na koniec pokażemy, jak Conway odkrył na nowo wielomian Alexandera, jednocześnie przyspieszając jego liczenie.


\subsection{Równania kolorujące (definicja pierwsza)}

\begin{definition}
\index{równanie kolorujące}%
    Wielomianowe równanie kolorujące związane ze skrzyżowaniem
\begin{comment}
    \[
        \LargePlusCrossingLabel
    \]
\end{comment}
    splotu zorientowanego to $g + tk - tg - h = 0$.
    Tylko orientacja górnego łuku ma znaczenie.
\end{definition}

Istotnie, wystarczy podstawić tutaj $t = -1$, by otrzymać szczególną definicję \ref{def:colouring_equation}.

\begin{definition}[wielomian Alexandera]
\label{def:alexander_polynomial}%
    Ustalmy diagram $D$ zorientowanego splotu $L$, gdzie nie ma żadnej krzywej zamkniętej (niewęzła $\SmallUnknot$).
    Przypiszmy etykiety $x_0, \ldots, x_m$ do włókien oraz $0, \ldots, m$ do skrzyżowań diagramu $D$.
    Niech $p_{ij}$ będzie współczynnikiem przy włóknie $x_j$ w~wielomianowym równaniu kolorującym nad wierzchołkiem $i$.
    Z macierzy $P=(p_{ij})$ wykreślmy po jednej kolumnie i~wierszu.
    Wyznacznik zmniejszonej macierzy to wielomian Alexandera, oznaczamy go $\alexander_L(t)$.
\end{definition}

Nasz nowy niezmiennik nie jest zwykłym wielomianem, tylko wielomianem Laurenta jednej zmiennej, czyli elementem pierścienia $\Z[t, t^{-1}]$.

\begin{proposition}
    \label{alexander_invariance}
    Wielomian Alexandera z~dokładnością do mnożenia przez jedności:
    \begin{equation}
        f(t) \equiv g(t) \iff \exists m \in \Z: f(t) = \pm t^m g(t)
    \end{equation}
    jest niezmiennikiem zorientowanych splotów.
\end{proposition}

W dowodzie niezmienniczości wyznacznika węzła skorzystaliśmy z~relacji między nim a~grupą kolorującą.
Poprzednie wydania książki zawierały sugestię, że elementarny (czyli taki, który nie korzysta z~teorii modułów) dowód niezmienniczości wielomianu Alexandera nie istnieje.
Sugestia ta była błędna.

Podany dowód dotyczy tylko splotów o spójnym diagramie.
Jeżeli mamy diagram, który nie spełnia tego założenia, to fakt \ref{prp:alexander_unlinks} orzeka, że jego wielomian Alexandera znika.

\begin{proof}
    Ustalmy diagram o~$k$ skrzyżowaniach, który rozcina płaszczyznę na $k+2$ obszarów i~utwórzmy macierz o~wymiarach $k \times k$, której kolumny odpowiadają obszarom, wiersze zaś skrzyżowaniom -- pomijając przy tym dwa sąsiadujące ze sobą obszary -- o~wyrazach ze współczynników równań kolorujących.
    Jej wyznacznik jest wielomianem Alexandera.

    Sąsiadującym ze sobą obszarom przypiszmy kolejne liczby całkowite tak, by obszar leżący po prawej stronie włókna miał niższy indeks.
    Pokażemy najpierw, że skasowanie kolumny indeksu $n$ oraz $n+1$ sprawia, że wyznacznik zmienia się co najwyżej o~czynnik $\pm t^m$ dla pewnego $m$.
    Niech $S_n$ oznacza sumę kolumn indeksu $n$.
    Każdy wiersz macierzy zawiera cztery niezerowe wyrazy: $\pm 1, \pm t$, zatem $\sum_n S_n = 0$.
    Równość ta zachodzi nawet po przemnożeniu kolumny indeksu $n$ przez $t^{-n}$: $\sum_n t^{-n}S_n = 0$, co prowadzi do relacji $\sum_n (t^{-n}-1) S_n = 0$.
    Jeśli więc indeks kolumny $v_j$ wynosi $n$, to $(t^{-n}-1)v_j$ jest kombinacją liniową innych kolumn niezerowego indeksu (ponieważ $t^0 - 1 = 0$).

    Rozpatrzmy macierze $M_{0,j}, M_{0,k}$, gdzie indeksy $j$-tej i~$k$-tej kolumny to odpowiednio $p$ i~$q$.
    Z powyższych rozważań wynika, że $(t^{-q}-1) \alexander_{0,j} = \pm (t^{-p}-1)\alexander_{0,k}$, ale indeksy obszarów są wyznaczone z~dokładnością do stałej addytywnej.
    Biorąc $i$-tą oraz $l$-tą kolumnę, indeksów $r$ oraz $s$, dostaniemy zależności
    \begin{align}
        (t^{r-q}-1) \alexander_{l,j} & = \pm (t^{r-p} - 1)\alexander_{l,i} \\
        (t^{q-s}-1) \alexander_{k,l} & = \pm (t^{q-r} - 1)\alexander_{k,i}
    \end{align}
    co prowadzi do
    \begin{equation}
        \alexander_{l,j} = \pm \frac{t^{q-r}(t^{r-p}-1)}{t^{q-s}-1} \alexander_{k,i}
    \end{equation}
    Położenie $p = r +1$, $s =q+1$ pokazuje, że różny wybór kolumn do skreślenia zmienia wyznacznik macierzy co najwyżej o~czynnik $\pm t^m$.

    Wprowadźmy jeszcze jedną techniczną definicję.
    Dwie kwadratowe macierze będą dla nas równoważne, jeśli można przejść od jednej do drugiej przy użyciu pięciu operacji:
    \begin{enumerate}
        \item przemnożenie wiersza lub kolumny przez $-1$;
        \item zamiana dwóch wierszy lub kolumn miejscami;
        \item dodanie jednego wiersza do innego (lub kolumny do innej);
        \item przemnożenie lub podzielenie kolumny przez $t$;
        \item rozszerzenie lub zmniejszenie macierzy o~$1$ na przekątnej i~zera w~innych miejscach.
    \end{enumerate}

    Ruchy Reidemeistera prowadzą do macierzy równoważnych wyjściowym.
    Każda z~tych operacji zmienia wyznacznik macierzy o~czynnik $\pm t^{-m}$, co kończy dowód.
\end{proof}

Zwyczajowo wielomian normalizuje się: bierze reprezentanta, który jest symetryczny w~zmiennych $t$ i $t^{-1}$ oraz przyjmuje w~punkcie $1$ wartość $\alexander_L(1) = 1$.
Odwrotnie, dowolny wielomian Laurenta z~całkowitymi współczynnikami o~takich własnościach jest wielomianem Alexandera pewnego węzła:

\begin{proposition}
\label{prp:alexander_hosokawa}%
    Każdy wielomian Laurenta $p(t)$ o~całkowitych współczynnikach taki, że $p(1/t) = p(t)$ i~$p(1) = \pm 1$ jest wielomianem Alexandera pewnego węzła.
\end{proposition}

\begin{proof}[Niedowód]
    Hosokawa w \cite{hosokawa58} udowodnił to dla pomocniczego wielomianu splotów
    \begin{equation}
        \frac{\Delta(t, \ldots, t)}{(1-t)^{\max(0, \mu - 2)}},
    \end{equation}
    gdzie $\mu$ oznacza liczbę ogniw.
    Książka \cite{rolfsen76} Rolfsena na stronach 171-172 zawiera natomiast jawną konstrukcję węzła o~danym wielomianie Alexandera.
\end{proof}

Nasz niezmiennik nie wykrywa niewęzła.
Na przykład $11_{471} = 11n_{34}$, $11_{473} = 11n_{42}$ albo $(-3, 5, 7)$-precel posiadają trywialny wielomian Alexandera, zjawisko to nie występuje wśród nietrywialnych węzłów o co najwyżej 10 skrzyżowaniach.
% MAKOTO SAKUMA - A SURVEY OF THE IMPACT OF THURSTON’S WORK ON KNOT THEORY
% In fact, H. Seifert [301], J. H.C. Whitehead [336], and Kinoshita-Terasaka [175] gave systematic construction of nontrivial knots with trivial Alexander polynomial.
% H. Seifert,  ̈Uber das Geschlecht von Knoten, Math. Ann. 110 (1934), 571–592.
% J. H. C. Whitehead, On doubled knots, J. London Math. Soc. 12, 63–71 (1937).
% S. Kinoshita and H.Terasaka, On unions of knots, Osaka Math. J. 9 (1957) 131–153

% koniec podsekcji Równania kolorujące



\subsection{Wielomian Alexandera a operacje na węzłach}

Wielomian Alexandera nie odróżnia luster i~rewersów od wyjściowych węzłów:

\begin{proposition}
    Niech $L$ będzie zorientowanym splotem.
    Wtedy $\alexander_{mL}(t) = \alexander_L(1/t) = \alexander_{rL}(t)$.
\end{proposition}

\begin{proof}
    Po odbiciu diagramu względem pionowej prostej skrzyżowanie z~definicji \ref{def:colouring_equation} też się odbija.
    Równanie związane z~nim zmienia się według schematu:
    \begin{equation}
        a + tc - ta - b = 0 \rightleftharpoons a + tb - ta - c = 0
    \end{equation}
    Pierwsze równanie z~$t$ zamienionym na $1/t$ staje się drugim równaniem przemnożonym przez $-1/t$.
    Dowód drugiej równości przebiega analogicznie.
\end{proof}

\begin{proposition}
    \label{prp:alexander_multiplicative}
    Niech $K_1, K_2$ będą zorientowanymi węzłami.
    Wtedy
    \begin{equation}
        \alexander_{K_1 \shrap K_2}(t) \equiv \alexander_{K_1}(t) \alexander_{K_2}(t)
    \end{equation}
\end{proposition}

\begin{proof}
    Wybierzmy poniższe diagramy dla węzłów $K_1$ oraz $K$:
\begin{comment}
    \[\begin{tikzpicture}[baseline=-0.65ex, scale=0.07]
    %\useasboundingbox (-5, -5) rectangle (5,5);
    \begin{knot}[clip width=5, end tolerance=1pt]
        \strand[semithick] (-70, -10) rectangle (-30, 10);
        \strand[semithick] ( 30, -10) rectangle ( 70, 10);
        \strand[semithick,Latex-] (-30, 5) .. controls (-22, 5) and (-18, -5) .. (-10, -5);
        \strand[semithick] (-30,-5) .. controls (-22, -5) and (-18, 5) .. (-10,  5);
        \strand[semithick] (-10, 5) [in=up, out=right] to (-5, 0) [in=right, out=down] to (-10, -5);

        % prawe strzalki
        \strand[semithick,-Latex] (30, 5) .. controls (22, 5) and (18, -5) .. (10, -5);
        \strand[semithick] (30,-5) .. controls (22, -5) and (18, 5) .. (10,  5);
        \strand[semithick] (10, 5) [in=up, out=left] to (5, 0) [in=left, out=down] to (10, -5);

        \node[darkblue] at (-50,5) {$x_1,\ldots,x_{m-1}$};
        \node[red] at (-50,-5) {$1,\ldots,m$};

        \node[darkblue] at (50,5) {$y_1,\ldots,y_{n-1}$};
        \node[red] at (50,-5) {$1,\ldots,n$};

        \node[darkblue] at (-30,-5)[below right] {$x_m$};
        \node[darkblue] at (-15,-5)[below] {$x_0$};
        \node[darkblue] at (30,-5)[below left] {$y_n$};
        \node[darkblue] at (15,-5)[below] {$y_0$};
        \node[red] at ( 19.5,  1)[above]{$0$};
        \node[red] at (-19.5,  1)[above]{$0$};
    \end{knot}
    \end{tikzpicture}
\]
\end{comment}
    Niech $A$ oraz $B$ oznaczają macierze otrzymane z~wielomianowych równań kolorujących dla $K_1$ oraz $K_2$ przez skreślenie skrajnie lewej kolumny i~górnego wiersza.
    Wtedy $\alexander_{K_1}(t) = \det A$ oraz $\alexander_{K_2}(t) = \det B$.
    Poniższy diagram przedstawia sumę $K_1 \shrap K_2$:

\begin{comment}
\[\begin{tikzpicture}[baseline=-0.65ex, scale=0.07]
    %\useasboundingbox (-5, -5) rectangle (5,5);
    \begin{knot}[clip width=5, end tolerance=1pt]
        \strand[semithick] (-70, -10) rectangle (-30, 10);
        \strand[semithick] ( 30, -10) rectangle ( 70, 10);
        \strand[semithick,Latex-] (-30, 5) .. controls (-22, 5) and (-18, -5) .. (-10, -5);
        \strand[semithick] (-30,-5) .. controls (-22, -5) and (-18, 5) .. (-10,  5);

        % prawe strzalki
        \strand[semithick] (30, 5) .. controls (22, 5) and (18, -5) .. (10, -5);
        \strand[semithick] (30,-5) .. controls (22, -5) and (18, 5) .. (10,  5);
        \strand[semithick] (10, 5) to (-10, 5);
        \strand[semithick,-Latex] (10, -5) to (-10, -5);

        \node[darkblue] at (-50,5) {$x_1,\ldots,x_{m-1}$};
        \node[red] at (-50,-5) {$1,\ldots,m$};

        \node[darkblue] at (50,5) {$y_1,\ldots,y_{n-1}$};
        \node[red] at (50,-5) {$1,\ldots,n$};

        \node[darkblue] at (-30,-5)[below right] {$x_m$};
        \node[darkblue] at (0,-5)[below] {$x_0 = y_0$};
        \node[darkblue] at (0, 5)[above] {$z$};
        \node[darkblue] at (30,-5)[below left] {$y_n$};
        \node[red] at ( 19.5,  1)[above]{$\zeta$};
        \node[red] at (-19.5,  1)[above]{$0$};
    \end{knot}
    \end{tikzpicture}\]
\end{comment}

    Uporządkujmy łuki na diagramie jako $x_0 = y_0$, $x_1, \ldots, x_m$, $y_1, \ldots, y_n$, $z$; skrzyżowania: $0, 1, \ldots, m$ (z $K_1$), $1, \ldots, n$ (z $K_2$), $\zeta$.
    Wielomianowe równanie kolorujące dla $K_1 \shrap K_2$ nad skrzyżowaniami $1, \ldots, m$ ($1, \ldots, n$) są takie same, jak przed dodaniem do siebie węzłów.
    Nad skrzyżowaniem $\zeta$ równanie orzeka, że $(1-t)y_0+t z-y_n=0$.

    Wynika stąd, że $\alexander_{K_1 \shrap K_2}(t)$ jest wyznacznikiem macierzy
    \begin{align*}
        M &= \left(\begin{array}{cc|cc|c}
            & & & & \\
            \multicolumn{2}{c|}{\smash{\raisebox{.5\normalbaselineskip}{$A$}}} & & \\
            \hline \\[-\normalbaselineskip]
            & & & & \\
            & & \multicolumn{2}{c|}{\smash{\raisebox{.5\normalbaselineskip}{$B$}}}\\ \hline
            & & & -1 & t
    \end{array}\right)
    \end{align*}

    Skreśliliśmy lewą kolumnę oraz górny wiersz.
    Zatem $\alexander_{K_1 \shrap K_2}(t) = t^?\alexander_{K_1}(t) \alexander_{K_2}(t)$, jeśli nie pomyliliśmy się w~obliczeniach.
\end{proof}

% koniec podsekcji Wielomian Alexandera a operacje na węzłach


\subsection{Relacja kłębiasta (definicja druga)}

\begin{definition}[relacja kłębiasta]
% TODO: uwspólnić definicje relacji kłębiastych?
\index{relacja kłębiasta}%
    Niech $L$ będzie zorientowanym splotem z ustalonym diagramem oraz skrzyżowaniem.
    Oznaczmy przez $L_+, L_-, L_0$ trzy diagramy splotów, które różnią się jedynie na małym obszarze wokół ustalonego skrzyżowania:
\begin{comment}
    \begin{figure}[H]
        \centering
        \begin{minipage}[b]{.3\linewidth}
            \centering
            \[\LargePlusCrossingArrows\]
            \subcaption{$L_+$}
        \end{minipage}
        \begin{minipage}[b]{.3\linewidth}
            \centering
            \[\LargeMinusCrossingArrows\]
            \subcaption{$L_-$}
        \end{minipage}
        \begin{minipage}[b]{.3\linewidth}
            \centering
            \[\LargeJustSmoothing\]
            \subcaption{$L_0$}
        \end{minipage}
    \end{figure}
\end{comment}
    Mówimy, że niezmiennik zorientowanych splotów $f$ spełnia relację kłębiastą, jeżeli wartości $f(L_+)$, $f(L_-)$ i $f(L_0)$ są związane pewnym wielomianowym równaniem, niezależnie od wyboru splotu $L$.
\end{definition}

% DICTIONARY;skein;kłąb;-
% DICTIONARY;skein relation;relacja kłębiasta;-
Termin ,,skein'' (kłąb) wprowadził Conway około roku 1970, kontynuując tradycję używania słów, które kojarzą się ze sznurkami.
\index[persons]{Conway, John}%
Czasami mówi się o relacji motkowej, my nie zamierzamy używać tego synonimu.

\begin{definition}
    Niech $L$ będzie zorientowanym splotem.
    Wielomian Laurenta $\alexander_L(t) \in \Z[t^{\pm 1/2}]$, który spełnia relację kłębiastą
    \begin{equation}
        \alexander_{L_+}(t) - \alexander_{L_-}(t) - (t^{1/2} - t^{-1/2}) \alexander_{L_0}(t) = 0
    \end{equation}
    z warunkiem brzegowym $\alexander_{\SmallUnknot}(t) = 1$, nazywamy wielomianem Alexandera.
\end{definition}

Wzór ten, choć znany był Alexanderowi, nie zyskał przez wiele dekad uwagi matematyków.
\index[persons]{Alexander, James}%
Mogło tak być, gdyż w pracy \cite{alexander28} znalazł się on na samym końcu, pod nagłówkiem ,,twierdzenia różne''.
Na nowo odkrył go Conway: chcąc szybko liczyć wielomian Alexandera zaproponował, by reparametryzować go wzorem $\alexander(x^2) = \conway(x - 1/x)$.
Spełnia wtedy zależność
\begin{equation}
    \conway_{L_+}(x)- \conway_{L_-}(x) = x \conway_{L_0}(x).
\end{equation}

Relacja kłębiasta wystarcza do wyznaczenia $\alexander_L$ każdego splotu na mocy lematu \ref{lem:unknotting_well_defined}.

\begin{proposition}
\index{splot!rozszczepialny}%
\label{prp:alexander_unlinks}
    Niech $L$ będzie splotem rozszczepialnym.
    Wtedy $\alexander_L(t) \equiv 0$.
\end{proposition}

\begin{proof}
    Skorzystamy z~relacji kłębiastej.
    Niech $L_0$ będzie splotem rozsczepialnym z~dwoma ogniwami.
    Wtedy węzły $L_+$ oraz $L_-$ powstałe przez dodanie skrzyżowania między ogniwami są tego samego typu, zatem
    \begin{equation}
        \alexander_{L_0} = \frac{\alexander_{L_+} - \alexander_{L_-}}{t^{1/2} - t^{-1/2}} = 0,
    \end{equation}
    a to chcieliśmy udowodnić.
\end{proof}

Implikacja w drugą stronę jest fałszywa.
Niech $\sigma_* = \sigma_{2} \sigma_{3}^{-2} \sigma_{2}$.
Domknięcie warkocza $\sigma_{1} \sigma_* \sigma_{1} \sigma_{3} \sigma_* \sigma_{1} \sigma_{3} \sigma_* \sigma_{3}$ nie jest rozszczepialne, ale jego wielomian Alexandera jest zerem.
% TODO: nie potrafię znaleźć, kto pierwszy odkrył takie warkocze
% patrz też https://math.stackexchange.com/questions/3740577/why-the-multivariate-alexander-polynomial-of-l14n63195-is-zero
\index{warkocz}%
Warkocze poznamy w rozdziale piątym.

\begin{corollary}
    Wielomian Alexandera nie odróżnia od siebie niesplotów.
\end{corollary}

Wady tej nie posiada wielomian Jonesa.

% koniec podsekcji Relacja kłębiasta




\subsection{Wielomian Alexandera a niezmienniki numeryczne}
\begin{proposition}
    \label{prp:alexander_determinant}
    Niech $L$ będzie zorientowanym splotem.
    Wtedy $|\alexander_L(-1)| = \det L$.
\end{proposition}

\begin{proof}
    Wystarczy porównać definicję dla $\alexander_L$ (\ref{def:alexander_polynomial}) oraz $\det L$ (\ref{def:determinant}).
\end{proof}

\begin{proposition}
    Wielomian Alexandera zadaje ograniczenie na indeks skrzyżowaniowy:
    \begin{equation}
        \deg \alexander_K(t) < \crossing K.
    \end{equation}
\end{proposition}

Być może istnieje bezpośredni dowód tej nierówności, ale jedyne uzasadnienie, jakie znam, opiera się na fakcie \ref{prp:alexander_genus} oraz wniosku \ref{cor:crossing_genus} -- czyli własnościach genusu.
\index{genus}

\begin{proposition}
    Tylko skończenie wiele węzłów alternujących może mieć ten sam wielomian Alexandera.
\end{proposition}

\begin{proof}
    Załóżmy nie wprost, że istnieje nieskończony ciąg $K_n$ węzłów alternujących o~tym samym wielomianie Alexandera $\alexander_K(t)$.
    Wszystkie jego wyrazy mają ten sam wyznacznik, ponieważ $\det K_n = |\alexander_K(-1)|$.
    Z faktu \ref{prp:bankwitz} wynika, że indeks skrzyżowaniowy węzłów $K_n$ jest wspólnie ograniczony: $c_k \le \det K_n = \det K$.
    To prowadzi do sprzeczności: węzłów o~danym indeksie skrzyżowaniowym jest tylko skończenie wiele.
\end{proof}

% koniec podsekcji Wielomian Alexandera a niezmienniki numeryczne




\subsection{Pochodna Foxa (definicja trzecia)}
\index{pochodna Foxa|(}%
Pojęcie grupy węzła oraz jej prezentacji wprowadzamy później, więc warto wrócić tutaj dopiero przy drugim lub następnym czytaniu!
Są dwa konkurencyjne podejścia do prezentowania grupy węzła.
Zgodnie z pomysłem pochodzącym jeszcze od Dehna, można przypisywać różne litery czterem częściom płaszczyzny, które są wycinane przez łuki skrzyżowania.
\index[persons]{Dehn, Max}%
Między innymi pierwsza praca Alexandera była bliska takiemu postępowaniu, ale dla oszczędności miejsca, nie opiszemy go wcale.
\index[persons]{Alexander, James}%

Alternatywne rozwiązanie każe etykietować nie obszary płaszczyzny, tylko łuki diagramu.
Klasyczne podręczniki teorii węzłów, takie jak \cite{crowell63}, macierz, a~co za tym idzie, także wielomian Alexandera wprowadzają właśnie w ten sposób: przy użyciu prezentacji Wirtingera i~pochodnej Foxa.
Jak sugeruje tytuł podsekcji, opiszemy teraz ten sposób.

Zakręcone!

% DICTIONARY;Fox derivative;pochodna Foxa;-
\begin{definition}[pochodna Foxa]
\label{def:fox_derivative}%
    Niech $G$ będzie wolną grupą generowaną przez (niekoniecznie skończony) podzbiór $\{g_i\}_{i \in I}$.
    Odwzorowanie $\partial/\partial g_i \colon G \to \Z G$ spełniające trzy aksjomaty:
    \begin{align}
        \frac{\partial}{\partial g_i} (e) & = 0 \\
        \frac{\partial}{\partial g_i} (g_j) & = \delta_{ij} \\
        \forall u, v \in G : \frac{\partial}{\partial g_i} (uv) & = \frac{\partial}{\partial g_i}(u) + u \frac{\partial}{\partial g_i} (w),
    \end{align}
    gdzie $\delta_{ij}$ oznacza deltę Kroneckera, nazywamy pochodną cząstkową Foxa.
\end{definition}

Fox opublikował w~Annals of Mathematics cykl pięciu artykułów \cite{fox53}, \cite{fox54}, \cite{fox56}, \cite{fox58}, \cite{fox60} poświęconych wolnemu rachunkowi różniczkowemu.
\index[persons]{Fox, Ralph}%
Definicja \ref{def:fox_derivative} jest tylko małym wycinkiem tego cyklu.
Strona nLab wspomina jeszcze o ,,\emph{a nice introduction in} \cite{crowell63}'', podręczniku Crowella, Foxa.

Ustalmy grupę $G$ oraz jej prezentację $\langle X | R \rangle = F/N$, gdzie $F = \langle X \rangle$ jest wolną grupą abelową, zaś $N$ to domknięcie normalne relacji $R$.
Mamy wtedy kanoniczny rzut $\varphi \colon F \to G$.
Niech $G^{ab} = G/[G, G]$ oznacza abelianizację, wtedy funkcja $\varphi^{ab} \colon F \to G^{ab}$ jest dobrze określona.
Ponieważ nie prowadzi to do zamieszania, tych samych liter będziemy używać także do funkcji $\varphi \colon \Z F \to \Z G$.
Definiujemy teraz macierz Jacobiego wymiaru $n \times n$:
\index{macierz!Jacobiego}%
\begin{equation}
    J = \left(\varphi \left(\frac{\partial r_i}{\partial x_j}\right) \right).
\end{equation}
oraz macierz $J^{ab}$, która jest obrazem $J$ nad $\Z G^{ab}$.
W pierścieniu $\Z G^{ab}$ wyróżnia się ideały generowane przez minory (wyznaczniki podmacierzy) rozmiaru $i \times i$ w $J^{ab}$.
Ciąg $D_1, D_2, \ldots$ tych ideałów jest z dokładnością do jakichś technicznych szczegółów niezmiennikiem, to znaczy nie zależy od prezentacji.

W szczególnym przypadku, kiedy $G$ jest grupą węzła, relacje $r_i$ pochodzą z prezentacji Wirtingera, zaś grupa $G^{ab}$ jest nieskończona, cykliczna.
Niech $t$ oznacza jej generator; wtedy najwyższy niezerowy ideał $D_i$ jest główny.
Generator tego ideału nazywamy wielomianem Alexandera.
\index{wielomian!Alexandera}%

Rachunki są trochę prostsze niż wydają się być.
Wystarczy najpierw wykreślić z macierzy $J$ jedną kolumnę oraz jeden wiersz, po czym podstawić za wszystkie litery zmienną $t$ i policzyć wyznacznik.
Otrzymaliśmy znowu wielomian Alexandera!

\index{pochodna Foxa|)}%

% koniec podsekcji pochodna Foxa




\subsection{Różniaste różności}
Istnieje odmiana wielomianu Alexandera, która liczy sobie tyle zmiennych, ile ogniw posiada splot (nie opisywaliśmy jej i~nie zamierzamy).
Klasyfikację \ref{prp:alexander_hosokawa} można częściowo uogólnić: Torres \cite{torres53} znalazł dwie geometryczne własności, nazwane później warunkami Torresa.
\index[persons]{Torres, Guillermo}%
\index{warunek!Torresa}%
Są warunkami koniecznymi, ale nie wystarczającymi, jak odkrył ponad ćwierć wieku później Hillman \cite{hillman81}: wielomian
\index[persons]{Hillman, Jonathan}%
\begin{equation}
    D(x,y) = \frac{(1 - x^6y^6)(x - 1 + 1/x) - 2(1 - x^5y^5)(1 - x)(1 - y)}{1-xy}
\end{equation}
spełnia warunki Torresa, ale nie jest wielomianem Alexandera żadnego splotu.

Fox \cite{fox62} podejrzewał, że
\index[persons]{Fox, Ralph}%
\begin{conjecture}
\index{hipoteza!trapezoidalna}%
    Ciąg współczynników wielomianu Alexandera węzła alternującego jest unimodalny.
\end{conjecture}
Dowód podano dla węzłów algebraicznych (Murasugi \cite{murasugi85}) oraz genusu dwa (Ozsváth i~Szabó \cite{ozsvath03}).
\index[persons]{Murasugi, Kunio}%
\index[persons]{Ozsváth, Peter}%
\index[persons]{Szabó, Zoltán}%
Hipoteza w~ogólnym przypadku pozostaje otwarta.

Wiemy natomiast, że kolejne współczynniki wielomianu Conwaya węzła alternującego są przeciwnych znaków i niezerowe (Murasugi \cite[s. 242]{murasugi96} odsyła do swojej wcześniejszej pracy \cite{murasugi59}, gdzie używa archaicznej nomenklatury: \emph{Schlauchknoten} zamiast węzłów satelitarnych jak w \cite[s. 245]{schubert53}).
\index{Schlauchknoten}%
Wynika stąd, że węzły $(p, q)$-torusowe dla $p > q > 2$ oraz pierwsze węzły satelitarne nie są alternujące.

% koniec podsekcji Różniaste różności



\index{wielomian!Alexandera|)}

% koniec sekcji Wielomian Alexandera

\section{Rewolucyjny wielomian Jonesa}
\index{wielomian!Jonesa|(}%
Jones odkrył wielomian określany teraz jego nazwiskiem podczas badań algebr operatorów: zauważył, że przypominają one warkocze.
Znalazł reprezentację grupy warkoczy w~swojej algebrze i~złożył ją ze śladem na niej, pierwsza definicja była więc czysto algebraiczna.

Rezultat Jonesa był przełomowy, ale my odłożymy przytoczenie szczegółów na później, najpierw zaś pokażemy, jak można uzyskać jego wielomian w~prostszy sposób: metodami kombinatorycznymi, pojawi się też relacja kłębiasta.

Na koniec zaprezentujemy, jak wielomian Jonesa umożliwia dowód hipotez Taita.

\subsection{Definicja algebraiczna -- algebra Temperleya-Lieba}

% https://en.wikipedia.org/wiki/Jones_polynomial
Jones otrzymał swój wielomian jako efekt uboczny badań nad algebrami operatorów: wziął ,,ślad'' pewnej reprezentacji warkoczy w~algebrę, która miała ważne znaczenie w~mechanice statystycznej.
Jego praca \cite{jones85} jest bardzo zwięzła, nam pomogła książka \cite[s. 85-103]{kauffman91} oraz strona ,,Aharonov–Jones–Landau algorithm'' z angielskiej Wikipedii.

% książka Kauffmana
\begin{definition}[algebra Temperleya-Lieba]
\index{algebra Temperleya-Lieba}%
    Niech $R$ będzie przemiennym pierścieniem, w~którym ustalono element $\delta \in R$.
    Wtedy $R$-algebrę $TL_n(\delta)$ generowaną przez elementy $e_1, \ldots, e_{n-1}$, które związane są relacjami
    \begin{align}
        U_i^2 & = \delta U_i, \\
        U_i U_{i \pm 1} U_i & = U_i, \\
        U_i U_j & = U_j U_i
    \end{align}
    dla $|i-j| \ge 2$, nazywamy algebrą Temperleya-Lieba.
\end{definition}

$TL_n(\tau)$ daje się przedstawić przy użyciu diagramów: prostokątów, których przeciwległe boki zawierają po $n$ punktów połączonych w~pary tak, by uniknąć samoprzecięć.
Mnożenie elementów algebry odpowiada sklejaniu dwóch diagramów, przy czym każdą zamkniętą pętlę zamieniamy na dodatkowy czynnik $\delta$.
To prawie są warkocze.


\begin{comment}
\begin{figure}[H]
\[
    \begin{tikzpicture}[baseline=-0.65ex, scale=0.2]
        \useasboundingbox (-6, -5) rectangle (6, 4);
        \begin{knot}[clip width=5, end tolerance=1pt]
            \strand[semithick] (-3, -4) to (3, -4);
            \strand[semithick] (-3, -2) to (3, -2);
            \strand[semithick] (-3, -0) to (3, +0);
            \strand[semithick] (-3, +2) to (3, +2);
            \strand[semithick] (-3, +4) to (3, +4);
            \node[darkblue] at (0, -6) {$1$};
    \end{knot}
    \end{tikzpicture}
    \begin{tikzpicture}[baseline=-0.65ex, scale=0.2]
        \useasboundingbox (-6, -5) rectangle (6, 4);
        \begin{knot}[clip width=5, end tolerance=1pt]
            \strand[semithick] (-3, -4) [in=down, out=right] to (-1, -3) [in=right, out=up] to (-3, -2);
            \strand[semithick] (3, -4) [in=down, out=left] to (1, -3) [in=left, out=up] to (3, -2);
            \strand[semithick] (-3, -0) to (3, +0);
            \strand[semithick] (-3, +2) to (3, +2);
            \strand[semithick] (-3, +4) to (3, +4);
            \node[darkblue] at (0, -6) {$e_1$};
    \end{knot}
    \end{tikzpicture}
    \begin{tikzpicture}[baseline=-0.65ex, scale=0.2]
        \useasboundingbox (-6, -5) rectangle (6, 4);
        \begin{knot}[clip width=5, end tolerance=1pt]
            \strand[semithick] (-3, -4) to (3, -4);
            \strand[semithick] (-3, -2) [in=down, out=right] to (-1, -1) [in=right, out=up] to (-3, 0);
            \strand[semithick] (3, -2) [in=down, out=left] to (1, -1) [in=left, out=up] to (3, 0);
            \strand[semithick] (-3, +2) to (3, +2);
            \strand[semithick] (-3, +4) to (3, +4);
            \node[darkblue] at (0, -6) {$e_2$};
    \end{knot}
    \end{tikzpicture}
    \begin{tikzpicture}[baseline=-0.65ex, scale=0.2]
        \useasboundingbox (-6, -5) rectangle (6, 4);
        \begin{knot}[clip width=5, end tolerance=1pt]
            \strand[semithick] (-3, -4) to (3, -4);
            \strand[semithick] (-3, -2) to (3, -2);
            \strand[semithick] (-3, 0) [in=down, out=right] to (-1, 1) [in=right, out=up] to (-3, 2);
            \strand[semithick] (3, 0) [in=down, out=left] to (1, 1) [in=left, out=up] to (3, 2);
            \strand[semithick] (-3, +4) to (3, +4);
            \node[darkblue] at (0, -6) {$e_3$};
    \end{knot}
    \end{tikzpicture}
    \begin{tikzpicture}[baseline=-0.65ex, scale=0.2]
        \useasboundingbox (-6, -5) rectangle (6, 4);
        \begin{knot}[clip width=5, end tolerance=1pt]
            \strand[semithick] (-3, -4) to (3, -4);
            \strand[semithick] (-3, -2) to (3, -2);
            \strand[semithick] (-3, -0) to (3, +0);
            \strand[semithick] (-3, 2) [in=down, out=right] to (-1, 3) [in=right, out=up] to (-3, 4);
            \strand[semithick] (3, 2) [in=down, out=left] to (1, 3) [in=left, out=up] to (3, 4);
            \node[darkblue] at (0, -6) {$e_4$};
    \end{knot}
    \end{tikzpicture}
\]
\caption{Diagramatyczne przedstawienie elementów bazowych algebry $TL_5(\delta)$}
\end{figure}
\end{comment}

Dla ustalonej liczby zespolonej $A$, funkcja $\rho_A \colon B_n \to TL_n(\delta)$ zadana na generatorach grupy warkoczy: $\rho_A(\sigma_i) = AU_i + A^{-1}$.
Bezpośredni rachunek pokazuje, że jest reprezentacją, jeśli $\delta = -A^2-A^{-2}$

% książka Kauffmana
\begin{definition}[algebra Jonesa]
\index{algebra Jonesa}%
    Niech $R$ będzie przemiennym pierścieniem, zaś $\tau$ skalarem, który komutuje ze wszystkimi innymi elementami.
    Wtedy $R$-algebrę $TL_n(\tau)$ generowaną przez elementy $e_1, \ldots, e_{n-1}$, związanymi relacjami
    \begin{align}
        e_i^2 & = e_i, \\
        e_i e_{i \pm 1} e_i & = \tau e_i, \\
        e_i e_j & = e_j e_i
    \end{align}
    dla $|i-j| \ge 2$, nazywamy algebrą Jonesa.
\end{definition}

Kauffman pisze, że algebry są ze sobą związane: jeśli położymy $e_i = \delta^{-1} U_i$ oraz $\tau = \delta^{-2}$, to aksjomaty (Jonesa) są spełnione.

% artykuł na Wiki
\begin{definition}[ślad Markowa]
\index{sZZZlad@ślad Markowa}%
    Niech $T \in TL_n(\delta)$ będzie elementem algebry Temperleya-Lieba będącym iloczynem generatorów $e_1, \ldots, e_{n-1}$, którego domknięcie rozpada się na $m$ składowych spójności.
    Śladem Markowa elementu $K$ nazywamy wielkość $\operatorname{tr} K = \delta^{m-n}$.
\end{definition}

Ślad Markowa przedłuża się liniowo do całej algebry Temperleya-Lieba i jest prawdziwym śladem: spełnia warunki $\operatorname{tr} (1) = 1$ oraz $\operatorname{tr} (T_1T_2) = \operatorname{tr} (T_2T_1)$.
Ma też dodatkową własność, że jeśli $w$ jest słowem na literach $e_1, e_2, \ldots, e_{i-1}$, to $\operatorname{tr}(we_i) = \delta^{-1} \operatorname{tr} (w)$.

Jones skorzystał z twierdzeń Alexandera i Markowa, złożył ze sobą funkcje $\operatorname{tr}$ oraz $\rho_A$ oraz znormalizował wynik.
Dostał tak niezmiennik splotów $L$ (domknięć warkoczy $B$):
\begin{equation}
    V_L(A^{-4}) = (-A)^{3 \writhe D} \delta^{n-1} (\operatorname{tr} \circ \rho_A)(B).
\end{equation}

% Koniec podsekcji Oryginalna praca Jonesa


\subsection{Definicja kombinatoryczna -- klamra Kauffmana}
\index{klamra Kauffmana|(}
Klamra Kauffmana to wielomian Laurenta jednej zmiennej zdefiniowany w pracy \cite{kauffman87} z 1987 roku, oparty na ruchach Reidemeistera.
Dzięki swojej prostocie mógł być odkryty na początku XX wieku, nim jeszcze maszyneria teorii węzłów została rozwinięta.

Poszukujemy niezmiennika dla splotów o~kilku prostych własnościach.
Przede wszystkim żądamy, by niewęzłowi przypisany był wielomian $1$: $\bracket{\SmallUnknot} = 1$.
Po drugie chcemy wyznaczać nawiasy znając je dla prostszych splotów, co zapiszemy symbolicznie:
\begin{comment}
\begin{equation}
    \bracket{\MediumMinusCrossing} = A \bracket{\MediumAlphaSmoothing} + B \bracket{\MediumBetaSmoothing}
\end{equation}
\end{comment}
Zależy nam też na tym, by móc dodać do splotu trywialną składową: $\langle L \cup \SmallUnknot \rangle = C \langle L \rangle$.
Prosty rachunek pokazuje wpływ drugiego ruchu Reidemeistera na klamrę:
\begin{comment}
\begin{equation}
    \bracket{\MediumKauffmanReidemeisterTwoA}
    = (A^2 + ABC + B^2) \bracket{\MediumBetaSmoothing} + BA \bracket{\MediumAlphaSmoothing}
    \stackrel{?}{=} \bracket{\MediumAlphaSmoothing}.
\end{equation}
\end{comment}

Aby zachodziła ostatnia równość wystarczy przyjąć $B = A^{-1}$, co wymusza na nas wartość trzeciego parametru: $C = -A^2 - A^{-2}$.
W ten sposób odkryliśmy definicję.

\begin{definition}[klamra Kauffmana]
    \label{def:kauffman_bracket}
    Wielomian Laurenta $\bracket{D}$ dla diagramu splotu $D$ zmiennej $A$,
    który jest niezmienniczy ze względu na gładkie deformacje diagramu,
    a~przy tym spełnia trzy poniższe aksjomaty:
\begin{comment}
    \begin{align}
        \bracket{\MediumUnknot} & = 1
        \label{eqn:kauffman_axiom_1}%
        \\
        \bracket{\MediumMinusCrossing} & =
        A \bracket{\MediumAlphaSmoothing} +
        A^{-1} \bracket{\MediumBetaSmoothing}
        \label{eqn:kauffman_axiom_2}%
        \\
        \bracket{D \sqcup \MediumUnknot} & =
        (-A^{-2} - A^2) \bracket{D}
        \label{eqn:kauffman_axiom_3}%
    \end{align}
\end{comment}
    nazywamy klamrą Kauffmana.
\end{definition}

Drugi aksjomat jest wariacją na temat relacji kłębiastej.

\begin{lemma}
    Klamra Kauffmana każdego diagramu wyznacza się w~skończenie wielu krokach.
\end{lemma}

\begin{proof}
    Najprościej dowieść tego indukcyjnie, ze względu na liczbę skrzyżowań na diagramie splotu.
    Baza indukcji to przypadek zero skrzyżowań, czyli niesplotów.
    Zauważmy, że ostatni (i później pierwszy) aksjomat pozwala wyznaczyć wartość klamry Kauffmana dla każdego niesplotu w tylu krokach, ile ogniw ma niesplot.

    Pozostał krok indukcyjny.
    Załóżmy, że wyznaczyliśmy już wartości klamry dla każdego diagramu o $n$ skrzyżowaniach i chcemy ją obliczyć dla kolejnego splotu z diagramem o~$n + 1$ skrzyżowaniach.
    Pozwala na to drugi aksjomat, usuwający jedno ze skrzyżowań.
\end{proof}

Przedstawimy teraz wpływ ruchów Reidemeistera na nasz nowy wielomian.

\begin{lemma}
    Drugi i~trzeci ruch Reidemeistera nie ma wpływu na klamrę Kauffmana,
    pierwszy ruch zmienia ją zgodnie z~regułą:
\begin{comment}
    \begin{equation}
        \bracket{\MediumReidemeisterOneLeft} = -A^{-3} \bracket{\,\MediumReidemeisterOneStraight\,}.
    \end{equation}
\end{comment}
\end{lemma}

\begin{proof}
Pierwszy ruch Reidemeistera:
\begin{comment}
\begin{align}
    \bracket{\MediumReidemeisterOneLeft} & \stackrel{K2}{=} A
    \bracket{\MediumReidemeisterOneSmoothA} +
    A^{-1} \bracket{\MediumReidemeisterOneSmoothB} \\ & \stackrel{K3}{=}
    A \bracket{\MediumReidemeisterOneStraight} +
    A^{-1}(-A^{-2}-A^2) \bracket{\MediumReidemeisterOneStraight} =
    -A^{-3}\bracket{\MediumReidemeisterOneStraight}
\end{align}
\end{comment}

Dla drugiego ruchu:
\begin{comment}
\begin{align}
    \bracket{\MediumKauffmanReidemeisterTwoA} & \stackrel{K2}{=}
    A \bracket{\MediumKauffmanReidemeisterTwoB} +
    A^{-1} \bracket{\MediumKauffmanReidemeisterTwoC} \\ & \stackrel{K1}{=}
    -A^{-2} \bracket{\MediumBetaSmoothing} +
    A^{-1} \bracket{\MediumKauffmanReidemeisterTwoC} \\ & \stackrel{K2}{=}
    -A^{-2} \bracket{\MediumBetaSmoothing} +
    A^{-1}A \bracket{\MediumAlphaSmoothing} +
    A^{-1}A^{-1} \bracket{\MediumBetaSmoothing} \\ & =
    \bracket{\MediumAlphaSmoothing}
\end{align}
\end{comment}

Dla trzeciego ruchu:
\begin{comment}
\begin{align}
\bracket{\MediumKauffmanReidemeisterThreeA} & \stackrel{K2}{=}
A \bracket{\MediumKauffmanReidemeisterThreeB} +
A^{-1} \bracket{\MediumKauffmanReidemeisterThreeC} \stackrel{R2}{=}
A \bracket{\MediumKauffmanReidemeisterThreeD} +
A^{-1} \bracket{\MediumKauffmanReidemeisterThreeE} \\ & \stackrel{R2}{=}
A \bracket{\MediumKauffmanReidemeisterThreeFlippedB} +
A^{-1} \bracket{\MediumKauffmanReidemeisterThreeFlippedC} \stackrel{K2}{=}
\bracket{\MediumKauffmanReidemeisterThreeFlippedA},
\end{align}
\end{comment}
korzystaliśmy tu z~własności drugiego ruchu.
\end{proof}

\begin{corollary}
    Rozpiętość klamry Kauffmana jest niezmiennikiem węzłów.
\end{corollary}

Klamra Kauffmana nie jest niezmiennikiem węzłów ze względu na I ruch Reidemeistera.
Jeżeli przypomnimy sobie, że na mocy lematu \ref{lem:writhe_reidemeister} spin także nie jest niezmiennikiem węzłów, odkryjemy ,,trik Kauffmana'': niedoskonałości tych dwóch obiektów znoszą się wzajemnie.
\index{trik Kauffmana}%
\index{spin}%

\begin{definition}
\label{def:jones_polynomial}%
    Niech $L$ będzie zorientowanym splotem.
    Wielomian Laurenta $\jones(L) \in \Z[t^{\pm 1/2}]$ określony przez
    \begin{equation}
        \jones(L)=\left[(-A)^{-3w(D)} \bracket{D}\right]_{t^{1/2}=A^{-2}},
    \end{equation}
    gdzie $D$ to dowolny diagram dla $L$, nazywamy wielomianem Jonesa.
\end{definition}

Sama klamra odegrała ważną rolę podczas unifikacji wielomianu Jonesa oraz innych niezmienników kwantowych.
W szczególności pozwoliła na uogólnienie go do niezmiennika 3-rozmaitości.

\begin{proposition}
    Wielomian Jonesa jest niezmiennikiem zorientowanych splotów.
\end{proposition}

\begin{proof}
    %Skorzystamy z~tego, że indeks zaczepienia jest niezmiennikiem.
    Wystarczy pokazać niezmienniczość $(-A)^{-3w(D)}\langle D\rangle$ na ruchy Reidemeistera.

    Niech
\begin{comment}
    \begin{equation}
        D_1 = \LargeReidemeisterOneLeft,
        \quad\quad\quad
        D_2 = \LargeReidemeisterOneStraight
    \end{equation}
\end{comment}
    Jak zauważyliśmy już wcześniej, II i III ruch nie zmienia ani spinu, ani klamry Kauffmana.
    Pozostało sprawdzić I ruch.
    Mamy:
    \begin{equation}
        (-A)^{-3 w\left(D_1\right)} \bracket{D_1} =
        (-A)^{-3 w\left(D_2\right) + 3} (-A)^{-3}\bracket{D_2} =
        (-A)^{-3 w\left(D_2\right)} \bracket{D_2},
    \end{equation}
    co kończy dowód.
\end{proof}

Zazwyczaj, ale nie zawsze, wielomian Jonesa lepiej radzi sobie z odróżnianiem od siebie splotów.
Zaczniemy od wyznaczenia bezpośrednio z definicji, jakie są wielomiany Jonesa niesplotów.
Dla porównania, wielomian Alexandera wszystkich splotów rozszczepialnych jest taki sam (stwierdzenie \ref{prp:alexander_unlinks}).

\begin{proposition}
    \label{prp:jones_trivial_link}
    Wielomianem Jonesa splotu trywialnego o $n$ ogniwach jest
    \begin{equation}
        \jones(K_n) = \left(-\sqrt{t} - \frac{1}{\sqrt {t}}\right)^{n-1}.
    \end{equation}
\end{proposition}

Co więcej, wielomian Jonesa odróżnia od siebie dowolne dwa węzły pierwsze o~co najwyżej 9 skrzyżowaniach.
Dalej występują już kolizje, oto pełna ich lista do 10 skrzyżowań:
$5_{1}$ -- $10_{132}$,
$8_{8}$ -- $10_{129}$,
$8_{16}$ -- $10_{156}$,
$10_{22}$ -- $10_{35}$,
$10_{25}$ -- $10_{56}$,
$10_{40}$ -- $10_{103}$,
$10_{41}$ -- $10_{94}$,
$10_{43}$ -- $10_{91}$,
$10_{59}$ -- $10_{106}$,
$10_{60}$ -- $10_{86}$,
$10_{71}$ -- $10_{104}$,
$10_{73}$ -- $10_{83}$,
$10_{81}$ -- $10_{109}$,
$10_{137}$ -- $10_{155}$.
Jones wiedział, że wielomianowe niezmienniki nie radzą sobie z~odróżnianiem od siebie mutantów, dlatego zapytał w~2000 roku, czy jego wielomian wykrywa niewęzły.
Pozostaje to otwartym problemem do dziś.

\begin{conjecture}
\index{hipoteza!o wielomianie Jonesa i niewęźle}%
\label{con:jones}%
    Niech $K$ będzie węzłem.
    Jeśli $\jones_K(t) \equiv 1$, to $K$ jest niewęzłem.
\end{conjecture}

Hipotezę zweryfikowano komputerowo dla węzłów o~małej liczbie skrzyżowań.
W latach dziewięćdziesiątych Hoste, Thistlethwaite, Weeks zrobili to dla węzłów spełniających $\operatorname{cr} \le 16$.
Wynik poprawiano: Dasbach, Hougardy w~1997 do $\operatorname{cr} = 17$; Yamada w~2000 do $\operatorname{cr} = 18$; wreszcie Tuzun, Sikora w~2016 do $\operatorname{cr} \le 22$.
Patrz \cite[s. 381]{ohtsuki02}.

% TODO: Argumentem przemawiającym za prawdziwością hipotezy jest twierdzenie ,,udowodnione'' przez Jørgena Andersena.
% TODO: \textbf{NIE Pokazał on, że rodzina okablowanych wielomianów Jonesa wykrywa niewęzeł.}
% TODO: Tutaj $n$-okablowanie węzła $K$ to $n$-komponentowy splot $K^n$, który powstaje z~$K$ po zamianie pojedynczej ,,żyły'' na $n$ równoległych żył.

Istnieją sploty o~trywialnym wielomianie Jonesa, jest ich nawet nieskończenie wiele, jak Eliahou, Kauffman i~Thistlethwaite pokazali w~pracy \cite{eliahou03}.

\begin{proposition}
    Niech $k \ge 2$ będzie liczbą naturalną.
    Istnieje nieskończenie wiele splotów pierwszych z $k$ ogniwami, których wielomian Jonesa nie odróżnia od niesplotu z $k$ ogniwami.

    Co więcej, można wymagać, by wszystkie te sploty były satelitami splotu Hopfa.
\index{splot!Hopfa}%
\end{proposition}

Niech $\jones$ będzie wielomianem Jonesa splotu $L$ o~$n$ składowych spójności.
Jego wartości w~niektórych pierwiastkach jedności są związane z~innymi niezmiennikami węzłów.
I tak przyjmując oznaczenie $\omega_k = \exp(2\pi i/k)$ mamy

\begin{proposition}
    \label{prp:jones_at_roots_of_unity}
    $\jones_L(\omega_3) = 1$.
\end{proposition}

\begin{proposition}
    $\jones_L(1) = (-2)^{n-1}$.
\end{proposition}

\begin{proof}
    Jak wkrótce się przekonamy, to proste wnioski z~relacji kłębiastej.
    Explicite wskazał je Jones w \cite[twierdzenie 14, 15]{jones85}.
\end{proof}

\begin{proposition}
    Pochodna w punkcie $t = 1$ znika: $\jones'_L(1) = 0$.
\end{proposition}

\begin{proof}
    Twierdzenie 16 w \cite{jones85}.
\end{proof}

\begin{proposition}
    $V_L(\omega_6) = \pm i^{n-1} \cdot (\sqrt 3i)^r$, gdzie $r$ jest rangą pierwszej grupy homologii podwójnego rozgałęzionego nakrycia $L$ nad $\Z_3$.
\end{proposition}

\begin{proof}
    Znak $\pm$ został wyznaczony przez Lipsona w \cite{lipson86}, praca ta zawiera też odsyłacz do wyprowadzenia reszty wzoru.
\end{proof}

\begin{proposition}
    Liczba trzy-kolorowań splotu $L$ wynosi $3|\jones_L(\omega_6)|^2$.
\end{proposition}

\begin{proof}
    Patrz \cite{przytycki98}.
\end{proof}

\begin{proposition}
    Jeśli $L$ jest właściwym splotem (indeks zaczepienia każdej składowej o~resztę splotu jest parzysty), to $\jones_L(i) = (-\sqrt 2)^{n-1}(-1)^{\operatorname{Arf} L}$.
    W przeciwnym razie $\jones_L(i) = 0$.
\end{proposition}

\begin{proof}
    Równość tę pokazał Murakami w~1986 roku (\cite{murakami86}).
\end{proof}

\begin{proposition}
    Niech $G$ będzie pierwszą grupą homologii podwójnego nakrycia $S^3$ rozgałęzionego nad składowymi.
    Jeśli $G$ jest torsyjna, to $\jones_L(-1) = |G|$.
    W przeciwnym razie $\jones_L(-1) = 0$.
\end{proposition}

\begin{proof}
    ???? % TODO
\end{proof}

Nie jest znana topologiczna interpretacja wielomianu Jonesa (którą posiada wielomian Alexandera) ani charakteryzacja poza warunkami koniecznymi z~pięciu faktów powyżej.

\begin{corollary}
    Niech $K$ będzie węzłem.
    Wtedy
    \begin{align}
        \jones(1) & = 1 \\
        \jones(-1) & = \pm \det K \\
        \jones(i) & = \begin{cases}
            1 & \text{dla } \alexander(-1) \equiv \pm 1 \mod 8 \\
            -1 & \text{w przeciwnym razie.}
        \end{cases}
    \end{align}
\end{corollary}

Poza powyżej opisanymi przypadkami, wartości wielomianu Jonesa nie można znaleźć w~czasie wielomianowym od ilości skrzyżowań na diagramie (jest to problem $\#P$-trudny).

% Czemu wielomian Jonesa jest wielomianem?
% Odpowiedniki wielomianu Jonesa dla węzłów w~3-rozmaitościach innych niż sfera $S^3$ nie są wielomianami, ale funkcjami z~pierwiastków jedności w~zbiór elementów całkowitch\footnote{algebraic integers} (jak podaje J. Roberts).

Dotychczas wyznaczyliśmy wielomian Jonesa jedynie dla splotów trywialnych (fakt \ref{prp:jones_trivial_link}).
Dlaczego?
Chociaż klamra Kauffmana to użyteczne narzędzie podczas dowodzenia różnych teoretycznych własności, niezbyt nadaje się do obliczeń, szczególnie ręcznych.
Na szczęście wtedy z pomocą przychodzi:

\begin{definition}
    Niech $L$ będzie zorientowanym splotem.
    Wielomian Laurenta $\jones_L(t) \in \Z[t^{\pm 1/2}]$, który spełnia relację kłębiastą
\index{relacja kłębiasta}%
    \begin{equation}
        t^{-1} \jones(L_+) - t\jones(L_-) + (t^{-1/2} - t^{1/2}) \jones(L_0) = 0
    \end{equation}
    z warunkiem brzegowym $\jones(\SmallUnknot) = 1$, nazywamy wielomianem Jonesa.
\end{definition}

\begin{proof}
Niech
\begin{comment}
\begin{equation}
    L_+ = \MediumMinusCrossing
    \quad\quad
    L_- = \MediumPlusCrossing
    \quad\quad
    L_0 = \MediumAlphaSmoothing
    \quad\quad
    L_\infty = \MediumBetaSmoothing
\end{equation}
\end{comment}
% TODO: zdefiniować je raz, a dobrze.
% ack -l 'L_\+'
% src/30-polynomials/alexander.tex
% src/30-polynomials/jones-kauffman.tex
% src/30-polynomials/blmho.tex
% src/30-polynomials/homfly.tex
% src/00-meta-latex/diagrams.tex
(oznaczenia te są standardowe i pozwalają oszczędzić trochę miejsca).
NIE SĄ - KOLIZJA OZNACZEŃ?Wyraźmy wielomian Jonesa przez klamrę Kauffmana i~spin.
Chcemy pokazać, że
\begin{align}
    & A^{4}(-A)^{-3w(L_+)}\bracket{L_+} \\
    - & A^{-4}(-A)^{-3w(L_-)}\bracket{L_0} \\
    + & (A^2-A^{-2})(-A)^{-3w(L_0)}\bracket{L_0} = 0.
\end{align}

Ale $w(L_\pm) = w(L_0)\pm 1$, zatem to jest równoważne z

\begin{equation}
    -A \bracket{L_+} +
    A^{-1} \bracket{L_-} +
    (A^2-A^{-2}) \bracket{L_0} =0.
\end{equation}

Z definicji klamry Kauffmana wnioskujemy, że

\begin{equation}
    \begin{cases}
        \bracket{L_+} = A\bracket{L_0} + A^{-1}\bracket{L_\infty} \\
        \bracket{L_-} = A\bracket{L_\infty} + A^{-1}\bracket{L_0}
    \end{cases}
\end{equation}

Pierwsze równanie przemnóżmy przez $A$, drugie przez $A^{-1}$, a~następnie dodajmy je do siebie.
Wtedy otrzymamy
\begin{equation}
    A\bracket{L_+} - A^{-1}\bracket{L_-} =
    A^2 (\bracket{L_0} - \bracket{L_\infty}),
\end{equation}
quod erat demonstrandum.
\end{proof}

% \subsection{Odwrotności, lustra i~sumy}
Wielomian Jonesa nie wykrywa orientacji splotu.

\begin{proposition}
    Niech $L$ będzie zorientowanym splotem.
    Wtedy $\jones(rL)=\jones(L)$.
\index{rewers}%
\end{proposition}

\begin{proof}
    Aby obliczyć wielomian rewersu, wykorzystujemy te same diagramy kłębiaste,
    jak dla zwykłego, a~przy tym nie zmieniamy znaku żadnego skrzyżowania.
\end{proof}

Ale czasami potrafi odróżnić splot od jego lustra:

\begin{proposition}
    Niech $L$ będzie zorientowanym splotem.
    Wtedy $\jones(mL)(t)=\jones(L)(t^{-1})$.
\index{lustro}%
\end{proposition}

\begin{proof}
    Zauważmy, że diagramy $L_-$ oraz $L_+$ są wzajemnymi lustrami.
    Dlatego każda relacja kłębiasta dla splotu postaci
    \begin{equation}
        t^{-1} \jones(L_+)(t) - t\jones(L_-)(t) + (t^{-1/2} - t^{1/2}) \jones(L_0)(t) = 0
    \end{equation}
    odpowiada pewnej relacji dla lustra splotu:
    \begin{equation}
        -t\jones(L_+)(t) + t^{-1} \jones(L_-)(t) + (t^{-1/2} - t^{1/2}) \jones(L_0)(t) = 0,
    \end{equation}
    co po zamianie zmiennych $t \mapsto t^{-1}$ i przemnożeniu przez $-1$ daje
    \begin{equation}
        -t^{-1} \jones(L_+)(t^{-1}) + t \jones(L_-)(t^{-1}) + (t^{1/2} - t^{-1/2}) \jones(L_0)(t^{-1}) = 0.
    \end{equation}

    Patrz też: Florian Gellert, Kombinatorische Invarianten, strona 12.
\end{proof}

\begin{corollary}
    \label{cor:joines_of_amphicheiral}
    Jeśli $K$ jest węzłem zwierciadlanym, to wielomian $\jones_K$ jest symetryczny.
\index{węzeł!zwierciadlany}
\end{corollary}

Implikacja odwrotna nie zachodzi na mocy wniosku \ref{cor:acheiral_signature}: węzeł $9_{42}$ ma symetryczny wielomian Jonesa, ale niezerową sygnaturę.
\index{sygnatura}%
Poniżej trzynastu skrzyżowań taka sytuacja ma miejsce dla dokładnie czternastu węzłów pierwszych.
% 9_42, 10_125, 11n_19, 11n_24, 11n_82, 12a_0669, 12a_1171, 12a_1179, 12a_1205, 12n_0362, 12n_0506, 12n_0562, 12n_0571, 12n_0821

Równość $\jones(mL)(t)=\jones(L)(t^{-1})$ nie jest spełniona dla trójlistnika, zatem ten nie jest równoważny ze swoim lustrem.
Wcześniej pokazał to z~dużo większym wysiłkiem Dehn, patrz przykład \ref{exm:trefoil_is_chiral}.

\begin{corollary}
    Wielomian Jonesa nie zależy od orientacji węzła.
    Nie jest to prawdą dla splotów.
\end{corollary}

\begin{proof}
    Każdy węzeł ma tylko dwie orientacje, splot może mieć ich $2^n$, gdzie $n$ to liczba składowych.
\end{proof}

\begin{proposition}
    \label{prp:jones_multiplicative_1}
    Niech $L_1, L_2$ będą zorientowanymi splotami.
    Wtedy
    \begin{equation}
        \jones(L_1 \sqcup L_2) = (-t^{1/2} - t^{-1/2}) \jones(L_1) \jones(L_2).
    \end{equation}
\end{proposition}

\begin{proof}
    Wybierzmy diagramy $D_1, D_2$ dla splotów $L_1, L_2$.
    Po podstawieniu $t^{1/2} = A^{-2}$ widzimy, że chcemy pokazać
    \begin{equation}
        (-A)^{-3w(D_1 \sqcup D_2)} \langle D_1 \sqcup D_2 \rangle
        =
        (-A^2 - A^{-2})(-A)^{-3(w(D_1) + w(D_2))} \langle D_1 \rangle \langle D_2 \rangle.
    \end{equation}

    Oczywiście $w(D_1 \sqcup D_2) = w(D_1) + w(D_2)$, więc wystarczy udowodnić, że
    \begin{equation}
        \langle D_1 \sqcup D_2 \rangle = (-A^2 - A^{-2}) \langle D_1 \rangle \langle D_2 \rangle.
    \end{equation}

    Oznaczmy przez $f_1(D_1)$, $f_2(D_1)$ odpowiednio lewą i~prawą stronę ostatniego równania.
    Są to wielomiany Laurenta, które zależą tylko od $D_1$.
    Aksjomaty Kauffmana pozwalają na pokazanie, że obie funkcje mają następujące własności:
    \begin{align}
        f_i(\SmallUnknot)            & = (-A^2 - A^{-2}) \langle D_2 \rangle \\
        f_i(D_1 \sqcup \SmallUnknot) & = (-A^2 - A^{-2}) f_i(D_1) \\
        f_i(\LittleRightCrossing)     & = A f_i(\LittleRightSmoothing) + A^{-1} f_i(\LittleLeftSmoothing).
    \end{align}
    Ponieważ powyższe tożsamości wystarczają do wyznaczenia wartości funkcji $f_i$ dla dowolnego diagramu $D_1$, dochodzimy do wniosku, że $f_1 \equiv f_2$.
    To kończy dowód.
\end{proof}

\begin{proposition}
\label{prp:jones_multiplicative_2}%
\index{relacja kłębiasta}%
    Niech $K_1, K_2$ będą zorientowanymi węzłami.
    Wtedy
    \begin{equation}
        \jones(K_1 \# K_2) = \jones(K_1) \jones(K_2).
    \end{equation}
\end{proposition}

\begin{proof}
    Rozpatrzmy sploty
\begin{comment}
    \begin{figure}[H]
    \centering
        %
        \begin{minipage}[b]{.3\linewidth}
            \[
                \MediumJonesShrapA
            \]
        \end{minipage}
        %
        \begin{minipage}[b]{.3\linewidth}
            \[
                \MediumJonesShrapB
            \]
        \end{minipage}
        %
        \begin{minipage}[b]{.3\linewidth}
            \[
                \MediumJonesShrapAB
            \]
        \end{minipage}
    \end{figure}
\end{comment}
    Relacja kłębiasta orzeka w tym przypadku, że
    \begin{equation}
        t^{-1} \jones(K_1 \# K_2) - t \jones(K_1 \# K_2) + (t^{-1/2} - t^{1/2}) \jones(K_1 \sqcup K_2) = 0.
    \end{equation}
    Ostatni składnik sumy można rozwinąć na mocy faktu \ref{prp:jones_multiplicative_1}.
    Po uporządkowaniu dostaniemy:
    \begin{equation}
        (t^{-1} - t) \jones(K_1 \# K_2) - (t^{-1} - t) \jones(K_1) \jones(K_2) = 0,
    \end{equation}
    a stąd widać już prawdziwość dowodzonej tezy.
\end{proof}

% Koniec sekcji Relacja kłębiasta
% Koniec podsekcji Wielomian Jonesa

\index{klamra Kauffmana|)}

% Koniec podsekcji Nawias Kauffmana


\subsection{Odróżnianie węzłów i splotów wielomianem Jonesa}
Wielomian Jonesa często (chociaż nie zawsze) odróżnia od siebie sploty lepiej niż wielomian Alexandera.
Na przykład: wielomian Alexandera wszystkich splotów rozszczepialnych jest taki sam (stwierdzenie \ref{prp:alexander_unlinks}), więc nie odróżnia wcale niesplotów.
Dla porównania, wielomian Jonesa odróżnia je wszystkie:

\begin{proposition}
\label{prp:jones_trivial_link}%
    Wielomianem Jonesa splotu trywialnego o $n$ ogniwach jest
    \begin{equation}
        \jones(K_n) = \left(-\sqrt{t} - \frac{1}{\sqrt {t}}\right)^{n-1}.
    \end{equation}
\end{proposition}

Co więcej, wielomian Jonesa odróżnia od siebie dowolne dwa węzły pierwsze o~co najwyżej 9 skrzyżowaniach.
Dalej występują już kolizje, oto pełna ich lista do 10 skrzyżowań:
$5_{1}$ -- $10_{132}$,
$8_{8}$ -- $10_{129}$,
$8_{16}$ -- $10_{156}$,
$10_{22}$ -- $10_{35}$,
$10_{25}$ -- $10_{56}$,
$10_{40}$ -- $10_{103}$,
$10_{41}$ -- $10_{94}$,
$10_{43}$ -- $10_{91}$,
$10_{59}$ -- $10_{106}$,
$10_{60}$ -- $10_{86}$,
$10_{71}$ -- $10_{104}$,
$10_{73}$ -- $10_{83}$,
$10_{81}$ -- $10_{109}$,
$10_{137}$ -- $10_{155}$.
Jones wiedział, że wielomianowe niezmienniki nie radzą sobie z~odróżnianiem od siebie mutantów, dlatego zapytał, czy jego wielomian wykrywa niewęzły.
Pozostaje to otwartym problemem do dziś.

\begin{conjecture}
\index{hipoteza!o wielomianie Jonesa i niewęźle}%
\label{con:jones}%
    Niech $K$ będzie węzłem.
    Jeśli $\jones_K(t) \equiv 1$, to $K$ jest niewęzłem.
\end{conjecture}

Hipotezę zweryfikowano komputerowo dla węzłów o~małej liczbie skrzyżowań.
W latach dziewięćdziesiątych Hoste, Thistlethwaite, Weeks zrobili to przy okazji tablicowania węzłów spełniających $\crossing K \le 16$.
Wynik poprawiano: Dasbach, Hougardy w~1997 do $\crossing K \le 17$; Yamada w~2000 do $\crossing K \le 18$; wreszcie Tuzun, Sikora w~2016 do $\crossing K \le 22$, potem w~2020 do $\crossing K \le 24$.
Patrz kolejno \cite{thistlethwaite98}, \cite{hougardy97}, \cite{yamada00}, \cite{tuzun18}, \cite{tuzun21}, ale też \cite[s. 381]{ohtsuki02}.

% TODO: Argumentem przemawiającym za prawdziwością hipotezy jest twierdzenie ,,udowodnione'' przez Jørgena Andersena.
% TODO: \textbf{NIE Pokazał on, że rodzina okablowanych wielomianów Jonesa wykrywa niewęzeł.}
% TODO: Tutaj $n$-okablowanie węzła $K$ to $n$-komponentowy splot $K^n$, który powstaje z~$K$ po zamianie pojedynczej ,,żyły'' na $n$ równoległych żył.

Istnieją sploty o~trywialnym wielomianie Jonesa.
Thistlethwaite wskazał dwa z~dwoma oraz jeden z~trzema ogniwami w~\cite{thistlethwaite01}.
Jest ich nawet nieskończenie wiele, jak Eliahou, Kauffman i~Thistlethwaite pokazali w~pracy \cite{eliahou03}.

\begin{proposition}
\index{splot!Hopfa}%
    Niech $k \ge 2$ będzie liczbą naturalną.
    Istnieje nieskończenie wiele splotów pierwszych z $k$ ogniwami, których wielomian Jonesa nie odróżnia od niesplotu z $k$ ogniwami.

    Co więcej, można wymagać, by wszystkie te sploty były satelitami splotu Hopfa.
\end{proposition}




\subsection{Relacja kłębiasta}
Dotychczas wyznaczyliśmy wielomian Jonesa jedynie dla splotów trywialnych (fakt \ref{prp:jones_trivial_link}).
Dlaczego?
Chociaż klamra Kauffmana to użyteczne narzędzie podczas dowodzenia różnych teoretycznych własności, niezbyt nadaje się do obliczeń, szczególnie ręcznych.
Na szczęście wtedy z pomocą przychodzi:

\begin{definition}
\index{relacja kłębiasta}%
    Niech $L$ będzie zorientowanym splotem.
    Wielomian Laurenta $\jones_L(t) \in \Z[t^{\pm 1/2}]$, który spełnia relację kłębiastą
    \begin{equation}
        t^{-1} \jones(L_+) - t\jones(L_-) + (t^{-1/2} - t^{1/2}) \jones(L_0) = 0
    \end{equation}
    z warunkiem brzegowym $\jones(\SmallUnknot) = 1$, nazywamy wielomianem Jonesa.
\end{definition}

\begin{proof}
Niech
\begin{comment}
\begin{equation}
    L_+ = \MediumMinusCrossing
    \quad\quad
    L_- = \MediumPlusCrossing
    \quad\quad
    L_0 = \MediumAlphaSmoothing
    \quad\quad
    L_\infty = \MediumBetaSmoothing
\end{equation}
\end{comment}
% TODO: zdefiniować je raz, a dobrze.
% ack -l 'L_\+'
% src/30-polynomials/alexander.tex
% src/30-polynomials/jones-kauffman.tex
% src/30-polynomials/blmho.tex
% src/30-polynomials/homfly.tex
% src/00-meta-latex/diagrams.tex
(oznaczenia te są standardowe i pozwalają oszczędzić trochę miejsca).
NIE SĄ - KOLIZJA OZNACZEŃ?Wyraźmy wielomian Jonesa przez klamrę Kauffmana i~spin.
Chcemy pokazać, że
\begin{align}
    & A^{4}(-A)^{-3w(L_+)}\bracket{L_+} \\
    - & A^{-4}(-A)^{-3w(L_-)}\bracket{L_0} \\
    + & (A^2-A^{-2})(-A)^{-3w(L_0)}\bracket{L_0} = 0.
\end{align}

Ale $w(L_\pm) = w(L_0)\pm 1$, zatem to jest równoważne z
\begin{equation}
    -A \bracket{L_+} +
    A^{-1} \bracket{L_-} +
    (A^2-A^{-2}) \bracket{L_0} =0.
\end{equation}

Z definicji klamry Kauffmana wnioskujemy, że
\begin{align}
    \bracket{L_+} & = A\bracket{L_0} + A^{-1}\bracket{L_\infty} \\
    \bracket{L_-} & = A\bracket{L_\infty} + A^{-1}\bracket{L_0}
\end{align}

Pierwsze równanie przemnóżmy przez $A$, drugie przez $A^{-1}$, a~następnie dodajmy je do siebie.
Wtedy otrzymamy
\begin{equation}
    A\bracket{L_+} - A^{-1}\bracket{L_-} =
    A^2 (\bracket{L_0} - \bracket{L_\infty}),
\end{equation}
quod erat demonstrandum.
\end{proof}




\subsection{Lustra, rewersy. Sumy}
Wielomian Jonesa nie wykrywa orientacji splotu:

\begin{proposition}
\index{rewers}%
    Niech $L$ będzie zorientowanym splotem.
    Wtedy $\jones(rL)=\jones(L)$.
\end{proposition}

\begin{proof}
    Aby obliczyć wielomian rewersu, wykorzystujemy te same diagramy kłębiaste,
    jak dla zwykłego, a~przy tym nie zmieniamy znaku żadnego skrzyżowania.
\end{proof}

Ale czasami potrafi odróżnić splot od jego lustra:

\begin{proposition}
\index{lustro}%
    Niech $L$ będzie zorientowanym splotem.
    Wtedy $\jones(mL)(t)=\jones(L)(t^{-1})$.
\end{proposition}

\begin{proof}
    Zauważmy, że diagramy $L_-$ oraz $L_+$ są wzajemnymi lustrami.
    Dlatego każda relacja kłębiasta dla splotu postaci
    \begin{equation}
        t^{-1} \jones(L_+)(t) - t\jones(L_-)(t) + (t^{-1/2} - t^{1/2}) \jones(L_0)(t) = 0
    \end{equation}
    odpowiada pewnej relacji dla lustra splotu:
    \begin{equation}
        -t\jones(L_+)(t) + t^{-1} \jones(L_-)(t) + (t^{-1/2} - t^{1/2}) \jones(L_0)(t) = 0,
    \end{equation}
    co po zamianie zmiennych $t \mapsto t^{-1}$ i przemnożeniu przez $-1$ daje
    \begin{equation}
        -t^{-1} \jones(L_+)(t^{-1}) + t \jones(L_-)(t^{-1}) + (t^{1/2} - t^{-1/2}) \jones(L_0)(t^{-1}) = 0.
    \end{equation}

    Patrz też: Florian Gellert, Kombinatorische Invarianten, strona 12.
\end{proof}

\begin{corollary}
\index{węzeł!zwierciadlany}
\label{cor:joines_of_amphicheiral}%
    Jeśli $K$ jest węzłem zwierciadlanym, to wielomian $\jones_K$ jest symetryczny.
\end{corollary}

Implikacja odwrotna nie zachodzi na mocy wniosku \ref{cor:acheiral_signature}: węzeł $9_{42}$ ma symetryczny wielomian Jonesa, ale niezerową sygnaturę.
\index{sygnatura}%
Poniżej trzynastu skrzyżowań taka sytuacja ma miejsce dla dokładnie czternastu węzłów pierwszych.
% 9_42, 10_125, 11n_19, 11n_24, 11n_82, 12a_0669, 12a_1171, 12a_1179, 12a_1205, 12n_0362, 12n_0506, 12n_0562, 12n_0571, 12n_0821
% TODO: uwzględnić kod programu, który to znalazł

Równość $\jones(mL)(t)=\jones(L)(t^{-1})$ nie jest spełniona dla trójlistnika, zatem ten nie jest równoważny ze swoim lustrem.
Wcześniej pokazał to z~dużo większym wysiłkiem Dehn, patrz przykład \ref{exm:trefoil_is_chiral}.
\index[persons]{Dehn, Max}%

\begin{corollary}
    Wielomian Jonesa nie zależy od orientacji węzła.
\end{corollary}

Nie jest to prawdą dla splotów.

\begin{proof}
    Każdy węzeł ma tylko dwie orientacje, splot może mieć ich $2^n$, gdzie $n$ to liczba składowych.
\end{proof}

\begin{proposition}
\index{suma niespójna}%
\label{prp:jones_multiplicative_1}%
    Niech $L_1, L_2$ będą zorientowanymi splotami.
    Wtedy
    \begin{equation}
        \jones(L_1 \sqcup L_2) = (-t^{1/2} - t^{-1/2}) \jones(L_1) \jones(L_2).
    \end{equation}
\end{proposition}

\begin{proof}
    Wybierzmy diagramy $D_1, D_2$ dla splotów $L_1, L_2$.
    Po podstawieniu $t^{1/2} = A^{-2}$ widzimy, że chcemy pokazać
    \begin{equation}
        (-A)^{-3w(D_1 \sqcup D_2)} \langle D_1 \sqcup D_2 \rangle
        =
        (-A^2 - A^{-2})(-A)^{-3(w(D_1) + w(D_2))} \langle D_1 \rangle \langle D_2 \rangle.
    \end{equation}

    Oczywiście $w(D_1 \sqcup D_2) = w(D_1) + w(D_2)$, więc wystarczy udowodnić, że
    \begin{equation}
        \langle D_1 \sqcup D_2 \rangle = (-A^2 - A^{-2}) \langle D_1 \rangle \langle D_2 \rangle.
    \end{equation}

    Oznaczmy przez $f_1(D_1)$, $f_2(D_1)$ odpowiednio lewą i~prawą stronę ostatniego równania.
    Są to wielomiany Laurenta, które zależą tylko od $D_1$.
    Aksjomaty Kauffmana pozwalają na pokazanie, że obie funkcje mają następujące własności:
    \begin{align}
        f_i(\SmallUnknot)            & = (-A^2 - A^{-2}) \langle D_2 \rangle \\
        f_i(D_1 \sqcup \SmallUnknot) & = (-A^2 - A^{-2}) f_i(D_1) \\
        f_i(\LittleRightCrossing)     & = A f_i(\LittleRightSmoothing) + A^{-1} f_i(\LittleLeftSmoothing).
    \end{align}
    Ponieważ powyższe tożsamości wystarczają do wyznaczenia wartości funkcji $f_i$ dla dowolnego diagramu $D_1$, dochodzimy do wniosku, że $f_1 \equiv f_2$.
    To kończy dowód.
\end{proof}

\begin{proposition}
\label{prp:jones_multiplicative_2}%
\index{relacja kłębiasta}%
\index{suma spójna}%
    Niech $K_1, K_2$ będą zorientowanymi węzłami.
    Wtedy
    \begin{equation}
        \jones(K_1 \# K_2) = \jones(K_1) \jones(K_2).
    \end{equation}
\end{proposition}

\begin{proof}
    Rozpatrzmy sploty
\begin{comment}
    \begin{figure}[H]
    \centering
        %
        \begin{minipage}[b]{.3\linewidth}
            \[
                \MediumJonesShrapA
            \]
        \end{minipage}
        %
        \begin{minipage}[b]{.3\linewidth}
            \[
                \MediumJonesShrapB
            \]
        \end{minipage}
        %
        \begin{minipage}[b]{.3\linewidth}
            \[
                \MediumJonesShrapAB
            \]
        \end{minipage}
    \end{figure}
\end{comment}
    Relacja kłębiasta orzeka w tym przypadku, że
    \begin{equation}
        t^{-1} \jones(K_1 \# K_2) - t \jones(K_1 \# K_2) + (t^{-1/2} - t^{1/2}) \jones(K_1 \sqcup K_2) = 0.
    \end{equation}
    Ostatni składnik sumy można rozwinąć na mocy faktu \ref{prp:jones_multiplicative_1}.
    Po uporządkowaniu dostaniemy:
    \begin{equation}
        (t^{-1} - t) \jones(K_1 \# K_2) - (t^{-1} - t) \jones(K_1) \jones(K_2) = 0,
    \end{equation}
    a stąd widać już prawdziwość dowodzonej tezy.
\end{proof}




\subsection{Wartości w pierwiastkach jedności}
Niech $\jones$ będzie wielomianem Jonesa splotu $L$ o~$n$ składowych spójności.
Jego wartości w~niektórych pierwiastkach jedności są związane z~innymi niezmiennikami węzłów.
I tak przyjmując oznaczenie $\omega_k = \exp(2\pi i/k)$ mamy

\begin{proposition}
    \label{prp:jones_at_roots_of_unity}
    $\jones_L(\omega_3) = 1$.
\end{proposition}

\begin{proposition}
    $\jones_L(1) = (-2)^{n-1}$.
\end{proposition}

\begin{proof}
\index{człowiek!Jones, Vaughan}%
    Jak wkrótce się przekonamy, to proste wnioski z~relacji kłębiastej.
    Explicite wskazał je Jones w \cite[twierdzenie 14, 15]{jones85}.
\end{proof}

\begin{proposition}
    Pochodna w punkcie $t = 1$ znika: $\jones'_L(1) = 0$.
\end{proposition}

\begin{proof}
    Twierdzenie 16 w \cite{jones85}.
\end{proof}

\begin{proposition}
    $V_L(\omega_6) = \pm i^{n-1} \cdot (\sqrt 3i)^r$, gdzie $r$ jest rangą pierwszej grupy homologii podwójnego rozgałęzionego nakrycia $L$ nad $\Z_3$.
\end{proposition}

\begin{proof}
\index{człowiek!Lipson, Andrew}%
    Znak $\pm$ został wyznaczony przez Lipsona w \cite{lipson86}, praca ta zawiera też odsyłacz do wyprowadzenia reszty wzoru.
\end{proof}

\begin{proposition}
    Liczba trzy-kolorowań splotu $L$ wynosi $3|\jones_L(\omega_6)|^2$.
\end{proposition}

\begin{proof}
\index{człowiek!Przytycki, Józef}%
    Przytycki w \cite{przytycki98}.
\end{proof}

\begin{proposition}
    Jeśli $L$ jest właściwym splotem (indeks zaczepienia każdej składowej o~resztę splotu jest parzysty), to $\jones_L(i) = (-\sqrt 2)^{n-1}(-1)^{\operatorname{Arf} L}$.
    W przeciwnym razie $\jones_L(i) = 0$.
\end{proposition}

\begin{proof}
\index{człowiek!Murakami, ?}%
    Równość tę pokazał Murakami w~1986 roku (\cite{murakami86}).
\end{proof}

\begin{proposition}
    Niech $G$ będzie pierwszą grupą homologii podwójnego nakrycia $S^3$ rozgałęzionego nad składowymi.
    Jeśli $G$ jest torsyjna, to $\jones_L(-1) = |G|$.
    W przeciwnym razie $\jones_L(-1) = 0$.
\end{proposition}

\begin{proof}
    ???? % TODO
\end{proof}

Nie jest znana topologiczna interpretacja wielomianu Jonesa (którą posiada wielomian Alexandera) ani charakteryzacja poza warunkami koniecznymi z~pięciu faktów powyżej.

\begin{corollary}
    Niech $K$ będzie węzłem.
    Wtedy
    \begin{align}
        \jones(1) & = 1 \\
        \jones(-1) & = \pm \det K \\
        \jones(i) & = \begin{cases}
            1 & \text{dla } \alexander(-1) \equiv \pm 1 \mod 8 \\
            -1 & \text{w przeciwnym razie.}
        \end{cases}
    \end{align}
\end{corollary}

Poza powyżej opisanymi przypadkami, wartości wielomianu Jonesa nie można znaleźć w~czasie wielomianowym od ilości skrzyżowań na diagramie (jest to problem $\#P$-trudny).

% Czemu wielomian Jonesa jest wielomianem?
% Odpowiedniki wielomianu Jonesa dla węzłów w~3-rozmaitościach innych niż sfera $S^3$ nie są wielomianami, ale funkcjami z~pierwiastków jedności w~zbiór elementów całkowitch\footnote{algebraic integers} (jak podaje J. Roberts).

% Koniec sekcji Relacja kłębiasta
% Koniec podsekcji Wielomian Jonesa

\index{klamra Kauffmana|)}

% Koniec podsekcji Nawias Kauffmana



\index{wielomian!Jonesa|)}%

\input{30-polynomials/302c-span}


\input{30-polynomials/303-homfly}
\input{30-polynomials/304-blmho}
\section{Wielomian Kauffmana}
\index{wielomian!Kauffmana|(}

Mniej więcej w~tym samym czasie, gdy odkryto wielomian BLM/Ho, Kauffman opisał w~\cite{kauffman90} sposób, jak uogólnić ten niezmiennik do odróżniającego lustra.
Wielomianu $F$ Kauffmana nie należy mylić z~klamrą Kauffmana $\langle \ldots \rangle$!

\begin{definition}[wielomian Kauffmana]
    Niech $L$ będzie zorientowanym splotem, zaś $D$ ustalonym diagramem o~spinie $\writhe D$.
    Istnieje wielomian $\Lambda(L)$ wyznaczony przez relację kłębiastą
    \begin{equation}
        \Lambda \left(\MediumLeftCrossing\right) + 
        \Lambda \left(\MediumRightCrossing\right) = 
        z \cdot \left(
        \Lambda \left(\MediumLeftSmoothing\right) + 
        \Lambda \left(\MediumRightSmoothing\right)
        \right),
    \end{equation}
    który jest niezmienniczy względem II i III ruchu Reidemeistera, spełnia równości:
    \begin{equation}
        \Lambda \left(\MediumReidemeisterIaRight\right) = 
        a \Lambda \left(\MediumReidemeisterIb\right),
        \quad\quad\quad
        \Lambda \left(\MediumReidemeisterIaLeft\right) = 
        \frac 1 a \Lambda \left(\MediumReidemeisterIb\right)
    \end{equation}
    oraz warunek brzegowy $\Lambda(\LittleUnknot) = 1$.
    Wtedy wielomian dwóch zmiennych
    \begin{equation}
        F_L(a, z) = a^{-\writhe D} \Lambda(D),
    \end{equation}
    nazywamy wielomianem Kauffmana.
    Jest niezmiennikiem splotów.
\end{definition}

Jego związki z~wielomianem HOMFLY pozostają nieznane.
Wiemy natomiast, że

\begin{proposition}
\index{wielomian!BLM/Ho}%
    Wielomian Kauffmana uogólnia wielomian BLM/Ho, zgodnie z podstawieniem
    \begin{equation}
        Q(x) = F(1, x).
    \end{equation}
\end{proposition}

\begin{proposition}
\index{wielomian!Jonesa}
    Wielomian Kauffmana uogólnia wielomian Jonesa, zgodnie z podstawieniem
    \begin{equation}
        \jones(t)=F(-t^{-3/4},t^{-1/4}+t^{1/4}).
    \end{equation}
\end{proposition}

Rozpatrzmy relację kłębiastą $\Lambda_+ - \Lambda_- = x(\Lambda_0 - \Lambda_\infty)$.
Prowadzi ona do wielomianu ,,z~Dubrownika'': Kauffman, na pocztówce napisanej do Lickorisha z Dubrownika w~1985 roku, opisał ten wielomian sądząc, że jest to nowy, niezależny od $F$ niezmiennik.

\index{wielomian!Kauffmana|)}

% Koniec sekcji Wielomian Kauffmana

\section{Niezmienniki Wasiljewa}
\index{niezmiennik!Wasiljewa|(}
\label{sec:vassiliev}

% DICTIONARY;finite type ...;... skończonego typu;niezmiennik
Niezmienniki Wasiljewa (znane także jako niezmienniki skończonego typu) umożliwiają radykalnie nowe podejście do węzłów.
Około 1989 roku odkryli je niezależnie Wiktor Wasiljew, w~oparciu o~teorię osobliwości, oraz Michaił Gusarow, metodami kombinatorycznymi.
\index[persons]{Wasiljew, Wiktor}%
\index[persons]{Gusarow, Michaił}%
% Wasiljew: V. A. Vassiliev, Cohomology of knot spaces, Theory of Singularities and its Applications (ed. V. I. Arnold), Advances in Soviet Math., 1 (1990) 23-69.
My przedstawimy to drugie podejście.
Niezmienniki Wasiljewa najlepiej zrozumieć, jeśli osłabimy trochę definicję węzła i będziemy rozważać także krzywe z~samoprzecięciami.

Uważam, że przyjemnym wprowadzeniem do niezmienników Wasiljewa jest artykuł \cite{chmutov12} Czmutowa.
\index[persons]{Czmutow, Siergiej}%
Za najlepsze uważa się pracę \cite{barnatan_95}: Bar-Natan, jej autor, pozostaje do dziś jedną z~najważniejszych osób dla rozwoju tego działu matematyki.
\index[persons]{Bar-Natan, Dror}%
Oprócz tego jest jeszcze pięknie zilustrowana książka \cite{duzhin12} dwóch rosyjskich i jednego nierosyjskiego matematyka, będąca obszernym kompendium wiedzy o~węzłach osobliwych.
\index[persons]{Dużin, Siergiej}%
\index[persons]{Mostovoy, Jacob}%

Szkielet sekcji oparliśmy o piętnasty rozdział podręcznika Murasugiego, \cite{murasugi96}.

% DICTIONARY;singular;osobliwy;węzeł
\begin{definition}[węzeł osobliwy]
\index{węzeł!osobliwy}%
    Niech $f \colon S^1 \to \R^3$ będzie kawałkami liniową funkcją, która jest różnowartościowa poza skończenie wieloma punktami.
    Załóżmy dodatkowo, że
    \begin{itemize}
        \item włókna -- przeciwobrazy punktów -- funkcji $f$ są co najwyżej dwuelementowe;
        \item jeżeli dwa różne punkty mają ten sam obraz, to węzeł $K = f(S^1)$ przecina się tam pod kątem prostym.
    \end{itemize}
    Obraz funkcji $f$ nazywamy węzłem osobliwym.
\end{definition}

Gdy używamy słowa węzeł, nigdy nie mamy na myśli węzła osobliwego.
Punkt, gdzie węzeł osobliwy tnie samego siebie, nazywamy wierzchołkiem i~oznaczamy pogrubioną kropką na diagramie.
\index{wierzchołek (węzła osobliwego)}%

\begin{comment}
\begin{figure}[H]
    \centering
    \begin{minipage}[b]{.14\linewidth}
        \centering
        \includegraphics[width=\linewidth]{../data/virtual_0_1.png}
        \subcaption{$0_1$}
    \end{minipage}
    \begin{minipage}[b]{.14\linewidth}
        \centering
        \includegraphics[width=\linewidth]{../data/virtual_2_1.png}
        \subcaption{$2_1$}
    \end{minipage}
    \begin{minipage}[b]{.14\linewidth}
        \centering
        \includegraphics[width=\linewidth]{../data/virtual_3_1.png}
        \subcaption{$3_1$}
    \end{minipage}
    \begin{minipage}[b]{.14\linewidth}
        \centering
        \includegraphics[width=\linewidth]{../data/virtual_3_7.png}
        \subcaption{$3_7$}
    \end{minipage}
    \begin{minipage}[b]{.14\linewidth}
        \centering
        \includegraphics[width=\linewidth]{../data/virtual_4_1.png}
        \subcaption{$4_1$}
    \end{minipage}
    \begin{minipage}[b]{.14\linewidth}
        \centering
        \includegraphics[width=\linewidth]{../data/virtual_4_108.png}
        \subcaption{$4_{108}$}
    \end{minipage}
    \caption{Garść węzłów osobliwych o małej liczbie skrzyżowań}
\end{figure}
\end{comment}

Podamy teraz definicję równoważności dla węzłów osobliwych, zupełnie analogicznie do definicji \ref{def:equivalent_knots_2} dla zwykłych węzłów.

\begin{definition}[płaski dysk]
    Niech $A$ będzie wierzchołkiem węzła osobliwego $K$, $B_A$ jego małym domkniętym otoczeniem, zaś $P_A$ płaszczyzną, która zawiera $B_A \cap K$, mały fragment węzła wokół wierzchołka.
    Dysk $P_A \cap B_A$ nazywamy płaskim dyskiem wokół wierzchołka $A$.
\end{definition}
% Murasugi

\begin{definition}
    Dwa osobliwe węzły $K, L$ są równoważne, co zapisujemy jako $K = L$, jeśli istnieje zachowujący orientację homeomorfizm $\varphi$ przestrzeni $\R^3$ w~siebie, który przenosi jeden węzeł na drugi: $\varphi(K) = L$ oraz indukuje bijekcję między rodzinami płaskich dysków $K$ i~$L$.
\end{definition}
% Murasugi

Ten drugi warunek gwarantuje nam, że skrzyżowania wokół podwójnych punktów nie ulegną zniszczeniu.
Istnieje podobne kryterium dla diagramów, odpowiednik twierdzenia Reidemeistera.

\begin{proposition}
\index{ruch!$\Omega$}%
\index{ruch!Reidemeistera}%
\index{twierdzenie!Reidemeistera}%
    Dwa osobliwe diagramy są równoważne dokładnie wtedy, kiedy można między nimi przejść przy użyciu ciągu izotopii otaczających, ruchów Reidemeistera, operacji $\Omega_4$ oraz $\Omega_5$.
\begin{comment}
    \begin{figure}[H]
    \centering
    \begin{minipage}[b]{.45\linewidth}
        \[
            \LargeVirtualReidemeisterThreeA \cong \LargeVirtualReidemeisterThreeB
        \]
        \subcaption{ruch $\Omega_{4a}$}
    \end{minipage}
    \begin{minipage}[b]{.45\linewidth}
        \[
            \LargeVirtualReidemeisterThreeC \cong \LargeVirtualReidemeisterThreeD
        \]
        \subcaption{ruch $\Omega_{4e}$}
    \end{minipage}
    \caption{Dwa osobliwe warianty III ruchu Reidemeistera}
    \end{figure}
    \begin{figure}[H]
        \[
            \LargeVirtualFlypeFiveA \cong \LargeVirtualFlypeFiveB
        \]
    \caption{Osobliwy wariant ruchu flype, $\Omega_5$}
    \end{figure}
\end{comment}
\end{proposition}

Uwaga: ufamy pracy Nelsona, Oyamaguchiego, Sazdanovic \cite{sazdanovic19} gdzie pokazano, że ten zestaw ruchów jest minimalny.
\index[persons]{Nelson, Sam}%
\index[persons]{Oyamaguchi, Natsumi}%
\index[persons]{Sazdanovic, Radmila}%
Natomist Murasugi podaje tylko jedną nową operację, wydaje nam się, że wszyscy nie mogą mieć racji.

Załóżmy teraz, że mamy jakiś niezmiennik węzłów $v$ o~wymiernych wartościach i~chcemy przedłużyć go do niezmiennika $\hat v$ węzłów osobliwych.
Najprościej zrobić to rekurencyjnie.
Niech $\hat v$ będzie już określony dla osobliwych węzłów o co najwyżej $n - 1$ wierzchołkach i~wybierzmy dowolny węzeł i diagram o~$n$ wierzchołkach.
Okazuje się, że jeżeli położymy

\begin{comment}
\begin{equation}
    \hat v\left( \MediumSingularCrossingArrows \right) =
    \hat v\left( \MediumPlusCrossingArrows \right) -
    \hat v\left( \MediumMinusCrossingArrows \right),
\end{equation}
\end{comment}
to dostaniemy dobrze określoną funkcję: jeżeli $L = K$ jest tym samym osobliwym węzłem z~innym diagramem, to $\hat v(L) = \hat v(K)$.
Funkcję $\hat v$ nazywamy niezmiennikiem osobliwym indukowanym przez niezmiennik węzłów $v_0$.

\begin{definition}[rząd niezmiennika]
\label{def:vassiliev_order}%
\index{rząd niezmiennika (osobliwego)}%
    Niech $v$ będzie niezmiennikiem osobliwym.
    Mówimy, że $v$ jest niezmiennikiem Wasiljewa rzędu co najwyżej $n$, jeśli dla dowolnego osobliwego węzła $K$ o $n + 1$ wierzchołkach zachodzi $v(K) = 0$.
    Jeśli dodatkowo $v$ nie jest rzędu co najwyżej $n - 1$, to mówimy, że jest rzędu dokładnie $n$.
\end{definition}




\subsection{Niezmienniki Wasiljewa małych rzędów}
Pokażemy najpierw, jakie są niezmienniki Wasiljewa rzędu 0, 1, 2.

\begin{proposition}
    Każdy niezmiennik Wasiljewa rzędu 0 jest funkcją stałą.
\end{proposition}

\begin{proof}
    Niech $v$ będzie niezmiennikiem rzędu zero i~znika na każdym osobliwym węźle o~jednym wierzchołku.
    Relacja kłębiasta mówi, że $v(\SmallPlusCrossing) = v(\SmallMinusCrossing)$, to znaczy odwrócenie dowolnego skrzyżowania nie zmienia wartości niezmiennika.

    Z lematu \ref{lem:unknotting_well_defined} wiemy jednak, że każdy węzeł można zmienić w niewęzeł odwracając niektóre skrzyżowania.
    Wynika stąd, że $v(K) = v(\SmallUnknot)$ dla każdego osobliwego węzła $K$, co należało udowodnić
\end{proof}

Każda zespolona krotność niezmiennika Wasiljewa znowu jest takim niezmiennikiem.
Niezmienniki rzędu zero tworzą przestrzeń liniową wymiaru 1 nad ciałem $\C$, zatem możemy krótko (choć nie dokładnie) powiedzieć, że jest jeden niezmiennik rzędu zero.

\begin{proposition}
    Nie istnieje niezmiennik Wasiljewa rzędu 1.
\end{proposition}

\begin{proposition}
    Istnieje dokładnie jeden niezmiennik Wasiljewa rzędów 2 i 3.
\end{proposition}

\begin{proof}
    Murasugi pokazuje to przy użyciu diagramów cięciw, które wprowadzimy później.
    Patrz \cite[s. 315-320]{murasugi96}.
\end{proof}



Czas na garść przykładów niezmienników skończonego typu oraz niezmienników, które nie są skończonego typu.


\subsection{Niezmienniki, które są skończonego typu}

Polyak, Viro \cite{polyak01} piszą, że jedyny niezmiennik Wasiljewa rzędu 2, to $v_2 = \frac 12 \alexander''(1)$ i~że występuje pod nazwą ,,niezmiennik węzłów Cassona''.
\index[persons]{Polyak, Michael}%
\index[persons]{Viro, Oleg}%
\index{niezmiennik Cassona}%
Modulo $2$ to niezmiennik Arfa.
\index{niezmiennik Arfa}%

\begin{example}
\index{wielomian!Conwaya}%
    Niech $K$ będzie węzłem, zaś $\conway_K(t) = \sum_k \conway_{2k} z^{2k}$ jego wielomianem Conwaya.
    Współczynnik $\conway_{2k}$ indukuje niezmiennik Wasiljewa rzędu dokładnie $2k$.
\end{example}

\begin{proof}
\index[persons]{Czmutow, Siergiej}%
\index[persons]{Bar-Natan, Dror}%
    Jak wspomina Czmutow w \cite{chmutov12}, porównanie relacji kłębiastej dla wielomianu Conwaya z tą dla niezmienników Wasiljewa pokazuje, że wielomian Conwaya osobliwego węzła o~$m$ punktach podwójnych jest podzielny przez $z^m$, co dowodzi już, że $c_{2k}$ jest niezmiennikiem rzędu co najwyżej $2k$.

    Bar-Natan w 1991 pokazał, że $\conway_{2k}$ jest niezmiennikiem rzędu dokładnie $2k$.
\end{proof}

Lin oraz Wang \cite{wang96} w~1994 roku na podstawie niezmienników małych rzędów, to jest $v_2$ oraz $v_3$, wysunęli następującą hipotezę: istnieje uniwersalna stała $C$ taka, że
\index[persons]{Lin, Xiao-Song}%
\index[persons]{Wang, Zhenghan}%
\index{hipoteza!Lin-Wanga}%

\begin{equation}
    |v_k(K)| \le C (\crossing K)^k.
\end{equation}

Hipotezę wkrótce udowodniono, najpierw dla węzłów (Bar-Natan, \cite{barnatan95}), nieco później także dla splotów (Stojmenow, \cite{stoimenow001}).
\index[persons]{Bar-Natan, Dror}%
\index[persons]{Stojmenow, Aleksander}%
Wartość stałej $C$ trudno obliczyć, dlatego Stojmenow zaproponował \cite[problem 1.17]{ohtsuki02} ograniczenie się do przypadku $v_k = \conway_k$.

\begin{conjecture}
    Niech $L$ będzie splotem.
    Wtedy
    \begin{equation}
        |\conway_k(L)| \le \frac{(\crossing L)^k}{2^kk!}.
    \end{equation}
\end{conjecture}

Nierówność jest nietrywialna tylko dla splotu $L$ z~$k+1, k-1, \ldots$ składowymi; trywialna dla $k = 0$, łatwa dla $k=1$ (wtedy $\conway_1$ jest indeksem zaczepienia splotów o~dwóch składowych) oraz udowodniona dla węzłów i~$k=2$ przez Polyaka, Viro w~2001 (\cite{polyak01}).
\index[persons]{Polyak, Michael}%
\index[persons]{Viro, Oleg}%

\begin{example}
\index{wielomian!Jonesa}%
    Niech $\jones_K(t)$ będzie wielomianem Jonesa węzła $K$.
    Dokonajmy podstawienia
    \begin{equation}
        t := e^x = 1 + x + \frac{x^2}{2} + \frac{x^3}{6} + \ldots,
    \end{equation}
    a~następnie rozwińmy wynik w~szereg Taylora:
    \begin{equation}
        \jones_K(e^x) = \sum_{k = 0}^\infty b_k x^k.
    \end{equation}
    Współczynnik $b_{k}$ indukuje niezmiennik Wasiljewa rzędu co najwyżej $k$.
\end{example}

Ten i podobne wyniki dla wielomianów HOMFLY, Kauffmana uzyskała Birman z~Linem w~\cite{birman93}, gdzie znacznie uprościli oryginalne techniki Wasiljewa.
\index[persons]{Birman, Joan}%
\index[persons]{Lin, Xiao-Song}%
Patrz też \cite[s. 56]{chmutov12}.
\index[persons]{Czmutow, Siergiej}%

\begin{example}[s. 311 w \cite{murasugi96}]
    Niech $K$ będzie węzłem, zaś $f(t)$ rozwinięciem Taylora wokół $t = 1$ dla wielomianu Jonesa:
    \begin{equation}
        f(t) = \sum_{k = 0}^\infty c_k (t-1)^k.
    \end{equation}
    Współczynnik $c_{k}$ indukuje niezmiennik Wasiljewa rzędu co najwyżej $k$.
\end{example}




\subsection{Niezmienniki, które nie są skończonego typu}
Pójdźmy teraz w~ślad za Murasugim i~zdefiniujmy nieskończoną rodzinę węzłów wirtualnych $K[p, q]$, gdzie $p$ jest liczbą wierzchołków, zaś $|q|$ liczbą klasycznych skrzyżowań.
Jeśli $q < 0$, wszystkie skrzyżowania odwracamy:
\begin{figure}[H]
    \centering
\begin{comment}
    \begin{tikzpicture}[baseline=-0.65ex, scale=0.1]
    \begin{knot}[clip width=5, end tolerance=1pt, flip crossing/.list={2}]
        % left part
        \draw[thick] (5, 0) [in=-60, out=-120] to (-5, 0) [in=60, out=120] to (-15, 0) [in=-60, out=-120] to (-25, 0) [in=60, out=120] to (-35, 0) [in=180, out=-120] to (-35, -10);
        \draw[thick] (5, 0) [in=60, out=120] to (-5, 0) [in=-60, out=-120] to (-15, 0) [in=60, out=120] to (-25, 0) [in=-60, out=-120] to (-35, 0) [in=-180, out=120] to (-35, 10);
        % right part
        \strand[thick] (5, 0) [in=120, out=60] to (15, 0) [in=-120, out=-60] to (25, 0) [in=120, out=60] to (35, 0) [in=0, out=-60] to (35, -10);
        \strand[thick] (5, 0) [in=-120, out=-60] to (15, 0) [in=120, out=60] to (25, 0) [in=-120, out=-60] to (35, 0) [in=0, out=60] to (35, 10);
        % external lines
        \draw[thick,Latex-] (-35, 10) to (35, 10);
        \draw[thick,Latex-] (-35, -10) to (35, -10);
        \draw[black,fill=black] (5,0) circle (0.5);
        \draw[black,fill=black] (-5,0) circle (0.5);
        \draw[black,fill=black] (-15,0) circle (0.5);
        \draw[black,fill=black] (-25,0) circle (0.5);
        \draw[black,fill=black] (-35,0) circle (0.5);
    \end{knot}
    \end{tikzpicture}
\end{comment}
    \caption{Węzeł osobliwy $K[p, q]$ dla $p = 2, q = 10$.}
\end{figure}

\begin{proposition}
\index{genus}%
\index{indeks skrzyżowaniowy}%
\index{indeks warkoczowy}%
\index{liczba gordyjska}%
\index{liczba mostowa}%
\index{sygnatura}%
    Następujące funkcje nie są niezmiennikami Wasiljewa: indeks skrzyżowaniowy $\crossing$, liczba gordyjska $\unknotting$, liczba mostowa $\bridge$, indeks warkoczowy $\braid$, genus $g$, sygnatura $\sigma$.
\end{proposition}

Wynik był znany już w latach 90., na przykład Birman, Lin \cite{birman93} pokazali, że liczba gordyjska nie jest niezmiennikiem Wasiljewa.
\index[persons]{Birman, Joan}%
\index[persons]{Lin, Xiao-Song}%

\begin{proof}
    Dowód dla sygnatury jest w~podręczniku Murasugiego \cite[s. 312]{murasugi96}.
    Natomiast żaden z~pozostałych niezmienników nie znika na osobliwym węźle $K[n+1, n]$.
\end{proof}




\subsection{Diagramy cięciw, układy ciężarów, algebra chińskich znaków}

Okazuje się, że wartość niezmiennika Wasiljewa $v$ nie zależy wprost od tego, jak zaplątany jest węzeł osobliwy $K$, ale od tego, jak ułożone są wierzchołki wzdłuż węzła. Standardową metodą kodowania tej informacji jest diagram cięciw.

\begin{definition}[diagram cięciw]
% DICTIONARY;chord;cięciw;diagram
\index{diagram cięciw}%
    Zorientowany okrąg razem z~$2n$ punktami leżącymi na nim (oraz~połączonymi w pary) z~dokładnością do zachowujących orientację homeomorfizmów nazywamy diagramem cięciw rzędu $n$, albo stopnia $n$.
\end{definition}

\begin{comment}
\begin{figure}[H]
    \centering
    \begin{minipage}[b]{.18\linewidth}
        \[\LargeChordDiagramA\]
        \subcaption{}
    \end{minipage}
    \begin{minipage}[b]{.18\linewidth}
        \[\LargeChordDiagramB\]
        \subcaption{}
    \end{minipage}
    \begin{minipage}[b]{.18\linewidth}
        \[\LargeChordDiagramC\]
        \subcaption{}
    \end{minipage}
    \begin{minipage}[b]{.18\linewidth}
        \[\LargeChordDiagramD\]
        \subcaption{}
    \end{minipage}
    \begin{minipage}[b]{.18\linewidth}
        \[\LargeChordDiagramE\]
        \subcaption{}
    \end{minipage}
    \caption{Wszystkie pięć diagramów cięciw stopnia 3}
\end{figure}
\end{comment}

Jak zamienić węzeł osobliwy w diagram cięciw?
Wybierzmy dowolny punkt na węźle, różny od wierzchołka i~przemierzmy węzeł.
Mijanym skrzyżowaniom przypiszmy liczby $1, 2, \ldots, 2n$.
Następnie na okręgu zaznaczmy kolejno te same punkty $1, 2, \ldots 2n$.
Wreszcie połączmy ze sobą liczby, które występują na tych samych skrzyżowaniach.

% TODO: wstawić obrazek.

\begin{proposition}
    Niech $K_1, K_2$ będą dwoma osobliwymi węzłami o~tym samym diagramie cięciw, zaś $v$~niezmiennikiem Wasiljewa.
    Wtedy $v(K_1) = v(K_2)$.
\end{proposition}

\begin{proof}
    Umieśćmy węzły osobliwe $K_1, K_2$ w przestrzeni tak, by ich wierzchołki oraz obie gałęzie wychodzące z wierzchołków leżały tak samo. Wtedy można tak zdeformować łuki $K_1$ tak, by jedynymi osobliwościami, jakie się pojawią lub znikną, były podwójne punkty.
    Teraz kłębiasta Wasiljewa mówi, że wartość $v$ nie zmienia się podczas tego procesu, zatem $v(K_1) = v(K_2)$, co należało okazać.
    % chmutov12
    % TODO: (\cite{duzhin12}, prop. 3.4.2)
\end{proof}

\begin{definition}[symbol niezmiennika]
\index{symbol niezmiennika (osobliwego)}%
    Niech $v$ będzie niezmiennikiem Wasiljewa.
    Obcięcie $v$ do zbioru węzłów osobliwych o~dokładnie $n$ wierzchołkach traktowane jako funkcja ze zbioru diagramów cięciw nazywamy symbolem tego niezmiennika.
\end{definition}

Jeśli $v_1, v_2$ są niezmiennikami Wasiljewa rzędu co najwyżej $n$ o~tych samych symbolach, to ich różnica jest niezmiennikiem rzędu co najwyżej $n - 1$.
Oznacza to, że przestrzeń $\mathcal V_n/\mathcal V_{n-1}$ pokrywa się z przestrzenią wszystkich symboli niezmienników Wasiljewa rzędu co najwyżej $n$.
Zbiór diagramów cięciw rzędu $n$ jest skończony, więc przestrzeń funkcji na tym zbiorze też jest skończona, a zatem przestrzenie $\mathcal V_n$ są skończonego wymiaru.

Stojmenow w \cite{stoimenow001} znalazł jakościowy wynik bez praktycznego znaczenia (ponieważ już dla $k = 4$ musielibyśmy znać wszystkie węzły o 36 skrzyżowaniach, a~jeszcze ich nie znamy, stan na 2021 rok).
\index[persons]{Stojmenow, Aleksander}%
Dokładniej:

\begin{proposition}
    Niezmiennik Wasiljewa rzędu co najwyżej $k$ jest jednoznacznie określony przez wartości, jakie przyjmuje na alternujących węzłach o co najwyżej $2k^2 + k$ skrzyżowaniach.
\end{proposition}

Symbol nie jest byle jaką funkcją, spełnia dwie relacje:
\index{relacje 1T i 4T}%
\begin{comment}
\begin{figure}[H]
    \[
        \LargeOneTerm \mapsto 0
    \]
    \caption{Relacja ,,one-term'' (1T albo FI?)}
\end{figure}
oraz
\begin{figure}[H]
    \[
        \LargeFourTermA - \LargeFourTermB + \LargeFourTermC - \LargeFourTermD \mapsto 0.
    \]
    \caption{Relacja ,,four-term'' (4T)}
\end{figure}
\end{comment}

Diagramy mogą mieć więcej cięciw z końcami tam, gdzie linia jest kropkowana, natomiast wszystkie końce cięciw na czarnych, pogrubionych łukach zostały zaznaczone explicite.

Zastosowaliśmy tutaj mały skrót dla oszczędności miejsca: oczywiście nie umiemy jeszcze odejmować od siebie diagramów, dlatego powyższe relacje należy rozumieć tak, że na każdym diagramie liczymy symbol niezmiennika i porównujemy tak otrzymane liczby zespolone.

\begin{definition}[układ ciężarów]
% DICTIONARY;weight system;układ ciężarów;-
\index{układ ciężarów}%
    Funkcję określoną na zbiorze diagramów $n$ cięciw, która spełnia relacje 1T oraz 4T, nazywamy układem ciężarów.
\end{definition}

Okazuje się, że wszystkie zależności, jakie występują między niezmiennikami Wasiljewa, są konsekwencjami relacji 1T oraz 4T.
Mówi o~tym głębokie twierdzenie Koncewicza:

\begin{proposition}
    Każdy układ ciężarów jest symbolem pewnego niezmiennika Wasiljewa. % rzędu co najwyżej $n$ - nie mieści się, przenosi samo $n$ do nowej linii.
\end{proposition}

\begin{proof}
    Koncewicz w \cite{kontsevich93}. % chmutow12/chmutov11 theorem 3.4
\end{proof}

% DICTIONARY;actuality table;tablica rzeczywistości;-
\index{tablica rzeczywistości}
Z tego, co napisaliśmy wyżej wynika, że wszystkie informacje o niezmienniku Wasiljewa można zakodować w~postaci tak zwanej ,,tablicy rzeczywistości''.
Dla każdego diagramu cięciw wybiera się reprezentanta, węzeł osobliwy, oraz podaje wartość niezmiennika na tym węźle.
Jest to pięknie zilustrowane w \cite[sekcja 3.7]{duzhin12}.

\begin{definition}[chiński znak]
    Spójny graf złożony z pojedynczego zorientowanego okręgu oraz pewnej liczby niezorientowanych, kreskowanych linii, które mogą się spotykać w~jednym z dwóch typów wierzchołków:
    \begin{itemize}
        \item wewnętrznych wierzchołkach, gdzie spotykają się trzy kreskowane linie;
        \item zewnętrznych wierzchołkach, gdzie kreskowane linie kończą się na okręgu.
    \end{itemize}
    Wierzchołki wewnętrzne są zorientowane, zgodnie lub przeciwnie do ruchu wskazówek zegara.
\end{definition}

Diagramy cięciw modulo relacja 4T jest tym samym, co algebra chińskich znaków modulo relacja STU.
% DICTIONARY;algebra of Chinese characters;algebra chińskich znaków;-
\index{algebra!chińskich znaków}%
W~tej drugiej spełnione są jeszcze relacje AS oraz IHX, nie mam siły tego rysować, ale wszystko można znaleźć w pracy Bar-Natana \cite{barnatan_95}.
\index[persons]{Bar-Natan, Dror}%




\subsection{Niezmienniki Wasiljewa wyższych rzędów}
(Na podstawie początku pracy Czmutowa \cite{chmutov12}).
\index[persons]{Czmutow, Siergiej}%
Niech $\mathcal V_n$ będzie zbiorem niezmienników Wasiljewa rzędu co najwyżej $n$, o~wartościach w zbiorze liczb zespolonych $\C$.
Z definicji \ref{def:vassiliev_order} wynika, że $\mathcal V_n$ jest przestrzenią wektorową nad ciałem $\C$ oraz $\mathcal V_n \subseteq \mathcal V_{n+1}$ i mamy rosnącą filtrację
\begin{equation}
    \mathcal V_0 \subseteq \mathcal V_1 \subseteq \mathcal V_2 \subseteq \ldots \subseteq \mathcal V := \bigcup_{n=0}^\infty \mathcal V_n.
\end{equation}

Oznaczmy wymiar przestrzeni $\mathcal V_n / \mathcal V_{n-1}$ przez $d_n$.
Dla wyższych rzędów nie dość, że nie znamy dokładnych wartości ciągu $d_n$, to dolne i górne ograniczenia asymptotyczne są od siebie bardzo różne: górne jest niemalże silnią, dolne natomiast jest podwykładnicze.

% https://people.math.osu.edu/chmutov.1/talks/2015/talk-KinW-XL-2015.pdf ? strona 22
% Chmutov, Duzhin. Mostovoy - Introduction to Vassiliev Knot Invariants, strona 432
\begin{proposition}
    $d_n < (2n-1)!!$.
\end{proposition}

\begin{proof}
\index[persons]{Czmutow, Siergiej}%
\index[persons]{Dużin, Siergiej}%
    Czmutow, Dużin \cite{duzhin94}.
\end{proof}

\begin{proposition}
    $d_n < (n-1)!$.
\end{proposition}

\begin{proof}
\index[persons]{Czmutow, Siergiej}%
\index[persons]{Dużin, Siergiej}%
    Czmutow, Dużin \cite{chmutovduzhin94}.
\end{proof}

\begin{proposition}
    $d_n < \frac 12 (n-2)!$.
\end{proposition}

\begin{proof}
\index[persons]{Ng, Ka}%
    Ng \cite{ng98}.
\end{proof}

\begin{proposition}
    Ciąg $d_n$ rośnie wolniej niż $n! \cdot (11/10)^n$.
\end{proposition}

\begin{proof}
\index[persons]{Stojmenow, Aleksander}%
    Stojmenow \cite{stoimenow98}.
\end{proof}

\begin{proposition}
    $d_n \lesssim n! / (2 \log 2 + O(1))^n$.
\end{proposition}

\begin{proof}
\index[persons]{Bollobás, Béla}%
\index[persons]{Riordan, Oliver}%
    Bollobás, Riordan \cite{bollobas00}.
\end{proof}

\begin{proposition}
    Niech $a < \frac 1 6 \pi^2$ będzie stałą.
    Wtedy
    \begin{equation}
        \dim \mathcal V_n / \mathcal V_{n-1} \lesssim \frac{n!}{a^n}.
    \end{equation}
\end{proposition}

\begin{proof}
\index[persons]{Zagier, Don}%
    Zagier \cite{zagier01} znalazł to ograniczenie przy użyciu szeregów Dirichleta.
\end{proof}

Zanim przejdziemy do ograniczeń z dołu, zdefinujmy jeszcze jedną przestrzeń, $\mathcal P_n \subseteq \mathcal V_n$.
Składa się z~tych niezmienników Wasiljewa, które są jednocześnie morfizmami, to znaczy spełniają równość $v(K_1 \shrap K_2) = v(K_1) + v(K_2)$.
Każdy niezmiennik jest wielomianową kombinacją niezmienników pierwotnych (elementów $\mathcal P_n$).

% https://people.math.osu.edu/chmutov.1/talks/2015/talk-KinW-XL-2015.pdf ? strona 32
% Chmutov, Duzhin. Mostovoy - Introduction to Vassiliev Knot Invariants, strona 434
\begin{proposition}
    $\dim \mathcal P_n \ge 1$.
\end{proposition}

\begin{proof}
\index[persons]{Czmutow, Siergiej}%
\index[persons]{Dużin, Siergiej}%
\index[persons]{Lando, Siergiej}%
    Czmutow, Dużin, Lando \cite{duzhin94}.
\end{proof}

\begin{proposition}
    $\dim \mathcal P_n \ge [n/2]$.
\end{proposition}

\begin{proof}
\index[persons]{Czmutow, Siergiej}%
\index[persons]{Melvin, Paul}%
\index[persons]{Morton, Hugh}%
\index[persons]{Warczenko, Aleksander}%
    Melvin, Morton \cite{melvin95}, Czmutow, Warczenko \cite{varchenko97}.
\end{proof}

\begin{proposition}
    $\dim \mathcal P_n \gtrsim \frac{1}{96} n^2$.
\end{proposition}

\begin{proof}
\index[persons]{Czmutow, Siergiej}%
    Czmutow \cite{duzhin96}.
\end{proof}

\begin{proposition}
    $\dim \mathcal P_n \gtrsim n^{\log_b n}$ dla $b > 4$.
\end{proposition}

\begin{proof}
\index[persons]{Czmutow, Siergiej}%
\index[persons]{Dużin, Siergiej}%
    Czmutow, Dużin \cite{duzhin99}.
\end{proof}

\begin{proposition}
    $\dim \mathcal P_n \gtrsim \exp (\pi \sqrt{n/3})$.
\end{proposition}

\begin{proof}
\index[persons]{Czmutow, Siergiej}%
\index[persons]{Koncewicz, Maksim}%
    Koncewicz w faksie do Czmutowa z 1997 roku. :)
\end{proof}

\begin{proposition}
    $\dim \mathcal P_n \gtrsim \exp (c \sqrt{n})$ dla każdej stałej $c < \pi \sqrt{2/3}$.
\end{proposition}

\begin{proof}
\index[persons]{Dasbach, Oliver}%
    Dasbach \cite{dasbach00} pracuje z~algebrą skończonych grafów, których wierzchołki są stopnia 1 lub 3, modulo relacje IHX i~antysymetrii.
    Pojawiają się też układy ciężarów pochodzące od algebry Liego $\mathfrak{gl}(N)$, dzięki którym można użyć pewnych dobrze znanych wyników z teorii algebr Liego i~teorii liczb.
\end{proof}

Praca \cite{dasbach00} ma tylko 11 stron, ale jest zamknięta w sobie i (zdaniem Birman) przykładem czytelności.
Ograniczenie Dasbacha pozostaje najlepsze (stan na 2011 rok).

\begin{corollary}
    Niech $a < \frac 1 6 \pi^2$ będzie stałą.
    Wtedy
    \begin{equation}
        \exp \left(\frac {n}{\log_a n} \right) \lesssim \dim \mathcal V_n / \mathcal V_{n-1}.
    \end{equation}
\end{corollary}

\begin{proof}
\index[persons]{Dasbach, Oliver}%
    Dasbach w \cite{dasbach00}.
\end{proof}

% Chmutov, Duzhin. Mostovoy - Introduction to Vassiliev Knot Invariants, strona 432
Dokładny wymiar przestrzeni $\mathcal V_n$ jest znany tylko dla $n \le 12$.
Poniższa tabela ma dość ciekawą historię.
Wasiljew znalazł ręcznie wartości w kolumnach dla $n \le 4$ w 1990 roku.
\index[persons]{Wasiljew, Wiktor}%
Potem Bar-Natan napisał komputerowy program rozwiazujący pewne równania liniowe i~znalazł tak wymiary przestrzeni $\mathcal V_n$ dla $n \le 9$, miało to miejsce w roku 1993.
\index[persons]{Bar-Natan, Dror}%
Wreszcie Kneissler cztery lata później znalazł dolne oraz górne ograniczenia: dolne oparte o znaczone powierzchnie, górne pochodzące od algebry Vogela (\cite{kneissler97}).
\index[persons]{Kneissler, Jan}%
\index{algebra Vogela}%
Dla $n \le 12$ ograniczenia te pokrywają się!

% Chmutov, Duzhin. Mostovoy - Introduction to Vassiliev Knot Invariants, strona 432
{
\renewcommand*{\arraystretch}{1.4} % bez tego wiersze mają minimalną wysokość, by pomieścić litery = ciasno
\footnotesize
\begin{longtable}{lcccccccccccccc}
\hline
    $n$ & $0$ & $1$ & $2$ & $3$ & $4$ & $5$ & $6$ & $7$ & $8$ & $9$ & $10$ & $11$ & $12$ \\ \hline \endhead
    $\dim \mathcal V_n$ & $1$ & $1$ & $2$ & $3$ & $6$ & $10$ & $19$ & $33$ & $60$ & $104$ & $184$ & $316$ & $548$ \\
    $\dim \mathcal V_n / \mathcal V_{n-1}$ & $1$ & $0$ & $1$ & $1$ & $3$ & $4$ & $9$ & $14$ & $27$ & $44$ & $80$ & $132$ & $232$ \\
    \hline
\end{longtable}
\normalsize
}


% kneissler97 podaje inny ciąg: 0, 1, 1, 2, 3, 5, 8, 12, 18, 27, 39, 55... (rk Pm), nasz nazywając (rk Am / rk Arm)
% Am: Z<circle diagrams of degree m> / Z<STU relations>
% Arm: Am / Z<FI relations>
% Pm: podmoduł Am generowany przez spójne diagramy




\subsection{Całka Koncewicza}

Żaden niezmiennik Wasiljewa nie jest zupełny:

\begin{proposition}
    Dla każdej liczby naturalnej $n$ i niezmiennika Wasiljewa $v$ rzędu $n$, istnieją różne od siebie węzły $K_1 \neq K_2$ takie, że $v(K_1) = v(K_2)$.
\end{proposition}

\begin{proof}
    \index[persons]{Ohyama, Yoshiyuki}%
    W pracy Ohyamy \cite{ohyama95}.
    Dla każdego węzła $K$ wskazano tam nieskończoną rodzinę złożonych węzłów $(K_n)$, których niezmienniki rzędu co najwyżej $n$ nie odróżniają od $K$.
\end{proof}

Jak można przeczytać w recenzji na portalu MathSciNet, Ohyama był świadomy istnienia preprintu Stanforda, wydanego później jako \cite{stanford96}: dowodzi się tam, że dla każdego splotu $L$ istnieje nieskończona rodzina pierwszych, nierozszczepialnych, alternujących splotów nieodróżnialnych takimi niezmiennikami.
\index[persons]{Stanford, Ted}%

Z drugiej strony, Czmutow i inni piszą w \cite{duzhin12}, że sześć niezmienników rzędu co najwyżej 4 wystarcza do odróżnienia dowolnych dwóch węzłów pierwszych do 8 skrzyżowań.
\index[persons]{Czmutow, Siergiej}%
\index[persons]{Dużin, Siergiej}%
\index[persons]{Mostovoy, Jacob}%
Kneissler twierdzi (\cite[wniosek 2.5]{kneissler97}), że niezmienniki rzędu co najwyżej 12 nie odróżniają węzłów od ich odwrotności.
\index[persons]{Kneissler, Jan}%
\index{węzeł!odwrotny}%

\index{całka Koncewicza|(}%
W 1993 roku Maxim Koncewicz pokazał, że dla każdego węzła można policzyć pewną całkę (teraz nazywaną całką Koncewicza), która jest niezmiennikiem uniwersalnym: z jej wartości można odtworzyć wszystkie inne niezmienniki skończonego typu.
\index[persons]{Koncewicz, Maksim}%
Bar-Natan w 1995 roku znalazł wartość tej całki dla niewęzła:
\index[persons]{Bar-Natan, Dror}%
\begin{equation}
    I (\SmallUnknot) = \exp \left(\sum_{n=0}^\infty b_{2n} w_{2n}\right),
\end{equation}
gdzie $b_{2n}$ to zmodyfikowane liczby Bernoulliego o funkcji tworzącej
\begin{equation}
    \sum_{n=0}^\infty b_{2n} x^{2n} = \frac 12 \log \frac {e^{x/2} - e^{-x/2}}{x/2},
\end{equation}
zaś $w_{2n}$ to ,,koła'': diagramy okręgu z doczepionymi $2n$ promieniami.
Liniową kombinację należy rozumieć jako element algebry chińskich znaków.
\index{algebra!chińskich znaków}%
Następnie Marché w~2003 roku znalazł wartości całki dla węzłów torusowych (\cite{marche04}).
\index[persons]{Marché, Julien}%
Wygląda na to, że nikt nie odważył się dokonać tego dla innych węzłów (stan na 2019).

\begin{conjecture}
    \label{con:vassilliev}
    Całka Koncewicza jest niezmiennikiem zupełnym.
\end{conjecture}

Całka Koncewicza jest mocniejsza od każdego wielomianowego niezmiennika, jaki dotąd poznaliśmy, a~nie wiemy nawet, czy wielomian Jonesa wykrywa niewęzły (hipoteza \ref{con:jones}).
\index{wielomian!Jonesa}%
Czmutow, Dużin wspominają w~dość czytelnie napisanym artykule \cite{chmutov05}, że hipoteza \ref{con:vassilliev} jest prawdziwa dla warkoczy (Kohno \cite{kohno87}) i~splotów sznurkowych (Bar-Natan \cite{barnatandror95}).
\index[persons]{Czmutow, Siergiej}%
\index[persons]{Dużin, Siergiej}%
\index[persons]{Kohno, Toshitake}%
\index[persons]{Bar-Natan, Dror}%
% DICTIONARY;string;sznurkowy;splot
%=% kohno87 nie zawiera nazwiska Koncewicz (!)
\index{splot!sznurkowy}%
\index{warkocz}%

Zbiór problemów Ohtsukiego \cite[s. 398-444]{ohtsuki02} poświęca wiele stron na niezmienniki skończonego typu oraz całkę Koncewicza.

\index{całka Koncewicza|)}%



\index{niezmiennik!Wasiljewa|)}

% koniec sekcji niezmienniki Wasiljewa



\chapter{Topologia algebraiczna}
W tym rozdziale poznamy niezmienniki wywodzące się z~topologii algebraicznej przy użyciu maszynerii topologii algebraicznej, na tyle, na ile to możliwe.
Zaczniemy od grupy splotu (czyli grupy jego dopełnienia), potem poznamy jej prezentację Wirtingera i~jeszcze raz spotkamy pochodną Foxa.
Następnie odkryjemy powierzchnie Seiferta, jeszcze jedno ,,źródło'' genusu, wyznacznika, sygnatury, niezmiennika Arfa czy przede wszystkim wielomianu Alexandera.
Na koniec powiemy krótko, czym są homologie, w szczególności homologie Chowanowa.

% koniec wstępu do rozdziału 4
\input{40-topology/401-group}

\section{Powierzchnie i macierze Seiferta}
W tej sekcji pogłębimy nasze rozumienie wielomianu Alexandera i~odkryjemy jego powiązania z topologicznymi własnościami węzłów.
Poznamy także zupełnie nowy sposób na wyznaczanie jego wartości, posłużą do tego powierzchnie oraz macierze Seiferta.


\subsection{Powierzchnia Seiferta}
Zaczniemy od przyjrzenia się powierzchniom.
Niektóre stwierdzenia będziemy przyjmować bez dowodu, by nie rozwodzić się za bardzo nad topologią.

\begin{definition}
    \index{powierzchnia}
    Powierzchnia to dwuwymiarowa rozmaitość topologiczna $M \subseteq \R^n$.
\end{definition}

Rozmaitość to obiekt, który wygląda lokalnie jak przestrzeń euklidesowa: każdy jej punkt $x \in M$ posiada otwarte otoczenie homeomorficzne z~otwartą kulą.
Przykładami powierzchni są sfera, brzeg torusa albo hiperboloida jednopowłokowa.
Istnieje ogólniejsze pojęcie, to jest rozmaitość z~brzegiem: każdy jej punkt posiada otoczenie homeomorficzne z otwartym podzbiorem górnej półpłaszczyzny $\{x \in \C: \mathfrak {Im} \ge 0\}$.
Zwartą powierzchnię bez brzegu nazywamy domkniętą.

Powierzchnię nazywamy orientowalną, jeśli nie istnieje na niej zamknięta krzywa, podczas pokonywania której odwraca się kierownica.
Orientowalne są dokładnie te powierzchnie, które nie zawierają w sobie kopii wstęgi Möbiusa.

\index{powierzchnia Seiferta|(}%
Najważniejsze dla nas są powierzchnie Seiferta:

\begin{definition}[powierzchnia Seiferta]
    Niech $L$ będzie splotem.
    Spójną, orientowalną powierzchnię zanurzoną w przestrzeni $\R^3$, której brzegiem jest splot $L$, nazywamy powierzchnią Seiferta splotu $L$.
    % R^3, nie R^n: patrz Kawauchi, 47
\end{definition}

% \begin{example}
% Powierzchnia Seiferta dla trójlistnika:
% \begin{center}
% \begin{tikzpicture}
% [scale=0.1]
%   \clip (-17,-15) rectangle (17,15);
%   \foreach \d in {0,180} {
%       \path[OBSZAR    ,rotate=\d] (-1.25,11.5)
%       .. controls (2,14) and (6,13.5) ..  (10,12)
%       .. controls (23,7) and (15,-20)  .. (3,-13)
%       -- (1.25, -11.5)
%       .. controls (4.5,-8) and (4.5,-4) .. (0,0)
%       .. controls (4,4) and (4.5,5.5) .. (-1.25,11.5);}
%   \path[TIKZ_ARCH] (0,10) .. controls (10,0) and (-10,0) .. (0,-10);
%   \foreach \d in {0,180} {
%   \path[TIKZ_ARCH, rotate=\d] (-1.5,1.5) .. controls (-6,6) and (-3,17) .. (10,12)
%   .. controls (23,7) and (15,-20)  .. (3,-13);}
% \end{tikzpicture}
% \end{center}
% \end{example}

Nie każde uszachowienie diagramu węzła prowadzi do powierzchni Seiferta:
widać to po standardowym diagramie trójlistnika.
Pomimo to...

\begin{proposition}
    \label{prp:seifert_exists}
    Każdy węzeł posiada powierzchnię Seiferta.
\end{proposition}

Powyższe stwierdzenie uzasadnili Pontriagin oraz Frankl w~1930 roku, my jednak podamy przyjemny i~konstruktywny dowód podany przez Seiferta \cite{seifert35} cztery lata później.
\index{człowiek!Seifert, Herbert}%
% Frankl, F.; Pontrjagin, L. (1930). "Ein Knotensatz mit Anwendung auf die Dimensionstheorie". Math. Annalen (in German). 102 (1): 785–789. doi:10.1007/BF01782377.

\begin{proof}
    Wybierzmy diagram $D$ dla węzła oraz orientację,
    a~następnie wyprostujmy wszystkie skrzyżowania zgodnie z~ich orientacją:
\begin{comment}
    \[
        \LargeMinusCrossingArrows, \LargePlusCrossingArrows \mapsto \LargeJustSmoothing
    \]
\end{comment}

    Otrzymany diagram składa się teraz z~pewnej liczby zamkniętych krzywych,
    zwanych okręgami Seiferta, które wypełniamy do dysków.
    Tam, gdzie jeden okrąg leżał wewnątrz drugiego, podnosimy wewnętrzny nad zewnętrzny.
    Przy każdym skrzyżowaniu pierwotnego diagramu doklejamy skręcony pasek do obydwu dysków.

    \begin{figure}[H]
        \centering
        \includegraphics[width=0.75\textwidth]{../data/seifert-algorithm.jpg}
        \caption[Smthing]{Kolejne kroki algorytmu Seiferta}
    \end{figure}

    Dyski są dwustronne, więc ich górnej stronie przypisujemy znak $+$,
    jeśli tylko brzeg jest zorientowany dodatnio i~$-$ w~przeciwnym razie.
\end{proof}

Powierzchnia Seiferta dziedziczy orientację po węźle.
Nawet niewinne odwrócenie jednego z ogniw splotu potrafi istotnie zmienić jego powierzchnię, dlatego potrzebna jest ostrożność!

\index{powierzchnia Seiferta|)}%

% Węzeł jest rozwłókniony dokładnie wtedy, gdy stanowi grzbiet pewnego 'open book decomposition' $S^3$.




\subsection{Węzły rozwłóknione}
\index{węzeł!włóknisty|see {węzeł rozwłókniony}}%
\index{węzeł!rozwłókniony|(}%
Wspomnijmy jeszcze krótko o~specjalnym rodzaju węzłów i splotów (patrz \cite[s. 49-50]{kawauchi96}).

% DICTIONARY;fibered;rozwłókniony, włóknisty;-
\begin{definition}
    Niech $L \subseteq S^3$ będzie splotem.
    Jeśli istnieje rodzina $F_t$ powierzchni Seiferta dla splotu $K$ sparametryzowana przez $t \in S^1$ taka, że $F_t \cap F_s = K$ dla $t \neq s$, to splot $K$ nazywamy rozwłóknionym albo włóknistym.
\end{definition}

\index{splot!Neuwirtha}%
Dawniej nazywano je splotami Neuwirtha, gdyż ten pokazał w~swojej pracy dyplomowej z~1959 roku, że można je scharakteryzować jako sploty, których komutant grupy podstawowej jest skończenie generowany, lub równoważnie, wolny.

\begin{example}
    Niewęzeł, trójlistnik $3_1$, ósemka $4_1$, $5_{1}$, $6_{2}$, $6_{3}$, $7_{1}$, $7_{6}$, $7_{7}$, $8_{2}$, $8_{5}$, $8_{7}$, $8_{9}$, $8_{10}$, $8_{12}$, $8_{16}$..$8_{21}$, splot Hopfa oraz wszystkie węzły torusowe są rozwłóknione.
\end{example}

(Jeśli węzeł pierwszy o co najwyżej ośmiu skrzyżowaniach nie został wymieniony w tym przykładzie, to nie jest rozwłókniony).
Rozkład liczby węzłów rozwłóknionych wśród węzłów pierwszych wygląda następująco:
\begin{itemize}
\item 9 skrzyżowań -- 23 węzły,
\item 10 skrzyżowań -- 74 węzły,
\item 11 skrzyżowań -- 256 węzłów,
\item 12 skrzyżowań -- 873 węzły.
\end{itemize}
% ZWERYFIKOWANO: funkcja count_fibered

Lwia część analizy węzłów o 12 skrzyżowaniach została wykonana przez Stojmenowa i~Hirasawę, jak podaje baza danych KnotInfo \cite{knotinfo22}.
% źródło: https://knotinfo.math.indiana.edu/descriptions/fibered.html
\index[persons]{Hirasawa, Mikami}%
\index[persons]{Stojmenow, Aleksander}%

\begin{proposition}
\index{wielomian!Alexandera}%
    Pierwszy i~ostatni współczynnik wielomianu Alexandera węzła rozwłóknionego to $\pm 1$.
\end{proposition}

% Kryterium to jest wystarczające dla węzłów pierwszych o co najwyżej 10 skrzyżowaniach oraz alternujących, ale znany jest przykład niewłóknistego węzła o 21 skrzyżowaniach, którego wielomian Alexandera ma postać $t^4 - t^3 + t^2 - t +1$.
% TODO: ustalić, czemu tak dużo skrzyżowań (z której książki ten fakt?). Sam wynik wydaje się być folklorem, tzn. nie wiadomo kto pierwszy to pokazał.

Kryterium to jest wystarczające dla węzłów pierwszych o co najwyżej 10 skrzyżowaniach, ale $11n_{34}$, $11n_{42}$, $11n_{73}$ oraz osiemnaście węzłów pierwszych o 12 skrzyżowaniach nie są rozwłóknione mimo tego, jak wygląda ich wielomian Alexandera.
% ZWERYFIKOWANO: funkcja alexander_fibered

\begin{example}
\index{węzeł!skręcony}%
    Niech $K$ będzie węzłem skręconym z $n$ półskrętami.
\index{wielomian!Alexandera}%
    Wtedy jego wielomianem Alexandera jest
    \begin{equation}
        \alexander_n(t) = n \cdot \left(t + \frac 1 t \right) - (2n+1),
    \end{equation}
    więc węzeł $K$ nie jest rozwłókniony, chyba że $n = 1$.
\end{example}

\begin{corollary}
% TODO: węzeł dokerski do indeksu?
    $2$-skręcony węzeł $6_1$ (węzeł dokerski) nie jest rozwłókniony.
\end{corollary}

Rolfsen \cite[s. 326]{rolfsen76} podaje jako ćwiczenie w swojej książce:

\begin{proposition}
\index{suma spójna}%
    Rodzina węzłów rozwłóknionych jest zamknięta na branie sum spójnych.
\end{proposition}

\index{węzeł!rozwłókniony|)}%

% koniec podsekcji Węzły rozwłóknione



\subsection{Genus}
\index{genus|(}%
\label{sec:genus}%
Zanim przejdziemy do zdefiniowania macierzy Seiferta, potrzebować będziemy krótkiego skoku w bok -- zrozumieć bardzo geometryczny niezmiennik węzłów, genus.

Zaczniemy od starego twierdzenia, które klasyfikuje powierzchnie domknięte.

\begin{proposition}
    Każda powierzchnia domknięta jest członkiem jednej z dwóch nieskończonych rodzin:
    \begin{enumerate}[leftmargin=*]
        \itemsep0em
        \item sumą spójną $g \ge 0$ torusów,
        \item sumą spójną $k \ge 1$ rzeczywistych płaszczyzn rzutowych.
    \end{enumerate}
\end{proposition}

Elementy pierwszej rodziny są orientowalne.
Sferę traktujemy dla wygody jako sumę spójną $g = 0$ torusów.
Wtedy sumę spójną $g$ torusów możemy wyobrazić sobie jako sferę, do której doklejono $g$ uchwytów.

\begin{definition}[genus powierzchni]
    Ilość torusów nazywamy genusem powierzchni i oznaczamy literą $\genus$.
\end{definition}

Podobna charakteryzacja istnieje dla powierzchni z~brzegiem.
Każdy taki obiekt jest homeomorficzny z~sumą spójną $g$ torusów, w~których wydrążono pewną liczbę otworów: tyle, ile składowych spójności ma brzeg powierzchni.
W~przypadku powierzchni Seiferta mamy do czynienia z jednym otworem.

Dla wygody przypomnijmy jeszcze definicję klasycznego niezmiennika powierzchni:

\begin{definition}[charakterystyka Eulera]
\index{charakterystyka Eulera}%
    Niech $M$ będzie domkniętą powierzchnią orientowalną.
    Po striangulowaniu, składa się z $k_0$ wierzchołków, $k_1$ krawędzi oraz $k_2$ ścian.
    Wielkość
    \begin{equation}
        \chi = k_0 - k_1 + k_2
    \end{equation}
    jest niezmiennikiem powierzchni, zwanym charakterystyką Eulera.
\end{definition}

Definicja ta nie jest wygodna podczas ręcznych obliczeń.
Mamy za to:

\begin{proposition}
    Charakterystykę Eulera powierzchni jednoznacznie wyznaczają cztery reguły:
    \begin{itemize}
        \item jeśli $M$ jest dyskiem, to $\chi(M) = 1$,
        \item jeśli $M_1, M_2$ są powierzchniami, to $\chi(M_1 \sqcup M_2) = \chi(M_1) + \chi(M_2)$,
        \item jeśli powierzchnia $M_2$ powstaje z $M_1$ przez dołączenie paska, to $\chi(M_2) = \chi(M_1) - 1$,
        \item jeśli powierzchnia $M_2$ powstaje z $M_1$ przez dołączenie dysku do całej składowej spójności brzegu, to $\chi(M_2) = \chi(M_1) + 1$.
    \end{itemize}
\end{proposition}

Genus oraz charakterystyka Eulera są ze sobą związane:

\begin{proposition}
    Niech $M$ będzie powierzchnią o genusie $\genus$ i $\mu$ składowych spójności brzegu.
    Wtedy
    \begin{equation}
        \chi = 2 - \mu - 2\genus.
    \end{equation}
\end{proposition}

Nas interesują głównie powierzchnie Seiferta węzłów:
\index{powierzchnia Seiferta}

\begin{proposition}
    \label{prp:seifert_euler_characteristics}
    Niech $K$ będzie węzłem z~diagramem $D$.
    Wtedy $\chi(M_D) = d - b$, gdzie $b$ jest liczbą skrzyżowań $D$, zaś $d$ jest liczbą okręgów Seiferta.
\end{proposition}

Można przeczytać o tym w \cite[s. 82]{murasugi96}.

\begin{proof}
    W~dowodzie faktu \ref{prp:seifert_exists} widzieliśmy, że liczba skrzyżowań $b$ jest jednocześnie liczbą pasków doklejonych do dysków.
    Bezpośredni rachunek pokazuje, że wtedy $k_0 = 4b$, $k_1 = 6b$ oraz $k_2 = b+d$.
    Wynika stąd, że $\chi = 4b - 6b + b + d = d - b$.
\end{proof}

Reszta tej podsekcji nie jest wymagana do zrozumienia macierzy Seiferta, przyjrzymy się genusowi jako obiektowi ciekawemu samemu w sobie.

\begin{definition}[3-genus]
    Niech $K$ będzie węzłem.
    Wśród wszystkich powierzchni Seiferta węzła $K$ istnieje co najmniej jedna o minimalnym genusie, jej genus nazywamy 3-genusem węzła $K$ i oznaczamy także przez $\genus$.
\end{definition}

Znalezienie 3-genusu dowolnego węzła sprawia te same trudności, co wyznaczenie jego liczby gordyjskiej.
Dowolna powierzchnia Seiferta zadaje ograniczenie z góry.
Z dołu 3-genus można szacować przy użyciu wielomianu Alexandera:
\index{wielomian!Alexandera}%

\begin{proposition}
    \label{prp:alexander_genus}
    Niech $K$ będzie węzłem.
    Wtedy $\operatorname{span} \alexander_K(t) \le 2\genus(K)$.
\end{proposition}

Fakt ten znalazłem w podręczniku Murasugiego \cite[s. 131]{murasugi96}.

\begin{proof}
    Załóżmy, że $F$ jest powierzchnią Seiferta węzła $K$ o genusie $g$.
    Wtedy macierz Seiferta powstała z $F$ jest stopnia $2g$, więc żaden ze składników jej wyznacznika nie może mieć stopnia (jako wielomian) większego niż $2g$.
\end{proof}

To dolne ograniczenie jest realizowane przez pewną powierzchnię Seiferta dla każdego pierwszego węzła o~co najwyżej 11 skrzyżowaniach poza siedmioma wyjątkami: 11n42, 11n67, 11n97 ($g = 2$), 11n34, 11n45, 11n73 oraz 11n152 ($g = 3$).
% warto byłoby dodać jakiś kod pozwalający sprawdzić, czemu akurat te węzły
Jeżeli nie powoduje to nieporozumień, zamiast 3-genus można pisać po prostu genus.

\begin{proposition}
    Niech $K$ będzie węzłem, zaś $M$ jego macierzą Seiferta.
    Równość $\operatorname{span} \alexander_K(t) = 2\genus(K)$ zachodzi wtedy i tylko wtedy, gdy wyznacznik $\det M \neq 0$ jest niezerowy.
\end{proposition}

Floer zdefiniował w~\cite{floer90} przestrzeń wektorową nazywaną teraz homologią Floera, jest ona wyposażona w~endomorfizm parzystego stopnia, który powstaje z 2-wymiarowej klasy homologii reprezentowanej przez powierzchnię Seiferta.
%~kanoniczną gradację modulo $2$ oraz
\index{homologia!Floera}
Ta homologia rozkłada się na sumę prostą przestrzeni własnych wyróżnionego endomorfizmu, ich charakterystyki Eulera są współczynnikami wielomianu Alexandera.
Pozwala to na dokładniejsze szacowanie genusu węzła, patrz prace Ozsvátha, Szabó \cite{szabo03} i Ghigginiego \cite{ghiggini08}.

Z góry genus ograniczony jest przez kilka klasycznych niezmienników numerycznych.
Zanim to pokażemy, przytoczymy techniczny lemat udowodniony przez Yamadę (\cite{yamada87}):

\begin{proposition}
    \label{prp:seifert_circles_braid}
    Niech $L$ będzie splotem, zaś $\operatorname{s} L$ minimalną liczbą okręgów Seiferta, które dostajemy ze wszystkich możliwych diagramów splotu $L$.
    Wtedy $\operatorname{s} L = \braid L$ jest równe indeksowi warkoczowemu.
\index{indeks warkoczowy}%
\end{proposition}

Powyższe stwierdzenie występuje bez dowodu (bez?) w \cite[s. 17]{kawauchi96}.

\begin{proposition}
    Niech $L$ będzie splotem.
    Wtedy $\crossing L - \braid L - \operatorname{\mu} L + 2 \ge 2 \genus L$.
\end{proposition}

\begin{proof}
    Ustalmy minimalny diagram $D$ dla splotu $L$ i zastosujmy do niego algorytm Seiferta.
    Dostaniemy tak $s$ okręgów Seiferta oraz powierzchnię o genusie $g$.
    Fakt \ref{prp:seifert_euler_characteristics} mówiący, że $\chi = s - c$, można przekształcić do
    \begin{equation}
        g = \frac{c + 2 - s - \mu(K)}{2}.
    \end{equation}
    Z~minimalności diagramu wynika, że $c = \crossing L$.
    Fakt \ref{prp:seifert_circles_braid} mówi, że $s \ge \braid L$.
    Nierówność $g \ge \genus L$ wynika z~definicji genusu.
    Z powyższych rozważań wynika, że
    \begin{equation}
        \crossing L + 2 \ge 2 \genus L + \braid L + \operatorname{\mu} L,
    \end{equation}
    a to jest równoważnie nierówności, której prawdziwości dowodzimy.
\end{proof}

\begin{corollary}
    \label{cor:crossing_genus}
    Niech $K$ będzie węzłem.
    Wtedy $\crossing K \ge 2 \genus K$.
\index{indeks skrzyżowaniowy}
\end{corollary}

Czy w definicji genusu można ograniczyć się do powierzchni Seiferta, które pochodzą od algorytmu Seiferta?
Niestety, poza pewnymi wyjątkami, nie.
Zanim przekonamy się, dlaczego tak jest, zdefiniujmy jeszcze dwa niezmienniki.

\begin{definition}[genus kanoniczny]
\index{genus!kanoniczny}%
    Niech $K$ będzie węzłem.
    Najmniejszy z genusów powierzchni Seiferta węzła $K$, które pochodzą z~algorytmu Seiferta, nazywamy genusem klasycznym i~oznaczamy symbolem $\operatorname{g_c} K$ lub krótko $g_c$.
\end{definition}

Stojmenow \cite{stoimenow08} opisał diagramy węzłów o~kanonicznym genusie równym 2.
Część z~jego wyników przenosi się na genus 3.
Jak sam pisze, sklasyfikowane wcześniej węzły o~genusie (kanonicznym) 1 okazały się być zbyt wąską klasą.

Pod koniec lat pięćdziesiątych Crowell i~Murasugi niezależnie zauważyli, że algorytm Seiferta zastosowany do alternującego diagramu zawsze daje powierzchnię o~minimalnej powierzchni.
Ich kombinatoryczne uzasadnienie było dość zawiłe, elementarny dowód podał Gabai w \cite{gabai86}.

Dubel trójlistnika ma genus równy $1$, ale algorytm Seiferta zastosowany wobec węzła produkuje powierzchnie o genusie co najmniej $3$, jak przewiduje ograniczenie znalezione przez Mortona w \cite[twierdzenie 2]{morton86}:

\begin{proposition}
    Niech $P(v, z)$ będzie wersją wielomianu HOMFLY spełniającą zależność
    \begin{equation}
        \frac 1v P_+ - vP_- = zP_0.
    \end{equation}
    Wtedy $M = \max \deg_z P(v, z) \le 2g_c$.
\end{proposition}

Nierówność Mortona jest równością dla wielu klas węzłów, w tym alternujących (Crowell, Murasugi), jednorodnych (które stanowią uogólnienie węzłów alternujących, Cromwell w \cite{cromwell89}), whiteheadowskich dubli węzłów dwumostowych (Nakamura w \cite{nakamura06}, Tripp w \cite{tripp02}) albo precli (Brittenham, Jensen \cite{brittenham06}).
\index{nierówność Mortona}%
\index{węzeł!alternujący}%
\index{węzeł!jednorodny}%
\index{dubel Whiteheada}%
\index{węzeł!dwumostowy}%
\index{precel}%
Stojmenow pokazał, że staje się równością dla węzłów o co najwyżej 12 skrzyżowaniach i znalazł przykład węzła, dla którego jest ostra.

\begin{definition}[genus wolny]
\index{genus!wolny}
    Niech $K$ będzie węzłem.
    Minimalny genus spośród powierzchni Seiferta węzła $K$, których dopełnienie w 3-sferze jest ciałem z rączkami, nazywamy genusem wolnym i~oznaczamy $g_f$.
\index{ciało z rączkami}
\end{definition}

Dopełnienie powierzchni Seiferta jest zawsze ciałem z rączkami, więc mamy oczywiste nierówności
\begin{equation}
    g \le g_f \le g_c.
\end{equation}

Morton w 1986 roku pokazał, że genus pewnych węzłów nie jest realizowany przez żaden diagram do którego stosuje się algorytm Seiferta, choćby $10_{165}$.
Patrz \cite{morton86}.

Moriah, matematyk izraelski, rozwiązał problem postawiony dekadę wcześniej przez Kirby'ego \cite{kirby78}: jak duża może być różnica $g_f - g$?

\begin{proposition}
    Niech $K$ będzie węzłem, $D_k(K)$ jego dublem Whiteheada z $k \neq 0$ skręceniami, zaś $B_n(K)$ to $n$-krotne nakrycie cykliczne sfery $S^3$ rozgałęzione nad węzłem $K$.
    Jeżeli ranga pierwszej grupy homologii $B_{|4k+1|}(K)$ wynosi $r$, to
    \begin{equation}
        g_f(D_k(K)) \ge \frac {2r-1} {|8k+2|}.
    \end{equation}
\end{proposition}

\begin{proof}
    Praca \cite{moriah87}.
    Dowód opiera się na chirurgii węzłów i splotów w sferze $S^3$.
\end{proof}

\begin{corollary}
    Niech $K$ bedzie sumą spójną $n$ trójlistników, połóżmy $k = -1$.
    Wtedy pierwsza grupa homologii ma rangę $r = 2n$ i~genus wolny jest nieograniczony
    \begin{equation}
        g_f(D_{-1}(3_1^n)) \ge \frac {4n-1} {6},
    \end{equation}
    podczas gdy zwykły genus to $g(D_{-1}(3_1^n)) = 1$.
\end{corollary}

Kobayashi oraz Kobayashi \cite{kobayashi96} wskazali nieskończoną rodzinę węzłów nieograniczonego genusu, dla której
\begin{equation}
    g_c(K) = \frac 32 g_f(K) = 2g(K).
\end{equation}
% znam ich ze Stojmenow - Knots of (canonical) genus two

\begin{proposition}
    \label{prp:genus_detects_unknot}
    Genus wykrywa niewęzły: $K$ jest niewęzłem wtedy i tylko wtedy, gdy $g(K) = 0$.
\end{proposition}

\begin{proof}
    Niech $K$ będzie węzłem o genusie $0$.
    Z~charakteryzacji powierzchni wynika, że jego powierzchnia Seiferta to suma spójna $0$ torusów, to znaczy kula z tyloma otworami, ile $K$ ma ogniw.
    Innymi słowy, powierzchnią Seiferta węzła $K$ jest dysk, którego brzeg stanowi niewęzeł.
    To pokazuje, że implikacja w lewo jest prawdziwa.

    Implikacja w prawo jest oczywista.
\end{proof}

\begin{proposition}
    \label{prp:genus_of_sum}
    Jeśli $J, K$ są węzłami, to $g (J \shrap K) = g(J) + g(K)$.
\end{proposition}

Poniższy dowód pochodzi od Schuberta (\cite{schubert49}), został tylko zapisany we współczesnym języku.
Przebiega w dwóch etapach: najpierw pokazuje się, że genus sumy nie jest większy od sumy genusów składników, a następnie, że nie jest od niej mniejszy.

\begin{proof}
    Pokażemy najpierw, że $g(J \# K) \le g(J) + g(K)$.
    Wybierzmy powierzchnie Seiferta $M_J$ oraz $M_K$ dla $J$ oraz $K$ o~minimalnym genusie.
    Suma $J \shrap K$ powstaje z~$J$ oraz $K$, podobnie jest z~powierzchniami Seiferta:
\begin{comment}
    \[
        \begin{tikzpicture}[baseline=-0.65ex,scale=0.12]
        \draw[semithick,-Latex] (-7, -5) to (-5, -5) [in=right, out=right] to (-5, 5) to (-7, 5);
        \draw[semithick,Latex-] ( 7, -5) to ( 5, -5) [in=left, out=left] to ( 5, 5) to ( 7, 5);
        \node at (-5, 0) {$J$};
        \node at (5, 0) {$K$};
        \end{tikzpicture}
        \longrightarrow
        \begin{tikzpicture}[baseline=-0.65ex,scale=0.12]
        \draw[semithick,-Latex] (-7, -5) to (-5, -5) to [out=right, in=left] (-2, -2) -- (2, -2) to [out=right, in=left] (5, -5) to (7, -5);
        \draw[semithick,Latex-] (-7, 5) to (-5,  5) to [out=right, in=left] (-2,  2) -- (2,  2) to [out=right, in=left] (5,  5) to (7, 5);
        \node at (0, -5) {$J \# K$};
        \end{tikzpicture}
        \quad\quad
        \begin{tikzpicture}[baseline=-0.65ex,scale=0.12]
        \draw[semithick,fill=blue!10!white] (-10, -5) to (-5, -5) [in=right, out=right] to (-5, 5) to (-10, 5);
        \draw[semithick,fill=blue!10!white] ( 10, -5) to ( 5, -5) [in=left, out=left] to ( 5, 5) to (10, 5);
        \node at (-6.5, 0) {$M_J$};
        \node at (6.5, 0) {$M_K$};
        \end{tikzpicture}
        \longrightarrow
        \begin{tikzpicture}[baseline=-0.65ex,scale=0.12]
        \fill[blue!10!white] (-7, -5) rectangle (7, 5);
        \draw[semithick,fill=white] (-7, -5) to (-5, -5) to [out=right, in=left] (-2, -2) -- (2, -2) to [out=right, in=left] (5, -5) to (7, -5);
        \draw[semithick,fill=white] (-7, 5) to (-5,  5) to [out=right, in=left] (-2,  2) -- (2,  2) to [out=right, in=left] (5,  5) to (7, 5);
        \node at (0, 0) {$M_{J \# K}$};
        \end{tikzpicture}
    \]
\end{comment}

    Skoro $M_{J\#K}$ powstaje z~$M_J \sqcup M_K$ przez dołączenie paska do brzegu, mamy
    \begin{equation}
        \chi(M_{J\#K}) = \chi(M_J \sqcup M_K) - 1 = \chi(M_J) + \chi(M_K)-1,
    \end{equation}
    a~przez to
    \begin{equation}
        g(M_{J\#K}) = \frac{1-\chi(M_{J\#K})}{2} =
        \frac{1-\chi(M_{J})}{2} + \frac{1-\chi(M_{K})}{2}
        % = %g(M_J)+g(M_K)
        = g(J) + g(K).
    \end{equation}
    To kończy dowód pierwszej nierówności.
    Pokażemy jeszcze, że $g(J \# K) \ge g(J)+g(K)$.
    Zaczynamy od powierzchni Seiferta $M_{J\#K}$ dla $J\#K$ o~minimalnym genusie $g(M_{J\#K})$ równym $g(J\#K)$.
    Poprzez wykonanie chirurgii na powierzchni, możemy przyjąć specjalną postać jak w~poprzednim dowodzie:
\begin{comment}
    \[
        \begin{tikzpicture}[baseline=-0.65ex,scale=0.16]
            \fill[blue!10!white] (-5, -5) rectangle(5, 5);
        \draw[semithick,fill=white] (-5, -5) to [out=right, in=left] (-2, -2) -- (2, -2) to [out=right, in=left] (5, -5);
        \draw[semithick,fill=white] (-5,  5) to [out=right, in=left] (-2,  2) -- (2,  2) to [out=right, in=left] (5,  5);
            \node at (0, 0) {$M_{J \# K}$};
        \end{tikzpicture}
    \]
\end{comment}

    Usunięcie paska daje powierzchnie Seiferta dla $M_J$ oraz $M_K$ takie, że
    \[
        g(M_J)+g(M_K)=g(M_{J\#K})=g(J\#K).
    \]
    Oznacza to, że $g(J)+g(K)\leqslant g(M_J)+g(M_K)=g(J\#K)$ i~tak naprawdę mamy równość.
\end{proof}

\begin{corollary}
    \label{cor:connected_sum_no_inverses}
    Jeśli suma spójna dwóch węzłów jest niewęzłem, to oba składniki także nim są.
\end{corollary}

Powrócimy teraz do węzłów pierwszych (definicja \ref{def:prime_knot}).
\index{węzeł!pierwszy}%

\begin{proposition}
    Niech $K$ będzie węzłem.
    Jeśli $g(K) = 1$, to $K$ jest węzłem pierwszym.
\end{proposition}

\begin{proof}
    Załóżmy nie wprost, że $K = K_1 \# K_2$ jest sumą dwóch nietrywialnych węzłów.
    Z~faktu \ref{prp:genus_of_sum} wynika wtedy, że $g(K) = g(K_1) + g(K_2)$.
    Zatem jeden z węzłów $K_1, K_2$ ma genus zero i jest trywialny, wbrew naszemu założeniu.
\end{proof}

Implikacja odwrotna jest fałszywa: pięciolistnik jest pierwszy, ale jego genus wynosi $2$.

\begin{proposition}
    Każdy węzeł można zapisać jako suma spójna pewnej liczby węzłów pierwszych (niewęzeł jest sumą pustej rodziny węzłów).
\end{proposition}

\begin{proof}
    Dowodzimy przez indukcję względem genusu $g(K)$.
    Przypadek bazowy $g(K) = 0$ jest oczywisty, gdyż wtedy $K$ to niewęzeł.
    Załóżmy więc, że fakt zachodzi dla węzłów $J$ genusu co najwyżej $n$.
    Niech $K$ będzie genusu $n + 1$.

    Jeśli $K$ jest pierwszy, nie ma czego dowodzić.
    W przeciwnym razie jest równoważny z~$J_1 \shrap J_2$, gdzie $J_1$ i~$J_2$ są nietrywialne.
    Mamy $g(J_1) + g(J_2) = g(K)$ oraz $g(J_1),g(J_2) \ge 1$.
    Zatem $g(J_1), g(J_2) \le n$.
    Na mocy hipotezy indukcyjnej, $J_1$ oraz $J_2$ są równoważne sumom
    \[
        J_1 \cong K_1\#\cdots\# K_s,\qquad
        J_2 \cong K_{s+1}\#\cdots\# K_r,
    \]
    gdzie $K_i$ są pierwsze.
    Zatem $K$ jest równoważny z~$K_1\#\cdots\# K_r$, co kończy dowód.
\end{proof}

Nasz aparat matematyczny jest niedostatecznie rozwinięty, by móc udowodnić jedyność rozkładu.

\begin{theorem}[Schubert, 1949]
    Każdy nietrywialny węzeł rozkłada się na węzły pierwsze.
    Rozkład jest, z dokładnością do kolejności składników, jednoznaczny.
\index{węzeł!pierwszy}%
\end{theorem}

Schubert podał geometryczny dowód oparty o powierzchnie Seiferta; wyraził go w języku PL-rozmaitości (\cite{schubert49}), ale niedużym wysiłkiem można dokonać adaptacji do gładkiego świata.
Praca Schuberta korzysta z twierdzenia Alexandera, że 2-sfera w przestrzeni $\R^3$ ogranicza dysk, i jego odpowiednika dla torusów w $S^3$.

Hashizume \cite{hashizume58} rozszeszył wyniki Schuberta do splotów.

\begin{proposition}
    \label{prp:infinitely_many_prime_knots}
    Istnieje nieskończenie wiele węzłów pierwszych.
\end{proposition}

\begin{proof}
    Pokażemy, że wszystkie węzły $(2n+2)_1$ są pierwsze, gdzie $n \ge 1$.
    Istotnie, algorytm Seiferta zastosowany do diagramu tego węzła wyprodukuje $2n+1$ okręgów.
\begin{comment}
    \[
        \begin{tikzpicture}[baseline=-0.65ex,scale=0.055]
        \begin{knot}[clip width=10, flip crossing/.list={1,4,5},end tolerance=1pt]
            \node at (0,10) {$\cdots$};
            \strand[semithick] (-30, -5) -- (-5, -5);
            \strand[semithick,-Latex]  (5, -5) -- (30, -5);
            \strand[semithick,Latex-]  (-30,-15) -- (-5,-15);
            \strand[semithick,Latex-]  (5,-15) -- (30,-15);

            \strand[semithick,domain=-90:90] plot ({7.5*cos(\x)-5}, {5*sin(\x)-10});
            \strand[semithick,domain=90:270] plot ({7.5*cos(\x)+5}, {5*sin(\x)-10});

            % zewnętrzne obręcze -- lewa strona
            \strand[semithick] (-30, 15) to [out=left, in=up]   (-45, 0);
            \strand[semithick] (-30,-15) to [out=left, in=down] (-45, 0);
            \strand[semithick] (-30,  5) to [out=left, in=up]   (-35, 0);
            \strand[semithick] (-30, -5) to [out=left, in=down] (-35, 0);

            % zewnętrzne obręcze -- prawastrona
            \strand[semithick] (30, 15) to [out=right, in=up]   (45,0);
            \strand[semithick] (30,-15) to [out=right, in=down] (45,0);
            \strand[semithick] (30,  5) to [out=right, in=up]   (35,0);
            \strand[semithick] (30, -5) to [out=right, in=down] (35,0);

            % jak w~drugim ruchu Reidemeistera - lewe
            \strand[semithick] (-30, 15) .. controls (-24, 15) and (-24,  5) .. (-20,  5);
            \strand[semithick] (-30,  5) .. controls (-24,  5) and (-24, 15) .. (-20, 15);
            \strand[semithick] (-10, 15) .. controls (-16, 15) and (-16,  5) .. (-20,  5);
            \strand[semithick] (-10,  5) .. controls (-16,  5) and (-16, 15) .. (-20, 15);

            % jak w~drugim ruchu Reidemeistera - prawe
            \strand[semithick] (30, 15) .. controls (24, 15) and (24,  5) .. (20,  5);
            \strand[semithick] (10, 15) .. controls (16, 15) and (16,  5) .. (20,  5);
            \strand[semithick] (30,  5) .. controls (24,  5) and (24, 15) .. (20, 15);
            \strand[semithick] (10,  5) .. controls (16,  5) and (16, 15) .. (20, 15);
        \end{knot}
        \end{tikzpicture}
        \longrightarrow
        \begin{tikzpicture}[baseline=-0.65ex,scale=0.055]
            \node at (0,10) {$\cdots$};
            \draw[semithick] (-30,  -5) -- (30, -5);
            \draw[semithick] (-30, -15) -- (30,-15);

            \draw[semithick] (0,-10) circle (3);

                % zewnętrzne obręcze -- lewa strona
            \draw[semithick] (-30, 15) to [out=left, in=up]   (-45, 0);
            \draw[semithick] (-30,-15) to [out=left, in=down] (-45, 0);
            \draw[semithick] (-30,  5) to [out=left, in=up]   (-35, 0);
            \draw[semithick] (-30, -5) to [out=left, in=down] (-35, 0);

                % zewnętrzne obręcze -- prawastrona
            \draw[semithick] (30, 15) to [out=right, in=up]   (45,0);
            \draw[semithick] (30,-15) to [out=right, in=down] (45,0);
            \draw[semithick] (30,  5) to [out=right, in=up]   (35,0);
            \draw[semithick] (30, -5) to [out=right, in=down] (35,0);

            \draw[semithick] (-30, 15) to [out=right, in=up] (-20,10);
            \draw[semithick] (-30,  5) to [out=right, in=down] (-20,10);

            \draw[semithick] (30, 15) to [out=left, in=up] (20,10);
            \draw[semithick] (30,  5) to [out=left, in=down] (20,10);

            \draw[semithick] (-10, 10) circle (5);
            \draw[semithick] (10,  10) circle (5);
        \end{tikzpicture}
    \]
\end{comment}
    Wynika stąd, że genus wynosi $\frac 12 (1 - (1+2n) + (2+2n)) = 1$, ponieważ wyznacznik ma wartość $4n+1$,
    węzły $(2n+2)_1$ nie są trywialne i~są parami różne.
\end{proof}

\index{genus|)}

% Koniec podsekcji Genus


\subsection{Macierz Seiferta}
\index{macierz Seiferta|(}
Niech $K$ będzie węzłem z diagramem $D$ i powierzchnią Seiferta $S$.

% Murasugi, s. 79
\begin{definition}[graf Seiferta]
\index{graf Seiferta}%
    Ściągnijmy dyski z dowodu faktu \ref{prp:seifert_exists} do punktów jednocześnie kurcząc doklejone paski, otrzymamy graf zwany grafem Seiferta diagramu $D$.
\end{definition}

Murasugi \cite[s. 79]{murasugi96} proponuje jako ćwiczenie dowód faktu:

\begin{proposition}
    Graf Seiferta jest dwudzielny i planarny.
\end{proposition}

% Murasugi 82, 83
Skoro graf Seiferta jest planarny, to dzieli sferę $S^2$ na $f$ obszarów.
Można wyznaczyć ich liczbę: skoro $\chi(S^2) = d - b + f = 2$, to $f - 1 = 1 - d + b$, pomijamy obszar nieograniczony.
Brzeg każdego obszaru jest zamkniętą krzywą, z których tworzymy krzywe $x_1, \ldots, x_m$ na powierzchni Seiferta.
Generują one grupę podstawową $\pi_1(S)$.

Niech $S$ będzie powierzchnią Seiferta z wyróżnioną jedną stroną.
Jeśli krzywa $x_i$ biegnie po powierzchni $S$, przez $x_i^*$ oznaczać będziemy dodatnie wypchnięcie: krzywą równoległą do $x_i$, która biegnie tuż nad nią.
Potrzebowaliśmy wyróżnić jedną ze stron powierzchni $S$, by słowo ,,nad'' miało sens.

\begin{definition}[macierz Seiferta]
    Przy zachowaniu powyższych oznaczeń, macierz, której wyrazy określa wzór $M_{i,j} = \operatorname{lk}(x_i, x_j^*)$, nazywamy macierzą Seiferta.
\end{definition}

Konstrukcja macierzy Seiferta zależy od wyboru diagramu oraz orientacji krzywych $x_i$, dlatego nie jest niezmiennikiem węzłów.
Stanie się nim, kiedy uwzględnimy jeszcze wpływ ruchów Reidemeistera.

\begin{proposition}
    Kwadratowa macierz $V$ o całkowitych wyrazach jest macierzą Seiferta węzła wtedy i~tylko wtedy, gdy $\det(V - V^t) = 1$.
\end{proposition}

\begin{proof}
    Kawauchi \cite[s. 62]{kawauchi96} pisze, że wynika to z~klasyfikacji macierzy Seiferta splotów.
\end{proof}

\begin{definition}
    Operacja $\Lambda_1$ dla pewnej odwracalnej macierzy $P$ o całkowitych wyrazach (czyli $\det P = \pm 1$) to
    \begin{equation}
        \Lambda_1 \colon M \mapsto PMP^t.
    \end{equation}
    Natomiast
    \begin{equation}
        \Lambda_2 \colon M \mapsto \begin{bmatrix}
  &   &  & 0 & 0 \\
  & M &  & \vdots & \vdots \\
  &   &  & 0 & 0 \\
* & \dots & * & 0 & 0 \\
0 & \dots & 0 & 1 & 0
\end{bmatrix} \textrm{albo} \begin{bmatrix}
  &   &  & * & 0 \\
  & M &  & \vdots & \vdots \\
  &   &  & * & 0 \\
0 & \dots & 0 & 0 & 1 \\
0 & \dots & 0 & 0 & 0
\end{bmatrix},
    \end{equation}
    gdzie gwiazdka zastępuje ustaloną liczbę całkowitą.
\end{definition}

\begin{definition}
\index{S-równoważność}%
    Niech $M_1, M_2$ będą macierzami.
    Jeśli $M_2$ można otrzymać z $M_1$ przez skończony ciąg operacji $\Lambda_1, \Lambda_2$ oraz ich odwrotności, to macierze nazywamy $S$-równoważnymi.
\end{definition}

Badania powyższej relacji równoważności prowadzili w~latach sześćdziesiątych ubiegłego stulecia Trotter \cite{trotter62}, Murasugi \cite{murasugi65} oraz Levine \cite{levine70}.
Litera $S$, jak nietrudno się domyślić, pochodzi od Seiferta.


\begin{proposition}
    Macierz Seiferta modulo $S$-równoważność jest niezmiennikiem splotów.
\end{proposition}

Dowód tego faktu jest elementarny, ale dość długi.
Razem z~ułatwiającymi zrozumienie diagramami można znaleźć go w podręczniku Murasugiego albo \cite[s. 64]{kawauchi96}, dlatego pomijamy go i skupimy się na tym, jakie niezmienniki można otrzymać z macierzy Seiferta.

Wyznacznik samej macierzy Seiferta nie jest niezmiennikiem.
Wykonując operację $\Lambda_2$ dostajemy macierz, której ostatnia kolumna albo ostatni wiersz są zerami, więc jej wyznacznik także jest zerem.
Jeśli jednak najpierw dokonamy jej symetryzacji, dostaniemy znany już niezmiennik.

\begin{proposition}
    Niech $M$ będzie macierzą Seiferta węzła $K$.
    Wtedy
    \begin{equation}
        \det K = |\det(M + M^t)|.
    \end{equation}
\end{proposition}

Przez wprowadzenie dodatkowej zmiennej $t \in \R$, ponownie uogólnimy wyznacznik do wielomianu Alexandera.

\begin{proposition}
    Niech $M$ będzie macierzą Seiferta rzędu $k$ węzła $K$.
    Wtedy
    \begin{equation}
        \alexander_K (t) = t^{-k/2}\det(M - tM^t).
    \end{equation}
\end{proposition}

Określimy jeszcze jeden, niewystępujący wcześniej niezmiennik (sygnaturę).

\index{macierz Seiferta|)}



\subsection{Sygnatura}
\index{sygnatura|(}%
Sygnatura jest kolejnym niezmiennikiem, do zdefiniowania których wystarczy znać macierz Seiferta.
Pochodzi prawdopodobnie z lat sześćdziesiątych (Trotter \cite{trotter62} dla węzłów, Murasugi \cite{murasugi65} dla splotów).
\index[persons]{Trotter, Hale}%
\index[persons]{Murasugi, Kunio}%
% z recenzji do 275415

% DICTIONARY;signature;sygnatura;-
\begin{definition}[sygnatura]
\label{def:signature}%
    Niech $M$ będzie macierzą Seiferta zorientowanego splotu $L$.
    Wielkość
    \begin{equation}
        \sigma_L := \operatorname{\sigma} (M + M^t),
    \end{equation}
    sygnaturę macierzy $M + M^t$, nazywamy sygnaturą splotu $L$.
\end{definition}

\begin{proposition}
\label{prp:signature_additive}%
    Sygnatura jest addytywna: $\sigma(K_1 \shrap \ldots \shrap K_n) = \sum_{k=1}^n \sigma(K_k)$.
\end{proposition}

Wiem o tym z \cite[s. 127]{murasugi96}.

\begin{proof}
    Bez straty ogólności ograniczmy się do przypadku $n = 2$ i~ustalmy powierzchnie Seiferta $F_1, F_2$ dla węzłów $K_1, K_2$ z~macierzami Seiferta $M_1, M_2$.
    Powierzchnia dla ich sumy spójnej $K_1 \shrap K_2$ powstaje przez sklejenie $F_1$ oraz $F_2$ paskiem.
    W języku macierzy oznacza to, że macierz Seiferta węzła $K_1 \shrap K_2$ ma postać $M = M_1 \oplus M_2$.
    Zatem:
    \begin{align}
        \sigma(K_1 \shrap K_2) & = \sigma(M + M^t) \\
                               & = \sigma(M_1 + M_1^t) + \sigma(M_2 + M_2^t) \\
                               & = \sigma(K_1) + \sigma(K_2),
    \end{align}
    co kończy dowód.
\end{proof}

\begin{corollary}
\index{liczba mostowa}%
\label{no_relation_signature_bridge}%
    Nie istnieje bezpośredni związek między sygnaturą i~liczbą mostową.
\end{corollary}

Patrz \cite[s. 145]{livingston93}.

\begin{proof}
    Węzeł torusowy $T_{2,n}$ jest dwumostowy, jego sygnatura wynosi $n - 1$.
    Suma spójna węzłów prostych (sumy przeciwnie zorientowanych trójlistników) ma zerową sygnaturę, ale na mocy faktu \ref{prp:bridge_additive} jej liczba mostowa jest nieograniczona.
\end{proof}

\begin{proposition}
\index{lustro}%
\index{rewers}%
\label{prp:signature_mirror_reverse}%
    Niech $L$ będzie splotem.
    Wtedy $\sigma(mL) = -\sigma(L)$ oraz $\sigma(rL) = \sigma(L)$.
\end{proposition}

O tym także wiem z \cite[s. 127]{murasugi96}.

\begin{proof}
    Wynika to z podobnych faktów dla macierzy Seiferta.
    Równoważność $M_{mL} \simeq - M_L^t$ wynika z tego, że zamiana nad- i podskrzyżowań odwraca wzajemne położenie krzywych, których indeksu zaczepienia szukamy.

    Podobnie pokazuje się, że $M_{rL} \simeq M_L^t$.
\end{proof}

\begin{corollary}
\index{węzeł!achiralny}%
\label{cor:acheiral_signature}%
    Jeśli $K$ jest węzłem achiralnym, to $\sigma(K) = 0$.
\end{corollary}

Węzły achiralne mają zerową sygnaturę, zatem trójlistnik nie jest achiralny.
Z faktów \ref{prp:signature_additive} oraz \ref{prp:signature_mirror_reverse} wynika, że suma tak samo zorientowanych trójlistników nie jest achiralna ($\sigma = \pm 4$).
Jak można przekonać się ze standardowego diagramu węzła prostego, ten jest achiralny.
Pisaliśmy coś o tym na stronie \pageref{two_sums_of_two_trefoils}.

\begin{proposition}
\index{wielomian!Alexandera}%
    Niech $L$ będzie węzłem.
    Jeśli $\alexander_K(t) \equiv 1$, to $\sigma (K) = 0$.
\end{proposition}

Założenie $\alexander_K(t) \equiv 1$ jest spełnione przez cztery węzły pierwsze do 12 skrzyżowań, są to $11n_{34}, 11n_{42}, 12n_{313}$ oraz $12n_{430}$, patrz wzmianka po fakcie \ref{alexander_no_detects_unknot}.
% ZWERYFIKOWANO: funkcja trivial_alexander

\begin{proof}
\index[persons]{Milnor, John}%
    Murasugi twierdzi, że zostało to udowodnione przez Milnora w \cite{milnor68}, nie jesteśmy jednak pewni, gdzie dokładnie, ale raczej poza sekcją piątą.
\end{proof}

Istnieje równoważna definicja, która nie wymaga czasochłonnego wyznaczania macierzy Seiferta.

\begin{proposition}
\index{relacja kłębiasta}%
    Sygnatura to niezmiennik topologiczny zadany kłębiastą relacją rekurencyjną:
    \begin{itemize}[leftmargin=*]
    \itemsep0em
        \item $\sigma (\SmallUnknot) = 0$,
        \item $\sigma (K_+) - \sigma (K_-) \in \{0, 2\}$,
        \item $4 \mid \sigma (K)$ wtedy i~tylko wtedy, gdy $\conway(2i) > 0$ (wielomian Conwaya).
    \end{itemize}
\end{proposition}

\begin{proof}
    Wystarczy pokazać, że sygnatura węzła spełnia trzy powyższe aksjomaty, a~następnie zauważyć, że korzystając z~nich jesteśmy w~stanie wyznaczyć jednoznacznie sygnaturę dla dowolnego węzła.
    Wynika to z~faktu, że każdy węzeł można zmienić w~niewęzeł odwracając pewne skrzyżowania.
    Pomysł opisał dokładnie Giller \cite[trzecie spostrzeżenie]{giller82}, sam oparł się o~\cite[twierdzenie 5.6]{murasugi65}.
\end{proof}

Sygnatura pozwala uzyskać proste oszacowanie liczby gordyjskiej od dołu:
\index{liczba gordyjska}%

\begin{proposition}
    Mamy $2 u(K) \ge |\sigma(K)|$.
\end{proposition}

Liczba gordyjska 83 z~801 węzłów pierwszych o mniej niż dwunastu skrzyżowaniach nie jest jeszcze znana.
Dla 272 spośród pozostałych mamy równość $2u = |\sigma|$.
% ZWERYFIKOWANO: funkcja unknotting_sigma 

\begin{proof}
    Ustalmy diagram $D$ dla węzła $K$.
    Odwrócenie dowolnego skrzyżowania polega na przejściu z~diagramu $D_+$ do $D_-$ lub z~$D_-$ do $D_+$.
    Zgodnie z relacją kłębiastą, sygnatura pozostaje taka sama lub zmienia wartość o $2$.
    Po wykonaniu $u$ odwróceń otrzymujemy diagram niewęzła o~sygnaturze zero, zatem sygnatura wyjściowego węzła nie mogła przekraczać $2u$.
    To kończy dowód.
\end{proof}

W~\cite{shinohara71} Shinohara pokazał, że dla każdej pary nieujemnych liczb całkowitych $m, n$ istnieje węzeł $K$ o wyznaczniku $4m+1$ ($8m+5$, $4m+3$) oraz sygnaturze bez znaku $8n$ ($8n+4$, $4n+2$).
\index[persons]{Shinohara, Yaichi}%
\index{wyznacznik}%
Ponadto, jeśli $m$ nie dzieli się przez $3$, istnieje węzeł o wyznaczniku $8m+1$ i sygnaturze bez znaku $8n+4$.
% skąd to? Ohtsuki?

Czas na raczej niezbyt użyteczną ciekawostkę.

\begin{conjecture}
    Czy istnieje węzeł o~sygnaturze $4$ i~wyznaczniku postaci $n = 4k + 1$?
\end{conjecture}

Stojmenow twierdzi, że jeśli tak jest, to wszystkie pierwsze dzielniki $n$ dają resztę $1$ z~dzielenia przez $24$ i~są większe od $2857$.
\index[persons]{Stojmenow, Aleksander}%
Patrz \cite[s. 540]{ohtsuki02}.

Czytając przeglądową pracę Conwaya \cite{conway19} dowiedzieliśmy się, że jeszcze w~latach sześćdziesiątych sygnatura została uogólniona do funkcji $\sigma_L \colon S^1 \to \Z$.
\index[persons]{Conway, John}%
Większość podręczników, a także prace Levine'a \cite{levine69} oraz Tristrama \cite{tristram69}, wprowadza ją przy użyciu macierzy Seiferta, więc my postąpimy dokładnie tak samo.
\index[persons]{Levine, Jerome}%
\index[persons]{Tristram, Andrew}%

\begin{definition}[sygnatura Levine'a-Tristrama]
\index{sygnatura!Levine'a-Tristrama}%
    Niech $M$ będzie macierzą Seiferta zorientowanego splotu $L$.
    Funkcję $\sigma_L \colon S^1 \to \Z$ daną wzorem
    \begin{equation}
        \sigma_L(\omega) := \operatorname{\sigma} [(1-\omega) M + (1 - \overline{\omega})M^t]
    \end{equation}
    nazywamy sygnaturą Levine'a-Tristrama splotu $L$.
    Jest niezmiennikiem splotów.
\end{definition}

Funkcja $\sigma_L$ jest kawałkami stała.
Conway pisze w \cite{conway19}, że wynika to ze wzoru na wielomian Alexandera $\Delta_L(t) = \det(tM - M^t)$.
\index[persons]{Conway, John}%
\index{wielomian!Alexandera}%
Jedynymi punktami nieciągłości są zera wielomianu $(t-1)\Delta_L(t)$, to świeży wynik Gilmera, Livingstona z~\cite{gilmer16}.
\index[persons]{Gilmer, Patrick}%
\index[persons]{Livingston, Charles}%

Mówimy, że funkcja zdefiniowana na okręgu jest zbalansowana, jeżeli w każdym punkcie nieciągłości przyjmuje wartość równą średniej z~lewo- oraz prawostronnej granicy w tym punkcie.
Livingston podał pełną charakteryzację zbilansowanych sygnatur Levine'a-Tristrama dla węzłów, analogiczny problem dla splotów wydaje się być wciąż otwarty.

\begin{proposition}
\label{balanced_iff_four_conditions}%
    Funkcja zbalansowana $\sigma \colon S^1 \to \Z$ jest realizowana jako sygnatura pewnego węzła wtedy i tylko wtedy, gdy:
    \begin{enumerate}
        \item dla każdego $\omega \in S^1$ mamy $\sigma(\omega) = \sigma(\overline{\omega})$
        \item $\sigma(1) = 0$
        \item każda nieciągłość funkcji $\sigma$ jest miejscem zerowym wielomianu Alexandera węzła
        \item jeżeli argumenty $\omega_1, \omega_2$ są sprzężone w sensie Galois, to $\sigma(\omega_1) \equiv \sigma(\omega_2)$ modulo $2$.
    \end{enumerate}
\end{proposition}

\begin{proof}
    Livingston pisze w \cite{livingston18}, że dowód w prawą stronę jest dość dobrze znany, natomiast w lewo korzysta z~wyników Kondo \cite{kondo79} i Sakaiego \cite{sakai77}, że każdy wielomian Alexandera węzła jest realizowany przez węzeł 1-gordyjski oraz zachowania zbalansowanej sygnatury podczas odwracania skrzyżowania.
\end{proof}

(Nie każdy wielomian Kauffmana/HOMFLY jest realizowany przez węzły 1-gordyjskie, Kawauchi \cite[s. 151]{kawauchi96} wspomina na przykład, że $\unknotting K \ge \log_3 |Q(-1)|$.)

\begin{proposition}
\index{węzeł!satelitarny}%
    Niech $S$ będzie satelitą z towarzyszem $C$, wzorcem $P$ oraz indeksem zaczepenia $n$.
    Wtedy
    \begin{equation}
        \sigma_S(\omega) = \sigma_P(\omega) + \sigma_C(\omega^n).
    \end{equation}
\end{proposition}

\begin{proof}
    Szczególny przypadek $\omega = -1$ rozpatrywał wcześniej Shinohara w~\cite{shinohara71}.
    Pełny dowód znajduje się w artykule \cite{litherland79} Litherlanda.
\end{proof}

Wreszcie:

\begin{proposition}
    Niech $L$ będzie splotem.
    Wtedy albo wielomian Alexandera $\Delta_L(t)$ jest tożsamościowo zerem, albo posiada co najmniej $|\sigma_L|$ zer, liczonych z krotnościami, na okręgu jednostkowym.
\end{proposition}

\begin{proof}
    Aneks w książce Liechtiego \cite{liechti16}, która nie wygląda na związaną z~teorią węzłów.
\end{proof}

\index{sygnatura|)}

% Koniec podsekcji Sygnatura



% koniec sekcji Powierzchnie i macierze Seiferta



\section{Niezmiennik Arfa}
\index{niezmiennik!Arfa|(}%

Cahit Arf wprowadził w 1941 roku pewien niezmiennik nieosobliwych form kwadratowych nad ciałem charakterystyki dwu.
\index[persons]{Arf, Cahit}%
Zrobił to między innymi po to, by sklasyfikować takie formy kwadratowe.
My poznamy wariant niezmiennika Arfy dla węzłów.

Niech $(v_{i,j})$ będzie macierzą Seiferta węzła $K$ o genusie $g$.
Wtedy jej wymiary wynoszą $2g \times 2g$ i macierz $V-V^t$ jest symplektyczna.

\begin{definition}
    Zachowując powyższe oznaczenia, niezmiennik Arfa to
    \begin{equation}
        \sum^g_{i=1}v_{2i-1,2i-1}v_{2i,2i} \pmod 2.
    \end{equation}
    % Przyjmuje on dwie wartości: 0, 1
\end{definition}

Niezmiennik Arfa dla węzłów można zdefiniować na kilka sposobów, z~których żaden nie jest istotnie lepszy od pozostałych.
Pierwszy był pomysł Robertello \cite{robertello65}:
\index[persons]{Robertello, Raymond}%

\begin{proposition}[Robertello, 1965]
    Niech $K$ będzie węzłem, zaś
    \begin{equation}
        \alexander_K(t)=c_{0}+c_{1}t+\cdots +c_{n}t^{n}+\cdots +c_{0}t^{2n}
    \end{equation}
    jego wielomianem Alexandera.
    Wtedy niezmiennik Arfa to $c_{n-1}+c_{n-3}+\cdots +c_{r} \mod 2$, gdzie $r = 0$ dla nieparzystych $n$, $r = 1$ w~przeciwnym razie.
\end{proposition}

Nieco później Murasugi \cite{murasugi69} zauważył, że warunek można uprościć:
\index[persons]{Murasugi, Kunio}%

\begin{proposition}[Murasugi, 1969]
    \label{prp:arf_murasugi}
    Niech $K$ będzie węzłem.
    Wtedy $\operatorname{Arf} K = 0$ wtedy i~tylko wtedy, gdy $\alexander_K(-1) \equiv \pm 1 \mod 8$.
\end{proposition}

Kauffman zaproponował trochę inne podejście z wykorzystaniem diagramów.
\index[persons]{Kauffman, Louis}%
Dwa węzły nazwiemy równoważnymi przez przejścia, jeśli są związane skończenie wieloma ,,przejściami'' \cite[s. 143]{kauffman83}:
\begin{comment}
\[
    \LargeTwoPassMoveA \cong \LargeTwoPassMoveB
    \quad\mbox{albo}\quad
    \LargeTwoPassMoveC \cong \LargeTwoPassMoveD
\]
\end{comment}

\begin{definition}[Kauffman, 1983]
    Każdy węzeł $K$ jest równoważny przez przejścia albo z niewęzłem (wtedy mówimy, że $\operatorname{Arf} K = 0$), albo z trójlistnikiem (wtedy, że $\operatorname{Arf} K = 1$).
\end{definition}

Wreszcie Jones zauważył \cite[tw. 19]{jones85}, że dzięki zespolonym algebrom Clifforda oraz pracy \cite{lannes85} niezmiennik Arfa jest specjalną wartością wielomianu Jonesa.
Jest to jedyna definicja, którą łatwo rozszerzyć do splotów.

\begin{proposition}[Jones, 1985]
\label{prp:arf_jones}%
    $\operatorname{Arf}(K) = \jones_K(i)$.
\end{proposition}

\begin{corollary}
    Niezmiennik Arfa jest $\shrap$-addytywny (modulo 2).
\end{corollary}

\begin{proof}
    Wynika to z faktu \ref{prp:arf_jones} oraz \ref{prp:jones_multiplicative_2}, ale bezpośredni dowód też istnieje. % gdzie?
\end{proof}

O niezmienniku Arfa usłyszymy jeszcze poznając węzły plastrowe.

\index{niezmiennik!Arfa|)}

% Koniec sekcji Niezmiennik Arfa



\section{Homologie}

Ta sekcja wymaga znajomości przynajmniej homologii: kompleksów łańcuchowych, różniczek, grup homologii.
Można się tego nauczyć z~każdego podręcznika topologii algebraicznej.

\subsection{Homologie Chowanowa}
\index{homologia!Chowanowa|(}
Viro napisał w 2004 piękną pracę \cite{viro04}, by objaśnić homologię Chowanowa używając tak mało algebry, jak to tylko możliwe.
\index[persons]{Viro, Oleg}%
Jest przyjazna dla początkujących, więc na niej opiera się reszta tej podsekcji\footnote{Viro wymienia kilka innych artykułów, z których można czerpać wiedzę. \emph{,,A good place to start: \cite{barnatan02} followed by \cite{shumakovitch12}, \cite{khovanov00}. Another possible starting point: \cite{turner17}.''}}.
\index[persons]{Chowanow, Michaił}%
\index[persons]{Bar-Natan, Dror}%
\index[persons]{Szumakowicz, Alexander}%
\index[persons]{Turner, Paul}%
Będziemy pracować ze stanami Kauffmana i obramowanymi węzłami, ponieważ autor sugeruje, że to bardziej naturalne.
Oryginalna praca Chowanowa to \cite{khovanov00}.

Niech $L$ będzie splotem, zaś $D$ jego diagramem.
Chowanow skonstruował rodzinę grup $\mathcal H^{i, j}(D)$ takich, że
\begin{equation}
    K(L, q) = \sum_{i, j} q^j (-1)^i \dim_\Q (\mathcal H^{i, j}(D) \otimes \Q),
\end{equation}
gdzie $K$ jest wersją wielomianu Jonesa.
Grupy $\mathcal H^{i, j}$ są u~niego grupami homologii pewnych kompleksów łańcuchowych.
Ich konstrukcja była przeładowana algebraicznymi szczegółami, później Bar-Natan \cite{barnatan02}, Viro podali jej warianty z~myślą o~topologach.
% Viro - Remarks on definition of Khovanov homology, arXiv

Homologię Chowanowa nazywa się kategoryfikacją wielomianu Jonesa.
Zanim zagłębimy się w szczegóły, rozpatrzmy prostszy przykład tego procesu.
Niech $X$ będzie przestrzenią topologiczną, wtedy charakterystykę Eulera oraz grupy homologii łączy zależność
\begin{equation}
    \chi(X) = \sum_{n = 0}^{\dim X} (-1)^n \operatorname{rk} H_n(X),
\end{equation}
a przy tym grupy homologii dostarczają więcej informacji, co więcej można o nich myśleć jako funktorach.
Homologie są kategoryfikacją charakterystyki Eulera.

Kategoryfikacja wielomianu Jonesa polega na zastąpieniu jakoś jego współczynników przez ciąg grup abelowych.
Wzór o sumowaniu stanów przypomina ostatnią równość, brakuje tylko przedstawienia składników po prawej stronie jako alternująca suma rang grup.

Viro zauważa, że powszechna definicja wielomianu Jonesa sprawia problem dla pustego splotu (którego nigdy wcześniej nie rozpatrywaliśmy).
Mamy:
\begin{equation}
    \jones_\varnothing = \frac{1}{-t^{1/2} - t^{-1/2}},
\end{equation}
a to nie jest wielomian Laurenta jednej zmiennej.
\index{wielomian Jonesa!powiększony}%
Dlatego definiuje powiększony wielomian Jonesa:
% DICTIONARY;Jones polynomial;wielomian Jonesa;-
% DICTIONARY;augmented;powiększony;wielomian Jonesa
\begin{equation}
    \widetilde{\jones_L}(t) = (-t^{1/2} - t^{-1/2}) \cdot \jones_L(t),
\end{equation}
i mówi, że będzie kategoryfikować powiększony wielomian Jonesa, a właściwie powiększoną klamrę Kauffmana.

Jak pisze dalej, pewne drobne trudności techniczne mogły skłonić Chowanowa do pozbycia się ułamkowych potęg przez zmianę zmiennej w~powiększonym wielomianie Jonesa: niech $q := -t^{1/2}$.
Dostaje się tak nowy wielomian, nazwijmy go $K$.
Spełnia trzy aksjomaty:
\begin{itemize}
\item (normalizacja) $K(\SmallUnknot) = q + 1/q$;
\item (stabilizacja) $K(L \sqcup \SmallUnknot) = (q + 1/q) K(L)$;
\item (relacja kłębiasta) \begin{equation}
    q^{-2}     K\left( \MediumPlusCrossingArrows \right) -
    q^{2}      K\left( \MediumMinusCrossingArrows \right) =
    (q^{-1}-q) K\left( \MediumJustSmoothing \right).
\end{equation}
\end{itemize}

Stąd widać już, jakie grupy dobrać dla niewęzła:
\begin{equation}
    H^{i,j} = \begin{cases}
        \Z & \textrm{ jeśli } i = 0, j = \pm 1 \\
        0  & \textrm{ w przeciwnym razie}.
    \end{cases}
\end{equation}
Wtedy spełniona jest równość
\begin{equation}
    K(L, q) = \sum_{i, j} (-1)^i q^j \operatorname{rk} H^{i, j} (L).
\end{equation}
Pozostało powtórzyć to dla dowolnego splotu.
Wzór o sumowaniu stanów przybiera postać:
\begin{equation}
    K(L, q) = \sum_s (-1)^{(\writhe D - |s|)/2} q^{(3\writhe D - |s|)/2} (q+1/q)^{|sD|}.
\end{equation}

Reprezentacja ta ma jedną wadę: każdy składnik z prawej strony przyczynia się do różnych jednomianów, zatem ma wpływ na różne grupy (których dopiero szukamy).
,,Surowe'' stany nie są prawdziwym odpowiednikiem sympleksów, jakie spotyka się podczas kategoryfikacji charakterystyki Eulera.
Najprostszym pomysłem, jak to naprawić, jest rozbicie ostatniej potęgi $q + 1/q$.
Zauważmy, że ma tyle czynników, ile wygładzenie diagramu ma składowych.
To motywuje definicję:

\begin{definition}[stan wzbogacony]
\index{stan!wzbogacony}%
    Stan diagramu $D$ razem z przypisaniem znaku $+$ lub $-$ do każdego okręgu $sD$ nazywamy stanem wzbogaconym.
\end{definition}

Dla ustalonego wzbogaconego stanu $S$ diagramu $D$ oznaczmy przez $\tau(S)$ sumę znaków przypisanych do okręgów\footnote{Oznaczenie wzięte z pracy Viro, żywimy nadzieję, że nikt nie weźmie $\tau$ za liczbę kolorowań z rodziału drugiego. Poza tym, Viro pisze $\sigma(s)$ zamiast naszego $|s|$ oraz $|s|$ zamiast naszego $|sD|$. Ostrożność wskazana.}.
Wtedy
\begin{equation}
    q^{(3 \writhe D - |s|)/2} (q + 1/q)^{|sD|} = \sum_{S/s} q^{(3 \writhe D - |s| + 2 \tau(S))/2},
\end{equation}
gdzie sumowanie odbywa się po wszystkich stanach $S$ wzbogacających stan $s$.
Niech
\begin{equation}
    j(S) := \frac 12 (3 \writhe D - |s| + 2 \tau(S)).
\end{equation}
Dobrnęliśmy do
\begin{equation}
    K(L, q) = \sum_S (-1)^{(\writhe D - |s|)/2} q^{j(S)},
\end{equation}
tym razem sumujemy po wszystkich wzbogaconych stanach diagramu $D$.

Potrzebujemy jeszcze trochę nowych obiektów.
Niech $C(D)$ oznacza wolną abelową grupę generowaną przez wzbogacone stany diagramu $D$, a $C^j(D)$ będzie jej podgrupą generowaną przez wzbogacone stany $S$ takie, że $j(S) = j$.

Czyni to $C(D)$ wolną grupą abelową z $\Z$-gradacją:
\begin{equation}
    C(D) = \bigoplus_{j \in \Z} C^j (D).
\end{equation}

Dla ustalonego stanu wzbogaconego $S$, niech $i(S) = (\writhe D - |s|)/2$.
Określmy ostatnią podgrupę, $C^{i,j}(D) \le C^j(S)$ generowaną przez wzbudzone stany $S$, dla których $i(S) = i$.
Dostajemy wreszcie
\begin{equation}
    K(L, q) = \sum_{j = -\infty}^\infty q^j \sum_{i = -\infty}^\infty (-1)^i \operatorname{rk} C^{i, j}(D).
\end{equation}

Teraz ,,wystarczy'' zdefiniować funkcję $d$ i sprawdzić, że jest różniczką, to znaczy że $d^2 = 0$.
Tak też robi Viro, nam brakuje sił, by przybliżyć konstrukcję.
To już koniec -- różniczka pozwala przejść z grup $C^{i,j}$ do grup homologii.
Pewne wyjaśnienia znaleźć można w~\cite[s. 42]{przytycki15}, gdzie podano przepis wymagający właściwie tylko ponumerowania skrzyżowań.
\index[persons]{Przytycki, Józef}%

Bar-Natan, topolog izraelski, napisał program liczący homologie Chowanowa szybko \cite{barnatan07}, przy czym szybko oznacza: chyba\footnote{Źródło: komentarze pod postem \url{https://mathoverflow.net/a/232267}} w~czasie $O(\exp(c \sqrt n))$, dla diagramu o~$n$ skrzyżowaniach.
\index[persons]{Bar-Natan, Dror}%
Nie możemy liczyć na istotne przyspieszenie:
znalezienie przybliżenia wielomianu Jonesa jest problemem \#P-trudnym (\cite{kuperberg15}, \cite{vertigan05}),
\index[persons]{Kuperberg, Greg}%
\index[persons]{Vertigan, Dirk}%
a przy znanych homologiach -- trywialnym.
(Ale patrz też fakt \ref{prp:jones_at_roots_of_unity}).

Kronheimer, Mrówka \cite{kronheimer11} pokazali:
\index[persons]{Kronheimer, Peter}%
\index[persons]{Mrówka, Tomasz}%

\begin{proposition}
\label{khovanov_detects_unknot}%
    Zredukowana kohomologia Chowanowa wykrywa niewęzeł.
\end{proposition}

\begin{proof}
% DICTIONARY;sutured;szwowa;rozmaitość
\index{rozmaitość szwowa}%
    Dowód składa się z dwóch kroków.
    W pierwszym panowie pokazują, że istnieje ciąg spektralny zaczynający się od zredukowanej kohomologii Chowanowa, po którym następuje koniec: homologia zdefiniowana osobliwymi instantonami.

    Potem dowodzą, że ta homologia jest izomorficzna z instantonową homologią Floera szwowego dopełnienia węzła, o której wiadomo, że wykrywa niewęzeł.
\end{proof}

\index{homologia!Chowanowa|)}

\subsection{Homologia Floera}
Do zrobienia...
\index[persons]{Floer, Andreas}%

% Koniec sekcji Homologie



\chapter{Wybrane rodziny węzłów}

Przyjęta przez nas definicja węzła czy splotu jest bardzo ogólna.
O tak określonych obiektach można udowodnić niewiele twierdzeń.
W dalszej części tego rozdziału będziemy rozpatrywali rozmaite klasy splotów.
Jest rzeczą jasną, że im węższa klasa, tym więcej twierdzeń o jej elementach można udowodnić.
Nakładane przez nas ograniczenia będą różnego charakteru.
Zaczniemy od warkoczy oraz supłów, które stanowią cegiełki do budowy splotów.
Z pojęciem supła mocno związane są sploty dwumostowe.
Potem poznamy precle, sploty złożone z wielu warkoczy połączonych ze sobą oraz bardzo symetryczne węzły Lissajous.
Później zbadamy węzły torusowe, klasę zrozumianą jako jedną z pierwszych, uogólnienie węzłów złożonych: węzły satelitarne i~opowiemy krótko o węzłach hiperbolicznych.
Na koniec przytoczymy kilka wyników czterowymiarowej teorii węzłów plastrowych i taśmowych.

% wiem o tym z "Aspects of topology, in memory of H. Dowker"
Z chronologicznego punktu widzenia, pierwszą nieskończoną rodziną węzłów poddaną rygorystycznej klasyfikacji były węzły torusowe (Schreier \cite{schreier24}), a trzy dekady później też węzły dwumostowe (Schubert \cite{schubert56}).
\index[persons]{Schreier, Otto}%
\index[persons]{Schubert, Horst}%



\section{Warkocze}
\label{sec:braid}%
\index{warkocz|(}%
Warkocze, a dokładniej grupa warkoczy, rozważane po raz pierwszy były niejawnie przez Hurwitza w~1885 roku oraz jawnie przez Artina czterdzieści lat później.
\index[persons]{Artin, Emil}%
Dla nas inspiracją była wspaniała praca przeglądowa Birman i Brendle poświęcona warkoczom, \cite{birman05}, ale także notatki Gonzáleza-Menesesa z~kursu zorganizowanego w~Pau \cite{gonzalez11}.
\index[persons]{Birman, Joan}%
\index[persons]{Brendle, Tara}%
\index[persons]{González-Meneses, Juan}%

O dwóch punktach $(d_1, t_1)$, $(d_2, t_2)$ w~walcu $B^2 \times [0, 1] \subseteq \R^3$ powiemy, że łączący je odcinek jest malejący, jeśli $t_1 > t_2$.
Łamana malejąca to taka, która jest złożona z~odcinków malejących.

% DICTIONARY;braid;warkocz;-
% DICTIONARY;strand;pasmo ...;warkocz
\begin{definition}[warkocz]
\index{pasmo warkocza}%
Teoriomnogościową sumę parami rozłącznych łamanych malejących, które łączą zbiory $\{x_1, \ldots, x_n\} \times \{1\}$ oraz $\{x_1, \ldots, x_n\} \times \{0\}$, nazywamy warkoczem o~$n$ pasmach.
\end{definition}

Poszczególne pasma warkocza możemy utożsamiać z~wykresami pewnych (gładkich) funkcji $f_i \colon [0, 1] \to \R^2$, jeśli zbiory $\{f_i(0) : 1 \le i \le n\} = \{f_i(1) : 1 \le i \le n\}$ są równe.
Wtedy dwa warkocze uznajemy za równoważne, jeśli istnieje między nimi izotopia: funkcje ciągłe dwóch zmiennych $F_i(t, s)$ określone na zbiorze $[0,1] \times [0,1]$ takie, że $F_i(t,0)= f_i(t)$ oraz $F_i(t, 1) = g_i(t)$.
Przez analogię do węzłów można zdefiniować diagramy warkoczy jako cienie bez katastrof.
Najczęściej rzutujemy prostopadle do odcinka $\{0\} \times [0, 1]$.

\begin{definition}
\index{grupa warkoczy}%
    Określmy pomocniczo dwie kontrakcje $B^2 \times [0,1] \to B^2 \times [0,1]$:
    \begin{align*}
        \psi_1(d, t)&  = (d, t/2) \\
        \psi_2(d, t)&  = (d, \frac12 (t+1))
    \end{align*}
    Klasy abstrakcji warkoczy z~mnożeniem danym wzorem $z_1z_2 = \psi_1(z_1) \cup \psi_2(z_2)$ tworzą grupę warkoczy $B_n$.
    Jej elementem neutralnym jest warkocz $1_n = \bigcup_{i = 1}^n \{x_1\} \times [0,1]$.
\end{definition}

Sprawdzenie aksjomatów grupy pozostawiamy Czytelnikowi,
pozostawiając mu małą wskazówkę graficzną:
\begin{comment}
\[
    \begin{tikzpicture}[baseline=-0.65ex, scale=0.2]
    \begin{knot}[clip width=5, end tolerance=1pt]
        \strand[semithick] (-6, 0) .. controls (-4, 0) and (-5, 2) .. (-3, 2);
        \strand[semithick] (-6, 2) .. controls (-4, 2) and (-5, 0) .. (-3, 0);
        \strand[semithick] (-6, -2) to (-3, -2);
        \strand[semithick] (-3, 0) .. controls (-1, 0) and (-2, -2) .. (0, -2);
        \strand[semithick] (-3, -2) .. controls (-1, -2) and (-2, 0) .. (0, 0);
        \strand[semithick] (-3, 2) to (0, 2);
        \strand[semithick] (+6, 0) .. controls (+4, 0) and (+5, 2) .. (+3, 2);
        \strand[semithick] (+6, 2) .. controls (+4, 2) and (+5, 0) .. (+3, 0);
        \strand[semithick] (+6, -2) to (+3, -2);
        \strand[semithick] (+3, 0) .. controls (+1, 0) and (+2, -2) .. (0, -2);
        \strand[semithick] (+3, -2) .. controls (+1, -2) and (+2, 0) .. (0, 0);
        \strand[semithick] (+3, 2) to (0, 2);
        \draw (+6, -3) rectangle (0, 3);
        \draw (-6, -3) rectangle (0, 3);
        \draw[semithick, decoration={brace,mirror,raise=3pt},decorate]  (-5.75, -3) -- node[below=6pt] {$\beta$} (-0.25, -3);
        \draw[semithick, decoration={brace,mirror,raise=3pt},decorate]  (0.25, -3) -- node[below=6pt] {$\beta^{-1}$} (5.75, -3);
    \end{knot}
    \end{tikzpicture}
    \cong
    \begin{tikzpicture}[baseline=-0.65ex, scale=0.2]
        \draw[semithick] (-3, -2) to (3, -2);
        \draw[semithick] (-3, 0) to (3, 0);
        \draw[semithick] (-3, 2) to (3, 2);
        \draw (-3, -3) rectangle (3, 3);
        \draw[semithick, decoration={brace,mirror,raise=3pt},decorate]  (-2.75, -3) -- node[below=6pt] {$1_3$} (2.75, -3);
    \end{tikzpicture}
    \quad\quad\quad
    \begin{tikzpicture}[baseline=-0.65ex, scale=0.2]
        \useasboundingbox (-6, -3) rectangle (12, 5);
\begin{knot}[clip width=5, end tolerance=1pt]
        \strand[semithick] (-6, 0) .. controls (-4, 0) and (-5, 2) .. (-3, 2);
        \strand[semithick] (-6, 2) .. controls (-4, 2) and (-5, 0) .. (-3, 0);
        \strand[semithick] (-6, -2) to (-3, -2);
        \strand[semithick] (-3, 0) .. controls (-1, 0) and (-2, -2) .. (0, -2);
        \strand[semithick] (-3, -2) .. controls (-1, -2) and (-2, 0) .. (0, 0);
        \strand[semithick] (-3, 2) to (0, 2);
        \draw (-6, -3) rectangle (0, 3);
        \draw[semithick, decoration={brace,mirror,raise=3pt},decorate]  (-5.75, -3) -- node[below=6pt] {$\beta_1$} (-0.25, -3);
        \strand[semithick] (+6, 0) .. controls (+4, 0) and (+5, 2) .. (+3, 2);
        \strand[semithick] (+6, 2) .. controls (+4, 2) and (+5, 0) .. (+3, 0);
        \strand[semithick] (+6, -2) to (+3, -2);
        \strand[semithick] (+3, 0) .. controls (+1, 0) and (+2, -2) .. (0, -2);
        \strand[semithick] (+3, -2) .. controls (+1, -2) and (+2, 0) .. (0, 0);
        \strand[semithick] (+3, 2) to (0, 2);
        \draw (+6, -3) rectangle (0, 3);
        \strand[semithick] (6+6, 0) .. controls (6+4, 0) and (6+5, 2) .. (6+3, 2);
        \strand[semithick] (6+6, 2) .. controls (6+4, 2) and (6+5, 0) .. (6+3, 0);
        \strand[semithick] (6+6, -2) to (6+3, -2);
        \strand[semithick] (6+3, 0) .. controls (6+1, 0) and (6+2, -2) .. (6+0, -2);
        \strand[semithick] (6+3, -2) .. controls (6+1, -2) and (6+2, 0) .. (6+0, 0);
        \strand[semithick] (6+3, 2) to (6+0, 2);
        \draw (6+6, -3) rectangle (6+0, 3);
        \draw[semithick, decoration={brace,mirror,raise=3pt},decorate]  (0.25, -3) -- node[below=6pt] {$\beta_2\beta_3$} (11.75, -3);
        \draw[semithick, decoration={brace,raise=3pt},decorate]  (6.25, 3) -- node[above=6pt] {$\beta_3$} (11.75, 3);
        \draw[semithick, decoration={brace,raise=3pt},decorate]  (-5.75, 3) -- node[above=6pt] {$\beta_1\beta_2$} (5.75, 3);
    \end{knot}
    \end{tikzpicture}
\]
\end{comment}

\begin{proposition}
    Grupa warkoczy $B_n$ ma prezentację
    \begin{equation}
        \{\sigma_1, \ldots, \sigma_{n-1} : \sigma_i\sigma_{i+1} \sigma_i = \sigma_{i+1} \sigma_i \sigma_{i+1}, |i-j| \neq 1 \implies \sigma_i \sigma_j = \sigma_j \sigma_i\}.
    \end{equation}
\end{proposition}

Pierwszy był sam Artin \cite{artin25}, wiele lat później Birman, Ko, Lee odkryli nową, ,,symetryczną'' prezentację, ale dla oszczędności miejsca nawet ich nie cytujemy.
% MR1870512 albo MR1654165
\index[persons]{Artin, Emil}%
\index[persons]{Birman, Joan}%
\index[persons]{Ko, Ki}%
\index[persons]{Lee, Sang}%
Wraz z~upływem wody w~Wiśle pojawiły się nowe dowody (w~kolejności chronologicznej): Magnusa \cite{magnus34}, Bohnenblusa \cite{bohnenblust47}, Chow \cite{chow48}, Fadella i van Buskirka \cite{fadell62}, Foxa i Neuwirtha \cite{fox62}.
\index[persons]{Bohnenblus, Frederic}%
\index[persons]{Chow, Wei-Liang}%
\index[persons]{Fadell, Edward}%
\index[persons]{Fox, Ralph}%
\index[persons]{Magnus, Wilhelm}%
\index[persons]{Neuwirth, Lee}%
\index[persons]{van Buskirk, James}%
Patrz też \cite{birman74}.

González-Meneses przytacza ze szczegółami niektóre dowody (przez czesanie warkoczy, grupy podstawowe kompleksów komórkowych i inne) w~\cite{gonzalez11}.

Generatory $\sigma_i$ posiadają prostą interpretację graficzną:
\begin{comment}
\[
    \begin{tikzpicture}[baseline=-0.65ex, scale=0.2]
    \begin{knot}[clip width=5, end tolerance=1pt]
        \strand[semithick] (-8, -4.5) to (8, -4.5);
        \strand[semithick] [in=left,out=right] (-8, -1.5) to (8, 1.5);
        \strand[semithick] [in=left,out=right] (-8, 1.5) to (8, -1.5);
        \strand[semithick] (-8, 4.5) to (8, 4.5);
        \node at (-10, -4.5) {$1$};
        \node at (-10, -3) {$\ldots$};
        \node at (-10, -1.5) {$i$};
        \node at (-10, 1.5) {$i+1$};
        \node at (-10, 3) {$\ldots$};
        \node at (-10, 4.5) {$n$};
        \node at (0, 3) {$\ldots$};
        \node at (0, -3) {$\ldots$};
    \end{knot}
    \end{tikzpicture}
\]
\end{comment}

Jeśli zapomnimy, jak poszczególne pasma krzyżują się, każdy warkocz staje się permutacją zbioru $n$-elementowego.
To odwzorowanie jest ,,na'' i zgodne ze składaniem warkoczy, dlatego wyznacza homomorfizm $B_n \to S_n$.
% DICTIONARY;pure;czysty;warkocz
\index{warkocz!czysty}%
Jego jądro stanowi podgrupa warkoczy czystych, czyli takich, że początek i~koniec każdego pasma znajdują się w~tej samej pozycji.

\begin{proposition}
    Grupa warkoczy $B_n$ jest przemienna wtedy i tylko wtedy, gdy $n < 3$.
\end{proposition}

\begin{proof}
    Dla $n = 1$ grupa warkoczowa jest trywialna, dla $n = 2$ mamy $B_2 \cong \Z$.
    Załóżmy, że $n \ge 3$. Wtedy grupa symetrii $S_n$ jest nieprzemienna, zatem grupa $B_n$ także taka jest.
\end{proof}

Obrazem generatora $\sigma_i \in B_n$ jest transpozycja $(i, i+1) \in S_n$, co pozwala przepisać prezentację Artina grupy $B_n$ do prezentacji Coxetera grupy symetrii:
\begin{equation}
    S_n = \langle s_1, \ldots, s_{n-1} \mid
    s_i^2 = 1,
    s_{i}s_{i+1}s_{i} = s_{i+1}s_{i}s_{i+1},
    s_is_j = s_js_i \mbox { dla } |i-j| \neq 1 \rangle
\end{equation}

\begin{proposition}
    Jeśli $n \ge 3$, to centrum grupy $B_n$ jest generowane przez warkocz $(\prod_{i = 1}^{n-1} \sigma_i)^n$.
\end{proposition}

\begin{proof}
\index[persons]{Garside, Frank}%
    Garside w~\cite{garside69}.
\end{proof}

Grupa $B_3$ jest izomorficzna z grupą podstawową trójlistnika (patrz przykład \ref{exm:trefoil_group}).
Nie istnieje żaden węzeł, którego grupą podstawową byłaby jednak $B_n$ dla $n \ge 4$: tam elementy $\sigma_1$, $\sigma_n$ oraz generator centrum rozpinają grupę izomorficzną z~$\Z^3$.
Natomiast asferyczna, niezwarta 3-rozmaitość nie może mieć grupy podstawowej $\Z^3$.
Musimy niestety pominąć czysto kohomologiczny dowód\footnote{Głodny wiedzy Czytelnik powinien odwiedzić \url{https://math.stackexchange.com/a/2119984}.} faktu, ale zaiste prowadzi to do sprzeczności.


\subsection{Warkocze a sploty}
% DICTIONARY;closure of ...;domknięcie ...;warkocz
Każdy warkocz można domknąć do węzła, łącząc ze sobą punkty $(x_i, 1)$ oraz $(x_i, 0)$
łamanymi, których rzuty do płaszczyzny diagramu nie przecinają się.
\index{warkocz!domknięcie warkocza}%
Nie wiadomo, kto wymyślił operację domykania warkoczy, ale była ona z pewnością znana Alexanderowi: rozpatrywano je jeszcze przed samymi warkoczami.
\index[persons]{Alexander, James}%
% TODO: rysunek w TikZ, jak się domyka.

Niech $b \in B_n$ będzie słowem zapisanym na standardowych generatorach.
Oznaczmy przez $b_+$, $b_-$ nieznakowaną sumę dodatnich, ujemnych wykładników.
Jeśli $b_+ - 3b_- \ge n$, to domknięcie warkocza $b$ nie jest achiralne (twierdzenie 5 z~\cite{jones85}).
\index{węzeł!achiralny}%

\begin{theorem}[Alexander, 1923]
\label{thm:alexander}
     Każdy splot powstaje przez domknięcie pewnego warkocza.
     \index{twierdzenie!Alexandera}
\end{theorem}

\begin{proof}[Niedowód]
\index[persons]{Alexander, James}%
\index[persons]{Morton, Hugh}%
\index[persons]{Yamada, Shuji}%
\index[persons]{Vogel, Pierre}%
    W kolejności chronologicznej:
    najpierw Alexander \cite{alexander23},
    a po blisko połowie wieku Morton \cite{mortonhr86},
    Yamada \cite{yamada87} (co daje łatwy do zaimplementowania program komputerowy)
    i~Vogel \cite{vogel90} (ulepszający algorym Yamady).
\end{proof}

Manturow \cite{manturov02} pokazał, że od warkocza można wymagać kwazitoryczności (warkocz nazywamy torycznym, jeżeli jest postaci $(\sigma_1 \ldots \sigma_{p-1})^q$ oraz kwazitorycznym, jeżeli powstaje przez odwrócenie niektórych skrzyżowań z~warkocza torycznego).
\index[persons]{Manturow, Wasilij}%
\index{warkocz!toryczny}%

\begin{theorem}[Markow, 1936]
\index{twierdzenie!Markowa}%
\label{markov_theorem}
    Dwa domknięte warkocze są równoważne jako sploty wtedy i~tylko wtedy,
    gdy jeden powstaje z~drugiego przez ciąg
    sprzężeń: $z_1 \mapsto z_2 z_1 z_2^{-1}$ oraz procesów Markowa,
    które zastępują $n$-warkocz $\beta$ przez $(n+1)$-warkocz $\beta\sigma_n^{\pm 1}$.
\end{theorem}

\begin{proof}
\index[persons]{Birman, Joan}%
\index[persons]{Menasco, William}%
\index[persons]{Morton, Hugh}%
\index[persons]{Traczyk, Paweł}%
    Kompletny i~godny naśladowania dowód znajduje się w~trudno dostępnej książce \cite{birman74} Birman, więc warto sprawdzić inne, opublikowane później materiały:
    Morton opisał w~\cite{mortonhr86} przepiękną, a~przy tym elementarną ideę ,,nitkowania'',
    potem Traczyk podał w~\cite{traczyk98} czysto kombinatoryczne, dwuwymiarowe uzasadnienie oparte o~okręgi Seiferta,
    wreszcie mamy też artykuł \cite{birman02} napisany przez Birman i~Menasco.
\end{proof}

Pierwsze sformułowanie twierdzenia pochodzące od Markowa \cite{markov36} korzystało z trzech ruchów, jeden z~nich stanowił uogólnienie II ruchu Reidemeistera.
Trzy lata później Weinberg zauważył w~\cite{weinberg39}, że wystarczą dwa ruchy.
\index[persons]{Weinberg, Noah}%
% Weinberg = Ной Вайнберг: http://www.mathsoc.spb.ru/history/Odynec_2020.pdf
Lambropoulou, Rourke przedstawili w~\cite{lambropoulou97} wersję twierdzenia wymagającą tylko jednego ruchu.
\index[persons]{Lambropoulou, Sofia}%
\index[persons]{Rourke, Colin}%

Historia twierdzenia Markowa jest raczej dramatyczna: Markow przedstawił swój dowód ustnie, ale nigdy go nie opublikował, zostawiając to zadanie swojemu uczniowi, Weinbergowi.
Ten jednak został zabity podczas wojny, wkrótce po opublikowaniu pierwszej pracy na temat teorii węzłów i na dokładny dowód trzeba było czekać do publikacji Birman \cite{birman74} blisko 40 lat.
\index[persons]{Birman, Joan}%

Twierdzenie \ref{markov_theorem} mówi nam, że teoria węzłów bada klasy równoważności w~grupie warkoczy.
Zarówno problem słowa (czy dwa słowa w~grupie przedstawiają ten sam jej element?) jak i~problem sprzężoności (czy dwa słowa w~grupie są sprzężone?) są rozwiązane, ale nadal nie mamy algorytmu, który mówiłby, czy dwa słowa w~grupie są równoważne w~sensie Markowa.
Cały czas chodzi o słowa w grupie warkoczy, oczywiście.
% TODO: kto to pokazał? wg Kawauchiego około strony 18, Murasugi w 1982 ale widziałem gdzieś informację, że Garside był pierwszy.
% a może Jones 1985?




\subsection{Reprezentacja Burau}
Na zakończenie sekcji wspomnijmy o~macierzowej reprezentacji Burau, wprowadzonej do matematyki w latach trzydziestych zeszłego wieku \cite{burau33}.
\index[persons]{Burau, Werner}%
\index{reprezentacja!Burau}%
Wyznaczona jest ona przez obrazy generatorów:
\begin{equation}
    \varphi(\sigma_i) = I_{i-1} \oplus \begin{pmatrix}
        1-t & t \\
        1   & 0
    \end{pmatrix} \oplus I_{n-i-1}
\end{equation}
Bezpośredni rachunek dowodzi, że reprezentacja $\varphi$ jest wierna dla $n \le 2$.
Magnus, Peluso \cite{peluso69} pokazali to samo dla $n = 3$, ale ich pracę czyta się tak trudno, że polecamy sięgnąć raczej po \cite[s. 129]{birman74} albo \cite[s. 110]{kassel08}.
Moody \cite{moody91} odkrył, że reprezentacja nie jest wierna dla $n \ge 9$, Long, Paton \cite{paton93} ulepszyli jego podejście i~poprawili jego wynik do $n \ge 6$.
\index[persons]{Moody, John}%
\index[persons]{Paton, Mark}%
\index[persons]{Long, Darren}%
% Paton = Mark https://www.genealogy.math.ndsu.nodak.edu/id.php?id=139714
Ich kontrukcja korzysta z~pewnej zamkniętej krzywej na sześciokrotnie przekłutym dysku o~pewnych cechach homologicznych.
Podobnymi metodami Bigelow pokazał u schyłku stulecia \cite{bigelow99}, że przypadek $n = 5$ też jest niewierny: jeśli
\index[persons]{Bigelow, Stephen}%
\begin{align}
    \psi_1 & = \sigma_3^{{-1}}\sigma_2\sigma_1^2\sigma_2\sigma_4^3\sigma_3\sigma_2, \\
\psi_2 & = \sigma_4^{{-1}}\sigma_3\sigma_2\sigma_1^{{-2}}\sigma_2\sigma_1^2\sigma_2^2\sigma_1\sigma_4^5,
\end{align}
to komutator $[\psi_1^{{-1}}\sigma_4\psi_1,\psi_2^{{-1}}\sigma_4\sigma_3\sigma_2\sigma_1^2\sigma_2\sigma_3\sigma_4\psi_2]$ należy do jądra.
Czy reprezentacja Burau dla $B_4$ jest wierna?
Negatywna odpowiedź na to pytanie prawie na pewno prowadziłaby do
nietrywialnego węzła, którego wielomianem HOMFLY jest $1$,
natomiast odpowiedź pozytywna raczej nie ma aż tak dramatycznych następstw.
% The first nonfaithful Burau representations were found by John A. Moody without the use of computer, using a notion of winding number or contour integration.[3] A more conceptual understanding, due to Darren D. Long and Mark Paton[4] interprets the linking or winding as coming from Poincaré duality in first homology relative to the basepoint of a covering space, and uses the intersection form (traditionally called Squier's Form as Craig Squier was the first to explore its properties).[5] Stephen Bigelow combined computer techniques and the Long–Paton theorem to show that the Burau representation is not faithful for n ≥ 5.[6][7][8] Bigelow moreover provides an explicit non-trivial element in the kernel as a word in the standard generators of the braid group: let




\subsection{Grupy warkoczy w algebrze}
Grupy $B_n$ mogą być obiektem badań algebry bez związku z~teorią węzłów.

\begin{proposition}
    Grupa warkoczy $B_n$ jest beztorsyjna dla każdego $n \ge 1$.
\end{proposition}

Istnieje wiele dowodów tego faktu: pierwszy korzystał z~krótkich ciągów dokładnych (Fadell, Neuwirth \cite{neuwirth62}), później podano oparty o~struktury Garside'a (Garside \cite{garside69}), czysto teoriogrupowy pochodzi od Dyera \cite{dyer80}.
\index[persons]{Dyer, Joan}%
\index[persons]{Fadell, Edward}%
\index[persons]{Garside, Frank}%
\index[persons]{Neuwirth, Lee}%
My przedstawimy inne rozumowanie, opisując przy tym ciekawy sam w~sobie porządek Dehornoya\footnote{W 1989 roku udowodniono, że pewien aksjomat teorii mnogości, $I_3$, dotyczący dużych liczb kardynalnych implikuje istnienie algebraicznej struktury zwanej acykliczną półką (nie mylić z naszymi półkami).
To pociąga decydowalność problemu słowa dla prawa $x(yz) = (xy)(xz)$, coś co nie jest wprost związane z dużymi liczbami kardynalnymi.
Dehornoy w 1992 roku podał przykład acyklicznej półki na grupie warkoczy $B_\infty$.}.% z artykułu "Dehornoy order na wiki"
\index{porządek Dehornoya}%

\begin{proof}
\index[persons]{Dehornoy, Patrick}%
    Mówimy, że grupa $G$ jest lewo-porządkowalna, jeśli można wyposażyć ją w~zupełny porządek $<$, niezmienniczy na mnożenie z lewej strony.
    To znaczy, dla każdych $a, b, c \in G$, z~nierówności $a < b$ wynika $ca < cb$.
    Wtedy zbiór $P = \{g \in G \mid e < g\}$ nazywamy półgrupą elementów dodatnich.
    Łatwo widać, że $G$ jest sumą rozłączną $P \sqcup \{e\} \sqcup P^{-1}$.
    Odwrotnie, każde takie rozbicie wyznacza porządek: wystarczy zdefiniować $a < b \iff a^{-1}b \in P$.

    Dehornoy znalazł taki porządek dla grupy warkoczowej $B_n$ w~\cite{dehornoy94}.
    Za zbiór $P$ elementów dodatnich wziął te słowa na standardowych generatorach, które dla pewnego $i$ zawierają $\sigma_i$, ale nie $\sigma_i^{-1}$ ani $\sigma_j^{\pm 1}$ dla $j < i$.
    Pokazanie, że $P$ jest półgrupą nie sprawia trudności, ale tego, że jest dobrze określonym zbiorem stanowi bardzo nietrywialne zadanie.

    Lewo-porządkowalna grupa jest beztorsyjna.
    Istotnie, ustalmy element $g \in G$ różny od elementu neutralnego.
    Bez straty ogólności niech $e < g$, przemnóżmy tę nierówność stronami przez $g$.
    Dostaniemy tak nową nierówność $g < g^2$.
    Powtarzając proces otrzymujemy łańcuch $e < g < g^2 < g^3 < \ldots$.
    Skoro $<$ jest porządkiem, nie jest możliwe by któryś z elementów $g^n$ był neutralny.
\end{proof}

Dowód uproszczono: Fenn, Greene, Rolfsen, Rourke i Wiest podali cztery lata później bezpośrednie geometryczne rozumowanie, które prowadzi do takiego samego porządku jak ten z pracy Dehornoya; patrz \cite{fenn99}.
\index[persons]{Fenn, Roger}%
\index[persons]{Greene, Joshua}%
\index[persons]{Rolfsen, Dale}%
\index[persons]{Rourke, Colin}%
\index[persons]{Wiest, Bertold}%
Dziesięć lat później dostaliśmy jeszcze pracę Bacardita, Dicksa \cite{bacardit09}
\index[persons]{Bacardit, Lluís}%
\index[persons]{Dicks, Warren}%

\begin{proposition}
    Grupa warkoczy $B_n$ jest grupą Hopfa dla każdego $n \ge 1$: nie jest izomorficzna z żadnym ze swoich właściwych ilorazów.
\end{proposition}

\begin{proof}
\index[persons]{Malcew, Anatolij}%
% https://en.wikipedia.org/wiki/Anatoly_Maltsev
\index[persons]{Magnus, Wilhelm}%
    Podręcznik \cite{magnus66} dobrze wyjaśnia różne idee stojące za dowodem, który podamy.

    Mówimy, że grupa $G$ jest rezydualnie skończona, jeśli przekrój jej podgrup skończonego indeksu jest trywialny.
    Łatwo widać, że własność ta przenosi się na wszystkie podgrupy grupy $G$.
    Baumslag zauważył, że jeśli grupa $G$ jest skończenie generowana i~rezydualnie skończona, to grupa jej automorfizmów $\operatorname{Aut} G$ jest rezydualnie skończona.
    Grupa $G = \Z^2$ spełnia te założenia.
    Wolna grupa $F_2$ jest podgrupą grupy automorfizmów $\Z^2$, na przykład
    \begin{equation}
        F_2 \simeq \left\langle
        \begin{pmatrix}
            1 & 2 \\
            0 & 1
        \end{pmatrix},
        \begin{pmatrix}
            1 & 0 \\
            2 & 1
        \end{pmatrix}
        \right\rangle \subseteq \operatorname{Aut} \Z^2.
    \end{equation}
    Wszystkie grupy wolne $F_n$, $n \in \N$, są podgrupami grupy $F_2$, dlatego także są rezydualnie skończone, a z nimi grupa warkoczy, gdyż $B_n \subseteq \operatorname{Aut} F_n$.

    Malcew pokazał, że skończenie generowana i~rezydualnie skończona grupa jest grupą Hopfa.
    Krótki dowód tego faktu można znaleźć w~sekcji 6.5 książki Magnusa \cite{magnus66}.
\end{proof}

Czy grupy warkoczy są liniowe?
Przez długi czas był to problem otwarty, potem Krammer \cite{krammer00} pokazał, że $B_4$ jest liniowa, następnie metodami topologicznymi Bigelow \cite{bigelow01} dowiódł, że wszystkie grupy $B_n$ są liniowe, a wkrótce po tym Krammer \cite{krammer02} uogólnił swoje algebraiczne rozumowanie także do wszystkich grup $B_n$.
\index[persons]{Krammer, Daan}%
\index[persons]{Bigelow, Stephen}%


\subsection{Liczba warkoczowa}
\index{liczba warkoczowa|(}%

% DICTIONARY;braid number;liczba warkoczowa;-
\begin{definition}
\label{def:braid_number}%
    Liczba warkoczowa to minimalna liczba pasm, na których można zapleść warkocz, którego domknięciem jest wyjściowy splot.
\end{definition}

Tylko jeden węzeł ma liczbę warkoczową $1$, jest to niewęzeł.
Dwuwarkoczowe są dokładnie węzły torusowe typu $(2, n)$ dla $|n| \ge 3$.
Węzły spełniające $\braid (K) = 3$ nie zostały jeszcze sklasyfikowane.
Liczba warkoczowa splotu zależy od orientacji ogniw i~trudno wyznacza się w~ogólnym przypadku.
Znamy ją między innymi dla węzłów torusowych (fakt \ref{cor:torus_braid_number}).

\begin{proposition}
    Węzeł o~$n$ skrzyżowaniach można zapleść na $n - 1$ pasmach: $\crossing K \ge 1 + \braid K$.
\end{proposition}

Powyższe ograniczenie nie jest zbyt użyteczne, równość mamy jedynie dla trójlistnika i~ósemki.
Ohyama \cite{ohyama93} pokazał:
\index[persons]{Ohyama, Yoshiyuki}%

\begin{proposition}
    Niech $L$ będzie nierozszczepionym\footnote{a może nierozszczepialnym?} splotem.
    Wtedy $\crossing L \ge 2 \braid L - 2$.
\end{proposition}

Dowód korzysta z grafu Seiferta splotu.
\index{graf Seiferta}%
Wśród pierwszych węzłów do 10 skrzyżowań mamy równość dziesięć razy: dla $4_1$, $6_1$, $8_1$, $8_3$, $8_{12}$, $10_1$, $10_3$, $10_{13}$, $10_{35}$, $10_{58}$.
%=% PANDAS
%=% >>> knots = [x for x in d.query('crossing_number == 2 * braid_index - 2 and crossing_number < 11')["name"]]; print(len(knots), knots)
%=% 10 ['4_1', '6_1', '8_1', '8_3', '8_12', '10_1', '10_3', '10_13', '10_35', '10_58']

\begin{proposition}[nierówność Mortona-Franksa-Williamsa]
\index{nierówność Mortona-Franksa-Williamsa}%
    Niech $L$ będzie zorientowanym splotem, zaś $\operatorname{span}_\alpha$ różnicą między największym i najmniejszym stopniem zmiennej $\alpha$ wielomianu HOMFLY zmiennych $\alpha, z$.
    Wtedy
    \begin{equation}
        \braid L \ge 1 + \frac 1 2 \operatorname{span}_\alpha P(\alpha, z).
    \end{equation}
\end{proposition}

Nierówność jest ostra dla wszystkich pierwszych węzłów do 10 skrzyżowań poza $9_{42}$, $9_{49}$, $10_{132}$, $10_{150}$ oraz $10_{156}$.
Dowód podali niezależnie Franks, Williams \cite{franks87} i Morton \cite{morton88}.
\index[persons]{Franks, John}%
\index[persons]{Morton, Hugh}%
\index[persons]{Williams, Robert}%
% TODO: zweryfikować z ./tools/knotinfo_parsed.json programem w Pythonie

Wielomian Alexandera wykrywa czasami węzły, których nie otrzyma się przez domykanie ,,małych'' warkoczy.
\index{wielomian!Alexandera}%
Przytoczone tu wyniki pochodzą z pracy \cite{jones85} Jonesa, gdzie nie ma jednak ich dowodów.

\begin{proposition}
    Jeśli $|\alexander_K(i)| > 3$, to węzeł $K$ nie jest domknięciem 3-warkocza.
\end{proposition}

Ta implikacja, \cite[wniosek 23]{jones85}, jest skuteczna przy 43 z 59 węzłów pierwszych o mniej niż 10 skrzyżowaniach.
Jones \cite[wniosek 24]{jones85} zasugerował, że domknięcia 4-warkoczy spełniają podobną nierówność $|\alexander (\exp (2\pi i / 5))| \le 13/2$, ale Stojmenow \cite{stoimenow02} odkrył po wielu latach, dlaczego nikt nigdy nie podał dowodu tego faktu.
Węzeł $13_{9221}$ ma liczbę warkoczową 4, jego wielomian Alexandera przyjmuje wartość $\alexander(\omega_5) = 19\sqrt{5} - 49 \approx -6.51$.

Udało mu się za to udowodnić słabszą implikację ($6 + 2 \sqrt{5} \approx 10.47$):

\begin{proposition}
    Jeśli $|\alexander_K(\exp(2\pi i/5))| > 6 + 2 \sqrt{5}$, to węzeł $K$ nie jest domknięciem 4-warkocza.
\end{proposition}

Prawdopodobnie nie istnieją podobne warunki dla 5-warkoczy.

\index{liczba warkoczowa|)}%

% Koniec podsekcji Liczba warkoczowa



\index{warkocz|)}%

% Koniec sekcji warkocze



\section{Supły}
\label{sec:tangle}
Na przełomie lat sześćdziesiątych i~siedemdziesiątych Conway szukał sposobu na zbudowanie kompletnej tablicy węzłów.
Niezmienniki znane w~tym czasie nie były dostatecznie mocne, by sprostać temu wyzwaniu.
Conway wprowadził pojęcie supła i~chociaż wszystkich węzłów nie można z~nich uzyskać, teoria została pchnięta do przodu.
Supły stanowią budulec splotów takich jak na przykład precle z~definicji \ref{def:pretzel}.

Sekcja oparta jest na podręczniku Murasugiego \cite{murasugi96} i~pracach Conwaya \cite{conway70}, Kauffmana, Goldmana \cite{kauffman97}, Kauffmana, Lambropoulou \cite{kauffman04}, a~także Schuberta \cite{schubert56}.
\index[persons]{Conway, John}%
\index[persons]{Goldman, Jay}%
\index[persons]{Kauffman, Louis}%
\index[persons]{Lambropoulou, Sofia}%
\index[persons]{Schubert, Horst}%

Supły występują także w polskojęzycznym artykule Janiak-Osajcy, Pogody \cite{janiak04}, ale ten zawiera nieprzyjemną pułapkę: wprowadza notację sprzeczną z~powszechnie akceptowaną.
\index[persons]{Janiak-Osajca, Agnieszka}%
\index[persons]{Pogoda, Zdzisław}%

% DICTIONARY;tangle;supeł;-
\begin{definition}[supeł]
    \label{def:tangle}
    \index{supeł}
    Zawarty w~kole fragment diagramu splotu o~dwóch łukach wyjściowych oraz dwóch wejściowych, nazywamy supłem.
\end{definition}

% z: AMPHICHEIRALS ACCORDING TO TAIT AND HASEMAN
Słowo ,,supeł'' zaproponowała Haseman, już ona rysowała supły wewnątrz pomocniczego okręgu, który tnie diagram w~czterech punktach.
\index[persons]{Haseman, Mary}%

Istnieją dwa rodzaje supłów:
\begin{comment}
\begin{figure}[H]
    \centering
    \begin{minipage}[b]{.48\linewidth}
        \[\LargeTangleAlternatingYes\]
        \subcaption{supeł naprzemienny}
    \end{minipage}
    \begin{minipage}[b]{.48\linewidth}
        \centering
        \[\LargeTangleAlternatingNo\]
        \subcaption{supeł sąsiądujący}
    \end{minipage}
\end{figure}
\end{comment}

Podobnie jak dla węzłów, pojawia się naturalne pytanie o~równoważność dwóch supłów.
Jest tak wtedy, gdy istnieje homeomorfizm kuli na siebie, który przekształca jeden supeł na drugi, ale nie rusza sfery otaczającej.
Dla diagramów odpowiada to ruchom Reidemeistera, nie mamy jednak prawa opuszczać kuli zawierającej supeł.

Dużo dokładniej mówi o tym Turajew \cite{turaev90}:
\index[persons]{Turajew, Władimir}%

\begin{proposition}
    Oznaczmy przez OTa kategorię zorientowanych supłów.
    Jej obiektami są skończone ciągi złożone z~$\pm 1$, razem z~ciągiem pustym.
    Morfizm ciągu $\varepsilon = (\varepsilon_1, \ldots, \varepsilon_k)$ w ciąg $\nu = (\nu_1, \ldots, \nu_l)$ jest klasą izotopii zorientowanego $(k, l)$-supła $L$ tak, że źródłem $L$ jest $\varepsilon$, zaś celem $\nu$.
    Na przykład supły $\curvearrowright$, $\curvearrowleft$ oraz $X_+$ opisane są przez morfizmy $\varnothing \to (-1, 1)$, $\varnothing \to (1, -1)$ oraz $(1, 1) \to (1, 1)$.
    Składanie morfizmów odpowiada mnożeniu supłów.

    W kategorii OTa wprowadza się iloczyn tensorowy $\otimes$. Iloczynem obiektów $\varepsilon, \nu$ (znaczących to, co wyżej) jest obiekt $(\varepsilon_1, \ldots, \varepsilon_k, \nu_1, \ldots, \nu_l)$.
    Iloczyn tensorowy morfizmów to iloczyn tensorowy splotów i łatwo widać, że $(OTa, \otimes, \varnothing)$ jest ściśle monoidalną kategorią.

    Zdefiniujmy cztery słowa:
    \begin{align}
        A & = (\downarrow \, \downarrow \, \curvearrowright) \circ (\downarrow \, \downarrow \, \uparrow \, \curvearrowright \, \downarrow) \circ (\downarrow \, \downarrow X_\pm \downarrow \, \downarrow) \circ (\downarrow \inversedcurvearrowright \, \uparrow \, \downarrow \, \downarrow) \circ (\inversedcurvearrowright \downarrow \, \downarrow) \\
        B & = (\curvearrowleft \, \downarrow \, \downarrow) \circ (\downarrow \, \curvearrowleft \, \uparrow \, \downarrow \, \downarrow) \circ (\downarrow \, \downarrow X_\pm \downarrow \, \downarrow) \circ (\downarrow \, \downarrow \, \uparrow  \inversedcurvearrowleft \downarrow) \circ (\downarrow \, \downarrow \inversedcurvearrowleft) \\
        T & = (\curvearrowleft \, \uparrow \, \downarrow) \circ (\downarrow X_- \downarrow) \circ (\downarrow \, \uparrow \inversedcurvearrowleft) \\
        Y & = (\uparrow \, \downarrow \, \curvearrowright) \circ (\downarrow X_+ \downarrow) \circ (\inversedcurvearrowright \uparrow \, \downarrow)
    \end{align}
    Kategoria OTa jest generowana przez morfizmy $\inversedcurvearrowright, \inversedcurvearrowleft, \curvearrowright, \curvearrowleft, X_+, X_-$ oraz przedstawiana przez nie razem z relacjami:
    \begin{align}
        (\curvearrowright \, \uparrow) \circ (\uparrow \inversedcurvearrowright) = & \uparrow \, = (\uparrow \, \curvearrowleft) \circ (\inversedcurvearrowleft \uparrow) \\
        (\curvearrowleft \, \downarrow) \circ (\downarrow \inversedcurvearrowleft) = & \downarrow \, = (\downarrow \, \curvearrowright) \circ (\inversedcurvearrowright \downarrow) \\
        A & = B \\
        X_+ \circ X_- & = X_- \circ X_+ = \, \uparrow \, \uparrow \\
        (X_+ \uparrow) \circ (\uparrow X_-) \circ (X_+ \uparrow) & = (\uparrow X_+) \circ (X_+ \uparrow) \circ (\uparrow X_+) \\
        (\uparrow \, \curvearrowright) \circ (X_\pm \downarrow) \circ (\uparrow \inversedcurvearrowleft) & = \, \uparrow \\
        Y \circ T = \, \downarrow \, \uparrow, & \quad T \circ Y = \, \uparrow \, \downarrow
    \end{align}
\end{proposition}


\begin{proof}
    Dowód twierdzenia oraz graficzne przedstawienie relacji z kategorii OTa zawiera praca Turajewa \cite{turaev90}.
    Wszystkie relacje odpowiadają ruchom Reidemeistera.
\index{ruchy Reidemeistera}%
    Trzecie od końca równanie to geometryczny wariant równania Yanga-Baxtera.
\index{równanie Yanga-Baxtera}%
% TODO: przerysować... do kodu

Patrz też \cite[s. 29-30]{duzhin12} (Czmutow, Dużin, Mostovoy przygotowali tam śliczne rysunki) albo \cite[s. 31]{schieber18} (gdzie Schieber przedstawił ruchy Reidemeistera i~cięte diagramy).
\index[persons]{Czmutow, Siergiej}%
\index[persons]{Dużin, Siergiej}%
\index[persons]{Mostovoy, Jacob}%
\index[persons]{Schieber, Nathaniel}%
% sliced diagrams
% DICTIONARY;sliced;cięty;diagram
\end{proof}

Wszystkich supłów jest bardzo dużo, więc ograniczymy się do końca rozdziału do pewnej ich regularnej rodziny.
Oto cztery podstawowe supły:
\begin{comment}
\begin{figure}[H]
    \centering
    \begin{minipage}[b]{.23\linewidth}
        \[
            \LargeTangleBasicZero
        \]
        \subcaption{$(0)$}
    \end{minipage}
    \begin{minipage}[b]{.23\linewidth}
        \[
            \LargeTangleBasicInfinity
        \]
        \subcaption{$(\infty) = (0, 0)$}
    \end{minipage}
    \begin{minipage}[b]{.23\linewidth}
        \[
            \LargeTangleBasicMinus
        \]
        \subcaption{$(-1)$}
    \end{minipage}
    \begin{minipage}[b]{.23\linewidth}
        \[
            \LargeTangleBasicPlus
        \]
        \subcaption{$(+1)$}
    \end{minipage}
\end{figure}
\end{comment}

\begin{definition}
    Supły powstające z~$(0)$ lub $(\infty)$ przez homeomorfizm kuli na siebie permutujący wejścia i~wyjścia nazywamy wymiernymi.
\end{definition}

Pokażemy teraz, jak zamienić dowolny skończony ciąg liczb całkowitych w~supeł, jako że jest to prostsze od procesu odwrotnego.
Nazwijmy jednak najpierw dwa rodzaje skrętów:
\begin{comment}
\begin{figure}[H]
    \centering
    \begin{minipage}[b]{.48\linewidth}
        \[\LargeTwistsRight\]
        \subcaption{skręty prawe}
    \end{minipage}
    \begin{minipage}[b]{.48\linewidth}
        \centering
        \[\LargeTwistsLeft\]
        \subcaption{skręty lewe}
    \end{minipage}
\end{figure}
\end{comment}

Mając ciąg $(a_1, a_2, \ldots, a_n)$ wykonujemy naprzemiennie obroty półsferą dolną (SW--SE, takie nazywamy pionowymi) oraz prawą (SW--NW, a takie poziomymi) tak, by ostatni był obrót poziomy.
Oto reguła zgodnie z którą wybieramy kierunek obrotów.
Podczas pionowych obrotów, prawy skręt jest dodatni, zaś lewy ujemny.
Podczas poziomych, zamieniamy znaki: prawy odpowiada ujemnym wyrazom ciągu, lewy dodatnim.
Wreszcie, jeżeli $n$ jest nieparzyste, zaczynamy od supła $T(0)$, w przeciwnym razie od supła $T(0, 0)$.

Różnym ciągom mogą odpowiadać te same supły, na przykład $T(-2, 3, 3) = T(3, -2)$, więc notacja nie jest jednoznaczna, ale to nic złego.
Każdemu supłowi przypiszmy pewną liczbę wymierną, według przepisu:
\begin{equation}
    T(a_1, a_2, \ldots, a_n) \mapsto a_n + \frac{1}{\ldots + 1/a_1} = \frac \alpha \beta.
\end{equation}

\begin{proposition}
    Istnieje bijekcja między typami supłów wymiernych oraz ułamkami łańcuchowymi.
\end{proposition}

\begin{proof}[Niedowód]
    Praca \cite{conway70} Conwaya, strony 331-332.
\end{proof}

\begin{proposition}[ćwiczenie 9.2.6 w \cite{murasugi96}]
    \label{prp:continued_fractions}
    Niech $T(a_1, a_2, \ldots, a_n)$ będzie supłem różnym od $0$ oraz $\infty$.
    Wtedy bez straty ogólności można założyć, że wszystkie liczby $a_i$ są tego samego znaku.
\end{proposition}

Z każdym supłem $T$ związane jest jego odbicie $\overline T$, obraz wyjściowego przez symetrię względem prostej $y = -x$.
Mając dwa supły obok siebie, można dokonać ich sklejenia wzdłuż połówek kul, w~których leżą:
\begin{comment}
\begin{figure}[H]
    \centering
    \begin{minipage}[b]{.23\linewidth}
        \[
            \LargeTangleSummandA
        \]
        \subcaption{jakiś supeł}
    \end{minipage}
    \begin{minipage}[b]{.23\linewidth}
        \centering
        \[
            \LargeTangleSummandB
        \]
        \subcaption{jakiś inny supeł}
    \end{minipage}
    \begin{minipage}[b]{.48\linewidth}
        \centering
        \[
            \LargeTangleSumAB
        \]
        \subcaption{suma tych supłów}
    \end{minipage}
\end{figure}
\end{comment}

Oznaczmy tak otrzymany splot przez $T_1 + T_2$.
Niektórzy definiują dalsze działania, jak produkt: $T_1 \cdot T_2 = \overline T_1 + T_2$ czy rozgałęzienie, $\overline T_1 + \overline T_2$.
Rodzina supłów wymiernych jest zamknięta na branie produktów, ale nie sum.
Wprowadzamy więc następującą, ogólniejszą definicję.
Supeł będący skończoną sumą supłów wymiernych, ich luster, odbić lub odbić luster nazywamy algebraicznym.

\begin{tobedone}[notacja Conwaya]
    % TODO: when changing tobedone to something else, remember to add prp: or other prefix!
    \label{conway_notation}
    ???
\end{tobedone}

Przez zszycie par łuków wejściowych (lub wyjściowych) zamieniamy supły w~węzły:
\begin{figure}[H]
    \centering
    \begin{minipage}[b]{.3\linewidth}
        \centering
        \LargeTangleFraction
        \subcaption{supeł $T$}
    \end{minipage}
    \begin{minipage}[b]{.3\linewidth}
        \centering
        \LargeTangleFractionNumerator
        \subcaption{licznik, $N(T)$}
    \end{minipage}
    \begin{minipage}[b]{.3\linewidth}
        \centering
        \LargeTangleFractionDenominator
        \subcaption{mianownik, $D(T)$}
    \end{minipage}
\end{figure}

% DICTIONARY;... numerator;licznik ...;supeł
% DICTIONARY;... denominator;mianownik ...;supeł
Oznaczenia $N(T)$ oraz $D(T)$ pochodzą od angielskich słów \emph{numerator}, \emph{denominator}.
Być może nie jest jasne, dlaczego terminy stosowane zazwyczaj do opisu ułamków stosujemy wobec diagramów splotów.
Nazewnictwo nie jest przypadkowe.

\begin{proposition}
    Ułamek supła zadany wzorem
    \begin{equation}
        F(A) = \frac{\conway_{N(A)}(z)}{\conway_{D(A)}(z)}
    \end{equation}
    spełnia zależność $F(A+B) = F(A) + F(B)$.
\end{proposition}

\begin{proof}
    Praca \cite{conway70} Conwaya.
\end{proof}

Praca \cite{conway70} zawiera jeszcze jeden ciekawy rezultat, uogólniony przez Lickorisha i~Milletta w~\cite[fakt 12]{lickorish87}.
Używamy tu wersji wielomianu HOMFLY o zmiennych $l, m$.

\begin{proposition}
    Niech $A, B$ będą supłami, zaś $T_n$ (odpowiednio: $T_d$)  wielomianem HOMFLY licznika (mianownika) supła $T$.
    Wtedy
    \begin{equation}
        (\mu^2 - 1)(A+B)_n = \mu(A_nB_n + A_dB_d) - (A_nB_d + A_dB_n),
    \end{equation}
    gdzie $\mu = -(l + 1/l)/m$. % oraz
    %\begin{equation}
        %(A+B)_d = A_dB_d.
    %\end{equation}
    % TODO: tego nie potrafię znaleźć w pracy Lickorisha
\end{proposition}


\subsection{Sploty o~dwóch mostach}
\label{sub:twobridge}%
\index{węzeł!wymierny|see {węzeł dwumostowy}}%
\index{węzeł!dwumostowy|(}%
Zajmiemy się teraz związkiem supłów z liczbą mostową.
Wiemy, że węzeł trywialny jest jednomostowy, następne w hierarchii są sploty dwumostowe.
Nazywa się je także wymiernymi, po angielsku czasami \emph{4-plats}.
Jako pierwszy studiował je Bankwitz z~Schumannem w~1934 roku.
% Kawauchi: as 4-plat presentations, which is just Conway's normal form.
Mają co najwyżej dwie składowe i~są odwracalne \cite[s. 211]{burde14}.
Patrz też \cite[s. 21-26]{kawauchi96}.

\begin{proposition}
    Sploty dwumostowe są pierwsze.
\end{proposition}

\begin{proof}
    Prosty wniosek z~tego, że liczba mostowa prawie jest addytywna (fakt \ref{prp:bridge_additive}).
\end{proof}

\begin{corollary}
    Pierwsze węzły dwu- lub trzymostowe są albo torusowe, albo hiperboliczne.
\end{corollary}

\begin{proof}
    Kawauchi \cite[s. 130]{kawauchi96} wnioskuje to z twierdzenia, którego nie znamy.
\end{proof}

%\todo[inline]{Murasugi Theorem 9.3.3 (138) lub Janiak-Osajca, Pogoda (34).}
% Aus der unten stehenden Klassifikation ergibt sich, dass man jede Verschlingung mit 2 Brücken wie im Bild rechts darstellen kann, wobei {\displaystyle a_{i}\in \mathbb {Z} } a_{i}\in \mathbb{Z }  die Anzahl der Halbtwists in der jeweiligen Box bezeichnet und für gerade bzw. ungerade {\displaystyle i} i~positive {\displaystyle a_{i}} a_{i} links- bzw. rechtshändigen Halbtwists entsprechen.
% Diese Darstellung wird als Conway-Normalform bezeichnet.
% Man kann stets erreichen, dass alle {\displaystyle a_{i}} a_{i} dasselbe Vorzeichen haben.[1] Insbesondere gibt die Conway-Normalform dann ein alternierendes Knotendiagramm.[2]
%Insbesondere ist ein 2-Brücken-Knoten genau dann amphichiral, wenn {\displaystyle q^{2}\equiv -1\ mod\ p} q^{2}\equiv -1\ mod\ p ist.

\begin{proposition}
    Sploty z~dwoma mostami to dokładnie sploty typu $D(T)$ dla pewnego supła wymiernego $T$.
\end{proposition}

Dowód tego stwierdzenia znaleźć można na przykład w książce \cite{murasugi96}, strony 183-187.
Wynika z niego, że każdy splot dwumostowy można przedstawić następującym diagramem:
\input{50-families/tangle_05}

Oto reguła, zgodnie z~którą wybieramy znaki liczb $a_i$:
jeśli $i$ jest nieparzyste, prawy skręt jest dodatni, jeśli parzyste -- lewy jest dodatni.
Sam diagram oznaczamy $C(a_1, \ldots, a_{2k+1})$ i~nazywamy postacią normalną Conwaya.

% Conway Normal Form: kawauchi96, strona 24

\begin{proposition}
    % Murasugi proposition 9.3.2
    Sploty dwumostowe są alternujące.
\end{proposition}

\begin{proof}
    Goodrick w~\cite{goodrick72} podał diagramatyczny dowód, gdzie ciąg ruchów zmienia diagram splotu dwumostowego w~alternujący.
    Wynika to też z faktu \ref{prp:continued_fractions}.
    % Burde, Zieschang 2013, strona 217, nazywają to twierdzeniem Bankwitza-Schumanna.
\end{proof}

Przez analogię do supłów, definiujemy ułamek łańcuchowy
\begin{equation}
    C(a_1, \ldots, a_{2k+1}) \mapsto a_1 + \frac{1}{a_2 + 1/\ldots} = \frac \alpha \beta.
\end{equation}

\begin{tobedone}
\index{postać normalna!Conwaya}%
\index{postać normalna!Schuberta}%
    To jest postać normalna Conwaya, ale mamy jeszcze postać Schuberta - \cite[s. 21]{kawauchi96}.
\end{tobedone}

Zauważmy, że wartość bezwzględna ułamka $\alpha/\beta$ zawsze przekracza $1$ i~odwrotnie, każdy taki ułamek pochodzi od pewnego węzła dwumostowego.
Parę względnie pierwszych liczb $(\alpha, \beta)$ nazywamy typem węzła dwumostowego.

\begin{proposition}
    \label{prp:tangle_equivalence}
    Dwumostowe sploty typów $(\alpha, \beta)$ oraz $(\alpha', \beta')$ są równoważne wtedy i tylko wtedy, gdy spełnione są warunki:
    \begin{equation}
        \begin{cases}
            \alpha = \alpha' \\
            \beta^{\pm 1} \equiv \beta' \pmod {2 \alpha}
        \end{cases}
    \end{equation}
    Gdyby rozpatrywać niezorientowane sploty, drugie przystawanie upraszcza się: wystarczy, że będzie zachodzić modulo $\alpha$.
\end{proposition}

\begin{proof}
    Słabsza wersja twierdzenia bierze się z klasyfikacji przestrzeni soczewkowych oraz tego, że podwójnie rozgałęziona przestrzeń nakryciowa dwumostowego splotu typu $(\alpha, \beta)$ to przestrzeń soczewkowa typu $(\alpha, \beta)$.
    % two-fold branched cyclic spaces?
    (Nie definiujemy w~tej książce, czym są przestrzenie soczewkowe).
\index{przestrzeń!soczewkowa}%
    Burde, Zieschang \cite[s. 212]{burde14} wspominają tu prace Reidemeistera \cite{reidemeisterXX}, Brody'ego \cite{brodyXX} oraz Turajewa \cite{turaevXX}.
    % TODO: Reidemeister, K., 1935: Homotopieringe und Linsenräume. Abh. Math. Sem. Univ. Hamburg, 11 (1935), 102–109
    % TODO: Brody, E. J., 1960: The topological classification of lens spaces. Ann. of Math., 71 (1960), 163–184

    Dowód mocnej wersji znajduje się u Schuberta \cite{schubert56} albo Turajewa \cite{turaevXX}, a zapewne także Murasugiego \cite[s. ?]{murasugi96}.
    % TODO: napisałem ?, bo nie wiem która strona i nie chce mi się dziś sprawdzać
    % TODO: schubert56 - to ma prawie 40 stron!
    % TODO: nie mam bibliografii do zieschanga-2013, a w zieschangu-2003 nie ma turaeva, więc nie wiem, co tu tak naprawdę jest cytowane :D
\end{proof}

\begin{proposition}
    Dwumostowy splot typu $(\alpha, \beta)$ jest achiralny dokładnie wtedy i tylko wtedy, gdy
    \begin{equation}
        \beta^2 \equiv -1 \mod \alpha.
    \end{equation}
\end{proposition}

\begin{proof}
    Wynika to z tego, że lustrem splotu typu $(\alpha, \beta)$ jest splot typu $(\alpha, -\beta)$ oraz faktu \ref{prp:tangle_equivalence}.
    Explicite pisze o tym Kawauchi \cite[s. 24]{kawauchi96}.
\end{proof}

\begin{proposition}
    Niech $b$ będzie dowolną liczbą całkowitą.
    Wtedy następujące sploty są tego samego typu:
    \begin{align}
        N(T(a_1, a_2, \ldots, a_{2k+1})) & \approx N(T(a_1, a_2, \ldots, a_{2k+1}, b, 0)) \\
                                         & \approx D(T(-a_1, -a_2, \ldots, -a_{2k+1}, b)) \\
                                         & \approx C(a_1, a_2, \ldots, a_{2k}-1, 1). \\
        N(T(a_1, a_2, \ldots, a_{2k}))   & \approx D(T(-a_1, -a_2, \ldots, -a_{2k}, b)) \\
                                         & \approx C(a_1, a_2, \ldots, a_{2k}-1, 1). \\
        D(T(a_1, a_2, \ldots, a_{2k+1})) & \approx D(T(a_1, a_2, \ldots, a_{2k}, 0)) \\
                                         & \approx C(1, a_1-1, a_2, \ldots, a_{2k}). \\
        D(T(a_1, a_2, \ldots, a_{2k}))   & \approx D(T(a_1, a_2, \ldots, a_{2k-1}, 0)) \\
                                         & \approx C(a_1, a_2, \ldots, a_{2k-1}).
    \end{align}
\end{proposition}

\begin{proof}
    \cite[fakt 9.3.4]{murasugi96}
\end{proof}

\begin{proposition}
    Niech $L$ będzie dwumostowym splotem typu $(\alpha, \beta)$.
    Wtedy $\det L = \alpha$.
\end{proposition}

Wynika stąd, że wyznacznik nie wystarcza do odróżniania splotów dwumostowych.

\begin{proof}
    % Chcąc oszczędzić niektórym Czytelnikom cierpień odsyłamy po prostu do \cite{schubert56}.
    \url{https://math.stackexchange.com/questions/3327846/}.
\end{proof}

Niech $A, B$ będą supłami.
Wiemy, że suma $A+B$ nie musi być supłem, zaś $D(A+B)$ niekoniecznie jest splotem dwumostowym.
Pomimo to, splot $N(A+B)$ jest dwumostowy, potrafimy nawet powiedzieć, jaki ma wyznacznik:

\begin{proposition}
    % Theorem 9.3.5 Murasugi

    Niech $A, B$ będą supłami, którym odpowiadają skrócone ułamki $p/q$ oraz $r/s$.
    Wtedy splot $L = N(A+B)$ jest dwumostowy, typu $(\alpha, \beta)$ i ma wyznacznik $\alpha = |ps + qr|$.
\end{proposition}

Murasugi (twierdzenie 9.3.5) twierdzi, że dowód znajduje się u Ernsta, Sumnersa \cite{ernst90}.
\index[persons]{Ernst, Claus}%
\index[persons]{Sumners, De Witt}%

\begin{proposition}
    Rozpatrzmy węzeł dwumostowy typu $(\alpha, \beta)$, gdzie $0 < \beta < \alpha$ i~$\beta$ jest nieparzyste.
    Niech $r_k$ będzie resztą z~dzielenia $k\beta$ przez $2\alpha$ leżącą w~przedziale $(-\alpha, \alpha)$ dla $k = 0, 1, \ldots, \alpha - 1$.
    Różnica między ilością dodatnich reszt i~ujemnych reszt to sygnatura węzła.
\end{proposition}

Wygląda na to, że jedynym niewyznaczonym do końca klasycznym niezmiennikiem jest liczba gordyjska.

\index{węzeł!dwumostowy|)}%

% Koniec podsekcji Sploty o~dwóch mostach




\subsection{Mutanty i mutacje}
\index{mutant|(}%
\index{mutacja|see {mutant}}%
\label{sec:mutant}%
Na zakończenie wspomnimy o~mutacjach.

\begin{definition}[mutacja]
    % \labelnotinuse{def:mutacja}
    Półobrót supła względem osi poziomej, pionowej albo też prostopadłej do płaszczyzny, w~jakiej leży diagram, nazywamy mutacją.
    W razie potrzeby zmieniamy orientację supła na przeciwną.
\end{definition}

\begin{definition}[mutant]
\label{def:mutant}%
    Niech $K$ będzie węzłem.
    Węzeł, który powstaje przez wykonanie ciągu mutacji na węźle $K$, nazywamy mutantem węzła $K$.
\end{definition}

Mutacja węzła o~co najwyżej dziesięciu skrzyżowaniach nie zmienia jego klasy abstrakcji.
Najprostszą, a zarazem najsłynniejszą parą różnych od siebie mutantów stanowią węzeł Conwaya $11n_{34}$ oraz Kinoshity-Terasakiego $11n_{42}$.
\index{węzeł!Conwaya}%
\index{węzeł!Kinoshity-Terasakiego}%

\begin{figure}[H]
\centering
\includegraphics[width=0.601\linewidth]{../data/mixed/knudemutation.png}
\caption{Węzły $11n_{42}$ oraz $11n_{34}$. Grafika Sørena Jørgensena, dostępna na licencji \href{https://creativecommons.org/licenses/by-sa/3.0/deed.en}{CC BY-SA 3.0} pod adresem \url{https://en.wikipedia.org/wiki/File:Knudemutation.svg}.}
\end{figure}
Conway zauważył podczas klasyfikacji niealternujących węzłów, że tylko one posiadają trywialny wielomian Alexandera.
Mają też taki sam wielomian Jonesa,
\begin{equation}
    \jones(t) = t^{6} -2t^5 +2t^4 -2t^3 +t^2 +2t^{-1} -2t^{-2} +2t^{-3} -t^{-4}.
\end{equation}
Kinoshita, Terasaki zdefiniowali nieskończoną rodzinę węzłów o trywialnym wielomianie Alexandera, której pierwszym wyrazem jest węzeł $11n_{42}$ w~\cite{kinoshita57}.
\index{człowiek!Kinoshita, Shinichi}%
\index{człowiek!Terasaka, Hidetaka}%
Dowód tego, że $11n_{34}$ oraz $11n_{42}$ są różne, jako pierwszy podał prawdopodobnie Riley w~1971 roku \cite{riley71}: wykorzystał on homomorfizmy z~grupy węzła w~$PSL(2, 7)$.
\index{człowiek!Riler, ?}%
%=% "The inequivalence of these knots was first observed by [Riley 1971]." - Kawauchi, strona 44
Genusy, odpowiednio: $3$ i~$2$, wyznaczył Gabai piętnaście lat później w~\cite{gabai86}, używał foliacji.
\index{człowiek!Gabai, David}%

Niedawno Stojmenow podjął się systematycznie szukania mutantów wśród węzłów o~mniej niż 19 skrzyżowaniach (praca \cite{stoimenow10} z~2010 roku).
\index{człowiek!Stoimenow, Alexander}%
Początkowo pracował sam, badając pewne subtelne przykłady postanowił uwikłać w swój projekt Toshifumiego Tanakę, a później także Daniela Mateię.
\index{człowiek!Tanaka, Toshifumi}%
\index{człowiek!Daniel, Matei}%
Praca \cite{stoimenow10} jest kontynuacją artykułu, który napisali wspólnymi siłami.

I tak na stronie 531 można przeczytać, że ,,niezmienniki Wasiljewa co najwyżej 8. stopnia nie rozróżniają mutantów węzłów \cite{chmutov94}'', ja tego nie widzę.
\index{niezmiennik!Wasiljewa}%
Mniej więcej sześć lat później wynik poprawił J. Murakami (nie mylić z H. Murakamim!) do 10. stopnia w~niezindeksowanej pracy \cite{murakami99}.
\index{człowiek!Murakami, J}
W międzyczasie Cromwell, Morton znaleźli niezmiennik stopnia 11., który odróżnia węzły Conwaya oraz Kinoshity-Terasakiego; patrz \cite{cromwell96}.
\index{człowiek!Cromwell, ?}%
\index{człowiek!Morton, ?}%
% czy Murakami potwierdził wynik Cromwella, Mortona?

Mutant węzła złożonego także jest złożony, co więcej istnieje bijekcja między czynnikami w ich rozkładach na węzły pierwsze Ruberman -- \cite{ruberman87}.
\index{człowiek!Ruberman, Daniel}%
Dzięki temu możemy bez straty ogólności założyć, że badamy tylko węzły pierwsze, niestety wciąż nie jest znana ogólna procedura pozwalająca wyliczyć wszystkie mutanty danego węzła.

Zaraz po rewolucji, jaką w latach 80. wywołała relacja kłębiasta, Ewing napisał z~Millettem komputerowy program w~języku C, który wyjątkowo szybko znajdował wielomiany HOMFLY oraz Kauffmana zadanego węzła.
\index{człowiek!Ewing, ?}%
\index{człowiek!Millett, ?}%
Nawet dziś program ten jest w stanie uporać się z węzłami, z którymi nie radzą sobie inne narzędzia.
Autorzy nie wiedzieli wtedy, że ktoś jeszcze będzie z nich korzystać w przyszłości, dlatego poczynili w kodzie liczne optymalizacje dla stacji roboczej Sun, jaką wtedy dysponowali.
Dzisiaj okazuje się, że dla węzłów o większej liczbie skrzyżowań program często kończy swoje działanie zrzutem pamięci, wpada w pętlę bez wyjścia albo zwraca niepoprawny wynik (składniki wielomianu Kauffmana są postaci $a^m z^n$, gdzie $m + n$ jest nieparzyste).
Stojmenow korzystał z tych programów podczas tablicowania mutantów.
Jak postępował?
\begin{enumerate}
    \item podzielił węzły na grupy o tej samej objętości, wielomianie Jonesa oraz Alexandera;
    \item w każdej z grup szukał ciągu mutacji pomiędzy diagramami minimalnymi;
    \item tam, gdzie nie udało się znaleźć mutantów, liczył 2-kablowy wielomian HOMFLY;
    \item jeśli wielomian był taki sam, szukał ciągu mutacji między nieminimalnymi diagramami do 18 skrzyżowań;
    \item wreszcie pozostałe grupy zostały potraktowane reprezentacjami grupy podstawowej dwukrotnego nakrycia.
\end{enumerate}

Podsumowanie jego pracy zawiera tabela:

\begin{table}[H]
    \centering
    \begin{tabular}{lccccc} \toprule
        skrzyżowania & 11 & 12 & 13  & 14   & 15    \\ \midrule
        pary         & 16 & 70 & 703 & 3917 & 24884 \\
        trójki       &    & 5  & 38  & 233  & 1000  \\
        czwórki      &    &    & 32  & 262  & 2909  \\
        szóstki      &    &    & 1   & 17   & 172   \\
        ósemki       &    &    &     & 6    & 84    \\
        łącznie      & 16 & 75 & 774 & 4435 & 29049 \\
        \bottomrule
        \hline
    \end{tabular}
    \caption{Liczba grup mutantów wśród pierwszych węzłów do 15 skrzyżowań}
\end{table}

\subsubsection{Rozróżnianie mutantów}
Żaden wielomianowy niezmiennik opisany w~tej książce nie potrafi odróżnić od siebie węzłów $11n_{34}$ oraz $11n_{42}$.
Okazuje się, że niewielomianowe niezmienniki też często są bezradne.

\begin{proposition}
    Mutacja węzła nie zmienia jego wielomianu Alexandera.
\index{wielomian!Alexandera}%
\end{proposition}

% w commicie 1fe48ad183cb592e897f4151f9c18439baa84274 wymieniam:
% +    kablowego wielomianu Jonesa, % menasco91
% +    2-kablowego wielomianu HOMFLY, % przytycki89
% +    kablowego wielomianu Kauffmana, % lipson87
% +    sygnatury Tristrama-Levine'a, % cooper99
% +    symplicjalnEj objętości Gromowa, % ruberman87
% +    instanton homologii Floera, % ruberman99
% +    niezmienników Wittena % rong94
% +    ani Cassona. % kirk89
% ale teraz nie potrafię sobie przypomnieć, jak znalazłem te niezmienniki/prace. :(
% wydaje mi się, że źródłem nie jest stoimenow10, może Math Overflow?

\begin{proof}
    Stojmenow, Tanaka piszą w \cite{tanaka09}, że to proste ćwiczenie teorii kłębiastej, oraz że rozumowanie łatwo przenosi się na odkryte później wielomiany Jonesa, HOMFLY, BLM/Ho, Kauffmana.
\end{proof}

Warto przytoczyć teraz obserwację 3.8.2 z \cite[s. 43]{kawauchi96}: jeśli sploty $L_1, L_2$ są mutantami, to podwójne przestrzenie nakryciowe nad $S^3$ rozgałęzione odpowiednio wzdłuż $L_1$ oraz $L_2$ są homeomorficzne z zachowaniem orientacji.
Co więcej, macierze Seiferta mutantów są $S$-równoważne.
\index{macierz!Seiferta}%
To tłumaczy czemu większość niezmienników nie radzi sobie z odróżnianiem mutantów.
% Viro: Two-fold branched coverings of three-sphere

% z tanaka09
Wzór kablowy\footnote{Niech $T$ będzie trywialnym torusem, zawierającym węzeł $K$, zaś $e \colon T \to S^3$ włożeniem $T$ na otoczenie węzła $C$ tak, że $e$ przenosi równoleżnik $T$ na równoleżnik $C$. Wtedy $\alexander_{eK} (t) = \alexander_K(t)\alexander_C(t^n)$.} \cite[tw. 6.15]{lickorish97} pokazuje, że wielomian Alexandera nie odróżnia satelitów zmutowanych węzłów.
Wielomian Jonesa nie spełnia żadnego wzoru kablowego (gdyż czasami odróżnia kable węzłów o~tym samym wielomianie), ale...:

\begin{proposition}
    Mutacja węzła nie zmienia jego kablowego wielomianu Jonesa.
\index{wielomian!Jonesa}%
\end{proposition}

\begin{proof}
\index{człowiek!Morton, ?}%
\index{człowiek!Traczyk, Paweł}%
    Morton, Traczyk \cite{traczyk88}.
    % kiedyś tu było niezdefiniowane menasco91 = Menasco, Thistlethwaite: The Tait flyping conjecture, ale w 2022 roku przeczytałem Stoimenow: Tabulating and distinguishing mutants zmieniłem zdanie: "nonetheless Morton and Traczyk [36] showed that..."
\end{proof}

Praca \cite{traczyk88} wspomina jeszcze, że to samo jest prawdą także dla ,,wielomianu Jonesa dwóch zmiennych'' (HOMFLY) i 2-kabli, ale nie dla dowolnych satelitów.
Fakt, że wielomian HOMFLY (a także Kauffmana) nie odróżniają 2-satelitów mutantów, odkryto rok wcześniej:

\begin{proposition}
\label{mutants_and_homfly}%
\index{wielomian!HOMFLY}%
    Mutacja węzła nie zmienia jego 2-kablowego wielomianu HOMFLY.
\end{proposition}

\begin{proof}
\index{człowiek!Lickorish, ?}%
\index{człowiek!Lipson, ?}%
\index{człowiek!Przytycki, Józef}%
    Lickorish, Lipson \cite{lipson87}, później też Przytycki \cite{przytycki89}.
    % TODO: skąd to o Przytyckim???
\end{proof}

Lepiej jest z~3-kablami: wielomian HOMFLY odróżnia tak węzły Kinoshity-Terasakiego i~Conwaya, ale wymaga takiej ilości rachunków, że mało komu chce się je przeprowadzać dla innych węzłów.
\index{węzeł!Conwaya}%
\index{węzeł!Kinoshity-Terasakiego}%

\begin{proposition}
    Mutacja węzła nie zmienia jego (2-?)kablowego wielomianu Kauffmana.
\index{wielomian!kablowy}%
\index{wielomian!Kauffmana}%
\end{proposition}

\begin{proof}
\index{człowiek!Lickorish, ?}%
\index{człowiek!Lipson, ?}%
    Lickorish, Lipson \cite{lipson87}.
\end{proof}

Morton w recenzji pracy Lickorisha, Lipsona wspomina, że dla satelitów owijających się więcej niż $2$ razy to nie jest prawda, jak później odkryto.

\begin{proposition}
\index{wielomian!BLM/Ho}%
    Mutacja węzła nie zmienia jego wielomianu BLM/Ho.
\end{proposition}

\begin{proof}
    \cite{tanaka09}, choć nie wiem gdzie dokładnie.
\end{proof}

\begin{proposition}
\index{sygnatura!Levine'a-Tristrama}%
    Mutacja węzła nie zmienia jego sygnatury Levine'a-Tristrama.
\end{proposition}

\begin{proof}
\index{człowiek!Cooper, ?}%
\index{człowiek!Lickorish, ?}%
    Cooper, Lickorish w~\cite{cooper99} podają klasyczny dowód, że mutacja zachowuje wielomian Alexandera, oparty o~macierz Seiferta.
    Wiedząc, jak wygląda ta macierz, autorzy wyciągają wniosek, że sygnatura splotów (!) w~homologicznej 3-sferze też jest zachowywana.
\end{proof}

\begin{proposition}
\index{objętość!symplicjalna Gromowa}%
\label{mutants_the_same_volume}%
    Mutacja węzła nie zmienia jego symplicjalnej objętości Gromowa.
\end{proposition}

\begin{proof}
\index{człowiek!Ruberman, Daniel}%
    Ruberman w \cite{ruberman87}.
    % Ruberman [42] showed that mutants have equal volume in all hyperbolic pieces of the JSJ decomposition.
\end{proof}

\begin{proposition}
\index{homologia!Floera}%
    Mutacja węzła nie zmienia jego instanton homologii Floera.
\end{proposition}

\begin{proof}
\index{człowiek!Ruberman, Daniel}%
    Jeszcze raz Ruberman, w \cite{ruberman99}.
\end{proof}

\begin{proposition}
\index{niezmiennik!Wittena}%
    Mutacja węzła nie zmienia jego niezmienników Wittena.
\end{proposition}

\begin{proof}[Niedowód]
    Rong dla wybranych mutacji (niech $L$ będzie obramowanym splotem w~3-rozmaitości $M$, zaś $F \subseteq M$ dwustronną powierzchnią, tnącą splot w 0, 1 lub 4 (transwersalnie) punktach, o genusie $g = 0$, $1$ lub $2$, wtedy rozcięcie $M$ wzdłuż $F$ i sklejenie po obrocie o 180 stopni daje parę $(M^\tau, K^\tau)$, która ma ten sam niezmiennik Wittena w $\mathrm{SU}(2)$ jak wyjściowa) w \cite{rong94}.
\end{proof}

\begin{proposition}
\index{niezmiennik!Cassona}%
    Mutacja węzła nie zmienia jego niezmienników Cassona.
\end{proposition}

Celowo nie podajemy definicji tego niezmiennika.
Nieformalnie, zlicza on co drugą klasę sprzężoności reprezentacji grupy podstawowej homologicznej 3-sfery w grupie $SU(2)$.

\begin{proof}
    Kirk w \cite{kirk89}.
\end{proof}

\begin{conjecture}
    Mutacja węzła nie zmienia jego liczby gordyjskiej.
\end{conjecture}

Jak czytamy w \cite[problem 1.69]{kirby78}, przypuszczenie to jest bardzo trudne do udowodnienia: wynika z niego inna stara hipoteza teorii węzłów, że liczba gordyjska splotów jest addytywna.
\index{hipoteza!o indeksie skrzyżowaniowym}%
Przytoczymy tylko dwa częściowe wyniki.
Najpierw Rolfsen zauważył, że jedynym mutantem niewęzła jest sam niewęzeł \cite{rolfsen93}.
\index{człowiek!Rolfsen, Dale}%
Dekadę później Gordon, Luecke pokazali, iż klasa węzłów $1$-gordyjskich jest zamknięta na przeprowadzanie mutacji \cite{gordon06}.
\index{człowiek!Gordon, ?}%
\index{człowiek!Luecke, ?}%
(Ohtsuki powtarza hipotezę w \cite[problem 12.15]{ohtsuki02}.)
\index{człowiek!Ohtsuki, ?}%

% to NIE jest z stoimenow10
\begin{proposition}
    Niech $D$ będzie alternującym diagramem.
    Wtedy każdy mutant $D$ też jest alternujący.
\end{proposition}

Wśród niezmienników, które mutacja czasami zmienia, znajduje się genus plastrowy:

\begin{proposition}
\index{genus!plastrowy}%
    Niech $m, n$ będą nieujemnymi liczbami całkowitymi.
    Wtedy istnieje węzeł $K$ o genusie plastrowym równym $m$, którego pewien mutant ma genus plastrowy równy $n$.
\end{proposition}

Stanowi to uogólnienie obserwacji Livingstona \cite{livingston83}, że istnieją mutanty o~różnym genusie plastrowym.
\index{człowiek!Livingston, ?}

\begin{proof}
    Kim, Livingston w \cite{kim05}.
\end{proof}

Zbiór problemów niskowymiarowej topologii opublikowany przez Kirby'ego \cite{kirby78} zawiera następujące pytanie:
\index{człowiek!Kirby, Rob}%

\begin{conjecture}[problem 1.91]
\index{węzeł!satelitarny}
    Niech $K$ będzie prostym\footnote{simple} węzłem bez orientacji.
    Czy istnieją węzły niebędące mutantami $K$, których nie można odróżnić od $K$ wielomianem Jonesa oraz wszystkimi jego satelitami?
\end{conjecture}

Stojmenow pisze, że tak: pierwszą chronologicznie parą jest $14_{41721}$, $14_{42125}$, dowód tego faktu opiera się na wzorze fuzyjnym Masbauma-Vogela odkrytym w pracy \cite{masbaum94}.
\index{człowiek!Stoimenow, Alexander}
\index{człowiek!Masbaum, Gregor}%
\index{człowiek!Vogel, Pierre}%
% fusion formula
Choć wzór ten zastosowany do konkretnej pary węzłów sprawia zazwyczaj trudności rachunkowe, to jest wystarczającym narzędziem, by rozszerzyć konstrukcję do ogólnego wyniku:

\begin{proposition}
    Istnieje nieskończenie wiele par prostych węzłów hiperbolicznych o tych samych kolorowych wielomianach Jonesa, które nie są swoimi mutantami.
\end{proposition}

\begin{proof}
    Stojmenow, Tanaka \cite[tw. 1.1]{tanaka09}.
\end{proof}

\index{mutant|)}





\section{Precle}
\index{węzeł!preclowy|see {precel}}%
\index{precel|(}%
Precle to sploty ze standardowym diagramem, na którym wyróżnić można co najmniej trzy warkocze na dokładnie dwóch pasmach.
Warkocze te ułożone są w sposób cykliczny.
Pojawiły się po raz pierwszy w~książce Reidemeistera z~1932 roku jako przykład węzła o~trywialnym wielomianie Alexandera.
Zanim podamy formalną definicję, wygodnie będzie przyjrzyć się ogólniejszej rodzinie splotów Montesinosa, nazwanych tak na cześć José Marii Montesinosa Amilibii, topologa hiszpańskiego.
\index[persons]{Montesinos, José}%
Wprowadził je do matematyki w~1973 roku \cite{montesinos73}.

Kawauchi \cite[s. 29]{kawauchi96} pisze, że Montesinos uogólnił precle, bo dwukrotne nakrycie nad $S^3$ rozgałęzione wzdłuż precla jest rozmaitością Seiferta.
\index{rozmaitość!Seiferta}%

\index{splot!Montesinosa|(}%
\begin{definition}[splot Montesinosa]
    Splotem Montesinosa nazywamy splot o~poniższym diagramie, gdzie wymierne liczby $\alpha_i/\beta_i$ oraz całkowita $e \in \Z$ odpowiadają supłom.
\begin{comment}
    \[
    \begin{tikzpicture}[baseline=-0.65ex, scale=0.1]
    %\useasboundingbox (-5, -9) rectangle (5, 5);
        \draw[semithick] (-5, 5) rectangle (5, 15);
        \foreach \x in {0,1,3,4} {
            \draw[semithick] (15*\x-35, -15) rectangle (15*\x-25, -5);
        }
        \foreach \x in {0,1,2,3,4,5} {
            \draw[semithick] (15*\x-35, -8) to (15*\x-40, -8);
            \draw[semithick] (15*\x-35, -12) to (15*\x-40, -12);
        }
        \draw[semithick] (-40, -8) [in=down, out=left] to (-45, -3);
        \draw[semithick] (-40, -12) [in=down, out=left] to (-49, -3);

        \draw[semithick] (-40, 8) [in=up, out=left] to (-45, 3);
        \draw[semithick] (-40, 12) [in=up, out=left] to (-49, 3);

        \draw[semithick] (40, 8) [in=up, out=right] to (45, 3);
        \draw[semithick] (40, 12) [in=up, out=right] to (49, 3);

        \draw[semithick] (40, -8) [in=down, out=right] to (45, -3);
        \draw[semithick] (40, -12) [in=down, out=right] to (49, -3);

        \draw[semithick] (-45, -3)  to (-45, 3);
        \draw[semithick] (-49, -3)  to (-49, 3);
        \draw[semithick] (45, -3)  to (45, 3);
        \draw[semithick] (49, -3)  to (49, 3);

        \draw[semithick] (-5, 8)  to (-40, 8);
        \draw[semithick] ( 5, 8)  to ( 40, 8);
        \draw[semithick] (-5, 12)  to (-40, 12);
        \draw[semithick] ( 5, 12)  to ( 40, 12);

        \node at (0, 10) {e};
        \node at (0, -10) {\ldots};
        \node at (-15, -10) {$\displaystyle \frac{\alpha_2}{\beta_2}$};
        \node at (-30, -10) {$\displaystyle \frac{\alpha_1}{\beta_1}$};
        \node at (15, -10) {$\displaystyle \frac{\alpha_{n-1}}{\beta_{n-1}}$};
        \node at (30, -10) {$\displaystyle \frac{\alpha_n}{\beta_n}$};
    \end{tikzpicture}
    \]
\end{comment}
\end{definition}

Cały dwunasty rozdział podręcznika Burdego i~Zieschanga \cite{burde14} jest poświęcony splotom Montesinosa.
Można tam znaleźć ich klasyfikację, zawiera jednak pułapkę: autorzy używają innej notacji dla supłów dwumostowych.
Podają przykład węzła $43/105 = [0, 2, 2, 3, 1, 4]$, dla nas to jest $105/22 = [4, 1, 3, 2, 2]$.

\begin{proposition}
    Sploty Montesinosa o $r \ge 3$ supłach $\beta_1/\alpha_1, \beta_2/\alpha_2, \ldots$ takich, że
    \begin{equation}
        \sum_{j=1}^r \frac{1}{\alpha_j} \le r - 2
    \end{equation}
    są sklasyfikowane (z dokładnością do cyklicznych permutacji oraz odwracania) przez uporządkowany zbior ułamków $\{\beta_i/\alpha_i \mod 1 : 1 \le i \le r\}$ razem z~wymierną liczbą
    \begin{equation}
        e_0 = e + \sum_{j=1}^r \frac{\beta_j}{\alpha_j}.
    \end{equation}
\end{proposition}

Powyższy fakt nie używa naszej notacji!
(Można przeczytać też \cite[s. 28]{kawauchi96} dla splotów.)

\begin{proof}
\index[persons]{Bonahon, Francis}%
    Praca doktorska Bonahona \cite{bonahon79}.
    W~internecie dostępny jest jej skan (gdyż była pisana odręcznie po francusku!), ale angielskie tłumaczenie nie istnieje.
\end{proof}

Używając nadal tej niestandardowej notacji można sklasyfikować węzły odwracalny czy zwierciadlane:

\begin{proposition}
\index{węzeł!zwierciadlany}%
    Splot Montesinosa jest zwierciadlany wtedy i tylko wtedy, gdy $e = 0$ oraz istnieje permutacja $\pi$, cykl długości $r$ lub odwrócenie, taka że
    \begin{equation}
        \frac{\beta_{\pi(i)}}{\alpha_{\pi(i)}} \equiv -\frac{\beta_i}{\alpha_i} \pmod 1
    \end{equation}
\end{proposition}

\begin{proof}
    Burde, Zieschang, Heusener \cite[s. 230]{burde14}.
\end{proof}

\begin{corollary}
    Splot Montesinosa z nieparzystą liczbą supłów nie jest zwierciadlany.
\end{corollary}

\begin{proposition}
\index{węzeł!odwracalny}%
    Splot Montesinosa jest odwracalny wtedy i tylko wtedy, gdy, po ewentualnej zmianie etykiet supłów, przynajmniej jedna z liczb $\alpha_i$ jest parzysta lub wszystkie liczby $\alpha_i$ są parzyste, a sam splot ma postać
    \begin{equation}
        M(e_0, \beta_1/\alpha_1, \ldots, \beta_p/\alpha_p, \beta_p/\alpha_p, \ldots, \beta_1/\alpha_1)
    \end{equation}
    (oraz $r = 2p$) lub
    \begin{equation}
        M(e_0, \beta_1/\alpha_1, \ldots, \beta_p/\alpha_p, \beta_{p+1}/\alpha_{p+1}, \beta_p/\alpha_p, \ldots, \beta_1/\alpha_1)
    \end{equation}
    (oraz $r = 2p+1$) lub
    \begin{equation}
         M(e_0, \beta_1/\alpha_1, \ldots, \beta_p/\alpha_p, \beta_{p+1}/\alpha_{p+1}, \beta_p/\alpha_p, \ldots, \beta_2/\alpha_2).
    \end{equation}
    (oraz $r = 2p$).
\end{proposition}

\begin{proof}
    Burde, Zieschang, Heusener \cite[s. 231]{burde14}.
\end{proof}

\index{splot!Montesinosa|)}%

% DICTIONARY;pretzel;preclowy;węzeł
\begin{definition}[precel]
\label{def:pretzel}%
    Splot Montesinosa o~całkowitych współczynnikach nazywamy preclem.
\end{definition}

Na standardowym diagramie precla $(p_1, p_2, \ldots, p_n)$ występuje $p_1$ lewych skrzyżowań w~pierwszym suple, $p_2$ w~drugim, i~tak dalej.
Taki precel jest węzłem dokładnie wtedy, gdy $n$ oraz $p_i$ są nieparzyste lub dokładnie jedna z~liczb $p_i$ jest parzysta (\cite[s. 27]{kawauchi96}).

\begin{proposition}
    Jeśli co najmniej dwa współczynniki $p_i, p_j$ zerują się, precel jest rozszczepialny.
\end{proposition}

\begin{proof}
    Widać to bezpośrednio z diagramu: jego część zawarta między supłem $p_i$ oraz $p_j$ jest rozłączna z~resztą diagramu.
    Nie jest jednak prawdziwa implikacja odwrotna.
    % TODO: podać przykład
\end{proof}

Precel $(1,1,1)$ to prawy trójlistnik, $(5, -1, -1)$ to węzeł dokerski $6_1$, $(-3, 0, -3)$ to splot dwóch trójlistników, zaś $(2p, 2q, 2r)$ jest splotem trzech niewęzłów.
Precle $(-2, 3, 2n+1)$ są szczególnie użyteczne jako narzędzie do badania 3-rozmaitości.
Wiele twierdzeń, które dotyczą takich rozmaitości, opiera się na przykład na chirurgii Dehna precla $(-2, 3, 7)$.
\index{chirurgia Dehna}%
\index{precel!(-2, 3, 7)}%

\begin{proposition}
\index{węzeł!torusowy}%
    Niech $K$ będzie węzłem torusowym.
    Jeśli $K$ jest jednocześnie $(-2, 3, k)$-preclem, to
    \begin{equation}
        K = 5_{1} = T_{2,5} = P(1, 3, -2)
    \end{equation}
    albo
    \begin{equation}
        K = 8_{19} = T_{3,4} = P(3, 3, -2)
    \end{equation}
    albo
    \begin{equation}
        K = 10_{124} = T_{3,5} = P(5, 3, -2).
    \end{equation}
\end{proposition}

\begin{proof}
\index[persons]{Garoufalidis, Stavros}%
\index[persons]{Koutschan, Christoph}%
    Garoufalidis, Koutschan \cite{garoufalidis12}.
\end{proof}

\begin{proposition}
    \label{prp:pretzel_not_invertible}
    Niech $p, q, r$ będą liczbami nieparzystymi takimi, że $|p|, |q|, |r|$ są parami różne i większe niż $1$.
    Wtedy $(p, q, r)$-precel jest nieodwracalny.
\end{proposition}

\begin{proof}
\index[persons]{Fox, Ralph}%
\index[persons]{Trotter, Hale}%
    Zgodnie z sugestią Foxa, Trotter przetłumaczył problem na język teorii grup w \cite{trotter63}.
    Wyróżnia w~grupie węzła dwa elementy -- (zorientowany) południk i równoleżnik.
    Jeśli dwa węzły są równoważne, to homeomorfizm $\R^3 \to \R^3$ posyłający jeden na drugi wyznacza izomorfizm ich grup podstawowych, który posyła południk na południk i równoleżnik na równoleżnik.
    W szczególności, jeśli węzeł jest odwracalny, to jego grupa posiada specjalny automorfizm (,,inwersję'') odwracający zarówno południk, jak i równoleżnik.
    To prowadzi do sprzeczności w przypadku rozpatrywanych precli.
\end{proof}

Wystarczający warunek z tego stwierdzenia jest prawie konieczny.
Jeśli $p = r$, węzeł można odwrócić przez półobrót wokół środkowej osi.
Cykliczne permutacje trójki $(p, q, r)$ nie zmieniają węzła, więc wszystkie trzy liczby $p, q, r$ muszą być różne.
Jeśli jedna z tych liczb jest parzysta, półobrót wokół poziomej osi odwraca węzeł.
Z pracy Bankwitza i Schumanna wynika, że jeśli któryś z parametrów ma wartość $\pm 1$, to węzeł też jest odwracalny.
\index[persons]{Bankwitz, Carl}%
\index[persons]{Schumann, Hans}%
% Bankwitz, Carl; Schumann, Hans Georg;
% Über viergeflechte. (German)
% Abh. Math. Sem. Univ. Hamburg 10 (1934), no. 1, 263–284.
Nie jest trudno pokazać to wprost.
Zatem nie wiemy jedynie jakie są precle $(p, q, -q)$, gdzie $|p| \neq |q|$ oraz $|p|, |q| \ge 3$.

Jeśli liczby $p, q, r$ są nieparzyste i tego samego znaku, to wyznacznik precla $(p, q, r)$ jest postaci $4n+3$.
\index{wyznacznik}%
Wtedy używając formy kwadratowej (jak Reidemeister w 1932!) można pokazać, że taki węzeł nie jest achiralny.
Niech $K$ będzie zorientowanym preclem $(3, 5, 7)$.
Wtedy $K$, $mK$, $rK$, $mrK$ są parami nierównoważne.
Węzeł $K \shrap mK$ jest dodatnio zwierciadlany, zaś $rK \shrap mK$ ujemnie zwierciadlany.
Trójlistnik jest odwracalny, ale nie zwierciadlany, ósemka jest zwierciadlana i~odwracalna.
To pokazuje, że wszystkie typy symetrii są realizowane przez precle lub sumy precli.

\begin{proposition}
\label{prp:pretzel_alexander}%
\index{wielomian!Alexandera}%
    Jeżeli liczby $p, q, r$ są nieprzyste, to wielomianem Alexandera $(p, q, r)$-precla jest
    \begin{equation}
        \alexander = \frac 14 ((pq+qr+pr) (t-1)^2 + (t+1)^2).
    \end{equation}
\end{proposition}

\begin{proof}
\index[persons]{Bae, Yongju}%
\index[persons]{Lee, In}%
    Bae, Lee pokazali w \cite[lemat 3.1]{bae20}, że macierz Seiferta $(p, q, r)$-precla to
    \begin{equation}
        M = \frac 1 2 \begin{bmatrix}
            p+q & p-1 \\
            p+1 & p+r
        \end{bmatrix},
    \end{equation}
    wystarczy więc użyć wzoru $\alexander = \det(M-tM^t)$.
\end{proof}

Wielomian Alexandera precla $(p_1, \ldots, p_n)$ nigdy nie zależy od kolejności współczynników, jest to ćwiczenie w~książce Livingstona \cite[s. 215]{livingston93}.

\begin{proposition}
\index{kolorowalność}%
    Niech $n$ będzie liczbą pierwszą.
    Węzeł $p, q, r$-preclowy jest $n$-kolorowalny wtedy i~tylko wtedy, gdy $n$ dzieli $|pq+qr+pr|$.
    Jeśli przynajmniej jedna z~liczba $p, q, r$ nie jest wielokrotnością $n$, kolorowanie z dokładnością do permutacji jest jedyne.
    W przeciwnym przypadku istnieją cztery różne kolorowania.
\end{proposition}

\begin{proof}
\index[persons]{Brownell, Kathryn}%
\index[persons]{O'Neil, Kaitlyn}%
\index[persons]{Taalman, Laura}%
    Pierwsza część jest wnioskiem ze stwierdzeeń~\ref{prp:colour_determinant},~\ref{prp:alexander_determinant} oraz~\ref{prp:pretzel_alexander}.
    Dowód drugiej zawiera praca Brownell, O'Neil, Taalman \cite{taalman05}, trzech Amerykanek.
\end{proof}

Podano tam także ogólny wzór na liczbę $n$-kolorowań dowolnego węzła.

\index{precel|)}%

% Koniec sekcji Precle


\input{50-families/lissajous}

\section{Węzły torusowe}
\index{węzeł!torusowy|(}%

W tej sekcji przyjrzymy się węzłom o~specjalnym ułożeniu w~przestrzeni $\R^3$.
Do ich określenia potrzebny jest torus trywialny, powierzchnia otrzymana przez obrót okręgu $(x-2)^2 + y^2 = 1$ wokół osi $y$.
Można go także uzyskać przez sklejenie podstaw walca tak, by go przy tym nie zapętlić.
Oczywiście istnieją też nietrywialne torusy, jak rurowe otoczenie trójlistnika.

\begin{definition}[splot torusowy]
    Splot, który leży na powierzchni trywialnego torusa, nazywamy torusowym.
\end{definition}

Na walcu $S^2 \times [0,1]$, którego podstawa leży w~płaszczyźnie $xy$, rozpatrzmy $r$ skierowanych odcinkach (dla $k = 0, 1, \ldots, r - 1$) o~końcach w~punktach
\begin{align*}
    \left(\cos \frac{2k \pi}{r}, \sin \frac{2k\pi}{r}, 0 \right), \quad
    \left(\cos \frac{2k \pi}{r}, \sin \frac{2k\pi}{r}, 1 \right).
\end{align*}
Przekręćmy górną podstawę walca wokół osi $z$ o~skierowany kąt $2\pi q / r$ oraz utożsammy ze sobą pary punktów $(x, y, 0) \sim (x, y, 1)$,
Uzyskaliśmy splot torusowy $T_{q, r}$: okrąża on $q$ razy rdzeń torusa i~$p$ razy jego oś symetrii obrotowej.
Określimy jeszcze kilka splotów torusowych.
Węzeł $T_{0, 0}$ leży na powierzchni torusa i~jest ściągalny do punktu, zaś $T_{1, 0}$ to nawinięta toroidalnie pętla.
Węzeł $T_{p, q}$ posiada następującą parametryzację:
\[
    x = (2+\cos q \phi) \cos p \phi, \quad
    y = (2+\cos q \phi) \sin p \phi, \quad
    z = - \sin q \phi, \quad
    0 \le \phi \le 2\pi.
\]
Poniżej przedstawiamy trzy węzły torusowe.

\begin{figure}[H]
    \begin{minipage}[b]{.3\linewidth}
        \centering
        \includegraphics[width=\linewidth]{../data/torus-p2-q3.pdf}
        \subcaption{trójlistnik: $p = 2, q = 3$}
    \end{minipage}
    \begin{minipage}[b]{.3\linewidth}
        \centering
        \includegraphics[width=\linewidth]{../data/torus-p2-q11.pdf}
        \subcaption{$p = 2, q = 11$}
    \end{minipage}
    \begin{minipage}[b]{.3\linewidth}
        \centering
        \includegraphics[width=\linewidth]{../data/torus-p11-q2.pdf}
        \subcaption{$p = 11, q = 2$}
    \end{minipage}
\end{figure}

%Węzeł ten leży na torusie $(r - 2)^2 + z^2 = 1$.
% p = 5;
% q = 3;
% ParametricPlot3D[
% {
% Cos [2 Pi p t] (2 + Cos[2 Pi q t]),
% (2 + Cos[2 Pi q t]) Sin[2 Pi p t],
% -Sin[2 Pi q t]},
% {t, 0, 1},
% ColorFunction -> "Rainbow",
% PlotStyle -> Thickness[0.02],
% Boxed -> False,
% Axes -> False
% ]

Okazuje się, że innych obiektów już nie ma.

\begin{proposition}
    Niech $K$ będzie splotem torusowym takim, że żadne z jego ogniw nie jest niewęzłem, czyli postaci $T_{1, 0}$.
    Wtedy dla pewnych całkowitch $p, q$, węzły $K$ oraz $T_{p, q}$ są tego samego typu.
\end{proposition}

\begin{proposition}
    Niech $d$ będzie największym wspólnym dzielnikiem liczb całkowitych $p, q$.
    Wtedy węzeł torusowy $T_{p, q}$ posiada dokładnie $d$ ogniw.
\end{proposition}

Siedem węzłów z tabeli na końcu książki to węzły torusowe.
Są to niewęzeł, $3_1 = T_{3,2}$, $5_1 = T_{5,2}$, $7_1 = T_{7,2}$, $8_{19} = T_{4,3}$, $9_1 = T_{9,2}$ oraz $10_{124} = T_{5, 3}$.

\begin{proposition}
    Niech $p, q$ będą względnie pierwszymi liczbami takimi, że $|p|, |q| \ge 2$.
    Wtedy splot $T_{p, q}$ oraz splot do niego odwrotny, $T_{-p, -q}$, są tego samego typu.
\end{proposition}

Sploty $T_{p, q}$ oraz $T_{q, p}$ również są równoważne.
Murasugi prezentuje w~swojej książce \cite{murasugi96} przyjemny dowód opierający się na następującym lemacie:

\begin{lemma}
    Sfera $S^3$ powstaje z~powierzchni dwóch węzłów trywialnych z~wnętrzem ($D^2 \times S^1$) przez wzajemne sklejenie południka i~równoleżnika z~równoleżnikiem i~południkiem.
\end{lemma}

\begin{proposition}
    Niech $K$ będzie nietrywialnym węzłem, którego grupa podstawowa $\pi$ ma nietrywialne centrum.
    Wtedy $K$ jest węzłem torusowym.
    % Kawauchi: The torus knots are characterized as the only knots whose groups have non-trivial centers (cf. Corollary 6.3.6).
\end{proposition}

\begin{proof}
\index[persons]{Aumann, Robert}
% https://mathscinet.ams.org/mathscinet-getitem?mr=96236
\index[persons]{Burde, Gerhard}%
\index[persons]{Murasugi, Kunio}%
\index[persons]{Neuwirth, Lee}%
\index[persons]{Nielsen, Jakob}%
\index[persons]{Stallings, John}%
\index[persons]{Zieschang, Heiner}%
    Najpierw pokazali to Murasugi \cite{murasugi61}, Neuwirth \cite{neuwirth61} przy dodatkowym założeniu, że węzeł $K$ jest alternujący,
    wkrótce po tym Burde, Zieschang znaleźli dowód w ogólnym przypadku \cite{burde66}.
    Ich dowód korzysta z wyników Neuwirtha (komutant grupy $\pi$ jest skończenie generowany), Stallingsa (dopełnienie $X$ tubularnego otoczenia $K$ można rozwłóknić nad $S^1$ z~włóknem: 2-rozmaitością $M$, z jedną krzywą na brzegu) i Nielsena.

    % TODO:
    (Burde, Zieschang piszą, że dla alternujących zrobił to już Aumann w 1956 roku).
\end{proof}

Przejdźmy do podania wartości różnych niezmienników.


\subsection{Niezmienniki liczbowe węzłów torusowych}
Podamy teraz wartości całkowitoliczbowych niezmienników dla węzłów torusowych przy założeniu, że $p$ lub $q$ nie jest zerem.
Nietrywialne węzły torusowe są pierwsze i~odwracalne, ale mają niezerową sygnaturę, więc nie są chiralne.
Wiedział to Schreier w 1924.
% TODO: \cite schreier24?

\begin{proposition}
\index{okres}%
    Węzeł torusowy $T_{p, q}$ ma okres $|p|$ oraz $|q|$.
\end{proposition}

\begin{proposition}
\index{sygnatura}%
    Niech $p, q > 0$ będą liczbami całkowitymi, zaś $R_2$ oznacza resztę z dzielenia przez dwa.
    Zdefiniujmy funkcję $\sigma(p, q) = - \sigma(T_{p, q})$.
    Spełnia zależność rekurencyjną
    \begin{equation}
        \sigma(p, q) = \begin{cases}
             q^2 + \sigma(p-2q, p) - R_2(p)       & \text{jeśli } 2q < p \\
             q^2 - 1                              & \text{jeśli } 2q = p \\
             q^2 - \sigma(2q - p, q) + R_2(r) - 2 & \text{jeśli } 2q > p > q \\
             q^2/2 + R_2(q)/2 - 1                 & \text{jeśli } p = q
             % czwarte stanowi algebraiczne przekształcenie trzeciego dla p >= q
        \end{cases}
    \end{equation}
    z warunkami brzegowymi: $\sigma(p, q) = \sigma(q, p)$, $\sigma(1, q) = 0$, $\sigma(2, q) = q-1$.
\end{proposition}

\begin{proof}[Niedowód]
\index[persons]{Litherland, Richard}%
\index[persons]{Gordon, Cameron}%
\index[persons]{Murasugi, Kunio}%
    Gordon, Litherland, Murasugi \cite[tw. 5.2]{litherland81} używają niezmiennika acyklicznego\footnote{Z angielskiego null-homologous, czyli o trywialnych zredukowanych grupach homologii.} splotu $L$ w zorientowanej 3-rozmaitości $M$ w połączeniu z jego $m$-krotnym rozgałęzionym nakryciem cyklicznym.
    Wspominają też, że można dowieść tego używając wzoru Hirzebrucha, ale nie robią tego.
\end{proof}

Borodzik niedawno przyjrzał się dokładniej sygnaturom węzłów torusowych.
\index[persons]{Borodzik, Maciej}%
W pracy \cite{borodzik10} napisanej z Oleszkiewiczem pokazał, że nie istnieje wymierna funkcja $R(p, q)$, która pokrywałaby się z sygnaturą węzła torusowego $T_{p, q}$ dla wszystkich względnie pierwszych, nieparzystych $p$ oraz $q$.
\index[persons]{Oleszkiewicz, Krzysztof}%

Uwaga: definicja funkcji $s$ z \cite{borodzik10} zawiera złośliwą literówkę.

\begin{proposition}
    Niech $p, q$ będą względnie pierwszymi liczbami, zaś $C \in [0, 1)$ stałą taką, że $Cpq$ nie jest liczbą całkowitą.
    Przyjmijmy $z = \exp (2 \pi i C)$ i zdefinujmy pomocnicze funkcje: niech $\{x\} = x - \lfloor x \rfloor$ oznacza część ułamkową, zaś
    \begin{equation}
        \langle x \rangle = \begin{cases}
            0 & \text{dla } x \in \Z \\
            \{x\} - 1/2 & \text{dla } x \not \in \Z
        \end{cases}
    \end{equation}
    funkcję piłę.
    Dalej, określmy sumę Dedekinda
    \begin{equation}
        s(p, q, x) = \sum_{j = 0}^{q-1} \left\langle \frac {j}{q} \right\rangle \left\langle \frac {jp}{q} + x \right\rangle.
    \end{equation}
    Przy tych oznaczeniach, sygnatura węzła $(p, q)$-torusowego wyznacza się wzorem
    \begin{align}
        \sigma(z) & = \frac{1}{3pq} \left (p^2 + q^2 + 6 \langle Cpq \rangle^2 - \frac {1}{2} \right)  + 2(C^2 - C) pq + (2-4C) \langle Cpq \rangle + {} \\
        & - 2s(p, q, Cp) - 2s(q, p, Cq) - 2s(p, q, p-pC) - 2s(q, p, q-qC). \nonumber
    \end{align}
\end{proposition}

\begin{corollary}
    Jeśli $p, q$ są nieparzyste i względnie pierwsze, to
    \begin{equation}
        \sigma(T_{p,q}) = \frac{1}{6pq} + \frac{2p}{3q} + \frac{2q}{3p} - \frac{pq}{2} - 4(s(2p, q, 0) + s(2q, p, 0)) - 1.
    \end{equation}
\end{corollary}

\begin{corollary}
    Jeśli $p$ jest nieparzyste, zaś $q > 2$ parzyste, to
    \begin{equation}
        \sigma(T_{p,q}) = - \frac{pq}{2} + 4s(2p, q, 0) - 8s(p, q, 0) + 1.
    \end{equation}
\end{corollary}

\begin{proposition}
\index{indeks skrzyżowaniowy}%
    Mamy $\crossing T_{p, q} = \min \{|pq| - |p|, |pq| - |q|\}$.
\end{proposition}

\begin{proof}
\index[persons]{Murasugi, Kunio}%
    Murasugi twierdzi, że udowodnił to w \cite{murasugi91}.
\end{proof}

Wyznaczenie indeksu rozwiązującego było dużo trudniejsze.
Murasugi pisze w~książce \cite{murasugi96}, że mamy nierówność
\begin{equation}
    u(T_{p, q}) \le \frac 12 (p-1)(q-1),
\end{equation}
z równością dla względnie pierwszych $p, q > 0$.
Hipoteza Milnora głosiła, że w~rzeczywistości równość zachodzi zawsze.
Dowód został odnaleziony w~latach 1993-1995 przy użyciu tzw. \emph{gauge theory} (działu teorii pola, gdzie lagranżjan jest niezmienniczy względem grup Liego lokalnych transformacji...).

\begin{proposition}
\index{liczba gordyjska}%
\label{prp:torus_unknotting_number}%
    Dla względnie pierwszych $p, q > 0$ mamy
    \begin{equation}
        \unknotting T_{p, q} = \frac 12 (p - 1)(q - 1),
    \end{equation}
\end{proposition}

% % Rasmussen podał nowy dowód hipotezy Milnora o plastrowym genusie węzłów torusowych, jest to pierwszy dowód który nie zależy od gauge theory.
% https://mathscinet.ams.org/mathscinet-getitem?mr=2729272

\begin{proof}
\index{hipoteza!Milnora}%
    ,,Jeśli $X$ jest jednospójną, gładką, domkniętą, zorientowaną 4-rozmaitością taką, że $b^+ < 3$ jest\footnote{wymiar maksymalnej dodatniej podprzedstrzeni dla formy przecięć (intersection form) drugiej homologii.} nieparzyste i taką, że wielomianowy niezmiennik Donaldsona jest nietrywialny, to genus każdej zorientowanej, gładko zanurzonej powierzchni $F$ (poza dwoma wyjątkami, których nie rozumiemy) spełnia nierówność $2g - 2 \ge F \cdot F$'' to stwierdzenie, jakie razem z~technicznymi lematami można znaleźć w \cite{kronheimer93}.
    Geometryczne elementy dowodu znalazły się w drugiej części, \cite{kronheimer95}.

    Wnioskiem z głównego twierdzenia jest dowód hipotezy Milnora.
    % This last result was proved by F. B. Kronheimer and T. S. Mrowka in [Kronheimer-Mrowka 1993], who determined the 4-dimensional genus of T(p, q) (defined in 12.3) by applying gauge theory to an embedded surface in a 4-manifold.
    % https://web.math.princeton.edu/~petero/GridHomologyBook.pdf strona 4
\end{proof}

Genus pokrywa się z~liczbą gordyjską dla węzłów torusowych, bo wyznacznik macierzy Seiferta jest niezerowy, więc genus to dokładnie stopień wielomianu Alexandera.

Patrz też \cite[s. 149]{murasugi96}.

\begin{proposition}
\index{liczba mostowa}%
\label{prp:torus_bridge_number}%
    $\bridge T_{p, q} = \min \{|p|, |q|\}$
\end{proposition}

Według Murasugiego dowód znalazł Schubert \cite{schubert54}.
\index[persons]{Schubert, Horst}%

\begin{corollary}
\index{indeks warkoczowy}%
\label{cor:torus_braid_number}%
    Niech $p, q \neq 0$ będą liczbami całkowitymi.
    Wtedy $\braid T_{p, q} = \min \{|p|, |q|\}$.
\end{corollary}

\begin{proof}
    Niech $K$ będzie węzłem torusowym typu $(p,q)$ z~minimalnym przedstawieniem jako warkocz $\beta$.
    Z konstrukcji domknięcia (czyli dołączenia rozłącznych półokręgów) wynika,
    że diagram $K$ ma dokładnie $b(K)$ lokalnych maksimów.
    Definicja liczby mostowej orzeka, iż $\bridge K \le \braid K$.
    Bez straty ogólności niech $p > q > 0$.
    Skoro węzeł $K$ powstaje z~$q$-warkocza $(\sigma_{q-1} \ldots \sigma_2\sigma_1)^p$,
    indeks $b(K)$ nie przekracza $q = br(K)$.
\end{proof}



\subsection{Niezmienniki wielomianowe węzłów torusowych}
\begin{proposition}
    Niech $L = T_{p, q}$ będzie splotem torusowym o $d$ ogniwach, różnym od $T_{0, 0}$.
\index{wielomian!Alexandera}%
    Wtedy jego wielomianem Alexandera jest
    \begin{equation}
        \alexander_L(t) = (-1)^{d-1} \frac{(1-t)(1 - t^{pq/d})^d}{(1-t^p)(1-t^q)} \cdot t^{-(p-1)(q-1)/2}.
    \end{equation}
\end{proposition}

Przypadek $p = 2$ wymaga prostego rozumowania indukcyjnego.
Samo ćwiczenie pojawia się w~wielu podręcznikach topologii.
Pełny dowód można znaleźć w~\cite[przykład 9.15]{burde14}, gdzie wyznaczono jakobian prezentacji grupy węzła $\langle x, y \mid x^py^{-q}\rangle$.

Inne podejście, formułę Seiferta-Torresa, prezentuje przeglądowa praca Turaewa \cite{turaev86}.
\index{formuła Seiferta-Torresa}

\begin{proof}
    Macierz Seiferta węzła torusowego $L = T_{p, q}$ ma nieskomplikowaną blokową budowę i posłuży nam do znalezienia wielomianu Alexandera wzorem $\alexander = \det (M - tM^t)$.
    % Rachunki są nieco uciążliwe.
    \begin{equation}
        M = \begin{bmatrix}
            B & & & & \\
            -B & B & & & \\
            & \ddots & \ddots & & \\
            & & \ddots & B & \\
            & & & -B & B
        \end{bmatrix},
    \end{equation}
    złożona z~$(q-1)^2$ bloków o~wymiarach $(p-1) \times (p-1)$:
    \begin{equation}
        B = \begin{bmatrix}
            -1 & & & & \\
            1 & -1 & & & \\
            & 1 & \ddots & & \\
            & & \ddots & -1 & \\
            & & & 1 & -1
        \end{bmatrix}.
    \end{equation}
    Rachunki pozostawiamy Czytelnikowi jako ćwiczenie.
\end{proof}

\begin{corollary}
    Niech $K = T_{p, q}$ będzie węzłem torusowym.
    Wtedy jego wielomianem Alexandera jest
    \begin{equation}
         \alexander(t) = \frac{(t^{pq}-1)(t-1)}{(t^p-1)(t^q-1)}.
    \end{equation}
\end{corollary}

\begin{corollary}
    Wielomian Alexandera odróżnia od siebie węzły $(2,n)$-torusowe.
\end{corollary}

\begin{proof}
    Mamy $\alexander(T_{2,n})(t) = (t^n+1) / (t+1)$, więc $\deg \alexander (T_{2,n}) = n - 1$.
\end{proof}

Znajomość wielomianu Alexandera wystarcza na szczęście do podania pełnej klasyfikacji węzłów torusowych bez uciążliwego dowodu.

\begin{proposition}
    Niech $p, q, r, s$ będą liczbami całkowitymi.
    Następujące warunki są równoważne:
    \begin{itemize}
        \item węzły torusowe $T_{q, r}$ oraz $T_{p, s}$ są równoważne,
        \item $\{q, r\} = \{p, s\}$ lub $\{q, r\} = \{-p, -s\}$.
    \end{itemize}
\end{proposition}

\begin{proof}
    Ograniczymy się do przypadku, gdy $p, q, r, s \ge 2$.
    Tylko jedna implikacja wymaga dowodu, w~prawo.
    Bez straty ogólności załóżmy więc, że $q > r$, $p > s$.
    Skoro węzły $T_{q, r}$ i~$T_{p,s}$ są równoważne, to porównanie najwyższych współczynników w~ich wielomianach Alexandera daje równość $(q-1)(r-1) = (p-1)(s-1)$.
    Wymnożenie wszystkiego prowadzi do czterech przypadków: $s = r$, $s = ps$, $qr = r$, $qr = ps$, z~których dwa środkowe nie mogą zachodzić (gdyż $p, q > 1$).
    Z czwartego wynika, że $qr \le s < ps$, czyli sprzeczność.
\end{proof}

Kawauchi pisze, że wcześniej klasyfikacja węzłów torusowych wynikała z klasyfikacji wolnych produktów $(\Z/p) * (\Z/q)$, które są ilorazami grup węzłów torusowych \cite{schreier24}.

Wartości wielomianu Jonesa podajemy bez dowodu:

\begin{proposition}
\index{klamra Kauffmana}%
    Klamra Kauffmana spełnia zależność rekurencyjną
    \begin{equation}
        \bracket{T_{2, n}} = A \bracket{T_{2,n-1}} + (-1)^{n-1} A^{2-3n}
    \end{equation}
    z warunkiem brzegowym $\bracket{T_{2,1}} = -A^3$.
\end{proposition}

\begin{proposition}
\index{wielomian!Jonesa}%
    Niech $L = T_{p, q}$ będzie węzłem torusowym.
    Wtedy jego wielomianem Jonesa jest
    \begin{equation}
        \jones(t) = \frac {{\sqrt t}^{(p-1)(q-1)}}{1-t^2} \cdot (1 - t^{p+1} - t^{q+1} + t^{p+q}).
    \end{equation}
\end{proposition}

\index{węzeł!torusowy|)}%

% Koniec sekcji Węzły torusowe



\section{Węzły satelitarne}

% DIKTJONARY;latitude;szerokość geograficzna;geografia
% DIKTJONARY;longitude;długość geograficzna;geografia
% DIKTJONARY;meridian (of longitude);południk;geografia
% DIKTJONARY;parallel (of latitude);równoleżnik;geografia
% DIKTJONARY;---;geografia;-
% to jest powtórzenie np. z pliku 103a
Załóżmy, że w dopełnieniu pewnego splotu został zanurzony torus.
Jeżeli jest ściśliwy, to albo równoleżnikl torusa ogranicza dysk w dopełnieniu splotu~i torus jest niezawęźlony, albo południk ogranicza dysk w dopełnieniu splotu i~splot nie przebiega wzdłuż torusa.
Żadna z~tych sytuacji nie jest ciekawa.
Inny zdegenerowany przypadek występuje, gdy torus stanowi rurowe otoczenie jednego z~ogniw splotu.
W przeciwnym razie splot można zbudować z~prostszych obiektów.

Oto formalny opis konstrukcji.
Niech $W$ będzie pełnym torusem.
Dysk zanurzony w $W$, którego brzeg stanowi nieściągalną pętlę w $\partial W$, nazywamy południkowym.
Mówimy, że zamknięta krzywa $\lambda \subseteq W$ jest właściwa, jeżeli przecina wszystkie dyski południkowe.

\begin{definition}[węzeł satelitarny]
    \index{węzeł!satelitarny}
    Niech $P$ będzie splotem zanurzonym w~niezawęźlonym torusie $W$ tak, by co najmniej jedno z~ogniw stanowiło właściwą pętlę w~$W$.
    Niech $C$ będzie węzłem, zaś $V$ jego rurowym otoczeniem.
    Wybierzmy dowolny homeomorfizm $h \colon W \to V$.
    Wtedy splot $S = h(P)$ nazywamy satelitą o~wzorcu $P$ oraz towarzyszu $C$.
\end{definition}

% \begin{definition}
%     Węzeł nazywamy satelitarnym, jeśli zawiera nieściśliwy, nierównoległy do brzegu torus we własnym dopełnieniu.
% \end{definition}

Hoste i inni podejrzewają w~\cite{thistlethwaite98}, że jeśli satelita owija się $m$-krotnie wokół torusa, zaś indeks skrzyżowaniowy towarzysza wynosi $k$, to satelita nie posiada diagramu o~mniej niż $km^2$ skrzyżowaniach.
\index[persons]{Hoste, Jim}%
\index[persons]{Thistlethwaite, Morwen}%
\index[persons]{Weeks, Jeff}%

Ponieważ dla trójlistnika $k = 3$, napotkali się tylko na satelity owijające się $m = 2$ razy podczas tablicowania pierwszych węzłów do 16 skrzyżowań.
Nie spodziewano się żadnego satelity ósemki, gdyż wtedy $k = 4$, zatem każdy satelita miałby co najmniej $4 \cdot 2^2 + 1 = 17$ skrzyżowań: dodatkowe $+1$ jest potrzebne, by nie dostać splotu o~dwóch ogniwach.

Najprostszy satelita ma 13 skrzyżowań.
Ze strony internetowej programu Regina można dowiedzieć się dokładniej, jak wygląda rozkład satelitów wśród małych węzłów:

\renewcommand*{\arraystretch}{1.4}
\footnotesize
\begin{longtable}{lccccccc}
    \hline
    \textbf{skrzyżowania} & 13 & 14 & 15 & 16 & 17 & 18 & 19 \\ \hline \endhead
    węzły pierwsze, satelitarne & 2 & 2 & 6 & 10 & 29 & 86 & 245 \\
    \hline
\end{longtable}
\normalsize

Razem 380 węzłów.

\begin{example}[torus połykająco-podążający]
% DICTIONARY;swallow-follow;połykająco-podążający;torus
% DICTIONARY;torus;torus;-
    Klasa węzłów satelitarnych obejmuje węzły złożone.
    W ich przypadku można wskazać pewien szczególny torus nieściśliwy -- połykający pierwszy składnik, a~potem podążający za drugim:\footnote{Źródło obrazka: \url{https://mcm-www.jwu.ac.jp/~hayashic/semi/07/07i/07i.html}}
    \begin{figure}[H]
        \centering
        \includegraphics[width=0.75\linewidth]{../data/mixed/follow-swallow.png}
        \caption[something]{Torus połykająco-podążający. Źródło: strona internetowa  C. Hayashiego.}
    \end{figure}
\end{example}

Schubert pokazał, że zorientowane klasy izotopii węzłów w~$S^3$ tworzą wolny przemienny monoid na przeliczalnie wielu generatorach.
\index[persons]{Schubert, Horst}%
Dowód to uważna analiza nieściśliwych torusów obecnych w~dopełnieniu sumy spójnej.
To doprowadziło go do definicji węzłów satelitarnych i~towarzyszących w~przełomowej pracy \cite{schubert53} oraz zunifikowało teorię 3-rozmaitości z teorią węzłów.
Może warto zapoznać się z pracą Motegiego \cite{motegi97}?
Wikipedia mówi, że zapis węzła jako satelity nie jest jednoznaczny, a~tam mogą być przykłady.
%=% https://en.wikipedia.org/wiki/Satellite_knot#cite_ref-7

Na brzegu torusa $V$ można wprowadzić pewien układ współrzędnych: południk to pętla właściwa w $\partial V$, która ogranicza dysk w $V$, natomiast równoleżnik to pętla w $\partial V$, która spotyka południk raz.
Z~dokładnością do izotopii południk jest jeden, ale równoleżnik nie.
Równoleżnik preferowany to taki, którego indeks zaczepienia z~rdzeniem torusa wynosi zero.

\begin{definition}[dubel Whiteheada]
\index{dubel Whiteheada}%
    Jeżeli $P \subseteq W$ jest skręconym jednokrotnie niewęzłem, to węzeł $S$ nazywamy dublem Whiteheada.
\end{definition}

Każdy węzeł posiada nieskończenie wiele dubli Whiteheada: wystarczy rozciąć torus $V$, skręcić jedną końcówkę i~ponownie zszyć, żaden z~nich nie jest odróżniany od niewęzła przez wielomian Alexandera.

Wyróżnia się pewien szczególny homeomorfizm $h$, który przenosi południk i preferowany równoleżnik $W$ na południk i preferowany równoleżnik $V$.
Nazywamy go wiernym.
% DICTIONARY;faithful homeomorphism;homomorfizm wierny;-
O dublu względem wiernego homeomorfizmu mówimy, że jest nieskręcony.
% TODO: skąd powyższe zdanie? jak jest nieskręcony po angielsku?

\begin{definition}[węzeł kablowy]
\index{węzeł!kablowy}%
    Niech $h \colon W \to V$ będzie wiernym homeomorfizmem, zaś $P$ węzłem $(p, q)$-torusowym.
    Satelitę $S$ nazywamy węzłem $(p, q)$-kablowym albo krótko kablem.
\end{definition}

\begin{proposition}
    Każdy kabel wyznacza jednoznacznie węzeł, z~którego powstał.
\end{proposition}

\begin{proof}
\index[persons]{Feustel, Charles}%
\index[persons]{Whitten, Wilbur}%
    Wniosek 2 z~pracy \cite{feustel78} Feustela, Whittena pokazuje, że na podstawie kabla można wyznaczyć parametry węzła torusowego $K'_{p,q}$ oraz topologię dopełnienia oryginalnego węzła.
    Wiemy jednak z~twierdzenia Gordona-Lueckego, że różne węzły PIERWSZE mają różne dopełnienia.
\end{proof}

Niewęzeł nie ma nietrywialnych węzłów towarzyszących.

\begin{definition}
    Towarzysza $C$ nietrywialnego splotu nazywamy właściwym, jeśli nie jest niewęzłem i~nie jest ogniwem tego splotu.
\end{definition}

Sploty bez właściwych towarzyszy określa się zazwyczaj terminem ,,atoroidalny''.
\index{splot!atoroidalny}%
Patrz też diagram przedstawiony w \cite{cromwell04} na stronie 83.

\begin{proposition}
    Duble nietrywialnych węzłów oraz kable są pierwsze.
\end{proposition}

\begin{proof}
    Prosty wniosek z~twierdzenia 4.4.1 w~\cite[s. 84]{cromwell04}: jeżeli wzorzec jest niewęzłem lub węzłem pierwszym, to każdy właściwy satelita jest pierwszy.
\end{proof}

Niektóre węzły przedstawiają się jako satelity w~dokładnie jeden sposób, inne nie.
Rok 1979 przyniósł amerykańską pracę \cite{jaco79} oraz niemiecką książkę\footnote{Według recenzji Hempela, najważniejsze tam jest twierdzenie klasyfikacyjne: niech $M_1, M_2$ będą 3-rozmaitościami Hakena \index{rozmaitość!Hakena} z brzegiem, zaś $V_1, V_2$ ich podrozmaitościami charakterystycznymi, wtedy każda homotopijna równoważność $f \colon M_1 \to M_2$ można zdeformować tak, że jest homeomorfizmem między domknięciami: $M_1 \setminus V_1$ oraz $M_2 \setminus V_2$ i~homotopijną równoważnością między $V_1$ oraz $V_2$.} \cite{johannson79}, gdzie niezależnie od siebie opisano jednoznaczny rozkład (nazywany teraz) Jaco-Shalena-Johannsona:
\index[persons]{Jaco, William}%
\index[persons]{Shalen, Peter}%
\index[persons]{Johannson, Klaus}%

\begin{proposition}
    Niech $M$ będzie nierozkładalną, orientowalną, domkniętą 3-rozmaitością.
    Istnieje wtedy jedyna z dokładnością do izotopii minimalna rodzina rozłącznie zanurzonych nieściśliwych torusów tak, że każda składowa 3-rozmaitości powstałej przez rozcinanie wzdłuż torusów jest atoroidalna lub włóknistą przestrzenią Seiferta ($S^1$-wiązką nad dwuwymiarowym orbifoldem).
\index{orbifold}%
\index{rozmaitość!atoroidalna}%
\index{przestrzeń!włóknista Seiferta}%
\index{wiązka ($S^1$)}%
\end{proposition}

Jest on związany z operacją złączania (ang. \emph{splicing}), będącej uogólnieniem budowania satelitów, sumy spójnej, dubli Whiteheada i kasowania ogniwa.
\index{złączanie (splicing)}%
% https://arxiv.org/pdf/math/0506523.pdf
% DICTIONARY;splicing;złączanie;-
Hipotezę o jedyności rozkładu wysnuł wcześniej Waldhausen.
\index[persons]{Waldhausen, Friedhelm}%

% Koniec sekcji Węzły satelitarne



\section{Węzły hiperboliczne}
\index{węzeł!hiperboliczny|(}
\label{sec:hyperbolic}
Jak pisaliśmy w~sekcji \ref{sec:mutant}, słynne węzły Conwaya oraz Kinoshity-Terasakiego odróżnił od siebie po raz pierwszy Riley.
\index[persons]{Riley, Robert}
\index{węzeł!Conwaya}%
\index{węzeł!Kinoshity-Terasakiego}%
Zbadał paraboliczne reprezentacje ich grup w~skończoną grupę prostą $PSL(2, 7)$, co doprowadziło go do odkrycia struktury hiperbolicznej w~dopełnieniu ósemki \cite{riley75}.
\index{ósemka}
% https://arxiv.org/pdf/2002.00564.pdf
Zainspirowany tym wynikiem Thurston najpierw rozłożył dopełnienie ósemki na dwa idealne wielościany, a~potem znacznie uogólnił swój przykład.

Reszta sekcji powstała na podstawie dwóch źródeł: przeglądowej pracy Kalfagianniego, Futera oraz Purcell \cite{purcell19} i~notatek z~wykładów, które były prowadzone przez samą Purcell.
\index[persons]{Futer, David}%
\index[persons]{Kalfagianni, Efstratia}%
\index[persons]{Purcell, Jessica}%
Wiedzę o~węzłach hiperbolicznych można czerpać także z~artykułu Weeksa \cite{weeks05}.
\index[persons]{Weeks, Jeff}%

% Badając sploty nie ograniczamy się tylko do diagramów, ale korzystamy też z ich dopełnień, to znaczy 3-rozmaitości $S^3 \setminus L$.
% Jest ona homeomorficzna z wnętrzem zwartej rozmaitości $X(L) = S^3 \setminus N(K)$, zwanej zewnętrzem splotu, gdzie $N(L)$ stanowi rurowe otoczenie splotu.
% Dalej możemy stosować maszynierę topologii 3-rozmaitości.

% \begin{definition}[ściśliwy]
%     Niech orientowalna powierzchnia $S$ będzie właściwie\footnote{properly} zanurzona w zwartej, orientowalnej 3-rozmaitości $M$.
%     Załóżmy, że dla każdego dysku $E \subseteq M$ z brzegiem $\partial E \subseteq S$ istnieje dysk $E' \subseteq S$ taki, że $\partial E = \partial E'$.
%     Mówimy wtedy, że powierzchnia $S$ jest nieściśliwa.
% \end{definition}

% $\partial$-ściśliwość

% essential

% Haken

% monodromy of fibration

% Węzły i sploty, które będziemy rozpatrywać dalej, mają szczególną strukturą geometryczną.

\begin{definition}[hiperboliczny]
    Splot $L$, na dopełnieniu którego można zadać zupełną metrykę o~stałej krzywiźnie $-1$ nazywamy hiperbolicznym.
\end{definition}

\begin{proposition}
    Niech $L$ będzie splotem, zaś $\mathbb H^3$ hiperboliczną 3-przestrzenią.
    Splot $L$ jest hiperboliczny wtedy i~tylko wtedy, gdy $S^3 \setminus L = \mathbb H^3 / \Gamma$, gdzie $\Gamma$ jest dyskretną, beztorsyjną grupą izometrii, izomorficzną z~$\pi_1(S^3 \setminus L)$.
\end{proposition}
% TODO: skąd to jest dokładnie?

Thurston podejrzewał, że każda 3-rozmaitość rozkłada się wzdłuż sfer i~nieściśliwych torusów na części wyposażone w~jedną z~ośmiu kanonicznych geometrii:
\begin{itemize}
\item sferyczną $S^3$, albo euklidesową $E^3$, albo hiperboliczną $H^3$,
\item $S^2 \times \R$, albo $H^2 \times \R$,
\item uniwersalne nakrycie $SL(2, \R)$,
\item geometrię Sol albo geometrię Nil.
\end{itemize}
Nie umiał podać pełnego uzasadnienia, w~pracy \cite{thurston82} udowodnił swoje przypuszczenie dla rozmaitości Hakena.
\index{rozmaitość Hakena}
Dowód hipotezy geometryzacyjnej dostarczył mniej więcej dwie dekady później Perelman, nie to jest jednak dla nas najważniejsze.
\index{hipoteza!geometryzacyjna Thurstona}
Z przełomowych prac Thurstona z~lat 70. oraz 80. wynika coś ciekawszego: że dopełnienie węzła jest rozmaitością włóknistą Seiferta, toroidalną albo hiperboliczną.
Innymi słowy, Thurston przedstawił trychotomię:

\begin{theorem}
    \index{twierdzenie!Thurstona}
    Każdy węzeł jest satelitarny, torusowy albo hiperboliczny.
    \index{węzeł!satelitarny}
    \index{węzeł!torusowy}
\end{theorem}
% luźno związane: http://www.deltami.edu.pl/temat/matematyka/topologia/2012/12/27/%William_Thurston_i_hipoteza_geometryzacyjna/

\begin{proof}
    Thurston w~\cite[wniosek 2.5]{thurston82}.
\end{proof}

% https://mathoverflow.net/a/289359 : hyperbolic, toroidal (that is, satellite), or Seifert fibered

Węzły hiperboliczne stanowią najliczniejszą i~najmniej zrozumianą rodzinę węzłów.
Sam Nead, użytkownik portalu MathOverflow napisał, że kryterium Thurstona dzięki maszynerii JSJ oraz pracom innych osób można wysłowić algebraicznie.

\begin{proposition}
    % There is a topological criterion due to Thurston.  Using the JSJ machine (and work of many others) this criterion can also be phrased algebraically.  I'll essay these below.  Please note that the situation is much simpler for knots.  To answer your question most directly, here is the desired reference to Wikipedia.
    % http://en.wikipedia.org/wiki/Hyperbolic_link
    % This page refers to the books of Colin Adams and William Thurston.  Both are excellent.
    % Now, here is Thurston's criterion. (EDIT: exposition improved after reading Bruno Martelli's answer.)
    % Suppose that $L$ is the link and $X$ is the link complement.  Suppose $\pi = \pi_1(X)$. We assume the following properties (and each property assumes the proceeding ones). $\newcommand{\ZZ}{\mathbb{Z}}$
    %  - $L$ is not a split link.  Equivalently, $X$ is contains no essential two-sphere.  Equivalently, $\pi$ is not a free product.
    %  - $L$ is not the unknot. Equivalently, $X$ contains no essential disk. Equivalently, $\pi$ is not $\ZZ$.
    %  - $L$ has no component that is an "undisturbed satellite knot".  Equivalently, $X$ contains no essential torus.
    %  - $L$ is not a torus knot. Equivalently, $X$ contains no essential annulus. These last two topological properties are equivalent to $\pi$ not containing a copy of $\ZZ^2$.
    % Then $X$ admits a hyperbolic structure.
    Niech $L$ będzie splotem, który nie rozszczepia się, nie jest niewęzłem, nie posiada wśród ogniw niezakłóconego węzła satelitarnego oraz nie jest węzłem torusowym.
    \index{splot!rozszczepialny}
    Wtedy $L$ jest hiperboliczny.
\end{proposition}

\begin{proposition}
    Niech $L$ będzie splotem takim, że jego dopełnienie $S^3 \setminus L$ nie zawiera właściwej 2-sfery, właściwego dysku, właściwego torusa oraz właściwego pierścienia.
    Wtedy $L$ jest hiperboliczny.
\end{proposition}

\begin{proposition}
    Niech $L$ będzie splotem takim, że jego grupa $\pi(S^3 \setminus L)$ nie jest produktem wolnym, nie jest izomorficzna z~$\Z$ oraz nie zawiera w~sobie kopii grupy $\Z \oplus \Z$.
    Wtedy $L$ jest hiperboliczny.
\end{proposition}

\begin{proof}
    Patrz \url{https://mathoverflow.net/a/153327}.
\end{proof}

Czas na podanie jakichś przykładów węzłów hiperbolicznych, za \cite{adams05}.

\begin{proposition}
    Każdy alternujący, pierwszy, oraz nierozszczepialny splot jest albo 2-warkoczem (a zatem, torusowy) albo hiperboliczny.
    \index{węzeł!alternujący}
    \index{węzeł!pierwszy}
    \index{węzeł!rozszczepialny}
\end{proposition}

\begin{proof}
\index[persons]{Menasco, William}%
    Menasco \cite{menasco84} pokazał, że dopełnienie alternującego węzła nie zawiera nieściśliwych nieperyferyjnych torusów.
    To w~połączeniu z~unifikacyjnym twierdzeniem Thurstona dla rozmaitości Hakena kończy dowód.
\index{rozmaitość Hakena}%
    % zarys dowodu zz MathSciNet
\end{proof}

\begin{proposition}
    Nietrywialne pierwsze prawie alternujące węzły są torusowe albo hiperboliczne.
    \index{węzeł!pierwszy}
    \index{węzeł!prawie alternujący}
\end{proposition}

\begin{proof}
    Grupa studentów pod opieką Adamsa w \cite{brock92}.
\end{proof}

\begin{proposition}
    Toroidalnie alternujące węzły pierwsze są torusowe albo hiperboliczne.
    \index{węzeł!pierwszy}
    \index{węzeł!toroidalnie alternujący}
\end{proposition}

Ze wszystkich węzłów pierwszych do 11 skrzyżowań i~pierwszych, nierozszczepialnych splotów do 10 skrzyżowań tylko 3 węzły i~2 sploty nie są toroidalnie alternujące, tak twierdzi Adams \cite{adams05}.

\begin{proof}
    Patrz \cite{adams994}.
\end{proof}

\begin{proposition}
\index{splot!Montesinosa}%
    Sploty Montesinosa są prawie zawsze torusowe albo hiperboliczne.
\end{proposition}

\begin{proof}
    Najpierw zidentyfikowano sploty Montesinosa torusowe, które są też torusowe \cite{boileau80}.
    % TODO: torusowe, które są torusowe???
    Potem w~pracy \cite{oertel84} znaleziono listę wyjątków (z chyba trochę inną notacją niż nasza):
    \begin{itemize}
        \item $K(1/2, 1/2, 21/2, 21/2)$,
        \item $K(2/3, 21/3, 21/3)$,
        \item $K(1/2, 21/4, 21/4)$,
        \item $K(1/2, 21/3, 21/6)$,
        \item lub lustra tych splotów. \qedhere
    \end{itemize}
\end{proof}

\begin{proposition}
    Mutant węzła hiperbolicznego jest węzłem hiperbolicznym.
    \index{mutant}
\end{proposition}

\begin{proof}
\index[persons]{Ruberman, Daniel}%
    Ruberman w~\cite{ruberman87}, patrz wniosek 1.4.
\end{proof}

\begin{proposition}
    Niech $G$ oznacza grupę izometrii wnętrza dopełnienia węzła hiperbolicznego.
    Wtedy $G$ jest diedralna lub skończona cykliczna.
\end{proposition}

\begin{proof}
\index[persons]{Riley, Robert}%
\index[persons]{Kodama, Kouzi}%
\index[persons]{Sakuma, Makoto}%
    Pierwszy był Riley w~artykule \cite[s. 124]{riley79}, można też zapoznać się z~późniejszą pracą \cite{kodama92} Kodamy i Sakumy.
    % Kodama - lemat_1.1
\end{proof}

Kawauchi \cite[s. 131]{kawauchi96} wprowadza jeszcze jedną grupę (grupę symetrii węzła): iloraz grupy PL automorfizmów pary $(S^3, K)$ przez podgrupę elementów, które są otaczająco izotopijne z~odwzorowaniem tożsamościowym.
Okazuje się, że nie wszystkie są skończone:

\begin{proposition}
    Węzeł $K$ ma skończoną grupę symetrii wtedy i~tylko wtedy, gdy jest hiperboliczny, torusowy lub kablem węzła torusowego.
\end{proposition}

Kawauchi nie podaje dowodu, ale zaleca zajrzeć do pracy Sakumy.
Ja zajrzałem i dalej nie mam pojęcia, jak ten dowód miałby wyglądać.

Z kryterium Thurstona mamy prosty wniosek (bo węzły złożone są satelitarne):

\begin{corollary}
    Każdy węzeł hiperboliczny jest pierwszy.
    \index{węzeł!pierwszy}
\end{corollary}

Prawie każdy węzeł pierwszy o~mniej niż 17 skrzyżowaniach jest hiperboliczny, na 32 wyjątki składa się 12 węzłów torusowych oraz 20 satelitów trójlistnika.
Te ostatnie mają co najmniej 13 skrzyżowań.
Baza ciągów liczb całkowitych OEIS zawiera informacje na temat liczności poszczególnych typów węzłów.
Analizując ciągi A051764, A051765 oraz A052408 można dojść do wniosku, że wraz ze wzrostem liczby skrzyżowań, stosunek liczby węzłów hiperbolicznych do wszystkich węzłów dąży do $1$:

\begin{figure}[H]
\renewcommand*{\arraystretch}{1.4}
\footnotesize
\begin{longtable}{lcccccccccccccc}
\hline
    \textbf{rodzaj} & 3 & 4 & 5 & 6 & 7 & 8  & 9  & 10  & 11  & 12   & 13   & 14    & 15     \\ \hline \endhead
    torusowe        & 1 & 0 & 1 & 0 & 1 & 1  & 1  & 1   & 1   & 0    & 1    & 1     & 2      \\
    satelitarne     & 0 & 0 & 0 & 0 & 0 & 0  & 0  & 0   & 0   & 0    & 2    & 2     & 6      \\
    hiperboliczne   & 0 & 1 & 1 & 3 & 6 & 20 & 48 & 164 & 551 & 2176 & 9985 & 46969 & 253285 \\
    \hline
\end{longtable}
\normalsize
\end{figure}

W pracy \cite{malyutin16} Malutin pokazał jednak, że to przypuszczenie jest sprzeczne z~wieloma innymi starymi hipotezami teorii węzłów: \ref{con:malyutin1} -- \ref{con:malyutin4}.
\index[persons]{Malyutin, Andrei}%

\begin{conjecture}
    \label{con:malyutin1}
    Indeks skrzyżowaniowy jest addytywny względem sumy spójnej.
    \index{indeks skrzyżowaniowy}
    \index{suma spójna}
\end{conjecture}

(To jest powtórzenie hipotezy \ref{con:crossing_additive}).

\begin{conjecture}
    Satelita ma większy (w słabszej wersji: nie mniejszy) indeks skrzyżowaniowy niż jego towarzysze.
    \index{węzeł!satelitarny}
\end{conjecture}

Lackenby pokazał w~\cite{lackenby14}, że jeśli $K$ jest satelitą z~towarzyszem $L$, to $\crossing K \ge 10^{-13} \crossing L$.
\index[persons]{Lackenby, Marc}%

\begin{conjecture}
%label{con:malyutin3}
    Węzeł złożony ma większy (w słabszej wersji: nie mniejszy) indeks skrzyżowaniowy niż jego składniki.
    \index{węzeł!pierwszy}
\end{conjecture}

Mówimy, że węzeł pierwszy $P$ jest $\lambda$-regularny, jeśli $\crossing K \ge \lambda \cdot \crossing P$ za każdym razem, gdy węzeł $P$ jest składnikiem węzła $K$.
\index{węzeł!regularny}
Zatem hipotezę można wysłowić krótko ,,węzły pierwsze są $1$-regularne''.
Z tego, co pisaliśmy po hipotezie \ref{con:crossing_additive} wynika, że hipoteza \ref{con:malyutin1} jest prawdziwa w~klasie węzłów alternujących czy torusowych i~że wszystkie węzły są $1/152$-regularne.

\begin{conjecture}
    \label{con:malyutin4}
    Węzły pierwsze są $2/3$-regularne.
\end{conjecture}

Rozwiązanie zagadki przyniosła praca samego Malutina \cite{malyutin19} opublikowana latem 2019 roku, przynajmniej dla splotów.
\index[persons]{Malyutin, Andrei}%
Pokazał w~niej, że jeśli oznaczymy liczbę splotów pierwszych i~nierozszczepialnych o~$n$ lub mniej skrzyżowaniach przez $P_n$, zaś liczbę hiperbolicznych splotów, także o~$n$ lub mniej skrzyżowaniach, przez $H_n$, prawdziwe będzie oszacowanie
\index{splot!rozszczepialny}
\index{węzeł!pierwszy}
\begin{equation}
    \liminf_{n \to \infty} \frac{H_n}{P_n} < 1 - 10^{-13}.
\end{equation}

Czwarty rozdział książki \cite{purcell20} zawiera ćwiczenie, by znaleźć dwuparametrową rodzinę zupełnych struktur hiperbolicznych na dziurawym torusie oraz czterokrotnie dziurawej sferze.
Elastyczność tego rodzaju nie występuje w~przestrzeniach wyższych wymiarów.
Z~twierdzenia o~sztywności, w~wersji algebraicznej:

\begin{theorem}[Mostow-Prasad]
    \index{twierdzenie!o sztywności}
    Niech $\Gamma_1, \Gamma_2$ będą dyskretnymi podgrupami grupy izometrii $\mathbb H^n$ dla $n \ge 3$ takimi, że ilorazy $\mathbb H^n/\Gamma_i$ mają skończone objętości.
    Załóżmy też, że istnieje izomorfizm grup $\varphi \colon \Gamma_1 \to \Gamma_2$.
    Wtedy podgrupy $\Gamma_1, \Gamma_2$ są sprzężone.
\end{theorem}
\index{twierdzenie!Mostowa-Prasada}

albo geometrycznej:

\begin{theorem}[Mostow-Prasad]
    Niech $M_1, M_2$ będą zupełnymi, hiperbolicznymi rozmaitościami o skończonych objętościach.
    Wtedy każdy izomorfizm grup podstawowych $\varphi \colon \pi_1(M_1) \to \pi_1(M_2)$ realizowany jest jednoznacznie przez izometrię.
\end{theorem}

wynika, że jeśli znaleźliśmy jakąś zupełną strukturę hiperboliczną na dopełnieniu splotu, to innych już nie ma.

\begin{proof}
\index[persons]{Benedetti, Riccardo}%
\index[persons]{Mostow, George}%
\index[persons]{Petronio, Carlo}%
\index[persons]{Prasad, Gopal}%
\index[persons]{Thurston, William}%
    Thurston przedstawił szkic rozumowania w~sekcji 5.9 swoich notatek, na bazie których powstała później książka \cite{thurston97}.
    Inne szczegółowe rozumowanie można znaleźć w~rozdziale C podręcznika Benedettiego, Petronio \cite{benedetti92}.
    Patrz także oryginalne prace: Mostowa \cite{mostow73}, Prasada \cite{prasad73}.
\end{proof}

Twierdzenie Mostowa-Prasada pozwala nam na wprowadzenie nowych niezmienników splotów hiperbolicznych: wystarczy wziąć dowolny geometryczny niezmiennik dopełnienia węzła.
Najważniejszym z~nich wydaje się być objętość.


\subsection{Objętość hiperboliczna}

\index{objętość|(}
\begin{definition}[objętość]
    Niech $L$ będzie splotem hiperbolicznym.
    Objętość dopełnienia $L$ względem zupełnej metryki hiperbolicznej nazywamy objętością splotu $L$ i~oznaczamy $\volume L$.
\end{definition}

Objętość jest zawsze skończoną liczbą rzeczywistą.
Dla wygody przyjmuje się czasami, że objętość węzłów torusowych oraz satelitarnych wynosi $0$.
Komputerowy program SnapPea napisany przez Weeksa pozwala na wyznaczenie objętości dowolnego splotu o~rozsądnej ilości skrzyżowań.
\index{SnapPea}

\begin{example}
    $\volume 4_1 = -6 \int_{0}^{\pi/3} \log |2\sin \theta| \,\mathrm{d}\theta \approx 2.0298832$.
\end{example}

Patrz też ciąg \href{https://oeis.org/A091518}{A091518} w~bazie danych OEIS.
Jak zobaczymy później, żaden węzeł nie ma mniejszej objętości.

\begin{example}
    $\volume 5_2 \approx 2.82812$.
\end{example}

W encyklopedii Wolfram Mathworld znajduje się informacja, że $5_2$ oraz pewien węzeł o~dwunastu skrzyżowaniach mają tę samą objętość, prawdopodobnie chodzi tu o~$12n_{242}$, który znany jest także jako $(-2, 3, 7)$-precel.
\index{precel!(-2, 3, 7)}

\begin{example}
    $\volume 6_1 \approx 3.16396$.
\end{example}

\begin{example}
    $\volume 6_2 \approx 4.40083$.
\end{example}

\begin{example}
    $\volume 6_3 \approx 5.69302$.
\end{example}

\begin{example}
    $\volume 7_4 \approx 5.13794$.
\end{example}

\begin{example}
    Niech $K$ będzie jednym z~dwóch węzłów w~parze Perko.
    Wtedy $\volume K \approx 5.63877$.
    \index{para Perko}
\end{example}

Praca \cite{purcell19} wspomina kilka przyjemnych ograniczeń, jakie musi spełniać objętość.
\index[persons]{Futer, David}%
\index[persons]{Kalfagianni, Efstratia}%
\index[persons]{Purcell, Jessica}%
Aby je przytoczyć, musimy najpierw zdefiniować dwie stałe: $v_4$ oraz $v_8$, odpowiednio objętość idealnego czworościanu\footnote{Albo rozmaitości Giesekinga, powstałej z czworościanu przez usunięcie  wierzchołków i sklejenie ściany 012 z 310 oraz 023 z 321. Dopełnienie ósemki jest podwójnym nakryciem tej rozmaitości.\index{rozmaitość Giesekinga}} oraz ośmiościanu foremnego w~$\mathbb H^3$.
Mamy
\begin{align}
    v_4 & = \int_{0}^{2\pi/3} \log(2 \cos(\theta/2)) \,\mathrm{d}\theta \approx 1.01494\,16064, \\
    % https://en.wikipedia.org/wiki/Gieseking_manifold
    v_8 & = 4 \sum_{n=0}^\infty \frac{(-1)^n}{(2n+1)^2} \approx 3.66386\,23767. % ... 08876060218414059729536443096597497126688537065 ... \ldots
\end{align}

I tak najpierw Adams pokazał w~swojej rozprawie doktorskiej \cite{adams83}:
\index[persons]{Adams, Colin}%

\begin{proposition}
    Niech $D$ będzie diagramem hiperbolicznego splotu o~$\crossing L \ge 5$ skrzyżowaniach.
    Wtedy
    \begin{equation}
        \volume L \le 4 (\crossing D - 4) v_4.
    \end{equation}
\end{proposition}

A trzy dekady później poprawił wswój wynik w~\cite{adams13}:

\begin{proposition}
    Niech $D$ będzie diagramem hiperbolicznego splotu o~$\crossing L \ge 5$ skrzyżowaniach.
    Wtedy
    \begin{equation}
        \volume L \le (\crossing D - 5) v_8 + 4v_4.
    \end{equation}
\end{proposition}

Jego metoda polega na podzieleniu dopełnienia splotu na czterościany i~ośmiościany oraz policzeniu ich.
To, w~połączeniu ze znanymi ograniczeniami na objętość ,,cegiełek'', wystarcza.
Podział na ośmiościany zaproponował Dylan (nie William!) Thurston.
% wiem to z purcell19

Thurston zauważył \cite[s. 365]{thurston82}, że tylko skończenie wiele hiperbolicznych 3-rozmaitości może mieć tę samą objętość -- wynika to z~prac Gromowa i~Jørgensena.
Następnie Wielenberg przedstawił w~\cite{wielenberg81} przykłady pokazujące, że istnieją dowolnie duże kolizje wśród węzłów hiperbolicznych: pewne podgrupy klasycznej grupy Picarda działają jako izometrie na górną półprzestrzestrzeń hiperboliczną wymiaru 3 mają podstawowe wielościany, które są takie same jako zbiory, ale różnią się jeśli chodzi o~utożsamienie ze sobą ścian.

Chociaż mutanty mają tę samą objętość hiperboliczną (fakt \ref{mutants_the_same_volume}), to praktyka pokazuje, że ten niezmiennik dobrze wspomaga proces tablicowania węzłów.

\begin{proposition}
    Zbiór
    \[
        \{\volume K: K \textrm{ jest hiperboliczny}\} \subseteq \R
    \]
    jest dobrze uporządkowany, typu porządkowego $\omega^\omega$.
\end{proposition}

\begin{proof}[Niedowód]
    Zdaniem angielskiej Wikipedii, dowód jest gdzieś w~\cite{neumann85} (gdzie Neumann znajduje eleganckie oszacowanie zmiany objętości po wykonaniu chirurgii Dehna), ja tego nie widzę.
    %=% wikipedia - angielski artykuł "hyperbolic volume" 
    Hodgson, Masa \cite{hodgson13} sugerują, że dowód da się znaleźć w notatkach Thurstona \cite{thurston02}.
    % TODO: https://mathscinet.ams.org/mathscinet-getitem?mr=648524 sugeruje, że to jest tam: "The order type of the set of all volumes of hyperbolic 3-manifolds is ω^ω."
\end{proof}

W dowolnej rodzinie węzłów istnieje element o~najmniejszej objętości.
Przytoczę teraz przykłady konkretnych rodzin i najmniejszych węzłów, za Futerem, Kalfagiannim, Purcell \cite[s. 16-17]{purcell19} oraz Hodgsonem, Masaiem\cite[s. 1-99]{hodgson13}.
\index[persons]{Futer, David}%
\index[persons]{Kalfagianni, Efstratia}%
\index[persons]{Purcell, Jessica}%
\index[persons]{Hodgson, Craig}%
\index[persons]{Masai, Hidetoshi}%

\begin{proposition}
%label{prp:eight_least_hyperbolic}
    Żaden węzeł nie ma mniejszej objętości hiperbolicznej od ósemki.
    \index{ósemka}
\end{proposition}

\begin{proof}
    Cao, Meyerhoff w~\cite{cao01} przeanalizowali pakowania horokul w~uniwersalnym nakryciu związanym z~rozmaitościami.
    Doszli do wniosku, że nie ma tam dostatecznieo wolnego miejsca, jeżeli szpic (cusp) nie jest odpowiedniego rozmiaru.
    Trzykrotnie wspierają się przy tym pomocą komputera, by sprawdzić, że określone warunki są spełnione we wszystkich punktach danej przestrzeni parametrów.
\end{proof}

\begin{proposition}
% DICTIONARY;cusped;szpiczasta;rozmaitość
% DICTIONARY;manifold;rozmaitość;-
\index{splot!Whiteheada}%
\index{ósemka}%
    Wśród orientowalnych 3-rozmaitości ze szpicem\footnote{rozmaitość szpiczasta -- niezwarte, zupełne hiperboliczne rozmaitości ze skończoną objętością Riemanna} najmniejszą objętość posiada dopełnienie ósemki oraz jego bliźniak, otrzymany przez $(5, 1)$-chirurgię jednego z~ogniw splotu Whiteheada.
% sformułowanie wygląda jak z "THE MINIMAL VOLUME ORIENTABLE HYPERBOLIC 3-MANIFOLD WITH 4 CUSPS"
\end{proposition}

Klasa rozmaitości wspomniana w fakcie obejmuje dopełnienia hiperbolicznych węzłów.
Powyższy fakt także został wzięty z~pracy \cite{cao01}.

Meyerhoff nie przestawał pracować nad rozmaitościami o~małych objętościach i~osiem lat później w~\cite{meyerhoff09} przedstawił z Gabaiem, Milleyem bez dowodu (obiecali pokazać go później):

\begin{proposition}
    Istnieje 10 orientowalnych 3-rozmaitości z~jednym szpicem o~objętości co najwyżej $2.848$: \texttt{m003}, \texttt{m004} ($\approx 2.02988$), \texttt{m006}, \texttt{m007} ($\approx 2.56897$), \texttt{m009}, \texttt{m010} ($\approx 2.66674$), \texttt{m011} ($\approx 2.78183$), \texttt{m015}, \texttt{m016} oraz \texttt{m017} ($\approx 2.82812$).
    Nazwy pochodzą ze spisu rozmaitości programu SnapPy.
\end{proposition}

Udało mi się rozszyfrować niektóre nazwy.
\texttt{m003} to siostra $4_1$, % https://hal.archives-ouvertes.fr/hal-02867890/document Michel Planat - Quantum computing thanks to Bianchi groups
\texttt{m004} to węzeł $4_1$, % SnapPy - also known as
% m006
% m007
% m009
% m010
% m011
\texttt{m015} to węzeł $5_2$,
\texttt{m016} to węzeł $12n242$, czyli znany nam już $(-2, 3, 7)$-precel,
\index{precel!(-2, 3, 7)}%
\texttt{m017} to siostra $5_2$. % https://arxiv.org/pdf/2107.03275.pdf

W tej samej pracy możemy jeszcze znaleźć informację, że:

\begin{proposition}
    Istnieje dokładnie jedna domknięta hiperboliczna 3-rozmaitość o najmniejszej objętości, rozmaitość Weeksa.
\end{proposition}

Rozmaitość Weeksa została odkryta przez Jeffreya Weeksa w jego rozprawie doktorskiej (1985) oraz niezależnie przez Matwiejewa, Fomenko (1988).
\index{rozmaitość Weeksa}
Powstaje ona przez wykonanie $(5, 2)$ oraz $(5, 1)$ chirurgii Dehna na dopełnieniu splotu Whiteheada, zaś jej objętość wynosi w~przybliżeniu $0.94270$. % https://oeis.org/A126774
\index{splot!Whiteheada}

Następna jest rozmaitość Meyerhoffa, powstała po $(5, 1)$ chirurgii na dopełnieniu ósemki.
\index{rozmaitość Meyerhoffa}
Meyerhoff sugerował w 1987, że ma najmniejszą objętość, ale okazało się potem, że ta wynosi $\approx 0.98136$.

\begin{proposition}
    Wśród orientowalnych 3-rozmaitości o~dwóch szpicach najmniejszą objętość mają splot Whiteheada oraz $(-2, 3, 8)$-precel.
\index{splot!Whiteheada}%
\index{precel!(-2, 3, 8)}%
\end{proposition}

% TODO: check cusped manifold in dictionary

Ich objętość wynosi $v_8$.

\begin{proof}
    Agol \cite{agol10} korzystając z metod topologicznych dowodzi istnienia ,,niezbędnej'' (z ang. essential) powierzchni, która zadaje dolne ograniczenie na objętość i skutecznie krępuje rozmaitości, które mogą to ograniczenie zrealizować.
\end{proof}

Przypadek trzech szpiców nie jest zbyt dobrze zrozumiany.

\begin{proposition}
    Wśród orientowalnych 3-rozmaitości o~czterech szpicach najmniejszą objętość posiada dopełnienie splotu $8_4^2$ wg numeracji Rolfsena (L8a13).
\end{proposition}

\begin{proof}
    Rozumowanie Yoshidy \cite{yoshida13} oparte o pracę Agola.
    Objętość splotu wynosi $2v_8$.
\end{proof}

\index{objętość|)}



\index{węzeł!hiperboliczny|)}

% Koniec sekcji Węzły hiperboliczne



\section{Węzły plastrowe i taśmowe}
\label{sec:slice}
Węzły plastrowe i taśmowe oraz pojęcie kobordyzmu należą do świata 4-wymiarowej teorii węzłów.
Sami nie rozumiemy go zbyt dobrze, dlatego zreferujemy tylko tekst Kawauchiego \cite[s. 154-169]{kawauchi96} i nie podamy ulubionego odniesienia do tematu w~literaturze.

% DICTIONARY;plastrowy;slice;węzeł
\begin{definition}[węzeł plastrowy]
\index{węzeł!plastrowy}%
    Węzeł $K$ w sferze $S^3$, który jest brzegiem lokalnie płaskiego, właściwego dysku $D$ w kuli $B^4$ nazywamy węzłem plastrowym. % Kawauchi 155
    % % z \cite{gompf86}
    % Niech $K$ będzie takim węzłem w $S^3 = \partial B^4$, który ogranicza gładko zanurzony 2-dysk w $B^4$.
    % O węźle $K$ mówimy wtedy, że jest plastrowy.
    % stare:
    % Niech $K \subseteq S^3$ będzie takim węzłem, że w kuli $B^4$ istnieje płaski dysk $D$ taki, że $K = \partial D = D \cap S^3$.
    % Wtedy $K$ nazywamy węzłem plastrowym.

    Dysk $D$, kiedy potrzebuje mieć nazwę, też jest dyskiem plastrowym.
\end{definition}

Następujące węzły o~mniej niż jedenastu skrzyżowaniach są plastrowe: $6_1$, $8_{8}$, $8_{9}$, $8_{20}$, $9_{27}$, $9_{41}$, $9_{46}$, $10_{3}$, $10_{22}$, $10_{35}$, $10_{42}$, $10_{48}$, $10_{75}$, $10_{87}$, $10_{99}$, $10_{123}$, $10_{129}$, $10_{137}$, $10_{140}$, $10_{153}$ oraz $10_{155}$.
\index{węzeł!Conwaya}
Wśród pierwszych węzłów do dwunastu skrzyżowań najdłużej opierał się węzeł Conwaya, aż Lisa Piccirillo \cite{piccirillo20} pokazała, że nie jest plastrowy.
\index[persons]{Piccirillo, Lisa}%

Wiele wyników, jakie podamy w tej sekcji, pochodzi z artykułu Foxa, Milnora \cite{fox66}, od którego wszystko się zaczęło.
\index[persons]{Fox, Ralph}%
\index[persons]{Milnor, John}%

\begin{proposition}
    Niech $K$ będzie węzłem.
    Wtedy $K \shrap mr K$ jest węzłem plastrowym.
\end{proposition}

\begin{proof}
    Kawauchi \cite[s. 155]{kawauchi96} pisze: wybierzmy 3-kulę $B \subseteq S^3$ taką, że $K \cap B$ jest trywialnym łukiem w $B$.
    Wtedy $(\operatorname{cl} (S^3 \setminus B), \operatorname{cl} (K \setminus K \cap B)) \times [0,1]$ jest parą złożoną z 4-kuli i lokalnie płaskiego dysku, który jest świadkiem plastrowości węzła $K \shrap mr K$.
\end{proof}

\begin{proposition}
    Albo wszystkie trzy węzły $K_1, K_2, K_1 \shrap K_2$ są plastrowe, albo co najwyżej jeden z~nich.
\end{proposition}

\begin{proof}
    Kawauchi \cite[s. 155]{kawauchi96} pisze: załóżmy, że $K_1, K_2$ są plastrowe, z plastrowymi dyskami $D_1, D_2 \subseteq B^4$.
    Wtedy suma brzegowa\footnote{Cokolwiek to jest...} $(B^4, D_1) \natural (B^4, D_2)$ pokazuje, że węzeł $K_1 \shrap K_2$ jest plastrowy.

    Załóżmy teraz, że $K_1, K_3 = K_1 \shrap K_2$ są plastrowe, z plastrowymi dyskami $D_1, D_3 \subseteq B_4$.
    Wybierzmy 3-kule $B_1, B_3$ wewnątrz $S^3$ tak, że domknięcie $K_1 \setminus B_1 \cap K_1$ jest trywialnym łukiem w domknięciu $S^3 \setminus B_1$.
    Wtedy coś tam dalej, ale nie warto tego przepisywać, bo i tak za trudne na tę książkę.
\end{proof}

Pierwszym poważnym wynikiem z dziedziny teorii węzłów plastrowych, pochodzącym jeszcze z pracy \cite{fox66}, był:

\begin{proposition}[warunek Foxa-Milnora]
\index{warunek!Foxa-Milnora}%
    Niech $K$ będzie węzłem plastrowym.
    Wtedy jego wielomian Alexandera jest postaci $\alexander(t) = f(t) f(1/t)$ dla pewnego wielomianu Laurenta $f \in \Z[t, 1/t]$.
\end{proposition}

\begin{corollary}
    \index{wyznacznik}
    Wyznacznik węzła plastrowego jest kwadratem.
\end{corollary}

\begin{proof}
    Mamy $\det K = |\alexander_K(-1)| = f(-1) f(-1)$.
\end{proof}

Ten prosty test stwierdza, że 2743 spośród 2977 węzłów o mniej niż 13 skrzyżowaniach nie jest plastrowych.
% TODO: podać program tłumaczący, czemu tak jest?

\begin{proposition}
% TODO: skąd to stwierdzenie?
\index{sygnatura}%
    Niech $K$ będzie węzłem plastrowym.
    Wtedy $\sigma(K) = 0$.
\end{proposition}

\begin{tobedone}[Szkic dowodu]
    Ustalmy odwzorowanie $f$, które jest nieosobliwe, symetryczne i~dwuliniowe, z~przestrzeni $V$ o~wymiarze $2n$ oraz wyznaczoną przez nie formę kwadratową.
    Jeśli znika ona na podprzestrzeni wymiaru $n$, to ma zerową sygnaturę.
    dowód znaleziony w~podręczniku Lickorisha.
    Patrz też twierdzenie 8.8 z~artykułu \cite{murasugi65}.
    Praca "Infinite Order Amphicheiral Knots". (Charles Livingston, 2001) -- chyba nie?
\end{tobedone}

Test ten eliminuje kolejne 45 węzłów poniżej 13 skrzyżowań.
% TODO: podać program tłumaczący, czemu tak jest?

\begin{proposition}
    \index{niezmiennik!Arfa}
    Niech $K$ będzie węzłem plastrowym.
    Wtedy $\operatorname{Arf} K = 0$.
\end{proposition}

\begin{proof}
    Ustalmy węzeł $K$, wiemy już, że jego wyznacznik jest kwadratem, a na mocy faktu \ref{cor:knot_determinant_odd} także tyle, że jest liczbą nieparzystą.
    Wynika stąd przystawanie $\det K \equiv 1 \mod 8$, które w~połączeniu z warunkiem Murasugiego (fakt \ref{prp:arf_murasugi}) daje $\operatorname{Arf} K = 0$.
\end{proof}

% TODO: podać program tłumaczący, czemu tak jest? tzn. czy coś to eliminuje więcej, jak tak to ile?

Ostatni fakt, jaki podamy we wprowadzeniu, to jedno z niewielu miejsc w całej książce, gdzie dotykamy różnic między kategorią TOP oraz PL.

\begin{proposition}
\label{prp:trivial_alexander_implies_slice}%
    Niech $K$ będzie węzłem w kategorii TOP.
    Jeżeli jego wielomian Alexandera jest trywialny: $\alexander_K(t) \equiv 1$, to węzeł $K$ jest plastrowy.
\end{proposition}

\begin{proof}
\index[persons]{Freedman, Michael}%
% Twierdzenie łatwo napotkać czytając o węzłach plastrowych, ale jawnie nikomu się nie chce wskazać dowodu. Informację, że to jest tw. 1.13b znalazłem wreszcie w https://arxiv.org/pdf/1504.01064.pdf
    Freedman w \cite[tw. 1.13]{freedman82}.
\end{proof}

Implikacja \ref{prp:trivial_alexander_implies_slice} przestaje być prawdziwa po przejściu z kategorii TOP do DIFF.
% Encyclopedia of Knot Theory pod redakcją Colin Adams, Erica Flapan, Allison Henrich, Louis H. Kauffman, Lewis D. Ludwig, Sam Nelson, około strony 453 lub 454: "but many knots with Alexander polynomial one are not smoothly slice [9]"
Wydawało mi się kiedyś, że Gompf \cite{gompf86} dobrze tłumaczy tę różnicę przy użyciu twierdzenia Donaldsona.
\index[persons]{Gompf, Robert}%
Dzisiaj wiem, że nic nie wiem.

Encyklopedia węzłów\footnote{,,Encyclopedia of knot theory'', niedostępna jeszcze w MathSciNet.} \cite{citation_missing_on_purpose} odsyła do Donaldsona \cite{donaldson83} właśnie.
% https://www.routledge.com/Encyclopedia-of-Knot-Theory/Adams-Flapan-Henrich-Kauffman-Ludwig-Nelson/p/book/9781138297845


%%% Kawauchi 156:
\subsection{Zgodność}
Zgodność jest relacją równoważności na zbiorze węzłów, która prowadzi do nowej definicji węzłów plastrowych (patrz fakt~\ref{prp:concordant_iff_sum_slice}).
My przytaczamy jej definicję z pracy Gompfa \cite{gompf86}:

\begin{definition}[zgodność]
\index{zgodność}%
\index{węzeł!zgodny|see {zgodność}}%
    Dwa węzły $K_0, K_1$ nazywamy (gładko) zgodnymi, jeżeli istnieje gładko zanurzony pierścień w $S^3 \times I$, którego brzegiem jest zbiór $K_0 \times \{0\} \cup K_1 \times \{1\}$.
\end{definition}

Kawauchi \cite[s. 156]{kawauchi96} pisze ,,Two knots (…) are knot cobordant (or concordant)'', więc tak jak wielu innych autorów nie odróżnia więc węzłów kobordantnych od zgodnych.
Mamy zamiar zrobić dokładnie to samo: różnica między tymi terminami jest subtelna; węzły zgodne są też kobordantne, ale implikacja w drugą stronę nie zachodzi (wiemy o~tym z~tekstu Blanlœila ,,Cobordism and Concordance of Knots'') chyba, że pracuje się z węzłami sferycznmi, a tak jest w klasycznej teorii węzłów.
\index[persons]{Blanloeil, Vincent}%
% https://www.maths.ed.ac.uk/~v1ranick/papers/blanloeil
% Concordant knots are cobordant, but the converse is not true in general.
% "Cobordism and Concordance of Knots" by Vincent Blanlœil

Dlatego my będziemy zawsze pisać o węzłach zgodnych i nigdy o kobordantnynch.

\begin{proposition}
\label{prp:concordant_iff_sum_slice}%
    Dwa węzły $K_1, K_2$ są zgodne wtedy i tylko wtedy, gdy suma $(mr K_0) \shrap K_1$ jest plastrowa.
\end{proposition}

\begin{proof}
    Ćwiczenie 12.1.3 w książce Kawauchiego \cite{kawauchi96}.
\end{proof}

\begin{definition}
    Węzeł zgodny z~niewęzłem nazywamy plastrowym.
\end{definition}

,,Bycie zgodnym'' jest relacją równoważności, słabszą od ,,bycia izotopijnym'', ale chyba mocniejszą od ,,bycia homotopijnym''.
% ale mocniejszą od homotopii?
% izotopia: https://encyclopediaofmath.org/wiki/Cobordism_of_knots
% homotopia: https://en.wikipedia.org/wiki/Link_concordance By its nature, link concordance is an equivalence relation. It is weaker than isotopy, and stronger than homotopy: isotopy implies concordance implies homotopy. A link is a slice link if it is concordant to the unlink.
Klasę abstrakcji węzła $K$ oznaczamy przez $[K]$.

\begin{definition}[grupa zgodności]
\index{grupa!zgodności}%
    Niech $C^1$ oznacza iloraz zbioru wszystkich węzłów przez relację zgodności.
    Zbiór $C^1$ wyposażony w~działanie
    \begin{equation}
        [K_1] + [K_2] = [K_1 \shrap K_2]
    \end{equation}
    staje się grupą abelową, nazywaną grupą zgodności.
    Jej elementem neutralnym jest klasa abstrakcji niewęzła.
    Elementem przeciwnym do $[K]$ jest $[mr K]$.
\end{definition}

%%% Kawauchi 157:

Niech $\Theta$ oznacza rodzinę macierzy Seiferta węzłów (czyli kwadratowych macierzy $V$ o~całkowitych wyrazach takich, że $\det (V - V^T) = 1$).
Mówimy, że macierz $V \in \Theta$ jest zerowo kobordantna, jeżeli jest postaci
\begin{equation}
    V = P \begin{pmatrix} 0 & V_{21} \\ V_{12} & V_{22} \end{pmatrix} P^{-1}
\end{equation}
dla pewnej całkowitoliczbowej macierzy $P$ o~wyznaczniku $\pm 1$; takie macierze nazywamy unimodularnie sprzężonymi.
\index{macierz!unimodularnie sprzężona}%
Każda zerowo kobordantna macierz $V \in \Theta$ stanowi macierz Seiferta pewnego plastrowego węzła $K$.
Kawauchi nazywa te węzły algebraicznie plastrowymi i~mówi, że to dokładnie węzły, które ograniczają izotropowe powierzchnie w kuli $B^4$, więc każdy węzeł plastrowy jest algebraicznie plastrowy.

Suma $(-V) \oplus V$ jest zerowo kobordantna dla każdej macierzy $V \in \Theta$.
To (chyba to) inspiruje Kawauchiego do wprowadzenia kolejnej definicji: dwie macierze $V_1, V_2 \in \Theta$ nazywa kobordantnymi, jeżeli $(-V_1) \oplus V_2$ jest zerowo kobordantna.
Kobordyzm stanowi relację równoważności na $\Theta$ -- iloraz $\Theta$ przez tę relację oznacza się $G_-$, jest grupą abelową.

\begin{proposition}
    % Kawauchi 12.2.8
    Odwzorowanie $\psi \colon C^1 \to G_-$ posyłające klasę abstrakcji węzła w klasę abstrakcji jego macierzy Seiferta jest dobrze określonym epimorfizmem.
\end{proposition}

\begin{proof}
    Nie umiem nic sam udowodnić, więc wymienię tylko trzy odsyłacze: z faktu~\ref{prp:cobordant_to_algebraic_is_algebraic} wynika, że odwzorowanie $\psi$ jest dobrze określone, dowód faktu~\ref{prp:signature_additive} pokazuje, że $\psi$ jest homomorfizmem, zaś w \cite[s. 62]{kawauchi96} można przeczytać, dlaczego jest ,,na''.
\end{proof}

Funkcję $\psi$ rozpatrywał Levine \cite{levine69} w latach sześćdziesiątych.
\index[persons]{Levine, Jerome}%
Po mniej niż dekadzie Casson, Gordon \cite{gordon78} wskazali nietrywialne elementy jądra.
\index[persons]{Gordon, Cameron}%
\index[persons]{Casson, Andrew}%
% to wyżej wiem z kawauchi98, "Supplementary notes for Chapter 12"
Potem był wynik Jianga \cite{jiang81}, że jądro nie jest skończenie generowalne, bo zawiera izomorficzną kopię $\Z^\infty$, a~jeszcze później Livingstona \cite{livingston99}, że zawiera też kopię $(\Z/2\Z)^\infty$.
% to wyżej wiem z https://mathscinet.ams.org/mathscinet-getitem?mr=2179265, pierwsze strony tekstu (nie recenzji)

\begin{proposition}
    $G_- \cong \Z^\infty \oplus (\Z/4\Z)^\infty \oplus (\Z/2\Z)^\infty$.
\end{proposition}

Kawauchi \cite[s. 161]{kawauchi96} bez uzasadnienia postanawia nie przytoczyć dowodu tego faktu, ale opowiada krótko, jaka jest idea przewodnia i odsyła wprost do pracy Levine'a.
Na dalszych stronach jego pracy przeglądowej pojawiają się jakieś formy kwadratowe oraz uogólnienia wszystkiego do zgodności splotów, ale ja wracam nocnym pociągiem, więc nie mam siły o tym pisać.




\subsection{Węzły taśmowe}
\index{węzeł!taśmowy|(}%
\begin{definition}
    Węzeł $K = f(S^1)$ będący brzegiem osobliwego dysku $f \colon D \to S^3$ posiadającego następującą własność: każda przecinająca siebie składowa jest łukiem $A \subseteq f(D^2)$, dla którego $f^{-1}(A)$ składa się z~dwóch łuków w~$D^2$ (jeden z~nich jest wewnętrzny), nazywamy taśmowym.
\end{definition}

Jak pisze Kawauchi, mamy oczywiste wynikanie:

\begin{proposition}
\index{węzeł!plastrowy}%
    Każdy węzeł taśmowy jest plastrowy.
\end{proposition}

Dawno temu Fox \cite[problem 1.33]{kirby78} zapytał, czy implikacja odwrotna jest prawdziwa:
\index[persons]{Fox, Ralph}%

\begin{conjecture}[slice-ribbon problem]
    \index{hipoteza!plastrowo-taśmowa}
    Czy każdy węzeł plastrowy jest taśmowy?
\end{conjecture}

Wprawdzie Lisca pokazał prawdziwość hipotezy dla węzłów dwumostowych \cite{lisca07},
\index[persons]{Lisca, Paolo}%
% korzystając ze słynnego tw. Donaldsona: that a definite intersection form of a compact, oriented, simply connected, smooth manifold of dimension 4 is diagonalisable
\index{węzeł!dwumostowy}%
zaś Greene oraz Jabuka zrobili to dla precli o trzech pasmach w~\cite{greene11};
\index[persons]{Greene, Joshua}%
\index[persons]{Jabuka, Stanisław}%
\index{precel}%
ale Gompf, Scharlemann i~Thompson zasugerowali w~\cite{gompf10} potencjalny kontrprzykład.
\index[persons]{Gompf, Robert}%
\index[persons]{Scharlemann, Martin}%
\index[persons]{Thompson, Abigail}%
\index{rozmaitość szwowa}%
Nie możemy przytoczyć tego kontrprzykładu, gdyż korzysta z~rozmaitości szwowych, opisanych w~\cite[s. 53-59]{kawauchi96}.

Teichner myśli\footnote{Patrz \url{https://mathoverflow.net/a/18154}.} o hipotezie plastrowo-taśmowej jako o~życzeniu, które uprościłoby pewne czterowymiarowe problemy, gdyby było prawdziwe.
\index[persons]{Teichner, Peter}%

\index{węzeł!taśmowy|)}

% koniec podsekcji węzły taśmowe



\input{50-families/slice-algebraic}

\subsection{Węzły skręcone}
\index{węzeł!skręcony|(}%

% DICTIONARY;twist knot;węzeł skręcony
Węzły skręcone uważa się za najprostszą (po torusowych) rodzinę węzłów.

\begin{definition}
%label{def:twist_knot}
    Węzeł powstały przez $n$-krotne półskręcanie domkniętej pętli oraz splecienie końców nazywamy węzłem skręconym.
\end{definition}

Węzły skręcone to dokładnie towarzyszące niewęzłowi w~węzłach satelitarnych, tak zwane whiteheadowskie duble niewęzła.
Wszystkie są odwracalne (ale tylko niewęzeł oraz ósemka są amfichiralne) i~mają liczbę gordyjską $1$, ponieważ wystarczy rozwiązać skrzyżowanie, które plotło końce.
\index{liczba gordyjska}%
Każdy jest dwumostowy (ćwiczenie w \cite[s. 114]{rolfsen76}) i~posiada zerową sygnaturę.
\index{węzeł!dwumostowy}%
\index{sygnatura}%
Dalsze własności węzłów skręconych zależą od $n$, ilości półskrętów.
Indeks skrzyżowaniowy wynosi $n + 2$.

\begin{proposition}
\index{wielomian!Conwaya}%
    Niech $K$ będzie węzłem $n$-skręconym.
    Wtedy
    \begin{equation}
    2 \conway (z) = \begin{cases}
        2 + (n+1) z^{2} & n \mbox{ nieparzyste} \\
        2 - nz^2 & n \mbox{ parzyste}
    \end{cases}
    \end{equation}
\end{proposition}

\begin{proposition}
\index{wielomian!Jonesa}%
    Niech $K$ będzie węzłem $n$-skręconym.
    Wtedy
    \begin{equation}
    (q+1)\jones(q) = \begin{cases}
        1+q^{-2}+q^{-n}-q^{-n-3} & n \mbox{ nieparzyste} \\
        q^3(1+q^{-2}-q^{-n}+q^{-n-3}) & n \mbox{ parzyste}
    \end{cases}
    \end{equation}
\end{proposition}

\begin{proposition}
\index{węzeł!plastrowy}%
    Niewęzeł oraz węzeł dokerski $6_1$ są jedynymi skręconymi węzłami plastrowymi.
\end{proposition}

\begin{proof}
    \cite{casson86}.
\end{proof}

\index{węzeł!skręcony|)}%

% koniec podsekcji Węzły skręcone




\appendix
\chapter{Tablice węzłów pierwszych}

\section{Wartości niezmienników}

Ta sekcja zawiera dwie tabele, zacznijmy od opisu drugiej.
Przedstawia ona węzły pierwsze o~co najwyżej dwunastu skrzyżowaniach oraz wartości ich niezmienników (całkowitoliczbowych lub wielomianowych).
Zgodnie z~oznaczeniami przyjętymi w reszcie książki, $\unknotting, \braid, \bridge$ to kolejno liczba gordyjska, warkoczowa i~mostowa.
Zapis $2..3$ mówi, że dokładna wartość nie jest znana i leży w przedziale $[2,3]$.
Jeśli liczba mostowa wynosi dokładnie $2$, zamiast niej podajemy nieskracalny ułamek $p/q$, który koduje węzeł.
Dalej, $\det$ jest wyznacznikiem, $\sigma$ sygnaturą.
Wielomian Conwaya $\conway(z)$ dla oszczędności miejsca podajemy jako ciąg współczynników, na przykład $1-1$ jest skrótem od $1-z^2$.
Ostatnia kolumna mówi, czy węzeł alternuje.

Pierwsza tabela stanowi podsumowanie drugiej: mówi, ile różnych wartości przybiera dany niezmiennik wśród węzłów o~danej liczbie skrzyżowań.
Jak widać, wielomian Conwaya radzi sobie najlepiej, ale nie doskonale.
Dane pochodzą z~portalu KnotInfo \cite{knotinfo22} (założonego w 2004 roku przez Charlesa Livingstona, do którego dołączył wkrótce Jae Choon Cha. W 2019 roku w~rozwoju strony zaczęli brać udział jeszcze Allison Moore oraz Eric Ost), gdzie znaleźć można opisy wszystkich niezmienników z~tabeli oraz wiele więcej danych, gorąco zachęcamy do odwiedzin tej strony internetowej.
% TODO: index persons może



Tabela pierwsza, podsumująca:


\renewcommand*{\arraystretch}{1.4}
\footnotesize
\begin{longtable}{cccccccccc}
\hline
nazwa & u & $\braid$ & $\bridge$ & det & $\sigma$ & $\conway$ & symetria & alt. & all \\ \hline
\endhead % all the lines above this will be repeated on every page
3 & 1 & 1 & 1 & 1 & 1 & 1 & 1 & 1 & 1 \\
4 & 1 & 1 & 1 & 1 & 1 & 1 & 1 & 1 & 1 \\
5 & 2 & 2 & 1 & 2 & 2 & 2 & 1 & 1 & 2 \\
6 & 1 & 2 & 1 & 3 & 2 & 3 & 2 & 1 & 3 \\
7 & 3 & 3 & 1 & 7 & 4 & 7 & 1 & 1 & 7 \\
8 & 3 & 3 & 2 & 16 & 5 & 21 & 3 & 2 & 21 \\
9 & 4 & 4 & 2 & 29 & 7 & 48 & 2 & 2 & 49 \\
10 & 5 & 4 & 2 & 56 & 7 & 150 & 4 & 2 & 165 \\
11 & 10 & 5 & 3 & 100 & 9 & 419 & 2 & 2 & 552 \\
12 & 10 & 5 & 3 & 167 & 8 & 1513 & 5 & 2 & 2176 \\
\hline
\end{longtable}
\normalsize



\newpage
Tabela druga, dokładna:

\renewcommand*{\arraystretch}{1.4}
\footnotesize
\begin{longtable}{ccccccccc}
\hline
nazwa & u & $\braid$ & $\bridge$ & det & $\sigma$ & $\conway$ & symetria & alt. \\ \hline
\endhead % all the lines above this will be repeated on every page
$3_{1}$ & $1$ & $2$ & ${}^{3}{\mskip -5mu/\mskip -3mu}_{1}$ & $3$ & $-2$ & $1+1$ & odwracalny & tak \\
$4_{1}$ & $1$ & $3$ & ${}^{5}{\mskip -5mu/\mskip -3mu}_{2}$ & $5$ & $0$ & $1-1$ & całkowicie & tak \\
$5_{1}$ & $2$ & $2$ & ${}^{5}{\mskip -5mu/\mskip -3mu}_{1}$ & $5$ & $-4$ & $1+3+1$ & odwracalny & tak \\
$5_{2}$ & $1$ & $3$ & ${}^{7}{\mskip -5mu/\mskip -3mu}_{3}$ & $7$ & $-2$ & $1+2$ & odwracalny & tak \\
$6_{1}$ & $1$ & $4$ & ${}^{9}{\mskip -5mu/\mskip -3mu}_{7}$ & $9$ & $0$ & $1-2$ & odwracalny & tak \\
$6_{2}$ & $1$ & $3$ & ${}^{11}{\mskip -5mu/\mskip -3mu}_{4}$ & $11$ & $-2$ & $1-1-1$ & odwracalny & tak \\
$6_{3}$ & $1$ & $3$ & ${}^{13}{\mskip -5mu/\mskip -3mu}_{5}$ & $13$ & $0$ & $1+1+1$ & całkowicie & tak \\
$7_{1}$ & $3$ & $2$ & ${}^{7}{\mskip -5mu/\mskip -3mu}_{1}$ & $7$ & $-6$ & $1+6+5+1$ & odwracalny & tak \\
$7_{2}$ & $1$ & $4$ & ${}^{11}{\mskip -5mu/\mskip -3mu}_{5}$ & $11$ & $-2$ & $1+3$ & odwracalny & tak \\
$7_{3}$ & $2$ & $3$ & ${}^{13}{\mskip -5mu/\mskip -3mu}_{9}$ & $13$ & $-4$ & $1+5+2$ & odwracalny & tak \\
$7_{4}$ & $2$ & $4$ & ${}^{15}{\mskip -5mu/\mskip -3mu}_{11}$ & $15$ & $-2$ & $1+4$ & odwracalny & tak \\
$7_{5}$ & $2$ & $3$ & ${}^{17}{\mskip -5mu/\mskip -3mu}_{7}$ & $17$ & $-4$ & $1+4+2$ & odwracalny & tak \\
$7_{6}$ & $1$ & $4$ & ${}^{19}{\mskip -5mu/\mskip -3mu}_{7}$ & $19$ & $-2$ & $1+1-1$ & odwracalny & tak \\
$7_{7}$ & $1$ & $4$ & ${}^{21}{\mskip -5mu/\mskip -3mu}_{8}$ & $21$ & $0$ & $1-1+1$ & odwracalny & tak \\
$8_{1}$ & $1$ & $5$ & ${}^{13}{\mskip -5mu/\mskip -3mu}_{11}$ & $13$ & $0$ & $1-3$ & odwracalny & tak \\
$8_{2}$ & $2$ & $3$ & ${}^{17}{\mskip -5mu/\mskip -3mu}_{6}$ & $17$ & $-4$ & $1+0-3-1$ & odwracalny & tak \\
$8_{3}$ & $2$ & $5$ & ${}^{17}{\mskip -5mu/\mskip -3mu}_{4}$ & $17$ & $0$ & $1-4$ & całkowicie & tak \\
$8_{4}$ & $2$ & $4$ & ${}^{19}{\mskip -5mu/\mskip -3mu}_{14}$ & $19$ & $2$ & $1-3-2$ & odwracalny & tak \\
$8_{5}$ & $2$ & $3$ & $3$ & $21$ & $-4$ & $1-1-3-1$ & odwracalny & tak \\
$8_{6}$ & $2$ & $4$ & ${}^{23}{\mskip -5mu/\mskip -3mu}_{10}$ & $23$ & $-2$ & $1-2-2$ & odwracalny & tak \\
$8_{7}$ & $1$ & $3$ & ${}^{23}{\mskip -5mu/\mskip -3mu}_{9}$ & $23$ & $2$ & $1+2+3+1$ & odwracalny & tak \\
$8_{8}$ & $2$ & $4$ & ${}^{25}{\mskip -5mu/\mskip -3mu}_{9}$ & $25$ & $0$ & $1+2+2$ & odwracalny & tak \\
$8_{9}$ & $1$ & $3$ & ${}^{25}{\mskip -5mu/\mskip -3mu}_{7}$ & $25$ & $0$ & $1-2-3-1$ & całkowicie & tak \\
$8_{10}$ & $2$ & $3$ & $3$ & $27$ & $2$ & $1+3+3+1$ & odwracalny & tak \\
$8_{11}$ & $1$ & $4$ & ${}^{27}{\mskip -5mu/\mskip -3mu}_{10}$ & $27$ & $-2$ & $1-1-2$ & odwracalny & tak \\
$8_{12}$ & $2$ & $5$ & ${}^{29}{\mskip -5mu/\mskip -3mu}_{12}$ & $29$ & $0$ & $1-3+1$ & całkowicie & tak \\
$8_{13}$ & $1$ & $4$ & ${}^{29}{\mskip -5mu/\mskip -3mu}_{11}$ & $29$ & $0$ & $1+1+2$ & odwracalny & tak \\
$8_{14}$ & $1$ & $4$ & ${}^{31}{\mskip -5mu/\mskip -3mu}_{12}$ & $31$ & $-2$ & $1+0-2$ & odwracalny & tak \\
$8_{15}$ & $2$ & $4$ & $3$ & $33$ & $-4$ & $1+4+3$ & odwracalny & tak \\
$8_{16}$ & $2$ & $3$ & $3$ & $35$ & $2$ & $1+1+2+1$ & odwracalny & tak \\
$8_{17}$ & $1$ & $3$ & $3$ & $37$ & $0$ & $1-1-2-1$ & -zwierciadlany & tak \\
$8_{18}$ & $2$ & $3$ & $3$ & $45$ & $0$ & $1+1-1-1$ & całkowicie & tak \\
$8_{19}$ & $3$ & $3$ & $3$ & $3$ & $-6$ & $1+5+5+1$ & odwracalny & nie \\
$8_{20}$ & $1$ & $3$ & $3$ & $9$ & $0$ & $1+2+1$ & odwracalny & nie \\
$8_{21}$ & $1$ & $3$ & $3$ & $15$ & $-2$ & $1+0-1$ & odwracalny & nie \\
$9_{1}$ & $4$ & $2$ & ${}^{9}{\mskip -5mu/\mskip -3mu}_{1}$ & $9$ & $-8$ & $1+10+15+7+1$ & odwracalny & tak \\
$9_{2}$ & $1$ & $5$ & ${}^{15}{\mskip -5mu/\mskip -3mu}_{7}$ & $15$ & $-2$ & $1+4$ & odwracalny & tak \\
$9_{3}$ & $3$ & $3$ & ${}^{19}{\mskip -5mu/\mskip -3mu}_{13}$ & $19$ & $-6$ & $1+9+9+2$ & odwracalny & tak \\
$9_{4}$ & $2$ & $4$ & ${}^{21}{\mskip -5mu/\mskip -3mu}_{5}$ & $21$ & $-4$ & $1+7+3$ & odwracalny & tak \\
$9_{5}$ & $2$ & $5$ & ${}^{23}{\mskip -5mu/\mskip -3mu}_{17}$ & $23$ & $-2$ & $1+6$ & odwracalny & tak \\
$9_{6}$ & $3$ & $3$ & ${}^{27}{\mskip -5mu/\mskip -3mu}_{5}$ & $27$ & $-6$ & $1+7+8+2$ & odwracalny & tak \\
$9_{7}$ & $2$ & $4$ & ${}^{29}{\mskip -5mu/\mskip -3mu}_{13}$ & $29$ & $-4$ & $1+5+3$ & odwracalny & tak \\
$9_{8}$ & $2$ & $5$ & ${}^{31}{\mskip -5mu/\mskip -3mu}_{11}$ & $31$ & $-2$ & $1+0-2$ & odwracalny & tak \\
$9_{9}$ & $3$ & $3$ & ${}^{31}{\mskip -5mu/\mskip -3mu}_{9}$ & $31$ & $-6$ & $1+8+8+2$ & odwracalny & tak \\
$9_{10}$ & $3$ & $4$ & ${}^{33}{\mskip -5mu/\mskip -3mu}_{23}$ & $33$ & $-4$ & $1+8+4$ & odwracalny & tak \\
$9_{11}$ & $2$ & $4$ & ${}^{33}{\mskip -5mu/\mskip -3mu}_{14}$ & $33$ & $4$ & $1+4-1-1$ & odwracalny & tak \\
$9_{12}$ & $1$ & $5$ & ${}^{35}{\mskip -5mu/\mskip -3mu}_{13}$ & $35$ & $-2$ & $1+1-2$ & odwracalny & tak \\
$9_{13}$ & $3$ & $4$ & ${}^{37}{\mskip -5mu/\mskip -3mu}_{27}$ & $37$ & $-4$ & $1+7+4$ & odwracalny & tak \\
$9_{14}$ & $1$ & $5$ & ${}^{37}{\mskip -5mu/\mskip -3mu}_{14}$ & $37$ & $0$ & $1-1+2$ & odwracalny & tak \\
$9_{15}$ & $2$ & $5$ & ${}^{39}{\mskip -5mu/\mskip -3mu}_{16}$ & $39$ & $2$ & $1+2-2$ & odwracalny & tak \\
$9_{16}$ & $3$ & $3$ & $3$ & $39$ & $-6$ & $1+6+7+2$ & odwracalny & tak \\
$9_{17}$ & $2$ & $4$ & ${}^{39}{\mskip -5mu/\mskip -3mu}_{14}$ & $39$ & $-2$ & $1-2+1+1$ & odwracalny & tak \\
$9_{18}$ & $2$ & $4$ & ${}^{41}{\mskip -5mu/\mskip -3mu}_{17}$ & $41$ & $-4$ & $1+6+4$ & odwracalny & tak \\
$9_{19}$ & $1$ & $5$ & ${}^{41}{\mskip -5mu/\mskip -3mu}_{16}$ & $41$ & $0$ & $1-2+2$ & odwracalny & tak \\
$9_{20}$ & $2$ & $4$ & ${}^{41}{\mskip -5mu/\mskip -3mu}_{15}$ & $41$ & $-4$ & $1+2-1-1$ & odwracalny & tak \\
$9_{21}$ & $1$ & $5$ & ${}^{43}{\mskip -5mu/\mskip -3mu}_{18}$ & $43$ & $2$ & $1+3-2$ & odwracalny & tak \\
$9_{22}$ & $1$ & $4$ & $3$ & $43$ & $-2$ & $1-1+1+1$ & odwracalny & tak \\
$9_{23}$ & $2$ & $4$ & ${}^{45}{\mskip -5mu/\mskip -3mu}_{19}$ & $45$ & $-4$ & $1+5+4$ & odwracalny & tak \\
$9_{24}$ & $1$ & $4$ & $3$ & $45$ & $0$ & $1+1-1-1$ & odwracalny & tak \\
$9_{25}$ & $2$ & $5$ & $3$ & $47$ & $-2$ & $1+0-3$ & odwracalny & tak \\
$9_{26}$ & $1$ & $4$ & ${}^{47}{\mskip -5mu/\mskip -3mu}_{18}$ & $47$ & $2$ & $1+0+1+1$ & odwracalny & tak \\
$9_{27}$ & $1$ & $4$ & ${}^{49}{\mskip -5mu/\mskip -3mu}_{19}$ & $49$ & $0$ & $1+0-1-1$ & odwracalny & tak \\
$9_{28}$ & $1$ & $4$ & $3$ & $51$ & $-2$ & $1+1+1+1$ & odwracalny & tak \\
$9_{29}$ & $2$ & $4$ & $3$ & $51$ & $2$ & $1+1+1+1$ & odwracalny & tak \\
$9_{30}$ & $1$ & $4$ & $3$ & $53$ & $0$ & $1-1-1-1$ & odwracalny & tak \\
$9_{31}$ & $2$ & $4$ & ${}^{55}{\mskip -5mu/\mskip -3mu}_{21}$ & $55$ & $-2$ & $1+2+1+1$ & odwracalny & tak \\
$9_{32}$ & $2$ & $4$ & $3$ & $59$ & $2$ & $1-1+0+1$ & chiralny & tak \\
$9_{33}$ & $1$ & $4$ & $3$ & $61$ & $0$ & $1+1+0-1$ & chiralny & tak \\
$9_{34}$ & $1$ & $4$ & $3$ & $69$ & $0$ & $1-1+0-1$ & odwracalny & tak \\
$9_{35}$ & $3$ & $5$ & $3$ & $27$ & $-2$ & $1+7$ & odwracalny & tak \\
$9_{36}$ & $2$ & $4$ & $3$ & $37$ & $4$ & $1+3-1-1$ & odwracalny & tak \\
$9_{37}$ & $2$ & $5$ & $3$ & $45$ & $0$ & $1-3+2$ & odwracalny & tak \\
$9_{38}$ & $3$ & $4$ & $3$ & $57$ & $-4$ & $1+6+5$ & odwracalny & tak \\
$9_{39}$ & $1$ & $5$ & $3$ & $55$ & $2$ & $1+2-3$ & odwracalny & tak \\
$9_{40}$ & $2$ & $4$ & $3$ & $75$ & $-2$ & $1-1-1+1$ & odwracalny & tak \\
$9_{41}$ & $2$ & $5$ & $3$ & $49$ & $0$ & $1+0+3$ & odwracalny & tak \\
$9_{42}$ & $1$ & $4$ & $3$ & $7$ & $2$ & $1-2-1$ & odwracalny & nie \\
$9_{43}$ & $2$ & $4$ & $3$ & $13$ & $-4$ & $1+1-3-1$ & odwracalny & nie \\
$9_{44}$ & $1$ & $4$ & $3$ & $17$ & $0$ & $1+0+1$ & odwracalny & nie \\
$9_{45}$ & $1$ & $4$ & $3$ & $23$ & $2$ & $1+2-1$ & odwracalny & nie \\
$9_{46}$ & $2$ & $4$ & $3$ & $9$ & $0$ & $1-2$ & odwracalny & nie \\
$9_{47}$ & $2$ & $4$ & $3$ & $27$ & $-2$ & $1-1+2+1$ & odwracalny & nie \\
$9_{48}$ & $2$ & $4$ & $3$ & $27$ & $2$ & $1+3-1$ & odwracalny & nie \\
$9_{49}$ & $3$ & $4$ & $3$ & $25$ & $-4$ & $1+6+3$ & odwracalny & nie \\
$10_{1}$ & $1$ & $6$ & ${}^{17}{\mskip -5mu/\mskip -3mu}_{15}$ & $17$ & $0$ & $1-4$ & odwracalny & tak \\
$10_{2}$ & $3$ & $3$ & ${}^{23}{\mskip -5mu/\mskip -3mu}_{8}$ & $23$ & $-6$ & $1+2-5-5-1$ & odwracalny & tak \\
$10_{3}$ & $2$ & $6$ & ${}^{25}{\mskip -5mu/\mskip -3mu}_{6}$ & $25$ & $0$ & $1-6$ & odwracalny & tak \\
$10_{4}$ & $2$ & $5$ & ${}^{27}{\mskip -5mu/\mskip -3mu}_{20}$ & $27$ & $2$ & $1-5-3$ & odwracalny & tak \\
$10_{5}$ & $2$ & $3$ & ${}^{33}{\mskip -5mu/\mskip -3mu}_{13}$ & $33$ & $4$ & $1+4+7+5+1$ & odwracalny & tak \\
$10_{6}$ & $3$ & $4$ & ${}^{37}{\mskip -5mu/\mskip -3mu}_{16}$ & $37$ & $-4$ & $1-1-6-2$ & odwracalny & tak \\
$10_{7}$ & $1$ & $5$ & ${}^{43}{\mskip -5mu/\mskip -3mu}_{16}$ & $43$ & $-2$ & $1-1-3$ & odwracalny & tak \\
$10_{8}$ & $2$ & $4$ & ${}^{29}{\mskip -5mu/\mskip -3mu}_{6}$ & $29$ & $-4$ & $1-3-7-2$ & odwracalny & tak \\
$10_{9}$ & $1$ & $3$ & ${}^{39}{\mskip -5mu/\mskip -3mu}_{28}$ & $39$ & $-2$ & $1-2-7-5-1$ & odwracalny & tak \\
$10_{10}$ & $1$ & $5$ & ${}^{45}{\mskip -5mu/\mskip -3mu}_{17}$ & $45$ & $0$ & $1+1+3$ & odwracalny & tak \\
$10_{11}$ & $2..3$ & $5$ & ${}^{43}{\mskip -5mu/\mskip -3mu}_{13}$ & $43$ & $-2$ & $1-5-4$ & odwracalny & tak \\
$10_{12}$ & $2$ & $4$ & ${}^{47}{\mskip -5mu/\mskip -3mu}_{17}$ & $47$ & $2$ & $1+4+6+2$ & odwracalny & tak \\
$10_{13}$ & $2$ & $6$ & ${}^{53}{\mskip -5mu/\mskip -3mu}_{22}$ & $53$ & $0$ & $1-5+2$ & odwracalny & tak \\
$10_{14}$ & $2$ & $4$ & ${}^{57}{\mskip -5mu/\mskip -3mu}_{22}$ & $57$ & $-4$ & $1+2-4-2$ & odwracalny & tak \\
$10_{15}$ & $2$ & $4$ & ${}^{43}{\mskip -5mu/\mskip -3mu}_{19}$ & $43$ & $2$ & $1+3+6+2$ & odwracalny & tak \\
$10_{16}$ & $2$ & $5$ & ${}^{47}{\mskip -5mu/\mskip -3mu}_{33}$ & $47$ & $-2$ & $1-4-4$ & odwracalny & tak \\
$10_{17}$ & $1$ & $3$ & ${}^{41}{\mskip -5mu/\mskip -3mu}_{9}$ & $41$ & $0$ & $1+2+7+5+1$ & całkowicie & tak \\
$10_{18}$ & $1$ & $5$ & ${}^{55}{\mskip -5mu/\mskip -3mu}_{23}$ & $55$ & $-2$ & $1-2-4$ & odwracalny & tak \\
$10_{19}$ & $2$ & $4$ & ${}^{51}{\mskip -5mu/\mskip -3mu}_{14}$ & $51$ & $-2$ & $1+1+5+2$ & odwracalny & tak \\
$10_{20}$ & $2$ & $5$ & ${}^{35}{\mskip -5mu/\mskip -3mu}_{16}$ & $35$ & $-2$ & $1-3-3$ & odwracalny & tak \\
$10_{21}$ & $2$ & $4$ & ${}^{45}{\mskip -5mu/\mskip -3mu}_{16}$ & $45$ & $-4$ & $1+1-5-2$ & odwracalny & tak \\
$10_{22}$ & $2$ & $4$ & ${}^{49}{\mskip -5mu/\mskip -3mu}_{36}$ & $49$ & $0$ & $1-4-6-2$ & odwracalny & tak \\
$10_{23}$ & $1$ & $4$ & ${}^{59}{\mskip -5mu/\mskip -3mu}_{23}$ & $59$ & $2$ & $1+3+5+2$ & odwracalny & tak \\
$10_{24}$ & $2$ & $5$ & ${}^{55}{\mskip -5mu/\mskip -3mu}_{24}$ & $55$ & $-2$ & $1-2-4$ & odwracalny & tak \\
$10_{25}$ & $2$ & $4$ & ${}^{65}{\mskip -5mu/\mskip -3mu}_{24}$ & $65$ & $-4$ & $1+0-4-2$ & odwracalny & tak \\
$10_{26}$ & $1$ & $4$ & ${}^{61}{\mskip -5mu/\mskip -3mu}_{44}$ & $61$ & $0$ & $1-3-5-2$ & odwracalny & tak \\
$10_{27}$ & $1$ & $4$ & ${}^{71}{\mskip -5mu/\mskip -3mu}_{27}$ & $71$ & $2$ & $1+2+4+2$ & odwracalny & tak \\
$10_{28}$ & $2$ & $5$ & ${}^{53}{\mskip -5mu/\mskip -3mu}_{19}$ & $53$ & $0$ & $1+3+4$ & odwracalny & tak \\
$10_{29}$ & $2$ & $5$ & ${}^{63}{\mskip -5mu/\mskip -3mu}_{26}$ & $63$ & $-2$ & $1-4-1+1$ & odwracalny & tak \\
$10_{30}$ & $1$ & $5$ & ${}^{67}{\mskip -5mu/\mskip -3mu}_{26}$ & $67$ & $-2$ & $1+1-4$ & odwracalny & tak \\
$10_{31}$ & $1$ & $5$ & ${}^{57}{\mskip -5mu/\mskip -3mu}_{25}$ & $57$ & $0$ & $1+2+4$ & odwracalny & tak \\
$10_{32}$ & $1$ & $4$ & ${}^{69}{\mskip -5mu/\mskip -3mu}_{29}$ & $69$ & $0$ & $1-1-4-2$ & odwracalny & tak \\
$10_{33}$ & $1$ & $5$ & ${}^{65}{\mskip -5mu/\mskip -3mu}_{18}$ & $65$ & $0$ & $1+0+4$ & całkowicie & tak \\
$10_{34}$ & $2$ & $5$ & ${}^{37}{\mskip -5mu/\mskip -3mu}_{13}$ & $37$ & $0$ & $1+3+3$ & odwracalny & tak \\
$10_{35}$ & $2$ & $6$ & ${}^{49}{\mskip -5mu/\mskip -3mu}_{20}$ & $49$ & $0$ & $1-4+2$ & odwracalny & tak \\
$10_{36}$ & $2$ & $5$ & ${}^{51}{\mskip -5mu/\mskip -3mu}_{20}$ & $51$ & $-2$ & $1+1-3$ & odwracalny & tak \\
$10_{37}$ & $2$ & $5$ & ${}^{53}{\mskip -5mu/\mskip -3mu}_{23}$ & $53$ & $0$ & $1+3+4$ & całkowicie & tak \\
$10_{38}$ & $2$ & $5$ & ${}^{59}{\mskip -5mu/\mskip -3mu}_{25}$ & $59$ & $-2$ & $1-1-4$ & odwracalny & tak \\
$10_{39}$ & $2$ & $4$ & ${}^{61}{\mskip -5mu/\mskip -3mu}_{22}$ & $61$ & $-4$ & $1+1-4-2$ & odwracalny & tak \\
$10_{40}$ & $2$ & $4$ & ${}^{75}{\mskip -5mu/\mskip -3mu}_{29}$ & $75$ & $2$ & $1+3+4+2$ & odwracalny & tak \\
$10_{41}$ & $2$ & $5$ & ${}^{71}{\mskip -5mu/\mskip -3mu}_{26}$ & $71$ & $-2$ & $1-2-1+1$ & odwracalny & tak \\
$10_{42}$ & $1$ & $5$ & ${}^{81}{\mskip -5mu/\mskip -3mu}_{31}$ & $81$ & $0$ & $1+0+1-1$ & odwracalny & tak \\
$10_{43}$ & $2$ & $5$ & ${}^{73}{\mskip -5mu/\mskip -3mu}_{27}$ & $73$ & $0$ & $1+2+1-1$ & całkowicie & tak \\
$10_{44}$ & $1$ & $5$ & ${}^{79}{\mskip -5mu/\mskip -3mu}_{30}$ & $79$ & $-2$ & $1+0-1+1$ & odwracalny & tak \\
$10_{45}$ & $2$ & $5$ & ${}^{89}{\mskip -5mu/\mskip -3mu}_{34}$ & $89$ & $0$ & $1-2+1-1$ & całkowicie & tak \\
$10_{46}$ & $3$ & $3$ & $3$ & $31$ & $-6$ & $1+0-6-5-1$ & odwracalny & tak \\
$10_{47}$ & $2..3$ & $3$ & $3$ & $41$ & $4$ & $1+6+8+5+1$ & odwracalny & tak \\
$10_{48}$ & $2$ & $3$ & $3$ & $49$ & $0$ & $1+4+8+5+1$ & odwracalny & tak \\
$10_{49}$ & $3$ & $4$ & $3$ & $59$ & $-6$ & $1+7+10+3$ & odwracalny & tak \\
$10_{50}$ & $2$ & $4$ & $3$ & $53$ & $-4$ & $1-1-5-2$ & odwracalny & tak \\
$10_{51}$ & $2..3$ & $4$ & $3$ & $67$ & $2$ & $1+5+5+2$ & odwracalny & tak \\
$10_{52}$ & $2$ & $4$ & $3$ & $59$ & $-2$ & $1+3+5+2$ & odwracalny & tak \\
$10_{53}$ & $3$ & $5$ & $3$ & $73$ & $-4$ & $1+6+6$ & odwracalny & tak \\
$10_{54}$ & $2..3$ & $4$ & $3$ & $47$ & $2$ & $1+4+6+2$ & odwracalny & tak \\
$10_{55}$ & $2$ & $5$ & $3$ & $61$ & $-4$ & $1+5+5$ & odwracalny & tak \\
$10_{56}$ & $2$ & $4$ & $3$ & $65$ & $-4$ & $1+0-4-2$ & odwracalny & tak \\
$10_{57}$ & $2$ & $4$ & $3$ & $79$ & $2$ & $1+4+4+2$ & odwracalny & tak \\
$10_{58}$ & $2$ & $6$ & $3$ & $65$ & $0$ & $1-4+3$ & odwracalny & tak \\
$10_{59}$ & $1$ & $5$ & $3$ & $75$ & $-2$ & $1-1-1+1$ & odwracalny & tak \\
$10_{60}$ & $1$ & $5$ & $3$ & $85$ & $0$ & $1-1+1-1$ & odwracalny & tak \\
$10_{61}$ & $2..3$ & $4$ & $3$ & $33$ & $-4$ & $1-4-7-2$ & odwracalny & tak \\
$10_{62}$ & $2$ & $3$ & $3$ & $45$ & $4$ & $1+5+8+5+1$ & odwracalny & tak \\
$10_{63}$ & $2$ & $5$ & $3$ & $57$ & $-4$ & $1+6+5$ & odwracalny & tak \\
$10_{64}$ & $2$ & $3$ & $3$ & $51$ & $-2$ & $1-3-8-5-1$ & odwracalny & tak \\
$10_{65}$ & $2$ & $4$ & $3$ & $63$ & $2$ & $1+4+5+2$ & odwracalny & tak \\
$10_{66}$ & $3$ & $4$ & $3$ & $75$ & $-6$ & $1+7+9+3$ & odwracalny & tak \\
$10_{67}$ & $2$ & $5$ & $3$ & $63$ & $-2$ & $1+0-4$ & chiralny & tak \\
$10_{68}$ & $2$ & $5$ & $3$ & $57$ & $0$ & $1+2+4$ & odwracalny & tak \\
$10_{69}$ & $2$ & $5$ & $3$ & $87$ & $2$ & $1+2-1+1$ & odwracalny & tak \\
$10_{70}$ & $2$ & $5$ & $3$ & $67$ & $2$ & $1-3-1+1$ & odwracalny & tak \\
$10_{71}$ & $1$ & $5$ & $3$ & $77$ & $0$ & $1+1+1-1$ & odwracalny & tak \\
$10_{72}$ & $2$ & $4$ & $3$ & $73$ & $-4$ & $1+2-3-2$ & odwracalny & tak \\
$10_{73}$ & $1$ & $5$ & $3$ & $83$ & $2$ & $1+1-1+1$ & odwracalny & tak \\
$10_{74}$ & $2$ & $5$ & $3$ & $63$ & $-2$ & $1+0-4$ & odwracalny & tak \\
$10_{75}$ & $2$ & $5$ & $3$ & $81$ & $0$ & $1+0+1-1$ & odwracalny & tak \\
$10_{76}$ & $2..3$ & $4$ & $3$ & $57$ & $-4$ & $1-2-5-2$ & odwracalny & tak \\
$10_{77}$ & $2..3$ & $4$ & $3$ & $63$ & $2$ & $1+4+5+2$ & odwracalny & tak \\
$10_{78}$ & $2$ & $5$ & $3$ & $69$ & $-4$ & $1+3+1-1$ & odwracalny & tak \\
$10_{79}$ & $2..3$ & $3$ & $3$ & $61$ & $0$ & $1+5+9+5+1$ & -zwierciadlany & tak \\
$10_{80}$ & $3$ & $4$ & $3$ & $71$ & $-6$ & $1+6+9+3$ & chiralny & tak \\
$10_{81}$ & $2$ & $5$ & $3$ & $85$ & $0$ & $1+3+2-1$ & -zwierciadlany & tak \\
$10_{82}$ & $1$ & $3$ & $3$ & $63$ & $-2$ & $1+0-4-4-1$ & chiralny & tak \\
$10_{83}$ & $2$ & $4$ & $3$ & $83$ & $2$ & $1+1+3+2$ & chiralny & tak \\
$10_{84}$ & $1$ & $4$ & $3$ & $87$ & $-2$ & $1+2+3+2$ & chiralny & tak \\
$10_{85}$ & $2$ & $3$ & $3$ & $57$ & $4$ & $1+2+4+4+1$ & chiralny & tak \\
$10_{86}$ & $2$ & $4$ & $3$ & $85$ & $0$ & $1-1-3-2$ & chiralny & tak \\
$10_{87}$ & $2$ & $4$ & $3$ & $81$ & $0$ & $1+0-3-2$ & chiralny & tak \\
$10_{88}$ & $1$ & $5$ & $3$ & $101$ & $0$ & $1-1+2-1$ & -zwierciadlany & tak \\
$10_{89}$ & $2$ & $5$ & $3$ & $99$ & $2$ & $1+1-2+1$ & odwracalny & tak \\
$10_{90}$ & $2$ & $4$ & $3$ & $77$ & $0$ & $1-3-4-2$ & chiralny & tak \\
$10_{91}$ & $1$ & $3$ & $3$ & $73$ & $0$ & $1+2+5+4+1$ & chiralny & tak \\
$10_{92}$ & $2$ & $4$ & $3$ & $89$ & $-4$ & $1+2-2-2$ & chiralny & tak \\
$10_{93}$ & $2$ & $4$ & $3$ & $67$ & $2$ & $1+1+4+2$ & chiralny & tak \\
$10_{94}$ & $2$ & $3$ & $3$ & $71$ & $-2$ & $1-2-5-4-1$ & chiralny & tak \\
$10_{95}$ & $1$ & $4$ & $3$ & $91$ & $2$ & $1+3+3+2$ & chiralny & tak \\
$10_{96}$ & $2$ & $5$ & $3$ & $93$ & $0$ & $1-3+1-1$ & odwracalny & tak \\
$10_{97}$ & $2$ & $5$ & $3$ & $87$ & $-2$ & $1+2-5$ & odwracalny & tak \\
$10_{98}$ & $2$ & $4$ & $3$ & $81$ & $-4$ & $1+0-3-2$ & chiralny & tak \\
$10_{99}$ & $2$ & $3$ & $3$ & $81$ & $0$ & $1+4+6+4+1$ & całkowicie & tak \\
$10_{100}$ & $2..3$ & $3$ & $3$ & $65$ & $4$ & $1+4+5+4+1$ & odwracalny & tak \\
$10_{101}$ & $3$ & $5$ & $3$ & $85$ & $-4$ & $1+7+7$ & odwracalny & tak \\
$10_{102}$ & $1$ & $4$ & $3$ & $73$ & $0$ & $1-2-4-2$ & chiralny & tak \\
$10_{103}$ & $3$ & $4$ & $3$ & $75$ & $2$ & $1+3+4+2$ & odwracalny & tak \\
$10_{104}$ & $1$ & $3$ & $3$ & $77$ & $0$ & $1+1+5+4+1$ & odwracalny & tak \\
$10_{105}$ & $2$ & $5$ & $3$ & $91$ & $-2$ & $1-1-2+1$ & odwracalny & tak \\
$10_{106}$ & $2$ & $3$ & $3$ & $75$ & $-2$ & $1-1-5-4-1$ & chiralny & tak \\
$10_{107}$ & $1$ & $5$ & $3$ & $93$ & $0$ & $1+1+2-1$ & chiralny & tak \\
$10_{108}$ & $2$ & $4$ & $3$ & $63$ & $-2$ & $1+0+4+2$ & odwracalny & tak \\
$10_{109}$ & $2$ & $3$ & $3$ & $85$ & $0$ & $1+3+6+4+1$ & -zwierciadlany & tak \\
$10_{110}$ & $2$ & $5$ & $3$ & $83$ & $-2$ & $1-3-2+1$ & chiralny & tak \\
$10_{111}$ & $2$ & $4$ & $3$ & $77$ & $-4$ & $1+1-3-2$ & odwracalny & tak \\
$10_{112}$ & $2$ & $3$ & $3$ & $87$ & $2$ & $1+2-1-3-1$ & odwracalny & tak \\
$10_{113}$ & $1$ & $4$ & $3$ & $111$ & $-2$ & $1+0+1+2$ & odwracalny & tak \\
$10_{114}$ & $1$ & $4$ & $3$ & $93$ & $0$ & $1+1-2-2$ & odwracalny & tak \\
$10_{115}$ & $2$ & $5$ & $3$ & $109$ & $0$ & $1+1+3-1$ & -zwierciadlany & tak \\
$10_{116}$ & $2$ & $3$ & $3$ & $95$ & $2$ & $1+0-2-3-1$ & odwracalny & tak \\
$10_{117}$ & $2$ & $4$ & $3$ & $103$ & $2$ & $1+2+2+2$ & chiralny & tak \\
$10_{118}$ & $1$ & $3$ & $3$ & $97$ & $0$ & $1+0+2+3+1$ & -zwierciadlany & tak \\
$10_{119}$ & $1$ & $4$ & $3$ & $101$ & $0$ & $1-1-2-2$ & chiralny & tak \\
$10_{120}$ & $3$ & $5$ & $3$ & $105$ & $-4$ & $1+6+8$ & odwracalny & tak \\
$10_{121}$ & $2$ & $4$ & $3$ & $115$ & $-2$ & $1+1+1+2$ & odwracalny & tak \\
$10_{122}$ & $2$ & $4$ & $3$ & $105$ & $0$ & $1+2-1-2$ & odwracalny & tak \\
$10_{123}$ & $2$ & $3$ & $3$ & $121$ & $0$ & $1-2-1+2+1$ & całkowicie & tak \\
$10_{124}$ & $4$ & $3$ & $3$ & $1$ & $-8$ & $1+8+14+7+1$ & odwracalny & nie \\
$10_{125}$ & $2$ & $3$ & $3$ & $11$ & $2$ & $1+3+4+1$ & odwracalny & nie \\
$10_{126}$ & $2$ & $3$ & $3$ & $19$ & $2$ & $1+5+4+1$ & odwracalny & nie \\
$10_{127}$ & $2$ & $3$ & $3$ & $29$ & $-4$ & $1+1-2-1$ & odwracalny & nie \\
$10_{128}$ & $3$ & $4$ & $3$ & $11$ & $-6$ & $1+7+9+2$ & odwracalny & nie \\
$10_{129}$ & $1$ & $4$ & $3$ & $25$ & $0$ & $1+2+2$ & odwracalny & nie \\
$10_{130}$ & $2$ & $4$ & $3$ & $17$ & $0$ & $1+4+2$ & odwracalny & nie \\
$10_{131}$ & $1$ & $4$ & $3$ & $31$ & $-2$ & $1+0-2$ & odwracalny & nie \\
$10_{132}$ & $1$ & $4$ & $3$ & $5$ & $0$ & $1+3+1$ & odwracalny & nie \\
$10_{133}$ & $1$ & $4$ & $3$ & $19$ & $-2$ & $1+1-1$ & odwracalny & nie \\
$10_{134}$ & $3$ & $4$ & $3$ & $23$ & $-6$ & $1+6+8+2$ & odwracalny & nie \\
$10_{135}$ & $2$ & $4$ & $3$ & $37$ & $0$ & $1+3+3$ & odwracalny & nie \\
$10_{136}$ & $1$ & $4$ & $3$ & $15$ & $2$ & $1+0-1$ & odwracalny & nie \\
$10_{137}$ & $1$ & $5$ & $3$ & $25$ & $0$ & $1-2+1$ & odwracalny & nie \\
$10_{138}$ & $2$ & $5$ & $3$ & $35$ & $-2$ & $1-3+1+1$ & odwracalny & nie \\
$10_{139}$ & $4$ & $3$ & $3$ & $3$ & $-6$ & $1+9+14+7+1$ & odwracalny & nie \\
$10_{140}$ & $2$ & $4$ & $3$ & $9$ & $0$ & $1+2+1$ & odwracalny & nie \\
$10_{141}$ & $1$ & $3$ & $3$ & $21$ & $0$ & $1-1-3-1$ & odwracalny & nie \\
$10_{142}$ & $3$ & $4$ & $3$ & $15$ & $-6$ & $1+8+9+2$ & odwracalny & nie \\
$10_{143}$ & $1$ & $3$ & $3$ & $27$ & $2$ & $1+3+3+1$ & odwracalny & nie \\
$10_{144}$ & $2$ & $4$ & $3$ & $39$ & $-2$ & $1-2-3$ & odwracalny & nie \\
$10_{145}$ & $2$ & $4$ & $3$ & $3$ & $2$ & $1+5+1$ & odwracalny & nie \\
$10_{146}$ & $1$ & $4$ & $3$ & $33$ & $0$ & $1+0+2$ & odwracalny & nie \\
$10_{147}$ & $1$ & $4$ & $3$ & $27$ & $-2$ & $1-1-2$ & chiralny & nie \\
$10_{148}$ & $2$ & $3$ & $3$ & $31$ & $2$ & $1+4+3+1$ & chiralny & nie \\
$10_{149}$ & $2$ & $3$ & $3$ & $41$ & $-4$ & $1+2-1-1$ & chiralny & nie \\
$10_{150}$ & $2$ & $4$ & $3$ & $29$ & $-4$ & $1+1-2-1$ & chiralny & nie \\
$10_{151}$ & $2$ & $4$ & $3$ & $43$ & $2$ & $1+3+2+1$ & chiralny & nie \\
$10_{152}$ & $4$ & $3$ & $3$ & $11$ & $-6$ & $1+7+13+7+1$ & odwracalny & nie \\
$10_{153}$ & $2$ & $4$ & $3$ & $1$ & $0$ & $1+4+5+1$ & chiralny & nie \\
$10_{154}$ & $3$ & $4$ & $3$ & $13$ & $-4$ & $1+5+6+1$ & odwracalny & nie \\
$10_{155}$ & $2$ & $3$ & $3$ & $25$ & $0$ & $1-2-3-1$ & odwracalny & nie \\
$10_{156}$ & $1$ & $4$ & $3$ & $35$ & $2$ & $1+1+2+1$ & odwracalny & nie \\
$10_{157}$ & $2$ & $3$ & $3$ & $49$ & $4$ & $1+4+0-1$ & odwracalny & nie \\
$10_{158}$ & $2$ & $4$ & $3$ & $45$ & $0$ & $1-3-2-1$ & odwracalny & nie \\
$10_{159}$ & $1$ & $3$ & $3$ & $39$ & $-2$ & $1+2+2+1$ & odwracalny & nie \\
$10_{160}$ & $2$ & $4$ & $3$ & $21$ & $-4$ & $1+3-2-1$ & odwracalny & nie \\
$10_{161}$ & $3$ & $3$ & $3$ & $5$ & $-4$ & $1+7+6+1$ & odwracalny & nie \\
$10_{162}$ & $2$ & $4$ & $3$ & $35$ & $2$ & $1-3-3$ & odwracalny & nie \\
$10_{163}$ & $2$ & $4$ & $3$ & $51$ & $-2$ & $1+1+1+1$ & odwracalny & nie \\
$10_{164}$ & $1$ & $4$ & $3$ & $45$ & $0$ & $1+1+3$ & odwracalny & nie \\
$10_{165}$ & $2$ & $4$ & $3$ & $39$ & $2$ & $1+2-2$ & odwracalny & nie \\
$11a_{1}$ & $2$ & $5$ & $3$ & $127$ & $-2$ & $1+0+0+2$ & odwracalny & tak \\
$11a_{2}$ & $2$ & $5$ & $3$ & $137$ & $-4$ & $1+2-3-3$ & chiralny & tak \\
$11a_{3}$ & $2$ & $4$ & $3$ & $115$ & $2$ & $1+1-3-3-1$ & chiralny & tak \\
$11a_{4}$ & $2$ & $5$ & $3$ & $97$ & $0$ & $1+0-2-2$ & odwracalny & tak \\
$11a_{5}$ & $2$ & $6$ & $3$ & $125$ & $0$ & $1-3+3-1$ & odwracalny & tak \\
$11a_{6}$ & $2$ & $5$ & $3$ & $135$ & $-2$ & $1-2-1+2$ & odwracalny & tak \\
$11a_{7}$ & $2$ & $4$ & $3$ & $95$ & $2$ & $1+0-2-3-1$ & odwracalny & tak \\
$11a_{8}$ & $1$ & $5$ & $3$ & $117$ & $0$ & $1-1-1-2$ & odwracalny & tak \\
$11a_{9}$ & $2$ & $4$ & $3$ & $65$ & $-4$ & $1+0+0+3+1$ & odwracalny & tak \\
$11a_{10}$ & $1$ & $5$ & $3$ & $107$ & $-2$ & $1-1+1+2$ & odwracalny & tak \\
$11a_{11}$ & $1$ & $5$ & $3$ & $113$ & $0$ & $1+0-1-2$ & odwracalny & tak \\
$11a_{12}$ & $1$ & $5$ & $3$ & $103$ & $-2$ & $1-2+1+2$ & odwracalny & tak \\
$11a_{13}$ & $2$ & $6$ & ${}^{61}{\mskip -5mu/\mskip -3mu}_{28}$ & $61$ & $0$ & $1-3+3$ & odwracalny & tak \\
$11a_{14}$ & $2..3$ & $4$ & $3$ & $133$ & $0$ & $1+3+5+3+1$ & odwracalny & tak \\
$11a_{15}$ & $2$ & $4$ & $3$ & $107$ & $2$ & $1-1-3-3-1$ & chiralny & tak \\
$11a_{16}$ & $2$ & $5$ & $3$ & $105$ & $0$ & $1-2-2-2$ & odwracalny & tak \\
$11a_{17}$ & $2$ & $6$ & $3$ & $123$ & $-2$ & $1-1-4+1$ & chiralny & tak \\
$11a_{18}$ & $2..3$ & $5$ & $3$ & $127$ & $-2$ & $1+4+5+3$ & chiralny & tak \\
$11a_{19}$ & $2$ & $4$ & $3$ & $155$ & $-2$ & $1-1-2-2-1$ & chiralny & tak \\
$11a_{20}$ & $2..3$ & $5$ & $3$ & $113$ & $-4$ & $1+0-5-3$ & chiralny & tak \\
$11a_{21}$ & $2$ & $6$ & $3$ & $75$ & $-2$ & $1-1-5$ & odwracalny & tak \\
$11a_{22}$ & $2$ & $4$ & $3$ & $101$ & $-4$ & $1+3+3+3+1$ & chiralny & tak \\
$11a_{23}$ & $2$ & $5$ & $3$ & $103$ & $-2$ & $1+2+2+2$ & odwracalny & tak \\
$11a_{24}$ & $2$ & $4$ & $3$ & $157$ & $0$ & $1+1+2+2+1$ & chiralny & tak \\
$11a_{25}$ & $2$ & $4$ & $3$ & $155$ & $-2$ & $1-1-2-2-1$ & chiralny & tak \\
$11a_{26}$ & $2$ & $4$ & $3$ & $157$ & $0$ & $1+1+2+2+1$ & chiralny & tak \\
$11a_{27}$ & $2$ & $5$ & $3$ & $143$ & $2$ & $1+0-1+2$ & chiralny & tak \\
$11a_{28}$ & $2$ & $4$ & $3$ & $121$ & $0$ & $1-2-1+2+1$ & chiralny & tak \\
$11a_{29}$ & $2$ & $5$ & $3$ & $115$ & $2$ & $1-3+0+2$ & chiralny & tak \\
$11a_{30}$ & $2$ & $5$ & $3$ & $149$ & $0$ & $1-1+1-2$ & chiralny & tak \\
$11a_{31}$ & $2$ & $5$ & $3$ & $125$ & $-4$ & $1+1-4-3$ & odwracalny & tak \\
$11a_{32}$ & $2$ & $5$ & $3$ & $139$ & $-2$ & $1+3+4+3$ & odwracalny & tak \\
$11a_{33}$ & $2$ & $4$ & $3$ & $95$ & $2$ & $1+0-2-3-1$ & odwracalny & tak \\
$11a_{34}$ & $1$ & $4$ & $3$ & $119$ & $-2$ & $1-2-4-3-1$ & odwracalny & tak \\
$11a_{35}$ & $1$ & $4$ & $3$ & $121$ & $0$ & $1+2+4+3+1$ & odwracalny & tak \\
$11a_{36}$ & $2$ & $5$ & $3$ & $121$ & $0$ & $1+2+0-2$ & odwracalny & tak \\
$11a_{37}$ & $2$ & $5$ & $3$ & $93$ & $0$ & $1-3-3-2$ & odwracalny & tak \\
$11a_{38}$ & $2$ & $5$ & $3$ & $117$ & $0$ & $1-1-1-2$ & chiralny & tak \\
$11a_{39}$ & $2$ & $5$ & $3$ & $101$ & $0$ & $1-1-6-3$ & odwracalny & tak \\
$11a_{40}$ & $2$ & $4$ & $3$ & $89$ & $-4$ & $1+2+2+3+1$ & odwracalny & tak \\
$11a_{41}$ & $1$ & $5$ & $3$ & $115$ & $-2$ & $1+1+1+2$ & odwracalny & tak \\
$11a_{42}$ & $2$ & $6$ & $3$ & $107$ & $-2$ & $1-1-3+1$ & odwracalny & tak \\
$11a_{43}$ & $3$ & $5$ & $4$ & $135$ & $-6$ & $1+6+9+4$ & odwracalny & tak \\
$11a_{44}$ & $2$ & $4$ & $4$ & $117$ & $0$ & $1+3+4+3+1$ & odwracalny & tak \\
$11a_{45}$ & $2..3$ & $6$ & $3$ & $87$ & $-2$ & $1-2-6$ & odwracalny & tak \\
$11a_{46}$ & $2$ & $5$ & $3$ & $87$ & $-2$ & $1+2+3+2$ & odwracalny & tak \\
$11a_{47}$ & $2$ & $4$ & $4$ & $117$ & $0$ & $1+3+4+3+1$ & odwracalny & tak \\
$11a_{48}$ & $2$ & $5$ & $3$ & $113$ & $4$ & $1+4+0-2$ & odwracalny & tak \\
$11a_{49}$ & $2..3$ & $5$ & $3$ & $105$ & $-4$ & $1+2-5-3$ & chiralny & tak \\
$11a_{50}$ & $2$ & $6$ & $3$ & $83$ & $-2$ & $1+1-5$ & odwracalny & tak \\
$11a_{51}$ & $2$ & $6$ & $3$ & $115$ & $2$ & $1+1-3+1$ & odwracalny & tak \\
$11a_{52}$ & $2$ & $5$ & $3$ & $137$ & $0$ & $1+2+1-2$ & chiralny & tak \\
$11a_{53}$ & $2..3$ & $4$ & $3$ & $97$ & $4$ & $1+0-2+2+1$ & chiralny & tak \\
$11a_{54}$ & $2$ & $5$ & $3$ & $139$ & $2$ & $1-1-1+2$ & chiralny & tak \\
$11a_{55}$ & $2$ & $4$ & $3$ & $71$ & $2$ & $1+2+0-3-1$ & odwracalny & tak \\
$11a_{56}$ & $1$ & $5$ & $3$ & $109$ & $0$ & $1+1-1-2$ & odwracalny & tak \\
$11a_{57}$ & $2$ & $4$ & $4$ & $99$ & $2$ & $1+1-2-3-1$ & odwracalny & tak \\
$11a_{58}$ & $2$ & $5$ & $3$ & $81$ & $0$ & $1+0-3-2$ & odwracalny & tak \\
$11a_{59}$ & $2$ & $6$ & ${}^{43}{\mskip -5mu/\mskip -3mu}_{23}$ & $43$ & $-2$ & $1-1-3$ & odwracalny & tak \\
$11a_{60}$ & $2..3$ & $5$ & $3$ & $85$ & $-4$ & $1+3-6-3$ & odwracalny & tak \\
$11a_{61}$ & $2$ & $6$ & $3$ & $103$ & $-2$ & $1+2-6$ & odwracalny & tak \\
$11a_{62}$ & $3$ & $4$ & $3$ & $55$ & $6$ & $1+6+2-3-1$ & odwracalny & tak \\
$11a_{63}$ & $2..3$ & $5$ & $3$ & $93$ & $4$ & $1+5-1-2$ & odwracalny & tak \\
$11a_{64}$ & $2..3$ & $5$ & $3$ & $97$ & $4$ & $1+4-1-2$ & odwracalny & tak \\
$11a_{65}$ & $2$ & $6$ & ${}^{59}{\mskip -5mu/\mskip -3mu}_{32}$ & $59$ & $2$ & $1+3-3$ & odwracalny & tak \\
$11a_{66}$ & $2$ & $4$ & $3$ & $119$ & $2$ & $1+2+1-2-1$ & chiralny & tak \\
$11a_{67}$ & $2$ & $5$ & $3$ & $125$ & $0$ & $1+1+0-2$ & chiralny & tak \\
$11a_{68}$ & $2$ & $4$ & $3$ & $103$ & $-2$ & $1+2+2-2-1$ & chiralny & tak \\
$11a_{69}$ & $1$ & $5$ & $3$ & $141$ & $0$ & $1+1+1-2$ & chiralny & tak \\
$11a_{70}$ & $1$ & $5$ & $3$ & $151$ & $-2$ & $1-2-2+2$ & chiralny & tak \\
$11a_{71}$ & $1$ & $4$ & $3$ & $159$ & $-2$ & $1+0-2-2-1$ & chiralny & tak \\
$11a_{72}$ & $2$ & $4$ & $3$ & $153$ & $0$ & $1+2+2+2+1$ & chiralny & tak \\
$11a_{73}$ & $1$ & $4$ & $3$ & $177$ & $0$ & $1+0-1+1+1$ & odwracalny & tak \\
$11a_{74}$ & $2$ & $4$ & $3$ & $73$ & $-4$ & $1-2+0+3+1$ & odwracalny & tak \\
$11a_{75}$ & $2$ & $5$ & ${}^{83}{\mskip -5mu/\mskip -3mu}_{36}$ & $83$ & $-2$ & $1-3+2+2$ & odwracalny & tak \\
$11a_{76}$ & $2$ & $4$ & $3$ & $145$ & $0$ & $1+0+1+2+1$ & chiralny & tak \\
$11a_{77}$ & $1$ & $5$ & ${}^{131}{\mskip -5mu/\mskip -3mu}_{76}$ & $131$ & $2$ & $1+1+0+2$ & odwracalny & tak \\
$11a_{78}$ & $1$ & $5$ & $3$ & $123$ & $-2$ & $1-1+0+2$ & odwracalny & tak \\
$11a_{79}$ & $2$ & $4$ & $3$ & $143$ & $2$ & $1+0-1-2-1$ & chiralny & tak \\
$11a_{80}$ & $1$ & $4$ & $3$ & $137$ & $0$ & $1-2+0+2+1$ & chiralny & tak \\
$11a_{81}$ & $2$ & $4$ & $3$ & $127$ & $-2$ & $1+0+0-2-1$ & chiralny & tak \\
$11a_{82}$ & $1$ & $4$ & $3$ & $95$ & $2$ & $1+0-2-3-1$ & odwracalny & tak \\
$11a_{83}$ & $2..3$ & $4$ & $3$ & $113$ & $-4$ & $1+4+4+3+1$ & odwracalny & tak \\
$11a_{84}$ & $2$ & $5$ & ${}^{101}{\mskip -5mu/\mskip -3mu}_{57}$ & $101$ & $0$ & $1-1-2-2$ & odwracalny & tak \\
$11a_{85}$ & $2$ & $5$ & ${}^{107}{\mskip -5mu/\mskip -3mu}_{47}$ & $107$ & $-2$ & $1+3+2+2$ & odwracalny & tak \\
$11a_{86}$ & $1$ & $4$ & $3$ & $91$ & $-2$ & $1-1-2-3-1$ & odwracalny & tak \\
$11a_{87}$ & $2$ & $5$ & $3$ & $121$ & $0$ & $1-2-1-2$ & odwracalny & tak \\
$11a_{88}$ & $1$ & $4$ & $3$ & $101$ & $0$ & $1-1+2+3+1$ & odwracalny & tak \\
$11a_{89}$ & $2$ & $5$ & ${}^{119}{\mskip -5mu/\mskip -3mu}_{44}$ & $119$ & $-2$ & $1-2+0+2$ & odwracalny & tak \\
$11a_{90}$ & $2$ & $5$ & ${}^{87}{\mskip -5mu/\mskip -3mu}_{64}$ & $87$ & $2$ & $1-2+2+2$ & odwracalny & tak \\
$11a_{91}$ & $1$ & $5$ & ${}^{129}{\mskip -5mu/\mskip -3mu}_{50}$ & $129$ & $0$ & $1+0+0-2$ & odwracalny & tak \\
$11a_{92}$ & $2$ & $4$ & $3$ & $103$ & $2$ & $1+2-2-3-1$ & odwracalny & tak \\
$11a_{93}$ & $2$ & $5$ & ${}^{93}{\mskip -5mu/\mskip -3mu}_{52}$ & $93$ & $0$ & $1+1-2-2$ & odwracalny & tak \\
$11a_{94}$ & $3$ & $4$ & $3$ & $107$ & $-6$ & $1+7+11+4$ & odwracalny & tak \\
$11a_{95}$ & $2$ & $5$ & ${}^{73}{\mskip -5mu/\mskip -3mu}_{33}$ & $73$ & $-4$ & $1+6+6$ & odwracalny & tak \\
$11a_{96}$ & $1$ & $6$ & ${}^{121}{\mskip -5mu/\mskip -3mu}_{50}$ & $121$ & $0$ & $1-2+3-1$ & odwracalny & tak \\
$11a_{97}$ & $2$ & $5$ & $3$ & $71$ & $-2$ & $1-2+3+2$ & odwracalny & tak \\
$11a_{98}$ & $1$ & $6$ & ${}^{77}{\mskip -5mu/\mskip -3mu}_{18}$ & $77$ & $0$ & $1-3+4$ & odwracalny & tak \\
$11a_{99}$ & $2$ & $4$ & $3$ & $135$ & $-2$ & $1-2-1-2-1$ & odwracalny & tak \\
$11a_{100}$ & $2$ & $5$ & $3$ & $141$ & $-4$ & $1+1-3-3$ & odwracalny & tak \\
$11a_{101}$ & $1$ & $5$ & $3$ & $167$ & $-2$ & $1+2+2+3$ & chiralny & tak \\
$11a_{102}$ & $2$ & $5$ & $3$ & $99$ & $-2$ & $1-3+1+2$ & chiralny & tak \\
$11a_{103}$ & $2$ & $6$ & $3$ & $81$ & $0$ & $1-4+4$ & chiralny & tak \\
$11a_{104}$ & $1$ & $5$ & $3$ & $125$ & $0$ & $1+1+0-2$ & chiralny & tak \\
$11a_{105}$ & $2..3$ & $5$ & $3$ & $109$ & $-4$ & $1+1-5-3$ & odwracalny & tak \\
$11a_{106}$ & $1$ & $4$ & $3$ & $93$ & $0$ & $1+1+2+3+1$ & odwracalny & tak \\
$11a_{107}$ & $2$ & $5$ & $3$ & $111$ & $-2$ & $1+0+1+2$ & odwracalny & tak \\
$11a_{108}$ & $2$ & $4$ & $3$ & $99$ & $2$ & $1+1-2-3-1$ & odwracalny & tak \\
$11a_{109}$ & $2$ & $4$ & $3$ & $117$ & $0$ & $1+3+4+3+1$ & odwracalny & tak \\
$11a_{110}$ & $2$ & $5$ & ${}^{97}{\mskip -5mu/\mskip -3mu}_{35}$ & $97$ & $0$ & $1+0-2-2$ & odwracalny & tak \\
$11a_{111}$ & $2$ & $5$ & ${}^{103}{\mskip -5mu/\mskip -3mu}_{37}$ & $103$ & $-2$ & $1+2+2+2$ & odwracalny & tak \\
$11a_{112}$ & $2$ & $4$ & $3$ & $125$ & $0$ & $1-3-1+2+1$ & chiralny & tak \\
$11a_{113}$ & $2$ & $4$ & $3$ & $109$ & $4$ & $1+1-1+2+1$ & chiralny & tak \\
$11a_{114}$ & $1$ & $5$ & $3$ & $151$ & $-2$ & $1+2+3+3$ & chiralny & tak \\
$11a_{115}$ & $2$ & $5$ & $3$ & $121$ & $0$ & $1-2-5-3$ & chiralny & tak \\
$11a_{116}$ & $2$ & $5$ & $3$ & $137$ & $-4$ & $1+2-3-3$ & chiralny & tak \\
$11a_{117}$ & $2$ & $5$ & ${}^{117}{\mskip -5mu/\mskip -3mu}_{49}$ & $117$ & $-4$ & $1+3+0-2$ & odwracalny & tak \\
$11a_{118}$ & $2$ & $5$ & $3$ & $87$ & $-2$ & $1-2+2+2$ & odwracalny & tak \\
$11a_{119}$ & $2$ & $6$ & ${}^{77}{\mskip -5mu/\mskip -3mu}_{34}$ & $77$ & $0$ & $1-3+4$ & odwracalny & tak \\
$11a_{120}$ & $2$ & $5$ & ${}^{109}{\mskip -5mu/\mskip -3mu}_{64}$ & $109$ & $4$ & $1+5+0-2$ & odwracalny & tak \\
$11a_{121}$ & $1$ & $6$ & ${}^{119}{\mskip -5mu/\mskip -3mu}_{50}$ & $119$ & $2$ & $1+2-3+1$ & odwracalny & tak \\
$11a_{122}$ & $1$ & $5$ & $3$ & $127$ & $2$ & $1+0+0+2$ & chiralny & tak \\
$11a_{123}$ & $3$ & $5$ & $3$ & $117$ & $-4$ & $1+7+9$ & odwracalny & tak \\
$11a_{124}$ & $3$ & $4$ & $3$ & $155$ & $-6$ & $1+7+12+5$ & odwracalny & tak \\
$11a_{125}$ & $2$ & $4$ & $3$ & $175$ & $-2$ & $1+0-3-2-1$ & chiralny & tak \\
$11a_{126}$ & $2..3$ & $4$ & $3$ & $145$ & $0$ & $1+4+6+3+1$ & odwracalny & tak \\
$11a_{127}$ & $2$ & $4$ & $3$ & $137$ & $-4$ & $1+2+1+2+1$ & chiralny & tak \\
$11a_{128}$ & $2$ & $6$ & $3$ & $129$ & $0$ & $1-4+3-1$ & odwracalny & tak \\
$11a_{129}$ & $2$ & $4$ & $3$ & $113$ & $4$ & $1+0-1+2+1$ & chiralny & tak \\
$11a_{130}$ & $2$ & $5$ & $3$ & $123$ & $2$ & $1-1+0+2$ & chiralny & tak \\
$11a_{131}$ & $2$ & $4$ & $3$ & $131$ & $2$ & $1+1+0-2-1$ & chiralny & tak \\
$11a_{132}$ & $2$ & $5$ & $3$ & $135$ & $-2$ & $1-2-1+2$ & chiralny & tak \\
$11a_{133}$ & $2$ & $6$ & $3$ & $79$ & $-2$ & $1+0-5$ & odwracalny & tak \\
$11a_{134}$ & $2$ & $5$ & $3$ & $123$ & $-2$ & $1+3+5+3$ & odwracalny & tak \\
$11a_{135}$ & $2$ & $5$ & $3$ & $153$ & $0$ & $1-2+1-2$ & odwracalny & tak \\
$11a_{136}$ & $1$ & $5$ & $3$ & $163$ & $2$ & $1+1-2+2$ & chiralny & tak \\
$11a_{137}$ & $2..3$ & $5$ & $3$ & $111$ & $2$ & $1-4+0+2$ & chiralny & tak \\
$11a_{138}$ & $1$ & $5$ & $3$ & $161$ & $0$ & $1+0+2-2$ & chiralny & tak \\
$11a_{139}$ & $1$ & $4$ & $3$ & $99$ & $-2$ & $1+1-2-3-1$ & odwracalny & tak \\
$11a_{140}$ & $2$ & $5$ & ${}^{65}{\mskip -5mu/\mskip -3mu}_{17}$ & $65$ & $-4$ & $1+0-4-2$ & odwracalny & tak \\
$11a_{141}$ & $2$ & $5$ & $3$ & $103$ & $2$ & $1-2+1+2$ & chiralny & tak \\
$11a_{142}$ & $3$ & $4$ & $3$ & $59$ & $6$ & $1+7+2-3-1$ & odwracalny & tak \\
$11a_{143}$ & $2$ & $5$ & $3$ & $89$ & $4$ & $1+6-1-2$ & odwracalny & tak \\
$11a_{144}$ & $2..3$ & $5$ & ${}^{73}{\mskip -5mu/\mskip -3mu}_{56}$ & $73$ & $4$ & $1+6-2-2$ & odwracalny & tak \\
$11a_{145}$ & $2$ & $6$ & ${}^{83}{\mskip -5mu/\mskip -3mu}_{22}$ & $83$ & $2$ & $1+5-4$ & odwracalny & tak \\
$11a_{146}$ & $1$ & $4$ & $3$ & $123$ & $2$ & $1+3+1-2-1$ & chiralny & tak \\
$11a_{147}$ & $2$ & $4$ & $3$ & $151$ & $-2$ & $1+2-1-2-1$ & chiralny & tak \\
$11a_{148}$ & $2$ & $6$ & $3$ & $115$ & $-2$ & $1+1-7$ & odwracalny & tak \\
$11a_{149}$ & $1$ & $5$ & $3$ & $127$ & $-2$ & $1+0+0+2$ & chiralny & tak \\
$11a_{150}$ & $2$ & $5$ & $3$ & $125$ & $4$ & $1+5+1-2$ & chiralny & tak \\
$11a_{151}$ & $2$ & $4$ & $3$ & $127$ & $2$ & $1+4+1-2-1$ & chiralny & tak \\
$11a_{152}$ & $2$ & $5$ & $3$ & $117$ & $0$ & $1+3+0-2$ & chiralny & tak \\
$11a_{153}$ & $2$ & $5$ & $3$ & $89$ & $0$ & $1+2-2-2$ & odwracalny & tak \\
$11a_{154}$ & $2$ & $6$ & ${}^{67}{\mskip -5mu/\mskip -3mu}_{37}$ & $67$ & $-2$ & $1+1-4$ & odwracalny & tak \\
$11a_{155}$ & $2$ & $5$ & $3$ & $171$ & $2$ & $1+3+2+3$ & odwracalny & tak \\
$11a_{156}$ & $2$ & $4$ & $3$ & $91$ & $2$ & $1+3-1-3-1$ & chiralny & tak \\
$11a_{157}$ & $2$ & $4$ & $3$ & $135$ & $2$ & $1+2+0-2-1$ & chiralny & tak \\
$11a_{158}$ & $2$ & $4$ & $3$ & $119$ & $2$ & $1-2-4-3-1$ & odwracalny & tak \\
$11a_{159}$ & $2$ & $6$ & ${}^{111}{\mskip -5mu/\mskip -3mu}_{65}$ & $111$ & $-2$ & $1+0-3+1$ & odwracalny & tak \\
$11a_{160}$ & $1$ & $4$ & $3$ & $145$ & $0$ & $1+0+1+2+1$ & odwracalny & tak \\
$11a_{161}$ & $2..3$ & $5$ & $3$ & $57$ & $4$ & $1+2-4-2$ & odwracalny & tak \\
$11a_{162}$ & $2$ & $4$ & $3$ & $167$ & $2$ & $1+2+2-1-1$ & chiralny & tak \\
$11a_{163}$ & $2$ & $4$ & $3$ & $119$ & $-2$ & $1+2+1-2-1$ & chiralny & tak \\
$11a_{164}$ & $2$ & $4$ & $3$ & $169$ & $0$ & $1-2-2+1+1$ & chiralny & tak \\
$11a_{165}$ & $2$ & $5$ & $3$ & $81$ & $0$ & $1+0-3-2$ & odwracalny & tak \\
$11a_{166}$ & $2$ & $6$ & ${}^{59}{\mskip -5mu/\mskip -3mu}_{45}$ & $59$ & $-2$ & $1-1-4$ & odwracalny & tak \\
$11a_{167}$ & $1$ & $5$ & $3$ & $113$ & $0$ & $1+0-1-2$ & chiralny & tak \\
$11a_{168}$ & $1$ & $5$ & $3$ & $125$ & $0$ & $1+1+0-2$ & chiralny & tak \\
$11a_{169}$ & $2$ & $5$ & $3$ & $121$ & $0$ & $1+2+0-2$ & chiralny & tak \\
$11a_{170}$ & $2$ & $4$ & $3$ & $185$ & $0$ & $1+2+4+2+1$ & odwracalny & tak \\
$11a_{171}$ & $2$ & $4$ & $3$ & $183$ & $-2$ & $1+2+1-1-1$ & odwracalny & tak \\
$11a_{172}$ & $2$ & $5$ & $3$ & $139$ & $-2$ & $1-1-1+2$ & chiralny & tak \\
$11a_{173}$ & $2$ & $5$ & $3$ & $135$ & $-2$ & $1+2+0+2$ & odwracalny & tak \\
$11a_{174}$ & $2$ & $4$ & ${}^{79}{\mskip -5mu/\mskip -3mu}_{51}$ & $79$ & $2$ & $1+0-1-3-1$ & odwracalny & tak \\
$11a_{175}$ & $2$ & $4$ & ${}^{105}{\mskip -5mu/\mskip -3mu}_{41}$ & $105$ & $0$ & $1+2+3+3+1$ & odwracalny & tak \\
$11a_{176}$ & $2$ & $4$ & ${}^{111}{\mskip -5mu/\mskip -3mu}_{31}$ & $111$ & $-2$ & $1+0-3-3-1$ & odwracalny & tak \\
$11a_{177}$ & $2$ & $4$ & ${}^{97}{\mskip -5mu/\mskip -3mu}_{21}$ & $97$ & $-4$ & $1+4+3+3+1$ & odwracalny & tak \\
$11a_{178}$ & $2$ & $5$ & ${}^{123}{\mskip -5mu/\mskip -3mu}_{89}$ & $123$ & $-2$ & $1+3+1+2$ & odwracalny & tak \\
$11a_{179}$ & $2$ & $4$ & ${}^{57}{\mskip -5mu/\mskip -3mu}_{20}$ & $57$ & $-4$ & $1-2-1+3+1$ & odwracalny & tak \\
$11a_{180}$ & $1$ & $4$ & ${}^{89}{\mskip -5mu/\mskip -3mu}_{64}$ & $89$ & $0$ & $1-2+1+3+1$ & odwracalny & tak \\
$11a_{181}$ & $2$ & $5$ & $3$ & $99$ & $-2$ & $1-3+1+2$ & odwracalny & tak \\
$11a_{182}$ & $2$ & $4$ & ${}^{73}{\mskip -5mu/\mskip -3mu}_{60}$ & $73$ & $4$ & $1+2+1+3+1$ & odwracalny & tak \\
$11a_{183}$ & $2$ & $5$ & ${}^{115}{\mskip -5mu/\mskip -3mu}_{34}$ & $115$ & $2$ & $1+1+1+2$ & odwracalny & tak \\
$11a_{184}$ & $1$ & $4$ & ${}^{87}{\mskip -5mu/\mskip -3mu}_{68}$ & $87$ & $2$ & $1+2-1-3-1$ & odwracalny & tak \\
$11a_{185}$ & $2$ & $5$ & ${}^{109}{\mskip -5mu/\mskip -3mu}_{30}$ & $109$ & $0$ & $1+1-1-2$ & odwracalny & tak \\
$11a_{186}$ & $3$ & $4$ & ${}^{95}{\mskip -5mu/\mskip -3mu}_{39}$ & $95$ & $-6$ & $1+8+12+4$ & odwracalny & tak \\
$11a_{187}$ & $1$ & $5$ & $3$ & $117$ & $0$ & $1-1-1-2$ & chiralny & tak \\
$11a_{188}$ & $2$ & $5$ & ${}^{67}{\mskip -5mu/\mskip -3mu}_{14}$ & $67$ & $-2$ & $1-3+3+2$ & odwracalny & tak \\
$11a_{189}$ & $1$ & $4$ & $3$ & $149$ & $0$ & $1-1+1+2+1$ & chiralny & tak \\
$11a_{190}$ & $1$ & $5$ & ${}^{85}{\mskip -5mu/\mskip -3mu}_{67}$ & $85$ & $0$ & $1-1-3-2$ & odwracalny & tak \\
$11a_{191}$ & $3$ & $4$ & ${}^{83}{\mskip -5mu/\mskip -3mu}_{19}$ & $83$ & $-6$ & $1+9+13+4$ & odwracalny & tak \\
$11a_{192}$ & $3$ & $5$ & ${}^{97}{\mskip -5mu/\mskip -3mu}_{71}$ & $97$ & $-4$ & $1+8+8$ & odwracalny & tak \\
$11a_{193}$ & $2$ & $5$ & ${}^{95}{\mskip -5mu/\mskip -3mu}_{66}$ & $95$ & $2$ & $1+0+2+2$ & odwracalny & tak \\
$11a_{194}$ & $2$ & $4$ & $3$ & $93$ & $4$ & $1+1+2+3+1$ & odwracalny & tak \\
$11a_{195}$ & $1$ & $6$ & ${}^{53}{\mskip -5mu/\mskip -3mu}_{8}$ & $53$ & $0$ & $1-1+3$ & odwracalny & tak \\
$11a_{196}$ & $2$ & $4$ & $3$ & $147$ & $2$ & $1+1-1-2-1$ & odwracalny & tak \\
$11a_{197}$ & $2..3$ & $5$ & $3$ & $143$ & $-2$ & $1+4+4+3$ & odwracalny & tak \\
$11a_{198}$ & $1$ & $5$ & $3$ & $115$ & $-2$ & $1+1+1+2$ & odwracalny & tak \\
$11a_{199}$ & $2$ & $5$ & $3$ & $99$ & $2$ & $1-3+1+2$ & chiralny & tak \\
$11a_{200}$ & $2$ & $5$ & $3$ & $85$ & $-4$ & $1+7+7$ & odwracalny & tak \\
$11a_{201}$ & $2$ & $6$ & $3$ & $81$ & $0$ & $1-4+4$ & chiralny & tak \\
$11a_{202}$ & $2..3$ & $5$ & $3$ & $111$ & $-2$ & $1-4+0+2$ & odwracalny & tak \\
$11a_{203}$ & $3$ & $4$ & ${}^{63}{\mskip -5mu/\mskip -3mu}_{11}$ & $63$ & $-6$ & $1+4+1-3-1$ & odwracalny & tak \\
$11a_{204}$ & $2$ & $5$ & ${}^{101}{\mskip -5mu/\mskip -3mu}_{71}$ & $101$ & $-4$ & $1+3-1-2$ & odwracalny & tak \\
$11a_{205}$ & $2$ & $5$ & ${}^{91}{\mskip -5mu/\mskip -3mu}_{66}$ & $91$ & $2$ & $1-1+2+2$ & odwracalny & tak \\
$11a_{206}$ & $3$ & $4$ & ${}^{47}{\mskip -5mu/\mskip -3mu}_{40}$ & $47$ & $6$ & $1+8+3-3-1$ & odwracalny & tak \\
$11a_{207}$ & $2$ & $5$ & ${}^{85}{\mskip -5mu/\mskip -3mu}_{26}$ & $85$ & $4$ & $1+7-1-2$ & odwracalny & tak \\
$11a_{208}$ & $2$ & $5$ & ${}^{105}{\mskip -5mu/\mskip -3mu}_{74}$ & $105$ & $4$ & $1+6+0-2$ & odwracalny & tak \\
$11a_{209}$ & $1$ & $6$ & $3$ & $141$ & $0$ & $1-3+4-1$ & chiralny & tak \\
$11a_{210}$ & $1$ & $6$ & ${}^{73}{\mskip -5mu/\mskip -3mu}_{16}$ & $73$ & $0$ & $1-2+4$ & odwracalny & tak \\
$11a_{211}$ & $2$ & $6$ & ${}^{67}{\mskip -5mu/\mskip -3mu}_{12}$ & $67$ & $2$ & $1+5-3$ & odwracalny & tak \\
$11a_{212}$ & $2$ & $5$ & $3$ & $153$ & $-4$ & $1+2-2-3$ & odwracalny & tak \\
$11a_{213}$ & $2$ & $5$ & $3$ & $145$ & $-4$ & $1+4-2-3$ & chiralny & tak \\
$11a_{214}$ & $2$ & $6$ & $3$ & $69$ & $0$ & $1-5+3$ & odwracalny & tak \\
$11a_{215}$ & $2$ & $4$ & $3$ & $133$ & $-4$ & $1+3+1+2+1$ & odwracalny & tak \\
$11a_{216}$ & $1$ & $4$ & $3$ & $147$ & $-2$ & $1+1-1-2-1$ & chiralny & tak \\
$11a_{217}$ & $2$ & $4$ & $3$ & $139$ & $2$ & $1+3+0-2-1$ & chiralny & tak \\
$11a_{218}$ & $2$ & $6$ & $3$ & $131$ & $2$ & $1+1-4+1$ & chiralny & tak \\
$11a_{219}$ & $2$ & $6$ & $3$ & $87$ & $2$ & $1+6-4$ & odwracalny & tak \\
$11a_{220}$ & $2$ & $5$ & ${}^{85}{\mskip -5mu/\mskip -3mu}_{23}$ & $85$ & $-4$ & $1+3-2-2$ & odwracalny & tak \\
$11a_{221}$ & $2$ & $4$ & $3$ & $79$ & $2$ & $1+4+0-3-1$ & odwracalny & tak \\
$11a_{222}$ & $1$ & $5$ & $3$ & $101$ & $0$ & $1+3-1-2$ & odwracalny & tak \\
$11a_{223}$ & $3$ & $4$ & $3$ & $75$ & $-6$ & $1+3+0-3-1$ & odwracalny & tak \\
$11a_{224}$ & $2$ & $5$ & ${}^{89}{\mskip -5mu/\mskip -3mu}_{27}$ & $89$ & $-4$ & $1+2-2-2$ & odwracalny & tak \\
$11a_{225}$ & $2$ & $5$ & ${}^{53}{\mskip -5mu/\mskip -3mu}_{42}$ & $53$ & $4$ & $1+3-4-2$ & odwracalny & tak \\
$11a_{226}$ & $1$ & $6$ & ${}^{71}{\mskip -5mu/\mskip -3mu}_{20}$ & $71$ & $2$ & $1+2-4$ & odwracalny & tak \\
$11a_{227}$ & $3$ & $4$ & $3$ & $143$ & $-6$ & $1+8+13+5$ & odwracalny & tak \\
$11a_{228}$ & $2$ & $6$ & $3$ & $133$ & $0$ & $1-1+4-1$ & odwracalny & tak \\
$11a_{229}$ & $2$ & $6$ & ${}^{71}{\mskip -5mu/\mskip -3mu}_{55}$ & $71$ & $-2$ & $1+2-4$ & odwracalny & tak \\
$11a_{230}$ & $1$ & $6$ & ${}^{51}{\mskip -5mu/\mskip -3mu}_{43}$ & $51$ & $-2$ & $1+1-3$ & odwracalny & tak \\
$11a_{231}$ & $2$ & $4$ & $4$ & $99$ & $2$ & $1+1-2-3-1$ & odwracalny & tak \\
$11a_{232}$ & $2$ & $4$ & $3$ & $123$ & $2$ & $1-1-4-3-1$ & chiralny & tak \\
$11a_{233}$ & $1$ & $4$ & $3$ & $173$ & $0$ & $1+1+3+2+1$ & chiralny & tak \\
$11a_{234}$ & $4$ & $3$ & ${}^{37}{\mskip -5mu/\mskip -3mu}_{5}$ & $37$ & $-8$ & $1+11+21+12+2$ & odwracalny & tak \\
$11a_{235}$ & $3$ & $4$ & ${}^{71}{\mskip -5mu/\mskip -3mu}_{49}$ & $71$ & $-6$ & $1+10+14+4$ & odwracalny & tak \\
$11a_{236}$ & $3$ & $4$ & ${}^{99}{\mskip -5mu/\mskip -3mu}_{29}$ & $99$ & $-6$ & $1+9+12+4$ & odwracalny & tak \\
$11a_{237}$ & $3$ & $5$ & $3$ & $93$ & $-4$ & $1+9+8$ & odwracalny & tak \\
$11a_{238}$ & $2$ & $5$ & ${}^{65}{\mskip -5mu/\mskip -3mu}_{53}$ & $65$ & $-4$ & $1+8+6$ & odwracalny & tak \\
$11a_{239}$ & $2$ & $4$ & $3$ & $195$ & $-2$ & $1+1+0-1-1$ & odwracalny & tak \\
$11a_{240}$ & $4$ & $3$ & $3$ & $61$ & $-8$ & $1+9+18+11+2$ & odwracalny & tak \\
$11a_{241}$ & $3$ & $4$ & $3$ & $95$ & $-6$ & $1+8+12+4$ & odwracalny & tak \\
$11a_{242}$ & $3$ & $4$ & ${}^{47}{\mskip -5mu/\mskip -3mu}_{9}$ & $47$ & $-6$ & $1+8+11+3$ & odwracalny & tak \\
$11a_{243}$ & $2$ & $5$ & ${}^{69}{\mskip -5mu/\mskip -3mu}_{49}$ & $69$ & $-4$ & $1+7+6$ & odwracalny & tak \\
$11a_{244}$ & $3$ & $4$ & $3$ & $147$ & $-6$ & $1+9+13+5$ & chiralny & tak \\
$11a_{245}$ & $3$ & $4$ & $3$ & $75$ & $-6$ & $1+7+9+3$ & odwracalny & tak \\
$11a_{246}$ & $2$ & $5$ & ${}^{41}{\mskip -5mu/\mskip -3mu}_{13}$ & $41$ & $-4$ & $1+6+4$ & odwracalny & tak \\
$11a_{247}$ & $1$ & $6$ & ${}^{19}{\mskip -5mu/\mskip -3mu}_{17}$ & $19$ & $-2$ & $1+5$ & odwracalny & tak \\
$11a_{248}$ & $2$ & $4$ & $3$ & $159$ & $2$ & $1+0-2-2-1$ & odwracalny & tak \\
$11a_{249}$ & $2$ & $5$ & $3$ & $117$ & $0$ & $1-1-1-2$ & odwracalny & tak \\
$11a_{250}$ & $2$ & $4$ & $3$ & $85$ & $-4$ & $1-1+1+3+1$ & odwracalny & tak \\
$11a_{251}$ & $2$ & $4$ & $3$ & $133$ & $0$ & $1-1+0+2+1$ & chiralny & tak \\
$11a_{252}$ & $2$ & $4$ & $3$ & $131$ & $-2$ & $1+1+0-2-1$ & chiralny & tak \\
$11a_{253}$ & $2$ & $4$ & $3$ & $133$ & $0$ & $1-1+0+2+1$ & chiralny & tak \\
$11a_{254}$ & $2$ & $4$ & $3$ & $131$ & $-2$ & $1+1+0-2-1$ & chiralny & tak \\
$11a_{255}$ & $1$ & $4$ & $3$ & $143$ & $2$ & $1+0-1-2-1$ & chiralny & tak \\
$11a_{256}$ & $1$ & $5$ & $3$ & $133$ & $0$ & $1-1+0-2$ & chiralny & tak \\
$11a_{257}$ & $1$ & $4$ & $3$ & $97$ & $0$ & $1+0+2+3+1$ & odwracalny & tak \\
$11a_{258}$ & $2$ & $5$ & $3$ & $75$ & $-2$ & $1-1+3+2$ & odwracalny & tak \\
$11a_{259}$ & $3$ & $4$ & $3$ & $79$ & $-6$ & $1+4+0-3-1$ & odwracalny & tak \\
$11a_{260}$ & $2$ & $5$ & $3$ & $69$ & $-4$ & $1+3-3-2$ & odwracalny & tak \\
$11a_{261}$ & $2$ & $4$ & $3$ & $129$ & $4$ & $1+0+0+2+1$ & chiralny & tak \\
$11a_{262}$ & $2$ & $5$ & $3$ & $107$ & $2$ & $1-1+1+2$ & chiralny & tak \\
$11a_{263}$ & $4$ & $4$ & $4$ & $81$ & $-8$ & $1+8+15+10+2$ & odwracalny & tak \\
$11a_{264}$ & $1$ & $4$ & $3$ & $135$ & $-2$ & $1+2+0-2-1$ & chiralny & tak \\
$11a_{265}$ & $2$ & $5$ & $3$ & $109$ & $0$ & $1+1-1-2$ & chiralny & tak \\
$11a_{266}$ & $1$ & $4$ & $3$ & $209$ & $0$ & $1+0+1+1+1$ & odwracalny & tak \\
$11a_{267}$ & $1$ & $4$ & $3$ & $191$ & $2$ & $1+0+0-1-1$ & chiralny & tak \\
$11a_{268}$ & $2$ & $4$ & $3$ & $139$ & $-2$ & $1-1-1-2-1$ & chiralny & tak \\
$11a_{269}$ & $2$ & $4$ & $3$ & $151$ & $-2$ & $1-2-2-2-1$ & chiralny & tak \\
$11a_{270}$ & $2$ & $5$ & $3$ & $137$ & $0$ & $1-2+0-2$ & chiralny & tak \\
$11a_{271}$ & $2$ & $5$ & $3$ & $171$ & $-2$ & $1-1+1+3$ & odwracalny & tak \\
$11a_{272}$ & $1$ & $5$ & $3$ & $149$ & $0$ & $1-1+1-2$ & chiralny & tak \\
$11a_{273}$ & $2$ & $5$ & $3$ & $159$ & $2$ & $1+0+2+3$ & chiralny & tak \\
$11a_{274}$ & $2$ & $4$ & $3$ & $165$ & $0$ & $1-1+2+2+1$ & chiralny & tak \\
$11a_{275}$ & $2$ & $5$ & $3$ & $129$ & $-4$ & $1+4-3-3$ & chiralny & tak \\
$11a_{276}$ & $2$ & $5$ & $3$ & $161$ & $-4$ & $1+4-1-3$ & chiralny & tak \\
$11a_{277}$ & $2$ & $4$ & $3$ & $135$ & $-2$ & $1-2-1-2-1$ & odwracalny & tak \\
$11a_{278}$ & $2$ & $5$ & $3$ & $141$ & $0$ & $1-3+0-2$ & odwracalny & tak \\
$11a_{279}$ & $2$ & $5$ & $3$ & $91$ & $-2$ & $1-1+6+3$ & odwracalny & tak \\
$11a_{280}$ & $1$ & $6$ & $3$ & $105$ & $0$ & $1-2+6$ & odwracalny & tak \\
$11a_{281}$ & $2$ & $4$ & $3$ & $155$ & $-2$ & $1-1-2-2-1$ & odwracalny & tak \\
$11a_{282}$ & $1$ & $4$ & $3$ & $127$ & $2$ & $1+0+0-2-1$ & chiralny & tak \\
$11a_{283}$ & $1$ & $5$ & $3$ & $147$ & $2$ & $1+1+3+3$ & chiralny & tak \\
$11a_{284}$ & $2$ & $4$ & $3$ & $179$ & $-2$ & $1+1+1-1-1$ & chiralny & tak \\
$11a_{285}$ & $2$ & $5$ & $3$ & $161$ & $0$ & $1+0+2-2$ & odwracalny & tak \\
$11a_{286}$ & $2$ & $4$ & $3$ & $147$ & $-2$ & $1+1-1-2-1$ & chiralny & tak \\
$11a_{287}$ & $1$ & $4$ & $3$ & $181$ & $0$ & $1-1-1+1+1$ & chiralny & tak \\
$11a_{288}$ & $2$ & $4$ & $3$ & $205$ & $0$ & $1+1+1+1+1$ & odwracalny & tak \\
$11a_{289}$ & $1$ & $4$ & $3$ & $145$ & $0$ & $1+0+1+2+1$ & chiralny & tak \\
$11a_{290}$ & $1$ & $5$ & $3$ & $141$ & $0$ & $1+1-3-3$ & chiralny & tak \\
$11a_{291}$ & $3..4$ & $4$ & $3$ & $99$ & $-6$ & $1+9+16+5$ & odwracalny & tak \\
$11a_{292}$ & $3$ & $5$ & $3$ & $129$ & $-4$ & $1+8+10$ & odwracalny & tak \\
$11a_{293}$ & $2..3$ & $4$ & $3$ & $81$ & $4$ & $1+4+2+3+1$ & odwracalny & tak \\
$11a_{294}$ & $2$ & $5$ & $3$ & $123$ & $2$ & $1+3+1+2$ & odwracalny & tak \\
$11a_{295}$ & $2$ & $5$ & $3$ & $109$ & $4$ & $1+5-4-3$ & chiralny & tak \\
$11a_{296}$ & $2$ & $6$ & $3$ & $111$ & $2$ & $1+4-6$ & chiralny & tak \\
$11a_{297}$ & $2$ & $5$ & $3$ & $175$ & $2$ & $1+0-3+2$ & odwracalny & tak \\
$11a_{298}$ & $3..4$ & $4$ & $3$ & $131$ & $-6$ & $1+9+14+5$ & chiralny & tak \\
$11a_{299}$ & $2..3$ & $5$ & $3$ & $97$ & $-4$ & $1+8+8$ & chiralny & tak \\
$11a_{300}$ & $1$ & $4$ & $3$ & $153$ & $0$ & $1-2+1+2+1$ & chiralny & tak \\
$11a_{301}$ & $2$ & $4$ & $3$ & $199$ & $2$ & $1+2+0-1-1$ & chiralny & tak \\
$11a_{302}$ & $2$ & $4$ & $3$ & $161$ & $4$ & $1+0-2+1+1$ & odwracalny & tak \\
$11a_{303}$ & $2$ & $5$ & $3$ & $149$ & $0$ & $1-1-3-3$ & chiralny & tak \\
$11a_{304}$ & $2..3$ & $5$ & $3$ & $117$ & $-4$ & $1+3-4-3$ & odwracalny & tak \\
$11a_{305}$ & $2$ & $4$ & $3$ & $135$ & $2$ & $1+2+0-2-1$ & odwracalny & tak \\
$11a_{306}$ & $2$ & $4$ & ${}^{105}{\mskip -5mu/\mskip -3mu}_{76}$ & $105$ & $4$ & $1+2+3+3+1$ & odwracalny & tak \\
$11a_{307}$ & $1$ & $5$ & ${}^{83}{\mskip -5mu/\mskip -3mu}_{18}$ & $83$ & $2$ & $1+1+3+2$ & odwracalny & tak \\
$11a_{308}$ & $3$ & $4$ & ${}^{71}{\mskip -5mu/\mskip -3mu}_{15}$ & $71$ & $-6$ & $1+6+1-3-1$ & odwracalny & tak \\
$11a_{309}$ & $2$ & $5$ & ${}^{93}{\mskip -5mu/\mskip -3mu}_{25}$ & $93$ & $-4$ & $1+5-1-2$ & odwracalny & tak \\
$11a_{310}$ & $2$ & $5$ & ${}^{61}{\mskip -5mu/\mskip -3mu}_{47}$ & $61$ & $-4$ & $1+5-3-2$ & odwracalny & tak \\
$11a_{311}$ & $1$ & $6$ & ${}^{79}{\mskip -5mu/\mskip -3mu}_{61}$ & $79$ & $-2$ & $1+4-4$ & odwracalny & tak \\
$11a_{312}$ & $2$ & $5$ & $3$ & $119$ & $2$ & $1+2+5+3$ & odwracalny & tak \\
$11a_{313}$ & $2$ & $6$ & $3$ & $77$ & $0$ & $1+1+5$ & odwracalny & tak \\
$11a_{314}$ & $2$ & $4$ & $3$ & $171$ & $-2$ & $1-1+1-1-1$ & odwracalny & tak \\
$11a_{315}$ & $2$ & $4$ & $3$ & $157$ & $0$ & $1+1+2+2+1$ & chiralny & tak \\
$11a_{316}$ & $2$ & $4$ & $3$ & $121$ & $0$ & $1+2+4+3+1$ & chiralny & tak \\
$11a_{317}$ & $2$ & $5$ & $3$ & $125$ & $0$ & $1+1-4-3$ & odwracalny & tak \\
$11a_{318}$ & $3$ & $4$ & $3$ & $135$ & $-6$ & $1+10+14+5$ & odwracalny & tak \\
$11a_{319}$ & $3..4$ & $4$ & $3$ & $119$ & $-6$ & $1+10+15+5$ & chiralny & tak \\
$11a_{320}$ & $2..3$ & $5$ & $3$ & $109$ & $-4$ & $1+9+9$ & chiralny & tak \\
$11a_{321}$ & $2$ & $5$ & $3$ & $121$ & $4$ & $1+6-3-3$ & odwracalny & tak \\
$11a_{322}$ & $2$ & $5$ & $3$ & $151$ & $2$ & $1+2-1+2$ & chiralny & tak \\
$11a_{323}$ & $2$ & $5$ & $3$ & $83$ & $2$ & $1+1+3+2$ & chiralny & tak \\
$11a_{324}$ & $2$ & $6$ & $3$ & $99$ & $2$ & $1+5-5$ & odwracalny & tak \\
$11a_{325}$ & $2$ & $6$ & $3$ & $95$ & $2$ & $1+0-6$ & odwracalny & tak \\
$11a_{326}$ & $1$ & $4$ & $3$ & $169$ & $0$ & $1+2+3+2+1$ & chiralny & tak \\
$11a_{327}$ & $2$ & $5$ & $3$ & $187$ & $-2$ & $1+3+1+3$ & odwracalny & tak \\
$11a_{328}$ & $2$ & $5$ & $3$ & $149$ & $-4$ & $1+3-2-3$ & chiralny & tak \\
$11a_{329}$ & $2..3$ & $5$ & $3$ & $145$ & $-4$ & $1+8+11$ & odwracalny & tak \\
$11a_{330}$ & $2$ & $4$ & $3$ & $89$ & $4$ & $1+2+2+3+1$ & chiralny & tak \\
$11a_{331}$ & $2$ & $5$ & $3$ & $115$ & $2$ & $1+1+1+2$ & chiralny & tak \\
$11a_{332}$ & $2$ & $4$ & $3$ & $189$ & $0$ & $1+1+0+1+1$ & odwracalny & tak \\
$11a_{333}$ & $2$ & $6$ & ${}^{65}{\mskip -5mu/\mskip -3mu}_{14}$ & $65$ & $0$ & $1+0+4$ & odwracalny & tak \\
$11a_{334}$ & $4$ & $3$ & ${}^{49}{\mskip -5mu/\mskip -3mu}_{9}$ & $49$ & $-8$ & $1+12+22+12+2$ & odwracalny & tak \\
$11a_{335}$ & $3$ & $4$ & ${}^{75}{\mskip -5mu/\mskip -3mu}_{17}$ & $75$ & $-6$ & $1+11+14+4$ & odwracalny & tak \\
$11a_{336}$ & $3..4$ & $4$ & ${}^{59}{\mskip -5mu/\mskip -3mu}_{11}$ & $59$ & $-6$ & $1+11+15+4$ & odwracalny & tak \\
$11a_{337}$ & $3$ & $5$ & ${}^{89}{\mskip -5mu/\mskip -3mu}_{63}$ & $89$ & $-4$ & $1+10+8$ & odwracalny & tak \\
$11a_{338}$ & $4$ & $3$ & $3$ & $69$ & $-8$ & $1+11+19+11+2$ & odwracalny & tak \\
$11a_{339}$ & $3$ & $4$ & ${}^{55}{\mskip -5mu/\mskip -3mu}_{13}$ & $55$ & $-6$ & $1+10+11+3$ & odwracalny & tak \\
$11a_{340}$ & $3..4$ & $4$ & $3$ & $87$ & $-6$ & $1+10+13+4$ & odwracalny & tak \\
$11a_{341}$ & $3$ & $5$ & ${}^{61}{\mskip -5mu/\mskip -3mu}_{19}$ & $61$ & $-4$ & $1+9+6$ & odwracalny & tak \\
$11a_{342}$ & $2$ & $5$ & ${}^{29}{\mskip -5mu/\mskip -3mu}_{25}$ & $29$ & $-4$ & $1+9+4$ & odwracalny & tak \\
$11a_{343}$ & $2$ & $6$ & ${}^{31}{\mskip -5mu/\mskip -3mu}_{27}$ & $31$ & $-2$ & $1+8$ & odwracalny & tak \\
$11a_{344}$ & $2$ & $5$ & $3$ & $129$ & $-4$ & $1+4-3-3$ & chiralny & tak \\
$11a_{345}$ & $1$ & $6$ & $3$ & $91$ & $-2$ & $1+3-5$ & chiralny & tak \\
$11a_{346}$ & $2..3$ & $4$ & $3$ & $93$ & $-4$ & $1+1+2+3+1$ & odwracalny & tak \\
$11a_{347}$ & $2$ & $5$ & $3$ & $111$ & $-2$ & $1+0+1+2$ & odwracalny & tak \\
$11a_{348}$ & $2$ & $4$ & $3$ & $145$ & $-4$ & $1+0-3+1+1$ & odwracalny & tak \\
$11a_{349}$ & $2$ & $5$ & $3$ & $155$ & $-2$ & $1-1-2+2$ & odwracalny & tak \\
$11a_{350}$ & $2$ & $4$ & $3$ & $185$ & $0$ & $1-2-1+1+1$ & odwracalny & tak \\
$11a_{351}$ & $1$ & $4$ & $3$ & $165$ & $0$ & $1-1-2+1+1$ & odwracalny & tak \\
$11a_{352}$ & $2$ & $5$ & $3$ & $135$ & $-2$ & $1-2-1+2$ & odwracalny & tak \\
$11a_{353}$ & $3..4$ & $4$ & $3$ & $123$ & $-6$ & $1+11+15+5$ & odwracalny & tak \\
$11a_{354}$ & $2..4$ & $5$ & $3$ & $105$ & $-4$ & $1+10+9$ & odwracalny & tak \\
$11a_{355}$ & $4$ & $3$ & ${}^{45}{\mskip -5mu/\mskip -3mu}_{7}$ & $45$ & $-8$ & $1+13+22+12+2$ & odwracalny & tak \\
$11a_{356}$ & $3..4$ & $4$ & ${}^{79}{\mskip -5mu/\mskip -3mu}_{55}$ & $79$ & $-6$ & $1+12+14+4$ & odwracalny & tak \\
$11a_{357}$ & $3..4$ & $4$ & ${}^{91}{\mskip -5mu/\mskip -3mu}_{27}$ & $91$ & $-6$ & $1+11+13+4$ & odwracalny & tak \\
$11a_{358}$ & $3$ & $4$ & ${}^{31}{\mskip -5mu/\mskip -3mu}_{5}$ & $31$ & $-6$ & $1+12+13+3$ & odwracalny & tak \\
$11a_{359}$ & $3$ & $5$ & ${}^{53}{\mskip -5mu/\mskip -3mu}_{43}$ & $53$ & $-4$ & $1+11+6$ & odwracalny & tak \\
$11a_{360}$ & $3$ & $5$ & ${}^{57}{\mskip -5mu/\mskip -3mu}_{47}$ & $57$ & $-4$ & $1+10+6$ & odwracalny & tak \\
$11a_{361}$ & $3..4$ & $5$ & $3$ & $69$ & $-4$ & $1+11+7$ & odwracalny & tak \\
$11a_{362}$ & $2..3$ & $6$ & $3$ & $39$ & $-2$ & $1+10$ & odwracalny & tak \\
$11a_{363}$ & $2..3$ & $6$ & ${}^{35}{\mskip -5mu/\mskip -3mu}_{29}$ & $35$ & $-2$ & $1+9$ & odwracalny & tak \\
$11a_{364}$ & $4$ & $3$ & ${}^{25}{\mskip -5mu/\mskip -3mu}_{3}$ & $25$ & $-8$ & $1+14+25+13+2$ & odwracalny & tak \\
$11a_{365}$ & $4$ & $4$ & ${}^{51}{\mskip -5mu/\mskip -3mu}_{35}$ & $51$ & $-6$ & $1+13+16+4$ & odwracalny & tak \\
$11a_{366}$ & $3..4$ & $5$ & $3$ & $81$ & $-4$ & $1+12+8$ & odwracalny & tak \\
$11a_{367}$ & $5$ & $2$ & ${}^{11}{\mskip -5mu/\mskip -3mu}_{1}$ & $11$ & $-10$ & $1+15+35+28+9+1$ & odwracalny & tak \\
$11n_{1}$ & $1$ & $5$ & $3$ & $27$ & $2$ & $1+3-1$ & odwracalny & nie \\
$11n_{2}$ & $2$ & $5$ & $3$ & $57$ & $-4$ & $1+2-4-2$ & odwracalny & nie \\
$11n_{3}$ & $1..2$ & $5$ & $3$ & $43$ & $2$ & $1-1-3$ & odwracalny & nie \\
$11n_{4}$ & $2$ & $4$ & $3$ & $49$ & $0$ & $1+0-1-1$ & chiralny & nie \\
$11n_{5}$ & $2$ & $4$ & $3$ & $71$ & $-2$ & $1-2-1+1$ & chiralny & nie \\
$11n_{6}$ & $2$ & $5$ & $3$ & $17$ & $0$ & $1+0-3-1$ & chiralny & nie \\
$11n_{7}$ & $2$ & $5$ & $3$ & $67$ & $-2$ & $1+1+0+1$ & chiralny & nie \\
$11n_{8}$ & $2$ & $4$ & $3$ & $53$ & $4$ & $1+3+0-1$ & chiralny & nie \\
$11n_{9}$ & $3$ & $4$ & $3$ & $5$ & $-4$ & $1+3-3-5-1$ & chiralny & nie \\
$11n_{10}$ & $2$ & $4$ & $3$ & $65$ & $4$ & $1+4+1-1$ & chiralny & nie \\
$11n_{11}$ & $2$ & $4$ & $3$ & $55$ & $-2$ & $1+2+1+1$ & chiralny & nie \\
$11n_{12}$ & $1$ & $4$ & $3$ & $13$ & $0$ & $1+1+1$ & odwracalny & nie \\
$11n_{13}$ & $3$ & $4$ & $3$ & $15$ & $-6$ & $1+4-4-5-1$ & odwracalny & nie \\
$11n_{14}$ & $2$ & $4$ & $3$ & $45$ & $4$ & $1+5+0-1$ & odwracalny & nie \\
$11n_{15}$ & $2$ & $4$ & $3$ & $35$ & $-2$ & $1+1+2+1$ & odwracalny & nie \\
$11n_{16}$ & $2$ & $5$ & $3$ & $37$ & $-4$ & $1+3-5-2$ & odwracalny & nie \\
$11n_{17}$ & $1..2$ & $5$ & $3$ & $47$ & $2$ & $1+4-2$ & odwracalny & nie \\
$11n_{18}$ & $1$ & $5$ & $3$ & $33$ & $0$ & $1+0+2$ & odwracalny & nie \\
$11n_{19}$ & $2$ & $4$ & $3$ & $5$ & $4$ & $1-1-4-1$ & odwracalny & nie \\
$11n_{20}$ & $1$ & $5$ & $3$ & $23$ & $2$ & $1-2-2$ & odwracalny & nie \\
$11n_{21}$ & $1$ & $4$ & $3$ & $49$ & $0$ & $1+0-1-1$ & chiralny & nie \\
$11n_{22}$ & $1$ & $4$ & $3$ & $55$ & $-2$ & $1+2+1+1$ & chiralny & nie \\
$11n_{23}$ & $2..3$ & $4$ & $3$ & $29$ & $-4$ & $1+5+7+5+1$ & chiralny & nie \\
$11n_{24}$ & $2$ & $4$ & $3$ & $23$ & $-2$ & $1+2+3+1$ & chiralny & nie \\
$11n_{25}$ & $1$ & $4$ & $3$ & $47$ & $-2$ & $1+0+1+1$ & chiralny & nie \\
$11n_{26}$ & $1$ & $4$ & $3$ & $41$ & $0$ & $1+2-1-1$ & chiralny & nie \\
$11n_{27}$ & $3$ & $4$ & $3$ & $19$ & $-6$ & $1+1-5-5-1$ & chiralny & nie \\
$11n_{28}$ & $1$ & $5$ & $3$ & $21$ & $0$ & $1-1+1$ & odwracalny & nie \\
$11n_{29}$ & $2$ & $5$ & $3$ & $51$ & $-2$ & $1+1-3$ & odwracalny & nie \\
$11n_{30}$ & $2..3$ & $5$ & $3$ & $33$ & $-4$ & $1+0-6-2$ & odwracalny & nie \\
$11n_{31}$ & $2$ & $5$ & $3$ & $3$ & $-2$ & $1+1-4-1$ & chiralny & nie \\
$11n_{32}$ & $2$ & $5$ & $3$ & $69$ & $0$ & $1-1+0-1$ & chiralny & nie \\
$11n_{33}$ & $2$ & $5$ & $3$ & $51$ & $-2$ & $1-3+0+1$ & chiralny & nie \\
$11n_{34}$ & $1$ & $4$ & $3$ & $1$ & $0$ & $1$ & chiralny & nie \\
$11n_{35}$ & $2$ & $4$ & $3$ & $89$ & $-4$ & $1+2-2-2$ & chiralny & nie \\
$11n_{36}$ & $2$ & $4$ & $3$ & $67$ & $-2$ & $1+1-4-4-1$ & chiralny & nie \\
$11n_{37}$ & $2$ & $4$ & $3$ & $25$ & $0$ & $1-2-3-1$ & chiralny & nie \\
$11n_{38}$ & $1$ & $4$ & $3$ & $3$ & $-2$ & $1-3-1$ & odwracalny & nie \\
$11n_{39}$ & $1$ & $4$ & $3$ & $25$ & $0$ & $1+2+2$ & chiralny & nie \\
$11n_{40}$ & $2$ & $4$ & $3$ & $79$ & $-2$ & $1+4+4+2$ & chiralny & nie \\
$11n_{41}$ & $2$ & $4$ & $3$ & $53$ & $-4$ & $1+3+4+4+1$ & chiralny & nie \\
$11n_{42}$ & $1$ & $4$ & $3$ & $1$ & $0$ & $1$ & chiralny & nie \\
$11n_{43}$ & $2$ & $4$ & $3$ & $89$ & $-4$ & $1+2-2-2$ & chiralny & nie \\
$11n_{44}$ & $2$ & $4$ & $3$ & $67$ & $-2$ & $1+1-4-4-1$ & chiralny & nie \\
$11n_{45}$ & $1$ & $4$ & $3$ & $25$ & $0$ & $1+2+2$ & odwracalny & nie \\
$11n_{46}$ & $2$ & $4$ & $3$ & $79$ & $-2$ & $1+4+4+2$ & chiralny & nie \\
$11n_{47}$ & $2$ & $4$ & $3$ & $53$ & $-4$ & $1+3+4+4+1$ & chiralny & nie \\
$11n_{48}$ & $1$ & $4$ & $3$ & $29$ & $0$ & $1-3-3-1$ & odwracalny & nie \\
$11n_{49}$ & $2$ & $5$ & $3$ & $1$ & $0$ & $1-4-1$ & odwracalny & nie \\
$11n_{50}$ & $1$ & $4$ & $3$ & $25$ & $0$ & $1+2+2$ & chiralny & nie \\
$11n_{51}$ & $1..2$ & $4$ & $3$ & $29$ & $0$ & $1+1-2-1$ & odwracalny & nie \\
$11n_{52}$ & $1$ & $4$ & $3$ & $59$ & $-2$ & $1-1+0+1$ & odwracalny & nie \\
$11n_{53}$ & $1$ & $4$ & $3$ & $37$ & $0$ & $1-1-2-1$ & odwracalny & nie \\
$11n_{54}$ & $1..2$ & $4$ & $3$ & $43$ & $-2$ & $1+3+2+1$ & odwracalny & nie \\
$11n_{55}$ & $1$ & $4$ & $3$ & $61$ & $0$ & $1+1+0-1$ & odwracalny & nie \\
$11n_{56}$ & $1$ & $4$ & $3$ & $35$ & $-2$ & $1+1+2+1$ & odwracalny & nie \\
$11n_{57}$ & $3$ & $4$ & $3$ & $7$ & $-6$ & $1+2-4-5-1$ & odwracalny & nie \\
$11n_{58}$ & $2$ & $4$ & $3$ & $35$ & $2$ & $1+1+2+1$ & odwracalny & nie \\
$11n_{59}$ & $2$ & $4$ & $3$ & $53$ & $-4$ & $1+3+0-1$ & odwracalny & nie \\
$11n_{60}$ & $1..2$ & $4$ & $3$ & $31$ & $-2$ & $1+0-6-5-1$ & odwracalny & nie \\
$11n_{61}$ & $2$ & $4$ & $3$ & $17$ & $-4$ & $1+4+6+5+1$ & odwracalny & nie \\
$11n_{62}$ & $1$ & $5$ & $3$ & $33$ & $0$ & $1+0+2$ & odwracalny & nie \\
$11n_{63}$ & $1$ & $5$ & $3$ & $39$ & $-2$ & $1+2-2$ & odwracalny & nie \\
$11n_{64}$ & $2$ & $5$ & $3$ & $21$ & $-4$ & $1-1-7-2$ & odwracalny & nie \\
$11n_{65}$ & $2..3$ & $4$ & $3$ & $33$ & $0$ & $1+4+3$ & odwracalny & nie \\
$11n_{66}$ & $2$ & $4$ & $3$ & $75$ & $-2$ & $1-1-1+1$ & odwracalny & nie \\
$11n_{67}$ & $2$ & $5$ & $3$ & $9$ & $0$ & $1-2$ & chiralny & nie \\
$11n_{68}$ & $2$ & $5$ & $3$ & $63$ & $-2$ & $1+0-4$ & chiralny & nie \\
$11n_{69}$ & $2$ & $5$ & $3$ & $45$ & $-4$ & $1+1-5-2$ & chiralny & nie \\
$11n_{70}$ & $2$ & $4$ & $3$ & $13$ & $-4$ & $1-3-4-1$ & odwracalny & nie \\
$11n_{71}$ & $2$ & $4$ & $4$ & $63$ & $-2$ & $1+4+5+2$ & odwracalny & nie \\
$11n_{72}$ & $2$ & $4$ & $4$ & $81$ & $-4$ & $1+0-3-2$ & odwracalny & nie \\
$11n_{73}$ & $2$ & $4$ & $4$ & $9$ & $0$ & $1+2+1$ & odwracalny & nie \\
$11n_{74}$ & $2$ & $4$ & $4$ & $9$ & $0$ & $1+2+1$ & odwracalny & nie \\
$11n_{75}$ & $2$ & $4$ & $4$ & $63$ & $2$ & $1+4+5+2$ & odwracalny & nie \\
$11n_{76}$ & $3$ & $4$ & $4$ & $45$ & $4$ & $1+5+8+5+1$ & odwracalny & nie \\
$11n_{77}$ & $4$ & $4$ & $4$ & $27$ & $-6$ & $1+7+12+7+1$ & odwracalny & nie \\
$11n_{78}$ & $2..3$ & $4$ & $4$ & $45$ & $-4$ & $1+5+8+5+1$ & odwracalny & nie \\
$11n_{79}$ & $2$ & $5$ & $3$ & $15$ & $-2$ & $1-4-2$ & odwracalny & nie \\
$11n_{80}$ & $2$ & $5$ & $3$ & $15$ & $2$ & $1+0-5-1$ & chiralny & nie \\
$11n_{81}$ & $3$ & $4$ & $4$ & $27$ & $-6$ & $1-1-6-5-1$ & odwracalny & nie \\
$11n_{82}$ & $1$ & $4$ & $3$ & $19$ & $-2$ & $1+1+3+1$ & odwracalny & nie \\
$11n_{83}$ & $2$ & $5$ & $3$ & $49$ & $0$ & $1+0+3$ & odwracalny & nie \\
$11n_{84}$ & $1$ & $4$ & $3$ & $35$ & $2$ & $1+1-2$ & odwracalny & nie \\
$11n_{85}$ & $1$ & $4$ & $3$ & $45$ & $0$ & $1+1-1-1$ & odwracalny & nie \\
$11n_{86}$ & $1$ & $4$ & $3$ & $33$ & $0$ & $1+0-2-1$ & odwracalny & nie \\
$11n_{87}$ & $1$ & $4$ & $3$ & $51$ & $2$ & $1+1+1+1$ & odwracalny & nie \\
$11n_{88}$ & $3$ & $4$ & $3$ & $11$ & $-6$ & $1+3-4-5-1$ & odwracalny & nie \\
$11n_{89}$ & $2$ & $4$ & $3$ & $61$ & $4$ & $1+5+1-1$ & odwracalny & nie \\
$11n_{90}$ & $2$ & $5$ & $3$ & $41$ & $-4$ & $1+2-5-2$ & odwracalny & nie \\
$11n_{91}$ & $2$ & $5$ & $3$ & $31$ & $2$ & $1+4-1$ & odwracalny & nie \\
$11n_{92}$ & $2$ & $4$ & $3$ & $15$ & $2$ & $1+0+3+1$ & odwracalny & nie \\
$11n_{93}$ & $3$ & $4$ & $3$ & $47$ & $-6$ & $1+8+11+3$ & chiralny & nie \\
$11n_{94}$ & $1..2$ & $4$ & $3$ & $57$ & $0$ & $1+2+0-1$ & odwracalny & nie \\
$11n_{95}$ & $2$ & $4$ & $3$ & $33$ & $-4$ & $1+4-1-1$ & odwracalny & nie \\
$11n_{96}$ & $1$ & $4$ & $3$ & $7$ & $-2$ & $1+2+4+1$ & odwracalny & nie \\
$11n_{97}$ & $2$ & $5$ & $3$ & $9$ & $0$ & $1-2$ & chiralny & nie \\
$11n_{98}$ & $2$ & $4$ & $3$ & $69$ & $0$ & $1+3+1-1$ & odwracalny & nie \\
$11n_{99}$ & $2..3$ & $4$ & $3$ & $39$ & $2$ & $1-2-3$ & odwracalny & nie \\
$11n_{100}$ & $1$ & $5$ & $3$ & $45$ & $0$ & $1-3+2$ & chiralny & nie \\
$11n_{101}$ & $1$ & $5$ & $3$ & $39$ & $2$ & $1+2-2$ & odwracalny & nie \\
$11n_{102}$ & $1..2$ & $5$ & $3$ & $3$ & $2$ & $1-3-1$ & odwracalny & nie \\
$11n_{103}$ & $2$ & $4$ & $3$ & $65$ & $4$ & $1+4+1-1$ & odwracalny & nie \\
$11n_{104}$ & $3$ & $4$ & $3$ & $3$ & $-6$ & $1+1-4-5-1$ & odwracalny & nie \\
$11n_{105}$ & $2$ & $4$ & $3$ & $69$ & $-4$ & $1+3+1-1$ & odwracalny & nie \\
$11n_{106}$ & $2$ & $4$ & $3$ & $27$ & $2$ & $1+3+3+1$ & odwracalny & nie \\
$11n_{107}$ & $2$ & $4$ & $3$ & $21$ & $-4$ & $1+3+6+5+1$ & odwracalny & nie \\
$11n_{108}$ & $2$ & $4$ & $3$ & $73$ & $4$ & $1+6+2-1$ & chiralny & nie \\
$11n_{109}$ & $2$ & $4$ & $3$ & $57$ & $4$ & $1+6+1-1$ & chiralny & nie \\
$11n_{110}$ & $1$ & $4$ & $3$ & $41$ & $0$ & $1-2-2-1$ & chiralny & nie \\
$11n_{111}$ & $1$ & $4$ & $3$ & $7$ & $-2$ & $1-2-5-1$ & odwracalny & nie \\
$11n_{112}$ & $1..2$ & $4$ & $3$ & $55$ & $-2$ & $1+2+1+1$ & odwracalny & nie \\
$11n_{113}$ & $2$ & $5$ & $3$ & $35$ & $2$ & $1+5-1$ & odwracalny & nie \\
$11n_{114}$ & $1$ & $5$ & $3$ & $53$ & $0$ & $1-1+3$ & chiralny & nie \\
$11n_{115}$ & $1..2$ & $5$ & $3$ & $77$ & $0$ & $1-3+0-1$ & odwracalny & nie \\
$11n_{116}$ & $1..2$ & $5$ & $3$ & $1$ & $0$ & $1-4-1$ & chiralny & nie \\
$11n_{117}$ & $2$ & $5$ & $3$ & $35$ & $-2$ & $1-3-3$ & odwracalny & nie \\
$11n_{118}$ & $2$ & $4$ & $3$ & $21$ & $-4$ & $1+3-2-1$ & odwracalny & nie \\
$11n_{119}$ & $1..2$ & $4$ & $3$ & $69$ & $0$ & $1-1+0-1$ & odwracalny & nie \\
$11n_{120}$ & $1..2$ & $4$ & $3$ & $47$ & $-2$ & $1+0-3-4-1$ & chiralny & nie \\
$11n_{121}$ & $2$ & $4$ & $3$ & $45$ & $4$ & $1+5+0-1$ & odwracalny & nie \\
$11n_{122}$ & $1..2$ & $4$ & $3$ & $27$ & $2$ & $1-1-2$ & chiralny & nie \\
$11n_{123}$ & $1$ & $5$ & $3$ & $57$ & $0$ & $1-2+3$ & odwracalny & nie \\
$11n_{124}$ & $1$ & $5$ & $3$ & $59$ & $-2$ & $1-1+0+1$ & chiralny & nie \\
$11n_{125}$ & $1$ & $4$ & $3$ & $63$ & $-2$ & $1+0+0+1$ & odwracalny & nie \\
$11n_{126}$ & $3$ & $4$ & $3$ & $27$ & $-6$ & $1+7+12+3$ & odwracalny & nie \\
$11n_{127}$ & $2$ & $4$ & $3$ & $55$ & $2$ & $1+2+1+1$ & odwracalny & nie \\
$11n_{128}$ & $1..2$ & $4$ & $3$ & $43$ & $-2$ & $1-1+1+1$ & chiralny & nie \\
$11n_{129}$ & $1..2$ & $4$ & $3$ & $43$ & $2$ & $1+3+2+1$ & chiralny & nie \\
$11n_{130}$ & $1$ & $4$ & $3$ & $53$ & $0$ & $1-1-1-1$ & chiralny & nie \\
$11n_{131}$ & $1$ & $4$ & $3$ & $67$ & $2$ & $1+1+0+1$ & chiralny & nie \\
$11n_{132}$ & $2$ & $4$ & $3$ & $25$ & $0$ & $1+2+2$ & chiralny & nie \\
$11n_{133}$ & $3$ & $4$ & $3$ & $25$ & $-4$ & $1+2+2+4+1$ & odwracalny & nie \\
$11n_{134}$ & $1$ & $4$ & $3$ & $47$ & $2$ & $1+0-3$ & chiralny & nie \\
$11n_{135}$ & $2$ & $4$ & $3$ & $5$ & $-4$ & $1-1-4-1$ & odwracalny & nie \\
$11n_{136}$ & $3$ & $4$ & $3$ & $63$ & $-6$ & $1+8+10+3$ & odwracalny & nie \\
$11n_{137}$ & $2..3$ & $4$ & $3$ & $57$ & $4$ & $1+6+1-1$ & odwracalny & nie \\
$11n_{138}$ & $1..2$ & $4$ & $3$ & $15$ & $-2$ & $1-4-2$ & odwracalny & nie \\
$11n_{139}$ & $1..2$ & $5$ & $3$ & $9$ & $0$ & $1-2$ & chiralny & nie \\
$11n_{140}$ & $2$ & $5$ & $3$ & $51$ & $2$ & $1+5-2$ & odwracalny & nie \\
$11n_{141}$ & $1..3$ & $5$ & $3$ & $21$ & $0$ & $1-5$ & odwracalny & nie \\
$11n_{142}$ & $1..2$ & $5$ & $3$ & $33$ & $0$ & $1-4+1$ & odwracalny & nie \\
$11n_{143}$ & $1$ & $4$ & $3$ & $9$ & $0$ & $1-2-4-1$ & odwracalny & nie \\
$11n_{144}$ & $2$ & $4$ & $3$ & $65$ & $-4$ & $1+4+1-1$ & odwracalny & nie \\
$11n_{145}$ & $1$ & $4$ & $3$ & $9$ & $0$ & $1+2+5+1$ & odwracalny & nie \\
$11n_{146}$ & $2$ & $4$ & $3$ & $63$ & $-2$ & $1+4+1+1$ & odwracalny & nie \\
$11n_{147}$ & $2$ & $4$ & $3$ & $37$ & $-4$ & $1+3+3+4+1$ & chiralny & nie \\
$11n_{148}$ & $3$ & $4$ & $3$ & $75$ & $-2$ & $1+3+0-3-1$ & odwracalny & nie \\
$11n_{149}$ & $2$ & $4$ & $3$ & $33$ & $-4$ & $1+0+2+4+1$ & odwracalny & nie \\
$11n_{150}$ & $2$ & $5$ & $3$ & $75$ & $-2$ & $1-1+3+2$ & odwracalny & nie \\
$11n_{151}$ & $1$ & $4$ & $3$ & $23$ & $-2$ & $1-2-2$ & chiralny & nie \\
$11n_{152}$ & $1$ & $4$ & $3$ & $23$ & $-2$ & $1-2-2$ & odwracalny & nie \\
$11n_{153}$ & $1$ & $4$ & $3$ & $57$ & $0$ & $1-2+3+4+1$ & odwracalny & nie \\
$11n_{154}$ & $1$ & $4$ & $3$ & $79$ & $-2$ & $1+0-1+1$ & chiralny & nie \\
$11n_{155}$ & $2$ & $5$ & $3$ & $51$ & $-2$ & $1-3+4+2$ & odwracalny & nie \\
$11n_{156}$ & $1$ & $4$ & $3$ & $77$ & $0$ & $1+1+1-1$ & odwracalny & nie \\
$11n_{157}$ & $2$ & $4$ & $3$ & $65$ & $0$ & $1+0+0-1$ & odwracalny & nie \\
$11n_{158}$ & $2$ & $4$ & $3$ & $45$ & $-4$ & $1+1+3+4+1$ & odwracalny & nie \\
$11n_{159}$ & $1$ & $4$ & $3$ & $71$ & $2$ & $1+2+0+1$ & chiralny & nie \\
$11n_{160}$ & $1..2$ & $4$ & $3$ & $67$ & $-2$ & $1+1+0+1$ & chiralny & nie \\
$11n_{161}$ & $1..2$ & $5$ & $3$ & $63$ & $-2$ & $1+0+4+2$ & odwracalny & nie \\
$11n_{162}$ & $1..2$ & $5$ & $3$ & $55$ & $-2$ & $1+2-3$ & odwracalny & nie \\
$11n_{163}$ & $2$ & $4$ & $3$ & $91$ & $-2$ & $1-1-2+1$ & odwracalny & nie \\
$11n_{164}$ & $2$ & $4$ & $3$ & $45$ & $-4$ & $1+1-1-1$ & odwracalny & nie \\
$11n_{165}$ & $2$ & $5$ & $3$ & $85$ & $0$ & $1-1+1-1$ & odwracalny & nie \\
$11n_{166}$ & $1..2$ & $4$ & $3$ & $59$ & $-2$ & $1-1-4-4-1$ & chiralny & nie \\
$11n_{167}$ & $2$ & $4$ & $3$ & $63$ & $-2$ & $1+4+1+1$ & odwracalny & nie \\
$11n_{168}$ & $2$ & $5$ & $3$ & $75$ & $-2$ & $1+3+0+1$ & odwracalny & nie \\
$11n_{169}$ & $3..4$ & $4$ & $3$ & $35$ & $-6$ & $1+9+12+3$ & odwracalny & nie \\
$11n_{170}$ & $2$ & $5$ & $3$ & $63$ & $2$ & $1+4-3$ & odwracalny & nie \\
$11n_{171}$ & $2..3$ & $5$ & $3$ & $65$ & $-4$ & $1+8+6$ & odwracalny & nie \\
$11n_{172}$ & $1..2$ & $4$ & $3$ & $49$ & $0$ & $1+0-1-1$ & chiralny & nie \\
$11n_{173}$ & $2..3$ & $4$ & $3$ & $45$ & $-4$ & $1+5+4+4+1$ & odwracalny & nie \\
$11n_{174}$ & $2$ & $4$ & $3$ & $97$ & $-4$ & $1+4-1-2$ & chiralny & nie \\
$11n_{175}$ & $2$ & $5$ & $3$ & $65$ & $-4$ & $1+4-3-2$ & odwracalny & nie \\
$11n_{176}$ & $1$ & $4$ & $3$ & $63$ & $2$ & $1+0+0+1$ & chiralny & nie \\
$11n_{177}$ & $1..2$ & $4$ & $3$ & $83$ & $-2$ & $1+1-1-3-1$ & chiralny & nie \\
$11n_{178}$ & $2$ & $4$ & $3$ & $95$ & $-2$ & $1+4+3+2$ & odwracalny & nie \\
$11n_{179}$ & $1..2$ & $4$ & $3$ & $77$ & $0$ & $1+1+1-1$ & odwracalny & nie \\
$11n_{180}$ & $3..4$ & $4$ & $3$ & $55$ & $-6$ & $1+10+11+3$ & odwracalny & nie \\
$11n_{181}$ & $2..3$ & $5$ & $3$ & $45$ & $-4$ & $1+9+5$ & odwracalny & nie \\
$11n_{182}$ & $1..2$ & $4$ & $3$ & $93$ & $0$ & $1-3+1+3+1$ & odwracalny & nie \\
$11n_{183}$ & $3$ & $4$ & $3$ & $21$ & $-4$ & $1+7+7+1$ & odwracalny & nie \\
$11n_{184}$ & $1$ & $4$ & $3$ & $87$ & $-2$ & $1+2+3+2$ & chiralny & nie \\
$11n_{185}$ & $2$ & $4$ & $3$ & $105$ & $-4$ & $1+2-1-2$ & odwracalny & nie \\
$12a_{1}$ & $1$ & $5$ & $3$ & $213$ & $0$ & $1-1+1+1+1$ & chiralny & tak \\
$12a_{2}$ & $1$ & $5$ & $3$ & $171$ & $-2$ & $1-1+1-1-1$ & chiralny & tak \\
$12a_{3}$ & $2$ & $6$ & $3$ & $169$ & $0$ & $1-2+2-2$ & odwracalny & tak \\
$12a_{4}$ & $2$ & $5$ & $3$ & $221$ & $0$ & $1-3+1+1+1$ & -zwierciadlany & tak \\
$12a_{5}$ & $2$ & $6$ & $3$ & $199$ & $-2$ & $1-2-1+3$ & chiralny & tak \\
$12a_{6}$ & $1$ & $5$ & $3$ & $203$ & $-2$ & $1-1-1-1-1$ & chiralny & tak \\
$12a_{7}$ & $2$ & $5$ & $3$ & $251$ & $2$ & $1-1+0+0-1$ & chiralny & tak \\
$12a_{8}$ & $1$ & $5$ & $3$ & $181$ & $0$ & $1-1-1+1+1$ & chiralny & tak \\
$12a_{9}$ & $2$ & $6$ & $3$ & $119$ & $-2$ & $1-2+0+2$ & odwracalny & tak \\
$12a_{10}$ & $2$ & $6$ & $3$ & $207$ & $-2$ & $1+0-1+3$ & chiralny & tak \\
$12a_{11}$ & $2$ & $5$ & $3$ & $165$ & $4$ & $1+3-1+1+1$ & chiralny & tak \\
$12a_{12}$ & $2$ & $6$ & $3$ & $167$ & $2$ & $1+2-2+2$ & odwracalny & tak \\
$12a_{13}$ & $2$ & $5$ & $3$ & $253$ & $0$ & $1+1+0+0+1$ & chiralny & tak \\
$12a_{14}$ & $2$ & $5$ & $3$ & $251$ & $2$ & $1-1+0+0-1$ & chiralny & tak \\
$12a_{15}$ & $2$ & $5$ & $3$ & $253$ & $0$ & $1+1+0+0+1$ & chiralny & tak \\
$12a_{16}$ & $2$ & $5$ & $3$ & $201$ & $0$ & $1-2+0+1+1$ & odwracalny & tak \\
$12a_{17}$ & $2$ & $5$ & $3$ & $223$ & $-2$ & $1+0-2-1-1$ & odwracalny & tak \\
$12a_{18}$ & $2$ & $5$ & $3$ & $151$ & $-2$ & $1-2+2-1-1$ & odwracalny & tak \\
$12a_{19}$ & $1$ & $5$ & $3$ & $191$ & $2$ & $1+0+0-1-1$ & odwracalny & tak \\
$12a_{20}$ & $1$ & $5$ & $3$ & $193$ & $0$ & $1+0+0+1+1$ & odwracalny & tak \\
$12a_{21}$ & $2$ & $5$ & $3$ & $197$ & $-4$ & $1+3-3-4$ & odwracalny & tak \\
$12a_{22}$ & $1$ & $6$ & $3$ & $149$ & $0$ & $1-1+1-2$ & odwracalny & tak \\
$12a_{23}$ & $2$ & $6$ & $3$ & $189$ & $0$ & $1-3+3-2$ & chiralny & tak \\
$12a_{24}$ & $2$ & $5$ & $3$ & $161$ & $-4$ & $1+0-2+1+1$ & odwracalny & tak \\
$12a_{25}$ & $1$ & $6$ & $3$ & $185$ & $0$ & $1-2+3-2$ & odwracalny & tak \\
$12a_{26}$ & $2$ & $5$ & $3$ & $145$ & $4$ & $1+4-2+1+1$ & odwracalny & tak \\
$12a_{27}$ & $1$ & $6$ & $3$ & $187$ & $2$ & $1+3-3+2$ & odwracalny & tak \\
$12a_{28}$ & $1$ & $6$ & $3$ & $175$ & $-2$ & $1+0-3+2$ & odwracalny & tak \\
$12a_{29}$ & $2$ & $5$ & $4$ & $219$ & $2$ & $1-1-2-1-1$ & odwracalny & tak \\
$12a_{30}$ & $2$ & $5$ & $4$ & $189$ & $0$ & $1+1+0+1+1$ & odwracalny & tak \\
$12a_{31}$ & $1$ & $6$ & $3$ & $139$ & $-2$ & $1-1-1+2$ & odwracalny & tak \\
$12a_{32}$ & $2$ & $6$ & $3$ & $143$ & $2$ & $1+0-1+2$ & odwracalny & tak \\
$12a_{33}$ & $2$ & $5$ & $4$ & $189$ & $0$ & $1+1+0+1+1$ & odwracalny & tak \\
$12a_{34}$ & $3$ & $4$ & $3$ & $115$ & $-6$ & $1+5-2-7-2$ & odwracalny & tak \\
$12a_{35}$ & $2..3$ & $5$ & $3$ & $177$ & $-4$ & $1+4-4-4$ & odwracalny & tak \\
$12a_{36}$ & $3$ & $5$ & $4$ & $159$ & $-6$ & $1+4-1-6-2$ & odwracalny & tak \\
$12a_{37}$ & $2..3$ & $5$ & $3$ & $133$ & $-4$ & $1+3-3-3$ & odwracalny & tak \\
$12a_{38}$ & $2$ & $6$ & ${}^{71}{\mskip -5mu/\mskip -3mu}_{33}$ & $71$ & $-2$ & $1+2-4$ & odwracalny & tak \\
$12a_{39}$ & $2$ & $5$ & $3$ & $187$ & $2$ & $1+3+1-1-1$ & chiralny & tak \\
$12a_{40}$ & $1$ & $5$ & $3$ & $211$ & $-2$ & $1+1-1-1-1$ & chiralny & tak \\
$12a_{41}$ & $2..3$ & $4$ & $3$ & $185$ & $4$ & $1+6+9+7+2$ & chiralny & tak \\
$12a_{42}$ & $2$ & $5$ & $3$ & $131$ & $2$ & $1+5+5+3$ & odwracalny & tak \\
$12a_{43}$ & $3$ & $5$ & $3$ & $203$ & $-6$ & $1+7+13+6$ & chiralny & tak \\
$12a_{44}$ & $2..3$ & $6$ & $3$ & $217$ & $-4$ & $1+2-2-4$ & odwracalny & tak \\
$12a_{45}$ & $2..3$ & $5$ & $3$ & $179$ & $2$ & $1+5+6+0-1$ & chiralny & tak \\
$12a_{46}$ & $2$ & $6$ & $3$ & $161$ & $0$ & $1+4+3-2$ & chiralny & tak \\
$12a_{47}$ & $2..3$ & $6$ & $3$ & $193$ & $-4$ & $1+4+1-3$ & chiralny & tak \\
$12a_{48}$ & $2$ & $5$ & $3$ & $261$ & $0$ & $1-1+0+0+1$ & chiralny & tak \\
$12a_{49}$ & $2..3$ & $5$ & $3$ & $231$ & $2$ & $1+6+7+5$ & chiralny & tak \\
$12a_{50}$ & $2..3$ & $5$ & $3$ & $151$ & $2$ & $1+6+8+0-1$ & chiralny & tak \\
$12a_{51}$ & $2$ & $6$ & $3$ & $189$ & $0$ & $1+5+5-2$ & chiralny & tak \\
$12a_{52}$ & $4$ & $4$ & $3$ & $109$ & $-8$ & $1+9+21+15+3$ & chiralny & tak \\
$12a_{53}$ & $3$ & $5$ & $3$ & $175$ & $-6$ & $1+8+15+6$ & chiralny & tak \\
$12a_{54}$ & $2$ & $6$ & $3$ & $169$ & $0$ & $1+2-1-3$ & chiralny & tak \\
$12a_{55}$ & $3$ & $5$ & $3$ & $155$ & $-6$ & $1+7+12+5$ & chiralny & tak \\
$12a_{56}$ & $2$ & $6$ & $3$ & $89$ & $-4$ & $1+6+7$ & odwracalny & tak \\
$12a_{57}$ & $2$ & $6$ & $3$ & $177$ & $0$ & $1+0-1-3$ & chiralny & tak \\
$12a_{58}$ & $1$ & $5$ & $3$ & $205$ & $0$ & $1+1+1+1+1$ & -zwierciadlany & tak \\
$12a_{59}$ & $2$ & $5$ & $3$ & $259$ & $-2$ & $1+1+0+0-1$ & chiralny & tak \\
$12a_{60}$ & $2$ & $5$ & $3$ & $261$ & $0$ & $1-1+0+0+1$ & chiralny & tak \\
$12a_{61}$ & $1$ & $5$ & $3$ & $179$ & $-2$ & $1+1+1-1-1$ & chiralny & tak \\
$12a_{62}$ & $2$ & $6$ & $3$ & $129$ & $0$ & $1+0+0-2$ & odwracalny & tak \\
$12a_{63}$ & $2$ & $5$ & $3$ & $259$ & $-2$ & $1+1+0+0-1$ & chiralny & tak \\
$12a_{64}$ & $2..3$ & $6$ & $3$ & $217$ & $-4$ & $1+2-2-4$ & chiralny & tak \\
$12a_{65}$ & $2..3$ & $5$ & $3$ & $179$ & $2$ & $1+5+6+0-1$ & chiralny & tak \\
$12a_{66}$ & $2$ & $5$ & $3$ & $237$ & $0$ & $1+1+3+1+1$ & chiralny & tak \\
$12a_{67}$ & $2$ & $5$ & $3$ & $219$ & $2$ & $1-1+2+0-1$ & chiralny & tak \\
$12a_{68}$ & $2$ & $6$ & $3$ & $185$ & $0$ & $1-2+3-2$ & chiralny & tak \\
$12a_{69}$ & $2$ & $5$ & $3$ & $239$ & $2$ & $1+0+1+0-1$ & chiralny & tak \\
$12a_{70}$ & $2$ & $5$ & $3$ & $233$ & $4$ & $1+2-1+0+1$ & chiralny & tak \\
$12a_{71}$ & $2$ & $5$ & $3$ & $207$ & $-2$ & $1+0+3+0-1$ & chiralny & tak \\
$12a_{72}$ & $2$ & $6$ & $3$ & $197$ & $0$ & $1-1+4-2$ & chiralny & tak \\
$12a_{73}$ & $1$ & $6$ & $3$ & $211$ & $-2$ & $1+1-5+2$ & chiralny & tak \\
$12a_{74}$ & $2$ & $5$ & $3$ & $255$ & $2$ & $1+0-4-1-1$ & odwracalny & tak \\
$12a_{75}$ & $2..3$ & $5$ & $3$ & $205$ & $-4$ & $1+5-6-5$ & chiralny & tak \\
$12a_{76}$ & $2$ & $6$ & $3$ & $143$ & $-2$ & $1+4-8$ & chiralny & tak \\
$12a_{77}$ & $2$ & $5$ & $3$ & $225$ & $0$ & $1+0+2+1+1$ & odwracalny & tak \\
$12a_{78}$ & $2..3$ & $4$ & $3$ & $149$ & $4$ & $1+7+11+8+2$ & odwracalny & tak \\
$12a_{79}$ & $2$ & $5$ & $3$ & $201$ & $0$ & $1-2+0+1+1$ & odwracalny & tak \\
$12a_{80}$ & $1$ & $5$ & $3$ & $173$ & $0$ & $1+1-1+1+1$ & chiralny & tak \\
$12a_{81}$ & $2$ & $6$ & $3$ & $159$ & $-2$ & $1+0-2+2$ & odwracalny & tak \\
$12a_{82}$ & $3$ & $5$ & $3$ & $191$ & $-6$ & $1+8+14+6$ & odwracalny & tak \\
$12a_{83}$ & $2$ & $6$ & $3$ & $205$ & $-4$ & $1+5+2-3$ & odwracalny & tak \\
$12a_{84}$ & $2$ & $5$ & $3$ & $199$ & $2$ & $1+2+0-1-1$ & odwracalny & tak \\
$12a_{85}$ & $2..3$ & $4$ & $3$ & $145$ & $4$ & $1+8+11+8+2$ & odwracalny & tak \\
$12a_{86}$ & $2..3$ & $5$ & $3$ & $171$ & $2$ & $1+7+7+4$ & odwracalny & tak \\
$12a_{87}$ & $2$ & $5$ & $3$ & $175$ & $2$ & $1+4+2-1-1$ & odwracalny & tak \\
$12a_{88}$ & $2$ & $5$ & $3$ & $291$ & $-2$ & $1+1-2+0-1$ & chiralny & tak \\
$12a_{89}$ & $2..3$ & $5$ & $3$ & $167$ & $2$ & $1+6+7+4$ & chiralny & tak \\
$12a_{90}$ & $2$ & $5$ & $3$ & $223$ & $-2$ & $1+0-2-1-1$ & chiralny & tak \\
$12a_{91}$ & $2$ & $5$ & $3$ & $221$ & $-4$ & $1+1-2+0+1$ & chiralny & tak \\
$12a_{92}$ & $2$ & $5$ & $3$ & $191$ & $2$ & $1+4+5+0-1$ & chiralny & tak \\
$12a_{93}$ & $4$ & $4$ & $3$ & $121$ & $-8$ & $1+10+22+15+3$ & chiralny & tak \\
$12a_{94}$ & $3..4$ & $5$ & $3$ & $163$ & $-6$ & $1+9+16+6$ & chiralny & tak \\
$12a_{95}$ & $2$ & $6$ & $3$ & $195$ & $-2$ & $1+1+0+3$ & odwracalny & tak \\
$12a_{96}$ & $3$ & $5$ & $3$ & $115$ & $-6$ & $1+9+15+5$ & odwracalny & tak \\
$12a_{97}$ & $2..3$ & $6$ & $3$ & $129$ & $-4$ & $1+8+10$ & odwracalny & tak \\
$12a_{98}$ & $2$ & $5$ & $3$ & $191$ & $-2$ & $1+0+0-1-1$ & odwracalny & tak \\
$12a_{99}$ & $2$ & $5$ & $3$ & $217$ & $0$ & $1+2+2+1+1$ & odwracalny & tak \\
$12a_{100}$ & $2$ & $6$ & $3$ & $225$ & $0$ & $1+4+3-3$ & chiralny & tak \\
$12a_{101}$ & $2$ & $5$ & $3$ & $229$ & $0$ & $1+3+3+1+1$ & chiralny & tak \\
$12a_{102}$ & $3..4$ & $5$ & $3$ & $239$ & $-6$ & $1+8+15+7$ & chiralny & tak \\
$12a_{103}$ & $2..3$ & $6$ & $3$ & $219$ & $-2$ & $1-1-2+3$ & odwracalny & tak \\
$12a_{104}$ & $2..3$ & $5$ & $3$ & $163$ & $2$ & $1+5+7+4$ & odwracalny & tak \\
$12a_{105}$ & $3$ & $5$ & $3$ & $131$ & $-6$ & $1+5+5-1-1$ & chiralny & tak \\
$12a_{106}$ & $2$ & $6$ & $3$ & $145$ & $-4$ & $1+4+2-2$ & odwracalny & tak \\
$12a_{107}$ & $3..4$ & $5$ & $3$ & $239$ & $-6$ & $1+8+15+7$ & chiralny & tak \\
$12a_{108}$ & $2$ & $5$ & $3$ & $235$ & $-2$ & $1+3+2+0-1$ & chiralny & tak \\
$12a_{109}$ & $2$ & $6$ & $3$ & $149$ & $0$ & $1+3+2-2$ & chiralny & tak \\
$12a_{110}$ & $2$ & $6$ & $3$ & $181$ & $-4$ & $1+3+0-3$ & odwracalny & tak \\
$12a_{111}$ & $2$ & $5$ & $3$ & $221$ & $-4$ & $1+1-2+0+1$ & chiralny & tak \\
$12a_{112}$ & $2$ & $5$ & $3$ & $167$ & $2$ & $1+2+2-1-1$ & odwracalny & tak \\
$12a_{113}$ & $2$ & $5$ & $4$ & $219$ & $-2$ & $1-1-2-1-1$ & odwracalny & tak \\
$12a_{114}$ & $2..3$ & $6$ & $4$ & $201$ & $-4$ & $1+2-3-4$ & odwracalny & tak \\
$12a_{115}$ & $2$ & $5$ & $3$ & $229$ & $0$ & $1+3+3+1+1$ & chiralny & tak \\
$12a_{116}$ & $2$ & $5$ & $4$ & $171$ & $2$ & $1+3+2-1-1$ & odwracalny & tak \\
$12a_{117}$ & $2..3$ & $6$ & $4$ & $201$ & $-4$ & $1+2-3-4$ & odwracalny & tak \\
$12a_{118}$ & $2$ & $6$ & $3$ & $137$ & $0$ & $1+2+1-2$ & odwracalny & tak \\
$12a_{119}$ & $2$ & $5$ & $4$ & $189$ & $-4$ & $1+1+0+1+1$ & odwracalny & tak \\
$12a_{120}$ & $2$ & $5$ & $3$ & $235$ & $-2$ & $1+3+2+0-1$ & chiralny & tak \\
$12a_{121}$ & $1$ & $6$ & $3$ & $141$ & $0$ & $1+1+1-2$ & odwracalny & tak \\
$12a_{122}$ & $2$ & $5$ & $4$ & $171$ & $2$ & $1+3+2-1-1$ & odwracalny & tak \\
$12a_{123}$ & $2$ & $5$ & $3$ & $149$ & $4$ & $1-1-3+1+1$ & chiralny & tak \\
$12a_{124}$ & $2$ & $6$ & $3$ & $151$ & $2$ & $1-2-2+2$ & odwracalny & tak \\
$12a_{125}$ & $2..3$ & $7$ & $3$ & $181$ & $0$ & $1-5+6-1$ & -zwierciadlany & tak \\
$12a_{126}$ & $2$ & $5$ & $3$ & $229$ & $0$ & $1-1-2+0+1$ & chiralny & tak \\
$12a_{127}$ & $2..3$ & $6$ & $3$ & $159$ & $-2$ & $1-4+1+3$ & chiralny & tak \\
$12a_{128}$ & $2$ & $7$ & $3$ & $101$ & $0$ & $1-5+5$ & odwracalny & tak \\
$12a_{129}$ & $2$ & $5$ & $3$ & $155$ & $-2$ & $1+3+3-1-1$ & chiralny & tak \\
$12a_{130}$ & $2$ & $6$ & $3$ & $153$ & $0$ & $1+2+2-2$ & odwracalny & tak \\
$12a_{131}$ & $2$ & $5$ & $3$ & $227$ & $2$ & $1+1+2+0-1$ & chiralny & tak \\
$12a_{132}$ & $2$ & $5$ & $3$ & $229$ & $0$ & $1-1-2+0+1$ & chiralny & tak \\
$12a_{133}$ & $2$ & $5$ & $3$ & $227$ & $2$ & $1+1+2+0-1$ & chiralny & tak \\
$12a_{134}$ & $2$ & $5$ & $3$ & $221$ & $0$ & $1+1-2+0+1$ & chiralny & tak \\
$12a_{135}$ & $2$ & $6$ & $3$ & $175$ & $2$ & $1+0-3+2$ & chiralny & tak \\
$12a_{136}$ & $2$ & $5$ & $3$ & $219$ & $2$ & $1-1+2+0-1$ & chiralny & tak \\
$12a_{137}$ & $2$ & $5$ & $3$ & $215$ & $2$ & $1-2+2+0-1$ & chiralny & tak \\
$12a_{138}$ & $1$ & $5$ & $3$ & $223$ & $-2$ & $1+0+2+0-1$ & chiralny & tak \\
$12a_{139}$ & $2$ & $5$ & $3$ & $257$ & $0$ & $1+0+0+0+1$ & chiralny & tak \\
$12a_{140}$ & $2$ & $6$ & $3$ & $181$ & $0$ & $1-1+3-2$ & chiralny & tak \\
$12a_{141}$ & $2$ & $5$ & $3$ & $161$ & $0$ & $1+0-2+1+1$ & odwracalny & tak \\
$12a_{142}$ & $2$ & $5$ & $3$ & $185$ & $-4$ & $1+2+0+1+1$ & odwracalny & tak \\
$12a_{143}$ & $4$ & $4$ & $3$ & $85$ & $-8$ & $1+11+24+16+3$ & odwracalny & tak \\
$12a_{144}$ & $3..4$ & $5$ & $3$ & $151$ & $-6$ & $1+10+17+6$ & odwracalny & tak \\
$12a_{145}$ & $3..4$ & $5$ & $3$ & $179$ & $-6$ & $1+9+15+6$ & odwracalny & tak \\
$12a_{146}$ & $3..4$ & $3$ & $3$ & $55$ & $6$ & $1+10+19+17+7+1$ & odwracalny & tak \\
$12a_{147}$ & $2..4$ & $4$ & $3$ & $109$ & $4$ & $1+9+13+9+2$ & odwracalny & tak \\
$12a_{148}$ & $2..4$ & $4$ & $3$ & $161$ & $4$ & $1+8+12+8+2$ & odwracalny & tak \\
$12a_{149}$ & $2$ & $5$ & $3$ & $199$ & $-2$ & $1+2+0-1-1$ & odwracalny & tak \\
$12a_{150}$ & $2..4$ & $5$ & $3$ & $159$ & $2$ & $1+8+8+4$ & odwracalny & tak \\
$12a_{151}$ & $1$ & $6$ & $3$ & $147$ & $-2$ & $1+1-1+2$ & odwracalny & tak \\
$12a_{152}$ & $2..3$ & $6$ & $3$ & $113$ & $-4$ & $1+8+9$ & odwracalny & tak \\
$12a_{153}$ & $2..3$ & $5$ & $3$ & $107$ & $2$ & $1+7+7+3$ & odwracalny & tak \\
$12a_{154}$ & $2$ & $6$ & $3$ & $241$ & $-4$ & $1+4+0-4$ & chiralny & tak \\
$12a_{155}$ & $2$ & $6$ & $3$ & $171$ & $-2$ & $1-1-3+2$ & chiralny & tak \\
$12a_{156}$ & $2..4$ & $6$ & $3$ & $141$ & $-4$ & $1+9+11$ & odwracalny & tak \\
$12a_{157}$ & $2$ & $5$ & $4$ & $189$ & $0$ & $1+1+0+1+1$ & odwracalny & tak \\
$12a_{158}$ & $3$ & $5$ & $3$ & $119$ & $-6$ & $1+6+6-1-1$ & odwracalny & tak \\
$12a_{159}$ & $2$ & $6$ & $3$ & $157$ & $-4$ & $1+5+3-2$ & odwracalny & tak \\
$12a_{160}$ & $2..4$ & $4$ & $3$ & $101$ & $4$ & $1+7+12+9+2$ & odwracalny & tak \\
$12a_{161}$ & $2..3$ & $5$ & $3$ & $151$ & $2$ & $1+6+8+4$ & odwracalny & tak \\
$12a_{162}$ & $2$ & $6$ & $3$ & $241$ & $-4$ & $1+4+0-4$ & chiralny & tak \\
$12a_{163}$ & $2$ & $6$ & $3$ & $231$ & $-2$ & $1+2+2+4$ & chiralny & tak \\
$12a_{164}$ & $2$ & $5$ & $4$ & $171$ & $-2$ & $1+3+2-1-1$ & odwracalny & tak \\
$12a_{165}$ & $2$ & $6$ & $3$ & $105$ & $-4$ & $1+2-1-2$ & odwracalny & tak \\
$12a_{166}$ & $2$ & $5$ & $4$ & $171$ & $-2$ & $1+3+2-1-1$ & odwracalny & tak \\
$12a_{167}$ & $2..4$ & $4$ & $4$ & $153$ & $4$ & $1+6+11+8+2$ & odwracalny & tak \\
$12a_{168}$ & $2..3$ & $5$ & $3$ & $99$ & $2$ & $1+5+7+3$ & odwracalny & tak \\
$12a_{169}$ & $2$ & $6$ & ${}^{49}{\mskip -5mu/\mskip -3mu}_{23}$ & $49$ & $0$ & $1+4+4$ & odwracalny & tak \\
$12a_{170}$ & $2$ & $6$ & $3$ & $179$ & $-2$ & $1-3+0+3$ & odwracalny & tak \\
$12a_{171}$ & $2$ & $6$ & $3$ & $197$ & $0$ & $1-1+0-3$ & odwracalny & tak \\
$12a_{172}$ & $2$ & $5$ & $3$ & $129$ & $4$ & $1+0-4+1+1$ & odwracalny & tak \\
$12a_{173}$ & $1$ & $5$ & $3$ & $169$ & $0$ & $1-2-2+1+1$ & odwracalny & tak \\
$12a_{174}$ & $1$ & $5$ & $3$ & $167$ & $2$ & $1+2+2-1-1$ & odwracalny & tak \\
$12a_{175}$ & $2$ & $6$ & $3$ & $183$ & $2$ & $1-2-4+2$ & odwracalny & tak \\
$12a_{176}$ & $2$ & $6$ & $3$ & $131$ & $2$ & $1-3-1+2$ & odwracalny & tak \\
$12a_{177}$ & $2$ & $6$ & $3$ & $171$ & $2$ & $1-1-3+2$ & chiralny & tak \\
$12a_{178}$ & $2..3$ & $6$ & $3$ & $139$ & $2$ & $1-5+2+3$ & odwracalny & tak \\
$12a_{179}$ & $2$ & $5$ & $3$ & $135$ & $-2$ & $1+2+4-1-1$ & odwracalny & tak \\
$12a_{180}$ & $1$ & $6$ & $3$ & $173$ & $0$ & $1+1+3-2$ & odwracalny & tak \\
$12a_{181}$ & $2..3$ & $7$ & $3$ & $165$ & $0$ & $1-5+5-1$ & odwracalny & tak \\
$12a_{182}$ & $2$ & $5$ & $4$ & $171$ & $2$ & $1+3+2-1-1$ & odwracalny & tak \\
$12a_{183}$ & $2..3$ & $7$ & $3$ & $121$ & $0$ & $1-6+6$ & odwracalny & tak \\
$12a_{184}$ & $1$ & $5$ & $3$ & $241$ & $0$ & $1+0-1+0+1$ & chiralny & tak \\
$12a_{185}$ & $2$ & $5$ & $3$ & $247$ & $-2$ & $1+2+1+0-1$ & chiralny & tak \\
$12a_{186}$ & $2..3$ & $5$ & $3$ & $193$ & $4$ & $1+4-3+0+1$ & chiralny & tak \\
$12a_{187}$ & $2$ & $6$ & $3$ & $203$ & $2$ & $1+3-4+2$ & chiralny & tak \\
$12a_{188}$ & $2$ & $5$ & $3$ & $221$ & $0$ & $1+1-2+0+1$ & chiralny & tak \\
$12a_{189}$ & $2$ & $5$ & $3$ & $225$ & $0$ & $1+0-2+0+1$ & chiralny & tak \\
$12a_{190}$ & $2$ & $5$ & $3$ & $217$ & $4$ & $1+2-2+0+1$ & chiralny & tak \\
$12a_{191}$ & $1$ & $5$ & $3$ & $263$ & $2$ & $1+2+0+0-1$ & chiralny & tak \\
$12a_{192}$ & $2$ & $6$ & $3$ & $179$ & $2$ & $1+1-3+2$ & chiralny & tak \\
$12a_{193}$ & $2..3$ & $5$ & $3$ & $105$ & $4$ & $1-2-6+1+1$ & odwracalny & tak \\
$12a_{194}$ & $2$ & $6$ & $3$ & $163$ & $2$ & $1-3-3+2$ & odwracalny & tak \\
$12a_{195}$ & $2..3$ & $5$ & $4$ & $141$ & $4$ & $1-3-4+1+1$ & odwracalny & tak \\
$12a_{196}$ & $2..3$ & $6$ & $3$ & $127$ & $2$ & $1-4-1+2$ & odwracalny & tak \\
$12a_{197}$ & $2$ & $7$ & ${}^{69}{\mskip -5mu/\mskip -3mu}_{32}$ & $69$ & $0$ & $1-5+3$ & odwracalny & tak \\
$12a_{198}$ & $2$ & $6$ & $3$ & $209$ & $0$ & $1-4+4-2$ & chiralny & tak \\
$12a_{199}$ & $2..3$ & $5$ & $3$ & $245$ & $-4$ & $1+3-4-5$ & odwracalny & tak \\
$12a_{200}$ & $2$ & $5$ & $3$ & $215$ & $2$ & $1-2-2-1-1$ & chiralny & tak \\
$12a_{201}$ & $2$ & $6$ & $3$ & $157$ & $0$ & $1-3+1-2$ & chiralny & tak \\
$12a_{202}$ & $1$ & $5$ & $3$ & $265$ & $0$ & $1-2+0+0+1$ & chiralny & tak \\
$12a_{203}$ & $2$ & $5$ & $3$ & $183$ & $2$ & $1-2+0-1-1$ & odwracalny & tak \\
$12a_{204}$ & $2$ & $6$ & ${}^{173}{\mskip -5mu/\mskip -3mu}_{76}$ & $173$ & $0$ & $1-3+2-2$ & odwracalny & tak \\
$12a_{205}$ & $3$ & $4$ & $3$ & $139$ & $-6$ & $1+3-4-7-2$ & odwracalny & tak \\
$12a_{206}$ & $2$ & $5$ & ${}^{105}{\mskip -5mu/\mskip -3mu}_{47}$ & $105$ & $-4$ & $1+2-5-3$ & odwracalny & tak \\
$12a_{207}$ & $2$ & $5$ & $3$ & $159$ & $-2$ & $1+0+2-1-1$ & odwracalny & tak \\
$12a_{208}$ & $1$ & $6$ & $3$ & $165$ & $0$ & $1-1+2-2$ & odwracalny & tak \\
$12a_{209}$ & $1$ & $5$ & $3$ & $273$ & $0$ & $1+0+1+0+1$ & odwracalny & tak \\
$12a_{210}$ & $2$ & $5$ & $3$ & $189$ & $-4$ & $1+1-4-4$ & odwracalny & tak \\
$12a_{211}$ & $2..3$ & $4$ & $3$ & $169$ & $0$ & $1+6+12+8+2$ & chiralny & tak \\
$12a_{212}$ & $2..3$ & $5$ & $3$ & $147$ & $-2$ & $1+5+8+4$ & odwracalny & tak \\
$12a_{213}$ & $2$ & $5$ & $3$ & $247$ & $-2$ & $1-2+0+0-1$ & chiralny & tak \\
$12a_{214}$ & $1$ & $5$ & $3$ & $257$ & $0$ & $1+0+0+0+1$ & chiralny & tak \\
$12a_{215}$ & $2$ & $5$ & $3$ & $195$ & $2$ & $1+1+0-1-1$ & odwracalny & tak \\
$12a_{216}$ & $2$ & $5$ & $3$ & $125$ & $-4$ & $1+1-4+1+1$ & odwracalny & tak \\
$12a_{217}$ & $1$ & $5$ & $3$ & $197$ & $0$ & $1-1+0+1+1$ & odwracalny & tak \\
$12a_{218}$ & $2$ & $6$ & $3$ & $175$ & $-2$ & $1+0-3+2$ & odwracalny & tak \\
$12a_{219}$ & $2$ & $5$ & $3$ & $181$ & $4$ & $1+3+0+1+1$ & odwracalny & tak \\
$12a_{220}$ & $1$ & $5$ & $3$ & $187$ & $-2$ & $1-1+0-1-1$ & odwracalny & tak \\
$12a_{221}$ & $2$ & $6$ & ${}^{169}{\mskip -5mu/\mskip -3mu}_{66}$ & $169$ & $0$ & $1-2+2-2$ & odwracalny & tak \\
$12a_{222}$ & $2$ & $5$ & $3$ & $281$ & $0$ & $1-2+1+0+1$ & chiralny & tak \\
$12a_{223}$ & $2$ & $6$ & $3$ & $189$ & $0$ & $1+1+0-3$ & odwracalny & tak \\
$12a_{224}$ & $2$ & $5$ & $3$ & $231$ & $-2$ & $1-2+1+0-1$ & chiralny & tak \\
$12a_{225}$ & $2$ & $5$ & $3$ & $227$ & $2$ & $1+1-2-1-1$ & odwracalny & tak \\
$12a_{226}$ & $2$ & $5$ & ${}^{181}{\mskip -5mu/\mskip -3mu}_{75}$ & $181$ & $-4$ & $1+3-4-4$ & odwracalny & tak \\
$12a_{227}$ & $2$ & $5$ & $3$ & $245$ & $-4$ & $1+3-4-5$ & chiralny & tak \\
$12a_{228}$ & $2$ & $5$ & $3$ & $223$ & $2$ & $1+0-2-1-1$ & chiralny & tak \\
$12a_{229}$ & $2$ & $6$ & $3$ & $201$ & $0$ & $1-2+4-2$ & odwracalny & tak \\
$12a_{230}$ & $2$ & $6$ & $3$ & $235$ & $2$ & $1+3-2+3$ & odwracalny & tak \\
$12a_{231}$ & $2..3$ & $5$ & $3$ & $213$ & $-4$ & $1+3-6-5$ & odwracalny & tak \\
$12a_{232}$ & $2$ & $5$ & $3$ & $185$ & $-4$ & $1+2-4-4$ & odwracalny & tak \\
$12a_{233}$ & $2$ & $5$ & $3$ & $249$ & $0$ & $1-2-1+0+1$ & chiralny & tak \\
$12a_{234}$ & $2$ & $4$ & $3$ & $141$ & $0$ & $1+5+10+8+2$ & odwracalny & tak \\
$12a_{235}$ & $2$ & $5$ & $3$ & $175$ & $-2$ & $1+4+6+4$ & odwracalny & tak \\
$12a_{236}$ & $3$ & $4$ & $3$ & $99$ & $-6$ & $1+1-6-8-2$ & odwracalny & tak \\
$12a_{237}$ & $2$ & $5$ & $3$ & $161$ & $-4$ & $1+0-6-4$ & odwracalny & tak \\
$12a_{238}$ & $2..3$ & $5$ & $3$ & $149$ & $-4$ & $1-1-7-4$ & odwracalny & tak \\
$12a_{239}$ & $2$ & $6$ & ${}^{87}{\mskip -5mu/\mskip -3mu}_{40}$ & $87$ & $-2$ & $1-2-6$ & odwracalny & tak \\
$12a_{240}$ & $2..3$ & $4$ & $3$ & $173$ & $4$ & $1+5+8+7+2$ & odwracalny & tak \\
$12a_{241}$ & $2$ & $5$ & ${}^{127}{\mskip -5mu/\mskip -3mu}_{57}$ & $127$ & $2$ & $1+4+5+3$ & odwracalny & tak \\
$12a_{242}$ & $2$ & $5$ & $3$ & $191$ & $-2$ & $1+0+0-1-1$ & odwracalny & tak \\
$12a_{243}$ & $2$ & $6$ & ${}^{133}{\mskip -5mu/\mskip -3mu}_{60}$ & $133$ & $0$ & $1-1+0-2$ & odwracalny & tak \\
$12a_{244}$ & $2..3$ & $5$ & $3$ & $243$ & $2$ & $1+5+6+5$ & odwracalny & tak \\
$12a_{245}$ & $2$ & $5$ & $3$ & $225$ & $0$ & $1+0+2+1+1$ & odwracalny & tak \\
$12a_{246}$ & $2$ & $5$ & $3$ & $153$ & $-4$ & $1+2-2+1+1$ & odwracalny & tak \\
$12a_{247}$ & $2$ & $6$ & ${}^{163}{\mskip -5mu/\mskip -3mu}_{71}$ & $163$ & $-2$ & $1+1-2+2$ & odwracalny & tak \\
$12a_{248}$ & $2$ & $6$ & $3$ & $145$ & $0$ & $1+0+1-2$ & odwracalny & tak \\
$12a_{249}$ & $2$ & $6$ & $3$ & $151$ & $2$ & $1+2-1+2$ & odwracalny & tak \\
$12a_{250}$ & $2$ & $5$ & $3$ & $157$ & $-4$ & $1+1-2+1+1$ & odwracalny & tak \\
$12a_{251}$ & $2$ & $6$ & ${}^{159}{\mskip -5mu/\mskip -3mu}_{59}$ & $159$ & $-2$ & $1+0-2+2$ & odwracalny & tak \\
$12a_{252}$ & $3$ & $4$ & $3$ & $131$ & $-6$ & $1+5-3-7-2$ & odwracalny & tak \\
$12a_{253}$ & $2$ & $5$ & $3$ & $161$ & $-4$ & $1+4-5-4$ & odwracalny & tak \\
$12a_{254}$ & $2..3$ & $5$ & ${}^{97}{\mskip -5mu/\mskip -3mu}_{23}$ & $97$ & $-4$ & $1+4-5-3$ & odwracalny & tak \\
$12a_{255}$ & $2$ & $6$ & ${}^{107}{\mskip -5mu/\mskip -3mu}_{28}$ & $107$ & $-2$ & $1+3-6$ & odwracalny & tak \\
$12a_{256}$ & $1$ & $6$ & $3$ & $219$ & $-2$ & $1-1-2+3$ & chiralny & tak \\
$12a_{257}$ & $2$ & $6$ & ${}^{191}{\mskip -5mu/\mskip -3mu}_{80}$ & $191$ & $-2$ & $1+0-4+2$ & odwracalny & tak \\
$12a_{258}$ & $2$ & $5$ & $3$ & $169$ & $0$ & $1-2-2+1+1$ & odwracalny & tak \\
$12a_{259}$ & $2$ & $6$ & ${}^{115}{\mskip -5mu/\mskip -3mu}_{52}$ & $115$ & $-2$ & $1-3+0+2$ & odwracalny & tak \\
$12a_{260}$ & $2$ & $5$ & $3$ & $183$ & $-2$ & $1-2+0-1-1$ & odwracalny & tak \\
$12a_{261}$ & $1$ & $5$ & $3$ & $241$ & $0$ & $1+0-1+0+1$ & chiralny & tak \\
$12a_{262}$ & $2$ & $5$ & $3$ & $191$ & $-2$ & $1+0+0-1-1$ & chiralny & tak \\
$12a_{263}$ & $2$ & $6$ & $3$ & $203$ & $-2$ & $1-1-1+3$ & odwracalny & tak \\
$12a_{264}$ & $2$ & $5$ & $3$ & $295$ & $-2$ & $1-2+1+1-1$ & odwracalny & tak \\
$12a_{265}$ & $2$ & $5$ & $3$ & $297$ & $0$ & $1-2+2+0+1$ & odwracalny & tak \\
$12a_{266}$ & $2$ & $4$ & $3$ & $221$ & $-4$ & $1+1+2+5+2$ & chiralny & tak \\
$12a_{267}$ & $2$ & $5$ & $3$ & $223$ & $-2$ & $1+0+2+4$ & chiralny & tak \\
$12a_{268}$ & $2$ & $5$ & $3$ & $277$ & $0$ & $1+3+2+0+1$ & -zwierciadlany & tak \\
$12a_{269}$ & $2..3$ & $5$ & $3$ & $153$ & $-4$ & $1+2-6-4$ & chiralny & tak \\
$12a_{270}$ & $2$ & $6$ & $3$ & $99$ & $-2$ & $1+1-6$ & chiralny & tak \\
$12a_{271}$ & $2$ & $5$ & $3$ & $185$ & $0$ & $1+2+0+1+1$ & chiralny & tak \\
$12a_{272}$ & $1$ & $6$ & $3$ & $163$ & $2$ & $1+1-2+2$ & chiralny & tak \\
$12a_{273}$ & $2$ & $5$ & $3$ & $205$ & $0$ & $1+1-3-4$ & -zwierciadlany & tak \\
$12a_{274}$ & $2$ & $4$ & $3$ & $137$ & $4$ & $1+6+10+8+2$ & odwracalny & tak \\
$12a_{275}$ & $2$ & $5$ & $3$ & $179$ & $2$ & $1+5+6+4$ & odwracalny & tak \\
$12a_{276}$ & $4$ & $4$ & $3$ & $157$ & $-8$ & $1+9+20+14+3$ & chiralny & tak \\
$12a_{277}$ & $3$ & $5$ & $3$ & $143$ & $-6$ & $1+8+13+5$ & odwracalny & tak \\
$12a_{278}$ & $2$ & $5$ & $3$ & $139$ & $2$ & $1+3+4-1-1$ & odwracalny & tak \\
$12a_{279}$ & $2$ & $6$ & $3$ & $169$ & $0$ & $1+2+3-2$ & odwracalny & tak \\
$12a_{280}$ & $1$ & $5$ & $3$ & $195$ & $-2$ & $1+1+0-1-1$ & odwracalny & tak \\
$12a_{281}$ & $2$ & $5$ & $3$ & $213$ & $-4$ & $1+3+2+1+1$ & odwracalny & tak \\
$12a_{282}$ & $1$ & $5$ & $3$ & $233$ & $0$ & $1-2-2+0+1$ & chiralny & tak \\
$12a_{283}$ & $2$ & $5$ & $3$ & $223$ & $2$ & $1+0+2+0-1$ & chiralny & tak \\
$12a_{284}$ & $2$ & $5$ & $3$ & $203$ & $-2$ & $1-1+3+0-1$ & chiralny & tak \\
$12a_{285}$ & $2$ & $5$ & $3$ & $227$ & $2$ & $1-3+1+0-1$ & chiralny & tak \\
$12a_{286}$ & $2$ & $6$ & $3$ & $233$ & $0$ & $1-2+2-3$ & chiralny & tak \\
$12a_{287}$ & $1$ & $5$ & $3$ & $243$ & $2$ & $1+1+1+0-1$ & chiralny & tak \\
$12a_{288}$ & $2$ & $5$ & $3$ & $255$ & $2$ & $1+0+0+0-1$ & chiralny & tak \\
$12a_{289}$ & $2..3$ & $5$ & $3$ & $193$ & $-4$ & $1+4-7-5$ & chiralny & tak \\
$12a_{290}$ & $1$ & $4$ & $3$ & $171$ & $-2$ & $1-1-7-7-2$ & chiralny & tak \\
$12a_{291}$ & $2$ & $5$ & $3$ & $153$ & $0$ & $1-2-7-4$ & chiralny & tak \\
$12a_{292}$ & $2$ & $5$ & $3$ & $219$ & $2$ & $1+3+3+0-1$ & chiralny & tak \\
$12a_{293}$ & $3$ & $5$ & $3$ & $227$ & $-6$ & $1+9+16+7$ & chiralny & tak \\
$12a_{294}$ & $2..3$ & $4$ & $3$ & $197$ & $4$ & $1+7+10+7+2$ & chiralny & tak \\
$12a_{295}$ & $3$ & $5$ & $3$ & $207$ & $-6$ & $1+8+13+6$ & odwracalny & tak \\
$12a_{296}$ & $2$ & $6$ & $3$ & $231$ & $-2$ & $1-2-3+3$ & chiralny & tak \\
$12a_{297}$ & $2..3$ & $4$ & $3$ & $153$ & $4$ & $1+6+11+8+2$ & odwracalny & tak \\
$12a_{298}$ & $2$ & $5$ & $3$ & $171$ & $2$ & $1+3+2-1-1$ & odwracalny & tak \\
$12a_{299}$ & $2$ & $5$ & $3$ & $161$ & $-4$ & $1+0-2+1+1$ & odwracalny & tak \\
$12a_{300}$ & $1$ & $6$ & ${}^{155}{\mskip -5mu/\mskip -3mu}_{68}$ & $155$ & $-2$ & $1-1-2+2$ & odwracalny & tak \\
$12a_{301}$ & $1$ & $6$ & $3$ & $221$ & $0$ & $1+1+2-3$ & chiralny & tak \\
$12a_{302}$ & $2..3$ & $5$ & ${}^{147}{\mskip -5mu/\mskip -3mu}_{61}$ & $147$ & $2$ & $1+5+8+4$ & odwracalny & tak \\
$12a_{303}$ & $1$ & $6$ & ${}^{153}{\mskip -5mu/\mskip -3mu}_{64}$ & $153$ & $0$ & $1+2+2-2$ & odwracalny & tak \\
$12a_{304}$ & $2..3$ & $5$ & $3$ & $137$ & $4$ & $1-2-4+1+1$ & odwracalny & tak \\
$12a_{305}$ & $2$ & $5$ & $3$ & $167$ & $-2$ & $1+2+2-1-1$ & odwracalny & tak \\
$12a_{306}$ & $2$ & $6$ & ${}^{147}{\mskip -5mu/\mskip -3mu}_{64}$ & $147$ & $2$ & $1-3-2+2$ & odwracalny & tak \\
$12a_{307}$ & $2$ & $6$ & ${}^{157}{\mskip -5mu/\mskip -3mu}_{69}$ & $157$ & $0$ & $1+1+2-2$ & odwracalny & tak \\
$12a_{308}$ & $1$ & $6$ & $3$ & $201$ & $0$ & $1-2+4-2$ & chiralny & tak \\
$12a_{309}$ & $1$ & $6$ & $3$ & $245$ & $0$ & $1-1+3-3$ & chiralny & tak \\
$12a_{310}$ & $1$ & $5$ & $3$ & $245$ & $0$ & $1-1-1+0+1$ & chiralny & tak \\
$12a_{311}$ & $3$ & $5$ & $3$ & $189$ & $-4$ & $1+5-7-5$ & odwracalny & tak \\
$12a_{312}$ & $2$ & $6$ & $3$ & $159$ & $-2$ & $1+4-9$ & odwracalny & tak \\
$12a_{313}$ & $2$ & $5$ & $3$ & $207$ & $-2$ & $1+4+4+4$ & chiralny & tak \\
$12a_{314}$ & $2$ & $6$ & $3$ & $217$ & $-4$ & $1+6+3-3$ & chiralny & tak \\
$12a_{315}$ & $2..3$ & $5$ & $3$ & $221$ & $-4$ & $1+1-6-5$ & chiralny & tak \\
$12a_{316}$ & $1$ & $5$ & $3$ & $281$ & $0$ & $1-2+1+0+1$ & chiralny & tak \\
$12a_{317}$ & $2$ & $5$ & $3$ & $259$ & $2$ & $1+5+5+5$ & chiralny & tak \\
$12a_{318}$ & $2$ & $5$ & $3$ & $199$ & $2$ & $1+2+0-1-1$ & chiralny & tak \\
$12a_{319}$ & $3..4$ & $5$ & $3$ & $183$ & $-6$ & $1+10+19+7$ & chiralny & tak \\
$12a_{320}$ & $2..3$ & $6$ & $3$ & $157$ & $-4$ & $1+9+12$ & chiralny & tak \\
$12a_{321}$ & $3$ & $4$ & $3$ & $171$ & $-6$ & $1+3-2-6-2$ & chiralny & tak \\
$12a_{322}$ & $2$ & $5$ & $3$ & $185$ & $-4$ & $1+2-4-4$ & chiralny & tak \\
$12a_{323}$ & $2$ & $5$ & $3$ & $253$ & $0$ & $1+1+0+0+1$ & odwracalny & tak \\
$12a_{324}$ & $2$ & $5$ & $3$ & $227$ & $2$ & $1+1+2+0-1$ & chiralny & tak \\
$12a_{325}$ & $2$ & $5$ & $3$ & $243$ & $2$ & $1+1+1+0-1$ & chiralny & tak \\
$12a_{326}$ & $2$ & $6$ & $3$ & $215$ & $-2$ & $1+2-1+3$ & chiralny & tak \\
$12a_{327}$ & $3$ & $5$ & $3$ & $175$ & $2$ & $1+4+6+4$ & odwracalny & tak \\
$12a_{328}$ & $2$ & $5$ & $3$ & $239$ & $-2$ & $1+0+1+0-1$ & chiralny & tak \\
$12a_{329}$ & $2..3$ & $5$ & $3$ & $141$ & $-4$ & $1+1-7-4$ & odwracalny & tak \\
$12a_{330}$ & $2$ & $6$ & ${}^{95}{\mskip -5mu/\mskip -3mu}_{43}$ & $95$ & $-2$ & $1+0-6$ & odwracalny & tak \\
$12a_{331}$ & $2..3$ & $5$ & $3$ & $177$ & $4$ & $1+4+0+1+1$ & odwracalny & tak \\
$12a_{332}$ & $2$ & $6$ & $3$ & $171$ & $2$ & $1+3-2+2$ & odwracalny & tak \\
$12a_{333}$ & $2$ & $5$ & $3$ & $241$ & $0$ & $1+0-1+0+1$ & chiralny & tak \\
$12a_{334}$ & $2$ & $5$ & $3$ & $249$ & $0$ & $1+2+0+0+1$ & chiralny & tak \\
$12a_{335}$ & $2$ & $5$ & $3$ & $271$ & $-2$ & $1+4+4+1-1$ & chiralny & tak \\
$12a_{336}$ & $2$ & $5$ & $3$ & $213$ & $-4$ & $1+3-2-4$ & chiralny & tak \\
$12a_{337}$ & $2$ & $6$ & $3$ & $259$ & $-2$ & $1+1-4+3$ & chiralny & tak \\
$12a_{338}$ & $2$ & $6$ & $3$ & $187$ & $-2$ & $1-1-4+2$ & chiralny & tak \\
$12a_{339}$ & $2$ & $6$ & $3$ & $155$ & $-2$ & $1+3-9$ & chiralny & tak \\
$12a_{340}$ & $1$ & $6$ & $3$ & $247$ & $-2$ & $1+2-3+3$ & chiralny & tak \\
$12a_{341}$ & $2$ & $5$ & $3$ & $229$ & $0$ & $1-1-2+0+1$ & -zwierciadlany & tak \\
$12a_{342}$ & $2$ & $5$ & $3$ & $237$ & $0$ & $1-3-2+0+1$ & chiralny & tak \\
$12a_{343}$ & $2$ & $6$ & $3$ & $199$ & $-2$ & $1-2-1+3$ & chiralny & tak \\
$12a_{344}$ & $3$ & $5$ & $3$ & $139$ & $-6$ & $1+7+13+5$ & chiralny & tak \\
$12a_{345}$ & $2$ & $6$ & $3$ & $105$ & $-4$ & $1+6+8$ & odwracalny & tak \\
$12a_{346}$ & $1$ & $4$ & $3$ & $213$ & $0$ & $1-1+1+5+2$ & odwracalny & tak \\
$12a_{347}$ & $2$ & $5$ & $3$ & $231$ & $-2$ & $1-2+1+4$ & odwracalny & tak \\
$12a_{348}$ & $2$ & $6$ & $3$ & $225$ & $0$ & $1-4+5-2$ & chiralny & tak \\
$12a_{349}$ & $2$ & $5$ & $3$ & $205$ & $4$ & $1+5-2+0+1$ & chiralny & tak \\
$12a_{350}$ & $2..3$ & $5$ & $3$ & $229$ & $-4$ & $1+3-5-5$ & chiralny & tak \\
$12a_{351}$ & $2$ & $5$ & $3$ & $221$ & $0$ & $1+1+2+1+1$ & chiralny & tak \\
$12a_{352}$ & $2$ & $5$ & $3$ & $259$ & $2$ & $1+1+0+0-1$ & chiralny & tak \\
$12a_{353}$ & $2$ & $5$ & $3$ & $187$ & $-2$ & $1+3+5+4$ & odwracalny & tak \\
$12a_{354}$ & $2$ & $5$ & $3$ & $259$ & $-2$ & $1+1+4+5$ & odwracalny & tak \\
$12a_{355}$ & $3$ & $5$ & $3$ & $127$ & $-6$ & $1+8+14+5$ & chiralny & tak \\
$12a_{356}$ & $2$ & $6$ & $3$ & $101$ & $-4$ & $1+7+8$ & odwracalny & tak \\
$12a_{357}$ & $2$ & $6$ & $3$ & $173$ & $0$ & $1+1-1-3$ & odwracalny & tak \\
$12a_{358}$ & $1$ & $5$ & $3$ & $255$ & $-2$ & $1+0+0+0-1$ & odwracalny & tak \\
$12a_{359}$ & $1$ & $5$ & $3$ & $305$ & $0$ & $1+0-1-1+1$ & odwracalny & tak \\
$12a_{360}$ & $1$ & $6$ & $3$ & $225$ & $0$ & $1+0+2-3$ & chiralny & tak \\
$12a_{361}$ & $2$ & $5$ & $3$ & $329$ & $0$ & $1-2+0-1+1$ & chiralny & tak \\
$12a_{362}$ & $2$ & $5$ & $3$ & $261$ & $4$ & $1+3-3-1+1$ & chiralny & tak \\
$12a_{363}$ & $2$ & $5$ & $3$ & $229$ & $4$ & $1+3-1+0+1$ & chiralny & tak \\
$12a_{364}$ & $2$ & $5$ & $3$ & $283$ & $2$ & $1+3-1+0-1$ & chiralny & tak \\
$12a_{365}$ & $2$ & $4$ & $3$ & $131$ & $2$ & $1+1-4-7-2$ & chiralny & tak \\
$12a_{366}$ & $2$ & $5$ & $3$ & $193$ & $0$ & $1+0-4-4$ & chiralny & tak \\
$12a_{367}$ & $4$ & $4$ & $3$ & $133$ & $-8$ & $1+11+23+15+3$ & odwracalny & tak \\
$12a_{368}$ & $3..4$ & $5$ & $3$ & $167$ & $-6$ & $1+10+16+6$ & odwracalny & tak \\
$12a_{369}$ & $3$ & $3$ & $3$ & $67$ & $6$ & $1+9+18+17+7+1$ & odwracalny & tak \\
$12a_{370}$ & $2..3$ & $4$ & $3$ & $113$ & $4$ & $1+8+13+9+2$ & odwracalny & tak \\
$12a_{371}$ & $2..3$ & $4$ & $3$ & $97$ & $4$ & $1+8+12+9+2$ & odwracalny & tak \\
$12a_{372}$ & $2..3$ & $5$ & $3$ & $155$ & $2$ & $1+7+8+4$ & odwracalny & tak \\
$12a_{373}$ & $3$ & $5$ & $3$ & $147$ & $-6$ & $1+5+4-1-1$ & odwracalny & tak \\
$12a_{374}$ & $1$ & $5$ & $3$ & $189$ & $0$ & $1+1+0+1+1$ & odwracalny & tak \\
$12a_{375}$ & $2..3$ & $4$ & $3$ & $117$ & $4$ & $1+7+13+9+2$ & odwracalny & tak \\
$12a_{376}$ & $2..3$ & $5$ & $3$ & $135$ & $2$ & $1+6+9+4$ & odwracalny & tak \\
$12a_{377}$ & $2$ & $5$ & $3$ & $225$ & $0$ & $1+0-2+0+1$ & chiralny & tak \\
$12a_{378}$ & $2$ & $6$ & ${}^{127}{\mskip -5mu/\mskip -3mu}_{45}$ & $127$ & $-2$ & $1+0+0+2$ & odwracalny & tak \\
$12a_{379}$ & $2..3$ & $5$ & ${}^{71}{\mskip -5mu/\mskip -3mu}_{17}$ & $71$ & $2$ & $1+6+9+3$ & odwracalny & tak \\
$12a_{380}$ & $2..3$ & $6$ & ${}^{77}{\mskip -5mu/\mskip -3mu}_{20}$ & $77$ & $0$ & $1+5+6$ & odwracalny & tak \\
$12a_{381}$ & $2..3$ & $6$ & $3$ & $155$ & $-2$ & $1-5+1+3$ & odwracalny & tak \\
$12a_{382}$ & $2..3$ & $5$ & $3$ & $133$ & $4$ & $1-1-4+1+1$ & odwracalny & tak \\
$12a_{383}$ & $2$ & $5$ & $3$ & $163$ & $2$ & $1+1+2-1-1$ & odwracalny & tak \\
$12a_{384}$ & $2$ & $6$ & ${}^{151}{\mskip -5mu/\mskip -3mu}_{62}$ & $151$ & $2$ & $1-2-2+2$ & odwracalny & tak \\
$12a_{385}$ & $2$ & $6$ & ${}^{161}{\mskip -5mu/\mskip -3mu}_{66}$ & $161$ & $0$ & $1+0+2-2$ & odwracalny & tak \\
$12a_{386}$ & $3$ & $5$ & $3$ & $189$ & $4$ & $1+5-3+0+1$ & odwracalny & tak \\
$12a_{387}$ & $2$ & $5$ & $3$ & $245$ & $0$ & $1-1-1+0+1$ & chiralny & tak \\
$12a_{388}$ & $1$ & $5$ & $3$ & $245$ & $0$ & $1-1-1+0+1$ & chiralny & tak \\
$12a_{389}$ & $2$ & $5$ & $3$ & $315$ & $2$ & $1+3+1+1-1$ & odwracalny & tak \\
$12a_{390}$ & $2$ & $4$ & $3$ & $243$ & $-2$ & $1+1-3-5-2$ & chiralny & tak \\
$12a_{391}$ & $3..4$ & $5$ & $3$ & $187$ & $-6$ & $1+11+19+7$ & odwracalny & tak \\
$12a_{392}$ & $2..4$ & $6$ & $3$ & $153$ & $-4$ & $1+10+12$ & odwracalny & tak \\
$12a_{393}$ & $2$ & $5$ & $3$ & $261$ & $-4$ & $1+3-3-5$ & chiralny & tak \\
$12a_{394}$ & $2$ & $4$ & $3$ & $181$ & $4$ & $1+7+9+7+2$ & chiralny & tak \\
$12a_{395}$ & $2$ & $5$ & $3$ & $183$ & $2$ & $1+6+6+4$ & chiralny & tak \\
$12a_{396}$ & $2$ & $5$ & $3$ & $189$ & $0$ & $1+1+0+1+1$ & odwracalny & tak \\
$12a_{397}$ & $2$ & $6$ & $3$ & $205$ & $-4$ & $1+5+2-3$ & odwracalny & tak \\
$12a_{398}$ & $2$ & $5$ & $3$ & $185$ & $-4$ & $1+2+0+1+1$ & chiralny & tak \\
$12a_{399}$ & $2$ & $5$ & $3$ & $211$ & $2$ & $1+5+8+5$ & chiralny & tak \\
$12a_{400}$ & $2$ & $6$ & $3$ & $203$ & $-2$ & $1+3+0+3$ & odwracalny & tak \\
$12a_{401}$ & $2$ & $6$ & $3$ & $159$ & $-2$ & $1+0-2+2$ & odwracalny & tak \\
$12a_{402}$ & $2$ & $5$ & $3$ & $271$ & $-2$ & $1+0-1+0-1$ & chiralny & tak \\
$12a_{403}$ & $2$ & $6$ & $3$ & $209$ & $0$ & $1+0+1-3$ & chiralny & tak \\
$12a_{404}$ & $2$ & $6$ & $3$ & $159$ & $2$ & $1-4+1+3$ & chiralny & tak \\
$12a_{405}$ & $2$ & $6$ & $3$ & $191$ & $-2$ & $1-4-1+3$ & chiralny & tak \\
$12a_{406}$ & $2$ & $6$ & ${}^{179}{\mskip -5mu/\mskip -3mu}_{74}$ & $179$ & $-2$ & $1-3-4+2$ & odwracalny & tak \\
$12a_{407}$ & $1$ & $6$ & $3$ & $251$ & $2$ & $1+3-3+3$ & chiralny & tak \\
$12a_{408}$ & $2$ & $5$ & $3$ & $285$ & $-4$ & $1+5-1-5$ & odwracalny & tak \\
$12a_{409}$ & $3$ & $4$ & $3$ & $147$ & $-6$ & $1+5+0-6-2$ & chiralny & tak \\
$12a_{410}$ & $2$ & $5$ & $3$ & $209$ & $-4$ & $1+4-2-4$ & chiralny & tak \\
$12a_{411}$ & $2$ & $4$ & $3$ & $153$ & $4$ & $1+6+7+7+2$ & chiralny & tak \\
$12a_{412}$ & $2$ & $5$ & $3$ & $211$ & $2$ & $1+5+4+4$ & chiralny & tak \\
$12a_{413}$ & $2$ & $5$ & $3$ & $279$ & $-2$ & $1+2-1+0-1$ & odwracalny & tak \\
$12a_{414}$ & $2..3$ & $6$ & $3$ & $147$ & $-2$ & $1+1-9$ & odwracalny & tak \\
$12a_{415}$ & $1$ & $5$ & $3$ & $249$ & $0$ & $1-2-1+0+1$ & chiralny & tak \\
$12a_{416}$ & $2$ & $5$ & $3$ & $207$ & $-2$ & $1+0+3+0-1$ & chiralny & tak \\
$12a_{417}$ & $2$ & $5$ & $3$ & $251$ & $2$ & $1-1+0+0-1$ & chiralny & tak \\
$12a_{418}$ & $2$ & $5$ & $3$ & $247$ & $2$ & $1+2+1+0-1$ & chiralny & tak \\
$12a_{419}$ & $2$ & $5$ & $3$ & $279$ & $-2$ & $1-2+2+1-1$ & chiralny & tak \\
$12a_{420}$ & $3$ & $5$ & $3$ & $111$ & $-6$ & $1+8+15+5$ & odwracalny & tak \\
$12a_{421}$ & $2..3$ & $6$ & $3$ & $133$ & $-4$ & $1+7+10$ & odwracalny & tak \\
$12a_{422}$ & $2..4$ & $4$ & $3$ & $69$ & $4$ & $1+7+14+10+2$ & odwracalny & tak \\
$12a_{423}$ & $2..3$ & $5$ & $3$ & $119$ & $2$ & $1+6+10+4$ & odwracalny & tak \\
$12a_{424}$ & $2..3$ & $5$ & $3$ & $131$ & $2$ & $1+5+9+4$ & odwracalny & tak \\
$12a_{425}$ & $2$ & $6$ & ${}^{81}{\mskip -5mu/\mskip -3mu}_{37}$ & $81$ & $0$ & $1+4+6$ & odwracalny & tak \\
$12a_{426}$ & $1$ & $5$ & $3$ & $273$ & $0$ & $1+0+1+0+1$ & odwracalny & tak \\
$12a_{427}$ & $2$ & $5$ & $3$ & $225$ & $0$ & $1+0-2+0+1$ & +zwierciadlany & tak \\
$12a_{428}$ & $2$ & $4$ & $3$ & $165$ & $0$ & $1+3+7+7+2$ & chiralny & tak \\
$12a_{429}$ & $2$ & $5$ & $3$ & $183$ & $-2$ & $1+2+5+4$ & chiralny & tak \\
$12a_{430}$ & $2$ & $4$ & $3$ & $221$ & $4$ & $1+5+7+6+2$ & odwracalny & tak \\
$12a_{431}$ & $3..4$ & $5$ & $3$ & $275$ & $-6$ & $1+9+17+8$ & chiralny & tak \\
$12a_{432}$ & $3$ & $5$ & $3$ & $211$ & $-6$ & $1+9+17+7$ & chiralny & tak \\
$12a_{433}$ & $3$ & $6$ & $3$ & $189$ & $-4$ & $1+5+1-3$ & odwracalny & tak \\
$12a_{434}$ & $2..3$ & $5$ & $3$ & $195$ & $2$ & $1+5+5+0-1$ & chiralny & tak \\
$12a_{435}$ & $2$ & $5$ & $3$ & $225$ & $0$ & $1+0-2+0+1$ & całkowicie & tak \\
$12a_{436}$ & $2..3$ & $5$ & $3$ & $143$ & $2$ & $1+4+4-1-1$ & odwracalny & tak \\
$12a_{437}$ & $2$ & $6$ & ${}^{149}{\mskip -5mu/\mskip -3mu}_{65}$ & $149$ & $0$ & $1+3+2-2$ & odwracalny & tak \\
$12a_{438}$ & $2$ & $5$ & $3$ & $209$ & $0$ & $1+0-3+0+1$ & odwracalny & tak \\
$12a_{439}$ & $1$ & $5$ & $3$ & $321$ & $0$ & $1+0+0-1+1$ & chiralny & tak \\
$12a_{440}$ & $2$ & $4$ & $3$ & $149$ & $-4$ & $1+3+6+7+2$ & chiralny & tak \\
$12a_{441}$ & $2$ & $5$ & $3$ & $199$ & $-2$ & $1+2+4+4$ & chiralny & tak \\
$12a_{442}$ & $3$ & $5$ & $3$ & $103$ & $-6$ & $1+10+16+5$ & odwracalny & tak \\
$12a_{443}$ & $2..3$ & $6$ & $3$ & $125$ & $-4$ & $1+9+10$ & odwracalny & tak \\
$12a_{444}$ & $2$ & $6$ & $3$ & $129$ & $-4$ & $1+4+1-2$ & odwracalny & tak \\
$12a_{445}$ & $2$ & $5$ & $3$ & $171$ & $2$ & $1+3+2-1-1$ & odwracalny & tak \\
$12a_{446}$ & $2$ & $5$ & $3$ & $231$ & $-2$ & $1+2+2+0-1$ & chiralny & tak \\
$12a_{447}$ & $2$ & $6$ & ${}^{121}{\mskip -5mu/\mskip -3mu}_{43}$ & $121$ & $0$ & $1+2+0-2$ & odwracalny & tak \\
$12a_{448}$ & $2$ & $7$ & $3$ & $105$ & $0$ & $1-6+5$ & odwracalny & tak \\
$12a_{449}$ & $2..3$ & $6$ & $3$ & $187$ & $2$ & $1-1-4+2$ & odwracalny & tak \\
$12a_{450}$ & $2$ & $6$ & $3$ & $207$ & $2$ & $1+4-4+2$ & odwracalny & tak \\
$12a_{451}$ & $1$ & $5$ & $3$ & $243$ & $-2$ & $1+1+1+0-1$ & chiralny & tak \\
$12a_{452}$ & $2$ & $5$ & $3$ & $253$ & $0$ & $1+1+0+0+1$ & chiralny & tak \\
$12a_{453}$ & $2$ & $5$ & $3$ & $149$ & $0$ & $1-5-4+1+1$ & odwracalny & tak \\
$12a_{454}$ & $2..3$ & $6$ & ${}^{103}{\mskip -5mu/\mskip -3mu}_{27}$ & $103$ & $-2$ & $1-6+0+2$ & odwracalny & tak \\
$12a_{455}$ & $1$ & $5$ & $3$ & $239$ & $2$ & $1+0+1+0-1$ & chiralny & tak \\
$12a_{456}$ & $2$ & $5$ & $3$ & $225$ & $0$ & $1-4-3+0+1$ & chiralny & tak \\
$12a_{457}$ & $2$ & $5$ & $3$ & $271$ & $2$ & $1+0-1+0-1$ & chiralny & tak \\
$12a_{458}$ & $2$ & $5$ & $3$ & $289$ & $0$ & $1+0-2-1+1$ & -zwierciadlany & tak \\
$12a_{459}$ & $2$ & $6$ & $3$ & $243$ & $2$ & $1+1-3+3$ & chiralny & tak \\
$12a_{460}$ & $2$ & $6$ & $3$ & $221$ & $0$ & $1-3+5-2$ & odwracalny & tak \\
$12a_{461}$ & $2..3$ & $5$ & $3$ & $233$ & $-4$ & $1+2-5-5$ & chiralny & tak \\
$12a_{462}$ & $2..3$ & $5$ & $3$ & $157$ & $0$ & $1-3-3+1+1$ & -zwierciadlany & tak \\
$12a_{463}$ & $2$ & $6$ & $3$ & $111$ & $-2$ & $1-4+0+2$ & odwracalny & tak \\
$12a_{464}$ & $2$ & $5$ & $3$ & $225$ & $0$ & $1+0+2+1+1$ & chiralny & tak \\
$12a_{465}$ & $2$ & $5$ & $3$ & $241$ & $0$ & $1-4-2+0+1$ & -zwierciadlany & tak \\
$12a_{466}$ & $2$ & $5$ & $3$ & $209$ & $4$ & $1+4-2+0+1$ & odwracalny & tak \\
$12a_{467}$ & $1$ & $5$ & $3$ & $271$ & $2$ & $1+0-1+0-1$ & chiralny & tak \\
$12a_{468}$ & $2$ & $5$ & $3$ & $191$ & $2$ & $1+4+5+0-1$ & chiralny & tak \\
$12a_{469}$ & $2$ & $5$ & $3$ & $217$ & $0$ & $1+2+2+1+1$ & chiralny & tak \\
$12a_{470}$ & $1$ & $5$ & $3$ & $247$ & $-2$ & $1+2+1+0-1$ & chiralny & tak \\
$12a_{471}$ & $2$ & $7$ & ${}^{85}{\mskip -5mu/\mskip -3mu}_{38}$ & $85$ & $0$ & $1-5+4$ & całkowicie & tak \\
$12a_{472}$ & $2$ & $6$ & $3$ & $237$ & $4$ & $1+5+4-3$ & odwracalny & tak \\
$12a_{473}$ & $2$ & $5$ & $3$ & $289$ & $0$ & $1+0+2+0+1$ & chiralny & tak \\
$12a_{474}$ & $2$ & $5$ & $3$ & $287$ & $2$ & $1+0+2+1-1$ & odwracalny & tak \\
$12a_{475}$ & $2$ & $5$ & $3$ & $255$ & $-2$ & $1+0+0+0-1$ & chiralny & tak \\
$12a_{476}$ & $2$ & $5$ & $3$ & $161$ & $4$ & $1+0-2+1+1$ & odwracalny & tak \\
$12a_{477}$ & $2..3$ & $7$ & ${}^{169}{\mskip -5mu/\mskip -3mu}_{70}$ & $169$ & $0$ & $1-6+5-1$ & całkowicie & tak \\
$12a_{478}$ & $2$ & $5$ & $3$ & $215$ & $2$ & $1+2+3+0-1$ & odwracalny & tak \\
$12a_{479}$ & $2$ & $5$ & $3$ & $257$ & $0$ & $1+0+0+0+1$ & chiralny & tak \\
$12a_{480}$ & $2$ & $5$ & $3$ & $287$ & $-2$ & $1+0-2+0-1$ & chiralny & tak \\
$12a_{481}$ & $2..3$ & $6$ & $3$ & $119$ & $2$ & $1-6-1+2$ & odwracalny & tak \\
$12a_{482}$ & $2..3$ & $7$ & ${}^{93}{\mskip -5mu/\mskip -3mu}_{22}$ & $93$ & $0$ & $1-7+4$ & odwracalny & tak \\
$12a_{483}$ & $1$ & $5$ & $3$ & $335$ & $2$ & $1+0-1+1-1$ & odwracalny & tak \\
$12a_{484}$ & $2$ & $5$ & $3$ & $289$ & $0$ & $1+0-2-1+1$ & chiralny & tak \\
$12a_{485}$ & $2$ & $5$ & $3$ & $255$ & $2$ & $1+0+0+0-1$ & odwracalny & tak \\
$12a_{486}$ & $2$ & $5$ & $3$ & $273$ & $0$ & $1+0-3-1+1$ & odwracalny & tak \\
$12a_{487}$ & $2$ & $5$ & $3$ & $263$ & $2$ & $1-2-1+0-1$ & chiralny & tak \\
$12a_{488}$ & $2$ & $5$ & $3$ & $175$ & $-2$ & $1-4+0-1-1$ & chiralny & tak \\
$12a_{489}$ & $2$ & $6$ & $3$ & $211$ & $-2$ & $1-3-2+3$ & chiralny & tak \\
$12a_{490}$ & $3$ & $5$ & $3$ & $207$ & $-6$ & $1+8+17+7$ & odwracalny & tak \\
$12a_{491}$ & $2$ & $6$ & $3$ & $177$ & $0$ & $1-4+2-2$ & odwracalny & tak \\
$12a_{492}$ & $2$ & $6$ & $3$ & $197$ & $0$ & $1-5+3-2$ & chiralny & tak \\
$12a_{493}$ & $2..3$ & $5$ & $3$ & $171$ & $-2$ & $1-5+0-1-1$ & odwracalny & tak \\
$12a_{494}$ & $2..3$ & $6$ & $3$ & $201$ & $0$ & $1-6+3-2$ & odwracalny & tak \\
$12a_{495}$ & $2$ & $4$ & $3$ & $245$ & $-4$ & $1+3+4+5+2$ & odwracalny & tak \\
$12a_{496}$ & $2..3$ & $4$ & $3$ & $267$ & $-2$ & $1+3+0-4-2$ & odwracalny & tak \\
$12a_{497}$ & $1$ & $5$ & ${}^{209}{\mskip -5mu/\mskip -3mu}_{81}$ & $209$ & $0$ & $1+0+1+1+1$ & odwracalny & tak \\
$12a_{498}$ & $1$ & $5$ & ${}^{207}{\mskip -5mu/\mskip -3mu}_{76}$ & $207$ & $-2$ & $1+0-1-1-1$ & odwracalny & tak \\
$12a_{499}$ & $2$ & $5$ & ${}^{233}{\mskip -5mu/\mskip -3mu}_{89}$ & $233$ & $0$ & $1+2+3+1+1$ & całkowicie & tak \\
$12a_{500}$ & $2$ & $5$ & ${}^{167}{\mskip -5mu/\mskip -3mu}_{60}$ & $167$ & $-2$ & $1-2+1-1-1$ & odwracalny & tak \\
$12a_{501}$ & $2$ & $5$ & ${}^{199}{\mskip -5mu/\mskip -3mu}_{55}$ & $199$ & $-2$ & $1-2-1-1-1$ & odwracalny & tak \\
$12a_{502}$ & $3$ & $4$ & ${}^{91}{\mskip -5mu/\mskip -3mu}_{37}$ & $91$ & $-6$ & $1+3-5-8-2$ & odwracalny & tak \\
$12a_{503}$ & $2$ & $5$ & $3$ & $135$ & $-2$ & $1-2+3-1-1$ & odwracalny & tak \\
$12a_{504}$ & $2$ & $5$ & $3$ & $153$ & $-4$ & $1+2-6-4$ & chiralny & tak \\
$12a_{505}$ & $2$ & $5$ & $3$ & $183$ & $2$ & $1+2+1-1-1$ & odwracalny & tak \\
$12a_{506}$ & $2$ & $5$ & ${}^{185}{\mskip -5mu/\mskip -3mu}_{68}$ & $185$ & $0$ & $1-2-1+1+1$ & całkowicie & tak \\
$12a_{507}$ & $3$ & $4$ & $3$ & $115$ & $-6$ & $1+1-7-8-2$ & odwracalny & tak \\
$12a_{508}$ & $2..3$ & $5$ & ${}^{129}{\mskip -5mu/\mskip -3mu}_{56}$ & $129$ & $-4$ & $1+0-8-4$ & odwracalny & tak \\
$12a_{509}$ & $1$ & $4$ & $3$ & $219$ & $-2$ & $1-1-6-6-2$ & chiralny & tak \\
$12a_{510}$ & $2$ & $5$ & ${}^{193}{\mskip -5mu/\mskip -3mu}_{81}$ & $193$ & $0$ & $1+0-4-4$ & całkowicie & tak \\
$12a_{511}$ & $2$ & $4$ & ${}^{125}{\mskip -5mu/\mskip -3mu}_{51}$ & $125$ & $4$ & $1+5+9+8+2$ & odwracalny & tak \\
$12a_{512}$ & $2$ & $5$ & ${}^{151}{\mskip -5mu/\mskip -3mu}_{64}$ & $151$ & $2$ & $1+2+3-1-1$ & odwracalny & tak \\
$12a_{513}$ & $2$ & $5$ & $3$ & $175$ & $2$ & $1+4+6+4$ & chiralny & tak \\
$12a_{514}$ & $2$ & $5$ & ${}^{187}{\mskip -5mu/\mskip -3mu}_{79}$ & $187$ & $2$ & $1+3+5+4$ & odwracalny & tak \\
$12a_{515}$ & $2$ & $5$ & $3$ & $177$ & $0$ & $1+0-1+1+1$ & odwracalny & tak \\
$12a_{516}$ & $2$ & $5$ & $3$ & $215$ & $-2$ & $1+2-1-1-1$ & odwracalny & tak \\
$12a_{517}$ & $2$ & $5$ & ${}^{145}{\mskip -5mu/\mskip -3mu}_{52}$ & $145$ & $-4$ & $1+0-3+1+1$ & odwracalny & tak \\
$12a_{518}$ & $2$ & $6$ & ${}^{157}{\mskip -5mu/\mskip -3mu}_{34}$ & $157$ & $0$ & $1-3+1-2$ & odwracalny & tak \\
$12a_{519}$ & $3$ & $4$ & ${}^{111}{\mskip -5mu/\mskip -3mu}_{25}$ & $111$ & $-6$ & $1+4-6-8-2$ & odwracalny & tak \\
$12a_{520}$ & $2$ & $5$ & ${}^{133}{\mskip -5mu/\mskip -3mu}_{36}$ & $133$ & $-4$ & $1+3-7-4$ & odwracalny & tak \\
$12a_{521}$ & $2$ & $5$ & ${}^{113}{\mskip -5mu/\mskip -3mu}_{48}$ & $113$ & $-4$ & $1+0-5+1+1$ & odwracalny & tak \\
$12a_{522}$ & $2$ & $5$ & ${}^{173}{\mskip -5mu/\mskip -3mu}_{73}$ & $173$ & $-4$ & $1+1-5-4$ & odwracalny & tak \\
$12a_{523}$ & $2$ & $4$ & $3$ & $213$ & $0$ & $1+3+6+6+2$ & chiralny & tak \\
$12a_{524}$ & $1$ & $6$ & $3$ & $171$ & $-2$ & $1-1-3+2$ & chiralny & tak \\
$12a_{525}$ & $2$ & $6$ & $3$ & $141$ & $0$ & $1+1+1-2$ & odwracalny & tak \\
$12a_{526}$ & $2$ & $4$ & $3$ & $211$ & $-2$ & $1+1-5-6-2$ & chiralny & tak \\
$12a_{527}$ & $2$ & $4$ & $3$ & $197$ & $-4$ & $1+3+5+6+2$ & chiralny & tak \\
$12a_{528}$ & $2$ & $5$ & ${}^{183}{\mskip -5mu/\mskip -3mu}_{67}$ & $183$ & $2$ & $1+2+1-1-1$ & odwracalny & tak \\
$12a_{529}$ & $2$ & $6$ & $3$ & $189$ & $0$ & $1-3+3-2$ & chiralny & tak \\
$12a_{530}$ & $2$ & $6$ & $3$ & $217$ & $0$ & $1-2+1-3$ & odwracalny & tak \\
$12a_{531}$ & $2$ & $4$ & $3$ & $151$ & $-2$ & $1-2-10-8-2$ & odwracalny & tak \\
$12a_{532}$ & $2$ & $5$ & ${}^{125}{\mskip -5mu/\mskip -3mu}_{33}$ & $125$ & $0$ & $1-3-9-4$ & odwracalny & tak \\
$12a_{533}$ & $2..3$ & $4$ & ${}^{137}{\mskip -5mu/\mskip -3mu}_{31}$ & $137$ & $4$ & $1+6+10+8+2$ & odwracalny & tak \\
$12a_{534}$ & $2$ & $5$ & ${}^{163}{\mskip -5mu/\mskip -3mu}_{44}$ & $163$ & $2$ & $1+5+7+4$ & odwracalny & tak \\
$12a_{535}$ & $1$ & $5$ & ${}^{175}{\mskip -5mu/\mskip -3mu}_{47}$ & $175$ & $2$ & $1+0+1-1-1$ & odwracalny & tak \\
$12a_{536}$ & $2$ & $5$ & ${}^{137}{\mskip -5mu/\mskip -3mu}_{29}$ & $137$ & $-4$ & $1+2-3+1+1$ & odwracalny & tak \\
$12a_{537}$ & $1$ & $6$ & ${}^{179}{\mskip -5mu/\mskip -3mu}_{50}$ & $179$ & $-2$ & $1+1-3+2$ & odwracalny & tak \\
$12a_{538}$ & $3$ & $4$ & ${}^{83}{\mskip -5mu/\mskip -3mu}_{13}$ & $83$ & $-6$ & $1+5-4-8-2$ & odwracalny & tak \\
$12a_{539}$ & $2$ & $5$ & ${}^{145}{\mskip -5mu/\mskip -3mu}_{44}$ & $145$ & $-4$ & $1+4-6-4$ & odwracalny & tak \\
$12a_{540}$ & $2$ & $5$ & ${}^{165}{\mskip -5mu/\mskip -3mu}_{49}$ & $165$ & $-4$ & $1+3-5-4$ & odwracalny & tak \\
$12a_{541}$ & $2$ & $5$ & ${}^{153}{\mskip -5mu/\mskip -3mu}_{41}$ & $153$ & $-4$ & $1-2-3+1+1$ & odwracalny & tak \\
$12a_{542}$ & $2$ & $6$ & $3$ & $123$ & $-2$ & $1+3-7$ & odwracalny & tak \\
$12a_{543}$ & $2$ & $6$ & $3$ & $239$ & $-2$ & $1+0-3+3$ & chiralny & tak \\
$12a_{544}$ & $2..3$ & $4$ & $3$ & $157$ & $0$ & $1+5+11+8+2$ & odwracalny & tak \\
$12a_{545}$ & $2..3$ & $5$ & ${}^{143}{\mskip -5mu/\mskip -3mu}_{63}$ & $143$ & $-2$ & $1+4+8+4$ & odwracalny & tak \\
$12a_{546}$ & $2$ & $6$ & $3$ & $231$ & $-2$ & $1-2-3+3$ & odwracalny & tak \\
$12a_{547}$ & $2$ & $5$ & $3$ & $233$ & $-4$ & $1+2-5-5$ & chiralny & tak \\
$12a_{548}$ & $2$ & $5$ & $3$ & $109$ & $-4$ & $1-3-10-4$ & odwracalny & tak \\
$12a_{549}$ & $2$ & $6$ & ${}^{111}{\mskip -5mu/\mskip -3mu}_{26}$ & $111$ & $-2$ & $1-4-8$ & odwracalny & tak \\
$12a_{550}$ & $2$ & $6$ & ${}^{149}{\mskip -5mu/\mskip -3mu}_{34}$ & $149$ & $0$ & $1-1+1-2$ & odwracalny & tak \\
$12a_{551}$ & $1$ & $6$ & ${}^{103}{\mskip -5mu/\mskip -3mu}_{18}$ & $103$ & $-2$ & $1+2-6$ & odwracalny & tak \\
$12a_{552}$ & $2$ & $6$ & ${}^{131}{\mskip -5mu/\mskip -3mu}_{30}$ & $131$ & $-2$ & $1-3-1+2$ & odwracalny & tak \\
$12a_{553}$ & $2..3$ & $4$ & $3$ & $205$ & $-4$ & $1+5+6+6+2$ & odwracalny & tak \\
$12a_{554}$ & $2..3$ & $6$ & $4$ & $189$ & $0$ & $1-3+3-2$ & odwracalny & tak \\
$12a_{555}$ & $2$ & $6$ & $3$ & $193$ & $0$ & $1+0+0-3$ & chiralny & tak \\
$12a_{556}$ & $2..3$ & $5$ & $3$ & $145$ & $-4$ & $1+0-7-4$ & chiralny & tak \\
$12a_{557}$ & $2$ & $6$ & $3$ & $107$ & $-2$ & $1-1-7$ & chiralny & tak \\
$12a_{558}$ & $2$ & $4$ & $3$ & $233$ & $-4$ & $1+2+3+5+2$ & chiralny & tak \\
$12a_{559}$ & $1$ & $5$ & $3$ & $211$ & $-2$ & $1+1+3+4$ & chiralny & tak \\
$12a_{560}$ & $1$ & $6$ & $3$ & $173$ & $0$ & $1+1+3-2$ & chiralny & tak \\
$12a_{561}$ & $3$ & $5$ & $3$ & $225$ & $-4$ & $1+4+3+1+1$ & odwracalny & tak \\
$12a_{562}$ & $2$ & $6$ & $3$ & $171$ & $-2$ & $1-1-3+2$ & chiralny & tak \\
$12a_{563}$ & $3$ & $5$ & $3$ & $117$ & $-4$ & $1+3-8-4$ & chiralny & tak \\
$12a_{564}$ & $2$ & $6$ & $3$ & $135$ & $-2$ & $1+2-8$ & chiralny & tak \\
$12a_{565}$ & $2$ & $5$ & $3$ & $193$ & $4$ & $1+4+1+1+1$ & chiralny & tak \\
$12a_{566}$ & $2$ & $6$ & $3$ & $155$ & $2$ & $1+3-1+2$ & chiralny & tak \\
$12a_{567}$ & $2$ & $4$ & $3$ & $249$ & $0$ & $1-2-1+4+2$ & odwracalny & tak \\
$12a_{568}$ & $2..3$ & $5$ & $3$ & $157$ & $-4$ & $1+1-6-4$ & chiralny & tak \\
$12a_{569}$ & $3$ & $5$ & $3$ & $153$ & $4$ & $1+6-1+1+1$ & odwracalny & tak \\
$12a_{570}$ & $2$ & $6$ & $3$ & $195$ & $2$ & $1+5-3+2$ & odwracalny & tak \\
$12a_{571}$ & $1$ & $4$ & $3$ & $279$ & $-2$ & $1+2-1-4-2$ & chiralny & tak \\
$12a_{572}$ & $2$ & $4$ & $3$ & $175$ & $-2$ & $1+0-7-7-2$ & chiralny & tak \\
$12a_{573}$ & $2$ & $4$ & $3$ & $143$ & $2$ & $1+0-5-7-2$ & chiralny & tak \\
$12a_{574}$ & $4$ & $4$ & $3$ & $117$ & $-8$ & $1+11+22+15+3$ & odwracalny & tak \\
$12a_{575}$ & $3$ & $5$ & $3$ & $183$ & $-6$ & $1+10+15+6$ & odwracalny & tak \\
$12a_{576}$ & $3$ & $3$ & $3$ & $63$ & $6$ & $1+8+18+17+7+1$ & odwracalny & tak \\
$12a_{577}$ & $2$ & $4$ & $3$ & $117$ & $4$ & $1+7+13+9+2$ & odwracalny & tak \\
$12a_{578}$ & $2$ & $4$ & $3$ & $153$ & $4$ & $1+6+11+8+2$ & odwracalny & tak \\
$12a_{579}$ & $1$ & $5$ & ${}^{177}{\mskip -5mu/\mskip -3mu}_{49}$ & $177$ & $0$ & $1+0-1+1+1$ & odwracalny & tak \\
$12a_{580}$ & $2..3$ & $4$ & ${}^{69}{\mskip -5mu/\mskip -3mu}_{11}$ & $69$ & $4$ & $1+7+14+10+2$ & odwracalny & tak \\
$12a_{581}$ & $2$ & $5$ & ${}^{119}{\mskip -5mu/\mskip -3mu}_{36}$ & $119$ & $2$ & $1+6+10+4$ & odwracalny & tak \\
$12a_{582}$ & $2$ & $5$ & ${}^{131}{\mskip -5mu/\mskip -3mu}_{39}$ & $131$ & $2$ & $1+5+9+4$ & odwracalny & tak \\
$12a_{583}$ & $2$ & $5$ & ${}^{161}{\mskip -5mu/\mskip -3mu}_{45}$ & $161$ & $0$ & $1-4-3+1+1$ & odwracalny & tak \\
$12a_{584}$ & $2$ & $5$ & ${}^{143}{\mskip -5mu/\mskip -3mu}_{31}$ & $143$ & $-2$ & $1+0+3-1-1$ & odwracalny & tak \\
$12a_{585}$ & $1$ & $6$ & ${}^{181}{\mskip -5mu/\mskip -3mu}_{50}$ & $181$ & $0$ & $1-1+3-2$ & odwracalny & tak \\
$12a_{586}$ & $3..4$ & $5$ & $3$ & $203$ & $-6$ & $1+11+18+7$ & chiralny & tak \\
$12a_{587}$ & $3$ & $4$ & $3$ & $175$ & $-6$ & $1+4-2-6-2$ & chiralny & tak \\
$12a_{588}$ & $1$ & $4$ & $3$ & $255$ & $-2$ & $1+0-4-5-2$ & chiralny & tak \\
$12a_{589}$ & $2$ & $4$ & $3$ & $185$ & $4$ & $1+6+9+7+2$ & chiralny & tak \\
$12a_{590}$ & $3$ & $4$ & $3$ & $151$ & $-6$ & $1+2-5-7-2$ & chiralny & tak \\
$12a_{591}$ & $2$ & $5$ & $3$ & $109$ & $-4$ & $1+1-5-3$ & odwracalny & tak \\
$12a_{592}$ & $2..3$ & $5$ & $3$ & $217$ & $-4$ & $1+2-6-5$ & chiralny & tak \\
$12a_{593}$ & $2$ & $5$ & $3$ & $181$ & $0$ & $1-1-5-4$ & chiralny & tak \\
$12a_{594}$ & $2$ & $5$ & $3$ & $99$ & $2$ & $1+5+7+3$ & odwracalny & tak \\
$12a_{595}$ & $2$ & $6$ & ${}^{139}{\mskip -5mu/\mskip -3mu}_{30}$ & $139$ & $-2$ & $1-1-1+2$ & odwracalny & tak \\
$12a_{596}$ & $2$ & $6$ & ${}^{81}{\mskip -5mu/\mskip -3mu}_{14}$ & $81$ & $0$ & $1+4+6$ & odwracalny & tak \\
$12a_{597}$ & $2..3$ & $6$ & ${}^{123}{\mskip -5mu/\mskip -3mu}_{26}$ & $123$ & $2$ & $1-5-1+2$ & odwracalny & tak \\
$12a_{598}$ & $2$ & $4$ & $3$ & $211$ & $-2$ & $1+1-5-6-2$ & chiralny & tak \\
$12a_{599}$ & $2$ & $4$ & $3$ & $209$ & $4$ & $1+4+6+6+2$ & chiralny & tak \\
$12a_{600}$ & $2..3$ & $5$ & ${}^{109}{\mskip -5mu/\mskip -3mu}_{25}$ & $109$ & $-4$ & $1+1-9-4$ & odwracalny & tak \\
$12a_{601}$ & $2$ & $6$ & ${}^{127}{\mskip -5mu/\mskip -3mu}_{34}$ & $127$ & $-2$ & $1+0-8$ & odwracalny & tak \\
$12a_{602}$ & $2$ & $4$ & $3$ & $229$ & $0$ & $1-1+2+5+2$ & chiralny & tak \\
$12a_{603}$ & $2$ & $5$ & $3$ & $215$ & $-2$ & $1-2+2+4$ & chiralny & tak \\
$12a_{604}$ & $1$ & $6$ & $3$ & $223$ & $-2$ & $1+0-2+3$ & chiralny & tak \\
$12a_{605}$ & $2$ & $4$ & $3$ & $155$ & $2$ & $1-1-6-7-2$ & chiralny & tak \\
$12a_{606}$ & $1$ & $5$ & $3$ & $169$ & $0$ & $1-2-6-4$ & chiralny & tak \\
$12a_{607}$ & $2$ & $4$ & $3$ & $173$ & $4$ & $1+5+8+7+2$ & chiralny & tak \\
$12a_{608}$ & $1$ & $5$ & $3$ & $191$ & $2$ & $1+4+5+4$ & chiralny & tak \\
$12a_{609}$ & $2$ & $5$ & $3$ & $223$ & $2$ & $1+4+7+5$ & chiralny & tak \\
$12a_{610}$ & $2..3$ & $6$ & $3$ & $137$ & $-4$ & $1+10+11$ & chiralny & tak \\
$12a_{611}$ & $2$ & $5$ & $3$ & $237$ & $-4$ & $1+5+0-4$ & chiralny & tak \\
$12a_{612}$ & $2$ & $5$ & $3$ & $145$ & $-4$ & $1+0-7-4$ & chiralny & tak \\
$12a_{613}$ & $2$ & $5$ & $3$ & $155$ & $2$ & $1+3+3+3$ & chiralny & tak \\
$12a_{614}$ & $1$ & $6$ & $3$ & $215$ & $-2$ & $1-2-2+3$ & chiralny & tak \\
$12a_{615}$ & $3$ & $5$ & $3$ & $243$ & $-6$ & $1+9+15+7$ & odwracalny & tak \\
$12a_{616}$ & $2..3$ & $4$ & $3$ & $265$ & $-4$ & $1+6+6+5+2$ & odwracalny & tak \\
$12a_{617}$ & $2$ & $4$ & $3$ & $177$ & $0$ & $1+4+8+7+2$ & chiralny & tak \\
$12a_{618}$ & $2$ & $4$ & $3$ & $257$ & $-4$ & $1+4+5+5+2$ & chiralny & tak \\
$12a_{619}$ & $2$ & $6$ & $3$ & $95$ & $-2$ & $1+0-6$ & chiralny & tak \\
$12a_{620}$ & $2$ & $4$ & $3$ & $269$ & $-4$ & $1+1+1+4+2$ & chiralny & tak \\
$12a_{621}$ & $2$ & $5$ & $3$ & $175$ & $2$ & $1+4+6+0-1$ & chiralny & tak \\
$12a_{622}$ & $2$ & $6$ & $3$ & $197$ & $0$ & $1+3+5-2$ & chiralny & tak \\
$12a_{623}$ & $2$ & $5$ & $3$ & $189$ & $-4$ & $1+1-4-4$ & chiralny & tak \\
$12a_{624}$ & $2$ & $5$ & $3$ & $179$ & $2$ & $1+5+6+4$ & chiralny & tak \\
$12a_{625}$ & $2$ & $6$ & $3$ & $209$ & $0$ & $1+0+1-3$ & odwracalny & tak \\
$12a_{626}$ & $1$ & $4$ & $3$ & $291$ & $-2$ & $1+1-2-4-2$ & chiralny & tak \\
$12a_{627}$ & $2$ & $5$ & $3$ & $229$ & $0$ & $1-1-2+0+1$ & -zwierciadlany & tak \\
$12a_{628}$ & $2$ & $5$ & $3$ & $195$ & $-2$ & $1+1+0+3$ & odwracalny & tak \\
$12a_{629}$ & $2$ & $5$ & $3$ & $297$ & $0$ & $1+2+3+0+1$ & odwracalny & tak \\
$12a_{630}$ & $2$ & $5$ & $3$ & $239$ & $-2$ & $1+0+1+0-1$ & chiralny & tak \\
$12a_{631}$ & $1$ & $5$ & $3$ & $225$ & $0$ & $1+0-2-4$ & chiralny & tak \\
$12a_{632}$ & $2$ & $5$ & $3$ & $149$ & $0$ & $1-1-7-4$ & chiralny & tak \\
$12a_{633}$ & $2$ & $5$ & $3$ & $219$ & $-2$ & $1+3+3+4$ & chiralny & tak \\
$12a_{634}$ & $2..3$ & $5$ & $3$ & $135$ & $2$ & $1+6+9+4$ & odwracalny & tak \\
$12a_{635}$ & $1$ & $6$ & $3$ & $201$ & $0$ & $1+2+1-3$ & chiralny & tak \\
$12a_{636}$ & $2..3$ & $6$ & $3$ & $93$ & $0$ & $1+5+7$ & odwracalny & tak \\
$12a_{637}$ & $1$ & $5$ & $3$ & $231$ & $2$ & $1+2+2+0-1$ & chiralny & tak \\
$12a_{638}$ & $2$ & $5$ & $3$ & $181$ & $-4$ & $1+3-4-4$ & chiralny & tak \\
$12a_{639}$ & $2..3$ & $4$ & $3$ & $199$ & $-2$ & $1+2-4-6-2$ & chiralny & tak \\
$12a_{640}$ & $2$ & $5$ & $3$ & $125$ & $0$ & $1+1-4-3$ & chiralny & tak \\
$12a_{641}$ & $2..3$ & $4$ & $3$ & $73$ & $4$ & $1+6+14+10+2$ & odwracalny & tak \\
$12a_{642}$ & $2$ & $5$ & $3$ & $115$ & $2$ & $1+5+10+4$ & odwracalny & tak \\
$12a_{643}$ & $2..3$ & $5$ & ${}^{99}{\mskip -5mu/\mskip -3mu}_{23}$ & $99$ & $2$ & $1+5+11+4$ & odwracalny & tak \\
$12a_{644}$ & $2$ & $6$ & ${}^{113}{\mskip -5mu/\mskip -3mu}_{30}$ & $113$ & $0$ & $1+4+8$ & odwracalny & tak \\
$12a_{645}$ & $2$ & $4$ & $3$ & $259$ & $-2$ & $1+1-4-5-2$ & chiralny & tak \\
$12a_{646}$ & $1$ & $4$ & $3$ & $169$ & $0$ & $1+2+7+7+2$ & chiralny & tak \\
$12a_{647}$ & $4$ & $4$ & $3$ & $153$ & $-8$ & $1+10+20+14+3$ & odwracalny & tak \\
$12a_{648}$ & $3$ & $5$ & $3$ & $147$ & $-6$ & $1+9+13+5$ & odwracalny & tak \\
$12a_{649}$ & $2..3$ & $5$ & ${}^{127}{\mskip -5mu/\mskip -3mu}_{27}$ & $127$ & $2$ & $1+4+5-1-1$ & odwracalny & tak \\
$12a_{650}$ & $2$ & $6$ & ${}^{165}{\mskip -5mu/\mskip -3mu}_{46}$ & $165$ & $0$ & $1+3+3-2$ & odwracalny & tak \\
$12a_{651}$ & $2..3$ & $5$ & ${}^{97}{\mskip -5mu/\mskip -3mu}_{17}$ & $97$ & $-4$ & $1-4-7+1+1$ & odwracalny & tak \\
$12a_{652}$ & $2$ & $6$ & ${}^{155}{\mskip -5mu/\mskip -3mu}_{46}$ & $155$ & $-2$ & $1-5-3+2$ & odwracalny & tak \\
$12a_{653}$ & $2..3$ & $6$ & $3$ & $145$ & $-4$ & $1+8+11$ & odwracalny & tak \\
$12a_{654}$ & $2$ & $5$ & $3$ & $165$ & $-4$ & $1+3-1-3$ & chiralny & tak \\
$12a_{655}$ & $2$ & $5$ & $3$ & $159$ & $2$ & $1+4+7+4$ & chiralny & tak \\
$12a_{656}$ & $1$ & $5$ & $3$ & $235$ & $2$ & $1+3+2+4$ & odwracalny & tak \\
$12a_{657}$ & $2$ & $4$ & $3$ & $181$ & $0$ & $1+3+8+7+2$ & chiralny & tak \\
$12a_{658}$ & $2$ & $5$ & $3$ & $167$ & $-2$ & $1+2+6+4$ & chiralny & tak \\
$12a_{659}$ & $3..4$ & $5$ & $3$ & $263$ & $-6$ & $1+10+18+8$ & chiralny & tak \\
$12a_{660}$ & $3$ & $4$ & $3$ & $163$ & $-6$ & $1+5-1-6-2$ & chiralny & tak \\
$12a_{661}$ & $2..3$ & $5$ & $3$ & $193$ & $-4$ & $1+4-3-4$ & chiralny & tak \\
$12a_{662}$ & $1$ & $5$ & $3$ & $241$ & $0$ & $1+0-1+0+1$ & chiralny & tak \\
$12a_{663}$ & $2$ & $6$ & $3$ & $175$ & $-2$ & $1+0+1+3$ & odwracalny & tak \\
$12a_{664}$ & $3$ & $5$ & $3$ & $169$ & $-4$ & $1-2-6+0+1$ & odwracalny & tak \\
$12a_{665}$ & $2..3$ & $6$ & $3$ & $195$ & $-2$ & $1-3-5+2$ & odwracalny & tak \\
$12a_{666}$ & $2$ & $6$ & $3$ & $207$ & $-2$ & $1-4-2+3$ & chiralny & tak \\
$12a_{667}$ & $2..4$ & $4$ & $3$ & $121$ & $0$ & $1+6+13+9+2$ & chiralny & tak \\
$12a_{668}$ & $2..3$ & $4$ & $3$ & $189$ & $0$ & $1+5+9+7+2$ & chiralny & tak \\
$12a_{669}$ & $2..3$ & $5$ & $3$ & $67$ & $2$ & $1+5+9+3$ & odwracalny & tak \\
$12a_{670}$ & $1$ & $4$ & $3$ & $237$ & $0$ & $1+1+3+5+2$ & chiralny & tak \\
$12a_{671}$ & $2$ & $4$ & $3$ & $173$ & $-4$ & $1+5+8+7+2$ & chiralny & tak \\
$12a_{672}$ & $2$ & $4$ & $3$ & $243$ & $-2$ & $1+1-3-5-2$ & chiralny & tak \\
$12a_{673}$ & $2$ & $5$ & $3$ & $203$ & $-2$ & $1-1+3+0-1$ & odwracalny & tak \\
$12a_{674}$ & $2$ & $5$ & $3$ & $221$ & $0$ & $1-3-3+0+1$ & chiralny & tak \\
$12a_{675}$ & $2$ & $4$ & $3$ & $209$ & $-4$ & $1+4+6+6+2$ & chiralny & tak \\
$12a_{676}$ & $2$ & $5$ & $3$ & $139$ & $-2$ & $1+3+4+3$ & chiralny & tak \\
$12a_{677}$ & $2$ & $5$ & $3$ & $175$ & $-2$ & $1+4+6+4$ & chiralny & tak \\
$12a_{678}$ & $2$ & $5$ & $3$ & $179$ & $-2$ & $1+1+5+4$ & chiralny & tak \\
$12a_{679}$ & $2$ & $6$ & $3$ & $81$ & $-4$ & $1+8+7$ & odwracalny & tak \\
$12a_{680}$ & $2..3$ & $4$ & $3$ & $199$ & $-2$ & $1+2-4-6-2$ & chiralny & tak \\
$12a_{681}$ & $2..3$ & $5$ & $3$ & $145$ & $-4$ & $1-4-4+1+1$ & odwracalny & tak \\
$12a_{682}$ & $2$ & $6$ & ${}^{107}{\mskip -5mu/\mskip -3mu}_{29}$ & $107$ & $-2$ & $1-5+0+2$ & odwracalny & tak \\
$12a_{683}$ & $3$ & $5$ & $3$ & $117$ & $-4$ & $1-5-6+1+1$ & odwracalny & tak \\
$12a_{684}$ & $2..3$ & $6$ & ${}^{135}{\mskip -5mu/\mskip -3mu}_{41}$ & $135$ & $-2$ & $1-6-2+2$ & odwracalny & tak \\
$12a_{685}$ & $2$ & $5$ & $3$ & $201$ & $-4$ & $1+2-3-4$ & chiralny & tak \\
$12a_{686}$ & $2..3$ & $5$ & $3$ & $209$ & $-4$ & $1+0-7-5$ & odwracalny & tak \\
$12a_{687}$ & $2..3$ & $6$ & $3$ & $223$ & $-2$ & $1-4-3+3$ & chiralny & tak \\
$12a_{688}$ & $2$ & $4$ & $3$ & $209$ & $-4$ & $1+4+6+6+2$ & chiralny & tak \\
$12a_{689}$ & $2..3$ & $6$ & $3$ & $123$ & $-2$ & $1-5-1+2$ & odwracalny & tak \\
$12a_{690}$ & $2$ & $7$ & ${}^{89}{\mskip -5mu/\mskip -3mu}_{20}$ & $89$ & $0$ & $1-6+4$ & odwracalny & tak \\
$12a_{691}$ & $2$ & $7$ & ${}^{77}{\mskip -5mu/\mskip -3mu}_{12}$ & $77$ & $0$ & $1-7+3$ & odwracalny & tak \\
$12a_{692}$ & $2..4$ & $4$ & $4$ & $153$ & $4$ & $1+6+11+8+2$ & odwracalny & tak \\
$12a_{693}$ & $2..3$ & $5$ & $4$ & $141$ & $4$ & $1-3-4+1+1$ & odwracalny & tak \\
$12a_{694}$ & $3$ & $5$ & $4$ & $159$ & $-6$ & $1+4-1-6-2$ & odwracalny & tak \\
$12a_{695}$ & $1$ & $5$ & $3$ & $275$ & $-2$ & $1+1-1+0-1$ & chiralny & tak \\
$12a_{696}$ & $1$ & $5$ & $3$ & $277$ & $0$ & $1-1+1+0+1$ & chiralny & tak \\
$12a_{697}$ & $1$ & $5$ & $3$ & $301$ & $0$ & $1+1-1-1+1$ & chiralny & tak \\
$12a_{698}$ & $2$ & $5$ & $3$ & $251$ & $2$ & $1-1+0+0-1$ & chiralny & tak \\
$12a_{699}$ & $2$ & $5$ & $3$ & $261$ & $0$ & $1-1+0+0+1$ & chiralny & tak \\
$12a_{700}$ & $2$ & $5$ & $3$ & $253$ & $0$ & $1+1+4+1+1$ & chiralny & tak \\
$12a_{701}$ & $2$ & $4$ & $3$ & $225$ & $4$ & $1+4+7+6+2$ & chiralny & tak \\
$12a_{702}$ & $2..3$ & $5$ & $3$ & $165$ & $4$ & $1-1-2+1+1$ & chiralny & tak \\
$12a_{703}$ & $2$ & $5$ & $3$ & $291$ & $2$ & $1+1+2+1-1$ & chiralny & tak \\
$12a_{704}$ & $2$ & $5$ & $3$ & $203$ & $2$ & $1+3+4+0-1$ & chiralny & tak \\
$12a_{705}$ & $2$ & $5$ & $3$ & $309$ & $0$ & $1-1-1-1+1$ & odwracalny & tak \\
$12a_{706}$ & $2..3$ & $5$ & $3$ & $213$ & $0$ & $1+3+2+1+1$ & odwracalny & tak \\
$12a_{707}$ & $2$ & $5$ & $3$ & $269$ & $0$ & $1+1+1+0+1$ & chiralny & tak \\
$12a_{708}$ & $2$ & $5$ & $3$ & $195$ & $-2$ & $1+1+0-1-1$ & odwracalny & tak \\
$12a_{709}$ & $2$ & $5$ & $3$ & $259$ & $2$ & $1+1+0+0-1$ & chiralny & tak \\
$12a_{710}$ & $2$ & $5$ & $3$ & $299$ & $2$ & $1-1+1+1-1$ & chiralny & tak \\
$12a_{711}$ & $2$ & $5$ & $3$ & $243$ & $2$ & $1+1+1+0-1$ & chiralny & tak \\
$12a_{712}$ & $2$ & $5$ & $3$ & $315$ & $2$ & $1-1+0+1-1$ & odwracalny & tak \\
$12a_{713}$ & $2$ & $4$ & ${}^{139}{\mskip -5mu/\mskip -3mu}_{39}$ & $139$ & $-2$ & $1-1-9-8-2$ & odwracalny & tak \\
$12a_{714}$ & $2$ & $4$ & ${}^{107}{\mskip -5mu/\mskip -3mu}_{19}$ & $107$ & $2$ & $1-1-7-8-2$ & odwracalny & tak \\
$12a_{715}$ & $1$ & $5$ & ${}^{169}{\mskip -5mu/\mskip -3mu}_{50}$ & $169$ & $0$ & $1-2-6-4$ & odwracalny & tak \\
$12a_{716}$ & $3$ & $3$ & ${}^{43}{\mskip -5mu/\mskip -3mu}_{5}$ & $43$ & $6$ & $1+7+15+16+7+1$ & odwracalny & tak \\
$12a_{717}$ & $2$ & $4$ & ${}^{89}{\mskip -5mu/\mskip -3mu}_{28}$ & $89$ & $4$ & $1+6+11+9+2$ & odwracalny & tak \\
$12a_{718}$ & $2$ & $4$ & ${}^{141}{\mskip -5mu/\mskip -3mu}_{41}$ & $141$ & $4$ & $1+5+10+8+2$ & odwracalny & tak \\
$12a_{719}$ & $2$ & $5$ & $3$ & $147$ & $2$ & $1+5+8+4$ & odwracalny & tak \\
$12a_{720}$ & $2$ & $4$ & ${}^{113}{\mskip -5mu/\mskip -3mu}_{21}$ & $113$ & $4$ & $1+4+8+8+2$ & odwracalny & tak \\
$12a_{721}$ & $1$ & $5$ & ${}^{171}{\mskip -5mu/\mskip -3mu}_{50}$ & $171$ & $2$ & $1+3+6+4$ & odwracalny & tak \\
$12a_{722}$ & $4$ & $3$ & ${}^{29}{\mskip -5mu/\mskip -3mu}_{3}$ & $29$ & $-8$ & $1+5-5-14-7-1$ & odwracalny & tak \\
$12a_{723}$ & $3$ & $4$ & ${}^{63}{\mskip -5mu/\mskip -3mu}_{20}$ & $63$ & $-6$ & $1+4-7-9-2$ & odwracalny & tak \\
$12a_{724}$ & $3$ & $4$ & ${}^{107}{\mskip -5mu/\mskip -3mu}_{31}$ & $107$ & $-6$ & $1+3-6-8-2$ & odwracalny & tak \\
$12a_{725}$ & $3$ & $5$ & $3$ & $117$ & $-4$ & $1+3-8-4$ & odwracalny & tak \\
$12a_{726}$ & $3$ & $4$ & ${}^{103}{\mskip -5mu/\mskip -3mu}_{19}$ & $103$ & $-6$ & $1+2-6-8-2$ & odwracalny & tak \\
$12a_{727}$ & $2$ & $5$ & ${}^{157}{\mskip -5mu/\mskip -3mu}_{46}$ & $157$ & $-4$ & $1+1-6-4$ & odwracalny & tak \\
$12a_{728}$ & $2$ & $4$ & ${}^{133}{\mskip -5mu/\mskip -3mu}_{29}$ & $133$ & $0$ & $1+3+9+8+2$ & odwracalny & tak \\
$12a_{729}$ & $1$ & $5$ & ${}^{167}{\mskip -5mu/\mskip -3mu}_{46}$ & $167$ & $-2$ & $1+2+6+4$ & odwracalny & tak \\
$12a_{730}$ & $2..3$ & $5$ & $3$ & $153$ & $-4$ & $1+2-6-4$ & chiralny & tak \\
$12a_{731}$ & $2$ & $5$ & ${}^{105}{\mskip -5mu/\mskip -3mu}_{22}$ & $105$ & $-4$ & $1-2-10-4$ & odwracalny & tak \\
$12a_{732}$ & $1$ & $5$ & ${}^{95}{\mskip -5mu/\mskip -3mu}_{18}$ & $95$ & $2$ & $1+4+7+3$ & odwracalny & tak \\
$12a_{733}$ & $2$ & $5$ & ${}^{73}{\mskip -5mu/\mskip -3mu}_{14}$ & $73$ & $-4$ & $1+2-7-3$ & odwracalny & tak \\
$12a_{734}$ & $2$ & $4$ & $3$ & $223$ & $-2$ & $1+0-6-6-2$ & chiralny & tak \\
$12a_{735}$ & $2$ & $6$ & $3$ & $99$ & $-2$ & $1+1-6$ & chiralny & tak \\
$12a_{736}$ & $1$ & $5$ & ${}^{141}{\mskip -5mu/\mskip -3mu}_{43}$ & $141$ & $0$ & $1-3-8-4$ & odwracalny & tak \\
$12a_{737}$ & $2$ & $4$ & $3$ & $165$ & $4$ & $1+3+7+7+2$ & odwracalny & tak \\
$12a_{738}$ & $2$ & $5$ & ${}^{119}{\mskip -5mu/\mskip -3mu}_{37}$ & $119$ & $2$ & $1+2+5+3$ & odwracalny & tak \\
$12a_{739}$ & $3$ & $4$ & $3$ & $147$ & $-6$ & $1+1-5-7-2$ & odwracalny & tak \\
$12a_{740}$ & $2$ & $5$ & ${}^{113}{\mskip -5mu/\mskip -3mu}_{35}$ & $113$ & $-4$ & $1+0-5-3$ & odwracalny & tak \\
$12a_{741}$ & $2..3$ & $5$ & $3$ & $237$ & $-4$ & $1+1-5-5$ & odwracalny & tak \\
$12a_{742}$ & $2$ & $4$ & $3$ & $135$ & $2$ & $1-2-9-8-2$ & odwracalny & tak \\
$12a_{743}$ & $2$ & $6$ & ${}^{79}{\mskip -5mu/\mskip -3mu}_{12}$ & $79$ & $-2$ & $1-4-6$ & odwracalny & tak \\
$12a_{744}$ & $1$ & $6$ & ${}^{61}{\mskip -5mu/\mskip -3mu}_{8}$ & $61$ & $0$ & $1+1+4$ & odwracalny & tak \\
$12a_{745}$ & $1$ & $6$ & ${}^{59}{\mskip -5mu/\mskip -3mu}_{8}$ & $59$ & $-2$ & $1-1-4$ & odwracalny & tak \\
$12a_{746}$ & $2$ & $4$ & $3$ & $217$ & $0$ & $1+2+6+6+2$ & odwracalny & tak \\
$12a_{747}$ & $2$ & $5$ & $3$ & $211$ & $2$ & $1+1+3+0-1$ & chiralny & tak \\
$12a_{748}$ & $2$ & $5$ & $3$ & $137$ & $0$ & $1-2-8-4$ & chiralny & tak \\
$12a_{749}$ & $2$ & $5$ & $3$ & $143$ & $2$ & $1+4+8+4$ & chiralny & tak \\
$12a_{750}$ & $2..3$ & $6$ & $4$ & $135$ & $-2$ & $1+2-8$ & odwracalny & tak \\
$12a_{751}$ & $1$ & $6$ & $3$ & $161$ & $0$ & $1+0+2-2$ & chiralny & tak \\
$12a_{752}$ & $2$ & $6$ & $3$ & $115$ & $-2$ & $1-3-8$ & chiralny & tak \\
$12a_{753}$ & $2$ & $6$ & $3$ & $85$ & $0$ & $1+3+6$ & chiralny & tak \\
$12a_{754}$ & $2$ & $5$ & $3$ & $263$ & $-2$ & $1+2+4+1-1$ & chiralny & tak \\
$12a_{755}$ & $2$ & $4$ & $3$ & $271$ & $-2$ & $1+0-5-5-2$ & chiralny & tak \\
$12a_{756}$ & $2$ & $5$ & $3$ & $187$ & $2$ & $1+3+5+0-1$ & chiralny & tak \\
$12a_{757}$ & $2$ & $4$ & $3$ & $163$ & $-2$ & $1+1-6-7-2$ & odwracalny & tak \\
$12a_{758}$ & $2$ & $5$ & ${}^{113}{\mskip -5mu/\mskip -3mu}_{31}$ & $113$ & $0$ & $1+0-5-3$ & odwracalny & tak \\
$12a_{759}$ & $2$ & $4$ & ${}^{61}{\mskip -5mu/\mskip -3mu}_{9}$ & $61$ & $4$ & $1+5+13+10+2$ & odwracalny & tak \\
$12a_{760}$ & $2$ & $5$ & ${}^{111}{\mskip -5mu/\mskip -3mu}_{34}$ & $111$ & $2$ & $1+4+10+4$ & odwracalny & tak \\
$12a_{761}$ & $1$ & $5$ & ${}^{139}{\mskip -5mu/\mskip -3mu}_{41}$ & $139$ & $2$ & $1+3+8+4$ & odwracalny & tak \\
$12a_{762}$ & $3..4$ & $4$ & ${}^{51}{\mskip -5mu/\mskip -3mu}_{7}$ & $51$ & $-6$ & $1+1-11-10-2$ & odwracalny & tak \\
$12a_{763}$ & $2..3$ & $5$ & ${}^{97}{\mskip -5mu/\mskip -3mu}_{30}$ & $97$ & $-4$ & $1+0-10-4$ & odwracalny & tak \\
$12a_{764}$ & $2..3$ & $5$ & ${}^{133}{\mskip -5mu/\mskip -3mu}_{39}$ & $133$ & $-4$ & $1-1-8-4$ & odwracalny & tak \\
$12a_{765}$ & $1$ & $5$ & $3$ & $277$ & $0$ & $1-1-3-1+1$ & chiralny & tak \\
$12a_{766}$ & $1$ & $5$ & $3$ & $213$ & $0$ & $1-1-3+0+1$ & chiralny & tak \\
$12a_{767}$ & $2..3$ & $6$ & $3$ & $123$ & $-2$ & $1-1-8$ & odwracalny & tak \\
$12a_{768}$ & $2$ & $5$ & $3$ & $183$ & $2$ & $1+2+5+0-1$ & chiralny & tak \\
$12a_{769}$ & $2$ & $6$ & $3$ & $189$ & $0$ & $1+1+4-2$ & chiralny & tak \\
$12a_{770}$ & $1$ & $6$ & $3$ & $199$ & $-2$ & $1-2-5+2$ & chiralny & tak \\
$12a_{771}$ & $2$ & $5$ & $3$ & $219$ & $2$ & $1+3+7+5$ & odwracalny & tak \\
$12a_{772}$ & $2$ & $5$ & $3$ & $129$ & $-4$ & $1+0-8-4$ & odwracalny & tak \\
$12a_{773}$ & $1$ & $6$ & ${}^{91}{\mskip -5mu/\mskip -3mu}_{20}$ & $91$ & $-2$ & $1-1-6$ & odwracalny & tak \\
$12a_{774}$ & $1$ & $6$ & ${}^{89}{\mskip -5mu/\mskip -3mu}_{16}$ & $89$ & $0$ & $1+2+6$ & odwracalny & tak \\
$12a_{775}$ & $2$ & $6$ & ${}^{87}{\mskip -5mu/\mskip -3mu}_{16}$ & $87$ & $-2$ & $1-2-6$ & odwracalny & tak \\
$12a_{776}$ & $2$ & $5$ & $3$ & $215$ & $-2$ & $1+2+3+0-1$ & chiralny & tak \\
$12a_{777}$ & $2$ & $5$ & $3$ & $177$ & $-4$ & $1+0-5+0+1$ & chiralny & tak \\
$12a_{778}$ & $2$ & $5$ & $3$ & $265$ & $0$ & $1-2-4-1+1$ & chiralny & tak \\
$12a_{779}$ & $2$ & $6$ & $3$ & $185$ & $0$ & $1+2+4-2$ & chiralny & tak \\
$12a_{780}$ & $3$ & $5$ & $3$ & $225$ & $-4$ & $1+0-6-5$ & odwracalny & tak \\
$12a_{781}$ & $1$ & $6$ & $3$ & $157$ & $0$ & $1+1+2-2$ & chiralny & tak \\
$12a_{782}$ & $2$ & $6$ & $3$ & $151$ & $-2$ & $1-2-2+2$ & chiralny & tak \\
$12a_{783}$ & $2$ & $6$ & $3$ & $187$ & $-2$ & $1-1-4+2$ & chiralny & tak \\
$12a_{784}$ & $2..3$ & $4$ & $3$ & $193$ & $0$ & $1+4+9+7+2$ & chiralny & tak \\
$12a_{785}$ & $2$ & $5$ & $3$ & $285$ & $-4$ & $1+1-2-1+1$ & odwracalny & tak \\
$12a_{786}$ & $2$ & $6$ & $3$ & $169$ & $0$ & $1+2+3-2$ & chiralny & tak \\
$12a_{787}$ & $2$ & $6$ & $3$ & $117$ & $0$ & $1+3+8$ & odwracalny & tak \\
$12a_{788}$ & $2$ & $5$ & $3$ & $299$ & $-2$ & $1-1-3+0-1$ & odwracalny & tak \\
$12a_{789}$ & $2..3$ & $4$ & $3$ & $109$ & $0$ & $1+5+12+9+2$ & odwracalny & tak \\
$12a_{790}$ & $2$ & $5$ & $3$ & $143$ & $-2$ & $1+4+8+4$ & odwracalny & tak \\
$12a_{791}$ & $2..3$ & $5$ & ${}^{63}{\mskip -5mu/\mskip -3mu}_{13}$ & $63$ & $2$ & $1+4+9+3$ & odwracalny & tak \\
$12a_{792}$ & $2$ & $6$ & ${}^{85}{\mskip -5mu/\mskip -3mu}_{24}$ & $85$ & $0$ & $1+3+6$ & odwracalny & tak \\
$12a_{793}$ & $1$ & $5$ & $3$ & $285$ & $0$ & $1+1-2-1+1$ & odwracalny & tak \\
$12a_{794}$ & $3..4$ & $4$ & $3$ & $91$ & $-6$ & $1-1-10-9-2$ & odwracalny & tak \\
$12a_{795}$ & $2..3$ & $5$ & $3$ & $137$ & $-4$ & $1-2-8-4$ & odwracalny & tak \\
$12a_{796}$ & $2..3$ & $5$ & ${}^{57}{\mskip -5mu/\mskip -3mu}_{11}$ & $57$ & $-4$ & $1-2-9-3$ & odwracalny & tak \\
$12a_{797}$ & $2$ & $6$ & ${}^{83}{\mskip -5mu/\mskip -3mu}_{24}$ & $83$ & $-2$ & $1-3-6$ & odwracalny & tak \\
$12a_{798}$ & $2..3$ & $5$ & $3$ & $213$ & $-4$ & $1-1-7-5$ & chiralny & tak \\
$12a_{799}$ & $1$ & $6$ & $3$ & $175$ & $-2$ & $1+0-3+2$ & odwracalny & tak \\
$12a_{800}$ & $2..3$ & $5$ & $3$ & $93$ & $-4$ & $1-3-7-3$ & odwracalny & tak \\
$12a_{801}$ & $3..4$ & $4$ & $4$ & $135$ & $-6$ & $1-2-9-8-2$ & odwracalny & tak \\
$12a_{802}$ & $2$ & $6$ & ${}^{47}{\mskip -5mu/\mskip -3mu}_{15}$ & $47$ & $-2$ & $1-4-4$ & odwracalny & tak \\
$12a_{803}$ & $1$ & $7$ & ${}^{21}{\mskip -5mu/\mskip -3mu}_{2}$ & $21$ & $0$ & $1-5$ & odwracalny & tak \\
$12a_{804}$ & $2$ & $5$ & $3$ & $249$ & $-4$ & $1+2+0+4+2$ & chiralny & tak \\
$12a_{805}$ & $2$ & $3$ & $3$ & $129$ & $4$ & $1+4+1-6-5-1$ & odwracalny & tak \\
$12a_{806}$ & $2$ & $4$ & $3$ & $219$ & $2$ & $1+3-1-5-2$ & odwracalny & tak \\
$12a_{807}$ & $1$ & $4$ & $3$ & $231$ & $2$ & $1+2-2-5-2$ & odwracalny & tak \\
$12a_{808}$ & $2$ & $5$ & $3$ & $141$ & $0$ & $1+1-3-3$ & odwracalny & tak \\
$12a_{809}$ & $2$ & $4$ & $3$ & $201$ & $-4$ & $1+2+1+5+2$ & odwracalny & tak \\
$12a_{810}$ & $2$ & $5$ & $3$ & $243$ & $-2$ & $1+1+1+4$ & odwracalny & tak \\
$12a_{811}$ & $4$ & $4$ & $3$ & $145$ & $-8$ & $1+8+19+14+3$ & odwracalny & tak \\
$12a_{812}$ & $2..3$ & $5$ & $3$ & $201$ & $-4$ & $1+2-7-5$ & chiralny & tak \\
$12a_{813}$ & $4$ & $4$ & $3$ & $125$ & $-8$ & $1+9+22+15+3$ & chiralny & tak \\
$12a_{814}$ & $3..4$ & $5$ & $3$ & $159$ & $-6$ & $1+8+16+6$ & chiralny & tak \\
$12a_{815}$ & $2..4$ & $3$ & $3$ & $95$ & $2$ & $1+8+20+18+7+1$ & chiralny & tak \\
$12a_{816}$ & $2..4$ & $4$ & $3$ & $149$ & $0$ & $1+7+15+9+2$ & chiralny & tak \\
$12a_{817}$ & $4$ & $4$ & $3$ & $145$ & $-8$ & $1+8+19+14+3$ & chiralny & tak \\
$12a_{818}$ & $2$ & $5$ & $3$ & $195$ & $-2$ & $1-3-1-1-1$ & chiralny & tak \\
$12a_{819}$ & $2$ & $3$ & $3$ & $169$ & $0$ & $1+2-1-7-5-1$ & -zwierciadlany & tak \\
$12a_{820}$ & $2$ & $4$ & $3$ & $179$ & $-2$ & $1+1-3-6-2$ & chiralny & tak \\
$12a_{821}$ & $1$ & $5$ & $3$ & $193$ & $0$ & $1+0-4-4$ & -zwierciadlany & tak \\
$12a_{822}$ & $3$ & $4$ & $3$ & $119$ & $-6$ & $1+2-7-8-2$ & odwracalny & tak \\
$12a_{823}$ & $2$ & $5$ & $3$ & $141$ & $-4$ & $1+1-7-4$ & odwracalny & tak \\
$12a_{824}$ & $2..3$ & $3$ & $3$ & $107$ & $2$ & $1+7+19+18+7+1$ & chiralny & tak \\
$12a_{825}$ & $2..3$ & $4$ & $3$ & $137$ & $0$ & $1+6+14+9+2$ & chiralny & tak \\
$12a_{826}$ & $2..3$ & $4$ & $3$ & $89$ & $0$ & $1+6+15+10+2$ & odwracalny & tak \\
$12a_{827}$ & $2..3$ & $5$ & $3$ & $99$ & $-2$ & $1+5+11+4$ & odwracalny & tak \\
$12a_{828}$ & $3..4$ & $5$ & $3$ & $195$ & $-6$ & $1+9+18+7$ & chiralny & tak \\
$12a_{829}$ & $3$ & $5$ & $3$ & $191$ & $-6$ & $1+4+1-5-2$ & chiralny & tak \\
$12a_{830}$ & $2$ & $4$ & $3$ & $205$ & $0$ & $1+5+10+7+2$ & chiralny & tak \\
$12a_{831}$ & $2$ & $4$ & $3$ & $205$ & $0$ & $1+5+10+7+2$ & chiralny & tak \\
$12a_{832}$ & $3$ & $5$ & $3$ & $191$ & $-6$ & $1+4+1-5-2$ & chiralny & tak \\
$12a_{833}$ & $2$ & $4$ & $3$ & $151$ & $-2$ & $1+2-5-7-2$ & chiralny & tak \\
$12a_{834}$ & $2$ & $5$ & $3$ & $173$ & $0$ & $1+1-5-4$ & chiralny & tak \\
$12a_{835}$ & $2..3$ & $3$ & $3$ & $71$ & $2$ & $1+6+17+17+7+1$ & odwracalny & tak \\
$12a_{836}$ & $2$ & $4$ & $3$ & $125$ & $0$ & $1+5+13+9+2$ & odwracalny & tak \\
$12a_{837}$ & $2$ & $4$ & $3$ & $145$ & $0$ & $1+4+10+8+2$ & odwracalny & tak \\
$12a_{838}$ & $4$ & $3$ & $3$ & $41$ & $-8$ & $1+2-9-15-7-1$ & odwracalny & tak \\
$12a_{839}$ & $3$ & $4$ & $3$ & $83$ & $-6$ & $1+1-9-9-2$ & odwracalny & tak \\
$12a_{840}$ & $3$ & $4$ & $3$ & $127$ & $-6$ & $1+0-8-8-2$ & odwracalny & tak \\
$12a_{841}$ & $2..3$ & $5$ & $3$ & $129$ & $-4$ & $1+0-8-4$ & odwracalny & tak \\
$12a_{842}$ & $2$ & $5$ & $3$ & $91$ & $-2$ & $1+3+7+3$ & odwracalny & tak \\
$12a_{843}$ & $2$ & $5$ & $3$ & $85$ & $-4$ & $1-1-7-3$ & odwracalny & tak \\
$12a_{844}$ & $2$ & $5$ & $3$ & $203$ & $-2$ & $1+3+4+0-1$ & chiralny & tak \\
$12a_{845}$ & $2..3$ & $5$ & $3$ & $111$ & $-2$ & $1+4+10+4$ & odwracalny & tak \\
$12a_{846}$ & $2$ & $5$ & $3$ & $203$ & $-2$ & $1+3+4+0-1$ & chiralny & tak \\
$12a_{847}$ & $2$ & $5$ & $3$ & $189$ & $-4$ & $1+1-4+0+1$ & chiralny & tak \\
$12a_{848}$ & $2$ & $4$ & $3$ & $161$ & $4$ & $1+4+7+7+2$ & chiralny & tak \\
$12a_{849}$ & $1$ & $5$ & $3$ & $187$ & $2$ & $1+3+5+4$ & chiralny & tak \\
$12a_{850}$ & $2$ & $3$ & $3$ & $89$ & $-4$ & $1+2-6-11-6-1$ & chiralny & tak \\
$12a_{851}$ & $1$ & $4$ & $3$ & $163$ & $-2$ & $1+1-6-7-2$ & chiralny & tak \\
$12a_{852}$ & $2$ & $4$ & $3$ & $207$ & $-2$ & $1+0-5-6-2$ & chiralny & tak \\
$12a_{853}$ & $2$ & $5$ & $3$ & $177$ & $0$ & $1+0-5-4$ & chiralny & tak \\
$12a_{854}$ & $2$ & $5$ & $3$ & $133$ & $0$ & $1-1-4-3$ & chiralny & tak \\
$12a_{855}$ & $2$ & $4$ & $3$ & $171$ & $-2$ & $1-1-3-6-2$ & chiralny & tak \\
$12a_{856}$ & $2$ & $5$ & $3$ & $233$ & $0$ & $1-2-2-4$ & chiralny & tak \\
$12a_{857}$ & $1$ & $4$ & $3$ & $239$ & $2$ & $1+0-3-5-2$ & chiralny & tak \\
$12a_{858}$ & $1$ & $5$ & $3$ & $165$ & $0$ & $1-1-2-3$ & chiralny & tak \\
$12a_{859}$ & $3$ & $3$ & $3$ & $79$ & $6$ & $1+4+8+11+6+1$ & chiralny & tak \\
$12a_{860}$ & $2$ & $4$ & $3$ & $149$ & $4$ & $1+3+6+7+2$ & chiralny & tak \\
$12a_{861}$ & $2$ & $4$ & $3$ & $201$ & $4$ & $1+2+5+6+2$ & chiralny & tak \\
$12a_{862}$ & $2..3$ & $5$ & $3$ & $183$ & $2$ & $1+2+5+4$ & chiralny & tak \\
$12a_{863}$ & $2$ & $5$ & $3$ & $131$ & $2$ & $1+1+4+3$ & chiralny & tak \\
$12a_{864}$ & $1$ & $3$ & $3$ & $159$ & $2$ & $1+0+2+7+5+1$ & chiralny & tak \\
$12a_{865}$ & $1$ & $4$ & $3$ & $229$ & $0$ & $1-1+2+5+2$ & chiralny & tak \\
$12a_{866}$ & $2$ & $4$ & $3$ & $295$ & $2$ & $1+2-2-4-2$ & chiralny & tak \\
$12a_{867}$ & $2$ & $5$ & $3$ & $313$ & $0$ & $1+2+0-1+1$ & chiralny & tak \\
$12a_{868}$ & $2$ & $5$ & $3$ & $377$ & $0$ & $1-2-1-2+1$ & -zwierciadlany & tak \\
$12a_{869}$ & $2..3$ & $3$ & $3$ & $161$ & $4$ & $1+4-1-7-5-1$ & odwracalny & tak \\
$12a_{870}$ & $2$ & $4$ & $3$ & $187$ & $2$ & $1+3-3-6-2$ & chiralny & tak \\
$12a_{871}$ & $2$ & $4$ & $3$ & $223$ & $2$ & $1+4-1-5-2$ & chiralny & tak \\
$12a_{872}$ & $2$ & $5$ & $3$ & $197$ & $0$ & $1+3-3-4$ & chiralny & tak \\
$12a_{873}$ & $2$ & $5$ & $3$ & $189$ & $0$ & $1+1-8-5$ & odwracalny & tak \\
$12a_{874}$ & $2$ & $4$ & $3$ & $283$ & $2$ & $1+3-1-4-2$ & chiralny & tak \\
$12a_{875}$ & $1$ & $5$ & $3$ & $307$ & $-2$ & $1+1+1+5$ & chiralny & tak \\
$12a_{876}$ & $4$ & $4$ & $3$ & $105$ & $-8$ & $1+10+25+16+3$ & odwracalny & tak \\
$12a_{877}$ & $3..4$ & $5$ & $3$ & $131$ & $-6$ & $1+9+18+6$ & odwracalny & tak \\
$12a_{878}$ & $2..3$ & $3$ & $3$ & $75$ & $2$ & $1+7+17+17+7+1$ & odwracalny & tak \\
$12a_{879}$ & $2..3$ & $4$ & $3$ & $121$ & $0$ & $1+6+13+9+2$ & odwracalny & tak \\
$12a_{880}$ & $2..3$ & $6$ & $3$ & $161$ & $-4$ & $1+8+12$ & odwracalny & tak \\
$12a_{881}$ & $2..3$ & $4$ & $3$ & $89$ & $4$ & $1+6+11+9+2$ & odwracalny & tak \\
$12a_{882}$ & $2$ & $5$ & $3$ & $147$ & $2$ & $1+5+8+4$ & odwracalny & tak \\
$12a_{883}$ & $2$ & $5$ & $3$ & $185$ & $0$ & $1+2-4-4$ & odwracalny & tak \\
$12a_{884}$ & $2$ & $5$ & $3$ & $299$ & $-2$ & $1-1+1+1-1$ & chiralny & tak \\
$12a_{885}$ & $2$ & $5$ & $3$ & $287$ & $-2$ & $1+0+2+1-1$ & chiralny & tak \\
$12a_{886}$ & $2..3$ & $5$ & $3$ & $207$ & $2$ & $1-4-2-1-1$ & odwracalny & tak \\
$12a_{887}$ & $1$ & $5$ & $3$ & $289$ & $0$ & $1+0-2-1+1$ & -zwierciadlany & tak \\
$12a_{888}$ & $1$ & $5$ & $3$ & $303$ & $-2$ & $1+0+1+1-1$ & chiralny & tak \\
$12a_{889}$ & $2$ & $4$ & $3$ & $167$ & $2$ & $1+2-2-6-2$ & chiralny & tak \\
$12a_{890}$ & $2$ & $5$ & $3$ & $205$ & $0$ & $1+1-3-4$ & -zwierciadlany & tak \\
$12a_{891}$ & $2$ & $4$ & $3$ & $233$ & $-4$ & $1+2+3+5+2$ & chiralny & tak \\
$12a_{892}$ & $2$ & $5$ & $3$ & $227$ & $-2$ & $1+1+2+4$ & chiralny & tak \\
$12a_{893}$ & $1$ & $5$ & $3$ & $309$ & $0$ & $1-1-1-1+1$ & chiralny & tak \\
$12a_{894}$ & $2$ & $4$ & $3$ & $273$ & $0$ & $1+0+1+4+2$ & chiralny & tak \\
$12a_{895}$ & $2$ & $4$ & $3$ & $243$ & $-2$ & $1+1-7-6-2$ & chiralny & tak \\
$12a_{896}$ & $2..3$ & $4$ & $3$ & $203$ & $-2$ & $1+3+0-5-2$ & odwracalny & tak \\
$12a_{897}$ & $2$ & $5$ & $3$ & $217$ & $0$ & $1+2-2-4$ & odwracalny & tak \\
$12a_{898}$ & $2..3$ & $3$ & $3$ & $167$ & $2$ & $1+6+15+14+6+1$ & chiralny & tak \\
$12a_{899}$ & $2..3$ & $4$ & $3$ & $173$ & $0$ & $1+5+12+8+2$ & chiralny & tak \\
$12a_{900}$ & $3..4$ & $5$ & $3$ & $247$ & $-6$ & $1+10+19+8$ & chiralny & tak \\
$12a_{901}$ & $2..3$ & $5$ & $3$ & $223$ & $2$ & $1+4+7+1-1$ & chiralny & tak \\
$12a_{902}$ & $2$ & $6$ & $3$ & $213$ & $0$ & $1+3+6-2$ & chiralny & tak \\
$12a_{903}$ & $2$ & $5$ & $3$ & $249$ & $-4$ & $1+2-4-5$ & chiralny & tak \\
$12a_{904}$ & $2$ & $4$ & $3$ & $183$ & $-2$ & $1+2-3-6-2$ & odwracalny & tak \\
$12a_{905}$ & $2$ & $5$ & $3$ & $189$ & $0$ & $1+1-4-4$ & odwracalny & tak \\
$12a_{906}$ & $2$ & $5$ & $3$ & $245$ & $0$ & $1-5-2+0+1$ & -zwierciadlany & tak \\
$12a_{907}$ & $3$ & $4$ & $3$ & $225$ & $-4$ & $1+4+3+5+2$ & odwracalny & tak \\
$12a_{908}$ & $2$ & $5$ & $3$ & $235$ & $-2$ & $1+3+2+4$ & odwracalny & tak \\
$12a_{909}$ & $2$ & $3$ & $3$ & $105$ & $-4$ & $1-2-10-12-6-1$ & chiralny & tak \\
$12a_{910}$ & $2$ & $4$ & $3$ & $179$ & $-2$ & $1-3-8-7-2$ & chiralny & tak \\
$12a_{911}$ & $2..3$ & $4$ & $3$ & $191$ & $-2$ & $1-4-9-7-2$ & chiralny & tak \\
$12a_{912}$ & $2..3$ & $5$ & $3$ & $117$ & $0$ & $1-5-6-3$ & chiralny & tak \\
$12a_{913}$ & $2$ & $5$ & $3$ & $219$ & $-2$ & $1-1-2-1-1$ & chiralny & tak \\
$12a_{914}$ & $2$ & $5$ & $3$ & $255$ & $2$ & $1+4+5+5$ & chiralny & tak \\
$12a_{915}$ & $2$ & $6$ & $3$ & $271$ & $-2$ & $1+0-5+3$ & chiralny & tak \\
$12a_{916}$ & $2..3$ & $3$ & $3$ & $147$ & $2$ & $1+5+12+13+6+1$ & chiralny & tak \\
$12a_{917}$ & $2$ & $4$ & $3$ & $193$ & $0$ & $1+4+9+7+2$ & chiralny & tak \\
$12a_{918}$ & $2$ & $5$ & $3$ & $247$ & $2$ & $1+2+5+1-1$ & chiralny & tak \\
$12a_{919}$ & $1$ & $6$ & $3$ & $189$ & $0$ & $1+1+4-2$ & chiralny & tak \\
$12a_{920}$ & $2$ & $3$ & $3$ & $125$ & $-4$ & $1+1-8-12-6-1$ & chiralny & tak \\
$12a_{921}$ & $3$ & $4$ & $3$ & $175$ & $-2$ & $1+0-7-7-2$ & chiralny & tak \\
$12a_{922}$ & $2$ & $4$ & $3$ & $283$ & $-2$ & $1-1-6-5-2$ & chiralny & tak \\
$12a_{923}$ & $3$ & $4$ & $3$ & $135$ & $-6$ & $1+2-4-7-2$ & chiralny & tak \\
$12a_{924}$ & $2$ & $5$ & $3$ & $189$ & $-4$ & $1+1-4-4$ & chiralny & tak \\
$12a_{925}$ & $2$ & $4$ & $3$ & $245$ & $0$ & $1+3+8+6+2$ & chiralny & tak \\
$12a_{926}$ & $2$ & $4$ & $3$ & $127$ & $-2$ & $1+0-8-8-2$ & chiralny & tak \\
$12a_{927}$ & $2$ & $5$ & $3$ & $165$ & $0$ & $1-1-6-4$ & chiralny & tak \\
$12a_{928}$ & $1$ & $5$ & $3$ & $309$ & $0$ & $1-1+3+0+1$ & chiralny & tak \\
$12a_{929}$ & $2$ & $4$ & $3$ & $203$ & $-2$ & $1-1-5-6-2$ & chiralny & tak \\
$12a_{930}$ & $3$ & $4$ & $3$ & $147$ & $-6$ & $1+1-5-7-2$ & chiralny & tak \\
$12a_{931}$ & $2$ & $5$ & $3$ & $177$ & $-4$ & $1+0-5-4$ & chiralny & tak \\
$12a_{932}$ & $2$ & $5$ & $3$ & $211$ & $-2$ & $1-3-2-1-1$ & chiralny & tak \\
$12a_{933}$ & $2$ & $5$ & $3$ & $227$ & $2$ & $1+1+2+0-1$ & chiralny & tak \\
$12a_{934}$ & $2$ & $4$ & $3$ & $235$ & $-2$ & $1-1-7-6-2$ & chiralny & tak \\
$12a_{935}$ & $1$ & $5$ & $3$ & $269$ & $0$ & $1+1+1+0+1$ & chiralny & tak \\
$12a_{936}$ & $2$ & $5$ & $3$ & $201$ & $0$ & $1-2-4-4$ & chiralny & tak \\
$12a_{937}$ & $3$ & $4$ & $3$ & $79$ & $-6$ & $1+0-9-9-2$ & odwracalny & tak \\
$12a_{938}$ & $2..3$ & $5$ & $3$ & $133$ & $-4$ & $1-1-8-4$ & odwracalny & tak \\
$12a_{939}$ & $2$ & $6$ & $3$ & $159$ & $-2$ & $1+0-10$ & odwracalny & tak \\
$12a_{940}$ & $2..3$ & $6$ & $3$ & $165$ & $0$ & $1-5+1-2$ & odwracalny & tak \\
$12a_{941}$ & $2$ & $4$ & $3$ & $135$ & $-2$ & $1-2-9-8-2$ & chiralny & tak \\
$12a_{942}$ & $2$ & $5$ & $3$ & $173$ & $0$ & $1-3-6-4$ & chiralny & tak \\
$12a_{943}$ & $2$ & $5$ & $3$ & $221$ & $0$ & $1-3-3-4$ & chiralny & tak \\
$12a_{944}$ & $1$ & $4$ & $3$ & $165$ & $0$ & $1+3+7+7+2$ & chiralny & tak \\
$12a_{945}$ & $2$ & $5$ & $3$ & $199$ & $-2$ & $1+2+4+4$ & chiralny & tak \\
$12a_{946}$ & $3$ & $4$ & $3$ & $151$ & $-6$ & $1+2-5-7-2$ & chiralny & tak \\
$12a_{947}$ & $2..3$ & $5$ & $3$ & $157$ & $-4$ & $1+1-6-4$ & chiralny & tak \\
$12a_{948}$ & $2$ & $5$ & $3$ & $251$ & $-2$ & $1+3+5+1-1$ & chiralny & tak \\
$12a_{949}$ & $2$ & $4$ & $3$ & $207$ & $-2$ & $1+0-5-6-2$ & chiralny & tak \\
$12a_{950}$ & $2$ & $5$ & $3$ & $197$ & $0$ & $1-1-4-4$ & chiralny & tak \\
$12a_{951}$ & $2$ & $5$ & $3$ & $295$ & $2$ & $1+2+2+1-1$ & chiralny & tak \\
$12a_{952}$ & $3$ & $4$ & $3$ & $123$ & $-6$ & $1+3-3-7-2$ & chiralny & tak \\
$12a_{953}$ & $2$ & $5$ & $3$ & $185$ & $-4$ & $1+2-4-4$ & chiralny & tak \\
$12a_{954}$ & $1$ & $4$ & $3$ & $211$ & $-2$ & $1+1-5-6-2$ & chiralny & tak \\
$12a_{955}$ & $2$ & $5$ & $3$ & $193$ & $0$ & $1+0-4-4$ & chiralny & tak \\
$12a_{956}$ & $2$ & $4$ & $3$ & $209$ & $4$ & $1+4+6+6+2$ & chiralny & tak \\
$12a_{957}$ & $2..3$ & $5$ & $3$ & $203$ & $2$ & $1+3+4+4$ & chiralny & tak \\
$12a_{958}$ & $2$ & $4$ & $3$ & $197$ & $4$ & $1+3+5+6+2$ & chiralny & tak \\
$12a_{959}$ & $2$ & $5$ & $3$ & $215$ & $2$ & $1+2+3+4$ & chiralny & tak \\
$12a_{960}$ & $2$ & $5$ & $3$ & $281$ & $0$ & $1-2+1-4$ & -zwierciadlany & tak \\
$12a_{961}$ & $2$ & $5$ & $3$ & $277$ & $0$ & $1-1-3-1+1$ & chiralny & tak \\
$12a_{962}$ & $2$ & $5$ & $3$ & $239$ & $-2$ & $1+4+6+1-1$ & chiralny & tak \\
$12a_{963}$ & $2$ & $5$ & $3$ & $221$ & $0$ & $1+1-6-5$ & chiralny & tak \\
$12a_{964}$ & $2$ & $5$ & $3$ & $255$ & $2$ & $1+0+4+1-1$ & chiralny & tak \\
$12a_{965}$ & $2$ & $5$ & $3$ & $297$ & $0$ & $1-2-2-1+1$ & chiralny & tak \\
$12a_{966}$ & $1$ & $4$ & $3$ & $217$ & $0$ & $1+2+6+6+2$ & chiralny & tak \\
$12a_{967}$ & $2..3$ & $5$ & $3$ & $253$ & $4$ & $1+5-3-5$ & chiralny & tak \\
$12a_{968}$ & $2$ & $5$ & $3$ & $193$ & $0$ & $1-4-5+0+1$ & chiralny & tak \\
$12a_{969}$ & $2$ & $6$ & $3$ & $139$ & $-2$ & $1-5-2+2$ & chiralny & tak \\
$12a_{970}$ & $2$ & $4$ & $3$ & $105$ & $4$ & $1+2+7+8+2$ & chiralny & tak \\
$12a_{971}$ & $2$ & $5$ & $3$ & $163$ & $2$ & $1+1+6+4$ & chiralny & tak \\
$12a_{972}$ & $2$ & $4$ & $3$ & $183$ & $2$ & $1-2-4-6-2$ & chiralny & tak \\
$12a_{973}$ & $3..4$ & $5$ & $3$ & $243$ & $-6$ & $1+9+19+8$ & odwracalny & tak \\
$12a_{974}$ & $2..3$ & $6$ & $3$ & $177$ & $-4$ & $1+8+13$ & odwracalny & tak \\
$12a_{975}$ & $2$ & $6$ & $3$ & $225$ & $0$ & $1+0-2-4$ & odwracalny & tak \\
$12a_{976}$ & $1$ & $6$ & $3$ & $279$ & $2$ & $1+2-1+4$ & odwracalny & tak \\
$12a_{977}$ & $3$ & $4$ & $3$ & $175$ & $-6$ & $1+4-2-6-2$ & odwracalny & tak \\
$12a_{978}$ & $2..3$ & $5$ & $3$ & $133$ & $-4$ & $1+3-3-3$ & odwracalny & tak \\
$12a_{979}$ & $2$ & $4$ & $3$ & $225$ & $0$ & $1+4+7+6+2$ & chiralny & tak \\
$12a_{980}$ & $2$ & $4$ & $3$ & $257$ & $4$ & $1+4+5+5+2$ & chiralny & tak \\
$12a_{981}$ & $2..3$ & $3$ & $3$ & $135$ & $2$ & $1+6+13+13+6+1$ & chiralny & tak \\
$12a_{982}$ & $2$ & $4$ & $3$ & $189$ & $0$ & $1+5+9+7+2$ & chiralny & tak \\
$12a_{983}$ & $2..3$ & $5$ & $3$ & $229$ & $-4$ & $1+3-5-5$ & chiralny & tak \\
$12a_{984}$ & $3$ & $3$ & $3$ & $103$ & $6$ & $1+6+11+12+6+1$ & chiralny & tak \\
$12a_{985}$ & $2..3$ & $4$ & $3$ & $157$ & $4$ & $1+5+7+7+2$ & chiralny & tak \\
$12a_{986}$ & $2$ & $4$ & $3$ & $285$ & $0$ & $1+1+2+4+2$ & chiralny & tak \\
$12a_{987}$ & $2$ & $4$ & $3$ & $225$ & $4$ & $1+4+7+6+2$ & chiralny & tak \\
$12a_{988}$ & $2..3$ & $4$ & $3$ & $125$ & $4$ & $1+5+9+8+2$ & chiralny & tak \\
$12a_{989}$ & $2..3$ & $5$ & $3$ & $175$ & $2$ & $1+4+6+4$ & chiralny & tak \\
$12a_{990}$ & $2$ & $5$ & $3$ & $225$ & $0$ & $1+0-2+0+1$ & -zwierciadlany & tak \\
$12a_{991}$ & $2$ & $5$ & $3$ & $209$ & $4$ & $1+0-3+0+1$ & chiralny & tak \\
$12a_{992}$ & $2$ & $5$ & $3$ & $265$ & $0$ & $1-2-4-1+1$ & chiralny & tak \\
$12a_{993}$ & $1$ & $4$ & $3$ & $193$ & $0$ & $1+0+4+6+2$ & chiralny & tak \\
$12a_{994}$ & $2$ & $4$ & $3$ & $287$ & $-2$ & $1+0-2-4-2$ & chiralny & tak \\
$12a_{995}$ & $3..4$ & $5$ & $3$ & $227$ & $-6$ & $1+9+20+8$ & chiralny & tak \\
$12a_{996}$ & $2..3$ & $6$ & $3$ & $193$ & $-4$ & $1+8+14$ & chiralny & tak \\
$12a_{997}$ & $2$ & $4$ & $3$ & $215$ & $-2$ & $1-2-6-6-2$ & chiralny & tak \\
$12a_{998}$ & $1$ & $4$ & $3$ & $251$ & $-2$ & $1-1-4-5-2$ & chiralny & tak \\
$12a_{999}$ & $2..3$ & $3$ & $3$ & $171$ & $2$ & $1+7+15+14+6+1$ & odwracalny & tak \\
$12a_{1000}$ & $2..3$ & $4$ & $3$ & $153$ & $0$ & $1+6+11+8+2$ & odwracalny & tak \\
$12a_{1001}$ & $2$ & $5$ & $3$ & $189$ & $0$ & $1-3-5-4$ & chiralny & tak \\
$12a_{1002}$ & $2$ & $3$ & $3$ & $183$ & $2$ & $1+2+5+8+5+1$ & chiralny & tak \\
$12a_{1003}$ & $2$ & $4$ & $3$ & $233$ & $4$ & $1+6+8+6+2$ & chiralny & tak \\
$12a_{1004}$ & $3..4$ & $5$ & $3$ & $259$ & $-6$ & $1+9+18+8$ & odwracalny & tak \\
$12a_{1005}$ & $1$ & $4$ & $3$ & $241$ & $0$ & $1+0+3+5+2$ & chiralny & tak \\
$12a_{1006}$ & $2$ & $4$ & $3$ & $205$ & $0$ & $1+1+5+6+2$ & chiralny & tak \\
$12a_{1007}$ & $2$ & $4$ & $3$ & $143$ & $2$ & $1+0-5-7-2$ & chiralny & tak \\
$12a_{1008}$ & $1$ & $5$ & $3$ & $197$ & $0$ & $1-1-4-4$ & -zwierciadlany & tak \\
$12a_{1009}$ & $2..3$ & $4$ & $3$ & $137$ & $-4$ & $1+2+5+7+2$ & chiralny & tak \\
$12a_{1010}$ & $2$ & $5$ & $3$ & $195$ & $-2$ & $1+1+4+4$ & odwracalny & tak \\
$12a_{1011}$ & $1$ & $3$ & $3$ & $121$ & $0$ & $1-2-9-12-6-1$ & chiralny & tak \\
$12a_{1012}$ & $2$ & $4$ & $3$ & $163$ & $-2$ & $1-3-7-7-2$ & chiralny & tak \\
$12a_{1013}$ & $2$ & $3$ & $3$ & $119$ & $-2$ & $1+2+9+12+6+1$ & chiralny & tak \\
$12a_{1014}$ & $2$ & $4$ & $3$ & $173$ & $0$ & $1+1+7+7+2$ & chiralny & tak \\
$12a_{1015}$ & $2$ & $4$ & $3$ & $131$ & $2$ & $1-3-9-8-2$ & chiralny & tak \\
$12a_{1016}$ & $2..3$ & $5$ & $3$ & $177$ & $0$ & $1-4-6-4$ & chiralny & tak \\
$12a_{1017}$ & $2..3$ & $4$ & $3$ & $109$ & $-4$ & $1+1+7+8+2$ & chiralny & tak \\
$12a_{1018}$ & $2$ & $5$ & $3$ & $159$ & $-2$ & $1+0+6+4$ & chiralny & tak \\
$12a_{1019}$ & $2$ & $5$ & $3$ & $361$ & $0$ & $1+2-1-2+1$ & całkowicie & tak \\
$12a_{1020}$ & $2$ & $5$ & $3$ & $319$ & $-2$ & $1+0+0+1-1$ & odwracalny & tak \\
$12a_{1021}$ & $2$ & $5$ & $3$ & $339$ & $-2$ & $1+1-1+1-1$ & chiralny & tak \\
$12a_{1022}$ & $2$ & $6$ & $3$ & $243$ & $-2$ & $1+1-3+3$ & odwracalny & tak \\
$12a_{1023}$ & $1$ & $4$ & ${}^{127}{\mskip -5mu/\mskip -3mu}_{29}$ & $127$ & $-2$ & $1+0-8-8-2$ & odwracalny & tak \\
$12a_{1024}$ & $2$ & $5$ & ${}^{149}{\mskip -5mu/\mskip -3mu}_{40}$ & $149$ & $0$ & $1-1-7-4$ & odwracalny & tak \\
$12a_{1025}$ & $2$ & $4$ & $3$ & $255$ & $2$ & $1+4+1-4-2$ & odwracalny & tak \\
$12a_{1026}$ & $2$ & $5$ & $3$ & $165$ & $0$ & $1+3-1-3$ & odwracalny & tak \\
$12a_{1027}$ & $2$ & $3$ & $3$ & $83$ & $2$ & $1+5+16+17+7+1$ & odwracalny & tak \\
$12a_{1028}$ & $2..3$ & $4$ & $3$ & $113$ & $0$ & $1+4+12+9+2$ & odwracalny & tak \\
$12a_{1029}$ & $2$ & $4$ & ${}^{81}{\mskip -5mu/\mskip -3mu}_{19}$ & $81$ & $0$ & $1+4+14+10+2$ & odwracalny & tak \\
$12a_{1030}$ & $2$ & $5$ & ${}^{91}{\mskip -5mu/\mskip -3mu}_{19}$ & $91$ & $-2$ & $1+3+11+4$ & odwracalny & tak \\
$12a_{1031}$ & $2..3$ & $4$ & $3$ & $97$ & $4$ & $1+4+11+9+2$ & odwracalny & tak \\
$12a_{1032}$ & $2$ & $5$ & $3$ & $139$ & $2$ & $1+3+8+4$ & odwracalny & tak \\
$12a_{1033}$ & $2$ & $5$ & ${}^{107}{\mskip -5mu/\mskip -3mu}_{25}$ & $107$ & $2$ & $1+3+10+4$ & odwracalny & tak \\
$12a_{1034}$ & $1$ & $6$ & ${}^{121}{\mskip -5mu/\mskip -3mu}_{32}$ & $121$ & $0$ & $1+2+8$ & odwracalny & tak \\
$12a_{1035}$ & $3..4$ & $5$ & $3$ & $155$ & $-6$ & $1+11+21+7$ & odwracalny & tak \\
$12a_{1036}$ & $3$ & $4$ & $3$ & $159$ & $-6$ & $1+4-1-6-2$ & chiralny & tak \\
$12a_{1037}$ & $2..4$ & $6$ & $3$ & $185$ & $-4$ & $1+10+14$ & odwracalny & tak \\
$12a_{1038}$ & $2$ & $5$ & $3$ & $197$ & $-4$ & $1+3-3-4$ & chiralny & tak \\
$12a_{1039}$ & $2..3$ & $5$ & ${}^{137}{\mskip -5mu/\mskip -3mu}_{37}$ & $137$ & $0$ & $1-6-5+1+1$ & całkowicie & tak \\
$12a_{1040}$ & $2..3$ & $6$ & ${}^{115}{\mskip -5mu/\mskip -3mu}_{26}$ & $115$ & $2$ & $1-7-1+2$ & odwracalny & tak \\
$12a_{1041}$ & $2$ & $4$ & $3$ & $265$ & $-4$ & $1+2+1+4+2$ & odwracalny & tak \\
$12a_{1042}$ & $2$ & $4$ & $3$ & $253$ & $0$ & $1+1+4+5+2$ & odwracalny & tak \\
$12a_{1043}$ & $2$ & $5$ & $3$ & $207$ & $-2$ & $1+0+3+4$ & odwracalny & tak \\
$12a_{1044}$ & $2$ & $5$ & $3$ & $195$ & $-2$ & $1+1+0+3$ & odwracalny & tak \\
$12a_{1045}$ & $2$ & $5$ & $3$ & $161$ & $-4$ & $1+0-6+0+1$ & odwracalny & tak \\
$12a_{1046}$ & $2$ & $6$ & $3$ & $203$ & $-2$ & $1-1-5+2$ & odwracalny & tak \\
$12a_{1047}$ & $2$ & $3$ & $3$ & $143$ & $2$ & $1+4+12+13+6+1$ & chiralny & tak \\
$12a_{1048}$ & $1$ & $4$ & $3$ & $197$ & $0$ & $1+3+9+7+2$ & chiralny & tak \\
$12a_{1049}$ & $2$ & $5$ & $3$ & $215$ & $2$ & $1+2+7+1-1$ & chiralny & tak \\
$12a_{1050}$ & $2..3$ & $5$ & $3$ & $285$ & $0$ & $1-3+1+0+1$ & odwracalny & tak \\
$12a_{1051}$ & $2$ & $3$ & $3$ & $113$ & $-4$ & $1+0-9-12-6-1$ & chiralny & tak \\
$12a_{1052}$ & $2$ & $4$ & $3$ & $187$ & $-2$ & $1-1-8-7-2$ & chiralny & tak \\
$12a_{1053}$ & $2$ & $4$ & $3$ & $233$ & $0$ & $1+2+7+6+2$ & chiralny & tak \\
$12a_{1054}$ & $2$ & $5$ & $3$ & $227$ & $2$ & $1+1+6+1-1$ & chiralny & tak \\
$12a_{1055}$ & $2$ & $6$ & $3$ & $209$ & $0$ & $1+0+5-2$ & chiralny & tak \\
$12a_{1056}$ & $1$ & $4$ & $3$ & $247$ & $-2$ & $1-2-8-6-2$ & chiralny & tak \\
$12a_{1057}$ & $2$ & $6$ & $3$ & $249$ & $0$ & $1+2+4-3$ & chiralny & tak \\
$12a_{1058}$ & $2$ & $6$ & $3$ & $221$ & $0$ & $1+1+6-2$ & chiralny & tak \\
$12a_{1059}$ & $3$ & $4$ & $3$ & $183$ & $-6$ & $1+2-3-6-2$ & chiralny & tak \\
$12a_{1060}$ & $1$ & $4$ & $3$ & $183$ & $-2$ & $1-2-8-7-2$ & chiralny & tak \\
$12a_{1061}$ & $2$ & $4$ & $3$ & $267$ & $-2$ & $1-1-5-5-2$ & chiralny & tak \\
$12a_{1062}$ & $2$ & $5$ & $3$ & $141$ & $-4$ & $1+1-3-3$ & chiralny & tak \\
$12a_{1063}$ & $1$ & $5$ & $3$ & $109$ & $0$ & $1-3-6-3$ & chiralny & tak \\
$12a_{1064}$ & $2$ & $4$ & $3$ & $167$ & $-2$ & $1-2-7-7-2$ & chiralny & tak \\
$12a_{1065}$ & $2$ & $5$ & $3$ & $301$ & $0$ & $1+1+3+0+1$ & chiralny & tak \\
$12a_{1066}$ & $2..3$ & $4$ & $3$ & $163$ & $-2$ & $1-3-11-8-2$ & chiralny & tak \\
$12a_{1067}$ & $2$ & $5$ & $3$ & $275$ & $-2$ & $1-3-2+0-1$ & chiralny & tak \\
$12a_{1068}$ & $2$ & $5$ & $3$ & $145$ & $0$ & $1-4-8-4$ & chiralny & tak \\
$12a_{1069}$ & $2$ & $6$ & $3$ & $235$ & $-2$ & $1-1-7+2$ & chiralny & tak \\
$12a_{1070}$ & $2$ & $5$ & $3$ & $263$ & $-2$ & $1-2-1+0-1$ & chiralny & tak \\
$12a_{1071}$ & $2$ & $4$ & $3$ & $223$ & $-2$ & $1+0-6-6-2$ & chiralny & tak \\
$12a_{1072}$ & $2$ & $4$ & $3$ & $241$ & $4$ & $1+4+4+5+2$ & chiralny & tak \\
$12a_{1073}$ & $2$ & $5$ & $3$ & $213$ & $-4$ & $1+3-2-4$ & chiralny & tak \\
$12a_{1074}$ & $2$ & $5$ & $3$ & $193$ & $-4$ & $1+0-4+0+1$ & chiralny & tak \\
$12a_{1075}$ & $2$ & $5$ & $3$ & $181$ & $0$ & $1-1-5-4$ & chiralny & tak \\
$12a_{1076}$ & $1$ & $5$ & $3$ & $273$ & $0$ & $1+0+1+0+1$ & chiralny & tak \\
$12a_{1077}$ & $2$ & $5$ & $3$ & $171$ & $2$ & $1+3+2+3$ & chiralny & tak \\
$12a_{1078}$ & $2$ & $4$ & $3$ & $277$ & $0$ & $1+3+6+5+2$ & chiralny & tak \\
$12a_{1079}$ & $2$ & $4$ & $3$ & $267$ & $-2$ & $1-1-5-5-2$ & chiralny & tak \\
$12a_{1080}$ & $2..3$ & $5$ & $3$ & $201$ & $-4$ & $1-2-4+0+1$ & chiralny & tak \\
$12a_{1081}$ & $1$ & $5$ & $3$ & $217$ & $0$ & $1-2-3+0+1$ & chiralny & tak \\
$12a_{1082}$ & $2$ & $5$ & $3$ & $185$ & $-4$ & $1-2-5+0+1$ & chiralny & tak \\
$12a_{1083}$ & $2$ & $4$ & $3$ & $169$ & $0$ & $1+2+7+7+2$ & chiralny & tak \\
$12a_{1084}$ & $2..3$ & $5$ & $3$ & $145$ & $-4$ & $1-4-8+0+1$ & chiralny & tak \\
$12a_{1085}$ & $2..3$ & $6$ & $3$ & $187$ & $-2$ & $1-5-5+2$ & chiralny & tak \\
$12a_{1086}$ & $2$ & $5$ & $3$ & $199$ & $2$ & $1+2+4+4$ & chiralny & tak \\
$12a_{1087}$ & $2$ & $5$ & $3$ & $225$ & $0$ & $1-4-3+0+1$ & chiralny & tak \\
$12a_{1088}$ & $2$ & $5$ & $3$ & $253$ & $0$ & $1-3-5-1+1$ & chiralny & tak \\
$12a_{1089}$ & $2..3$ & $5$ & $3$ & $157$ & $-4$ & $1-3-7+0+1$ & chiralny & tak \\
$12a_{1090}$ & $2$ & $6$ & $3$ & $175$ & $-2$ & $1-4-4+2$ & chiralny & tak \\
$12a_{1091}$ & $2$ & $4$ & $3$ & $237$ & $0$ & $1+1+7+6+2$ & chiralny & tak \\
$12a_{1092}$ & $2$ & $5$ & $3$ & $243$ & $-2$ & $1+1+5+1-1$ & chiralny & tak \\
$12a_{1093}$ & $2$ & $5$ & $3$ & $207$ & $-2$ & $1+0+3+0-1$ & chiralny & tak \\
$12a_{1094}$ & $2$ & $6$ & $3$ & $147$ & $-2$ & $1-3-2+2$ & chiralny & tak \\
$12a_{1095}$ & $2$ & $5$ & $3$ & $99$ & $2$ & $1+1+6+3$ & chiralny & tak \\
$12a_{1096}$ & $2$ & $5$ & $3$ & $271$ & $2$ & $1+0-1+0-1$ & chiralny & tak \\
$12a_{1097}$ & $2..4$ & $6$ & $3$ & $217$ & $-4$ & $1+10+16$ & chiralny & tak \\
$12a_{1098}$ & $2$ & $5$ & $3$ & $293$ & $-4$ & $1+3-1-5$ & chiralny & tak \\
$12a_{1099}$ & $2..3$ & $4$ & $3$ & $217$ & $4$ & $1+6+7+6+2$ & chiralny & tak \\
$12a_{1100}$ & $2$ & $4$ & $3$ & $261$ & $0$ & $1+3+5+5+2$ & odwracalny & tak \\
$12a_{1101}$ & $2$ & $4$ & $3$ & $237$ & $4$ & $1+5+4+5+2$ & chiralny & tak \\
$12a_{1102}$ & $2$ & $5$ & $3$ & $305$ & $0$ & $1-4+2+0+1$ & -zwierciadlany & tak \\
$12a_{1103}$ & $2..3$ & $5$ & $3$ & $195$ & $2$ & $1+5+5+4$ & chiralny & tak \\
$12a_{1104}$ & $2$ & $5$ & $3$ & $215$ & $2$ & $1+2+3+0-1$ & chiralny & tak \\
$12a_{1105}$ & $2$ & $5$ & $3$ & $289$ & $0$ & $1+0+2+0+1$ & całkowicie & tak \\
$12a_{1106}$ & $2$ & $6$ & $3$ & $205$ & $0$ & $1-3+0-3$ & odwracalny & tak \\
$12a_{1107}$ & $2..3$ & $4$ & $3$ & $113$ & $4$ & $1+4+8+8+2$ & chiralny & tak \\
$12a_{1108}$ & $2$ & $5$ & $3$ & $155$ & $2$ & $1+3+7+4$ & chiralny & tak \\
$12a_{1109}$ & $2..3$ & $4$ & $3$ & $247$ & $-2$ & $1+2-3-5-2$ & odwracalny & tak \\
$12a_{1110}$ & $2$ & $5$ & $3$ & $283$ & $-2$ & $1+3+3+5$ & chiralny & tak \\
$12a_{1111}$ & $2$ & $4$ & $3$ & $207$ & $2$ & $1+0-1-5-2$ & chiralny & tak \\
$12a_{1112}$ & $3..4$ & $5$ & $3$ & $215$ & $-6$ & $1+10+21+8$ & odwracalny & tak \\
$12a_{1113}$ & $2..4$ & $6$ & $3$ & $205$ & $-4$ & $1+9+15$ & odwracalny & tak \\
$12a_{1114}$ & $3..4$ & $3$ & $3$ & $95$ & $6$ & $1+8+12+12+6+1$ & odwracalny & tak \\
$12a_{1115}$ & $2..4$ & $4$ & $3$ & $165$ & $4$ & $1+7+8+7+2$ & odwracalny & tak \\
$12a_{1116}$ & $2..4$ & $4$ & $3$ & $185$ & $4$ & $1+6+9+7+2$ & odwracalny & tak \\
$12a_{1117}$ & $2$ & $6$ & $3$ & $273$ & $-4$ & $1+4+2-4$ & odwracalny & tak \\
$12a_{1118}$ & $2..3$ & $5$ & $3$ & $115$ & $2$ & $1+5+6+3$ & odwracalny & tak \\
$12a_{1119}$ & $2$ & $4$ & $3$ & $169$ & $0$ & $1+2+7+7+2$ & chiralny & tak \\
$12a_{1120}$ & $2$ & $3$ & $3$ & $111$ & $2$ & $1+4+10+12+6+1$ & chiralny & tak \\
$12a_{1121}$ & $2$ & $4$ & $3$ & $181$ & $0$ & $1+3+8+7+2$ & chiralny & tak \\
$12a_{1122}$ & $2$ & $5$ & $3$ & $285$ & $0$ & $1+1+2+0+1$ & chiralny & tak \\
$12a_{1123}$ & $2$ & $5$ & $3$ & $261$ & $0$ & $1-1-4-1+1$ & -zwierciadlany & tak \\
$12a_{1124}$ & $2..3$ & $7$ & $3$ & $205$ & $0$ & $1-7+7-1$ & -zwierciadlany & tak \\
$12a_{1125}$ & $2$ & $5$ & ${}^{101}{\mskip -5mu/\mskip -3mu}_{23}$ & $101$ & $-4$ & $1-1-10-4$ & odwracalny & tak \\
$12a_{1126}$ & $1$ & $6$ & ${}^{119}{\mskip -5mu/\mskip -3mu}_{26}$ & $119$ & $-2$ & $1-2-8$ & odwracalny & tak \\
$12a_{1127}$ & $2..3$ & $7$ & ${}^{97}{\mskip -5mu/\mskip -3mu}_{22}$ & $97$ & $0$ & $1-8+4$ & całkowicie & tak \\
$12a_{1128}$ & $1$ & $3$ & ${}^{59}{\mskip -5mu/\mskip -3mu}_{9}$ & $59$ & $2$ & $1+3+13+16+7+1$ & odwracalny & tak \\
$12a_{1129}$ & $1$ & $4$ & ${}^{105}{\mskip -5mu/\mskip -3mu}_{23}$ & $105$ & $0$ & $1+2+11+9+2$ & odwracalny & tak \\
$12a_{1130}$ & $1$ & $4$ & ${}^{125}{\mskip -5mu/\mskip -3mu}_{27}$ & $125$ & $0$ & $1+1+8+8+2$ & odwracalny & tak \\
$12a_{1131}$ & $2$ & $4$ & ${}^{73}{\mskip -5mu/\mskip -3mu}_{11}$ & $73$ & $4$ & $1+2+9+9+2$ & odwracalny & tak \\
$12a_{1132}$ & $2$ & $5$ & ${}^{131}{\mskip -5mu/\mskip -3mu}_{40}$ & $131$ & $2$ & $1+1+8+4$ & odwracalny & tak \\
$12a_{1133}$ & $2$ & $5$ & ${}^{159}{\mskip -5mu/\mskip -3mu}_{47}$ & $159$ & $2$ & $1+0+6+4$ & odwracalny & tak \\
$12a_{1134}$ & $2$ & $3$ & ${}^{53}{\mskip -5mu/\mskip -3mu}_{7}$ & $53$ & $-4$ & $1-1-13-16-7-1$ & odwracalny & tak \\
$12a_{1135}$ & $1$ & $4$ & ${}^{103}{\mskip -5mu/\mskip -3mu}_{32}$ & $103$ & $-2$ & $1-2-11-9-2$ & odwracalny & tak \\
$12a_{1136}$ & $2$ & $4$ & ${}^{147}{\mskip -5mu/\mskip -3mu}_{43}$ & $147$ & $-2$ & $1-3-10-8-2$ & odwracalny & tak \\
$12a_{1137}$ & $2$ & $5$ & $3$ & $141$ & $0$ & $1-3-8-4$ & odwracalny & tak \\
$12a_{1138}$ & $2$ & $5$ & ${}^{79}{\mskip -5mu/\mskip -3mu}_{14}$ & $79$ & $-2$ & $1+0+7+3$ & odwracalny & tak \\
$12a_{1139}$ & $1$ & $6$ & ${}^{101}{\mskip -5mu/\mskip -3mu}_{18}$ & $101$ & $0$ & $1-1+6$ & odwracalny & tak \\
$12a_{1140}$ & $1$ & $5$ & ${}^{97}{\mskip -5mu/\mskip -3mu}_{18}$ & $97$ & $0$ & $1-4-7-3$ & odwracalny & tak \\
$12a_{1141}$ & $2$ & $5$ & $3$ & $167$ & $-2$ & $1+2+6+0-1$ & chiralny & tak \\
$12a_{1142}$ & $2$ & $5$ & $3$ & $99$ & $-2$ & $1+1+10+4$ & odwracalny & tak \\
$12a_{1143}$ & $2$ & $6$ & $3$ & $205$ & $0$ & $1+1+5-2$ & chiralny & tak \\
$12a_{1144}$ & $2$ & $6$ & $3$ & $129$ & $0$ & $1+0+8$ & odwracalny & tak \\
$12a_{1145}$ & $2$ & $4$ & ${}^{79}{\mskip -5mu/\mskip -3mu}_{15}$ & $79$ & $-2$ & $1-4-14-10-2$ & odwracalny & tak \\
$12a_{1146}$ & $2$ & $5$ & ${}^{117}{\mskip -5mu/\mskip -3mu}_{34}$ & $117$ & $0$ & $1-5-10-4$ & odwracalny & tak \\
$12a_{1147}$ & $2..3$ & $4$ & $3$ & $123$ & $-2$ & $1-5-13-9-2$ & odwracalny & tak \\
$12a_{1148}$ & $2..3$ & $5$ & ${}^{73}{\mskip -5mu/\mskip -3mu}_{23}$ & $73$ & $0$ & $1-6-9-3$ & odwracalny & tak \\
$12a_{1149}$ & $2$ & $6$ & ${}^{35}{\mskip -5mu/\mskip -3mu}_{4}$ & $35$ & $2$ & $1-7-4$ & odwracalny & tak \\
$12a_{1150}$ & $2$ & $5$ & $3$ & $197$ & $0$ & $1-5-5+0+1$ & chiralny & tak \\
$12a_{1151}$ & $2..3$ & $6$ & $3$ & $135$ & $2$ & $1-6-2+2$ & chiralny & tak \\
$12a_{1152}$ & $2$ & $5$ & $3$ & $325$ & $0$ & $1-1+0-1+1$ & -zwierciadlany & tak \\
$12a_{1153}$ & $2$ & $5$ & $3$ & $137$ & $-4$ & $1-2-8+0+1$ & chiralny & tak \\
$12a_{1154}$ & $2$ & $6$ & $3$ & $195$ & $-2$ & $1-3-5+2$ & chiralny & tak \\
$12a_{1155}$ & $2$ & $5$ & $3$ & $273$ & $-4$ & $1+4-2-5$ & chiralny & tak \\
$12a_{1156}$ & $2$ & $5$ & $3$ & $205$ & $-4$ & $1+1-3+0+1$ & odwracalny & tak \\
$12a_{1157}$ & $3$ & $4$ & ${}^{39}{\mskip -5mu/\mskip -3mu}_{5}$ & $39$ & $-6$ & $1-2-15-11-2$ & odwracalny & tak \\
$12a_{1158}$ & $2$ & $5$ & ${}^{77}{\mskip -5mu/\mskip -3mu}_{16}$ & $77$ & $-4$ & $1-3-12-4$ & odwracalny & tak \\
$12a_{1159}$ & $2..3$ & $5$ & ${}^{113}{\mskip -5mu/\mskip -3mu}_{24}$ & $113$ & $-4$ & $1-4-10-4$ & odwracalny & tak \\
$12a_{1160}$ & $2..3$ & $6$ & $3$ & $111$ & $-2$ & $1-4-8$ & odwracalny & tak \\
$12a_{1161}$ & $2$ & $6$ & ${}^{75}{\mskip -5mu/\mskip -3mu}_{14}$ & $75$ & $-2$ & $1-5-6$ & odwracalny & tak \\
$12a_{1162}$ & $2..3$ & $5$ & ${}^{69}{\mskip -5mu/\mskip -3mu}_{13}$ & $69$ & $-4$ & $1-5-13-4$ & odwracalny & tak \\
$12a_{1163}$ & $2..3$ & $6$ & ${}^{103}{\mskip -5mu/\mskip -3mu}_{24}$ & $103$ & $-2$ & $1-6-8$ & odwracalny & tak \\
$12a_{1164}$ & $2..4$ & $5$ & $3$ & $105$ & $-4$ & $1-6-11-4$ & odwracalny & tak \\
$12a_{1165}$ & $2..3$ & $6$ & ${}^{67}{\mskip -5mu/\mskip -3mu}_{16}$ & $67$ & $-2$ & $1-7-6$ & odwracalny & tak \\
$12a_{1166}$ & $2$ & $7$ & ${}^{33}{\mskip -5mu/\mskip -3mu}_{4}$ & $33$ & $0$ & $1-8$ & odwracalny & tak \\
$12a_{1167}$ & $2$ & $5$ & $3$ & $313$ & $0$ & $1-2-5-2+1$ & -zwierciadlany & tak \\
$12a_{1168}$ & $2$ & $3$ & $3$ & $155$ & $2$ & $1+3+11+13+6+1$ & chiralny & tak \\
$12a_{1169}$ & $1$ & $4$ & $3$ & $137$ & $0$ & $1+2+9+8+2$ & chiralny & tak \\
$12a_{1170}$ & $2$ & $4$ & $3$ & $153$ & $0$ & $1+2+10+8+2$ & chiralny & tak \\
$12a_{1171}$ & $2$ & $5$ & $3$ & $115$ & $-2$ & $1+1+9+4$ & chiralny & tak \\
$12a_{1172}$ & $2$ & $4$ & $3$ & $221$ & $0$ & $1+1+6+6+2$ & chiralny & tak \\
$12a_{1173}$ & $2$ & $6$ & $3$ & $231$ & $-2$ & $1-2-3+3$ & odwracalny & tak \\
$12a_{1174}$ & $2$ & $4$ & $3$ & $145$ & $0$ & $1+4+10+8+2$ & chiralny & tak \\
$12a_{1175}$ & $2$ & $4$ & $3$ & $223$ & $-2$ & $1+0-6-6-2$ & chiralny & tak \\
$12a_{1176}$ & $1$ & $3$ & $3$ & $119$ & $2$ & $1+2+9+12+6+1$ & odwracalny & tak \\
$12a_{1177}$ & $2$ & $4$ & $3$ & $189$ & $0$ & $1+1+8+7+2$ & odwracalny & tak \\
$12a_{1178}$ & $2$ & $4$ & $3$ & $161$ & $0$ & $1+0+6+7+2$ & odwracalny & tak \\
$12a_{1179}$ & $2$ & $5$ & $3$ & $91$ & $-2$ & $1-1+6+3$ & odwracalny & tak \\
$12a_{1180}$ & $2$ & $4$ & $3$ & $165$ & $0$ & $1+3+11+8+2$ & odwracalny & tak \\
$12a_{1181}$ & $2$ & $5$ & $3$ & $135$ & $-2$ & $1+2+8+4$ & odwracalny & tak \\
$12a_{1182}$ & $1$ & $4$ & $3$ & $187$ & $-2$ & $1-1-8-7-2$ & odwracalny & tak \\
$12a_{1183}$ & $2$ & $5$ & $3$ & $121$ & $-4$ & $1-2-9-4$ & odwracalny & tak \\
$12a_{1184}$ & $2..3$ & $4$ & $3$ & $177$ & $-4$ & $1+4+4+6+2$ & chiralny & tak \\
$12a_{1185}$ & $2$ & $5$ & $3$ & $235$ & $-2$ & $1+3+2+4$ & chiralny & tak \\
$12a_{1186}$ & $2$ & $4$ & $3$ & $245$ & $-4$ & $1+3+4+5+2$ & chiralny & tak \\
$12a_{1187}$ & $1$ & $4$ & $3$ & $303$ & $2$ & $1+0-3-4-2$ & chiralny & tak \\
$12a_{1188}$ & $1$ & $5$ & $3$ & $353$ & $0$ & $1+0+2-1+1$ & -zwierciadlany & tak \\
$12a_{1189}$ & $2$ & $5$ & $3$ & $267$ & $-2$ & $1+3+0+4$ & chiralny & tak \\
$12a_{1190}$ & $2$ & $5$ & $3$ & $279$ & $-2$ & $1+2+3+1-1$ & chiralny & tak \\
$12a_{1191}$ & $2$ & $3$ & $3$ & $153$ & $4$ & $1+2-2-7-5-1$ & chiralny & tak \\
$12a_{1192}$ & $2$ & $4$ & $3$ & $227$ & $2$ & $1+1-2-5-2$ & chiralny & tak \\
$12a_{1193}$ & $2$ & $4$ & $3$ & $323$ & $2$ & $1+1+0-3-2$ & odwracalny & tak \\
$12a_{1194}$ & $3$ & $4$ & $3$ & $175$ & $2$ & $1+0-3-6-2$ & odwracalny & tak \\
$12a_{1195}$ & $2$ & $5$ & $3$ & $239$ & $-2$ & $1+0+5+1-1$ & chiralny & tak \\
$12a_{1196}$ & $2$ & $4$ & $3$ & $297$ & $0$ & $1-2-2+3+2$ & odwracalny & tak \\
$12a_{1197}$ & $2$ & $4$ & $3$ & $237$ & $0$ & $1+1+3+5+2$ & chiralny & tak \\
$12a_{1198}$ & $1$ & $5$ & $3$ & $223$ & $-2$ & $1+0+2+4$ & chiralny & tak \\
$12a_{1199}$ & $1$ & $3$ & $3$ & $185$ & $0$ & $1-2-5-8-5-1$ & odwracalny & tak \\
$12a_{1200}$ & $2$ & $4$ & $3$ & $195$ & $2$ & $1-3-5-6-2$ & odwracalny & tak \\
$12a_{1201}$ & $2$ & $4$ & $3$ & $213$ & $0$ & $1+3+6+6+2$ & chiralny & tak \\
$12a_{1202}$ & $2..3$ & $7$ & $3$ & $169$ & $0$ & $1-6+9$ & całkowicie & tak \\
$12a_{1203}$ & $2$ & $3$ & $3$ & $143$ & $2$ & $1+4+12+13+6+1$ & odwracalny & tak \\
$12a_{1204}$ & $2$ & $4$ & $3$ & $149$ & $0$ & $1+3+10+8+2$ & chiralny & tak \\
$12a_{1205}$ & $2..3$ & $5$ & $3$ & $119$ & $-2$ & $1+2+9+4$ & odwracalny & tak \\
$12a_{1206}$ & $2$ & $5$ & $3$ & $245$ & $-4$ & $1+3+0-4$ & odwracalny & tak \\
$12a_{1207}$ & $2$ & $4$ & $3$ & $213$ & $-4$ & $1+3+2+5+2$ & chiralny & tak \\
$12a_{1208}$ & $2$ & $5$ & $3$ & $247$ & $-2$ & $1+2+1+4$ & chiralny & tak \\
$12a_{1209}$ & $2$ & $3$ & $3$ & $181$ & $0$ & $1+3+0-7-5-1$ & -zwierciadlany & tak \\
$12a_{1210}$ & $2..3$ & $3$ & $3$ & $165$ & $4$ & $1+3-1-7-5-1$ & odwracalny & tak \\
$12a_{1211}$ & $2$ & $3$ & $3$ & $265$ & $0$ & $1+2+1-4-4-1$ & -zwierciadlany & tak \\
$12a_{1212}$ & $2$ & $3$ & $3$ & $231$ & $-2$ & $1-2-3+3+4+1$ & odwracalny & tak \\
$12a_{1213}$ & $2$ & $4$ & $3$ & $253$ & $0$ & $1-3-1+4+2$ & odwracalny & tak \\
$12a_{1214}$ & $4$ & $3$ & $3$ & $45$ & $-8$ & $1+1-9-15-7-1$ & odwracalny & tak \\
$12a_{1215}$ & $2$ & $3$ & $3$ & $117$ & $-4$ & $1-1-9-12-6-1$ & chiralny & tak \\
$12a_{1216}$ & $2$ & $4$ & $3$ & $183$ & $2$ & $1+2+1-5-2$ & odwracalny & tak \\
$12a_{1217}$ & $1$ & $5$ & $3$ & $237$ & $0$ & $1+1-1-4$ & odwracalny & tak \\
$12a_{1218}$ & $2$ & $3$ & $3$ & $109$ & $0$ & $1+1-5-11-6-1$ & -zwierciadlany & tak \\
$12a_{1219}$ & $1$ & $3$ & $3$ & $123$ & $2$ & $1+3+9+12+6+1$ & chiralny & tak \\
$12a_{1220}$ & $3$ & $3$ & $3$ & $91$ & $6$ & $1+3+7+11+6+1$ & odwracalny & tak \\
$12a_{1221}$ & $2$ & $3$ & $3$ & $195$ & $-2$ & $1+1+4+8+5+1$ & chiralny & tak \\
$12a_{1222}$ & $2$ & $3$ & $3$ & $205$ & $0$ & $1+1-3-8-5-1$ & odwracalny & tak \\
$12a_{1223}$ & $2..3$ & $3$ & $3$ & $137$ & $-4$ & $1-2-12-13-6-1$ & chiralny & tak \\
$12a_{1224}$ & $2$ & $4$ & $3$ & $147$ & $-2$ & $1-3-10-8-2$ & chiralny & tak \\
$12a_{1225}$ & $2$ & $3$ & $3$ & $225$ & $0$ & $1+0-6-9-5-1$ & -zwierciadlany & tak \\
$12a_{1226}$ & $2$ & $3$ & $3$ & $147$ & $2$ & $1+5+12+13+6+1$ & chiralny & tak \\
$12a_{1227}$ & $2$ & $3$ & $3$ & $191$ & $2$ & $1+0+4+8+5+1$ & chiralny & tak \\
$12a_{1228}$ & $1$ & $4$ & $3$ & $197$ & $0$ & $1-1+4+6+2$ & chiralny & tak \\
$12a_{1229}$ & $2$ & $3$ & $3$ & $221$ & $0$ & $1-3-7-9-5-1$ & -zwierciadlany & tak \\
$12a_{1230}$ & $2$ & $3$ & $3$ & $215$ & $-2$ & $1+2+7+9+5+1$ & chiralny & tak \\
$12a_{1231}$ & $1$ & $3$ & $3$ & $193$ & $0$ & $1+0-4-8-5-1$ & chiralny & tak \\
$12a_{1232}$ & $2$ & $4$ & $3$ & $187$ & $-2$ & $1-1-4-6-2$ & chiralny & tak \\
$12a_{1233}$ & $2$ & $3$ & $3$ & $77$ & $-4$ & $1-3-16-17-7-1$ & odwracalny & tak \\
$12a_{1234}$ & $2$ & $4$ & $3$ & $127$ & $-2$ & $1-4-13-9-2$ & odwracalny & tak \\
$12a_{1235}$ & $1$ & $3$ & $3$ & $133$ & $0$ & $1-1-8-12-6-1$ & chiralny & tak \\
$12a_{1236}$ & $2$ & $4$ & $3$ & $167$ & $2$ & $1-2-7-7-2$ & chiralny & tak \\
$12a_{1237}$ & $2$ & $4$ & $3$ & $225$ & $0$ & $1+0+2+5+2$ & chiralny & tak \\
$12a_{1238}$ & $2$ & $3$ & $3$ & $131$ & $-2$ & $1+1+8+12+6+1$ & chiralny & tak \\
$12a_{1239}$ & $1$ & $4$ & $3$ & $177$ & $0$ & $1+0+7+7+2$ & chiralny & tak \\
$12a_{1240}$ & $2..3$ & $4$ & $3$ & $97$ & $-4$ & $1+0+6+8+2$ & chiralny & tak \\
$12a_{1241}$ & $2$ & $5$ & $3$ & $155$ & $-2$ & $1-1+6+4$ & chiralny & tak \\
$12a_{1242}$ & $3$ & $4$ & $3$ & $47$ & $-6$ & $1-4-16-11-2$ & odwracalny & tak \\
$12a_{1243}$ & $2..3$ & $5$ & $3$ & $85$ & $-4$ & $1-5-12-4$ & odwracalny & tak \\
$12a_{1244}$ & $2$ & $4$ & $3$ & $119$ & $-2$ & $1-2-8-8-2$ & chiralny & tak \\
$12a_{1245}$ & $1$ & $5$ & $3$ & $173$ & $0$ & $1-3-6-4$ & chiralny & tak \\
$12a_{1246}$ & $2$ & $3$ & $3$ & $143$ & $-2$ & $1+4+12+13+6+1$ & odwracalny & tak \\
$12a_{1247}$ & $2..3$ & $4$ & $3$ & $117$ & $-4$ & $1+3+8+8+2$ & odwracalny & tak \\
$12a_{1248}$ & $2$ & $3$ & $3$ & $255$ & $2$ & $1+0+0+4+4+1$ & odwracalny & tak \\
$12a_{1249}$ & $1$ & $3$ & $3$ & $257$ & $0$ & $1+0+0-4-4-1$ & -zwierciadlany & tak \\
$12a_{1250}$ & $1$ & $3$ & $3$ & $217$ & $0$ & $1-2-7-9-5-1$ & odwracalny & tak \\
$12a_{1251}$ & $2$ & $5$ & $3$ & $265$ & $0$ & $1+2+1-4$ & -zwierciadlany & tak \\
$12a_{1252}$ & $2$ & $4$ & $3$ & $231$ & $-2$ & $1-2-7-6-2$ & chiralny & tak \\
$12a_{1253}$ & $2$ & $3$ & $3$ & $207$ & $2$ & $1+4+8+9+5+1$ & chiralny & tak \\
$12a_{1254}$ & $2..3$ & $3$ & $3$ & $145$ & $0$ & $1-4-12-13-6-1$ & -zwierciadlany & tak \\
$12a_{1255}$ & $2..3$ & $3$ & $3$ & $177$ & $-4$ & $1+0-5-8-5-1$ & chiralny & tak \\
$12a_{1256}$ & $2..3$ & $4$ & $3$ & $139$ & $2$ & $1-5-10-8-2$ & chiralny & tak \\
$12a_{1257}$ & $2$ & $4$ & $3$ & $203$ & $-2$ & $1-1-5-6-2$ & chiralny & tak \\
$12a_{1258}$ & $1$ & $3$ & $3$ & $157$ & $0$ & $1-3-11-13-6-1$ & chiralny & tak \\
$12a_{1259}$ & $2$ & $4$ & $3$ & $143$ & $2$ & $1-4-10-8-2$ & chiralny & tak \\
$12a_{1260}$ & $2$ & $3$ & $3$ & $153$ & $0$ & $1-2-11-13-6-1$ & -zwierciadlany & tak \\
$12a_{1261}$ & $2$ & $4$ & $3$ & $195$ & $-2$ & $1-3-9-7-2$ & chiralny & tak \\
$12a_{1262}$ & $2$ & $4$ & $3$ & $147$ & $2$ & $1-3-10-8-2$ & chiralny & tak \\
$12a_{1263}$ & $2$ & $5$ & $3$ & $211$ & $2$ & $1+1+3+4$ & chiralny & tak \\
$12a_{1264}$ & $2$ & $4$ & $3$ & $127$ & $-2$ & $1-4-9-8-2$ & chiralny & tak \\
$12a_{1265}$ & $2$ & $5$ & $3$ & $165$ & $0$ & $1-5-7-4$ & chiralny & tak \\
$12a_{1266}$ & $2..3$ & $5$ & $3$ & $129$ & $-4$ & $1-4-9-4$ & chiralny & tak \\
$12a_{1267}$ & $2$ & $5$ & $3$ & $145$ & $0$ & $1-4-8-4$ & -zwierciadlany & tak \\
$12a_{1268}$ & $1$ & $4$ & $3$ & $221$ & $0$ & $1+1+6+6+2$ & chiralny & tak \\
$12a_{1269}$ & $2..3$ & $5$ & $3$ & $169$ & $0$ & $1-6-7-4$ & -zwierciadlany & tak \\
$12a_{1270}$ & $2$ & $4$ & $3$ & $255$ & $-2$ & $1+0-4-5-2$ & odwracalny & tak \\
$12a_{1271}$ & $1$ & $5$ & $3$ & $191$ & $2$ & $1+0+4+4$ & chiralny & tak \\
$12a_{1272}$ & $2$ & $4$ & $3$ & $201$ & $0$ & $1+2+5+6+2$ & chiralny & tak \\
$12a_{1273}$ & $1$ & $3$ & ${}^{61}{\mskip -5mu/\mskip -3mu}_{11}$ & $61$ & $0$ & $1-3-13-16-7-1$ & całkowicie & tak \\
$12a_{1274}$ & $2$ & $4$ & ${}^{95}{\mskip -5mu/\mskip -3mu}_{17}$ & $95$ & $2$ & $1-4-11-9-2$ & odwracalny & tak \\
$12a_{1275}$ & $2$ & $5$ & ${}^{149}{\mskip -5mu/\mskip -3mu}_{44}$ & $149$ & $0$ & $1-5-8-4$ & całkowicie & tak \\
$12a_{1276}$ & $2$ & $4$ & ${}^{75}{\mskip -5mu/\mskip -3mu}_{13}$ & $75$ & $2$ & $1-5-14-10-2$ & odwracalny & tak \\
$12a_{1277}$ & $2$ & $5$ & ${}^{121}{\mskip -5mu/\mskip -3mu}_{37}$ & $121$ & $0$ & $1-6-10-4$ & odwracalny & tak \\
$12a_{1278}$ & $2..3$ & $5$ & ${}^{41}{\mskip -5mu/\mskip -3mu}_{6}$ & $41$ & $4$ & $1-6-11-3$ & odwracalny & tak \\
$12a_{1279}$ & $2..3$ & $6$ & ${}^{67}{\mskip -5mu/\mskip -3mu}_{10}$ & $67$ & $2$ & $1-7-6$ & odwracalny & tak \\
$12a_{1280}$ & $2$ & $5$ & $3$ & $241$ & $0$ & $1-4-6-1+1$ & -zwierciadlany & tak \\
$12a_{1281}$ & $2..3$ & $5$ & ${}^{109}{\mskip -5mu/\mskip -3mu}_{33}$ & $109$ & $0$ & $1-7-11-4$ & całkowicie & tak \\
$12a_{1282}$ & $2..3$ & $6$ & ${}^{63}{\mskip -5mu/\mskip -3mu}_{10}$ & $63$ & $2$ & $1-8-6$ & odwracalny & tak \\
$12a_{1283}$ & $2$ & $3$ & $3$ & $81$ & $0$ & $1-4-16-17-7-1$ & odwracalny & tak \\
$12a_{1284}$ & $2$ & $4$ & $3$ & $123$ & $2$ & $1-5-13-9-2$ & odwracalny & tak \\
$12a_{1285}$ & $2..3$ & $4$ & $3$ & $87$ & $2$ & $1-6-15-10-2$ & odwracalny & tak \\
$12a_{1286}$ & $2..3$ & $5$ & $3$ & $45$ & $4$ & $1-7-11-3$ & odwracalny & tak \\
$12a_{1287}$ & $2..3$ & $7$ & ${}^{37}{\mskip -5mu/\mskip -3mu}_{6}$ & $37$ & $0$ & $1-9$ & całkowicie & tak \\
$12a_{1288}$ & $2..3$ & $3$ & $3$ & $117$ & $0$ & $1-5-18-18-7-1$ & całkowicie & tak \\
$12n_{1}$ & $1$ & $5$ & $3$ & $57$ & $0$ & $1-2-1-1$ & chiralny & nie \\
$12n_{2}$ & $1..2$ & $5$ & $3$ & $83$ & $-2$ & $1+1-1-3-1$ & chiralny & nie \\
$12n_{3}$ & $1..2$ & $5$ & $3$ & $73$ & $0$ & $1+2+1-1$ & chiralny & nie \\
$12n_{4}$ & $1$ & $5$ & $3$ & $81$ & $0$ & $1+0+1-1$ & chiralny & nie \\
$12n_{5}$ & $1$ & $5$ & $3$ & $87$ & $2$ & $1+2-1+1$ & chiralny & nie \\
$12n_{6}$ & $2$ & $5$ & $3$ & $45$ & $-4$ & $1+1-1+3+1$ & chiralny & nie \\
$12n_{7}$ & $1$ & $5$ & $3$ & $39$ & $2$ & $1-2+1+1$ & chiralny & nie \\
$12n_{8}$ & $2..3$ & $5$ & $3$ & $61$ & $-4$ & $1-3-1+3+1$ & chiralny & nie \\
$12n_{9}$ & $1..2$ & $5$ & $3$ & $71$ & $2$ & $1-2-1+1$ & chiralny & nie \\
$12n_{10}$ & $2$ & $5$ & $3$ & $39$ & $2$ & $1-2+1+1$ & chiralny & nie \\
$12n_{11}$ & $1..2$ & $6$ & $3$ & $61$ & $0$ & $1-3+3$ & odwracalny & nie \\
$12n_{12}$ & $1..2$ & $6$ & $3$ & $79$ & $-2$ & $1-4+2+2$ & odwracalny & nie \\
$12n_{13}$ & $1$ & $6$ & $3$ & $29$ & $0$ & $1-3+1$ & odwracalny & nie \\
$12n_{14}$ & $1..2$ & $5$ & $3$ & $89$ & $0$ & $1-2+1-1$ & chiralny & nie \\
$12n_{15}$ & $2$ & $5$ & $3$ & $111$ & $2$ & $1+0-3+1$ & chiralny & nie \\
$12n_{16}$ & $2$ & $5$ & $3$ & $21$ & $-4$ & $1-1-3+3+1$ & odwracalny & nie \\
$12n_{17}$ & $2$ & $5$ & $3$ & $91$ & $2$ & $1-1-2+1$ & chiralny & nie \\
$12n_{18}$ & $1..2$ & $6$ & $3$ & $109$ & $0$ & $1-3+2-1$ & chiralny & nie \\
$12n_{19}$ & $1..2$ & $6$ & $3$ & $1$ & $0$ & $1-4+3+1$ & chiralny & nie \\
$12n_{20}$ & $1$ & $4$ & $3$ & $47$ & $2$ & $1+0+1+1$ & chiralny & nie \\
$12n_{21}$ & $1..2$ & $5$ & $3$ & $119$ & $-2$ & $1-2+0+2$ & chiralny & nie \\
$12n_{22}$ & $1..2$ & $5$ & $3$ & $141$ & $0$ & $1-3+0+2+1$ & chiralny & nie \\
$12n_{23}$ & $1$ & $5$ & $3$ & $9$ & $0$ & $1-2$ & chiralny & nie \\
$12n_{24}$ & $1$ & $4$ & $3$ & $49$ & $0$ & $1+0-1-1$ & chiralny & nie \\
$12n_{25}$ & $1$ & $5$ & $3$ & $11$ & $2$ & $1-1-1$ & odwracalny & nie \\
$12n_{26}$ & $1..2$ & $5$ & $3$ & $97$ & $0$ & $1+0-2-2$ & chiralny & nie \\
$12n_{27}$ & $2$ & $5$ & $3$ & $123$ & $-2$ & $1-1+0-2-1$ & chiralny & nie \\
$12n_{28}$ & $1$ & $5$ & $3$ & $33$ & $0$ & $1+0+2$ & chiralny & nie \\
$12n_{29}$ & $1..2$ & $5$ & $3$ & $119$ & $-2$ & $1-2+0+2$ & chiralny & nie \\
$12n_{30}$ & $1..2$ & $5$ & $3$ & $141$ & $0$ & $1-3+0+2+1$ & chiralny & nie \\
$12n_{31}$ & $1$ & $5$ & $3$ & $9$ & $0$ & $1-2$ & chiralny & nie \\
$12n_{32}$ & $1..2$ & $5$ & $3$ & $97$ & $0$ & $1+0-2-2$ & chiralny & nie \\
$12n_{33}$ & $2$ & $5$ & $3$ & $123$ & $-2$ & $1-1+0-2-1$ & chiralny & nie \\
$12n_{34}$ & $1$ & $5$ & $3$ & $33$ & $0$ & $1+0+2$ & chiralny & nie \\
$12n_{35}$ & $1$ & $5$ & $3$ & $69$ & $0$ & $1-1+0-1$ & odwracalny & nie \\
$12n_{36}$ & $1$ & $5$ & $3$ & $91$ & $2$ & $1-1-2+1$ & odwracalny & nie \\
$12n_{37}$ & $2$ & $5$ & $3$ & $41$ & $-4$ & $1-2-2+3+1$ & odwracalny & nie \\
$12n_{38}$ & $1$ & $5$ & $3$ & $67$ & $2$ & $1+1+0+1$ & odwracalny & nie \\
$12n_{39}$ & $1$ & $5$ & $3$ & $93$ & $0$ & $1+1+2-1$ & odwracalny & nie \\
$12n_{40}$ & $2$ & $5$ & $3$ & $63$ & $-2$ & $1+0+0-3-1$ & odwracalny & nie \\
$12n_{41}$ & $1..2$ & $5$ & $3$ & $19$ & $2$ & $1-3+2+1$ & odwracalny & nie \\
$12n_{42}$ & $1..2$ & $5$ & $3$ & $59$ & $-2$ & $1-1+0+1$ & odwracalny & nie \\
$12n_{43}$ & $1..2$ & $5$ & $3$ & $81$ & $0$ & $1-4+0+3+1$ & odwracalny & nie \\
$12n_{44}$ & $1..2$ & $5$ & $3$ & $51$ & $-2$ & $1-3+0+1$ & odwracalny & nie \\
$12n_{45}$ & $1..2$ & $5$ & $3$ & $37$ & $0$ & $1-1-2-1$ & odwracalny & nie \\
$12n_{46}$ & $1$ & $6$ & $3$ & $41$ & $0$ & $1-2+2$ & odwracalny & nie \\
$12n_{47}$ & $1..2$ & $6$ & $3$ & $59$ & $-2$ & $1-5+3+2$ & odwracalny & nie \\
$12n_{48}$ & $1..2$ & $6$ & $3$ & $49$ & $0$ & $1-4+2$ & odwracalny & nie \\
$12n_{49}$ & $1..2$ & $6$ & $3$ & $81$ & $0$ & $1-4+4$ & chiralny & nie \\
$12n_{50}$ & $1..2$ & $6$ & $3$ & $99$ & $-2$ & $1-3+1+2$ & chiralny & nie \\
$12n_{51}$ & $1..2$ & $6$ & $3$ & $9$ & $0$ & $1-2$ & chiralny & nie \\
$12n_{52}$ & $2$ & $5$ & $3$ & $123$ & $2$ & $1+3-3+1$ & odwracalny & nie \\
$12n_{53}$ & $1..2$ & $5$ & $3$ & $57$ & $0$ & $1-2+3$ & odwracalny & nie \\
$12n_{54}$ & $2$ & $4$ & $3$ & $29$ & $4$ & $1+1-2-1$ & odwracalny & nie \\
$12n_{55}$ & $2$ & $5$ & $4$ & $111$ & $-2$ & $1+0+1+2$ & odwracalny & nie \\
$12n_{56}$ & $2$ & $5$ & $4$ & $9$ & $0$ & $1+2+1$ & odwracalny & nie \\
$12n_{57}$ & $2$ & $5$ & $4$ & $9$ & $0$ & $1+2+1$ & odwracalny & nie \\
$12n_{58}$ & $2$ & $5$ & $4$ & $129$ & $4$ & $1+4+1-2$ & odwracalny & nie \\
$12n_{59}$ & $3$ & $5$ & $4$ & $21$ & $-4$ & $1+3-2-5-1$ & odwracalny & nie \\
$12n_{60}$ & $2$ & $5$ & $4$ & $99$ & $-2$ & $1+1-2-3-1$ & odwracalny & nie \\
$12n_{61}$ & $2$ & $5$ & $4$ & $99$ & $-2$ & $1+1-2-3-1$ & odwracalny & nie \\
$12n_{62}$ & $2$ & $5$ & $4$ & $81$ & $0$ & $1+0-3-2$ & odwracalny & nie \\
$12n_{63}$ & $2$ & $5$ & $4$ & $39$ & $2$ & $1-2-3$ & odwracalny & nie \\
$12n_{64}$ & $2$ & $5$ & $4$ & $69$ & $-4$ & $1-1+0+3+1$ & odwracalny & nie \\
$12n_{65}$ & $1$ & $5$ & $3$ & $31$ & $2$ & $1+0-2$ & odwracalny & nie \\
$12n_{66}$ & $2$ & $5$ & $4$ & $81$ & $0$ & $1+0-3-2$ & odwracalny & nie \\
$12n_{67}$ & $3$ & $5$ & $4$ & $51$ & $6$ & $1+5+2-3-1$ & odwracalny & nie \\
$12n_{68}$ & $3$ & $5$ & $3$ & $35$ & $-6$ & $1+5+3-3-1$ & chiralny & nie \\
$12n_{69}$ & $1$ & $5$ & $3$ & $73$ & $0$ & $1+2+1-1$ & chiralny & nie \\
$12n_{70}$ & $1$ & $5$ & $3$ & $79$ & $-2$ & $1+0-1+1$ & chiralny & nie \\
$12n_{71}$ & $1..2$ & $5$ & $3$ & $59$ & $-2$ & $1+3+1-3-1$ & chiralny & nie \\
$12n_{72}$ & $2$ & $5$ & $3$ & $97$ & $-4$ & $1+4+3-1$ & chiralny & nie \\
$12n_{73}$ & $2$ & $5$ & $3$ & $55$ & $2$ & $1+2+1+1$ & chiralny & nie \\
$12n_{74}$ & $4$ & $4$ & $3$ & $49$ & $-8$ & $1+8+17+11+2$ & chiralny & nie \\
$12n_{75}$ & $1..3$ & $4$ & $3$ & $83$ & $2$ & $1+5+8+3$ & chiralny & nie \\
$12n_{76}$ & $2$ & $4$ & $3$ & $53$ & $-4$ & $1+3+0-1$ & chiralny & nie \\
$12n_{77}$ & $3$ & $5$ & $3$ & $43$ & $-6$ & $1+7+11+3$ & odwracalny & nie \\
$12n_{78}$ & $2$ & $5$ & $3$ & $65$ & $0$ & $1+4+5$ & odwracalny & nie \\
$12n_{79}$ & $1$ & $5$ & $3$ & $23$ & $-2$ & $1+2-1$ & odwracalny & nie \\
$12n_{80}$ & $1..2$ & $4$ & $3$ & $67$ & $2$ & $1+5+5+2$ & chiralny & nie \\
$12n_{81}$ & $2$ & $4$ & $3$ & $101$ & $-4$ & $1+3-1-2$ & chiralny & nie \\
$12n_{82}$ & $1..2$ & $5$ & $3$ & $35$ & $2$ & $1+5+7+2$ & chiralny & nie \\
$12n_{83}$ & $2$ & $5$ & $3$ & $105$ & $-4$ & $1+2-1-2$ & chiralny & nie \\
$12n_{84}$ & $1..3$ & $4$ & $3$ & $63$ & $-2$ & $1+4+5+2$ & chiralny & nie \\
$12n_{85}$ & $2$ & $5$ & $3$ & $65$ & $0$ & $1+0-4-2$ & chiralny & nie \\
$12n_{86}$ & $1..2$ & $5$ & $3$ & $103$ & $-2$ & $1+2+2+2$ & chiralny & nie \\
$12n_{87}$ & $1..2$ & $5$ & $3$ & $49$ & $0$ & $1+4+4$ & chiralny & nie \\
$12n_{88}$ & $3$ & $4$ & $3$ & $119$ & $-6$ & $1+6+10+4$ & odwracalny & nie \\
$12n_{89}$ & $3$ & $4$ & $3$ & $43$ & $-6$ & $1+3-2-4-1$ & chiralny & nie \\
$12n_{90}$ & $2..3$ & $4$ & $3$ & $77$ & $4$ & $1+5+6+4+1$ & chiralny & nie \\
$12n_{91}$ & $4$ & $4$ & $3$ & $59$ & $-6$ & $1+7+14+8+1$ & chiralny & nie \\
$12n_{92}$ & $1..3$ & $4$ & $3$ & $109$ & $0$ & $1+5+8+4+1$ & chiralny & nie \\
$12n_{93}$ & $3$ & $5$ & $3$ & $11$ & $-6$ & $1+3+0-4-1$ & chiralny & nie \\
$12n_{94}$ & $2$ & $5$ & $3$ & $73$ & $-4$ & $1+2-3-2$ & chiralny & nie \\
$12n_{95}$ & $2$ & $5$ & $3$ & $95$ & $2$ & $1+4+3+2$ & chiralny & nie \\
$12n_{96}$ & $3$ & $5$ & $3$ & $7$ & $-6$ & $1+6+9+2$ & chiralny & nie \\
$12n_{97}$ & $1..2$ & $5$ & $3$ & $41$ & $0$ & $1+2-1-1$ & chiralny & nie \\
$12n_{98}$ & $2..3$ & $5$ & $3$ & $93$ & $-4$ & $1+1-2+2+1$ & chiralny & nie \\
$12n_{99}$ & $2$ & $5$ & $3$ & $135$ & $-2$ & $1+2+0+2$ & chiralny & nie \\
$12n_{100}$ & $3..4$ & $5$ & $3$ & $111$ & $-6$ & $1+8+15+5$ & chiralny & nie \\
$12n_{101}$ & $1..3$ & $5$ & $3$ & $141$ & $0$ & $1+5+6-1$ & chiralny & nie \\
$12n_{102}$ & $2$ & $5$ & $3$ & $21$ & $-4$ & $1+3-2-1$ & chiralny & nie \\
$12n_{103}$ & $3$ & $4$ & $3$ & $47$ & $-6$ & $1+4-2-4-1$ & chiralny & nie \\
$12n_{104}$ & $2..3$ & $4$ & $3$ & $73$ & $4$ & $1+6+6+4+1$ & chiralny & nie \\
$12n_{105}$ & $4$ & $4$ & $3$ & $31$ & $-6$ & $1+8+16+8+1$ & chiralny & nie \\
$12n_{106}$ & $1..2$ & $4$ & $3$ & $81$ & $0$ & $1+4+6+4+1$ & chiralny & nie \\
$12n_{107}$ & $3$ & $4$ & $3$ & $39$ & $-6$ & $1+2-2-4-1$ & chiralny & nie \\
$12n_{108}$ & $2$ & $5$ & $3$ & $69$ & $-4$ & $1+3-3-2$ & chiralny & nie \\
$12n_{109}$ & $2$ & $5$ & $3$ & $99$ & $2$ & $1+5+3+2$ & chiralny & nie \\
$12n_{110}$ & $3$ & $5$ & $3$ & $21$ & $-4$ & $1+7+11+2$ & chiralny & nie \\
$12n_{111}$ & $1..2$ & $5$ & $3$ & $91$ & $-2$ & $1+3+3+2$ & chiralny & nie \\
$12n_{112}$ & $2$ & $5$ & $3$ & $77$ & $-4$ & $1+1-3-2$ & chiralny & nie \\
$12n_{113}$ & $2..3$ & $3$ & $3$ & $53$ & $4$ & $1+7+9+5+1$ & chiralny & nie \\
$12n_{114}$ & $3$ & $3$ & $3$ & $67$ & $-6$ & $1+5+1-3-1$ & chiralny & nie \\
$12n_{115}$ & $1..3$ & $4$ & $3$ & $11$ & $2$ & $1+7+13+7+1$ & chiralny & nie \\
$12n_{116}$ & $2$ & $4$ & $3$ & $71$ & $2$ & $1+6+5+2$ & chiralny & nie \\
$12n_{117}$ & $2$ & $4$ & $3$ & $97$ & $-4$ & $1+4-1-2$ & chiralny & nie \\
$12n_{118}$ & $1..3$ & $5$ & $3$ & $7$ & $2$ & $1+6+9+2$ & chiralny & nie \\
$12n_{119}$ & $2$ & $4$ & $3$ & $17$ & $-4$ & $1+0-3-1$ & chiralny & nie \\
$12n_{120}$ & $2$ & $4$ & $3$ & $19$ & $2$ & $1+5+4+1$ & chiralny & nie \\
$12n_{121}$ & $1$ & $4$ & $3$ & $1$ & $0$ & $1+4+1$ & odwracalny & nie \\
$12n_{122}$ & $1..2$ & $5$ & $3$ & $107$ & $-2$ & $1+3+2-2-1$ & chiralny & nie \\
$12n_{123}$ & $2$ & $5$ & $3$ & $145$ & $-4$ & $1+4+2-2$ & chiralny & nie \\
$12n_{124}$ & $1$ & $5$ & $3$ & $7$ & $2$ & $1+2$ & chiralny & nie \\
$12n_{125}$ & $2..3$ & $5$ & $3$ & $93$ & $-4$ & $1+1-2+2+1$ & chiralny & nie \\
$12n_{126}$ & $2$ & $5$ & $3$ & $135$ & $-2$ & $1+2+0+2$ & chiralny & nie \\
$12n_{127}$ & $1..2$ & $5$ & $3$ & $107$ & $-2$ & $1+3+2-2-1$ & chiralny & nie \\
$12n_{128}$ & $2$ & $5$ & $3$ & $145$ & $-4$ & $1+4+2-2$ & chiralny & nie \\
$12n_{129}$ & $1$ & $5$ & $3$ & $7$ & $2$ & $1+2$ & chiralny & nie \\
$12n_{130}$ & $2$ & $5$ & $3$ & $65$ & $0$ & $1+0-4-2$ & chiralny & nie \\
$12n_{131}$ & $1..2$ & $5$ & $3$ & $103$ & $-2$ & $1+2+2+2$ & chiralny & nie \\
$12n_{132}$ & $1..2$ & $5$ & $3$ & $49$ & $0$ & $1+4+4$ & chiralny & nie \\
$12n_{133}$ & $3$ & $4$ & $3$ & $119$ & $-6$ & $1+6+10+4$ & chiralny & nie \\
$12n_{134}$ & $3$ & $4$ & $3$ & $43$ & $-6$ & $1+3-2-4-1$ & chiralny & nie \\
$12n_{135}$ & $2..3$ & $4$ & $3$ & $77$ & $4$ & $1+5+6+4+1$ & chiralny & nie \\
$12n_{136}$ & $4$ & $4$ & $3$ & $59$ & $-6$ & $1+7+14+8+1$ & chiralny & nie \\
$12n_{137}$ & $1..3$ & $4$ & $3$ & $109$ & $0$ & $1+5+8+4+1$ & chiralny & nie \\
$12n_{138}$ & $3$ & $5$ & $3$ & $11$ & $-6$ & $1+3+0-4-1$ & chiralny & nie \\
$12n_{139}$ & $2..3$ & $5$ & $3$ & $69$ & $-4$ & $1+3+1+3+1$ & chiralny & nie \\
$12n_{140}$ & $1..3$ & $5$ & $3$ & $111$ & $-2$ & $1+0-3+1$ & chiralny & nie \\
$12n_{141}$ & $1..2$ & $5$ & $3$ & $57$ & $0$ & $1-2-1-1$ & chiralny & nie \\
$12n_{142}$ & $1$ & $5$ & $3$ & $43$ & $-2$ & $1-1+1+1$ & odwracalny & nie \\
$12n_{143}$ & $2$ & $5$ & $3$ & $13$ & $-4$ & $1+1-3-1$ & odwracalny & nie \\
$12n_{144}$ & $2$ & $5$ & $3$ & $75$ & $2$ & $1-1-1+1$ & odwracalny & nie \\
$12n_{145}$ & $2$ & $5$ & $3$ & $25$ & $0$ & $1-2+1$ & odwracalny & nie \\
$12n_{146}$ & $1$ & $5$ & $3$ & $47$ & $2$ & $1+4-2$ & chiralny & nie \\
$12n_{147}$ & $3$ & $4$ & $3$ & $75$ & $-2$ & $1+3+4+2$ & odwracalny & nie \\
$12n_{148}$ & $3$ & $4$ & $3$ & $17$ & $4$ & $1+8+7+1$ & chiralny & nie \\
$12n_{149}$ & $2$ & $5$ & $3$ & $5$ & $4$ & $1+7+2$ & chiralny & nie \\
$12n_{150}$ & $2$ & $5$ & $3$ & $57$ & $-4$ & $1+2+0+3+1$ & odwracalny & nie \\
$12n_{151}$ & $2$ & $5$ & $3$ & $99$ & $-2$ & $1+1-2+1$ & odwracalny & nie \\
$12n_{152}$ & $1$ & $5$ & $3$ & $69$ & $0$ & $1-1+0-1$ & odwracalny & nie \\
$12n_{153}$ & $4$ & $4$ & $3$ & $13$ & $-8$ & $1+9+19+12+2$ & odwracalny & nie \\
$12n_{154}$ & $1..2$ & $4$ & $3$ & $47$ & $2$ & $1+4+6+2$ & odwracalny & nie \\
$12n_{155}$ & $2$ & $4$ & $3$ & $89$ & $-4$ & $1+2-2-2$ & odwracalny & nie \\
$12n_{156}$ & $2$ & $5$ & $3$ & $33$ & $-4$ & $1+0-2+3+1$ & odwracalny & nie \\
$12n_{157}$ & $2$ & $5$ & $3$ & $75$ & $-2$ & $1+3+0+1$ & odwracalny & nie \\
$12n_{158}$ & $1..2$ & $5$ & $3$ & $93$ & $0$ & $1+1+2-1$ & odwracalny & nie \\
$12n_{159}$ & $1..2$ & $4$ & $3$ & $55$ & $2$ & $1+6+6+2$ & odwracalny & nie \\
$12n_{160}$ & $1..2$ & $5$ & $3$ & $53$ & $0$ & $1+3+0-1$ & odwracalny & nie \\
$12n_{161}$ & $2$ & $5$ & $3$ & $91$ & $-2$ & $1-1-2+1$ & odwracalny & nie \\
$12n_{162}$ & $1..2$ & $5$ & $3$ & $61$ & $0$ & $1+1+0-1$ & odwracalny & nie \\
$12n_{163}$ & $3$ & $5$ & $3$ & $47$ & $-6$ & $1+4+2-3-1$ & odwracalny & nie \\
$12n_{164}$ & $2$ & $5$ & $3$ & $85$ & $0$ & $1+3+2-1$ & odwracalny & nie \\
$12n_{165}$ & $1$ & $5$ & $3$ & $67$ & $-2$ & $1+1+0+1$ & odwracalny & nie \\
$12n_{166}$ & $4$ & $4$ & $3$ & $41$ & $-8$ & $1+10+21+12+2$ & odwracalny & nie \\
$12n_{167}$ & $1..3$ & $4$ & $3$ & $67$ & $2$ & $1+5+9+3$ & odwracalny & nie \\
$12n_{168}$ & $2$ & $4$ & $3$ & $37$ & $-4$ & $1+3-1-1$ & odwracalny & nie \\
$12n_{169}$ & $3..4$ & $5$ & $3$ & $51$ & $-6$ & $1+9+15+4$ & odwracalny & nie \\
$12n_{170}$ & $1..2$ & $5$ & $3$ & $81$ & $0$ & $1+4+6$ & odwracalny & nie \\
$12n_{171}$ & $2$ & $5$ & $3$ & $39$ & $-2$ & $1+2-2$ & odwracalny & nie \\
$12n_{172}$ & $3$ & $5$ & $3$ & $23$ & $-6$ & $1+6+4-3-1$ & odwracalny & nie \\
$12n_{173}$ & $2$ & $5$ & $3$ & $123$ & $-2$ & $1-1+0-2-1$ & chiralny & nie \\
$12n_{174}$ & $2$ & $5$ & $3$ & $165$ & $0$ & $1-1+2+2+1$ & chiralny & nie \\
$12n_{175}$ & $2$ & $5$ & $3$ & $3$ & $-2$ & $1+1+0+4+1$ & chiralny & nie \\
$12n_{176}$ & $1..2$ & $5$ & $3$ & $63$ & $-2$ & $1+0-4$ & chiralny & nie \\
$12n_{177}$ & $3$ & $5$ & $3$ & $79$ & $-6$ & $1+8+13+4$ & chiralny & nie \\
$12n_{178}$ & $1..2$ & $5$ & $3$ & $101$ & $0$ & $1+3+3-1$ & chiralny & nie \\
$12n_{179}$ & $2$ & $5$ & $3$ & $13$ & $-4$ & $1+1-3-1$ & chiralny & nie \\
$12n_{180}$ & $1..2$ & $4$ & $3$ & $97$ & $0$ & $1+0-2-2$ & chiralny & nie \\
$12n_{181}$ & $1..2$ & $4$ & $3$ & $83$ & $-2$ & $1+1+3+2$ & chiralny & nie \\
$12n_{182}$ & $1..2$ & $5$ & $3$ & $71$ & $-2$ & $1+2+0-3-1$ & chiralny & nie \\
$12n_{183}$ & $2$ & $5$ & $3$ & $109$ & $-4$ & $1+5+4-1$ & chiralny & nie \\
$12n_{184}$ & $1..2$ & $5$ & $3$ & $43$ & $2$ & $1+3+2+1$ & chiralny & nie \\
$12n_{185}$ & $3$ & $4$ & $3$ & $55$ & $-6$ & $1+2-3-4-1$ & chiralny & nie \\
$12n_{186}$ & $2..3$ & $4$ & $3$ & $89$ & $4$ & $1+6+7+4+1$ & chiralny & nie \\
$12n_{187}$ & $4$ & $4$ & $3$ & $47$ & $-6$ & $1+8+15+8+1$ & chiralny & nie \\
$12n_{188}$ & $2$ & $4$ & $3$ & $65$ & $-4$ & $1+4+5+4+1$ & chiralny & nie \\
$12n_{189}$ & $1..2$ & $4$ & $3$ & $79$ & $-2$ & $1+0-5-4-1$ & chiralny & nie \\
$12n_{190}$ & $2..3$ & $3$ & $3$ & $65$ & $4$ & $1+8+10+5+1$ & chiralny & nie \\
$12n_{191}$ & $3$ & $3$ & $3$ & $79$ & $-6$ & $1+4+0-3-1$ & chiralny & nie \\
$12n_{192}$ & $1..3$ & $4$ & $3$ & $23$ & $2$ & $1+6+12+7+1$ & chiralny & nie \\
$12n_{193}$ & $2..3$ & $4$ & $3$ & $59$ & $2$ & $1+7+6+2$ & chiralny & nie \\
$12n_{194}$ & $2$ & $4$ & $3$ & $85$ & $-4$ & $1+3-2-2$ & chiralny & nie \\
$12n_{195}$ & $1..2$ & $5$ & $3$ & $19$ & $2$ & $1+5+8+2$ & chiralny & nie \\
$12n_{196}$ & $2..3$ & $5$ & $3$ & $37$ & $-4$ & $1-1-6-2$ & chiralny & nie \\
$12n_{197}$ & $1..2$ & $5$ & $3$ & $107$ & $-2$ & $1+3+2+2$ & chiralny & nie \\
$12n_{198}$ & $1$ & $5$ & $3$ & $27$ & $-2$ & $1-1+2+1$ & odwracalny & nie \\
$12n_{199}$ & $2$ & $4$ & $3$ & $11$ & $2$ & $1+7+5+1$ & odwracalny & nie \\
$12n_{200}$ & $2$ & $5$ & $3$ & $9$ & $0$ & $1+6+2$ & odwracalny & nie \\
$12n_{201}$ & $2$ & $5$ & $3$ & $105$ & $-4$ & $1+2-5-3$ & chiralny & nie \\
$12n_{202}$ & $2..3$ & $5$ & $3$ & $135$ & $2$ & $1+6+5+3$ & chiralny & nie \\
$12n_{203}$ & $3..4$ & $5$ & $3$ & $15$ & $-6$ & $1+8+13+3$ & chiralny & nie \\
$12n_{204}$ & $2..3$ & $5$ & $3$ & $93$ & $-4$ & $1+5+3-1$ & odwracalny & nie \\
$12n_{205}$ & $2$ & $5$ & $3$ & $77$ & $-4$ & $1+1+1+3+1$ & chiralny & nie \\
$12n_{206}$ & $2$ & $5$ & $3$ & $115$ & $2$ & $1+1-3-3-1$ & chiralny & nie \\
$12n_{207}$ & $3$ & $5$ & $3$ & $37$ & $-4$ & $1+3-1-5-1$ & odwracalny & nie \\
$12n_{208}$ & $2..3$ & $4$ & $3$ & $103$ & $2$ & $1+6+7+3$ & chiralny & nie \\
$12n_{209}$ & $2$ & $4$ & $3$ & $137$ & $-4$ & $1+2-3-3$ & chiralny & nie \\
$12n_{210}$ & $1..2$ & $5$ & $3$ & $1$ & $0$ & $1+4+5+1$ & chiralny & nie \\
$12n_{211}$ & $2$ & $5$ & $3$ & $51$ & $-2$ & $1-3+0+1$ & odwracalny & nie \\
$12n_{212}$ & $2..3$ & $4$ & $3$ & $103$ & $2$ & $1+6+7+3$ & chiralny & nie \\
$12n_{213}$ & $2$ & $4$ & $3$ & $137$ & $-4$ & $1+2-3-3$ & chiralny & nie \\
$12n_{214}$ & $1..2$ & $5$ & $3$ & $1$ & $0$ & $1+4+5+1$ & chiralny & nie \\
$12n_{215}$ & $2$ & $5$ & $3$ & $61$ & $-4$ & $1+1-4-2$ & chiralny & nie \\
$12n_{216}$ & $1..3$ & $5$ & $3$ & $83$ & $2$ & $1+5+4+2$ & chiralny & nie \\
$12n_{217}$ & $3$ & $5$ & $3$ & $5$ & $-4$ & $1+7+10+2$ & chiralny & nie \\
$12n_{218}$ & $1$ & $5$ & $3$ & $29$ & $0$ & $1+1-2-1$ & odwracalny & nie \\
$12n_{219}$ & $2$ & $5$ & $4$ & $99$ & $-2$ & $1+1-2-3-1$ & odwracalny & nie \\
$12n_{220}$ & $3$ & $5$ & $4$ & $21$ & $-4$ & $1+3-2-5-1$ & odwracalny & nie \\
$12n_{221}$ & $2$ & $5$ & $4$ & $9$ & $0$ & $1+2+1$ & odwracalny & nie \\
$12n_{222}$ & $2$ & $5$ & $4$ & $129$ & $-4$ & $1+4+1-2$ & odwracalny & nie \\
$12n_{223}$ & $2$ & $5$ & $4$ & $111$ & $-2$ & $1+0+1+2$ & odwracalny & nie \\
$12n_{224}$ & $2$ & $5$ & $4$ & $81$ & $0$ & $1+0-3-2$ & odwracalny & nie \\
$12n_{225}$ & $2$ & $5$ & $4$ & $39$ & $-2$ & $1-2-3$ & odwracalny & nie \\
$12n_{226}$ & $2$ & $5$ & $3$ & $77$ & $-4$ & $1+1+1+3+1$ & chiralny & nie \\
$12n_{227}$ & $2$ & $5$ & $3$ & $115$ & $2$ & $1+1-3-3-1$ & chiralny & nie \\
$12n_{228}$ & $3$ & $5$ & $3$ & $37$ & $-4$ & $1+3-1-5-1$ & odwracalny & nie \\
$12n_{229}$ & $3$ & $5$ & $4$ & $51$ & $-6$ & $1+5+2-3-1$ & odwracalny & nie \\
$12n_{230}$ & $1..2$ & $6$ & $3$ & $19$ & $-2$ & $1-3+2+1$ & odwracalny & nie \\
$12n_{231}$ & $1$ & $5$ & $3$ & $31$ & $2$ & $1+0-2$ & chiralny & nie \\
$12n_{232}$ & $1$ & $5$ & $3$ & $31$ & $2$ & $1+0-2$ & chiralny & nie \\
$12n_{233}$ & $2..3$ & $3$ & $3$ & $29$ & $4$ & $1+9+12+6+1$ & odwracalny & nie \\
$12n_{234}$ & $3$ & $3$ & $3$ & $43$ & $-6$ & $1+3-2-4-1$ & odwracalny & nie \\
$12n_{235}$ & $2..3$ & $3$ & $3$ & $13$ & $4$ & $1+5+10+6+1$ & odwracalny & nie \\
$12n_{236}$ & $2..3$ & $4$ & $3$ & $47$ & $2$ & $1+8+7+2$ & odwracalny & nie \\
$12n_{237}$ & $2$ & $4$ & $3$ & $73$ & $-4$ & $1+2-3-2$ & odwracalny & nie \\
$12n_{238}$ & $1..2$ & $4$ & $3$ & $31$ & $2$ & $1+4+7+2$ & odwracalny & nie \\
$12n_{239}$ & $2..3$ & $4$ & $3$ & $43$ & $2$ & $1+7+7+2$ & odwracalny & nie \\
$12n_{240}$ & $2$ & $4$ & $3$ & $77$ & $-4$ & $1+1-3-2$ & odwracalny & nie \\
$12n_{241}$ & $1..2$ & $4$ & $3$ & $59$ & $2$ & $1+3+5+2$ & odwracalny & nie \\
$12n_{242}$ & $5$ & $3$ & $3$ & $1$ & $-8$ & $1+12+31+27+9+1$ & odwracalny & nie \\
$12n_{243}$ & $4$ & $4$ & $3$ & $5$ & $-8$ & $1+11+23+13+2$ & odwracalny & nie \\
$12n_{244}$ & $4$ & $4$ & $3$ & $25$ & $-8$ & $1+10+20+12+2$ & odwracalny & nie \\
$12n_{245}$ & $3..4$ & $5$ & $3$ & $39$ & $-6$ & $1+10+16+4$ & odwracalny & nie \\
$12n_{246}$ & $1..2$ & $5$ & $3$ & $69$ & $0$ & $1+3+5$ & odwracalny & nie \\
$12n_{247}$ & $2$ & $5$ & $3$ & $51$ & $-2$ & $1+1-3$ & odwracalny & nie \\
$12n_{248}$ & $2..3$ & $5$ & $3$ & $25$ & $0$ & $1+6+3$ & odwracalny & nie \\
$12n_{249}$ & $1$ & $5$ & $3$ & $47$ & $-2$ & $1+0-3$ & odwracalny & nie \\
$12n_{250}$ & $1$ & $5$ & $3$ & $41$ & $0$ & $1+2+3$ & odwracalny & nie \\
$12n_{251}$ & $3$ & $5$ & $3$ & $19$ & $-6$ & $1+9+13+3$ & odwracalny & nie \\
$12n_{252}$ & $1$ & $5$ & $3$ & $89$ & $0$ & $1+2-2-2$ & chiralny & nie \\
$12n_{253}$ & $2..3$ & $4$ & $3$ & $55$ & $2$ & $1+6+6+2$ & chiralny & nie \\
$12n_{254}$ & $2$ & $4$ & $3$ & $65$ & $-4$ & $1+0-4-2$ & chiralny & nie \\
$12n_{255}$ & $1$ & $5$ & $3$ & $127$ & $-2$ & $1+0+0+2$ & chiralny & nie \\
$12n_{256}$ & $2$ & $5$ & $3$ & $25$ & $0$ & $1+2+2$ & chiralny & nie \\
$12n_{257}$ & $2$ & $4$ & $3$ & $25$ & $0$ & $1+2+2$ & chiralny & nie \\
$12n_{258}$ & $1..2$ & $5$ & $3$ & $15$ & $-2$ & $1+0-1$ & odwracalny & nie \\
$12n_{259}$ & $3..4$ & $4$ & $3$ & $95$ & $-6$ & $1+8+12+4$ & chiralny & nie \\
$12n_{260}$ & $2..3$ & $5$ & $3$ & $21$ & $0$ & $1+7+3$ & odwracalny & nie \\
$12n_{261}$ & $2$ & $5$ & $4$ & $69$ & $-4$ & $1-1+0+3+1$ & odwracalny & nie \\
$12n_{262}$ & $1$ & $5$ & $3$ & $89$ & $0$ & $1+2-2-2$ & chiralny & nie \\
$12n_{263}$ & $1$ & $5$ & $3$ & $127$ & $-2$ & $1+0+0+2$ & chiralny & nie \\
$12n_{264}$ & $2$ & $5$ & $3$ & $25$ & $0$ & $1+2+2$ & chiralny & nie \\
$12n_{265}$ & $1..2$ & $5$ & $3$ & $63$ & $-2$ & $1+0+4+2$ & chiralny & nie \\
$12n_{266}$ & $2$ & $5$ & $3$ & $105$ & $0$ & $1-2-2-2$ & chiralny & nie \\
$12n_{267}$ & $2$ & $5$ & $3$ & $63$ & $-2$ & $1+0-4$ & chiralny & nie \\
$12n_{268}$ & $2$ & $6$ & $3$ & $9$ & $0$ & $1-2$ & chiralny & nie \\
$12n_{269}$ & $2$ & $5$ & $3$ & $117$ & $0$ & $1-1+3-1$ & odwracalny & nie \\
$12n_{270}$ & $2..3$ & $5$ & $3$ & $63$ & $2$ & $1+4-3$ & odwracalny & nie \\
$12n_{271}$ & $1$ & $5$ & $3$ & $53$ & $0$ & $1-1-1-1$ & odwracalny & nie \\
$12n_{272}$ & $1$ & $5$ & $3$ & $83$ & $2$ & $1+1-1+1$ & odwracalny & nie \\
$12n_{273}$ & $2$ & $5$ & $3$ & $5$ & $-4$ & $1+3-3-1$ & odwracalny & nie \\
$12n_{274}$ & $2$ & $5$ & $3$ & $55$ & $-2$ & $1+2-3$ & odwracalny & nie \\
$12n_{275}$ & $1$ & $5$ & $3$ & $37$ & $0$ & $1-1+2$ & chiralny & nie \\
$12n_{276}$ & $3$ & $4$ & $3$ & $25$ & $4$ & $1+6+7+1$ & odwracalny & nie \\
$12n_{277}$ & $1$ & $4$ & $3$ & $67$ & $-2$ & $1+1+4+2$ & chiralny & nie \\
$12n_{278}$ & $2$ & $5$ & $3$ & $65$ & $0$ & $1+0+4$ & chiralny & nie \\
$12n_{279}$ & $1$ & $5$ & $3$ & $25$ & $0$ & $1-2+1$ & odwracalny & nie \\
$12n_{280}$ & $1..2$ & $5$ & $3$ & $51$ & $-2$ & $1-3+0+1$ & odwracalny & nie \\
$12n_{281}$ & $1..2$ & $5$ & $3$ & $77$ & $0$ & $1-3+0-1$ & chiralny & nie \\
$12n_{282}$ & $1..2$ & $5$ & $3$ & $21$ & $0$ & $1-1+1$ & odwracalny & nie \\
$12n_{283}$ & $2$ & $5$ & $3$ & $87$ & $-2$ & $1-2+2-2-1$ & odwracalny & nie \\
$12n_{284}$ & $1..2$ & $4$ & $3$ & $57$ & $0$ & $1-2-1-1$ & odwracalny & nie \\
$12n_{285}$ & $1..2$ & $5$ & $3$ & $39$ & $-2$ & $1-2+1+1$ & odwracalny & nie \\
$12n_{286}$ & $1..2$ & $5$ & $3$ & $45$ & $0$ & $1-3+2$ & chiralny & nie \\
$12n_{287}$ & $1..2$ & $5$ & $3$ & $111$ & $2$ & $1+0+1-2-1$ & odwracalny & nie \\
$12n_{288}$ & $1..2$ & $5$ & $3$ & $49$ & $0$ & $1+4+4$ & odwracalny & nie \\
$12n_{289}$ & $3$ & $5$ & $3$ & $91$ & $-6$ & $1+7+12+4$ & odwracalny & nie \\
$12n_{290}$ & $2$ & $4$ & $3$ & $69$ & $-4$ & $1+3+1-1$ & chiralny & nie \\
$12n_{291}$ & $1..3$ & $4$ & $3$ & $99$ & $2$ & $1+5+7+3$ & chiralny & nie \\
$12n_{292}$ & $4$ & $4$ & $3$ & $1$ & $-8$ & $1+8+18+12+2$ & chiralny & nie \\
$12n_{293}$ & $2$ & $4$ & $3$ & $7$ & $-2$ & $1+2$ & odwracalny & nie \\
$12n_{294}$ & $2..3$ & $5$ & $3$ & $45$ & $-4$ & $1-3-2+3+1$ & odwracalny & nie \\
$12n_{295}$ & $1..2$ & $5$ & $3$ & $107$ & $2$ & $1-1-3+1$ & odwracalny & nie \\
$12n_{296}$ & $2$ & $5$ & $3$ & $37$ & $-4$ & $1-1-2+3+1$ & odwracalny & nie \\
$12n_{297}$ & $2$ & $6$ & $3$ & $95$ & $-2$ & $1-4+1+2$ & odwracalny & nie \\
$12n_{298}$ & $1..2$ & $5$ & $3$ & $89$ & $0$ & $1-2+1-1$ & chiralny & nie \\
$12n_{299}$ & $2$ & $5$ & $3$ & $85$ & $-4$ & $1+3-6-3$ & odwracalny & nie \\
$12n_{300}$ & $1$ & $5$ & $3$ & $77$ & $0$ & $1+1+1-1$ & odwracalny & nie \\
$12n_{301}$ & $1..2$ & $5$ & $3$ & $67$ & $-2$ & $1+1+0-3-1$ & odwracalny & nie \\
$12n_{302}$ & $1..2$ & $5$ & $3$ & $111$ & $-2$ & $1+0-3+1$ & odwracalny & nie \\
$12n_{303}$ & $2$ & $4$ & $3$ & $41$ & $-4$ & $1+2-1-1$ & odwracalny & nie \\
$12n_{304}$ & $1..3$ & $4$ & $3$ & $71$ & $2$ & $1+6+9+3$ & odwracalny & nie \\
$12n_{305}$ & $4$ & $4$ & $3$ & $29$ & $-8$ & $1+9+20+12+2$ & odwracalny & nie \\
$12n_{306}$ & $1$ & $5$ & $3$ & $35$ & $-2$ & $1+1-2$ & odwracalny & nie \\
$12n_{307}$ & $1..3$ & $5$ & $3$ & $77$ & $0$ & $1+5+6$ & odwracalny & nie \\
$12n_{308}$ & $3$ & $5$ & $3$ & $63$ & $-6$ & $1+8+14+4$ & odwracalny & nie \\
$12n_{309}$ & $1..2$ & $4$ & $3$ & $1$ & $0$ & $1+4+5+1$ & odwracalny & nie \\
$12n_{310}$ & $1$ & $5$ & $3$ & $21$ & $0$ & $1+3+2$ & odwracalny & nie \\
$12n_{311}$ & $1$ & $5$ & $3$ & $51$ & $2$ & $1+1-3$ & odwracalny & nie \\
$12n_{312}$ & $1..2$ & $5$ & $3$ & $49$ & $0$ & $1+0-1-1$ & chiralny & nie \\
$12n_{313}$ & $1$ & $5$ & $3$ & $1$ & $0$ & $1$ & chiralny & nie \\
$12n_{314}$ & $1..2$ & $4$ & $3$ & $55$ & $2$ & $1-2+0+1$ & odwracalny & nie \\
$12n_{315}$ & $1..2$ & $4$ & $3$ & $61$ & $0$ & $1+1+0-1$ & chiralny & nie \\
$12n_{316}$ & $2$ & $4$ & $3$ & $65$ & $-4$ & $1+4+1-1$ & chiralny & nie \\
$12n_{317}$ & $2$ & $4$ & $3$ & $85$ & $0$ & $1-1-3-2$ & chiralny & nie \\
$12n_{318}$ & $1..2$ & $4$ & $3$ & $1$ & $0$ & $1+4+5+1$ & chiralny & nie \\
$12n_{319}$ & $2$ & $4$ & $3$ & $41$ & $-4$ & $1-2-6-2$ & chiralny & nie \\
$12n_{320}$ & $1$ & $4$ & $3$ & $109$ & $0$ & $1+1-1-2$ & chiralny & nie \\
$12n_{321}$ & $2$ & $4$ & $3$ & $11$ & $-2$ & $1+3$ & odwracalny & nie \\
$12n_{322}$ & $1$ & $5$ & $3$ & $23$ & $-2$ & $1-2-2$ & chiralny & nie \\
$12n_{323}$ & $1..2$ & $4$ & $3$ & $53$ & $0$ & $1+3+4$ & chiralny & nie \\
$12n_{324}$ & $1..2$ & $5$ & $3$ & $67$ & $-2$ & $1-3-5$ & chiralny & nie \\
$12n_{325}$ & $1$ & $4$ & $3$ & $113$ & $0$ & $1+0-1-2$ & chiralny & nie \\
$12n_{326}$ & $2$ & $4$ & $3$ & $53$ & $-4$ & $1+3+4+4+1$ & chiralny & nie \\
$12n_{327}$ & $2$ & $4$ & $3$ & $61$ & $4$ & $1+5+5+4+1$ & chiralny & nie \\
$12n_{328}$ & $4$ & $4$ & $3$ & $19$ & $-6$ & $1+9+17+8+1$ & odwracalny & nie \\
$12n_{329}$ & $3$ & $4$ & $3$ & $21$ & $4$ & $1+7+7+1$ & odwracalny & nie \\
$12n_{330}$ & $1..2$ & $4$ & $3$ & $87$ & $-2$ & $1+2+3+2$ & odwracalny & nie \\
$12n_{331}$ & $2$ & $5$ & $3$ & $45$ & $4$ & $1+1-5-2$ & chiralny & nie \\
$12n_{332}$ & $2$ & $5$ & $3$ & $9$ & $4$ & $1+6+2$ & odwracalny & nie \\
$12n_{333}$ & $2$ & $5$ & $3$ & $45$ & $0$ & $1+1+3$ & odwracalny & nie \\
$12n_{334}$ & $2$ & $5$ & $3$ & $63$ & $2$ & $1+0-4$ & chiralny & nie \\
$12n_{335}$ & $2$ & $4$ & $3$ & $95$ & $-2$ & $1+0+2+2$ & chiralny & nie \\
$12n_{336}$ & $2$ & $5$ & $3$ & $5$ & $4$ & $1-1-4-1$ & chiralny & nie \\
$12n_{337}$ & $1..2$ & $5$ & $3$ & $103$ & $2$ & $1-2-3+1$ & chiralny & nie \\
$12n_{338}$ & $4$ & $4$ & $3$ & $25$ & $-8$ & $1+10+20+12+2$ & odwracalny & nie \\
$12n_{339}$ & $2$ & $4$ & $3$ & $95$ & $-2$ & $1+4+7+3$ & odwracalny & nie \\
$12n_{340}$ & $1$ & $4$ & $3$ & $17$ & $0$ & $1+0-3-1$ & odwracalny & nie \\
$12n_{341}$ & $3$ & $5$ & $3$ & $67$ & $-6$ & $1+9+14+4$ & odwracalny & nie \\
$12n_{342}$ & $1$ & $5$ & $3$ & $53$ & $0$ & $1+3+4$ & odwracalny & nie \\
$12n_{343}$ & $1$ & $5$ & $3$ & $59$ & $-2$ & $1-1-4$ & odwracalny & nie \\
$12n_{344}$ & $2$ & $3$ & $3$ & $77$ & $4$ & $1+5+6+4+1$ & chiralny & nie \\
$12n_{345}$ & $1..2$ & $3$ & $3$ & $67$ & $-2$ & $1+1-4-4-1$ & chiralny & nie \\
$12n_{346}$ & $1..2$ & $4$ & $3$ & $13$ & $0$ & $1+5+10+6+1$ & chiralny & nie \\
$12n_{347}$ & $1$ & $4$ & $3$ & $31$ & $2$ & $1+4+3+1$ & odwracalny & nie \\
$12n_{348}$ & $1..2$ & $5$ & $3$ & $79$ & $2$ & $1+0-1+1$ & chiralny & nie \\
$12n_{349}$ & $2$ & $5$ & $3$ & $13$ & $4$ & $1+5-2-1$ & chiralny & nie \\
$12n_{350}$ & $1$ & $5$ & $3$ & $103$ & $2$ & $1+2-2+1$ & chiralny & nie \\
$12n_{351}$ & $1..2$ & $5$ & $3$ & $31$ & $-2$ & $1-4+1+1$ & chiralny & nie \\
$12n_{352}$ & $1$ & $5$ & $3$ & $7$ & $-2$ & $1-2+3+1$ & odwracalny & nie \\
$12n_{353}$ & $1..2$ & $5$ & $3$ & $103$ & $2$ & $1-2-3+1$ & chiralny & nie \\
$12n_{354}$ & $1..2$ & $5$ & $3$ & $23$ & $2$ & $1-2+2+1$ & odwracalny & nie \\
$12n_{355}$ & $2$ & $5$ & $3$ & $11$ & $2$ & $1-1+3+1$ & odwracalny & nie \\
$12n_{356}$ & $2$ & $5$ & $3$ & $61$ & $0$ & $1-3-1-1$ & -zwierciadlany & nie \\
$12n_{357}$ & $2$ & $5$ & $3$ & $99$ & $-2$ & $1-3-3+1$ & chiralny & nie \\
$12n_{358}$ & $1$ & $4$ & $3$ & $43$ & $2$ & $1-1-3$ & chiralny & nie \\
$12n_{359}$ & $1..2$ & $6$ & $3$ & $69$ & $0$ & $1-5+3$ & chiralny & nie \\
$12n_{360}$ & $1$ & $5$ & $3$ & $49$ & $0$ & $1+0+3$ & chiralny & nie \\
$12n_{361}$ & $2..3$ & $5$ & $3$ & $73$ & $-4$ & $1+2-7-3$ & chiralny & nie \\
$12n_{362}$ & $1$ & $4$ & $3$ & $31$ & $2$ & $1+0+2+1$ & chiralny & nie \\
$12n_{363}$ & $1..2$ & $5$ & $3$ & $87$ & $2$ & $1-2-2+1$ & chiralny & nie \\
$12n_{364}$ & $2$ & $5$ & $3$ & $99$ & $-2$ & $1-3+1+2$ & chiralny & nie \\
$12n_{365}$ & $2$ & $5$ & $3$ & $99$ & $2$ & $1-3+1+2$ & chiralny & nie \\
$12n_{366}$ & $3$ & $4$ & $3$ & $29$ & $4$ & $1+9+8+1$ & chiralny & nie \\
$12n_{367}$ & $1..2$ & $5$ & $3$ & $125$ & $0$ & $1+1+4-1$ & chiralny & nie \\
$12n_{368}$ & $3$ & $4$ & $3$ & $27$ & $-6$ & $1+3-1-4-1$ & chiralny & nie \\
$12n_{369}$ & $1$ & $4$ & $3$ & $75$ & $-2$ & $1-1-5-4-1$ & chiralny & nie \\
$12n_{370}$ & $1$ & $4$ & $3$ & $5$ & $0$ & $1-1-4-1$ & odwracalny & nie \\
$12n_{371}$ & $1$ & $4$ & $3$ & $11$ & $-2$ & $1+3+4+1$ & odwracalny & nie \\
$12n_{372}$ & $1$ & $4$ & $3$ & $43$ & $-2$ & $1+3+2+1$ & chiralny & nie \\
$12n_{373}$ & $2$ & $4$ & $3$ & $77$ & $-4$ & $1+5+2-1$ & odwracalny & nie \\
$12n_{374}$ & $4$ & $4$ & $3$ & $11$ & $-6$ & $1+11+22+13+2$ & odwracalny & nie \\
$12n_{375}$ & $2$ & $4$ & $3$ & $57$ & $-4$ & $1+2+0-1$ & odwracalny & nie \\
$12n_{376}$ & $1..2$ & $4$ & $3$ & $91$ & $-2$ & $1-1-6-4-1$ & chiralny & nie \\
$12n_{377}$ & $1..2$ & $4$ & $3$ & $39$ & $2$ & $1+2+2+1$ & odwracalny & nie \\
$12n_{378}$ & $1..2$ & $4$ & $3$ & $107$ & $2$ & $1+3+6+3$ & chiralny & nie \\
$12n_{379}$ & $2$ & $4$ & $3$ & $63$ & $2$ & $1+4+5+2$ & odwracalny & nie \\
$12n_{380}$ & $2$ & $4$ & $3$ & $81$ & $0$ & $1+0-3-2$ & odwracalny & nie \\
$12n_{381}$ & $1$ & $5$ & $3$ & $59$ & $-2$ & $1+3-3$ & chiralny & nie \\
$12n_{382}$ & $1..2$ & $5$ & $3$ & $19$ & $-2$ & $1+1-1$ & chiralny & nie \\
$12n_{383}$ & $1..2$ & $5$ & $3$ & $51$ & $-2$ & $1+1-3$ & odwracalny & nie \\
$12n_{384}$ & $2$ & $5$ & $3$ & $113$ & $-4$ & $1+0-5-3$ & chiralny & nie \\
$12n_{385}$ & $1..2$ & $4$ & $3$ & $73$ & $0$ & $1+2+1-1$ & chiralny & nie \\
$12n_{386}$ & $4$ & $4$ & $3$ & $9$ & $-8$ & $1+10+19+12+2$ & odwracalny & nie \\
$12n_{387}$ & $3$ & $5$ & $3$ & $27$ & $-6$ & $1+7+4-3-1$ & odwracalny & nie \\
$12n_{388}$ & $2$ & $5$ & $3$ & $45$ & $0$ & $1+1-1-1$ & odwracalny & nie \\
$12n_{389}$ & $2$ & $5$ & $3$ & $99$ & $-2$ & $1-3-3+1$ & odwracalny & nie \\
$12n_{390}$ & $1..2$ & $4$ & $3$ & $59$ & $-2$ & $1-1+0+1$ & chiralny & nie \\
$12n_{391}$ & $2$ & $5$ & $3$ & $115$ & $2$ & $1+5+6+3$ & odwracalny & nie \\
$12n_{392}$ & $1$ & $5$ & $3$ & $99$ & $-2$ & $1+1-2+1$ & odwracalny & nie \\
$12n_{393}$ & $2$ & $5$ & $3$ & $49$ & $0$ & $1+0+3$ & chiralny & nie \\
$12n_{394}$ & $2$ & $5$ & $3$ & $25$ & $0$ & $1-2+1$ & odwracalny & nie \\
$12n_{395}$ & $2$ & $5$ & $3$ & $77$ & $4$ & $1+5+2-1$ & odwracalny & nie \\
$12n_{396}$ & $1..2$ & $5$ & $3$ & $67$ & $2$ & $1-3-1+1$ & chiralny & nie \\
$12n_{397}$ & $2$ & $5$ & $3$ & $49$ & $0$ & $1+0-1-1$ & odwracalny & nie \\
$12n_{398}$ & $2$ & $5$ & $3$ & $21$ & $4$ & $1+3-2-1$ & chiralny & nie \\
$12n_{399}$ & $1..2$ & $5$ & $3$ & $81$ & $0$ & $1+0+1-1$ & chiralny & nie \\
$12n_{400}$ & $1$ & $5$ & $3$ & $95$ & $2$ & $1+0-2+1$ & chiralny & nie \\
$12n_{401}$ & $2$ & $5$ & $3$ & $87$ & $-2$ & $1+2-1+1$ & chiralny & nie \\
$12n_{402}$ & $3$ & $4$ & $3$ & $9$ & $4$ & $1+10+7+1$ & odwracalny & nie \\
$12n_{403}$ & $2$ & $4$ & $3$ & $9$ & $4$ & $1+2-3-1$ & odwracalny & nie \\
$12n_{404}$ & $2..3$ & $5$ & $3$ & $3$ & $2$ & $1+9+2$ & odwracalny & nie \\
$12n_{405}$ & $2$ & $5$ & $3$ & $73$ & $-4$ & $1+2-3+2+1$ & chiralny & nie \\
$12n_{406}$ & $3..4$ & $5$ & $3$ & $83$ & $-6$ & $1+9+17+5$ & chiralny & nie \\
$12n_{407}$ & $2$ & $4$ & $3$ & $37$ & $-4$ & $1+3-1-1$ & chiralny & nie \\
$12n_{408}$ & $1..2$ & $5$ & $3$ & $101$ & $0$ & $1+3+3-1$ & chiralny & nie \\
$12n_{409}$ & $2$ & $4$ & $3$ & $39$ & $2$ & $1+6+3+1$ & chiralny & nie \\
$12n_{410}$ & $2$ & $5$ & $3$ & $111$ & $-2$ & $1+4+2+2$ & chiralny & nie \\
$12n_{411}$ & $1..2$ & $5$ & $3$ & $19$ & $2$ & $1+5$ & odwracalny & nie \\
$12n_{412}$ & $1$ & $4$ & $3$ & $75$ & $2$ & $1+3+4+2$ & chiralny & nie \\
$12n_{413}$ & $2$ & $4$ & $3$ & $75$ & $-2$ & $1+3+4+2$ & chiralny & nie \\
$12n_{414}$ & $2$ & $5$ & $3$ & $25$ & $0$ & $1+2+2$ & odwracalny & nie \\
$12n_{415}$ & $1..2$ & $4$ & $3$ & $103$ & $-2$ & $1+2+2+2$ & chiralny & nie \\
$12n_{416}$ & $2..3$ & $4$ & $3$ & $77$ & $-4$ & $1+5+6+4+1$ & odwracalny & nie \\
$12n_{417}$ & $4$ & $3$ & $3$ & $35$ & $-6$ & $1+9+16+8+1$ & odwracalny & nie \\
$12n_{418}$ & $3$ & $4$ & $3$ & $19$ & $-6$ & $1+5+0-4-1$ & chiralny & nie \\
$12n_{419}$ & $3$ & $4$ & $3$ & $3$ & $-6$ & $1+5+1-4-1$ & chiralny & nie \\
$12n_{420}$ & $2$ & $5$ & $3$ & $81$ & $0$ & $1+0+1-1$ & odwracalny & nie \\
$12n_{421}$ & $1..2$ & $5$ & $3$ & $117$ & $0$ & $1-1-1-2$ & chiralny & nie \\
$12n_{422}$ & $1..2$ & $5$ & $3$ & $117$ & $0$ & $1-1-1-2$ & chiralny & nie \\
$12n_{423}$ & $2$ & $5$ & $3$ & $15$ & $2$ & $1+0+3+1$ & chiralny & nie \\
$12n_{424}$ & $1$ & $4$ & $3$ & $95$ & $-2$ & $1+0-2-3-1$ & chiralny & nie \\
$12n_{425}$ & $2$ & $4$ & $3$ & $33$ & $4$ & $1+4+3+4+1$ & odwracalny & nie \\
$12n_{426}$ & $4$ & $4$ & $3$ & $15$ & $-6$ & $1+8+17+8+1$ & odwracalny & nie \\
$12n_{427}$ & $2$ & $4$ & $3$ & $113$ & $-4$ & $1+4+4+3+1$ & chiralny & nie \\
$12n_{428}$ & $3$ & $4$ & $3$ & $31$ & $-6$ & $1+4-1-4-1$ & chiralny & nie \\
$12n_{429}$ & $1$ & $5$ & $3$ & $71$ & $2$ & $1+2-4$ & chiralny & nie \\
$12n_{430}$ & $1$ & $5$ & $3$ & $1$ & $0$ & $1$ & chiralny & nie \\
$12n_{431}$ & $1..2$ & $5$ & $3$ & $101$ & $0$ & $1-1+2-1$ & chiralny & nie \\
$12n_{432}$ & $2..3$ & $5$ & $3$ & $17$ & $4$ & $1+8+3$ & odwracalny & nie \\
$12n_{433}$ & $2$ & $5$ & $3$ & $11$ & $2$ & $1+3-4-1$ & odwracalny & nie \\
$12n_{434}$ & $1$ & $5$ & $3$ & $65$ & $0$ & $1+0+0-1$ & chiralny & nie \\
$12n_{435}$ & $1$ & $5$ & $3$ & $57$ & $0$ & $1+2+0-1$ & chiralny & nie \\
$12n_{436}$ & $2$ & $5$ & $3$ & $29$ & $4$ & $1+5-1-1$ & odwracalny & nie \\
$12n_{437}$ & $1..2$ & $5$ & $3$ & $29$ & $0$ & $1+5+7+1$ & chiralny & nie \\
$12n_{438}$ & $2$ & $4$ & $3$ & $21$ & $-4$ & $1-1-3-1$ & odwracalny & nie \\
$12n_{439}$ & $1..3$ & $4$ & $3$ & $3$ & $2$ & $1+5+5+1$ & odwracalny & nie \\
$12n_{440}$ & $2$ & $4$ & $3$ & $81$ & $0$ & $1+0-3-2$ & chiralny & nie \\
$12n_{441}$ & $2..3$ & $4$ & $3$ & $93$ & $-4$ & $1+5+3-1$ & odwracalny & nie \\
$12n_{442}$ & $2$ & $5$ & $3$ & $39$ & $-2$ & $1-2-3$ & odwracalny & nie \\
$12n_{443}$ & $1..3$ & $4$ & $3$ & $33$ & $0$ & $1+4+3$ & odwracalny & nie \\
$12n_{444}$ & $1..2$ & $5$ & $3$ & $27$ & $-2$ & $1-1-2$ & chiralny & nie \\
$12n_{445}$ & $2$ & $5$ & $3$ & $129$ & $-4$ & $1+4+1-2$ & odwracalny & nie \\
$12n_{446}$ & $1$ & $5$ & $3$ & $7$ & $-2$ & $1+2-4-1$ & odwracalny & nie \\
$12n_{447}$ & $1$ & $4$ & $3$ & $73$ & $0$ & $1-2-4-2$ & chiralny & nie \\
$12n_{448}$ & $1..2$ & $5$ & $3$ & $109$ & $0$ & $1-3-2+2+1$ & chiralny & nie \\
$12n_{449}$ & $1..2$ & $5$ & $3$ & $35$ & $-2$ & $1+1+2+1$ & odwracalny & nie \\
$12n_{450}$ & $1$ & $5$ & $3$ & $95$ & $-2$ & $1+0-2+1$ & odwracalny & nie \\
$12n_{451}$ & $2$ & $4$ & $3$ & $35$ & $2$ & $1+5+3+1$ & odwracalny & nie \\
$12n_{452}$ & $1..2$ & $4$ & $3$ & $55$ & $-2$ & $1-2-4$ & odwracalny & nie \\
$12n_{453}$ & $3..4$ & $5$ & $3$ & $115$ & $-6$ & $1+9+15+5$ & odwracalny & nie \\
$12n_{454}$ & $1..3$ & $4$ & $3$ & $47$ & $2$ & $1+4+2+1$ & chiralny & nie \\
$12n_{455}$ & $2$ & $4$ & $3$ & $97$ & $-4$ & $1+0-6-3$ & chiralny & nie \\
$12n_{456}$ & $1..2$ & $5$ & $3$ & $17$ & $0$ & $1+4+6+1$ & chiralny & nie \\
$12n_{457}$ & $1..2$ & $5$ & $3$ & $11$ & $2$ & $1+3$ & odwracalny & nie \\
$12n_{458}$ & $1..2$ & $5$ & $3$ & $129$ & $0$ & $1+0+0+2+1$ & odwracalny & nie \\
$12n_{459}$ & $2$ & $5$ & $3$ & $105$ & $-4$ & $1+2-1+2+1$ & odwracalny & nie \\
$12n_{460}$ & $2$ & $5$ & $3$ & $63$ & $2$ & $1+0+0+1$ & odwracalny & nie \\
$12n_{461}$ & $1$ & $5$ & $3$ & $85$ & $0$ & $1-1+1-1$ & chiralny & nie \\
$12n_{462}$ & $2$ & $5$ & $3$ & $25$ & $0$ & $1-2+1$ & -zwierciadlany & nie \\
$12n_{463}$ & $1..2$ & $5$ & $3$ & $77$ & $0$ & $1+1+1-1$ & chiralny & nie \\
$12n_{464}$ & $1$ & $4$ & $3$ & $15$ & $-2$ & $1+0-1$ & odwracalny & nie \\
$12n_{465}$ & $1..2$ & $5$ & $3$ & $83$ & $-2$ & $1+1+3-2-1$ & chiralny & nie \\
$12n_{466}$ & $2$ & $3$ & $3$ & $53$ & $4$ & $1+7+9+5+1$ & odwracalny & nie \\
$12n_{467}$ & $1$ & $3$ & $3$ & $43$ & $-2$ & $1-1-7-5-1$ & odwracalny & nie \\
$12n_{468}$ & $1$ & $3$ & $3$ & $37$ & $0$ & $1+3+7+5+1$ & odwracalny & nie \\
$12n_{469}$ & $2$ & $4$ & $3$ & $55$ & $2$ & $1+6+6+2$ & odwracalny & nie \\
$12n_{470}$ & $1..2$ & $4$ & $3$ & $41$ & $0$ & $1-2-6-2$ & odwracalny & nie \\
$12n_{471}$ & $1..2$ & $4$ & $3$ & $71$ & $-2$ & $1+2+4+2$ & odwracalny & nie \\
$12n_{472}$ & $5$ & $3$ & $3$ & $13$ & $-8$ & $1+13+32+27+9+1$ & odwracalny & nie \\
$12n_{473}$ & $4$ & $4$ & $3$ & $1$ & $-8$ & $1+12+23+13+2$ & odwracalny & nie \\
$12n_{474}$ & $4$ & $4$ & $3$ & $17$ & $-8$ & $1+12+24+13+2$ & odwracalny & nie \\
$12n_{475}$ & $1..2$ & $4$ & $3$ & $7$ & $2$ & $1+2+4+1$ & odwracalny & nie \\
$12n_{476}$ & $2$ & $4$ & $3$ & $73$ & $-4$ & $1-2-8-3$ & odwracalny & nie \\
$12n_{477}$ & $3..4$ & $5$ & $3$ & $43$ & $-6$ & $1+11+16+4$ & odwracalny & nie \\
$12n_{478}$ & $1..2$ & $5$ & $3$ & $29$ & $0$ & $1+1+2$ & odwracalny & nie \\
$12n_{479}$ & $1..2$ & $5$ & $3$ & $83$ & $-2$ & $1-3-6$ & odwracalny & nie \\
$12n_{480}$ & $2$ & $5$ & $3$ & $81$ & $0$ & $1-4+0-1$ & odwracalny & nie \\
$12n_{481}$ & $2$ & $5$ & $3$ & $65$ & $0$ & $1-4-1-1$ & odwracalny & nie \\
$12n_{482}$ & $1$ & $4$ & $3$ & $65$ & $0$ & $1+0+0-1$ & odwracalny & nie \\
$12n_{483}$ & $1$ & $4$ & $3$ & $15$ & $-2$ & $1+0-1$ & odwracalny & nie \\
$12n_{484}$ & $1$ & $4$ & $3$ & $51$ & $-2$ & $1+1+1+1$ & chiralny & nie \\
$12n_{485}$ & $1$ & $4$ & $3$ & $71$ & $-2$ & $1+2+4+2$ & chiralny & nie \\
$12n_{486}$ & $2$ & $4$ & $3$ & $55$ & $2$ & $1+2+5+2$ & odwracalny & nie \\
$12n_{487}$ & $1$ & $4$ & $3$ & $11$ & $-2$ & $1-1-5-1$ & odwracalny & nie \\
$12n_{488}$ & $1$ & $4$ & $3$ & $5$ & $0$ & $1+3+5+1$ & odwracalny & nie \\
$12n_{489}$ & $1..2$ & $5$ & $3$ & $119$ & $-2$ & $1+2+1+2$ & odwracalny & nie \\
$12n_{490}$ & $1..2$ & $5$ & $3$ & $77$ & $0$ & $1+1+5$ & odwracalny & nie \\
$12n_{491}$ & $1..2$ & $5$ & $3$ & $69$ & $0$ & $1+3+1-1$ & chiralny & nie \\
$12n_{492}$ & $1$ & $4$ & $3$ & $89$ & $0$ & $1-2-3-2$ & chiralny & nie \\
$12n_{493}$ & $1$ & $4$ & $3$ & $93$ & $0$ & $1-3-3-2$ & chiralny & nie \\
$12n_{494}$ & $3$ & $5$ & $3$ & $45$ & $-4$ & $1-3-2+3+1$ & odwracalny & nie \\
$12n_{495}$ & $2$ & $5$ & $3$ & $27$ & $2$ & $1+3+3+1$ & odwracalny & nie \\
$12n_{496}$ & $3$ & $5$ & $3$ & $117$ & $-4$ & $1+7+5-1$ & odwracalny & nie \\
$12n_{497}$ & $2$ & $5$ & $3$ & $105$ & $0$ & $1+2+3-1$ & odwracalny & nie \\
$12n_{498}$ & $2$ & $5$ & $3$ & $79$ & $-2$ & $1+0-5$ & chiralny & nie \\
$12n_{499}$ & $1$ & $5$ & $3$ & $45$ & $0$ & $1+1-1-1$ & odwracalny & nie \\
$12n_{500}$ & $1..2$ & $4$ & $3$ & $51$ & $2$ & $1+5+6+2$ & odwracalny & nie \\
$12n_{501}$ & $1..2$ & $5$ & $3$ & $49$ & $0$ & $1+4+4$ & odwracalny & nie \\
$12n_{502}$ & $4$ & $4$ & $3$ & $9$ & $-8$ & $1+10+23+13+2$ & odwracalny & nie \\
$12n_{503}$ & $3$ & $5$ & $3$ & $35$ & $-6$ & $1+9+16+4$ & odwracalny & nie \\
$12n_{504}$ & $1$ & $5$ & $3$ & $121$ & $0$ & $1-2-1+2+1$ & chiralny & nie \\
$12n_{505}$ & $2$ & $5$ & $3$ & $99$ & $2$ & $1+1-2+1$ & odwracalny & nie \\
$12n_{506}$ & $1..2$ & $5$ & $3$ & $59$ & $-2$ & $1-1+0+1$ & chiralny & nie \\
$12n_{507}$ & $1..2$ & $5$ & $3$ & $75$ & $2$ & $1-1-1+1$ & chiralny & nie \\
$12n_{508}$ & $2$ & $4$ & $3$ & $81$ & $-4$ & $1+4+6+4+1$ & odwracalny & nie \\
$12n_{509}$ & $2..3$ & $5$ & $3$ & $49$ & $-4$ & $1-4-2+3+1$ & odwracalny & nie \\
$12n_{510}$ & $2$ & $5$ & $3$ & $121$ & $4$ & $1+6+5-1$ & odwracalny & nie \\
$12n_{511}$ & $1$ & $4$ & $3$ & $85$ & $0$ & $1-1-3-2$ & odwracalny & nie \\
$12n_{512}$ & $2$ & $4$ & $3$ & $67$ & $2$ & $1+5+5+2$ & chiralny & nie \\
$12n_{513}$ & $2$ & $4$ & $3$ & $125$ & $-4$ & $1+5+1-2$ & chiralny & nie \\
$12n_{514}$ & $1..2$ & $4$ & $3$ & $35$ & $2$ & $1+5+3+1$ & chiralny & nie \\
$12n_{515}$ & $2$ & $5$ & $3$ & $89$ & $-4$ & $1+2-2-2$ & chiralny & nie \\
$12n_{516}$ & $2$ & $5$ & $3$ & $97$ & $-4$ & $1+4-1-2$ & chiralny & nie \\
$12n_{517}$ & $1..2$ & $4$ & $3$ & $101$ & $0$ & $1+3+7+4+1$ & chiralny & nie \\
$12n_{518}$ & $4$ & $4$ & $3$ & $63$ & $-6$ & $1+8+14+8+1$ & odwracalny & nie \\
$12n_{519}$ & $1..2$ & $5$ & $3$ & $7$ & $2$ & $1+2$ & odwracalny & nie \\
$12n_{520}$ & $1..2$ & $5$ & $3$ & $71$ & $2$ & $1+2-4$ & chiralny & nie \\
$12n_{521}$ & $1..2$ & $5$ & $3$ & $95$ & $-2$ & $1+0+2-2-1$ & odwracalny & nie \\
$12n_{522}$ & $1..3$ & $4$ & $3$ & $23$ & $2$ & $1+6+4+1$ & odwracalny & nie \\
$12n_{523}$ & $1..2$ & $5$ & $3$ & $13$ & $0$ & $1+5+2$ & odwracalny & nie \\
$12n_{524}$ & $1..3$ & $6$ & $3$ & $91$ & $-2$ & $1-5+1+2$ & odwracalny & nie \\
$12n_{525}$ & $1..2$ & $6$ & $3$ & $117$ & $0$ & $1-5+2-1$ & odwracalny & nie \\
$12n_{526}$ & $2$ & $4$ & $3$ & $29$ & $-4$ & $1+1+2+4+1$ & odwracalny & nie \\
$12n_{527}$ & $2..3$ & $5$ & $3$ & $93$ & $-4$ & $1+1-6-3$ & odwracalny & nie \\
$12n_{528}$ & $3$ & $4$ & $3$ & $37$ & $4$ & $1+7+8+1$ & chiralny & nie \\
$12n_{529}$ & $1..2$ & $5$ & $3$ & $115$ & $-2$ & $1-3-4+1$ & chiralny & nie \\
$12n_{530}$ & $1..2$ & $5$ & $3$ & $83$ & $-2$ & $1-3+2+2$ & chiralny & nie \\
$12n_{531}$ & $1..3$ & $4$ & $3$ & $97$ & $0$ & $1+4+7+4+1$ & chiralny & nie \\
$12n_{532}$ & $1..2$ & $5$ & $3$ & $125$ & $0$ & $1-3-1-2$ & chiralny & nie \\
$12n_{533}$ & $2$ & $4$ & $3$ & $63$ & $-2$ & $1+0-4-4-1$ & odwracalny & nie \\
$12n_{534}$ & $1$ & $4$ & $3$ & $109$ & $0$ & $1+1-1-2$ & chiralny & nie \\
$12n_{535}$ & $1$ & $5$ & $3$ & $27$ & $2$ & $1+3-1$ & odwracalny & nie \\
$12n_{536}$ & $1$ & $4$ & $3$ & $81$ & $0$ & $1+0+1-1$ & chiralny & nie \\
$12n_{537}$ & $1..2$ & $5$ & $3$ & $115$ & $-2$ & $1-3+0+2$ & odwracalny & nie \\
$12n_{538}$ & $2$ & $4$ & $3$ & $65$ & $-4$ & $1+0+0+3+1$ & odwracalny & nie \\
$12n_{539}$ & $1..2$ & $4$ & $3$ & $127$ & $-2$ & $1+0-4-3-1$ & chiralny & nie \\
$12n_{540}$ & $2$ & $4$ & $3$ & $91$ & $-2$ & $1-1+2+2$ & chiralny & nie \\
$12n_{541}$ & $1..2$ & $4$ & $3$ & $83$ & $2$ & $1+1+3+2$ & chiralny & nie \\
$12n_{542}$ & $1..2$ & $5$ & $3$ & $115$ & $2$ & $1+1-3+1$ & odwracalny & nie \\
$12n_{543}$ & $2$ & $4$ & $3$ & $63$ & $2$ & $1+4+1+1$ & odwracalny & nie \\
$12n_{544}$ & $1..2$ & $4$ & $3$ & $119$ & $-2$ & $1-2+0+2$ & odwracalny & nie \\
$12n_{545}$ & $2$ & $5$ & $3$ & $85$ & $-4$ & $1+3-2-2$ & chiralny & nie \\
$12n_{546}$ & $2$ & $4$ & $3$ & $63$ & $2$ & $1+0+4+2$ & odwracalny & nie \\
$12n_{547}$ & $1$ & $5$ & $3$ & $69$ & $0$ & $1-1+4$ & odwracalny & nie \\
$12n_{548}$ & $1..2$ & $4$ & $3$ & $85$ & $0$ & $1+3+2-1$ & odwracalny & nie \\
$12n_{549}$ & $2$ & $4$ & $3$ & $27$ & $-2$ & $1+3+3+1$ & odwracalny & nie \\
$12n_{550}$ & $1..2$ & $5$ & $3$ & $33$ & $0$ & $1+0+2$ & odwracalny & nie \\
$12n_{551}$ & $2$ & $5$ & $3$ & $105$ & $-4$ & $1+2-5-3$ & odwracalny & nie \\
$12n_{552}$ & $1$ & $5$ & $3$ & $9$ & $0$ & $1+2-3-1$ & odwracalny & nie \\
$12n_{553}$ & $3$ & $5$ & $4$ & $81$ & $0$ & $1-4+4$ & odwracalny & nie \\
$12n_{554}$ & $3$ & $6$ & $4$ & $27$ & $2$ & $1-1-2$ & odwracalny & nie \\
$12n_{555}$ & $3$ & $5$ & $4$ & $135$ & $2$ & $1+6-3+1$ & odwracalny & nie \\
$12n_{556}$ & $3$ & $5$ & $4$ & $81$ & $0$ & $1-4+4$ & odwracalny & nie \\
$12n_{557}$ & $2..3$ & $5$ & $3$ & $97$ & $-4$ & $1+0-6-3$ & chiralny & nie \\
$12n_{558}$ & $2$ & $5$ & $3$ & $35$ & $-2$ & $1+5+3+1$ & chiralny & nie \\
$12n_{559}$ & $2$ & $4$ & $3$ & $53$ & $-4$ & $1-1-5-2$ & chiralny & nie \\
$12n_{560}$ & $1$ & $4$ & $3$ & $133$ & $0$ & $1-1-4-3$ & chiralny & nie \\
$12n_{561}$ & $2$ & $5$ & $3$ & $55$ & $-2$ & $1-2-4$ & chiralny & nie \\
$12n_{562}$ & $2$ & $5$ & $3$ & $35$ & $-2$ & $1-3-3$ & chiralny & nie \\
$12n_{563}$ & $1..2$ & $5$ & $3$ & $71$ & $-2$ & $1-2-5$ & chiralny & nie \\
$12n_{564}$ & $1..2$ & $5$ & $3$ & $39$ & $2$ & $1+2-2$ & odwracalny & nie \\
$12n_{565}$ & $2$ & $4$ & $3$ & $45$ & $4$ & $1+1-5-2$ & chiralny & nie \\
$12n_{566}$ & $1$ & $5$ & $3$ & $63$ & $2$ & $1+0-4$ & chiralny & nie \\
$12n_{567}$ & $2$ & $4$ & $3$ & $117$ & $0$ & $1-1-5-3$ & chiralny & nie \\
$12n_{568}$ & $1..2$ & $4$ & $3$ & $107$ & $-2$ & $1+3+6+3$ & chiralny & nie \\
$12n_{569}$ & $1..3$ & $4$ & $3$ & $83$ & $2$ & $1+5+8+3$ & chiralny & nie \\
$12n_{570}$ & $2$ & $3$ & $3$ & $45$ & $4$ & $1+5+8+5+1$ & odwracalny & nie \\
$12n_{571}$ & $2$ & $3$ & $3$ & $27$ & $2$ & $1-1-6-5-1$ & odwracalny & nie \\
$12n_{572}$ & $1$ & $4$ & $3$ & $63$ & $2$ & $1+4+5+2$ & odwracalny & nie \\
$12n_{573}$ & $1$ & $4$ & $3$ & $57$ & $0$ & $1-2-5-2$ & odwracalny & nie \\
$12n_{574}$ & $5$ & $3$ & $3$ & $9$ & $-8$ & $1+14+32+27+9+1$ & odwracalny & nie \\
$12n_{575}$ & $4$ & $4$ & $3$ & $3$ & $-6$ & $1+13+23+13+2$ & odwracalny & nie \\
$12n_{576}$ & $4$ & $4$ & $3$ & $33$ & $-8$ & $1+12+21+12+2$ & odwracalny & nie \\
$12n_{577}$ & $1..2$ & $4$ & $3$ & $27$ & $-2$ & $1+3+3+1$ & odwracalny & nie \\
$12n_{578}$ & $1..2$ & $4$ & $3$ & $93$ & $0$ & $1-3-7-3$ & odwracalny & nie \\
$12n_{579}$ & $1$ & $5$ & $3$ & $9$ & $0$ & $1-2+4+1$ & odwracalny & nie \\
$12n_{580}$ & $2$ & $5$ & $3$ & $65$ & $0$ & $1+4+5$ & odwracalny & nie \\
$12n_{581}$ & $3$ & $5$ & $3$ & $27$ & $-6$ & $1+11+13+3$ & odwracalny & nie \\
$12n_{582}$ & $2$ & $5$ & $3$ & $9$ & $0$ & $1+2+1$ & odwracalny & nie \\
$12n_{583}$ & $2$ & $5$ & $3$ & $63$ & $-2$ & $1-4-5$ & odwracalny & nie \\
$12n_{584}$ & $1$ & $5$ & $3$ & $91$ & $-2$ & $1-1-2+1$ & chiralny & nie \\
$12n_{585}$ & $3$ & $5$ & $3$ & $55$ & $-6$ & $1+10+15+4$ & odwracalny & nie \\
$12n_{586}$ & $1..2$ & $5$ & $3$ & $101$ & $0$ & $1-1+2-1$ & chiralny & nie \\
$12n_{587}$ & $1..2$ & $5$ & $3$ & $117$ & $0$ & $1+3+4-1$ & odwracalny & nie \\
$12n_{588}$ & $1$ & $4$ & $3$ & $85$ & $0$ & $1-1-3-2$ & chiralny & nie \\
$12n_{589}$ & $2$ & $4$ & $3$ & $113$ & $-4$ & $1+4+0-2$ & chiralny & nie \\
$12n_{590}$ & $3$ & $4$ & $3$ & $23$ & $-6$ & $1+6+0-4-1$ & chiralny & nie \\
$12n_{591}$ & $4$ & $4$ & $3$ & $7$ & $-6$ & $1+10+18+8+1$ & odwracalny & nie \\
$12n_{592}$ & $2$ & $4$ & $3$ & $49$ & $4$ & $1+4+4+4+1$ & odwracalny & nie \\
$12n_{593}$ & $2$ & $5$ & $3$ & $61$ & $-4$ & $1+5-3-2$ & chiralny & nie \\
$12n_{594}$ & $3$ & $4$ & $3$ & $13$ & $-4$ & $1+9+11+2$ & odwracalny & nie \\
$12n_{595}$ & $1..2$ & $5$ & $3$ & $91$ & $2$ & $1+3+3+2$ & odwracalny & nie \\
$12n_{596}$ & $2$ & $5$ & $3$ & $95$ & $2$ & $1+4+7+3$ & odwracalny & nie \\
$12n_{597}$ & $1..2$ & $4$ & $3$ & $75$ & $-2$ & $1+3+4+2$ & chiralny & nie \\
$12n_{598}$ & $2$ & $5$ & $3$ & $99$ & $-2$ & $1+1-2+1$ & odwracalny & nie \\
$12n_{599}$ & $1$ & $4$ & $3$ & $55$ & $-2$ & $1+2-3-4-1$ & chiralny & nie \\
$12n_{600}$ & $3..4$ & $5$ & $3$ & $63$ & $-6$ & $1+12+15+4$ & odwracalny & nie \\
$12n_{601}$ & $2$ & $5$ & $3$ & $27$ & $-2$ & $1+3+3+1$ & odwracalny & nie \\
$12n_{602}$ & $2$ & $5$ & $3$ & $99$ & $-2$ & $1-3-3+1$ & odwracalny & nie \\
$12n_{603}$ & $3$ & $4$ & $3$ & $15$ & $-6$ & $1+4+0-4-1$ & odwracalny & nie \\
$12n_{604}$ & $2$ & $3$ & $3$ & $81$ & $4$ & $1+4+6+4+1$ & odwracalny & nie \\
$12n_{605}$ & $2$ & $4$ & $3$ & $9$ & $0$ & $1-2-8-6-1$ & odwracalny & nie \\
$12n_{606}$ & $2$ & $4$ & $3$ & $79$ & $2$ & $1+4+4+2$ & chiralny & nie \\
$12n_{607}$ & $1..2$ & $4$ & $3$ & $59$ & $2$ & $1+3+5+2$ & chiralny & nie \\
$12n_{608}$ & $1..2$ & $5$ & $3$ & $41$ & $0$ & $1+2+3$ & chiralny & nie \\
$12n_{609}$ & $2$ & $4$ & $3$ & $41$ & $-4$ & $1+2+3+4+1$ & chiralny & nie \\
$12n_{610}$ & $1$ & $4$ & $3$ & $65$ & $0$ & $1+0+4+4+1$ & odwracalny & nie \\
$12n_{611}$ & $2$ & $5$ & $3$ & $75$ & $-2$ & $1-1+3+2$ & odwracalny & nie \\
$12n_{612}$ & $2$ & $5$ & $3$ & $95$ & $-2$ & $1-4-3+1$ & odwracalny & nie \\
$12n_{613}$ & $1$ & $5$ & $3$ & $117$ & $0$ & $1-1-1+2+1$ & odwracalny & nie \\
$12n_{614}$ & $1$ & $5$ & $3$ & $73$ & $0$ & $1-2+0-1$ & chiralny & nie \\
$12n_{615}$ & $2$ & $5$ & $3$ & $99$ & $-2$ & $1-3+1-2-1$ & odwracalny & nie \\
$12n_{616}$ & $1$ & $5$ & $3$ & $65$ & $0$ & $1+0+0-1$ & chiralny & nie \\
$12n_{617}$ & $2$ & $5$ & $3$ & $5$ & $4$ & $1+3-3-1$ & chiralny & nie \\
$12n_{618}$ & $1..2$ & $5$ & $3$ & $107$ & $-2$ & $1-1+1-2-1$ & chiralny & nie \\
$12n_{619}$ & $2$ & $5$ & $3$ & $85$ & $4$ & $1+3-2+2+1$ & chiralny & nie \\
$12n_{620}$ & $1$ & $5$ & $3$ & $143$ & $-2$ & $1+0-1+2$ & chiralny & nie \\
$12n_{621}$ & $2$ & $5$ & $3$ & $95$ & $2$ & $1+4-1+1$ & chiralny & nie \\
$12n_{622}$ & $2$ & $5$ & $3$ & $99$ & $-2$ & $1+1+2-2-1$ & chiralny & nie \\
$12n_{623}$ & $2$ & $5$ & $3$ & $53$ & $-4$ & $1+3+0+3+1$ & chiralny & nie \\
$12n_{624}$ & $3$ & $4$ & $3$ & $13$ & $-4$ & $1+5-2-5-1$ & odwracalny & nie \\
$12n_{625}$ & $1..2$ & $5$ & $3$ & $91$ & $2$ & $1-1-2-3-1$ & odwracalny & nie \\
$12n_{626}$ & $3$ & $5$ & $3$ & $153$ & $4$ & $1+6+3-2$ & chiralny & nie \\
$12n_{627}$ & $1..2$ & $4$ & $3$ & $117$ & $0$ & $1+3+8+4+1$ & chiralny & nie \\
$12n_{628}$ & $1$ & $5$ & $3$ & $105$ & $0$ & $1-2+2-1$ & odwracalny & nie \\
$12n_{629}$ & $1..2$ & $5$ & $3$ & $27$ & $2$ & $1-1+2+1$ & odwracalny & nie \\
$12n_{630}$ & $2..3$ & $5$ & $3$ & $15$ & $-2$ & $1-4+2+1$ & odwracalny & nie \\
$12n_{631}$ & $1..3$ & $5$ & $3$ & $81$ & $0$ & $1-4+0-1$ & chiralny & nie \\
$12n_{632}$ & $1..2$ & $5$ & $3$ & $147$ & $2$ & $1+1+3-1-1$ & odwracalny & nie \\
$12n_{633}$ & $1$ & $4$ & $3$ & $97$ & $0$ & $1+0-2-2$ & chiralny & nie \\
$12n_{634}$ & $1..2$ & $5$ & $3$ & $47$ & $-2$ & $1+0+1+1$ & chiralny & nie \\
$12n_{635}$ & $1..2$ & $5$ & $3$ & $157$ & $0$ & $1-3-3+1+1$ & chiralny & nie \\
$12n_{636}$ & $2$ & $5$ & $3$ & $81$ & $0$ & $1+0+1-1$ & odwracalny & nie \\
$12n_{637}$ & $2$ & $5$ & $3$ & $135$ & $-2$ & $1-2-1+2$ & odwracalny & nie \\
$12n_{638}$ & $3$ & $4$ & $3$ & $15$ & $-6$ & $1+8+9+2$ & odwracalny & nie \\
$12n_{639}$ & $3$ & $4$ & $3$ & $35$ & $-6$ & $1+5-1-4-1$ & odwracalny & nie \\
$12n_{640}$ & $4$ & $3$ & $3$ & $27$ & $-6$ & $1+11+17+8+1$ & chiralny & nie \\
$12n_{641}$ & $2$ & $4$ & $3$ & $69$ & $4$ & $1+3+5+4+1$ & chiralny & nie \\
$12n_{642}$ & $3..4$ & $5$ & $4$ & $27$ & $2$ & $1+11+1$ & odwracalny & nie \\
$12n_{643}$ & $2$ & $5$ & $3$ & $49$ & $-4$ & $1+4-4-2$ & odwracalny & nie \\
$12n_{644}$ & $3$ & $4$ & $3$ & $7$ & $-6$ & $1+10+10+2$ & chiralny & nie \\
$12n_{645}$ & $1$ & $5$ & $3$ & $71$ & $2$ & $1+2+4+2$ & chiralny & nie \\
$12n_{646}$ & $1$ & $4$ & $3$ & $103$ & $-2$ & $1+2-2-3-1$ & chiralny & nie \\
$12n_{647}$ & $4$ & $3$ & $3$ & $23$ & $-6$ & $1+10+17+8+1$ & chiralny & nie \\
$12n_{648}$ & $3$ & $4$ & $3$ & $11$ & $-6$ & $1+7+1-4-1$ & chiralny & nie \\
$12n_{649}$ & $2$ & $4$ & $3$ & $77$ & $-4$ & $1+1+1+3+1$ & chiralny & nie \\
$12n_{650}$ & $1$ & $4$ & $3$ & $17$ & $0$ & $1+0+1$ & odwracalny & nie \\
$12n_{651}$ & $2$ & $5$ & $3$ & $119$ & $-2$ & $1+2+1+2$ & odwracalny & nie \\
$12n_{652}$ & $2$ & $4$ & $3$ & $139$ & $-2$ & $1-1-5-3-1$ & chiralny & nie \\
$12n_{653}$ & $3$ & $4$ & $3$ & $39$ & $-6$ & $1+6-1-4-1$ & odwracalny & nie \\
$12n_{654}$ & $3$ & $5$ & $3$ & $45$ & $-4$ & $1+5-4-2$ & odwracalny & nie \\
$12n_{655}$ & $3$ & $4$ & $3$ & $3$ & $-6$ & $1+9+10+2$ & chiralny & nie \\
$12n_{656}$ & $1$ & $5$ & $3$ & $99$ & $2$ & $1+1+2+2$ & chiralny & nie \\
$12n_{657}$ & $1..2$ & $4$ & $3$ & $81$ & $0$ & $1+0+5+4+1$ & chiralny & nie \\
$12n_{658}$ & $2$ & $4$ & $3$ & $89$ & $-4$ & $1+2+2+3+1$ & chiralny & nie \\
$12n_{659}$ & $2$ & $5$ & $3$ & $73$ & $-4$ & $1+6-2-2$ & chiralny & nie \\
$12n_{660}$ & $3$ & $4$ & $3$ & $45$ & $4$ & $1+9+9+1$ & odwracalny & nie \\
$12n_{661}$ & $2$ & $5$ & $3$ & $29$ & $-4$ & $1+1-6-2$ & chiralny & nie \\
$12n_{662}$ & $1..2$ & $5$ & $3$ & $59$ & $-2$ & $1-1+4+2$ & chiralny & nie \\
$12n_{663}$ & $1..2$ & $5$ & $3$ & $115$ & $-2$ & $1+1+1-2-1$ & odwracalny & nie \\
$12n_{664}$ & $1..2$ & $5$ & $3$ & $165$ & $0$ & $1-1-2+1+1$ & odwracalny & nie \\
$12n_{665}$ & $2$ & $5$ & $3$ & $39$ & $2$ & $1+2-2$ & chiralny & nie \\
$12n_{666}$ & $2$ & $3$ & $3$ & $81$ & $-4$ & $1+4+6+4+1$ & odwracalny & nie \\
$12n_{667}$ & $1$ & $4$ & $3$ & $75$ & $-2$ & $1+3+4+2$ & odwracalny & nie \\
$12n_{668}$ & $2$ & $4$ & $3$ & $105$ & $-4$ & $1+2+3+3+1$ & odwracalny & nie \\
$12n_{669}$ & $2$ & $5$ & $3$ & $99$ & $-2$ & $1+1+2+2$ & odwracalny & nie \\
$12n_{670}$ & $1..3$ & $4$ & $3$ & $25$ & $0$ & $1+6+11+6+1$ & chiralny & nie \\
$12n_{671}$ & $3$ & $4$ & $3$ & $95$ & $-6$ & $1+4+3-2-1$ & odwracalny & nie \\
$12n_{672}$ & $2$ & $5$ & $3$ & $51$ & $2$ & $1+5+2+1$ & chiralny & nie \\
$12n_{673}$ & $1..3$ & $4$ & $3$ & $5$ & $0$ & $1+7+14+7+1$ & chiralny & nie \\
$12n_{674}$ & $3$ & $3$ & $3$ & $75$ & $-6$ & $1+3+0-3-1$ & chiralny & nie \\
$12n_{675}$ & $1..3$ & $3$ & $3$ & $45$ & $0$ & $1+5+8+5+1$ & chiralny & nie \\
$12n_{676}$ & $1..3$ & $5$ & $3$ & $9$ & $0$ & $1+6+10+2$ & chiralny & nie \\
$12n_{677}$ & $2$ & $4$ & $3$ & $89$ & $-4$ & $1+2-2-2$ & chiralny & nie \\
$12n_{678}$ & $1..2$ & $4$ & $3$ & $79$ & $-2$ & $1+4+4+2$ & chiralny & nie \\
$12n_{679}$ & $5$ & $3$ & $3$ & $25$ & $-8$ & $1+10+28+26+9+1$ & chiralny & nie \\
$12n_{680}$ & $4$ & $4$ & $3$ & $19$ & $-6$ & $1+9+21+13+2$ & chiralny & nie \\
$12n_{681}$ & $1..3$ & $4$ & $3$ & $25$ & $0$ & $1+6+11+6+1$ & chiralny & nie \\
$12n_{682}$ & $3$ & $4$ & $3$ & $95$ & $-6$ & $1+4+3-2-1$ & chiralny & nie \\
$12n_{683}$ & $2$ & $3$ & $3$ & $89$ & $4$ & $1+6+7+4+1$ & chiralny & nie \\
$12n_{684}$ & $2$ & $3$ & $3$ & $77$ & $4$ & $1+5+6+4+1$ & chiralny & nie \\
$12n_{685}$ & $2$ & $4$ & $3$ & $79$ & $2$ & $1+4+4+2$ & chiralny & nie \\
$12n_{686}$ & $2$ & $4$ & $3$ & $97$ & $-4$ & $1+4+3+3+1$ & chiralny & nie \\
$12n_{687}$ & $1$ & $5$ & $3$ & $107$ & $2$ & $1+3+2+2$ & chiralny & nie \\
$12n_{688}$ & $5$ & $3$ & $3$ & $37$ & $-8$ & $1+11+29+26+9+1$ & chiralny & nie \\
$12n_{689}$ & $4$ & $4$ & $3$ & $7$ & $-6$ & $1+10+22+13+2$ & chiralny & nie \\
$12n_{690}$ & $1..2$ & $5$ & $3$ & $45$ & $0$ & $1+5+8+1$ & chiralny & nie \\
$12n_{691}$ & $4$ & $4$ & $3$ & $35$ & $-6$ & $1+9+20+13+2$ & odwracalny & nie \\
$12n_{692}$ & $4$ & $4$ & $3$ & $35$ & $-6$ & $1+9+20+13+2$ & odwracalny & nie \\
$12n_{693}$ & $3$ & $5$ & $3$ & $13$ & $-4$ & $1+5+2-4-1$ & chiralny & nie \\
$12n_{694}$ & $4$ & $4$ & $3$ & $35$ & $-6$ & $1+9+16+8+1$ & chiralny & nie \\
$12n_{695}$ & $1..2$ & $4$ & $3$ & $85$ & $0$ & $1+3+6+4+1$ & chiralny & nie \\
$12n_{696}$ & $3$ & $5$ & $3$ & $13$ & $-4$ & $1+5+2-4-1$ & chiralny & nie \\
$12n_{697}$ & $2$ & $5$ & $3$ & $21$ & $-4$ & $1+3+2+4+1$ & chiralny & nie \\
$12n_{698}$ & $2..3$ & $4$ & $3$ & $109$ & $4$ & $1+5+0-2$ & chiralny & nie \\
$12n_{699}$ & $2$ & $5$ & $3$ & $47$ & $2$ & $1+4-2$ & chiralny & nie \\
$12n_{700}$ & $2..3$ & $4$ & $3$ & $93$ & $4$ & $1+5-1-2$ & odwracalny & nie \\
$12n_{701}$ & $2..3$ & $5$ & $3$ & $63$ & $2$ & $1+4-3$ & odwracalny & nie \\
$12n_{702}$ & $1..2$ & $4$ & $3$ & $121$ & $0$ & $1+2+0-2$ & chiralny & nie \\
$12n_{703}$ & $2$ & $4$ & $3$ & $73$ & $-4$ & $1+2+1+3+1$ & chiralny & nie \\
$12n_{704}$ & $1$ & $5$ & $3$ & $61$ & $0$ & $1+1+4$ & chiralny & nie \\
$12n_{705}$ & $1$ & $5$ & $3$ & $109$ & $0$ & $1+1+3-1$ & chiralny & nie \\
$12n_{706}$ & $2$ & $5$ & $3$ & $49$ & $0$ & $1-4+2+4+1$ & całkowicie & nie \\
$12n_{707}$ & $2..3$ & $3$ & $3$ & $85$ & $4$ & $1+7+7+4+1$ & odwracalny & nie \\
$12n_{708}$ & $1..2$ & $3$ & $3$ & $49$ & $0$ & $1+4+8+5+1$ & chiralny & nie \\
$12n_{709}$ & $1$ & $3$ & $3$ & $61$ & $0$ & $1+1+4+4+1$ & chiralny & nie \\
$12n_{710}$ & $1$ & $4$ & $3$ & $95$ & $-2$ & $1+0+2+2$ & chiralny & nie \\
$12n_{711}$ & $1..2$ & $4$ & $3$ & $113$ & $0$ & $1+0+3+3+1$ & chiralny & nie \\
$12n_{712}$ & $2$ & $4$ & $3$ & $119$ & $2$ & $1+2+5+3$ & chiralny & nie \\
$12n_{713}$ & $2$ & $4$ & $3$ & $91$ & $2$ & $1+3+7+3$ & chiralny & nie \\
$12n_{714}$ & $2$ & $4$ & $3$ & $131$ & $-2$ & $1+1+4+3$ & chiralny & nie \\
$12n_{715}$ & $1$ & $5$ & $3$ & $89$ & $0$ & $1+2+6$ & chiralny & nie \\
$12n_{716}$ & $1..2$ & $4$ & $3$ & $79$ & $-2$ & $1+0+3+2$ & chiralny & nie \\
$12n_{717}$ & $2$ & $4$ & $3$ & $51$ & $2$ & $1+5+2+1$ & odwracalny & nie \\
$12n_{718}$ & $1$ & $4$ & $3$ & $55$ & $2$ & $1+2+5+2$ & odwracalny & nie \\
$12n_{719}$ & $1$ & $4$ & $3$ & $55$ & $-2$ & $1+2+1+1$ & chiralny & nie \\
$12n_{720}$ & $2$ & $5$ & $3$ & $101$ & $-4$ & $1+3-1-2$ & chiralny & nie \\
$12n_{721}$ & $1..3$ & $3$ & $3$ & $25$ & $0$ & $1+6+11+6+1$ & odwracalny & nie \\
$12n_{722}$ & $3$ & $3$ & $3$ & $55$ & $-6$ & $1+2-3-4-1$ & odwracalny & nie \\
$12n_{723}$ & $1..2$ & $4$ & $3$ & $19$ & $-2$ & $1+5+8+2$ & odwracalny & nie \\
$12n_{724}$ & $2$ & $4$ & $3$ & $61$ & $-4$ & $1+1-4-2$ & odwracalny & nie \\
$12n_{725}$ & $5$ & $3$ & $3$ & $5$ & $-8$ & $1+11+31+27+9+1$ & odwracalny & nie \\
$12n_{726}$ & $2$ & $5$ & $3$ & $63$ & $-2$ & $1+0-4$ & odwracalny & nie \\
$12n_{727}$ & $1$ & $5$ & $3$ & $91$ & $-2$ & $1-1+2+2$ & chiralny & nie \\
$12n_{728}$ & $1..2$ & $4$ & $3$ & $107$ & $-2$ & $1-1-3-3-1$ & chiralny & nie \\
$12n_{729}$ & $2$ & $4$ & $3$ & $73$ & $-4$ & $1+2-3-2$ & chiralny & nie \\
$12n_{730}$ & $1..2$ & $4$ & $3$ & $47$ & $-2$ & $1+4+2+1$ & chiralny & nie \\
$12n_{731}$ & $1..2$ & $5$ & $3$ & $121$ & $0$ & $1-2-1-2$ & chiralny & nie \\
$12n_{732}$ & $1$ & $5$ & $3$ & $115$ & $-2$ & $1+1-3+1$ & chiralny & nie \\
$12n_{733}$ & $1..2$ & $5$ & $3$ & $131$ & $-2$ & $1+1+0+2$ & chiralny & nie \\
$12n_{734}$ & $2..3$ & $4$ & $3$ & $85$ & $-4$ & $1+3-2-2$ & chiralny & nie \\
$12n_{735}$ & $1..2$ & $5$ & $3$ & $71$ & $-2$ & $1+2-4$ & chiralny & nie \\
$12n_{736}$ & $1$ & $5$ & $3$ & $151$ & $2$ & $1+2-1+2$ & chiralny & nie \\
$12n_{737}$ & $2$ & $4$ & $3$ & $45$ & $0$ & $1+1-5-2$ & odwracalny & nie \\
$12n_{738}$ & $2$ & $4$ & $3$ & $129$ & $-4$ & $1+4+1-2$ & chiralny & nie \\
$12n_{739}$ & $2$ & $4$ & $3$ & $13$ & $4$ & $1+1+1+4+1$ & odwracalny & nie \\
$12n_{740}$ & $1$ & $4$ & $3$ & $105$ & $0$ & $1-2-2-2$ & odwracalny & nie \\
$12n_{741}$ & $1..2$ & $4$ & $3$ & $95$ & $2$ & $1+4+7+3$ & odwracalny & nie \\
$12n_{742}$ & $2$ & $4$ & $3$ & $91$ & $2$ & $1+3+3+2$ & odwracalny & nie \\
$12n_{743}$ & $2$ & $5$ & $3$ & $85$ & $0$ & $1+3+6$ & odwracalny & nie \\
$12n_{744}$ & $2$ & $5$ & $3$ & $69$ & $-4$ & $1+3-3-2$ & odwracalny & nie \\
$12n_{745}$ & $2$ & $5$ & $3$ & $125$ & $0$ & $1+1+4-1$ & odwracalny & nie \\
$12n_{746}$ & $2$ & $5$ & $3$ & $119$ & $2$ & $1+2+1+2$ & odwracalny & nie \\
$12n_{747}$ & $2$ & $3$ & $3$ & $101$ & $4$ & $1+3+3+3+1$ & chiralny & nie \\
$12n_{748}$ & $1$ & $3$ & $3$ & $79$ & $-2$ & $1+0-5-4-1$ & chiralny & nie \\
$12n_{749}$ & $2$ & $3$ & $3$ & $7$ & $-2$ & $1+6+5+1$ & odwracalny & nie \\
$12n_{750}$ & $3$ & $3$ & $3$ & $21$ & $-4$ & $1+3+2$ & odwracalny & nie \\
$12n_{751}$ & $1..2$ & $3$ & $3$ & $69$ & $0$ & $1+3+5+4+1$ & odwracalny & nie \\
$12n_{752}$ & $2$ & $4$ & $3$ & $87$ & $-2$ & $1+2+3+2$ & odwracalny & nie \\
$12n_{753}$ & $2$ & $4$ & $3$ & $105$ & $-4$ & $1+2-1-2$ & chiralny & nie \\
$12n_{754}$ & $1$ & $4$ & $3$ & $105$ & $0$ & $1+2+3+3+1$ & odwracalny & nie \\
$12n_{755}$ & $2$ & $5$ & $3$ & $119$ & $-2$ & $1-2-4+1$ & chiralny & nie \\
$12n_{756}$ & $2$ & $5$ & $3$ & $99$ & $-2$ & $1+1+2+2$ & odwracalny & nie \\
$12n_{757}$ & $2$ & $5$ & $3$ & $115$ & $-2$ & $1+1-3+1$ & chiralny & nie \\
$12n_{758}$ & $3..4$ & $5$ & $3$ & $103$ & $-6$ & $1+10+16+5$ & chiralny & nie \\
$12n_{759}$ & $2$ & $5$ & $3$ & $111$ & $-2$ & $1+0-7$ & odwracalny & nie \\
$12n_{760}$ & $2$ & $5$ & $3$ & $125$ & $0$ & $1-3-1-2$ & odwracalny & nie \\
$12n_{761}$ & $2$ & $5$ & $3$ & $61$ & $-4$ & $1+1+0+3+1$ & chiralny & nie \\
$12n_{762}$ & $2$ & $5$ & $3$ & $69$ & $-4$ & $1-1+0+3+1$ & chiralny & nie \\
$12n_{763}$ & $2$ & $5$ & $3$ & $149$ & $0$ & $1+3+6-1$ & chiralny & nie \\
$12n_{764}$ & $3..4$ & $4$ & $3$ & $23$ & $-6$ & $1+10+13+3$ & odwracalny & nie \\
$12n_{765}$ & $2$ & $5$ & $3$ & $137$ & $-4$ & $1+2-3-3$ & chiralny & nie \\
$12n_{766}$ & $2$ & $5$ & $3$ & $97$ & $-4$ & $1+4-5-3$ & chiralny & nie \\
$12n_{767}$ & $1$ & $3$ & $3$ & $75$ & $2$ & $1-1-5-4-1$ & chiralny & nie \\
$12n_{768}$ & $1..2$ & $4$ & $3$ & $25$ & $0$ & $1-2-3-1$ & chiralny & nie \\
$12n_{769}$ & $2..3$ & $4$ & $3$ & $87$ & $2$ & $1+6+4+2$ & chiralny & nie \\
$12n_{770}$ & $2$ & $4$ & $3$ & $153$ & $-4$ & $1+2-2-3$ & chiralny & nie \\
$12n_{771}$ & $1$ & $4$ & $3$ & $103$ & $2$ & $1+2+2+2$ & chiralny & nie \\
$12n_{772}$ & $1..2$ & $4$ & $3$ & $127$ & $-2$ & $1+4+5+3$ & chiralny & nie \\
$12n_{773}$ & $2$ & $4$ & $3$ & $89$ & $4$ & $1-2-7-3$ & chiralny & nie \\
$12n_{774}$ & $1..2$ & $5$ & $3$ & $51$ & $2$ & $1-3-4$ & chiralny & nie \\
$12n_{775}$ & $1..2$ & $5$ & $3$ & $131$ & $-2$ & $1-3-1-2-1$ & chiralny & nie \\
$12n_{776}$ & $1..2$ & $5$ & $3$ & $107$ & $-2$ & $1-1+1-2-1$ & chiralny & nie \\
$12n_{777}$ & $1..2$ & $4$ & $3$ & $65$ & $0$ & $1+0-4-2$ & odwracalny & nie \\
$12n_{778}$ & $2$ & $4$ & $3$ & $85$ & $-4$ & $1-1-7-3$ & chiralny & nie \\
$12n_{779}$ & $2$ & $4$ & $3$ & $111$ & $-2$ & $1+0+1+2$ & chiralny & nie \\
$12n_{780}$ & $1..2$ & $5$ & $3$ & $81$ & $0$ & $1+4+2-1$ & chiralny & nie \\
$12n_{781}$ & $2$ & $4$ & $3$ & $93$ & $0$ & $1+1+6+4+1$ & odwracalny & nie \\
$12n_{782}$ & $1..2$ & $4$ & $3$ & $81$ & $0$ & $1-4-4-2$ & chiralny & nie \\
$12n_{783}$ & $1..2$ & $5$ & $3$ & $127$ & $-2$ & $1+4+1+2$ & chiralny & nie \\
$12n_{784}$ & $1..2$ & $5$ & $3$ & $123$ & $-2$ & $1+3+1-2-1$ & chiralny & nie \\
$12n_{785}$ & $1$ & $5$ & $3$ & $91$ & $2$ & $1-1-6$ & odwracalny & nie \\
$12n_{786}$ & $1$ & $5$ & $3$ & $139$ & $-2$ & $1+3+0+2$ & chiralny & nie \\
$12n_{787}$ & $1$ & $4$ & $3$ & $71$ & $-2$ & $1-2-5-4-1$ & chiralny & nie \\
$12n_{788}$ & $1..2$ & $5$ & $3$ & $109$ & $0$ & $1-3-2-2$ & chiralny & nie \\
$12n_{789}$ & $1$ & $5$ & $3$ & $89$ & $0$ & $1-2-3-2$ & chiralny & nie \\
$12n_{790}$ & $1..2$ & $4$ & $3$ & $133$ & $0$ & $1+3+5+3+1$ & chiralny & nie \\
$12n_{791}$ & $2..3$ & $4$ & $3$ & $93$ & $-4$ & $1+5+3+3+1$ & chiralny & nie \\
$12n_{792}$ & $1..2$ & $4$ & $3$ & $79$ & $2$ & $1-4-6-4-1$ & chiralny & nie \\
$12n_{793}$ & $2..3$ & $4$ & $3$ & $61$ & $-4$ & $1-3-9-3$ & chiralny & nie \\
$12n_{794}$ & $1..2$ & $5$ & $3$ & $101$ & $0$ & $1-5-3-2$ & chiralny & nie \\
$12n_{795}$ & $1..2$ & $5$ & $3$ & $79$ & $-2$ & $1-4-6$ & chiralny & nie \\
$12n_{796}$ & $2..3$ & $4$ & $3$ & $97$ & $-4$ & $1+4-1-2$ & chiralny & nie \\
$12n_{797}$ & $2$ & $5$ & $3$ & $59$ & $-2$ & $1+3-3$ & chiralny & nie \\
$12n_{798}$ & $2$ & $5$ & $3$ & $91$ & $2$ & $1+3-1+1$ & odwracalny & nie \\
$12n_{799}$ & $1..2$ & $5$ & $3$ & $135$ & $-2$ & $1-2-1+2$ & chiralny & nie \\
$12n_{800}$ & $1..2$ & $4$ & $3$ & $115$ & $2$ & $1+1-3-3-1$ & chiralny & nie \\
$12n_{801}$ & $2$ & $4$ & $3$ & $19$ & $-2$ & $1+1-1$ & odwracalny & nie \\
$12n_{802}$ & $1..2$ & $4$ & $3$ & $121$ & $0$ & $1+2+4+3+1$ & chiralny & nie \\
$12n_{803}$ & $2..3$ & $4$ & $3$ & $105$ & $-4$ & $1+6+4+3+1$ & odwracalny & nie \\
$12n_{804}$ & $1..3$ & $4$ & $3$ & $95$ & $-2$ & $1-4-7-4-1$ & odwracalny & nie \\
$12n_{805}$ & $1..3$ & $5$ & $3$ & $85$ & $0$ & $1-5-4-2$ & odwracalny & nie \\
$12n_{806}$ & $3..4$ & $4$ & $3$ & $63$ & $-6$ & $1+8+14+4$ & odwracalny & nie \\
$12n_{807}$ & $2$ & $4$ & $3$ & $57$ & $-4$ & $1+2-4-2$ & odwracalny & nie \\
$12n_{808}$ & $1..2$ & $4$ & $3$ & $31$ & $-2$ & $1+0-2$ & odwracalny & nie \\
$12n_{809}$ & $1..3$ & $5$ & $3$ & $15$ & $-2$ & $1-4-2$ & odwracalny & nie \\
$12n_{810}$ & $2$ & $5$ & $3$ & $55$ & $-2$ & $1-2+4+2$ & odwracalny & nie \\
$12n_{811}$ & $1..3$ & $4$ & $3$ & $63$ & $-2$ & $1+4+5+2$ & odwracalny & nie \\
$12n_{812}$ & $1..2$ & $4$ & $3$ & $9$ & $0$ & $1+2+1$ & chiralny & nie \\
$12n_{813}$ & $2$ & $5$ & $3$ & $81$ & $0$ & $1+0-3-2$ & odwracalny & nie \\
$12n_{814}$ & $1..3$ & $5$ & $3$ & $95$ & $-2$ & $1+0-6$ & odwracalny & nie \\
$12n_{815}$ & $1..2$ & $4$ & $3$ & $7$ & $-2$ & $1-2-1$ & odwracalny & nie \\
$12n_{816}$ & $1$ & $5$ & $3$ & $113$ & $0$ & $1+0-1-2$ & chiralny & nie \\
$12n_{817}$ & $2$ & $4$ & $3$ & $49$ & $0$ & $1-4-6-2$ & odwracalny & nie \\
$12n_{818}$ & $1..3$ & $5$ & $3$ & $91$ & $-2$ & $1-5-3+1$ & odwracalny & nie \\
$12n_{819}$ & $1..2$ & $4$ & $3$ & $107$ & $-2$ & $1+3+6+3$ & chiralny & nie \\
$12n_{820}$ & $3$ & $3$ & $3$ & $91$ & $6$ & $1+7+4-2-1$ & odwracalny & nie \\
$12n_{821}$ & $1..2$ & $3$ & $3$ & $35$ & $2$ & $1-3-7-5-1$ & odwracalny & nie \\
$12n_{822}$ & $1..2$ & $3$ & $3$ & $55$ & $-2$ & $1-2-8-5-1$ & odwracalny & nie \\
$12n_{823}$ & $2$ & $4$ & $3$ & $105$ & $4$ & $1+6+0-2$ & odwracalny & nie \\
$12n_{824}$ & $1..2$ & $4$ & $3$ & $65$ & $0$ & $1-4-5-2$ & odwracalny & nie \\
$12n_{825}$ & $1..2$ & $4$ & $3$ & $45$ & $0$ & $1-3-6-2$ & odwracalny & nie \\
$12n_{826}$ & $1..2$ & $5$ & $3$ & $99$ & $-2$ & $1-3-3+1$ & chiralny & nie \\
$12n_{827}$ & $2$ & $4$ & $3$ & $119$ & $-2$ & $1-2-4-3-1$ & chiralny & nie \\
$12n_{828}$ & $1$ & $4$ & $3$ & $109$ & $0$ & $1-3-6-3$ & chiralny & nie \\
$12n_{829}$ & $1..2$ & $3$ & $3$ & $91$ & $2$ & $1+3-1-3-1$ & odwracalny & nie \\
$12n_{830}$ & $3$ & $3$ & $3$ & $17$ & $-4$ & $1+4+2$ & odwracalny & nie \\
$12n_{831}$ & $1$ & $3$ & $3$ & $99$ & $-2$ & $1+1-2-3-1$ & odwracalny & nie \\
$12n_{832}$ & $2$ & $4$ & $3$ & $105$ & $4$ & $1+2-5-3$ & odwracalny & nie \\
$12n_{833}$ & $1..3$ & $4$ & $3$ & $99$ & $2$ & $1+5+7+3$ & odwracalny & nie \\
$12n_{834}$ & $2$ & $5$ & $3$ & $113$ & $-4$ & $1+4+0-2$ & chiralny & nie \\
$12n_{835}$ & $1..2$ & $4$ & $3$ & $13$ & $0$ & $1+5+6+1$ & odwracalny & nie \\
$12n_{836}$ & $2$ & $4$ & $3$ & $117$ & $-4$ & $1+3+0-2$ & chiralny & nie \\
$12n_{837}$ & $2$ & $5$ & $3$ & $169$ & $0$ & $1+2-1+1+1$ & odwracalny & nie \\
$12n_{838}$ & $2$ & $5$ & $3$ & $25$ & $0$ & $1-2+1$ & odwracalny & nie \\
$12n_{839}$ & $2$ & $5$ & $3$ & $121$ & $4$ & $1+2-4+1+1$ & odwracalny & nie \\
$12n_{840}$ & $2$ & $5$ & $3$ & $119$ & $2$ & $1-2+0-2-1$ & odwracalny & nie \\
$12n_{841}$ & $1..2$ & $5$ & $3$ & $81$ & $0$ & $1+0-3-2$ & chiralny & nie \\
$12n_{842}$ & $1..2$ & $5$ & $3$ & $181$ & $0$ & $1-1-1+1+1$ & chiralny & nie \\
$12n_{843}$ & $2$ & $5$ & $3$ & $147$ & $-2$ & $1+1+3+3$ & odwracalny & nie \\
$12n_{844}$ & $1..3$ & $6$ & $3$ & $75$ & $2$ & $1-5+6+3$ & odwracalny & nie \\
$12n_{845}$ & $2$ & $4$ & $3$ & $99$ & $-2$ & $1+1+6+3$ & odwracalny & nie \\
$12n_{846}$ & $2$ & $5$ & $3$ & $81$ & $0$ & $1+0+5$ & odwracalny & nie \\
$12n_{847}$ & $2$ & $5$ & $3$ & $75$ & $-2$ & $1-1-1+1$ & odwracalny & nie \\
$12n_{848}$ & $2$ & $4$ & $3$ & $85$ & $-4$ & $1+3+2+3+1$ & odwracalny & nie \\
$12n_{849}$ & $2$ & $4$ & $3$ & $161$ & $-4$ & $1+4-1-3$ & odwracalny & nie \\
$12n_{850}$ & $4$ & $3$ & $3$ & $15$ & $-6$ & $1+12+18+8+1$ & odwracalny & nie \\
$12n_{851}$ & $3..4$ & $4$ & $3$ & $5$ & $-4$ & $1+11+11+2$ & odwracalny & nie \\
$12n_{852}$ & $1$ & $4$ & $3$ & $125$ & $0$ & $1+1+4+3+1$ & chiralny & nie \\
$12n_{853}$ & $1..2$ & $4$ & $3$ & $125$ & $0$ & $1-3-5-3$ & chiralny & nie \\
$12n_{854}$ & $1..3$ & $5$ & $3$ & $143$ & $-2$ & $1+4+4+3$ & chiralny & nie \\
$12n_{855}$ & $1..3$ & $4$ & $3$ & $85$ & $0$ & $1-5-8-3$ & odwracalny & nie \\
$12n_{856}$ & $1..3$ & $5$ & $3$ & $55$ & $2$ & $1-6-5$ & odwracalny & nie \\
$12n_{857}$ & $1..2$ & $5$ & $3$ & $139$ & $2$ & $1-1-1-2-1$ & chiralny & nie \\
$12n_{858}$ & $1$ & $5$ & $3$ & $111$ & $2$ & $1+0+1+2$ & chiralny & nie \\
$12n_{859}$ & $1..3$ & $5$ & $3$ & $149$ & $0$ & $1-5+0+2+1$ & chiralny & nie \\
$12n_{860}$ & $2$ & $4$ & $3$ & $143$ & $2$ & $1+0+3+3$ & odwracalny & nie \\
$12n_{861}$ & $1..2$ & $4$ & $3$ & $95$ & $-2$ & $1+4+3+2$ & chiralny & nie \\
$12n_{862}$ & $2$ & $4$ & $3$ & $127$ & $-2$ & $1+4+5+3$ & chiralny & nie \\
$12n_{863}$ & $2..3$ & $4$ & $3$ & $145$ & $-4$ & $1+4-2-3$ & chiralny & nie \\
$12n_{864}$ & $1..2$ & $4$ & $3$ & $131$ & $-2$ & $1-3-5-3-1$ & chiralny & nie \\
$12n_{865}$ & $1..2$ & $4$ & $3$ & $99$ & $2$ & $1-3-7-4-1$ & chiralny & nie \\
$12n_{866}$ & $1..2$ & $4$ & $3$ & $141$ & $0$ & $1+1+5+3+1$ & chiralny & nie \\
$12n_{867}$ & $2..3$ & $4$ & $3$ & $133$ & $4$ & $1+7+2-2$ & odwracalny & nie \\
$12n_{868}$ & $1..2$ & $4$ & $3$ & $33$ & $0$ & $1+0+2$ & odwracalny & nie \\
$12n_{869}$ & $2$ & $5$ & $3$ & $135$ & $-2$ & $1+2+0+2$ & odwracalny & nie \\
$12n_{870}$ & $2$ & $5$ & $3$ & $25$ & $0$ & $1-2-3-1$ & chiralny & nie \\
$12n_{871}$ & $1$ & $5$ & $3$ & $119$ & $-2$ & $1+2-3+1$ & chiralny & nie \\
$12n_{872}$ & $1..2$ & $5$ & $3$ & $131$ & $2$ & $1+1+0-2-1$ & chiralny & nie \\
$12n_{873}$ & $1..3$ & $5$ & $3$ & $85$ & $0$ & $1-5+0+3+1$ & -zwierciadlany & nie \\
$12n_{874}$ & $2$ & $4$ & $3$ & $123$ & $-2$ & $1-1-4-3-1$ & chiralny & nie \\
$12n_{875}$ & $1..2$ & $4$ & $3$ & $99$ & $-2$ & $1-3-7-4-1$ & odwracalny & nie \\
$12n_{876}$ & $2$ & $5$ & $3$ & $81$ & $0$ & $1-4-4-2$ & odwracalny & nie \\
$12n_{877}$ & $2$ & $5$ & $3$ & $155$ & $2$ & $1-1+2-1-1$ & chiralny & nie \\
$12n_{878}$ & $2$ & $5$ & $3$ & $139$ & $-2$ & $1-1-5+1$ & chiralny & nie \\
$12n_{879}$ & $2$ & $4$ & $3$ & $143$ & $2$ & $1+0-1-2-1$ & odwracalny & nie \\
$12n_{880}$ & $1..2$ & $4$ & $3$ & $133$ & $0$ & $1-1-4-3$ & odwracalny & nie \\
$12n_{881}$ & $2..3$ & $5$ & $3$ & $121$ & $-4$ & $1+6+9$ & odwracalny & nie \\
$12n_{882}$ & $3$ & $3$ & $3$ & $99$ & $6$ & $1+5+3-2-1$ & odwracalny & nie \\
$12n_{883}$ & $2$ & $4$ & $3$ & $99$ & $-2$ & $1+1+6+3$ & chiralny & nie \\
$12n_{884}$ & $1$ & $5$ & $3$ & $81$ & $0$ & $1+0+5$ & chiralny & nie \\
$12n_{885}$ & $1..2$ & $5$ & $3$ & $149$ & $0$ & $1-1+1-2$ & chiralny & nie \\
$12n_{886}$ & $2$ & $5$ & $3$ & $159$ & $-2$ & $1+4-1+2$ & odwracalny & nie \\
$12n_{887}$ & $2$ & $3$ & $3$ & $125$ & $-4$ & $1+1+0+2+1$ & odwracalny & nie \\
$12n_{888}$ & $5$ & $3$ & $3$ & $45$ & $-8$ & $1+13+30+26+9+1$ & odwracalny & nie \\
\hline
\end{longtable}
\normalsize


\input{90-appendix/table_prime}

\chapter{Tablice węzłów wirtualnych}
\begin{comment}
\input{90-appendix/table_virtuals}
\end{comment}

\chapter{Notacja, użyte symbole}

\begin{itemize}
    \item $\N \subset \Z \subset \Q \subset \R \subset \C$: zbiór liczb naturalnych, całkowitych, wymiernych, rzeczywistych, zespolonych,
    \item $[a, b] = \{x \in \R \mid a \le x \le b\}$: przedział domknięty,
    \item $S^n = \{x \in \R^{n+1} \mid \|x\| = 1\}$, $n$-sfera
    \item $mK, rK$ lustro i rewers węzła,
    \item $\crossing, \braid, \bridge, \unknotting, \linking, \stick$ liczba/indeks skrzyżowaniowy, warkoczowy, mostowy, gordyjska, zaczepienia, patykowa,
    \item $\det$ wyznacznik,
    \item $\sigma$ sygnatura, być może Levine'a-Tristrama,
    \item $g, \chi$ genus i charakterystyka Eulera,
    \item $\#$ suma spójna,
    \item $\tau_p$ liczba $p$-kolorowań,
    \item $\operatorname{span}$ rozpiętość wielomianu,
    \item $\alexander$, $\conway$, $\jones$, $P$, $F$, $Q$ wielomian Alexandera, Conwaya, Jonesa, HOMFLY-PT, Kauffmana, BLM/Ho.
\end{itemize}

% $$PSL(2, 7)$ to rzutowa specjalna grupa liniowa nad ciałem $F_7$.



\chapter{Słownik angielsko-polski}
\begin{compactitem}
\item \textbf{2-przejście}: 2-pass move
\item \textbf{band sum}: suma paskowa
\item \textbf{cień}: shadow
\item \textbf{diagram}: diagram
\item \textbf{długość sznurowa}: ropelength
\item \textbf{indeks skrzyżowaniowy}: crossing number
\item \textbf{indeks zaczepienia}: linking number
\item \textbf{izotopia}: isotopy
(\emph{otaczająca}: ambient)
\item \textbf{liczba gordyjska}: unknotting number
\item \textbf{liczba mostowa}: bridge number
\item \textbf{liczba patykowa}: stick number
\item \textbf{niewęzeł}: unknot
\item \textbf{niezmiennik}: invariant
\item \textbf{nieściśliwy}: incompressible
\item \textbf{obramowanie}: framing
\item \textbf{ogniwo splotu}: component
\item \textbf{okres}: period
\item \textbf{pięciolistnik}: cinquefoil knot
\item \textbf{południk}: meridian
\item \textbf{ruch Reidemeistera/Turajewa/...}: Reidemeister/Turaev/... move
\item \textbf{równoleżnik}: latitude
\item \textbf{skrzyżowanie}: crossing
(\emph{znak}: sign)
\item \textbf{spin}: writhe
\item \textbf{splot}: link
(\emph{rozszczepialny}: splittable)
\item \textbf{suma niespójna}: distant union
\item \textbf{suma spójna}: connected sum
\item \textbf{szwindel Mazura}: Mazur swindle
\item \textbf{tablica węzłów}: knot table
\item \textbf{trójlistnik}: trefoil knot
\item \textbf{wygładzenie}: smoothing
\item \textbf{węzeł}: knot
(\emph{adekwatny}: adequate, \emph{alternujący}: alternating, \emph{dziki}: wild, \emph{długi}: long, \emph{ilorazowy}: quotient, \emph{lustro/lustrzany}: mirror, \emph{obramowany}: frame, \emph{odwracalny}: reversible, \emph{okresowy}: periodic, \emph{pierwszy}: prime, \emph{poskromiony}: tame, \emph{rewers/odwrotny}: reverse, \emph{skrętny/chiralny}: chiral, \emph{wirtualny}: virtual, \emph{zespawany}: welded, \emph{zorientowany}: oriented, \emph{zwierciadlany}: achiral/amphicheiral, \emph{złożony}: composite)
\item \textbf{węzeł babski}: granny knot
\item \textbf{węzeł dokerski}: stevedore knot
\item \textbf{ósemka}: figure-eight

\end{compactitem}


\raggedright
\bibliographystyle{plain}
\bibliography{knot_theory}

\indexprologue{\small Większość przymiotników, które pasują do węzłów, splotów, itd. wymieniona jest tylko raz, pod hasłem węzeł, chyba że to nie ma sensu (jak na przykład dla rozszczepialności), wtedy wymieniona jest raz, tam gdzie ma sens.}
\printindex

\indexprologue{\small Te wystąpienia nazwisk, gdzie chodziło o autora klasycznych podręczników, pominięto.}
\printindex[persons]

\end{document}



% https://www.mi.sanu.ac.rs/vismath/sl/l26.htm The recent progress is made by A.Stiomenow, who succeeded to prove that the Conjecture holds for a restricted class of knots: a rational knot of unknotting number one has an unknotting number one minimal diagram (A.Stoimenow: Vassiliev Invariants and Rational Knots of Unknotting Number One).

% Thistlethwaite https://doi.org/10.1017/CBO9781107359925.003:
% na stronie 67 wymienione jest 13 grup, których użył do klasyfikacji
% dopisać tę pozycję do historii tabel węzłów z dopiskiem, że jest napisana przyjemnym, niesuchym językiem

% There are many other points of interest concerning the group of a knot, a few of which we indicate here. It follows immediately from Dehn's lemma (Papa) that a longitude is trivial if and only if the knot is trivial. This gives us the celebrated unknotti-ng theorem, that TTjC IR3  - K] is infinite cyclic if and only if K is trivial. If a composite K# L is trivial, the amalgamated-free-product structure of *n"i C IR3  - K # L) gives us that the groups of K and L are both isomorphic to Z . This, together with the unknotting theorem, gives us the non-canoellation theorem: if K# L is unknotted, then K and L are unknotted. For another proof, which does not use the big guns of Dehn's lemma, but which uses the fact that the genus of a knot, i.e. the

% As a consequence of the sphere theorem (Papa), the complement of a Knot in S3  is aspherical, i.e. all its higher homotopy groups are trivial. Thus the complement of a Knot in S3  is an Eileriberg-Maelane space K(G^,1) . It follows that the homotopy type of S3  - K is determined by the isomorphism class of the group G . Also it is Known that if a group G contains non^trivial elements of finite order, then K(G,1) is infinite-dimensional ((Hem), p.76). Hence we have the interesting result that a Knot group is torsion-free. Hence each meridian-longitude pair in the group of a non-trivial Knot generates a subgroup isomorphic to Z + Z . A fairly recent result (Sha) is that a Knot group cannot contain an infinitely divisible element other than the identity (an element g is infinitely divisible if it is an nth power for infinitely many n).

"Arborescent knots are the ones which can be represented in terms of double fat graphs or equivalently as tree Feynman diagrams.". Find out whether they are useful or not. - węzeł drzewiasty?

In mathematics, Mostow's rigidity theorem, or strong rigidity theorem, or Mostow–Prasad rigidity theorem, essentially states that the geometry of a complete, finite-volume hyperbolic manifold of dimension greater than two is determined by the fundamental group and hence unique. The theorem was proven for closed manifolds by Mostow (1968) and extended to finite volume manifolds by Marden (1974) in 3 dimensions, and by Prasad (1973) in all dimensions at least 3. Gromov (1981) gave an alternate proof using the Gromov norm. Isomorphism pi M1 -> pi M2 can be lifted to isometry int M1 -> int M2.

https://mathoverflow.net/questions/104172/knot-diagrams-sets-of-moves-and-equivalence-relations

Niech K \subseteq S^3 będzie zorientowanym węzłem, który pogrubiamy do V(K), otoczenia tubularnego. Wtedy T = \partial V jest torusem. Kładziemy l = T \cap powierzchnia Seiferta dla K, "prostopadła" krzywa do l to m (longitude, merridian). Niech \hat V = S^3 \setminus int V. Wtedy cl \hat V \cap cl V = T (wspólny brzeg). Definiujemy m - merridian dla \hat V jako merridian dla V.

T0, torus trywialny z węzłem K_{q,r} na sobie. Odwzorowanie h: T_0 -> \hat T które posyła m na m, l na l ma tę własność, że h(K) =: \hat K (leży na \hat T jak K leży na T). Związane nazwisko - Kirby

. . . chirurgia Dehna, lens spaces, klasyfikacja. Twierdzenie: każda domknięta orientowalna spójna 3-rozmaitość powstaje w ch. Dehna

węzłów teoria,
mat. dział topologii zajmujący się badaniem, gł. za pomocą pojęć algebraicznych, położeń krzywych zwykłych zamkniętych, czyli tzw. węzłów, w trójwymiarowej przestrzeni euklidesowej ℝ3.
Podstawowymi pojęciami teorii węzłów są: węzeł, czyli krzywa zwykła zamknięta, oraz splot, czyli skończona liczba parami rozłącznych węzłów. Jako wzorzec dla mat. definicji węzła można przyjąć kawałek zawiązanego sznurka, w którym końce po zawiązaniu zostały sklejone. Podstawowym zagadnieniem teorii węzłów jest rozróżnienie i klasyfikacja węzłów. Dwa węzły są równoważne, jeśli można jeden z nich zdeformować w taki sposób, by otrzymać drugi (bez rozcinania) lub jeśli istnieje homeomorfizm ℝ3, zachowujący orientację, przekształcający jeden węzeł na drugi. Klasyfikacja węzłów polega na poszukiwaniu różnych wielkości (liczbowych, algebraicznych i in.), zw. niezmiennikami, pozostających takimi samymi dla węzłów równoważnych.
Badanie i klasyfikacja węzłów zostały zapoczątkowane w XIX w., gdy P.G. Tait zdefiniował rząd węzła, jako najmniejszą liczbę przecięć na tzw. diagramie, czyli rysunku rzutu węzła na płaszczyznę, przy czym przy rzutowaniu co najwyżej 2 punkty zlepiają się i tych zlepień jest skończona liczba. Jednak niezmiennik ten jest bardzo trudny do stosowania, zwłaszcza że wraz ze wzrostem rzędu bardzo gwałtownie (co najmniej wykładniczo) rośnie liczba nierównoważnych węzłów. Pełną klasyfikację węzłów do rzędu 9. oprac. w końcu lat 20. XX w. K. Reidemeister. Na pocz. XX w. zostało zdefiniowanych wiele innych niezmienników, m.in. J.V. Alexander przyporządkował węzłom pewne wielomiany (1928); w latach 70. i 80. XX w. wprowadzono kolejne wielomiany niezmiennicze dla węzłów, lepiej je odróżniające: wielomian J.H. Conwaya, V. Jonesa (który za prace nad teorią węzłów otrzymał 1990 Medal Fieldsa), HOMFLY (nazwa od inicjałów autorów), L.H. Kaufmanna. Istnieje związek między teorią węzłów a niektórymi działami fizyki, chemii, biologii; w samej matematyce teoria węzłów odgrywa szczególną rolę przy badaniu rozmaitości trójwymiarowych. (PWN)

http://stoimenov.net/stoimeno/homepage/ptab/

Fox p-colourings form a finite vector space over Z/p thus their number equals some power of p.

