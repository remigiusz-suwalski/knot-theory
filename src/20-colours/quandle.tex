\section{Kwandle i wraki} % (fold)
% TODO: Problems  on invariants of knots and 3-manifolds, rozdział 3
\label{sec:quandle}
Sekcja ta powstała częściowo w~oparciu o~notatki autorstwa Andrew Bergera, Chrisa Geriga\footnote{dostępne pod adresem \url{https://math.berkeley.edu/~cgerig/notes}} oraz Andrew Bergera, Brandona Flannery'ego i~Chrisa Sumnichta\footnote{dostępne pod adresem \url{https://github.com/thyrgle/191_Final_Project/blob/master/paper.pdf}}.
Kwandle, z~angielskiego \emph{quandle}, są strukturami algebraicznymi przypominającymi grupy.
Aksjomaty grupy stanowią uogólnienie symetrii -- symetrie są odwracalne, można je składać, identyczność jest symetrią.
Aksjomaty kwandli będą odzwierciedlać ruchy Reidemeistera.

David Joyce zapytany o znaczenie słowa \emph{quandle} odpowiedział: ,,I needed a usable word. “Distributive algebra” had too many syllables. Piffle was already taken. I tried trindle and quagle, but they didn’t seem right, so I went with quandle.''.

\begin{definition}[kwandl]
    \index{quandle}
    Zbiór $X$ wyposażony w dwuargumentowe działanie $\triangleright$ taki, że dla wszystkich elementów $x, y, z \in X$ zachodzi:
    \begin{enumerate}
        \item $x \triangleright x = x$,
        \item odwzorowanie $\beta_y \colon X \to X$ dane wzorem $\beta_y(x) = x \triangleright y$ jest odwracalne,
        \item $(x \triangleright y) \triangleright z = (x \triangleright z) \triangleright (y \triangleright z)$
    \end{enumerate}
    nazywamy kwandlem.
\end{definition}

Kwandle można rozpatrywać jako samodzielne konstrukcje algebraiczne.
My pokażemy, że są naturalnym niezmiennikiem węzłów.

Niech $X$ będzie skończonym kwandlem, zaś $K$ węzłem.
Elementy $x \in X$ będą dla nas kolorami, którymi oznaczymy długie łuki na diagramie węzła $K$.
Gdy trzy kolory spotykają się przy jednym skrzyżowaniu, definiujemy funkcję $\triangleright \colon X \times X \to X$, jak na rysunku.
To znaczy: kiedy łuk o kolorze $x$ przechodzi pod łukiem koloru $y$, staje się łukiem w kolorze $x \triangleright y$.

\[
    \begin{tikzpicture}[scale=0.18, baseline=0]
        \path[TIKZ_ARCH,Latex-] (-4,0) -- (4,8);
        \path[TIKZ_ARCH] (4,0) -- (1,3);
        \path[TIKZ_ARCH] (-1,5) -- (-4,8);
        \node[darkblue] at (-4, 0)[above left] {$y$};
        \node[darkblue] at (4, 0)[above right] {$x \triangleright y$};
        \node[darkblue] at (-4, 8)[below left] {$x$};
    \end{tikzpicture}
\]

Ta definicja pochodzi z~nieopublikowanej korespondencji między Johnem Conwayem i~Gavenem Wraithem, którzy w 1959 byli studentami I stopnia na uniwersytecie w Cambridge.
Ponownie odkryto ją w latach 80. XX wieku: Joyce w 1982 po raz pierwszy nazwał te obiekty kwandlami, Matwiejow w tym samym roku jako grupoidy rozdzielne, Brieskorn w 1986 jako zbiory automorficzne.

Drugi aksjomat nazywa się czasem odwracalnością z prawej strony: znając $x \triangleright y$ oraz $y$ możemy odtworzyć element $x$, jednak znając $x$ być może nie jesteśmy w stanie odtworzyć elementu $y$.
Jedyny element $x$ taki, że $x \triangleright y = z$ nazwijmy $y \triangleleft z$.
To pozwala podać trochę inną definicję kwandli, my nie będziemy jej używać.

\begin{definition}
    Zbiór $X$ z dwuargumentowymi działaniami $\triangleright, \triangleleft$ taki, że dla wszystkich $x, y, z \in X$ zachodzi:
    \begin{align*}
    x \triangleright x = x \triangleleft x & = x \\
    (x \triangleleft y) \triangleright x & = y \\
    x \triangleleft (y \triangleright x) & = y \\
     (x \triangleright z) \triangleright (y \triangleright z) & = (x \triangleright y) \triangleright z \\
    (x \triangleleft y) \triangleleft (x \triangleleft z) & = x \triangleleft (y \triangleleft z)
    \end{align*}
    nazywamy kwandlem.
\end{definition}

Teraz możemy przetłumaczyć ruchy Reidemeistera w aksjomaty kwandli.

\begin{proposition}
    Niech $X$ będzie skończonym kwandlem.
    Liczba etykietowań diagramu elementami kwandla $X$ jest niezmiennikiem węzłów, zwanym niezmiennikiem zliczającym.
\end{proposition}

\begin{proof}
    Musimy pokazać, że etykiety na diagramiem przed każdym ruchem Reidemeistera wyznaczają jednoznacznie układ etykiet po tym ruchu.
    Pierwszy ruch:
    \[
        \begin{tikzpicture}[baseline=-0.65ex,scale=0.07]
        \begin{knot}[clip width=5,flip crossing/.list={1}]
            \strand[semithick,Latex-] (15, 0) [in=up,out=left] to (-5, -7);
            \strand[semithick] (-5, -7) [in=down,out=down] to (5, -7);
            \strand[semithick] (5, -7) [in=right,out=up] to (-15, 0);
            \node[darkblue] at (-15, 0)[left] {$x$};
            \node[darkblue] at (15, 0)[right] {$x \triangleright x$};
        \end{knot}
        \end{tikzpicture}
        \stackrel{R_1}{\cong} \,\,
        \begin{tikzpicture}[baseline=-0.65ex,scale=0.07]
        \begin{knot}[clip width=5]
            \strand[semithick,-Latex] (-15, 0) to (15, 0);
            \node[darkblue] at (-15, 0)[left] {$x$};
        \end{knot}
        \end{tikzpicture}
    \]
    Drugi ruch:
    \[
        \begin{tikzpicture}[baseline=-0.65ex,scale=0.07]
        \begin{knot}[clip width=4, flip crossing/.list={1,2}]
            \strand[semithick,-Latex] (-10, -4) to (-7, -4) [in=left, out=right] to (0, 4) [in=left, out=right] to (7, -4) to (10, -4) to (15, -4);
            \strand[semithick] (-10, 4) to (-7, 4) [in=left, out=right] to (0, -4) [in=left, out=right] to (7, 4) to (10, 4) to (15, 4);
            \node[darkblue] at (-10, -4)[left] {$y$};
            \node[darkblue] at (-10, 4)[left] {$x$};
            \node[darkblue] at (0, 4) [above] {$y \triangleright x$};
            \node[darkblue] at (15, -4) [right] {$x \triangleleft (y \triangleright x)$};
        \end{knot}
        \end{tikzpicture}
        \stackrel{R_2}{\cong} \,\,
        \begin{tikzpicture}[baseline=-0.65ex,scale=0.07]
        \begin{knot}[clip width=4]
            \strand[semithick,-Latex] (-10, -4) to (-7, -4) [in=left, out=right] to (0, -1) [in=left, out=right] to (7, -4) to (10, -4) to (15, -4);
            \strand[semithick] (-10, 4) to (-7, 4) [in=left, out=right] to (0, 1) [in=left, out=right] to (7, 4) to (10, 4) to (15, 4);
            \node[darkblue] at (-10, -4)[left] {$y$};
            \node[darkblue] at (-10, 4)[left] {$x$};
        \end{knot}
        \end{tikzpicture}
    \]
    Trzeci ruch:
    \[
        \begin{tikzpicture}[baseline=-0.65ex,scale=0.07]
        \begin{knot}[clip width=5, flip crossing/.list={3}]
            \node[darkblue] at (-10, 10) [left] {$z$};
            \strand[semithick,-Latex] (-10, 10) [in=left, out=right] to (10,-10) to (15,-10);
            \node[darkblue] at (-10, 0) [left] {$y$};
            \strand[semithick,-Latex] (-10, 0) [in=left, out=right] to (0, 10) [in=left, out=right] to (10, 0) to (15, 0);
            \node[darkblue] at (-10, -10) [left] {$x$};
            \strand[semithick,-Latex] (-10, -10) [in=left, out=right] to (10,10) to (15,10);
            \node[darkblue] at (0, 10) [above] {$y \triangleright z$};
            \node[darkblue] at (15, 10) [right] {$(x \triangleright z) \triangleright (y \triangleright z)$};
        \end{knot}
        \end{tikzpicture}
        \stackrel{R_3}{\cong} \,\,
        \begin{tikzpicture}[baseline=-0.65ex,scale=0.07]
        \begin{knot}[clip width=5, flip crossing/.list={3}]
            \node[darkblue] at (-10, 10) [left] {$z$};
            \strand[semithick,-Latex] (-10, 10) [in=left, out=right] to (10,-10)  to (15, -10);
            \node[darkblue] at (-10, 0) [left] {$y$};
            \strand[semithick,-Latex] (-10, 0) [in=left, out=right] to (0, -10) [in=left, out=right] to (10, 0) to (15, 0);
            \node[darkblue] at (-10, -10) [left] {$x$};
            \strand[semithick,-Latex] (-10, -10) [in=left, out=right] to (10,10) to (15, 10);
            \node[darkblue] at (15, 10) [right] {$(x \triangleright y) \triangleright z$};
        \end{knot}
        \end{tikzpicture}
        \qedhere
    \]
\end{proof}

Homomorfizmy definiujemy standardowo, przez analogię do grup:

\begin{definition}
    Niech $Q_1, Q_2$ będą kwandlami.
    Odwzorowanie $f \colon Q_1 \to Q_2$, które dla waszystkich $x,y \in Q_1$ spełnia warunek $f(x \triangleright y) = f(x) \triangleright f(y)$, nazywamy homomorfizmem.
\end{definition}

Wiele znanych struktur algebraicznych okazuje się być źródłem kwandli.

\begin{example}[kwandl cykliczny/diedralny]
    Grupa abelowa z działaniem $x \triangleright y = 2y - x$.
\end{example}

\begin{example}[kwandle sprzężone]
    Grupa z działaniem $x \triangleright y = y^{-n} x y^n$.
\end{example}

\begin{example}[kwandl Alexandera]
    Moduł nad pierścieniem $\Z[t, 1/t]$ wielomianów Laurenta z~działaniem $x \triangleright y =tx + (1-t) y$.
\end{example}

\begin{example}[kwandl symplektyczny]
    Przestrzeń liniowa i antysymetryczna forma dwuliniowa $\langle \cdot | \cdot \rangle$ z działaniem $x \triangleright y = x + \langle x | y \rangle y$.
\end{example}

D. Joyce w swojej rozprawie doktorskiej przypisał każdemu węzłowi $K$ pewien szczególny kwandl $Q(K)$, kwandl podstawowy.
Definicja tego obiektu jest dość zawiła: łuki diagramu są generatorami, zaś skrzyżowania odpowiadają za relacje.
Joyce pokazał, że kwandl $Q(K)$ wyznacza węzeł $K$ jednoznacznie z dokładnością do orientacji.
Nie czyni to jednak nowego niezmiennika użytecznym, gdyż wyznaczenie go nawet w najprostszych przypadkach stanowi trudność.
Niebrzydowski, Przytycki pokazali w 2008 roku, że kwandl podstawowy trójlistnika jest izomorficzny z~rzutowym pierwotnym podkwandlem pewnych odwzorowań liniowych przestrzeni symplektycznej $\Z \oplus \Z$, cokolwiek to znaczy.

Aksjomaty grupy można wzmacniać (grupy abelowe) lub osłabiać (monoidy).
Podobnie czyni się z aksjomatami kwandli.
Kwandle inwolutywne odpowiadają węzłom bez orientacji, wraki dobrze opisują węzły obramowane (\emph{framed}), i tak dalej.

\begin{definition}[kwandl inwolutywny]
    Kwandl $Q$, w którym dla wszystkich $x, y \in Q$ zachodzi $x \triangleleft (x \triangleleft y) = y$, nazywamy inwolutywnym (albo kei).
\end{definition}

Kwandle inwolutywne badał jako pierwszy Mituhisa Takasaki (1943).
Szukał niełącznej struktury, która dobrze opisywałaby odbicia w skończonej geometrii.

\begin{definition}[półka]
    \index{shelf}
    Zbiór $X$ wyposażony w dwuargumentowe działanie $\triangleright$ taki, że dla wszystkich elementów $x, y, z \in X$ zachodzi $(x \triangleright y) \triangleright z = (x \triangleright z) \triangleright (y \triangleright z)$, nazywamy półką.
\end{definition}

\begin{example}
    Niech $B_\infty$ oznacza grupę wszystkich warkoczy, zaś $\phi$ będzie jej endomorfizmem posyłającym generator $\sigma_k$ na $\sigma_{k+1}$.
    Zbiór $B_\infty$ z działaniem $a \triangleleft b = a\phi(b)\sigma_1 \phi{a} ^{-1}$ jest półką.
\end{example}

Półki zdefiniowała Alissa Crans w pracy doktorskiej.
To nieprzetłumaczalna gra słów: dwie półki (\emph{shelves}), lewa i prawa, które dobrze do siebie pasują, dają stojak (\emph{rack}).
Półka stanowi uogólnienie dwóch obiektów -- wraków i wrzecion.

\begin{definition}[wrak]
    \index{wrack}
    Zbiór $X$ z dwuargumentowym działaniem $\triangleright$ taki, że dla wszystkich elementów $x, y, z \in X$ zachodzi:
    \begin{enumerate}
        \item odwzorowanie $\beta_y \colon X \to X$ dane wzorem $\beta_y(x) = x \triangleright y$ jest odwracalne,
        \item $(x \triangleright y) \triangleright z = (x \triangleright z) \triangleright (y \triangleright z)$
    \end{enumerate}
    nazywamy wrakiem (z angielskiego \emph{wrack}).
\end{definition}

\begin{example}
    Zbiór $X = \{1, 2, \ldots, n\}$ i permutacja $\sigma \in S_n$ z działaniem $x \triangleright y = \sigma(x)$.
\end{example}

\begin{example}
    Moduł nad pierścieniem $\Z[t^{\pm 1}, s]/(s^2 - (1-t)s)$ z działaniem $x \triangleright y = tx+sy$.
\end{example}

Wraki dobrze współgrają z podwójnym pierwszym ruchem Reidemeistera, który to nie zmienia spinu diagramu:
\[
    \begin{tikzpicture}[baseline=-0.65ex,scale=0.07]
    \begin{knot}[clip width=5,flip crossing/.list={1}]
        \strand[semithick] (15, 0) [in=up,out=left] to (-5, -7);
        \strand[semithick] (-5, -7) [in=down,out=down] to (5, -7);
        \strand[semithick] (5, -7) [in=right,out=up] to (-15, 0);
        \strand[semithick,Latex-] (45, 0) [in=up,out=left] to (25, -7);
        \strand[semithick] (25, -7) [in=down,out=down] to (35, -7);
        \strand[semithick] (35, -7) [in=right,out=up] to (15, 0);
        \node[darkblue] at (-15, 0)[left] {$x$};
        \node[darkblue] at (15, 0)[above] {$x \triangleright x$};
        \node[darkblue] at (45, 0)[right] {$x$};
    \end{knot}
    \end{tikzpicture}
    \cong
    \begin{tikzpicture}[baseline=-0.65ex,scale=0.07]
    \begin{knot}[clip width=5]
        \strand[semithick,-Latex] (-15, 0) to (15, 0);
        \node[darkblue] at (-15, 0)[left] {$x$};
    \end{knot}
    \end{tikzpicture}
\]

\begin{definition}[wrzeciono]
    \index{spindle}
    Zbiór $X$ z dwuargumentowym działaniem $\triangleright$ taki, że dla wszystkich elementów $x, y, z \in X$ zachodzi:
    \begin{enumerate}
        \item $x \triangleright x = x$,
        \item $(x \triangleright y) \triangleright z = (x \triangleright z) \triangleright (y \triangleright z)$
    \end{enumerate}
    nazywamy wrzecionem (z angielskiego \emph{spindle}).
\end{definition}

Zatem kwandle to wraki, które są też wrzecionami.
Muszę w~tym miejscu wtrącić uwagę językową.
Conway nazwał wraki wrakami (\emph{wracks}), by częściowo zażartować z~nazwiska jego kolegi Gavina Wraitha, a częściowo by zaznaczyć, że są one tym, co zostaje z~grupy, w~której zapomniano o~mnożeniu, ale nie sprzęganiu (w~języku angielskim co najmniej od XVI wieku funkcjonuje zwrot ,,wrack and ruin'' oznaczający zniszczenie).
Obecnie dominuje określenie \emph{racks}.

%Pokażemy, że kwandle uogólniają kolorowania.
%Niech $X$ będzie zbiorem kolorów z~operacją $\triangleright$, które spełnia aksjomaty z~definicji kwandli.
%Wtedy przy każdym skrzyżowaniu występują trzy kolory: $x$, $y$ oraz $x \triangleright y$.

%Przypomnijmy, że 3-kolorowanie diagramu polegało na przypisaniu każdemu włóknu pewnego koloru (z trzech) tak, by każdy został użyty, a~żadne skrzyżowanie nie stało się dwubarwne.
%Ogólniej, jeśli kolorami były liczby $0, \ldots, n - 1$, żądaliśmy od skrzyżowań, by kolor $y$ po przejściu pod kolorem $x$ stawał się $z$, gdzie $z \equiv 2x - y$ modulo $n$.
%Można to uogólnić jeszcze bardziej, właśnie do quandli: $\Z/n$-kolorowanie węzła to quandle związany z~pierścieniem $\Z/n$ operacją $x \triangleright y =  2y - x$}

% Koniec sekcji Wraki i~kwandle