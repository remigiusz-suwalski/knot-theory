\section{Grupa splotu}
Ponieważ dopełnienie dowolnego splotu, zarówno w przestrzeni $\R^3$ jak i $S^3$, jest łukowo spójne, jego grupa podstawowa nie zależy od wyboru punktu bazowego.
Dzięki temu poniższa definicja ma sens:

\begin{definition}
    \index{grupa!splotu}
    Niech $L$ będzie splotem.
    Grupę podstawową jego dopełnienia, $\pi_1(\R^3 \setminus L)$, nazywamy grupą splotu i oznaczamy $\pi(L)$.
\end{definition}

Kiedy mówimy o~grupie węzła, zazwyczaj mamy na myśli obiekt opisany powyżej, a nie grupę kolorującą z~definicji \ref{def:colouring_group}.
Nie należy ich mylić, grupa węzła ma bowiem dużo większe znaczenie.

Podamy najpierw kilka przykładów węzłów oraz ich grup.

\begin{example}
    Niewęzeł: $\Z$.
\end{example}

Z twierdzenia o pętli, czyli uogólnienia lematu Dehna wynika, że niewęzeł jest jedynym węzłem, którego grupą podstawową jest $\Z$.
% https://math.stackexchange.com/questions/3468034/knot-group-is-mathbbz-iff-k-is-the-unknot

\begin{example}
    \label{exm:trefoil_group}
    Trójlistnik: $\langle x, y \mid x^2 = y^3\rangle$.
\end{example}

\begin{proof}
    Wynika to z równości
    % https://en.wikipedia.org/wiki/Tietze_transformations
    \begin{align}
        \pi_1(S^3 \setminus 3_1) & = \langle x, y, z \mid xz = yx, zy = xz, yx = zy \rangle \\
                                 & = \langle x, y \mid xyx = yxy \rangle \\
                                 & = \langle x, y, a, b \mid xyx = yxy, a = yx, b = xyx \rangle \\
                                 & = \langle x, a, b \mid xa = a^2x^{-1}, b = xa \rangle \\
                                 & = \langle a, b \mid b = a^2(ba^{-1})^{-1} \rangle \\
                                 & = \langle a, b \mid a^3 = b^2 \rangle,
    \end{align}
    prawdziwych na mocy transformacji Tietzego.
\end{proof}

Trójlistnik jest węzłem $(3, 2)$-torusowym, więc powyższy przykład stanowi szczególny przypadek grupy węzła torusowego:

\begin{example}
    Węzeł $(p,q)$-torusowy: $\langle x, y \mid x^p = y^q \rangle$.
\end{example}

\begin{proof}
    Wniosek z twierdzenia Seiferta-van Kampena, patrz \cite[s. 77]{kawauchi96}.
\end{proof}

\begin{example}
    Węzeł ósemkowy: $\langle x, y \mid yxy^{{-1}}xy=xyx^{{-1}}yx \rangle$.
\end{example}

\begin{example}
    Splot Hopfa: $\Z \oplus \Z$.
\end{example}

Jak wspomina Kawauchi, z klasyfikacji abelowych grup podstawowych 3-rozmaitości wynika, że $\Z$ oraz $\Z \oplus \Z$ są jedynymi przemiennymi grupami splotów i podaje w formie ćwiczenia informację, że splot Hopfa oraz niewęzeł to jedyne sploty o przemiennej grupie \cite[s. 83]{kawauchi96}.

\begin{proposition}
    \label{prop:knot_group_invariant}
    Jeżeli sploty $L_1, L_2$ są równoważne, to grupy $\pi(L_1), \pi(L_2)$ są izomorficzne.
    Innymi słowy, grupa jest niezmiennikiem splotów.
\end{proposition}

\begin{proof}
    Gdy dwa sploty są równoważne, istnieje izotopijny z~identycznością homeomorfizm $\R^3 \to \R^3$, który posyła pierwszy splot na drugi.
    Obcięty do dopełnień splotów indukuje izomorfizm grup podstawowych.
\end{proof}

Na przykładzie grupy $\langle x,y,z \mid xyx=yxy,xzx=zxz\rangle$, która odpowiada zarówno sumie prostej różno-, jak i~jednoskrętnych trójlistników, widać że implikacja odwrotna nie zachodzi: mają one różne sygnatury (patrz uwaga za wnioskiem \ref{cor:acheiral_signature}).
Prawdziwe jest nawet ogólniejsze stwierdzenie:

\begin{proposition}

    Niech $K_1, K_2$ będą zorientowanymi węzłami.
    Wtedy węzłom $K_1 \shrap K_2$, $K_1 \shrap mr K_2$ odpowiadają izomorficzne grupy.
\end{proposition}

\begin{proof}
    Wniosek z twierdzeń o podgrupie południkowo-równoleżnikowej \cite[s. 75]{kawauchi96}.
\end{proof}

Twierdzenie odwrotne do faktu \ref{prop:knot_group_invariant} jest prawdziwe w klasie węzłów pierwszych:

\begin{proposition}
    Niech $K_1, K_2$ będą węzłami pierwszymi.
    Jeżeli ich grupy są izomorficzne, to same węzły są równoważne.
\end{proposition}

,,\emph{The group of a prime knot does not, however, necessarily determine the topological type of the exterior. Dehn hips on certain “essential” solid tori in the exteriors of torus knots and of cable knots produce Haken manifolds that are homotopically equivalent but not homeomorphic to the original exteriors and that, in fact, cannot be imbedded in $S^3$}'' (Whitten, \cite{whitten87}).

\begin{proof}
    % to jest kopia \cite[s. 76]{kawauchi96}
    Whitten pokazał w \cite{whitten87}, że węzły o izomorficznych grupach mają homeomorficzne dopełnienia.
    Wkrótce po tym Gordon, Luecke udowodnili w \cite{gordon89}, że nietrywialna chirurgia Dehna na nietrywialnym węźle nigdy nie daje sfery $S^3$, a stąd wynika, że każdy homeomorfizm dopełnień węzłów można przedłużyć do homeomorfizmu $S^3$ w siebie posyłającego jeden węzeł na drugi jako zbiory.
\end{proof}

Kawauchi proponuje jako proste ćwiczenie \cite[s. 73]{kawauchi96} pokazanie, że grupa sumy niespójnej splotów $L_1, L_2$ jest izomorficzna z wolnym produktem $\pi(L_1) * \pi(L_2)$, a następnie przytacza twierdzenie:

\begin{proposition}

    Niech $L \subseteq S^3$ będzie splotem.
    Następujące warunki są równoważne:
    \begin{enumerate}
        \item splot $L$ nie jest rozszczepialny,
        \item splot $L$ jest niewęzłem lub jego dopełnienie jest rozmaitością Hakena o~nieściśliwym brzegu,
        \item grupa podstawowa splotu $L$ jest nierozkładalna względem produktu wolnego.
    \end{enumerate}
\end{proposition}

\begin{proof}
    Z twierdzenia o pętli (niech $M$ będzie spójną 3-rozmaitością o niepustym brzegu, zaś $F$ powierzchnią na $\partial M$; jeżeli homomorfizm $\pi_1(F) \to \pi_1(M)$ indukowany przez inkluzję nie jest różnowartościowy, to istnieje dysk ściskający dla $F$ w $M$) i sferze (niech $M$ będzie spójną zorientowaną 3-rozmaitością, jeżeli $\pi_2(M)$ jest nietrywialna, to istnieje sfera właściwa w $M$) wynika, że $1 \implies 2$.
    Implikacja $2 \implies 3$ jest wnioskiem z hipotezy Knesera (niech $M$ będzie zwartą, spójną 3-rozmaitością, której brzeg jest pusty lub złożony z nieściśliwych powierzchni; jeśli $\pi_1(M) \cong G_1 * G_2$, to istnieje rozkład $M$ na sumę $M_1 \shrap M_2$ taką, że $\pi_1(M_i) \cong G_i$), zaś wynikanie $3 \implies 1$ jest oczywiste.
    To kończy dowód.
\end{proof}

\begin{corollary}

    Niech $L \subseteq S^3$ będzie splotem.
    Następujące warunki są równoważne:
    \begin{enumerate}
        \item grupa podstawowa splotu $L$ jest wolna, rangi $n$,
        \item splot $L$ jest trywialny, złożony z $n$ ogniw.
    \end{enumerate}
\end{corollary}

\begin{proposition}
    Niech $L \subseteq S^3$ będzie splotem.
    Następujące warunki są równoważne:
    \begin{enumerate}
        \item splot $L$ jest prosty i niepierścieniowaty\footnote{simple, anannular}
        \item grupa $\pi(L)$ jest nieprzemienną, nierozkładalną względem produktu wolnego podgrupą dyskretną $PSL_2(\C)$.
    \end{enumerate}
\end{proposition}

\begin{proof}
    Wniosek z twierdzenia Thurstona o hiperbolizacji, patrz \cite[s. 76]{kawauchi96}.
\end{proof}

Kawauchi wspomina jeszcze w \cite[s. 85]{kawauchi96}:

\begin{proposition}
    Grupa splotu jest rezydualnie skończona i lokalnie indeksowalna: dla każdej  nietrywialnej skończenie generowanej podgrupy, istnieje epimorfizm z niej w $\Z$.
\end{proposition}

\begin{proposition}
    Grupa węzła jest niezmiennikiem mocniejszym od genusu, a~w~przypadku węzłów złożonych, także od indeksu mostowego.
\end{proposition}

\begin{proof}[Niedowód]
    Jest to wniosek 3 z~pracy \cite{feustel78}.
\end{proof}


\subsection{Prezentacja Wirtingera}
\index{prezentacja Wirtingera|(}
Wiemy więc już trochę o~nowym niezmienniku, ale nie umiemy go jeszcze wyznaczać.
Jak zauważył Wilhelm Wirtinger około roku 1905, a więc jeszcze przed narodzinami teorii węzłów, grupa węzła zawsze posiada pewną specjalną prezentację, nazwaną na jego cześć prezentacją Wirtingera.
Jest to skończona prezentacja, w~której wszystkie relacje są postaci $w g_i w^{-1} = g_j$, gdzie $w$ to pewne słowo na generatorach, $g_1, \ldots, g_k$.
Przedstawimy ją zaraz ze względu na użyteczność w~rachunkach, dowodząc jednocześnie jej istnienia.

\begin{proposition}
    Grupa każdego węzła posiada prezentację Wirtingera.
\end{proposition}

\begin{proof}
    Oto zarys konstruktywnego dowodu.
    Przedstawiony algorytm jest bardzo wygodnym sposobem na wyznaczenie grupy węzła.
    Niech $K$ będzie węzłem z~diagramem o~$n$ łukach i~$m$ skrzyżowaniach.
    Wtedy
    \begin{equation}
        \pi_1(K) \cong \langle a_1, \ldots, a_n \mid r_1, \ldots, r_m\rangle,
    \end{equation}
    gdzie $a_i$ to włókna diagramu, zaś $r_x$ to relacje Wirtingera: $a_ia_ja_i^{-1}a_k^{-1}=1$,
\begin{comment}
    \begin{figure}[H]
    \begin{minipage}[b]{.48\linewidth}
        \[
            \LargeWirtingerRelationA
        \]
    \end{minipage}
    \begin{minipage}[b]{.48\linewidth}
        \[
            \LargeWirtingerRelationB
        \]
    \end{minipage}
    \end{figure}
\end{comment}
    w~których łuk $a_i$ biegnie górą, zaś $a_j$ leży po jego lewej stronie.
\end{proof}

\begin{comment}
\begin{figure}[H]
    \begin{minipage}[b]{.48\linewidth}
        \[
            \HugeWirtingerPlus
        \]
        \subcaption{skrzyżowanie dodatnie: $x_j = x_k x_{j+1} x_k^{-1}$}
    \end{minipage}
    \begin{minipage}[b]{.48\linewidth}
        \[
            \HugeWirtingerMinus
        \]
        \subcaption{skrzyżowanie ujemne: $x_j = x_k^{-1} x_{j+1} x_k$}
    \end{minipage}
\end{figure}
\end{comment}

\begin{corollary}
    Niech $G$ będzie grupą węzła.
    Wtedy jej abelianizacją jest $G^{\operatorname{ab}} = \Z$.
\end{corollary}

\begin{proof}
    Relacja $a_ia_ja_i^{-1}a_k^{-1}=1$ po przejściu do abelianizacji przyjmuje postać $a_j = a_k$.
    Oznacza to, że etykieta łuku nie zmienia się podczas przejścia pod każdym skrzyżowaniem, zatem wszystkie etykiety są takie same.

    Można też zauważyć, że abelianizacją grupy podstawowej węzła jest pierwsza grupa homologii okręgu, czyli $\Z$.
\end{proof}

Michel Kervaire w \cite{kervaire65} pokazał, jakie własności musi posiadać grupa węzła (i~wiemy o~tym, bo przeczytaliśmy książkę Kawauchiego, w tym \cite[tw. 14.1.1]{kawauchi96}):
\index[persons]{Kervaire, Michel}%

\begin{proposition}
    Niech $G$ będzie grupą węzła $S^n \subseteq S^{n+2}$.
    Wtedy:
    \begin{enumerate}[leftmargin=*]
        \itemsep0em
        \item grupa $G$ jest skończenie prezentowana,
        \item abelianizacja $G/G'$ jest nieskończoną grupą cykliczną,
        \item druga grupa homologii $H_2(G) = 0$ jest trywialna,
        \item istnieje element $x \in G$ zwany południkiem taki, że $G$ jest najmniejszą podgrupą normalną $G$, która zawiera $x$.
    \end{enumerate}
\end{proposition}

Wyżej wymienione warunki konieczne są także wystarczające, jeżeli $n \ge 3$, jednakże problem pełnej charakteryzacji w~czwartym wymiarze jest otwarty.
Warunki 2. i 3. wynikają z~dualności Alexandera, zaś 1. i 4. stanowią przeformułowanie prezentacji Wirtingera.

\index{prezentacja Wirtingera|)}

% koniec podsekcji Prezentacja Wirtingera




\subsection{Pochodna Foxa}
\index{pochodna Foxa|(}%
Istnieje alternatywna prezentacja grupy węzła, która pochodzi od Dehna, gdzie zamiast etykietować łuki, przypisuje się różne litery czterem częściom płaszczyzny, które są rozcinane przez skrzyżowanie.
Pomijamy tę prezentację dla oszczędności miejsca.
Klasycznie, jak na przykład w~\cite{crowell63}, macierz, a~co za tym idzie, także wielomian Alexandera wprowadza się przy użyciu prezentacji Wirtingera i~pochodnej Foxa.
Oryginalna praca Alexandera była jednak bliższa duchem pomysłom Dehna.

% DICTIONARY;Fox derivative;pochodna Foxa;-
\begin{definition}[pochodna Foxa]
    Niech $G$ będzie wolną grupą generowaną przez (niekoniecznie skończony) podzbiór $\{g_i\}_{i \in I}$.
    Odwzorowanie $\partial/\partial g_i \colon G \to \Z G$ spełniające trzy aksjomaty:
    \begin{align}
        \frac{\partial}{\partial g_i} (e) & = 0 \\
        \frac{\partial}{\partial g_i} (g_j) & = \delta_{ij} \\
        \forall u, v \in G : \frac{\partial}{\partial g_i} (uv) & = \frac{\partial}{\partial g_i}(u) + u \frac{\partial}{\partial g_i} (w),
    \end{align}
    gdzie $\delta_{ij}$ oznacza deltę Kroneckera, nazywamy pochodną cząstkową Foxa.
\end{definition}

Ustalmy prezentację grupy węzła z $n$ relacjami (słowami) $w_1, \ldots, w_n$ nad $n$-literowym alfabetem $x_1, \ldots, x_n$.
Zdefiniujmy następnie macierz Jacobiego wymiaru $n \times n$, elementami której są pochodne Foxa słów $w_i$ względem zmiennych $x_j$:
\begin{equation}
    J = \left(\frac{\partial w_i}{\partial x_j}\right).
\end{equation}

Wykreślmy z macierzy $J$ najpierw jedną kolumnę oraz jeden wiersz z tej macierzy, po czym podstawmy za wszystkie litery zmienną $t$ i policzmy wyznacznik.
Otrzymaliśmy znowu wielomian Alexandera.
Fox napisał cykl pięciu artykułów \cite{fox53}, \cite{fox54}, \cite{fox56}, \cite{fox58}, \cite{fox60} poświęcony wolnemu rachunkowi różniczkowemu, powyższa definicja jest tylko małym wycinkiem tego cyklu opublikowanego w Annals of Mathematics.

nLab wspomina jeszcze o \emph{a nice introduction in} \cite{crowell63}, podręczniku Crowella, Foxa.

\index{pochodna Foxa|)}%

% koniec podsekcji pochodna Foxa



% Koniec sekcji Grupa splotu
