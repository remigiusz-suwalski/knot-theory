\section{Grupa węzła. Prezentacja Wirtingera} % (fold)
\label{sec:group_wirtinger}
Kiedy mówimy o grupie węzła,
zazwyczaj mamy na myśli obiekt opisany poniżej,
a nie grupę kolorującą z definicji \ref{colgrp_def}.
Nie należy ich mylić.

\begin{definition}
    \label{def:knot_group}
    \index{grupa!węzła}
    Grupa węzła $K$ to grupa podstawowa jego dopełnienia, $\pi_1(\R^3 \setminus K)$.
\end{definition}

Przestrzeń $\R^3 \setminus K$ jest łukowo spójna,
zatem powyższa definicja nie zależy od wyboru punktu bazowego.
Oto grupy kilku najprostszych węzłów.

\begin{example}
    Niewęzeł: $\Z$.
\end{example}

\begin{example}
    Trójlistnik: grupa warkoczowa $B_3 \cong \langle x, y \mid x^2 = y^3\rangle$.
\end{example}

\begin{example}
    Węzeł $(p,q)$-torusowy: $\langle x, y \mid x^p = y^q \rangle$.
\end{example}

\begin{example}
    Węzeł ósemkowy: $\langle x, y \mid yxy^{{-1}}xy=xyx^{{-1}}yx \rangle$.
\end{example}

\begin{proposition}
    \label{prop:knot_group_invariant}
    Grupa węzła jest niezmiennikiem węzłów.
\end{proposition}

\begin{proof}
    Gdy dwa węzły są równoważne,
    istnieje izotopijny z identycznością homeomorfizm $\R^3 \to \R^3$,
    który posyła pierwszy węzeł na drugi.
    Obcięty do dopełnień węzłów indukuje izomorfizm grup podstawowych.
\end{proof}

\begin{proposition}
    Grupa węzła jest niezmiennikiem mocniejszym od genusu, a w przypadku węzłów złożonych, także od indeksu mostowego.
\end{proposition}

\begin{proof}[Niedowód]
    Jest to wniosek 3 z pracy \cite{feustel78}.
\end{proof}

Grupa podstawowa odróżnia od siebie dowolne dwa węzły pierwsze.
Na przykładzie grupy $\langle x,y,z \mid xyx=yxy,xzx=zxz\rangle$,
która odpowiada zarówno sumie prostej różno-, jak i jednoskrętnych trójlistników,
widać jednak, że nie jest niezmiennikiem zupełnym.
Prawdziwe jest ogólniejsze stwierdzenie:

\begin{proposition}
    \label{prop:knot_group_sum}
    Jeśli $K, L$ są węzłami, to grupa podstawowa nie odróżnia węzła $K \shrap L$ od $K \shrap -L^*$.
\end{proposition}

\index{prezentacja Wirtingera}
Wiemy więc już trochę o nowym niezmienniku, ale nie umiemy go jeszcze wyznaczać.
Jak zauważył Wilhelm Wirtinger około roku 1905 (!),
grupa węzła zawsze posiada pewną specjalną prezentację,
nazwaną na jego cześć prezentacją Wirtingera.
Jest to skończona prezentacja, w której wszystkie relacje są postaci $w g_i w^{-1} = g_j$,
gdzie $w$ to pewne słowo na generatorach, $g_1, \ldots, g_k$.
Przedstawimy ją zaraz ze względu na użyteczność w rachunkach,
dowodząc jednocześnie jej istnienia.

\begin{proposition}
    \label{prop:wirtinger}
    Każdy węzeł posiada prezentację Wirtingera.
\end{proposition}

\begin{proof}
    Oto zarys konstruktywnego dowodu.
    Przedstawiony algorytm jest bardzo wygodnym sposobem na wyznaczenie grupy węzła.
    Niech $K$ będzie węzłem z diagramem o $n$ łukach i $m$ skrzyżowaniach.
    Wtedy
    \begin{equation}
        \pi_1(K) \cong \langle a_1, \ldots, a_n \mid r_1, \ldots, r_m\rangle,
    \end{equation}
    gdzie $a_i$ to włókna diagramu, zaś $r_x$ to relacje Wirtingera: $a_ia_ja_i^{-1}a_k^{-1}=1$, \[
    \begin{tikzpicture}[baseline=-0.65ex,scale=0.15]
    \begin{knot}[clip width=15]
        \strand[semithick,-Latex] (-5, -5) to (5, 5);
        \strand[semithick,-Latex] (-5, 5) to (5, -5);
        \node[darkblue] at (5, 5)[below right] {$a_i$};
        \node[darkblue] at (5, -5)[above right] {$a_k$};
        \node[darkblue] at (-5, 5)[below left] {$a_j$};
    \end{knot}
    \end{tikzpicture}
    \quad\quad
    \begin{tikzpicture}[baseline=-0.65ex,scale=0.15]
    \begin{knot}[clip width=15, flip crossing/.list={1}]
        \strand[semithick,-Latex] (-5, -5) to (5, 5);
        \strand[semithick,-Latex] (-5, 5) to (5, -5);
        \node[darkblue] at (5, 5)[below right] {$a_j$};
        \node[darkblue] at (-5, -5)[above left] {$a_k$};
        \node[darkblue] at (-5, 5)[below left] {$a_i$};
    \end{knot}
    \end{tikzpicture}
    \]
    w których łuk $a_i$ biegnie górą, zaś $a_j$ leży po jego lewej stronie.
\end{proof}

Istnieje alternatywna prezentacja grupy węzła, która pochodzi od Dehna,
gdzie zamiast etykietować łuki,
przypisuje się różne litery czterem częściom płaszczyzny,
które są rozcinane przez skrzyżowanie.
Pomijamy tę prezentację dla oszczędności miejsca.
Klasycznie (jak na przykład w \cite{crowell63}) macierz, a co za tym idzie,
także wielomian Alexandera wprowadza się przy użyciu prezentacji Wirtingera i pochodnej Foxa.
Oryginalna praca Alexandera była jednak bliższa duchem pomysłom Dehna.

\begin{definition}
    \index{pochodna Foxa}
    Niech $G$ będzie grupą generowaną przez (niekoniecznie skończony) podzbiór $\{g_i\}_{i \in I}$..
    Pochodna Foxa to różniczkowanie $G \to \Z G$: addytywne odwzorowanie spełniające trzy aksjomaty:
    \begin{enumerate}
        \item $(\partial/\partial g_i)(e) = 0$;
        \item $(\partial/\partial g_i)(g_j) = 1$, jeśli $i = j$, $0$ w przeciwnym razie;
        \item $(\partial/\partial g_i)(uw) = (\partial/\partial g_i)(u) + u(\partial/\partial g_i)(w)$ dla dowolnych słów $u, w \in G$.
    \end{enumerate}
\end{definition}

\todo[inline]{Macierz pochodnych, potem zamiana wszystkich zmiennych na $t$, potem policzenie wyznacznika: to jest wielomian Alexandera?}

Nietrywialne węzły, których grupa ma nietrywialne centrum, to dokładnie węzły torusowe.
Wszystkie one są węzłami Neuwirtha
(z grupą o skończenie generowanym komutancie, wprowadzone w 1965 roku).

\begin{corollary}
    \label{prop:knot_group_abelianisation}
    Abelianizacją grupy węzła jest pierwsza grupa homologii okręgu, $\Z$.
\end{corollary}

Dwa następne stwierdzenia są już trudniejsze w dowodzie,
na przykład uzasadnienie pierwszego może wymagać:
twierdzenia o sferze, o pętli oraz hipotezy Knesera.

\begin{proposition}
    \label{prop:knot_group_split}
    Następujące warunki są równoważne:
    splot $L$ nie jest rozszczepialny;
    grupa podstawowa splotu $L$ nie jest produktem wolnym;
    $L$ jest rozmaitością Hakena o nieściśliwym brzegu.
\end{proposition}

\begin{proposition}
    \label{prop:knot_group_free}
    Splot o wolnej grupie podstawowej rangi $n$ to trywialny splot o $n$ ogniwach.
\end{proposition}

Twierdzenie Dehna z 1915 mówi, że jedynym węzłem,
którego grupą są liczby całkowite $\mathbb Z$, jest niewęzeł.
Wynik ten został później istotnie uogólniony.
Michael Kervaire pokazał w 1966 roku (w \cite{kervaire65}),
że abstrakcyjna grupa $G$ jest grupą podstawową dopełnienia węzła
(co najmniej pięciowymiarowego) wtedy i tylko wtedy, gdy:
\begin{enumerate}[leftmargin=*]
    \itemsep0em
    \item abelianizacją $G$ jest grupa liczb całkowitych, $\mathbb Z$,
    \item druga grupa homologii $G$ jest trywialna,
    \item $G$ jest skończenie prezentowana oraz
    \item $G$ stanowi domknięcie normalne pojedynczego generatora.
\end{enumerate}

Wiadomo, że w wymiarach $n \ge 3$ warunki te są co najmniej wystarczające,
jednakże problem  pełnej charakteryzacji w czwartym wymiarze jest otwarty.
Pierwsze dwa wynikają z dualności Alexandera,
dwa kolejne stanowią przeformułowanie prezentacji Wirtingera.

% Koniec sekcji Grupa węzła. Prezentacja Wirtingera