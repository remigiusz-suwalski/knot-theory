\section{Homologie}

% TODO: wstęp do homologii

Z powodu naszego ograniczonego rozumienia topologii algebraicznej oraz teorii kategorii wyłożony poniżej materiał jest tak naprawdę tylko przytoczeniem podstawowych definicji i~faktów.

Kompleks łańcuchowy to ciąg grup abelowych $C_n$ indeksowanych liczbami całkowitymi
wraz z~różniczkami, morfizmami $\partial_n \colon C_n \to C_{n-1}$ takimi,
że złożenie $\partial_{n-1} \circ \partial_n = 0$ jest trywialne.
Iloraz $\ker \partial_n / \operatorname{im} \partial_{n+1}$ nazywamy $n$-tą grupą homologii, $H_n$.

\subsection{Homologie Chowanowa}
\index{homologia!Chowanowa|(}
Viro napisał w 2004 piękną pracę \cite{viro04}, by objaśnić homologię Chowanowa używając tak mało algebry, jak to tylko możliwe.
Jest przyjazna dla początkujących, więc na niej opiera się reszta tej podsekcji.
Będziemy pracować ze stanami Kauffmana i obramowanymi węzłami, ponieważ autor sugeruje, że to bardziej naturalne.
Oryginalna praca Chowanowa to \cite{khovanov00}.

Niech $L$ będzie splotem, zaś $D$ jego diagramem.
Chowanow skonstruował rodzinę grup $\mathcal H^{i, j}(D)$ takich, że
\begin{equation}
    K(L, q) = \sum_{i, j} q^j (-1)^i \dim_\Q (\mathcal H^{i, j}(D) \otimes \Q),
\end{equation}
gdzie $K$ jest wersją wielomianu Jonesa.
Grupy $\mathcal H^{i, j}$ są u~niego grupami homologii pewnych kompleksów łańcuchowych.
Ich konstrukcja była przeładowana algebraicznymi szczegółami, później Bar-Natan \cite{barnatan02}, Viro podali jej warianty z~myślą o~topologach.
% Viro - Remarks on definition of Khovanov homology, arXiv

Homologię Chowanowa nazywa się kategoryfikacją wielomianu Jonesa.
Zanim zagłębimy się w szczegóły, rozpatrzmy prostszy przykład tego procesu.
Niech $X$ będzie przestrzenią topologiczną, wtedy charakterystykę Eulera oraz grupy homologii łączy zależność
\begin{equation}
    \chi(X) = \sum_{n = 0}^{\dim X} (-1)^n \operatorname{rk} H_n(X),
\end{equation}
a przy tym grupy homologii dostarczają więcej informacji, co więcej można o nich myśleć jako funktorach.

Kategoryfikacja wielomianu Jonesa polega na zastąpieniu jakoś jego współczynników przez ciąg grup abelowych.
Wzór o sumowaniu stanów przypomina ostatnią równość, brakuje tylko przedstawienia składników po prawej stronie jako alternująca suma rang grup.

Pewne drobne trudności techniczne skłoniły (być może) Chowanowa do zmiany zmiennej w~powiększonym wielomianie Jonesa: $q = -t^{1/2}$.
Dostał tak nowy wielomian $K$ o trzech własnościach:

% TODO

Stąd widać już, jakie grupy dobrać dla niewęzła:
\begin{equation}
    H^{i,j} = \begin{cases}
        \Z & \textrm{ jeśli } i = 0, j = \pm 1 \\
        0  & \textrm{ w przeciwnym razie}.
    \end{cases}
\end{equation}
Wtedy spełniona jest równość
\begin{equation}
    K(L) = \sum_{i, j} (-1)^i q^j \operatorname{rk} H^{i, j} (L).
\end{equation}
Pozostało powtórzyć to dla dowolnego splotu.

Wzór o sumowaniu stanów Kauffmana przybiera postać:
\begin{equation}
    K(L) = \sum_s (-1)^{(\writhe D - |s|)/2} q^{(3\writhe D - |s|)/2} (q+1/q)^{|sD|}.
\end{equation}

Każdy składnik z prawej strony przyczynia się do różnych jednomianów, zatem ma wpływ na różne grupy (które dopiero budujemy).
Stany nie są prawdziwym odpowiednikiem sympleksów z kategoryfikacji charakterystyki Eulera.
Najprostszym pomysłem, jak to naprawić, jest rozbicie ostatniej potęgi $q + 1/q$.
Zauważmy, że ma tyle czynników, ile wygładzenie diagramu ma składowych.
To motywuje definicję:

\begin{definition}[stan wzbogacony]
    Stan diagramu $D$ razem z przypisaniem znaku $+$ lub $-$ do każdego okręgu $sD$ nazywamy stanem wzbogaconym. 
\end{definition}

Dla ustalonego wzbogaconego stanu $S$ diagramu $D$ oznaczmy przez $\tau(S)$ sumę znaków przypisanych do okręgów\footnote{Oznaczenie wzięte z pracy Viro, żywimy nadzieję, że nikt nie weźmie $\tau$ za liczbę kolorowań z rodziału drugiego}.
Wtedy
\begin{equation}
    q^{(3 \writhe D - |s|)/2} (q + 1/q)^{|sD|} = \sum_{S/s} q^{(3 \writhe D - |s| + 2 \tau(S))/2},
\end{equation}
gdzie sumowanie odbywa się po wszystkich stanach $S$ wzbogacających stan $s$.
Niech
\begin{equation}
    j(S) := \frac 12 (3 \writhe D - |s| + 2 \tau(S)).
\end{equation}

Dobrnęliśmy do
\begin{equation}
    K(L, q) = \sum_S (-1)^{(\writhe D - |s|)/2} q^{j(S)},
\end{equation}
gdzie sumujemy po wszystkich wzbogaconych stanach diagramu $D$.

Zdefiniujmy jeszcze trochę obiektów.
Niech $C(D)$ oznacza wolną abelową grupę generowaną przez wzbogacone stany diagramu $D$, zaś $C^j(D)$ będzie jej podgrupą generowaną przez wzbogacone stany $S$ takie, że $j(S) = j$.
Czyni to $C(D)$ wolną grupą abelową z $\Z$-gradacją:
\begin{equation}
    C(D) = \bigoplus_{j \in \Z} C^j (D).
\end{equation}

Dla ustalonego stanu wzbogaconego $S$, niech $i(S) = (\writhe D - |s|)/2$.
Określmy jeszcze jedną podgrupę, $C^{i,j}(D) \le C^j(S)$ generowaną przez wzbudzone stany $S$, dla których $i(S) = i$.
Dostajemy wreszcie
\begin{equation}
    K(L, q) = \sum_{j = -\infty}^\infty q^j \sum_{i = -\infty}^\infty (-1)^i \operatorname{rk} C^{i, j}(D).
\end{equation}

Teraz ,,wystarczy'' zdefiniować funkcję $d$ i sprawdzić, że jest różniczką, to znaczy że $d^2 = 0$.
Tak też robi Viro, nam brakuje sił, by przybliżyć konstrukcję.
To już koniec -- różniczka pozwala przejść z grup $C^{i,j}$ do grup homologii.

\begin{definition}[obramowanie]
\index{obramowanie|see {węzeł obramowany}}%
\index{węzeł!obramowany}%
    Każde nieznikające normalne pole wektorowe na splocie nazywamy obramowaniem.
    Jeżeli wszystkie wektory są równoległe do płaszczyzny, na której leży diagram tego splotu, obramowanie nazywamy płaskim\footnote{Po angielsku ,,blackboard framing'', ale nie da się tego sensownie przetłumaczyć...}.
\end{definition}

Liczba samozaczepienia zorientowanego obramowanego splotu $L$ to indeks zaczepienia tego splotu ze sobą pchniętym w kierunku obramowania, jej wartość to dokładnie spin.

Viro zauważa, że powszechna definicja wielomianu Jonesa sprawia problem dla pustego splotu (którego nigdy wcześniej nie rozpatrywaliśmy).
Mamy:
\begin{equation}
    \jones_\varnothing = \frac{1}{-t^{1/2} - t^{-1/2}},    
\end{equation}
a to nie jest wielomian Laurenta jednej zmiennej.
\index{wielomian Jonesa!powiększony}%
Dlatego definiuje powiększony wielomian Jonesa:
\begin{equation}
    \widetilde{\jones_L}(t) = (-t^{1/2} - t^{-1/2}) \cdot \jones_L(t),    
\end{equation}

% Rasmussen podał nowy dowód hipotezy Milnora o plastrowym genusie węzłów torusowych, jest to pierwszy dowód który nie zależy od gauge theory.
% https://mathscinet.ams.org/mathscinet-getitem?mr=2729272

Innym narzędziem wykrywającym niewęzły jest homologia Chowanowa (opisana później),
jak pokazał Kronheimer z~Mrówką \cite{kronheimer11}.
\textbf{We prove that a knot is the unknot if and only if its reduced Khovanov cohomology has rank 1. The proof has two steps. We show first that there is a spectral sequence beginning with the reduced Khovanov cohomology and abutting to a knot homology defined using singular instantons. We then show that the latter homology is isomorphic to the instanton Floer homology of the sutured knot complement: an invariant that is already known to detect the unknot.}
Bar-Natan, topolog izraelski, napisał program liczący te homologie szybko \cite{barnatan07},
zapewne w~czasie $O(\exp(c \sqrt n))$, dla diagramu o~$n$ skrzyżowaniach.
Nie możemy liczyć na istotne przyspieszenie:
znalezienie przybliżenia wielomianu Jonesa jest problemem \#P-trudnym (\cite{kuperberg15}, \cite{vertigan05}),
a przy znanych homologiach -- wręcz trywialnym.
Patrz też \ref{prp:jones_at_roots_of_unity}.

\begin{definition}
    Niech $D$ będzie diagram splotu.
    Niezredukowanym nawiasem Kauffmana nazywamy wielomian
    \[
        [D] = (-A^2 - A^{-2}) \langle D \rangle = \sum_s A^{\sigma(s)} (-A^2 - A^{-2})^{|D_s|}.
    \]
\end{definition}

\begin{definition}
    Rozszerzonym stanem Kauffmana nazywamy parę uporządkowaną $S = (s, r)$,
    gdzie $s$ to stan Kauffmana,
    zaś $r$ to odwzorowanie $D_s \to \{\pm 1\}$ ze zbioru składowych diagramu.
\end{definition}

\begin{definition}
    Zbiór rozszerzonych stanów Kauffmana $\mathcal S(D)$:
    rozbija się na podzbiory indeksowane przez pary liczb całkowitych:
    $\mathcal S_{i, j} = \{S : \sigma(s) = i, \sigma(s) + 2 \tau(s) = j\}$.
\end{definition}

Tutaj lokalnie $\sigma(s) = |s|$, oraz $\tau(s) = |r^{-1}[1]| - |r^{-1}[-1]|$.

%%% Zauważmy, że $|D_s| \equiv r(s) \mod 2$. DLACZEGO?

\begin{definition}
    Niech $C_{i, j}$ będzie wolną grupą abelową generowaną przez zbiór $\mathcal S_{i, j}$.
    Ponumerujmy skrzyżowania diagramu $D$ liczbami $1, 2, \ldots, n$.
    Homologie Chowanowa splotu o~diagramie $D$ to homologie kompleksu
    \[
        C(D) = \bigoplus_{i, j \in \Z} C_{i, j}(D),
    \]
    z różniczkami $\partial_{i, j} \colon C_{i,j} \to C_{i-2, j}$ danymi wzorem
    \[
        \partial_{i, j}(S) = \sum_{S'} (-1)^{t(S, S')}  S'.
    \]
    Sumowanie odbywa się po tych $S' \in \mathcal S_{i-2, j}$,
    które różnią się od $S$ na dokładnie jednym skrzyżowaniu $v$:
    $S(v) = 1$, $S'(v) = -1$ oraz $\tau(S') = 1 + \tau (S)$.

    Liczba $t(S, S')$ to liczba skrzyżowań $D$ mniejszych od $v$, dla których $S$ przyjmuje wartość $-1$.
\end{definition}

\index{homologia!Chowanowa|)}

\subsection{Homologia Floera}
Do zrobienia...

% Koniec sekcji Homologie
