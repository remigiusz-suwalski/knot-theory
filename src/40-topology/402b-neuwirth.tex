
\subsection{Węzły rozwłóknione}
\index{węzeł!włóknisty|see {węzeł rozwłókniony}}%
\index{węzeł!rozwłókniony|(}%
Wspomnijmy jeszcze krótko o~specjalnym rodzaju węzłów i splotów.

% DICTIONARY;fibered;rozwłókniony, włóknisty;-
\begin{definition}
    Niech $L \subseteq S^3$ będzie splotem.
    Jeśli istnieje rodzina $F_t$ powierzchni Seiferta dla splotu $K$ sparametryzowana przez $t \in S^1$ taka, że $F_t \cap F_s = K$ dla $t \neq s$, to splot $K$ nazywamy rozwłóknionym albo włóknistym.
\end{definition}

\index{splot!Neuwirtha}%
Dawniej nazywano je splotami Neuwirtha, gdyż ten pokazał w~swojej pracy dyplomowej z~1959 roku, że można je scharakteryzować jako sploty, których komutant grupy podstawowej jest skończenie generowany, lub równoważnie, wolny.

\begin{example}
    Niewęzeł, trójlistnik $3_1$, ósemka $4_1$, $5_{1}$, $6_{2}$, $6_{3}$, $7_{1}$, $7_{6}$, $7_{7}$, $8_{2}$, $8_{5}$, $8_{7}$, $8_{9}$, $8_{10}$, $8_{12}$, $8_{16}$..$8_{21}$, splot Hopfa oraz wszystkie węzły torusowe są rozwłóknione.
\end{example}

(Jeśli węzeł pierwszy o co najwyżej ośmiu skrzyżowaniach nie został wymieniony w tym przykładzie, to nie jest rozwłókniony).
Rozkład liczby węzłów rozwłóknionych wśród węzłów pierwszych wygląda następująco:
\begin{itemize}
\item 9 skrzyżowań -- 23 węzły,
\item 10 skrzyżowań -- 74 węzły,
\item 11 skrzyżowań -- 256 węzłów,
\item 12 skrzyżowań -- 873 węzły.
\end{itemize}
% query = 'fibered == True' + grep + nl

Lwia część analizy węzłów o 12 skrzyżowaniach została wykonana przez Stojmenowa i~Hirasawę, jak podaje baza danych KnotInfo \cite{knotinfo22}.
% źródło: https://knotinfo.math.indiana.edu/descriptions/fibered.html
\index[persons]{Hirasawa, Mikami}%
\index[persons]{Stojmenow, Aleksander}%

\begin{proposition}
\index{wielomian!Alexandera}%
    Pierwszy i~ostatni współczynnik wielomianu Alexandera węzła rozwłóknionego to $\pm 1$.
\end{proposition}

Kryterium to jest wystarczające dla węzłów pierwszych o co najwyżej 10 skrzyżowaniach oraz alternujących, ale znany jest przykład niewłóknistego węzła o 21 skrzyżowaniach, którego wielomian Alexandera ma postać $t^4 - t^3 + t^2 - t +1$.

\begin{proposition}
\index{węzeł!skręcony}%
    Niech $K$ będzie węzłem skręconym z $n$ półskrętami.
    Wtedy jego wielomianem Alexandera jest
    \begin{equation}
        \alexander_n(t) = n \cdot \left(t + \frac 1 t \right) - (2n+1),
    \end{equation}
    więc węzeł $K$ nie jest rozwłókniony, chyba że $n = 1$.
\end{proposition}

\begin{corollary}
    $2$-skręcony węzeł $6_1$ (węzeł dokerski) nie jest rozwłókniony.
\end{corollary}

Rolfsen \cite[s. 326]{rolfsen76} podaje jako ćwiczenie w swojej książce:

\begin{proposition}
    Rodzina węzłów rozwłóknionych jest zamknięta na branie sum spójnych.
\end{proposition}

\index{węzeł!rozwłókniony|)}%

% koniec podsekcji Węzły rozwłóknione

