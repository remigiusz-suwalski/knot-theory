
\subsection{Węzły rozwłóknione}
\index{węzeł!rozwłókniony|(}
Wspomnijmy jeszcze krótko o~specjalnym rodzaju węzłów i splotów.

% DICTIONARY;fibered;rozwłókniony, włóknisty;-
\begin{definition}
    Niech $L \subseteq S^3$ będzie splotem.
    Jeśli istnieje rodzina $F_t$ powierzchni Seiferta dla splotu $K$ sparametryzowana przez $t \in S^1$ taka, że $F_t \cap F_s = K$ dla $t \neq s$, to splot $K$ nazywamy rozwłóknionym albo włóknistym
\end{definition}

\index{splot!Neuwirtha}
Dawniej nazywano je splotami Neuwirtha, gdyż ten pokazał w~swojej pracy dyplomowej z~1959 roku, że można je scharakteryzować jako sploty, których komutant grupy podstawowej jest skończenie generowany, lub równoważnie, wolny.

\begin{example}
    Niewęzeł, trójlistnik $3_1$, ósemka $4_1$, splot Hopfa oraz wszystkie węzły torusowe są rozwłóknione.
\end{example}

\begin{proposition}
\index{wielomian!Alexandera}
    Pierwszy i~ostatni współczynnik wielomianu Alexandera węzła rozwłóknionego to $\pm 1$.
\end{proposition}

Kryterium to jest wystarczające dla węzłów pierwszych o co najwyżej 10 skrzyżowaniach oraz alternujących, ale znany jest przykład niewłóknistego węzła o 21 skrzyżowaniach, którego wielomian Alexandera ma postać $t^4 - t^3 + t^2 - t +1$.

\begin{proposition}
\index{węzeł!skręcony}%
    Niech $K$ będzie węzłem skręconym z $n$ półskrętami.
    Wtedy jego wielomianem Alexandera jest
    \begin{equation}
        \alexander_n(t) = n \cdot \left(t + \frac 1 t\right)  - (2n+1),
    \end{equation}
    więc nie jest rozwłókniony, chyba że $n = 1$.
\end{proposition}

\begin{corollary}
    Skręcony węzeł $6_2$ nie jest rozwłókniony.
\end{corollary}

Rolfsen \cite[s. 326]{rolfsen76} podaje jako ćwiczenie w swojej książce:

\begin{proposition}
    Rodzina węzłów rozwłóknionych jest zamknięta na branie sum spójnych.
\end{proposition}

\index{węzeł!rozwłókniony|)}

% koniec podsekcji Węzły rozwłóknione

