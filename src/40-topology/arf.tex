\section{Niezmiennik Cahita Arfa} % (fold)

Niezmiennik Arfa dla węzłów można zdefiniować na kilka sposobów, z~których żaden jest istotnie lepszy od pozostałych.
Przyjmuje on dwie wartości: $0$ i $1$.
Pierwszy był pomysł Robertello \cite{robertello65}:

\begin{proposition}[Robertello, 1965]
    Niech $K$ będzie węzłem, zaś
    \begin{equation}
        \alexander_K(t)=c_{0}+c_{1}t+\cdots +c_{n}t^{n}+\cdots +c_{0}t^{2n}
    \end{equation}
    jego wielomianem Alexandera.
    Wtedy niezmiennik Arfa to $c_{n-1}+c_{n-3}+\cdots +c_{r} \mod 2$, gdzie $r = 0$ dla nieparzystych $n$, $r = 1$ w~przeciwnym razie.
\end{proposition}

Nieco później Murasugi \cite{murasugi69} zauważył, że warunek można uprościć:

\begin{proposition}[Murasugi, 1969]
    \label{arf_murasugi}
    Niech $K$ będzie węzłem.
    Wtedy $\operatorname{Arf} K = 0$ wtedy i~tylko wtedy, gdy $\alexander_K(-1) \equiv \pm 1 \mod 8$.
\end{proposition}

Louis Kauffman zaproponował inne podejście w~1983 roku, z wykorzystaniem diagramów.
Dwa węzły nazwiemy równoważnymi przez przejścia, jeśli są związane skończenie wieloma ,,przejściami'' \cite[s. 143]{kauffman83}:
\begin{comment}
\[
    \begin{tikzpicture}[baseline=-0.65ex,scale=0.35]
    \begin{knot}[clip width=7]
        \strand[-latex, thick] (-2.5,-1.0) to (2.5,-1.0);
        \strand[-latex, thick] (2.5,1.0) to (-2.5,1.0);
        \strand[-latex, thick] (-1.0,-2.5) to (-1.0,2.5);
        \strand[-latex, thick] (1.0,2.5) to (1.0,-2.5);
    \end{knot}
    \end{tikzpicture}
    \cong
    \begin{tikzpicture}[baseline=-0.65ex,scale=0.35]
    \begin{knot}[clip width=7, flip crossing/.list={1,2,3,4}]
        \strand[-latex, thick] (-2.5,-1.0) to (2.5,-1.0);
        \strand[-latex, thick] (2.5,1.0) to (-2.5,1.0);
        \strand[-latex, thick] (-1.0,-2.5) to (-1.0,2.5);
        \strand[-latex, thick] (1.0,2.5) to (1.0,-2.5);
    \end{knot}
    \end{tikzpicture}
    \quad\mbox{albo}\quad
    \begin{tikzpicture}[baseline=-0.65ex,scale=0.35]
    \begin{knot}[clip width=7]
        \strand[-latex, thick] (-2.5,-1.0) to (2.5,-1.0);
        \strand[-latex, thick] (2.5,1.0) to (-2.5,1.0);
        \strand[latex-, thick] (-1.0,-2.5) to (-1.0,2.5);
        \strand[latex-, thick] (1.0,2.5) to (1.0,-2.5);
    \end{knot}
    \end{tikzpicture}
    \cong
    \begin{tikzpicture}[baseline=-0.65ex,scale=0.35]
    \begin{knot}[clip width=7, flip crossing/.list={1,2,3,4}]
        \strand[-latex, thick] (-2.5,-1.0) to (2.5,-1.0);
        \strand[-latex, thick] (2.5,1.0) to (-2.5,1.0);
        \strand[latex-, thick] (-1.0,-2.5) to (-1.0,2.5);
        \strand[latex-, thick] (1.0,2.5) to (1.0,-2.5);
    \end{knot}
    \end{tikzpicture}
\]
\end{comment}

\begin{definition}[Kauffman, 1983]
    \index{niezmiennik!Arfa}
    Każdy węzeł $K$ jest równoważny przez przejścia albo z niewęzłem, albo z trójlistnikiem.
    W pierwszym przypadku mówimy, że niezmiennik Arfa znika: $\operatorname{Arf} K = 0$, w drugim, że nie: $\operatorname{Arf} K = 1$.
\end{definition}

Wreszcie Jones zauważył \cite{jones85}, że wielomian $\jones$ także pozwala na określenie niezmiennika Arfa, dzięki zespolonym algebrom Clifforda oraz pracy \cite{lannes85}.
Jest to jedyna definicja, którą łatwo rozszerzyć do splotów.

\begin{proposition}[Jones, 1985]
    \label{arf_jones}
    $\operatorname{Arf}(K) = \jones_K(i)$.
\end{proposition}

\begin{corollary}
    Niezmiennik Arfa jest $\shrap$-addytywny.
\end{corollary}

\begin{proof}
    Wynika to z faktu \ref{arf_jones} oraz \ref{prp:jones_multiplicative_2}, ale istnieją też bardziej bezpośrednie dowody.
\end{proof}

Na zakończenie zostawiliśmy definicję zakorzenioną w topologii algebraicznej.

\begin{proposition}
    Niech $(v_{ij})$ będzie macierzą Seiferta powstałą z~krzywych genusu $g$, które reprezentują bazę pierwszej grupy homologii powierzchni.
    To oznacza, że macierz $V$ wymiaru $2g \times 2g$ ma następującą własność: różnica $V - V^t$ jest symplektyczna.
    Niezmiennik Arfa to
    \begin{equation}
        \sum^g_{i=1}v_{2i-1,2i-1}v_{2i,2i} \pmod 2.
    \end{equation}
\end{proposition}

O niezmienniku Arfa usłyszymy jeszcze poznając węzły plastrowe, w~sekcji \ref{sec:slice}.

% https://macsphere.mcmaster.ca/handle/11375/25082 geometryczna interpretacja

% OEIS http://oeis.org/A131433, ...1434
% Koniec sekcji Niezmiennik Cahita Arfa
