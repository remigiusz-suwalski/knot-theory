
\subsubsection{Genus kanoniczny, genus wolny}
% DICTIONARY;free;wolny;genus
% DICTIONARY;canonical;kanoniczny;genus
% DICTIONARY;genus;genus;-
Czy w definicji genusu można ograniczyć się do tych powierzchni Seiferta, które pochodzą od algorytmu Seiferta?
\index{algorytm!Seiferta}%
Niestety, poza pewnymi wyjątkami, nie.
Zanim przekonamy się, dlaczego tak jest, zdefiniujmy jeszcze dwa niezmienniki.

\begin{definition}[genus kanoniczny]
\index{genus!kanoniczny}%
    Niech $K$ będzie węzłem.
    Najmniejszy z genusów powierzchni Seiferta węzła $K$, które pochodzą z~algorytmu Seiferta, nazywamy genusem klasycznym i~oznaczamy symbolem $\operatorname{g_c} K$ lub krótko $g_c$.
\end{definition}

Stojmenow \cite{stoimenow08} opisał diagramy węzłów o~kanonicznym genusie równym 2.
\index[persons]{Stojmenow, Aleksander}%
Część z~jego wyników przenosi się na genus 3.
Jak sam pisze, sklasyfikowane wcześniej węzły o~genusie (kanonicznym) 1 okazały się być zbyt wąską klasą.

Pod koniec lat pięćdziesiątych Crowell i~Murasugi niezależnie zauważyli, że algorytm Seiferta zastosowany do alternującego diagramu zawsze daje powierzchnię o~minimalnej powierzchni.
\index[persons]{Crowell, Richard}%
\index[persons]{Murasugi, Kunio}%
Ich kombinatoryczne uzasadnienie było dość zawiłe, elementarny dowód podał Gabai w \cite{gabai86}.
\index[persons]{Gabai, David}%

Dubel trójlistnika ma genus równy $1$, ale algorytm Seiferta zastosowany wobec węzła produkuje powierzchnie o genusie co najmniej $3$, jak przewiduje ograniczenie znalezione przez Mortona w \cite[twierdzenie 2]{morton86}:
\index[persons]{Morton, Hugh}%

\begin{proposition}
    Niech $P(v, z)$ będzie wersją wielomianu HOMFLY spełniającą zależność
    \begin{equation}
        \frac 1v P_+ - vP_- = zP_0.
    \end{equation}
    Wtedy $M = \max \deg_z P(v, z) \le 2g_c$.
\index{nierówność Mortona}%
\end{proposition}

Wspólnymi siłami udało się pokazać, że dla wielu klas węzłów nierówność Mortona jest tak naprawdę równością.
Chronologicznie, były to: węzły alternujące (niezależnie Crowell \cite{crowellrichard59}, Murasugi \cite{murasugi58} około 1959),
\index[persons]{Crowell, Richard}%
\index[persons]{Murasugi, Kunio}%
\index{węzeł!alternujący}%
jednorodne\footnote{Uogólnienie węzłów alternujących.} (Cromwell \cite{cromwell89} w 1989),
\index{węzeł!jednorodny}%
\index[persons]{Cromwell, Peter}%
whiteheadowskich dubli węzłów dwumostowych (Nakamura \cite{nakamura06}, Tripp \cite{tripp02} na początku XXI wieku)
\index{dubel Whiteheada}%
\index{węzeł!dwumostowy}%
\index[persons]{Nakamura, Takuji}%
\index[persons]{Tripp, James}%
czy wreszcie dubli alternujących precli (Brittenham, Jensen \cite{brittenham06}).
\index[persons]{Brittenham, Mark}%
\index[persons]{Jensen, Jacqueline}%
\index{precel}%
Stojmenow \cite{stoimenow02} sprawdził, że nierówność Mortona jest równością dla wszystkich węzłów do 12 skrzyżowań i znalazł przykłady węzłów, dla których jest ostra: $15_{100154}$, $15_{167945}$.
\index[persons]{Stojmenow, Aleksander}%
% wiem to wszystko z <Families of knots for which Morton’s inequality is strict>:
% Morton’s inequality ... has since been shown to be an equality for many classes of knots.
% These include all of the knots having 12 or fewer crossings [St2], all alternating knots [Cr],[Mu], and, more generally, all homogeneous knots [Cm], and the Whitehead doubles of 2-bridge knots [Na],[Tr] and pretzel knots [BJ].

\begin{definition}[genus wolny]
\index{ciało z rączkami}%
\index{genus!wolny}%
    Niech $K$ będzie węzłem.
    Minimalny genus spośród powierzchni Seiferta węzła $K$, których dopełnienie w 3-sferze jest ciałem z rączkami, nazywamy genusem wolnym i~oznaczamy $g_f$.
\end{definition}

Dopełnienie powierzchni Seiferta jest zawsze ciałem z rączkami, więc mamy oczywiste nierówności
\begin{equation}
    g \le g_f \le g_c.
\end{equation}
Jak duża może być różnica między kolejnymi genusami?
Już Kirby \cite[problem 1.20a]{kirby78} chciał znać oszacowania różnicy $g_f - g$.
Morton \cite{morton86} pokazał, że genus pewnych węzłów nie jest realizowany przez żaden diagram do którego stosuje się algorytm Seiferta, choćby $10_{165}$.
\index[persons]{Morton, Hugh}%
Moriah, matematyk izraelski, rozwiązał problem Kirby'ego dekadę później:
\index[persons]{Moriah, Yoav}%

\begin{proposition}
    Niech $K$ będzie węzłem, $D_k(K)$ jego dublem Whiteheada z $k \neq 0$ skręceniami, zaś $B_n(K)$ to $n$-krotne nakrycie cykliczne sfery $S^3$ rozgałęzione nad węzłem $K$.
    Jeżeli ranga pierwszej grupy homologii $B_{|4k+1|}(K)$ wynosi $r$, to
    \begin{equation}
        g_f(D_k(K)) \ge \frac {2r-1} {|8k+2|}.
    \end{equation}
\end{proposition}

\begin{proof}
    Praca \cite{moriah87}.
    Dowód opiera się na chirurgii węzłów i splotów w sferze $S^3$.
\end{proof}

\begin{corollary}
    Niech $K$ bedzie sumą spójną $n$ trójlistników, połóżmy $k = -1$.
    Wtedy pierwsza grupa homologii ma rangę $r = 2n$ i~genus wolny jest nieograniczony
    \begin{equation}
        g_f(D_{-1}(3_1^n)) \ge \frac {4n-1} {6},
    \end{equation}
    podczas gdy zwykły genus to $g(D_{-1}(3_1^n)) = 1$.
\end{corollary}

Kawauchi \cite{kawauchi94} zbadał węzeł $K_m$, sumę spójną $m$ kopii skręconego whiteheadowskiego dubla trójlistnika, i policzył, że różnica $g_c(K_m) - g(K_m)$ wynosi $2m$.
\index[persons]{Kawauchi, Akio}%
Wreszcie Kobayashi oraz Kobayashi \cite{kobayashi96} wskazali nieskończoną rodzinę węzłów nieograniczonego genusu, dla której
\index[persons]{Kobayashi, Masako}%
\index[persons]{Kobayashi, Tsuyoshi}%
\begin{equation}
    g_c(K) = \frac 32 g_f(K) = 2g(K).
\end{equation}
% znam ich ze Stojmenow - Knots of (canonical) genus two

