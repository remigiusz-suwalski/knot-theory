
\subsection{Sploty o~dwóch mostach}
\label{sub:twobridge}%
\index{węzeł!wymierny|see {węzeł dwumostowy}}%
\index{węzeł!dwumostowy|(}%
Zajmiemy się teraz związkiem supłów z liczbą mostową.
Wiemy, że węzeł trywialny jest jednomostowy, następne w hierarchii są sploty dwumostowe.
Nazywa się je także wymiernymi, po angielsku czasami \emph{4-plats}.
Jako pierwszy studiował je Bankwitz z~Schumannem w~1934 roku.
% Kawauchi: as 4-plat presentations, which is just Conway's normal form.
Mają co najwyżej dwie składowe i~są odwracalne \cite[s. 211]{burde14}.
Patrz też \cite[s. 21-26]{kawauchi96}.

\begin{proposition}
    Sploty dwumostowe są pierwsze.
\end{proposition}

\begin{proof}
    Prosty wniosek z~tego, że liczba mostowa prawie jest addytywna (fakt \ref{prp:bridge_additive}).
\end{proof}

\begin{corollary}
    Pierwsze węzły dwu- lub trzymostowe są albo torusowe, albo hiperboliczne.
\end{corollary}

\begin{proof}
    Kawauchi \cite[s. 130]{kawauchi96} wnioskuje to z twierdzenia, którego nie znamy.
\end{proof}

%\todo[inline]{Murasugi Theorem 9.3.3 (138) lub Janiak-Osajca, Pogoda (34).}
% Aus der unten stehenden Klassifikation ergibt sich, dass man jede Verschlingung mit 2 Brücken wie im Bild rechts darstellen kann, wobei {\displaystyle a_{i}\in \mathbb {Z} } a_{i}\in \mathbb{Z }  die Anzahl der Halbtwists in der jeweiligen Box bezeichnet und für gerade bzw. ungerade {\displaystyle i} i~positive {\displaystyle a_{i}} a_{i} links- bzw. rechtshändigen Halbtwists entsprechen.
% Diese Darstellung wird als Conway-Normalform bezeichnet.
% Man kann stets erreichen, dass alle {\displaystyle a_{i}} a_{i} dasselbe Vorzeichen haben.[1] Insbesondere gibt die Conway-Normalform dann ein alternierendes Knotendiagramm.[2]
%Insbesondere ist ein 2-Brücken-Knoten genau dann amphichiral, wenn {\displaystyle q^{2}\equiv -1\ mod\ p} q^{2}\equiv -1\ mod\ p ist.

\begin{proposition}
    Sploty z~dwoma mostami to dokładnie sploty typu $D(T)$ dla pewnego supła wymiernego $T$.
\end{proposition}

Dowód tego stwierdzenia znaleźć można na przykład w książce \cite{murasugi96}, strony 183-187.
Wynika z niego, że każdy splot dwumostowy można przedstawić następującym diagramem:
\input{50-families/tangle_05}

Oto reguła, zgodnie z~którą wybieramy znaki liczb $a_i$:
jeśli $i$ jest nieparzyste, prawy skręt jest dodatni, jeśli parzyste -- lewy jest dodatni.
Sam diagram oznaczamy $C(a_1, \ldots, a_{2k+1})$ i~nazywamy postacią normalną Conwaya.

% Conway Normal Form: kawauchi96, strona 24

\begin{proposition}
    % Murasugi proposition 9.3.2
    Sploty dwumostowe są alternujące.
\end{proposition}

\begin{proof}
    Goodrick w~\cite{goodrick72} podał diagramatyczny dowód, gdzie ciąg ruchów zmienia diagram splotu dwumostowego w~alternujący.
    Wynika to też z faktu \ref{prp:continued_fractions}.
    % Burde, Zieschang 2013, strona 217, nazywają to twierdzeniem Bankwitza-Schumanna.
\end{proof}

Przez analogię do supłów, definiujemy ułamek łańcuchowy
\begin{equation}
    C(a_1, \ldots, a_{2k+1}) \mapsto a_1 + \frac{1}{a_2 + 1/\ldots} = \frac \alpha \beta.
\end{equation}

\begin{tobedone}
\index{postać normalna!Conwaya}%
\index{postać normalna!Schuberta}%
    To jest postać normalna Conwaya, ale mamy jeszcze postać Schuberta - \cite[s. 21]{kawauchi96}.
\end{tobedone}

Zauważmy, że wartość bezwzględna ułamka $\alpha/\beta$ zawsze przekracza $1$ i~odwrotnie, każdy taki ułamek pochodzi od pewnego węzła dwumostowego.
Parę względnie pierwszych liczb $(\alpha, \beta)$ nazywamy typem węzła dwumostowego.

\begin{proposition}
    \label{prp:tangle_equivalence}
    Dwumostowe sploty typów $(\alpha, \beta)$ oraz $(\alpha', \beta')$ są równoważne wtedy i tylko wtedy, gdy spełnione są warunki:
    \begin{equation}
        \begin{cases}
            \alpha = \alpha' \\
            \beta^{\pm 1} \equiv \beta' \pmod {2 \alpha}
        \end{cases}
    \end{equation}
    Gdyby rozpatrywać niezorientowane sploty, drugie przystawanie upraszcza się: wystarczy, że będzie zachodzić modulo $\alpha$.
\end{proposition}

\begin{proof}
    Słabsza wersja twierdzenia bierze się z klasyfikacji przestrzeni soczewkowych oraz tego, że podwójnie rozgałęziona przestrzeń nakryciowa dwumostowego splotu typu $(\alpha, \beta)$ to przestrzeń soczewkowa typu $(\alpha, \beta)$.
    % two-fold branched cyclic spaces?
    (Nie definiujemy w~tej książce, czym są przestrzenie soczewkowe).
\index{przestrzeń!soczewkowa}%
    Burde, Zieschang \cite[s. 212]{burde14} wspominają tu prace Reidemeistera \cite{reidemeisterXX}, Brody'ego \cite{brodyXX} oraz Turajewa \cite{turaevXX}.
    % TODO: Reidemeister, K., 1935: Homotopieringe und Linsenräume. Abh. Math. Sem. Univ. Hamburg, 11 (1935), 102–109
    % TODO: Brody, E. J., 1960: The topological classification of lens spaces. Ann. of Math., 71 (1960), 163–184

    Dowód mocnej wersji znajduje się u Schuberta \cite{schubert56} albo Turajewa \cite{turaevXX}, a zapewne także Murasugiego \cite[s. ?]{murasugi96}.
    % TODO: napisałem ?, bo nie wiem która strona i nie chce mi się dziś sprawdzać
    % TODO: schubert56 - to ma prawie 40 stron!
    % TODO: nie mam bibliografii do zieschanga-2013, a w zieschangu-2003 nie ma turaeva, więc nie wiem, co tu tak naprawdę jest cytowane :D
\end{proof}

\begin{proposition}
    Dwumostowy splot typu $(\alpha, \beta)$ jest achiralny dokładnie wtedy i tylko wtedy, gdy
    \begin{equation}
        \beta^2 \equiv -1 \mod \alpha.
    \end{equation}
\end{proposition}

\begin{proof}
    Wynika to z tego, że lustrem splotu typu $(\alpha, \beta)$ jest splot typu $(\alpha, -\beta)$ oraz faktu \ref{prp:tangle_equivalence}.
    Explicite pisze o tym Kawauchi \cite[s. 24]{kawauchi96}.
\end{proof}

\begin{proposition}
    Niech $b$ będzie dowolną liczbą całkowitą.
    Wtedy następujące sploty są tego samego typu:
    \begin{align}
        N(T(a_1, a_2, \ldots, a_{2k+1})) & \approx N(T(a_1, a_2, \ldots, a_{2k+1}, b, 0)) \\
                                         & \approx D(T(-a_1, -a_2, \ldots, -a_{2k+1}, b)) \\
                                         & \approx C(a_1, a_2, \ldots, a_{2k}-1, 1). \\
        N(T(a_1, a_2, \ldots, a_{2k}))   & \approx D(T(-a_1, -a_2, \ldots, -a_{2k}, b)) \\
                                         & \approx C(a_1, a_2, \ldots, a_{2k}-1, 1). \\
        D(T(a_1, a_2, \ldots, a_{2k+1})) & \approx D(T(a_1, a_2, \ldots, a_{2k}, 0)) \\
                                         & \approx C(1, a_1-1, a_2, \ldots, a_{2k}). \\
        D(T(a_1, a_2, \ldots, a_{2k}))   & \approx D(T(a_1, a_2, \ldots, a_{2k-1}, 0)) \\
                                         & \approx C(a_1, a_2, \ldots, a_{2k-1}).
    \end{align}
\end{proposition}

\begin{proof}
    \cite[fakt 9.3.4]{murasugi96}
\end{proof}

\begin{proposition}
    Niech $L$ będzie dwumostowym splotem typu $(\alpha, \beta)$.
    Wtedy $\det L = \alpha$.
\end{proposition}

Wynika stąd, że wyznacznik nie wystarcza do odróżniania splotów dwumostowych.

\begin{proof}
    % Chcąc oszczędzić niektórym Czytelnikom cierpień odsyłamy po prostu do \cite{schubert56}.
    \url{https://math.stackexchange.com/questions/3327846/}.
\end{proof}

Niech $A, B$ będą supłami.
Wiemy, że suma $A+B$ nie musi być supłem, zaś $D(A+B)$ niekoniecznie jest splotem dwumostowym.
Pomimo to, splot $N(A+B)$ jest dwumostowy, potrafimy nawet powiedzieć, jaki ma wyznacznik:

\begin{proposition}
    % Theorem 9.3.5 Murasugi

    Niech $A, B$ będą supłami, którym odpowiadają skrócone ułamki $p/q$ oraz $r/s$.
    Wtedy splot $L = N(A+B)$ jest dwumostowy, typu $(\alpha, \beta)$ i ma wyznacznik $\alpha = |ps + qr|$.
\end{proposition}

Murasugi (twierdzenie 9.3.5) twierdzi, że dowód znajduje się u Ernsta, Sumnersa \cite{ernst90}.
\index[persons]{Ernst, Claus}%
\index[persons]{Sumners, De Witt}%

\begin{proposition}
    Rozpatrzmy węzeł dwumostowy typu $(\alpha, \beta)$, gdzie $0 < \beta < \alpha$ i~$\beta$ jest nieparzyste.
    Niech $r_k$ będzie resztą z~dzielenia $k\beta$ przez $2\alpha$ leżącą w~przedziale $(-\alpha, \alpha)$ dla $k = 0, 1, \ldots, \alpha - 1$.
    Różnica między ilością dodatnich reszt i~ujemnych reszt to sygnatura węzła.
\end{proposition}

Wygląda na to, że jedynym niewyznaczonym do końca klasycznym niezmiennikiem jest liczba gordyjska.

\index{węzeł!dwumostowy|)}%

% Koniec podsekcji Sploty o~dwóch mostach

