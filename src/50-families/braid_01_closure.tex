
\subsection{Warkocze a sploty}
% DICTIONARY;closure of ...;domknięcie ...;warkocz
Każdy warkocz można domknąć do węzła, łącząc ze sobą punkty $(x_i, 1)$ oraz $(x_i, 0)$
łamanymi, których rzuty do płaszczyzny diagramu nie przecinają się.
\index{warkocz!domknięcie warkocza}%
Nie wiadomo, kto wymyślił operację domykania warkoczy, ale była ona z pewnością znana Alexanderowi: rozpatrywano je jeszcze przed samymi warkoczami.
\index[persons]{Alexander, James}%
% TODO: rysunek w TikZ, jak się domyka.

Niech $b \in B_n$ będzie słowem zapisanym na standardowych generatorach.
Oznaczmy przez $b_+$, $b_-$ nieznakowaną sumę dodatnich, ujemnych wykładników.
Jeśli $b_+ - 3b_- \ge n$, to domknięcie warkocza $b$ nie jest achiralne (twierdzenie 5 z~\cite{jones85}).
\index{węzeł!achiralny}%

\begin{theorem}[Alexander, 1923]
\label{thm:alexander}
     Każdy splot powstaje przez domknięcie pewnego warkocza.
     \index{twierdzenie!Alexandera}
\end{theorem}

\begin{proof}[Niedowód]
    W kolejności chronologicznej:
    najpierw Alexander \cite{alexander23},
\index[persons]{Alexander, James}%
    a po blisko połowie wieku Morton \cite{mortonhr86},
\index[persons]{Morton, Hugh}%
    Yamada \cite{yamada87} (co daje łatwy do zaimplementowania program komputerowy)
\index[persons]{Yamada, Shuji}%
    i~Vogel \cite{vogel90} (ulepszający algorym Yamady).
\index[persons]{Vogel, Pierre}%
\end{proof}

Manturow \cite{manturov02} pokazał, że od warkocza można wymagać kwazitoryczności (warkocz nazywamy torycznym, jeżeli jest postaci $(\sigma_1 \ldots \sigma_{p-1})^q$ oraz kwazitorycznym, jeżeli powstaje przez odwrócenie niektórych skrzyżowań z~warkocza torycznego).
\index[persons]{Manturow, Wasilij}%
\index{warkocz!toryczny}%

\begin{theorem}[Markow, 1936]
\index{twierdzenie!Markowa}%
\label{markov_theorem}
    Dwa domknięte warkocze są równoważne jako sploty wtedy i~tylko wtedy,
    gdy jeden powstaje z~drugiego przez ciąg
    sprzężeń: $z_1 \mapsto z_2 z_1 z_2^{-1}$ oraz procesów Markowa,
    które zastępują $n$-warkocz $\beta$ przez $(n+1)$-warkocz $\beta\sigma_n^{\pm 1}$.
\end{theorem}

\begin{proof}
\index[persons]{Birman, Joan}%
\index[persons]{Menasco, William}%
\index[persons]{Morton, Hugh}%
\index[persons]{Traczyk, Paweł}%
    Kompletny i~godny naśladowania dowód znajduje się w~trudno dostępnej książce \cite{birman74} Birman, więc warto sprawdzić inne, opublikowane później materiały:
    Morton opisał w~\cite{mortonhr86} przepiękną, a~przy tym elementarną ideę ,,nitkowania'',
    potem Traczyk podał w~\cite{traczyk98} czysto kombinatoryczne, dwuwymiarowe uzasadnienie oparte o~okręgi Seiferta,
    wreszcie mamy też artykuł \cite{birman02} napisany przez Birman i~Menasco.
\end{proof}

Pierwsze sformułowanie twierdzenia pochodzące od Markowa \cite{markov36} korzystało z trzech ruchów, jeden z~nich stanowił uogólnienie II ruchu Reidemeistera.
Trzy lata później Weinberg zauważył w~\cite{weinberg39}, że wystarczą dwa ruchy.
\index[persons]{Weinberg, Noah}%
% Weinberg = Ной Вайнберг: http://www.mathsoc.spb.ru/history/Odynec_2020.pdf
Lambropoulou, Rourke przedstawili w~\cite{lambropoulou97} wersję twierdzenia wymagającą tylko jednego ruchu.
\index[persons]{Lambropoulou, Sofia}%
\index[persons]{Rourke, Colin}%

Historia twierdzenia Markowa jest raczej dramatyczna: Markow przedstawił swój dowód ustnie, ale nigdy go nie opublikował, zostawiając to zadanie swojemu uczniowi, Weinbergowi.
Ten jednak został zabity podczas wojny, wkrótce po opublikowaniu pierwszej pracy na temat teorii węzłów i na dokładny dowód trzeba było czekać do publikacji Birman \cite{birman74} blisko 40 lat.
\index[persons]{Birman, Joan}%

Twierdzenie~\ref{markov_theorem} mówi nam, że teoria węzłów bada klasy równoważności w~grupie warkoczy.
Zarówno problem słowa (czy dwa słowa w~grupie przedstawiają ten sam jej element?) jak i~problem sprzężoności (czy dwa słowa w~grupie są sprzężone?) są rozwiązane, ale nadal nie mamy algorytmu, który mówiłby, czy dwa słowa w~grupie są równoważne w~sensie Markowa.
Cały czas chodzi o słowa w grupie warkoczy, oczywiście.
% TODO: kto to pokazał? wg Kawauchiego około strony 18, Murasugi w 1982 ale widziałem gdzieś informację, że Garside był pierwszy.
% a może Jones 1985?

