\subsection{Węzły skręcone}
\begin{definition}
    \index{węzeł!skręcony}
    \label{def:twist_knot}
    Węzeł powstały przez $n$-krotne półskręcanie domkniętej pętli oraz splecienie końców nazywamy węzłem skręconym.
\end{definition}

Węzły skręcone to dokładnie towarzyszące niewęzłowi w~węzłach satelitarnych, tak zwane whiteheadowskie duble niewęzła.
Wszystkie są odwracalne (ale tylko niewęzeł oraz ósemka są amfichiralne) i~mają liczbę gordyjską $1$, ponieważ wystarczy rozwiązać skrzyżowanie, które plotło końce.
Każdy jest $2$-mostowy i~posiada zerową sygnaturę.
Dalsze własności węzłów skręconych zależą od $n$, ilości półskrętów.
Indeks skrzyżowaniowy wynosi $n + 2$.

\begin{proposition}
    Wielomianowymi niezmiennikami węzłów skręconych są:
    \begin{align*}
    (q+1)\jones(q) & = \begin{cases}
        1+q^{-2}+q^{-n}-q^{-n-3} & n \mbox{ nieparzyste} \\
        q^{3}+q-q^{3-n}+q^{-n} & n \mbox{ parzyste}
    \end{cases} \\
    2 \conway (z) & = \begin{cases}
        (n+1) z^{2} + 2 & n \mbox{ nieparzyste} \\
        2 - nz^2 & n \mbox{ parzyste}
    \end{cases}
    \end{align*}
\end{proposition}

\begin{proposition}
    Niewęzeł oraz węzeł dokerski $6_1$ są jedynymi skręconymi węzłami plastrowymi.
\end{proposition}

\begin{proof}
    \cite{casson86}.
\end{proof}
