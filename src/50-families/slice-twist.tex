
\subsection{Węzły skręcone}
\index{węzeł!skręcony|(}%

% DICTIONARY;twist;skręcony;węzeł
Węzły skręcone uważa się za najprostszą (po torusowych) rodzinę węzłów.
Wspomina o~nich bardzo krótko Kawauchi \cite[s. 31]{kawauchi96}.

\begin{definition}
    Węzeł powstały przez $n$-krotne półskręcanie domkniętej pętli oraz splecienie końców nazywamy węzłem skręconym.
\end{definition}

Węzły skręcone to dokładnie towarzyszące niewęzłowi w~węzłach satelitarnych, tak zwane whiteheadowskie duble niewęzła.
Wszystkie są odwracalne (ale tylko niewęzeł oraz ósemka są zwierciadlane) i~mają liczbę gordyjską $1$, ponieważ wystarczy rozwiązać skrzyżowanie, które plotło końce.
\index{liczba gordyjska}%
Każdy jest dwumostowy (ćwiczenie u~Rolfsena \cite[s. 114]{rolfsen76}) i~posiada zerową sygnaturę.
\index{węzeł!dwumostowy}%
\index{sygnatura}%
Dalsze własności węzłów skręconych zależą od $n$, ilości półskrętów.
Indeks skrzyżowaniowy wynosi $n + 2$.

\begin{proposition}
\index{wielomian!Conwaya}%
    Niech $K$ będzie węzłem $n$-skręconym.
    Wtedy
    \begin{equation}
    2 \conway (z) = \begin{cases}
        2 + (n+1) z^{2} & n \mbox{ nieparzyste} \\
        2 - nz^2 & n \mbox{ parzyste}
    \end{cases}
    \end{equation}
\end{proposition}

\begin{proposition}
\index{wielomian!Jonesa}%
    Niech $K$ będzie węzłem $n$-skręconym.
    Wtedy
    \begin{equation}
    (q+1)\jones(q) = \begin{cases}
        1+q^{-2}+q^{-n}-q^{-n-3} & n \mbox{ nieparzyste} \\
        q^3(1+q^{-2}-q^{-n}+q^{-n-3}) & n \mbox{ parzyste}
    \end{cases}
    \end{equation}
\end{proposition}

\begin{proposition}
\index{węzeł!plastrowy}%
    Niewęzeł oraz węzeł dokerski $6_1$ są jedynymi skręconymi węzłami plastrowymi.
\end{proposition}

\begin{proof}
    Casson, Gordon w~\cite{casson86}.
\end{proof}

\index{węzeł!skręcony|)}%

% koniec podsekcji Węzły skręcone

