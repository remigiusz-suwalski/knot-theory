
\section{Węzły hiperboliczne}
\index{węzeł!hiperboliczny|(}
\label{sec:hyperbolic}
Jak pisaliśmy w~sekcji \ref{sec:mutant}, słynne węzły Conwaya oraz Kinoshity-Terasakiego odróżnił od siebie po raz pierwszy Riley.
\index{człowiek!Riley, Robert}
\index{węzeł!Conwaya}%
\index{węzeł!Kinoshity-Terasakiego}%
Zbadał paraboliczne reprezentacje ich grup w~skończoną grupę prostą $PSL(2, 7)$, co doprowadziło go do odkrycia struktury hiperbolicznej w~dopełnieniu ósemki \cite{riley75}.
\index{ósemka}
% https://arxiv.org/pdf/2002.00564.pdf
Zainspirowany tym wynikiem Thurston najpierw rozłożył dopełnienie ósemki na dwa idealne wielościany, a~potem znacznie uogólnił swój przykład.

Reszta sekcji powstała na podstawie dwóch źródeł: przeglądowej pracy Kalfagianniego, Futera oraz Purcell \cite{purcell19} i~notatek z~wykładów, które były prowadzone przez samą Purcell.
\index{człowiek!Futer, David}%
\index{człowiek!Kalfagianni, Efstratia}%
\index{człowiek!Purcell, Jessica}%
Wiedzę o~węzłach hiperbolicznych można czerpać także z~artykułu Weeksa \cite{weeks05}.
\index{człowiek!Weeks, Jeff}%

% Badając sploty nie ograniczamy się tylko do diagramów, ale korzystamy też z ich dopełnień, to znaczy 3-rozmaitości $S^3 \setminus L$.
% Jest ona homeomorficzna z wnętrzem zwartej rozmaitości $X(L) = S^3 \setminus N(K)$, zwanej zewnętrzem splotu, gdzie $N(L)$ stanowi rurowe otoczenie splotu.
% Dalej możemy stosować maszynierę topologii 3-rozmaitości.

% \begin{definition}[ściśliwy]
%     Niech orientowalna powierzchnia $S$ będzie właściwie\footnote{properly} zanurzona w zwartej, orientowalnej 3-rozmaitości $M$.
%     Załóżmy, że dla każdego dysku $E \subseteq M$ z brzegiem $\partial E \subseteq S$ istnieje dysk $E' \subseteq S$ taki, że $\partial E = \partial E'$.
%     Mówimy wtedy, że powierzchnia $S$ jest nieściśliwa.
% \end{definition}

% $\partial$-ściśliwość

% essential

% Haken

% monodromy of fibration

% Węzły i sploty, które będziemy rozpatrywać dalej, mają szczególną strukturą geometryczną.

\begin{definition}[hiperboliczny]
    Splot $L$, na dopełnieniu którego można zadać zupełną metrykę o~stałej krzywiźnie $-1$ nazywamy hiperbolicznym.
\end{definition}

\begin{proposition}
    Niech $L$ będzie splotem, zaś $\mathbb H^3$ hiperboliczną 3-przestrzenią.
    Splot $L$ jest hiperboliczny wtedy i~tylko wtedy, gdy $S^3 \setminus L = \mathbb H^3 / \Gamma$, gdzie $\Gamma$ jest dyskretną, beztorsyjną grupą izometrii, izomorficzną z~$\pi_1(S^3 \setminus L)$.
\end{proposition}
% TODO: skąd to jest dokładnie?

Thurston podejrzewał, że każda 3-rozmaitość rozkłada się wzdłuż sfer i~nieściśliwych torusów na części wyposażone w~jedną z~ośmiu kanonicznych geometrii:
\begin{itemize}
\item sferyczną $S^3$, albo euklidesową $E^3$, albo hiperboliczną $H^3$,
\item $S^2 \times \R$, albo $H^2 \times \R$,
\item uniwersalne nakrycie $SL(2, \R)$,
\item geometrię Sol albo geometrię Nil.
\end{itemize}
Nie umiał podać pełnego uzasadnienia, w~pracy \cite{thurston82} udowodnił swoje przypuszczenie dla rozmaitości Hakena.
\index{rozmaitość Hakena}
Dowód hipotezy geometryzacyjnej dostarczył mniej więcej dwie dekady później Perelman, nie to jest jednak dla nas najważniejsze.
\index{hipoteza!geometryzacyjna Thurstona}
Z przełomowych prac Thurstona z~lat 70. oraz 80. wynika coś ciekawszego: że dopełnienie węzła jest rozmaitością włóknistą Seiferta, toroidalną albo hiperboliczną.
Innymi słowy, Thurston przedstawił trychotomię:

\begin{theorem}
    \index{twierdzenie!Thurstona}
    Każdy węzeł jest satelitarny, torusowy albo hiperboliczny.
    \index{węzeł!satelitarny}
    \index{węzeł!torusowy}
\end{theorem}
% luźno związane: http://www.deltami.edu.pl/temat/matematyka/topologia/2012/12/27/%William_Thurston_i_hipoteza_geometryzacyjna/

\begin{proof}
    Thurston w~\cite[wniosek 2.5]{thurston82}.
\end{proof}

% https://mathoverflow.net/a/289359 : hyperbolic, toroidal (that is, satellite), or Seifert fibered

Węzły hiperboliczne stanowią najliczniejszą i~najmniej zrozumianą rodzinę węzłów.
Sam Nead, użytkownik portalu MathOverflow napisał, że kryterium Thurstona dzięki maszynerii JSJ oraz pracom innych osób można wysłowić algebraicznie.

\begin{proposition}
    % There is a topological criterion due to Thurston.  Using the JSJ machine (and work of many others) this criterion can also be phrased algebraically.  I'll essay these below.  Please note that the situation is much simpler for knots.  To answer your question most directly, here is the desired reference to Wikipedia.
    % http://en.wikipedia.org/wiki/Hyperbolic_link
    % This page refers to the books of Colin Adams and William Thurston.  Both are excellent.
    % Now, here is Thurston's criterion. (EDIT: exposition improved after reading Bruno Martelli's answer.)
    % Suppose that $L$ is the link and $X$ is the link complement.  Suppose $\pi = \pi_1(X)$. We assume the following properties (and each property assumes the proceeding ones). $\newcommand{\ZZ}{\mathbb{Z}}$
    %  - $L$ is not a split link.  Equivalently, $X$ is contains no essential two-sphere.  Equivalently, $\pi$ is not a free product.
    %  - $L$ is not the unknot. Equivalently, $X$ contains no essential disk. Equivalently, $\pi$ is not $\ZZ$.
    %  - $L$ has no component that is an "undisturbed satellite knot".  Equivalently, $X$ contains no essential torus.
    %  - $L$ is not a torus knot. Equivalently, $X$ contains no essential annulus. These last two topological properties are equivalent to $\pi$ not containing a copy of $\ZZ^2$.
    % Then $X$ admits a hyperbolic structure.
    Niech $L$ będzie splotem, który nie rozszczepia się, nie jest niewęzłem, nie posiada wśród ogniw niezakłóconego węzła satelitarnego oraz nie jest węzłem torusowym.
    \index{splot!rozszczepialny}
    Wtedy $L$ jest hiperboliczny.
\end{proposition}

\begin{proposition}
    Niech $L$ będzie splotem takim, że jego dopełnienie $S^3 \setminus L$ nie zawiera właściwej 2-sfery, właściwego dysku, właściwego torusa oraz właściwego pierścienia.
    Wtedy $L$ jest hiperboliczny.
\end{proposition}

\begin{proposition}
    Niech $L$ będzie splotem takim, że jego grupa $\pi(S^3 \setminus L)$ nie jest produktem wolnym, nie jest izomorficzna z~$\Z$ oraz nie zawiera w~sobie kopii grupy $\Z \oplus \Z$.
    Wtedy $L$ jest hiperboliczny.
\end{proposition}

\begin{proof}
    Patrz \url{https://mathoverflow.net/a/153327}.
\end{proof}

Czas na podanie jakichś przykładów węzłów hiperbolicznych, za \cite{adams05}.

\begin{proposition}
    Każdy alternujący, pierwszy, oraz nierozszczepialny splot jest albo 2-warkoczem (a zatem, torusowy) albo hiperboliczny.
    \index{węzeł!alternujący}
    \index{węzeł!pierwszy}
    \index{węzeł!rozszczepialny}
\end{proposition}

\begin{proof}
\index{człowiek!Menasco, William}%
    Menasco \cite{menasco84} pokazał, że dopełnienie alternującego węzła nie zawiera nieściśliwych nieperyferyjnych torusów.
    To w~połączeniu z~unifikacyjnym twierdzeniem Thurstona dla rozmaitości Hakena kończy dowód.
\index{rozmaitość Hakena}%
    % zarys dowodu zz MathSciNet
\end{proof}

\begin{proposition}
    Nietrywialne pierwsze prawie alternujące węzły są torusowe albo hiperboliczne.
    \index{węzeł!pierwszy}
    \index{węzeł!prawie alternujący}
\end{proposition}

\begin{proof}
    Grupa studentów pod opieką Adamsa w \cite{brock92}.
\end{proof}

\begin{proposition}
    Toroidalnie alternujące węzły pierwsze są torusowe albo hiperboliczne.
    \index{węzeł!pierwszy}
    \index{węzeł!toroidalnie alternujący}
\end{proposition}

Ze wszystkich węzłów pierwszych do 11 skrzyżowań i~pierwszych, nierozszczepialnych splotów do 10 skrzyżowań tylko 3 węzły i~2 sploty nie są toroidalnie alternujące, tak twierdzi Adams \cite{adams05}.

\begin{proof}
    Patrz \cite{adams994}.
\end{proof}

\begin{proposition}
    Sploty Montesinosa są prawie zawsze torusowe albo hiperboliczne.
    \index{splot!Montesinosa}
\end{proposition}

\begin{proof}
    Najpierw zidentyfikowano sploty Montesinosa torusowe, które są też torusowe \cite{boileau80}.
    Potem w~pracy \cite{oertel84} znaleziono listę wyjątków (z chyba trochę inną notacją niż nasza):
    \begin{itemize}
        \item $K(1/2, 1/2, 21/2, 21/2)$,
        \item $K(2/3, 21/3, 21/3)$,
        \item $K(1/2, 21/4, 21/4)$,
        \item $K(1/2, 21/3, 21/6)$,
        \item lub lustra tych splotów. \qedhere
    \end{itemize}
\end{proof}

\begin{proposition}
    Mutant węzła hiperbolicznego jest węzłem hiperbolicznym.
    \index{mutant}
\end{proposition}

\begin{proof}
\index{człowiek!Ruberman, Daniel}%
    Ruberman w~\cite{ruberman87}, patrz wniosek 1.4.
\end{proof}

\begin{proposition}
    Niech $G$ oznacza grupę izometrii wnętrza dopełnienia węzła hiperbolicznego.
    Wtedy $G$ jest diedralna lub skończona cykliczna.
\end{proposition}

\begin{proof}
\index{człowiek!Riley, Robert}%
\index{człowiek!Kodama, ?}%
\index{człowiek!Sakuma, ?}%
    Pierwszy był Riley w~artykule \cite[s. 124]{riley79}, można też zapoznać się z~późniejszą pracą \cite{kodama92} Kodamy i Sakumy.
    % Kodama - lemat_1.1
\end{proof}

Kawauchi \cite[s. 131]{kawauchi96} wprowadza jeszcze jedną grupę (grupę symetrii węzła): iloraz grupy PL automorfizmów pary $(S^3, K)$ przez podgrupę elementów, które są otaczająco izotopijne z~odwzorowaniem tożsamościowym.
Okazuje się, że nie wszystkie są skończone:

\begin{proposition}
    Węzeł $K$ ma skończoną grupę symetrii wtedy i~tylko wtedy, gdy jest hiperboliczny, torusowy lub kablem węzła torusowego.
\end{proposition}

Kawauchi nie podaje dowodu, ale zaleca zajrzeć do pracy Sakumy.
Ja zajrzałem i dalej nie mam pojęcia, jak ten dowód miałby wyglądać.

Z kryterium Thurstona mamy prosty wniosek (bo węzły złożone są satelitarne):

\begin{corollary}
    Każdy węzeł hiperboliczny jest pierwszy.
    \index{węzeł!pierwszy}
\end{corollary}

Prawie każdy węzeł pierwszy o~mniej niż 17 skrzyżowaniach jest hiperboliczny, na 32 wyjątki składa się 12 węzłów torusowych oraz 20 satelitów trójlistnika.
Te ostatnie mają co najmniej 13 skrzyżowań.
Baza ciągów liczb całkowitych OEIS zawiera informacje na temat liczności poszczególnych typów węzłów.
Analizując ciągi A051764, A051765 oraz A052408 można dojść do wniosku, że wraz ze wzrostem liczby skrzyżowań, stosunek liczby węzłów hiperbolicznych do wszystkich węzłów dąży do $1$:

\begin{figure}[H]
\renewcommand*{\arraystretch}{1.4}
\footnotesize
\begin{longtable}{lcccccccccccccc}
\hline
    \textbf{rodzaj} & 3 & 4 & 5 & 6 & 7 & 8  & 9  & 10  & 11  & 12   & 13   & 14    & 15     \\ \hline \endhead
    torusowe        & 1 & 0 & 1 & 0 & 1 & 1  & 1  & 1   & 1   & 0    & 1    & 1     & 2      \\
    satelitarne     & 0 & 0 & 0 & 0 & 0 & 0  & 0  & 0   & 0   & 0    & 2    & 2     & 6      \\
    hiperboliczne   & 0 & 1 & 1 & 3 & 6 & 20 & 48 & 164 & 551 & 2176 & 9985 & 46969 & 253285 \\
    \hline
\end{longtable}
\normalsize
\end{figure}

W pracy \cite{malyutin16} Malutin pokazał jednak, że to przypuszczenie jest sprzeczne z~wieloma innymi starymi hipotezami teorii węzłów: \ref{con:malyutin1} -- \ref{con:malyutin4}.
\index{człowiek!Malyutin, Andrei}%

\begin{conjecture}
    \label{con:malyutin1}
    Indeks skrzyżowaniowy jest addytywny względem sumy spójnej.
    \index{indeks skrzyżowaniowy}
    \index{suma spójna}
\end{conjecture}

(To jest powtórzenie hipotezy \ref{con:crossing_additive}).

\begin{conjecture}
    Satelita ma większy (w słabszej wersji: nie mniejszy) indeks skrzyżowaniowy niż jego towarzysze.
    \index{węzeł!satelitarny}
\end{conjecture}

Lackenby pokazał w~\cite{lackenby14}, że jeśli $K$ jest satelitą z~towarzyszem $L$, to $\crossing K \ge 10^{-13} \crossing L$.
\index{człowiek!Lackenby, Marc}%

\begin{conjecture}
%label{con:malyutin3}
    Węzeł złożony ma większy (w słabszej wersji: nie mniejszy) indeks skrzyżowaniowy niż jego składniki.
    \index{węzeł!pierwszy}
\end{conjecture}

Mówimy, że węzeł pierwszy $P$ jest $\lambda$-regularny, jeśli $\crossing K \ge \lambda \cdot \crossing P$ za każdym razem, gdy węzeł $P$ jest składnikiem węzła $K$.
\index{węzeł!regularny}
Zatem hipotezę można wysłowić krótko ,,węzły pierwsze są $1$-regularne''.
Z tego, co pisaliśmy po hipotezie \ref{con:crossing_additive} wynika, że hipoteza \ref{con:malyutin1} jest prawdziwa w~klasie węzłów alternujących czy torusowych i~że wszystkie węzły są $1/152$-regularne.

\begin{conjecture}
    \label{con:malyutin4}
    Węzły pierwsze są $2/3$-regularne.
\end{conjecture}

Rozwiązanie zagadki przyniosła praca samego Malutina \cite{malyutin19} opublikowana latem 2019 roku, przynajmniej dla splotów.
\index{człowiek!Malyutin, Andrei}%
Pokazał w~niej, że jeśli oznaczymy liczbę splotów pierwszych i~nierozszczepialnych o~$n$ lub mniej skrzyżowaniach przez $P_n$, zaś liczbę hiperbolicznych splotów, także o~$n$ lub mniej skrzyżowaniach, przez $H_n$, prawdziwe będzie oszacowanie
\index{splot!rozszczepialny}
\index{węzeł!pierwszy}
\begin{equation}
    \liminf_{n \to \infty} \frac{H_n}{P_n} < 1 - 10^{-13}.
\end{equation}

Czwarty rozdział książki \cite{purcell20} zawiera ćwiczenie, by znaleźć dwuparametrową rodzinę zupełnych struktur hiperbolicznych na dziurawym torusie oraz czterokrotnie dziurawej sferze.
Elastyczność tego rodzaju nie występuje w~przestrzeniach wyższych wymiarów.
Z~twierdzenia o~sztywności, w~wersji algebraicznej:

\begin{theorem}[Mostow-Prasad]
    \index{twierdzenie!o sztywności}
    Niech $\Gamma_1, \Gamma_2$ będą dyskretnymi podgrupami grupy izometrii $\mathbb H^n$ dla $n \ge 3$ takimi, że ilorazy $\mathbb H^n/\Gamma_i$ mają skończone objętości.
    Załóżmy też, że istnieje izomorfizm grup $\varphi \colon \Gamma_1 \to \Gamma_2$.
    Wtedy podgrupy $\Gamma_1, \Gamma_2$ są sprzężone.
\end{theorem}
\index{twierdzenie!Mostowa-Prasada}

albo geometrycznej:

\begin{theorem}[Mostow-Prasad]
    Niech $M_1, M_2$ będą zupełnymi, hiperbolicznymi rozmaitościami o skończonych objętościach.
    Wtedy każdy izomorfizm grup podstawowych $\varphi \colon \pi_1(M_1) \to \pi_1(M_2)$ realizowany jest jednoznacznie przez izometrię.
\end{theorem}

wynika, że jeśli znaleźliśmy jakąś zupełną strukturę hiperboliczną na dopełnieniu splotu, to innych już nie ma.

\begin{proof}
\index{człowiek!Benedetti, Riccardo}%
\index{człowiek!Mostow, George}%
\index{człowiek!Petronio, Carlo}%
\index{człowiek!Prasad, Gopal}%
\index{człowiek!Thurston, William}%
    Thurston przedstawił szkic rozumowania w~sekcji 5.9 swoich notatek, na bazie których powstała później książka \cite{thurston97}.
    Inne szczegółowe rozumowanie można znaleźć w~rozdziale C podręcznika Benedettiego, Petronio \cite{benedetti92}.
    Patrz także oryginalne prace: Mostowa \cite{mostow73}, Prasada \cite{prasad73}.
\end{proof}

Twierdzenie Mostowa-Prasada pozwala nam na wprowadzenie nowych niezmienników splotów hiperbolicznych: wystarczy wziąć dowolny geometryczny niezmiennik dopełnienia węzła.
Najważniejszym z~nich wydaje się być objętość.

\index{objętość|(}
\begin{definition}[objętość]
    Niech $L$ będzie splotem hiperbolicznym.
    Objętość dopełnienia $L$ względem zupełnej metryki hiperbolicznej nazywamy objętością splotu $L$ i~oznaczamy $\volume L$.
\end{definition}

Objętość jest zawsze skończoną liczbą rzeczywistą.
Dla wygody przyjmuje się czasami, że objętość węzłów torusowych oraz satelitarnych wynosi $0$.
Komputerowy program SnapPea napisany przez Weeksa pozwala na wyznaczenie objętości dowolnego splotu o~rozsądnej ilości skrzyżowań.
\index{SnapPea}

\begin{example}
    $\volume 4_1 = -6 \int_{0}^{\pi/3} \log |2\sin \theta| \,\mathrm{d}\theta \approx 2.0298832$.
\end{example}

Patrz też ciąg A091518 w~bazie danych OEIS.
Jak zobaczymy później, żaden węzeł nie ma mniejszej objętości.

\begin{example}
    $\volume 5_2 \approx 2.82812$.
\end{example}

W encyklopedii Wolfram Mathworld znajduje się informacja, że $5_2$ oraz pewien węzeł o~dwunastu skrzyżowaniach mają tę samą objętość, prawdopodobnie chodzi tu o~$12n_{242}$, który znany jest także jako $(-2, 3, 7)$-precel.
\index{precel!(-2, 3, 7)}

\begin{example}
    $\volume 6_1 \approx 3.16396$.
\end{example}

\begin{example}
    $\volume 6_2 \approx 4.40083$.
\end{example}

\begin{example}
    $\volume 6_3 \approx 5.69302$.
\end{example}

\begin{example}
    $\volume 7_4 \approx 5.13794$.
\end{example}

\begin{example}
    Niech $K$ będzie jednym z~dwóch węzłów w~parze Perko.
    Wtedy $\volume K \approx 5.63877$.
    \index{para Perko}
\end{example}

Praca \cite{purcell19} wspomina kilka przyjemnych ograniczeń, jakie musi spełniać objętość.
\index{człowiek!Futer, David}%
\index{człowiek!Kalfagianni, Efstratia}%
\index{człowiek!Purcell, Jessica}%
Aby je przytoczyć, musimy najpierw zdefiniować dwie stałe: $v_4$ oraz $v_8$, odpowiednio objętość idealnego czworościanu\footnote{Albo rozmaitości Giesekinga, powstałej z czworościanu przez usunięcie  wierzchołków i sklejenie ściany 012 z 310 oraz 023 z 321. Dopełnienie ósemki jest podwójnym nakryciem tej rozmaitości.\index{rozmaitość Giesekinga}} oraz ośmiościanu foremnego w~$\mathbb H^3$.
Mamy
\begin{align}
    v_4 & = \int_{0}^{2\pi/3} \log(2 \cos(\theta/2)) \,\mathrm{d}\theta \approx 1.01494\,16064, \\
    % https://en.wikipedia.org/wiki/Gieseking_manifold
    v_8 & = 4 \sum_{n=0}^\infty \frac{(-1)^n}{(2n+1)^2} \approx 3.66386\,23767. % ... 08876060218414059729536443096597497126688537065 ... \ldots
\end{align}

I tak najpierw Adams pokazał w~swojej rozprawie doktorskiej \cite{adams83}:
\index{człowiek!Adams, Colin}%

\begin{proposition}
    Niech $D$ będzie diagramem hiperbolicznego splotu o~$\crossing L \ge 5$ skrzyżowaniach.
    Wtedy
    \begin{equation}
        \volume L \le 4 (\crossing D - 4) v_4.
    \end{equation}
\end{proposition}

A trzy dekady później poprawił wswój wynik w~\cite{adams13}:

\begin{proposition}
    Niech $D$ będzie diagramem hiperbolicznego splotu o~$\crossing L \ge 5$ skrzyżowaniach.
    Wtedy
    \begin{equation}
        \volume L \le (\crossing D - 5) v_8 + 4v_4.
    \end{equation}
\end{proposition}

Jego metoda polega na podzieleniu dopełnienia splotu na czterościany i~ośmiościany oraz policzeniu ich.
To, w~połączeniu ze znanymi ograniczeniami na objętość ,,cegiełek'', wystarcza.
Podział na ośmiościany zaproponował Dylan (nie William!) Thurston.
% wiem to z purcell19

Thurston zauważył \cite[s. 365]{thurston82}, że tylko skończenie wiele hiperbolicznych 3-rozmaitości może mieć tę samą objętość -- wynika to z~prac Gromowa i~Jørgensena.
Następnie Wielenberg przedstawił w~\cite{wielenberg81} przykłady pokazujące, że istnieją dowolnie duże kolizje wśród węzłów hiperbolicznych: pewne podgrupy klasycznej grupy Picarda działają jako izometrie na górną półprzestrzestrzeń hiperboliczną wymiaru 3 mają podstawowe wielościany, które są takie same jako zbiory, ale różnią się jeśli chodzi o~utożsamienie ze sobą ścian.

Chociaż mutanty mają tę samą objętość hiperboliczną (fakt \ref{mutants_the_same_volume}), to praktyka pokazuje, że ten niezmiennik dobrze wspomaga proces tablicowania węzłów.

\begin{proposition}
    Zbiór
    \[
        \{\volume K: K \textrm{ jest hiperboliczny}\} \subseteq \R
    \]
    jest dobrze uporządkowany, typu porządkowego $\omega^\omega$.
\end{proposition}

\begin{proof}[Niedowód]
    Zdaniem angielskiej Wikipedii, dowód jest gdzieś w~\cite{neumann85} (gdzie Neumann znajduje eleganckie oszacowanie zmiany objętości po wykonaniu chirurgii Dehna), ja tego nie widzę.
    %=% wikipedia - angielski artykuł "hyperbolic volume" 
    Hodgson, Masa \cite{hodgson13} sugerują, że dowód da się znaleźć w notatkach Thurstona \cite{thurston02}.
\end{proof}

W dowolnej rodzinie węzłów istnieje element o~najmniejszej objętości.
Przytoczę teraz przykłady konkretnych rodzin i najmniejszych węzłów, za Futerem, Kalfagiannim, Purcell \cite[s. 16-17]{purcell19} oraz Hodgsonem, Masaiem\cite[s. 1-99]{hodgson13}.
\index{człowiek!Futer, David}%
\index{człowiek!Kalfagianni, Efstratia}%
\index{człowiek!Purcell, Jessica}%
\index{człowiek!Hodgson, Craig}%
\index{człowiek!Masai, Hidetoshi}%

\begin{proposition}
%label{prp:eight_least_hyperbolic}
    Żaden węzeł nie ma mniejszej objętości hiperbolicznej od ósemki.
    \index{ósemka}
\end{proposition}

\begin{proof}
    Cao, Meyerhoff w~\cite{cao01} przeanalizowali pakowania horokul w~uniwersalnym nakryciu związanym z~rozmaitościami.
    Doszli do wniosku, że nie ma tam dostatecznieo wolnego miejsca, jeżeli szpic (cusp) nie jest odpowiedniego rozmiaru.
    Trzykrotnie wspierają się przy tym pomocą komputera, by sprawdzić, że określone warunki są spełnione we wszystkich punktach danej przestrzeni parametrów.
\end{proof}

\begin{proposition}
% DICTIONARY;cusped;szpiczasta;rozmaitość
% DICTIONARY;manifold;rozmaitość;-
\index{splot!Whiteheada}%
\index{ósemka}%
    Wśród orientowalnych 3-rozmaitości ze szpicem\footnote{rozmaitość szpiczasta -- niezwarte, zupełne hiperboliczne rozmaitości ze skończoną objętością Riemanna} najmniejszą objętość posiada dopełnienie ósemki oraz jego bliźniak, otrzymany przez $(5, 1)$-chirurgię jednego z~ogniw splotu Whiteheada.
% sformułowanie wygląda jak z "THE MINIMAL VOLUME ORIENTABLE HYPERBOLIC 3-MANIFOLD WITH 4 CUSPS"
\end{proposition}

Klasa rozmaitości wspomniana w fakcie obejmuje dopełnienia hiperbolicznych węzłów.
Powyższy fakt także został wzięty z~pracy \cite{cao01}.

Meyerhoff nie przestawał pracować nad rozmaitościami o~małych objętościach i~osiem lat później w~\cite{meyerhoff09} przedstawił z Gabaiem, Milleyem bez dowodu (obiecali pokazać go później):

\begin{proposition}
    Istnieje 10 orientowalnych 3-rozmaitości z~jednym szpicem o~objętości co najwyżej $2.848$: \texttt{m003}, \texttt{m004} ($\approx 2.02988$), \texttt{m006}, \texttt{m007} ($\approx 2.56897$), \texttt{m009}, \texttt{m010} ($\approx 2.66674$), \texttt{m011} ($\approx 2.78183$), \texttt{m015}, \texttt{m016} oraz \texttt{m017} ($\approx 2.82812$).
    Nazwy pochodzą ze spisu rozmaitości programu SnapPy.
\end{proposition}

Udało mi się rozszyfrować niektóre nazwy.
\texttt{m003} to siostra $4_1$, % https://hal.archives-ouvertes.fr/hal-02867890/document Michel Planat - Quantum computing thanks to Bianchi groups
\texttt{m004} to węzeł $4_1$, % SnapPy - also known as
% m006
% m007
% m009
% m010
% m011
\texttt{m015} to węzeł $5_2$,
\texttt{m016} to węzeł $12n242$, czyli znany nam już $(-2, 3, 7)$-precel,
\index{precel!(-2, 3, 7)}%
\texttt{m017} to siostra $5_2$. % https://arxiv.org/pdf/2107.03275.pdf

W tej samej pracy możemy jeszcze znaleźć informację, że:

\begin{proposition}
    Istnieje dokładnie jedna domknięta hiperboliczna 3-rozmaitość o najmniejszej objętości, rozmaitość Weeksa.
\end{proposition}

Rozmaitość Weeksa została odkryta przez Jeffreya Weeksa w jego rozprawie doktorskiej (1985) oraz niezależnie przez Matwiejewa, Fomenko (1988).
\index{rozmaitość Weeksa}
Powstaje ona przez wykonanie $(5, 2)$ oraz $(5, 1)$ chirurgii Dehna na dopełnieniu splotu Whiteheada, zaś jej objętość wynosi w~przybliżeniu $0.94270$. % https://oeis.org/A126774
\index{splot!Whiteheada}

Następna jest rozmaitość Meyerhoffa, powstała po $(5, 1)$ chirurgii na dopełnieniu ósemki.
\index{rozmaitość Meyerhoffa}
Meyerhoff sugerował w 1987, że ma najmniejszą objętość, ale okazało się potem, że ta wynosi $\approx 0.98136$.

\begin{proposition}
    Wśród orientowalnych 3-rozmaitości o~dwóch szpicach najmniejszą objętość mają splot Whiteheada oraz $(-2, 3, 8)$-precel.
\index{splot!Whiteheada}%
\index{precel!(-2, 3, 8)}%
\end{proposition}

% TODO: check cusped manifold in dictionary

Ich objętość wynosi $v_8$.

\begin{proof}
    Agol \cite{agol10} korzystając z metod topologicznych dowodzi istnienia ,,niezbędnej'' (z ang. essential) powierzchni, która zadaje dolne ograniczenie na objętość i skutecznie krępuje rozmaitości, które mogą to ograniczenie zrealizować.
\end{proof}

Przypadek trzech szpiców nie jest zbyt dobrze zrozumiany.

\begin{proposition}
    Wśród orientowalnych 3-rozmaitości o~czterech szpicach najmniejszą objętość posiada dopełnienie splotu $8_4^2$ wg numeracji Rolfsena (L8a13).
\end{proposition}

\begin{proof}
    Rozumowanie Yoshidy \cite{yoshida13} oparte o pracę Agola.
    Objętość splotu wynosi $2v_8$.
\end{proof}

\index{objętość|)}

\index{węzeł!hiperboliczny|)}

% Koniec sekcji Węzły hiperboliczne

