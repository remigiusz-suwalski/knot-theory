\section{Węzły hiperboliczne} % (fold)
\label{sec:hyperbolic}
Jak pisaliśmy w sekcji \ref{mutants_mutation}, słynne węzły Conwaya oraz Kinoshity-Terasakiego odróżnił od siebie po raz pierwszy przez Rileya.
Zbadał paraboliczne reprezentacje ich grup w skończoną grupę prostą $PSL(2, 7)$, co doprowadziło go do odkrycia w dopełnieniu ósemki struktury hiperbolicznej \cite{riley75}.
% https://arxiv.org/pdf/2002.00564.pdf
Zainspirowany tym wynikiem Thurston najpierw rozłożył dopełnienie ósemki na dwa idealne wielościany, a potem znacznie uogólnił swój przykład.

Reszta sekcji powstała na podstawie przeglądowej pracy \cite{purcell19} Futera, Kalfagianniego, Purcell i innych oraz notatek z~wykładów, które były prowadzone przez samą Purcell.
Wiedzę o~węzłach hiperbolicznych można czerpać także z~artykułu: \cite{weeks05} Weeksa.
Poprawiona wersja dostępna w~serwisie arXiv.

% Badając sploty nie ograniczamy się tylko do diagramów, ale korzystamy też z ich dopełnień, to znaczy 3-rozmaitości $S^3 \setminus L$.
% Jest ona homeomorficzna z wnętrzem zwartej rozmaitości $X(L) = S^3 \setminus N(K)$, zwanej zewnętrzem splotu, gdzie $N(L)$ stanowi rurowe otoczenie splotu.
% Dalej możemy stosować maszynierę topologii 3-rozmaitości.

% \begin{definition}[ściśliwy]
%     Niech orientowalna powierzchnia $S$ będzie właściwie\footnote{properly} zanurzona w zwartej, orientowalnej 3-rozmaitości $M$.
%     Załóżmy, że dla każdego dysku $E \subseteq M$ z brzegiem $\partial E \subseteq S$ istnieje dysk $E' \subseteq S$ taki, że $\partial E = \partial E'$.
%     Mówimy wtedy, że powierzchnia $S$ jest nieściśliwa.
% \end{definition}

% $\partial$-ściśliwość

% essential

% Haken

% monodromy of fibration

% Węzły i sploty, które będziemy rozpatrywać dalej, mają szczególną strukturą geometryczną.

\begin{definition}[hiperboliczny]
    \index{węzeł!hiperboliczny}
    Splot $L$, na dopełnieniu którego można zadać zupełną metrykę o stałej krzywiźnie $-1$ nazywamy hiperbolicznym.
\end{definition}

\begin{proposition}
    Splot $L$ jest hiperboliczny wtedy i tylko wtedy, gdy $S^3 \setminus L = \mathbb H^3 / \Gamma$, gdzie $\mathbb H^3$ to hiperboliczna 3-przestrzeń, zaś $\Gamma$ jest dyskretną, beztorsyjną grupą izometrii, izomorficzną z $\pi_1(S^3 \setminus L)$.
\end{proposition}

Thurston podejrzewał, że każda 3-rozmaitość rozkłada się wzdłuż sfer i nieściśliwych torusów na części wyposażone w jedną z ośmiu kanonicznych geometrii:
\begin{itemize}
\item sferyczną $S^3$,
\item euklidesową $E^3$,
\item hiperboliczną $H^3$,
\item $S^2 \times \R$,
\item $H^2 \times \R$,
\item uniwersalne nakrycie $SL(2, \R)$,
\item geometrię Sol albo
\item geometrię Nil.
\end{itemize}
Nie umiał podać pełnego uzasadnienia, w 1982 roku udowodnił swoje przypuszczenie dla rozmaitości Hakena.
Dowód hipotezy geometryzacyjnej dostarczył mniej więcej dwie dekady później Perelman, nie to jest jednak dla nas najważniejsze.

Thurston pokazał też, że dopełnienie węzła jest rozmaitością włóknistą Seiferta, toroidalną albo hiperboliczną.
Innymi słowy, przedstawił trychotomię:
\begin{theorem}
    \index{twierdzenie!Thurstona}
    Każdy węzeł jest satelitarny, torusowy albo hiperboliczny.
\end{theorem}

Węzły hiperboliczne stanowią najliczniejszą i najmniej zrozumianą rodzinę węzłów.

\begin{proof}
    Thurston w \cite{thurston82}.
\end{proof}

Dwa lata później Menasco \cite{menasco84} pokazał, że każdy pierwszy, alternujący splot jest albo 2-warkoczem (a zatem, torusowy) albo hiperboliczny.

\begin{corollary}
    Każdy węzeł hiperboliczny jest pierwszy.
\end{corollary}

Prawie każdy węzeł pierwszy o~mniej niż 17 skrzyżowaniach jest hiperboliczny, na 32 wyjątki składa się 12 węzłów torusowych oraz 20 satelitów trójlistnika.
Te ostatnie mają co najmniej 11 skrzyżowań.
Baza ciągów liczb całkowitych OEIS zawiera informacje na temat liczności poszczególnych typów węzłów.
Analizując ciągi A051764, A051765 oraz A052408 można dojść do wniosku, że wraz ze wzrostem liczby skrzyżowań, stosunek liczby węzłów hiperbolicznych do wszystkich węzłów dąży do $1$:

\begin{figure}[H]
\renewcommand*{\arraystretch}{1.4}
\footnotesize
\begin{longtable}{lcccccccccccccc}
\hline
    \textbf{rodzaj} & 3 & 4 & 5 & 6 & 7 & 8  & 9  & 10  & 11  & 12   & 13   & 14    & 15     \\ \hline \endhead
    torusowe        & 1 & 0 & 1 & 0 & 1 & 1  & 1  & 1   & 1   & 0    & 1    & 1     & 2      \\
    satelitarne     & 0 & 0 & 0 & 0 & 0 & 0  & 0  & 0   & 0   & 0    & 2    & 2     & 6      \\
    hiperboliczne   & 0 & 1 & 1 & 3 & 6 & 20 & 48 & 164 & 551 & 2176 & 9985 & 46969 & 253285 \\
    \hline
\end{longtable}
\normalsize
\end{figure}

W pracy \cite{malyutin16} A. Malyutin pokazał jednak, że to przypuszczenie jest sprzeczne z~wieloma innymi starymi hipotezami teorii węzłów: \ref{malyutin1} -- \ref{malyutin4}.

\begin{conjecture}
    \label{malyutin1}
    Indeks skrzyżowaniowy jest addytywny względem sumy spójnej.
\end{conjecture}

(To jest powtórzenie hipotezy \ref{cnj:crossing_additive}).
Murasugi dowiódł prawdziwości hipotezy dla węzłów alternujących, w~pracy \cite{murasugi87} jest to wniosek z~dowodu hipotezy Taita.
Krótko po tym Lickorish, Thistlethwaite powtórzyli to dla węzłów adekwatnych w \cite{lickorish88}.
Na początku XX wieku Diao \cite{diao04} oraz Gruber \cite{gruber03} niezależnie udowodnili hipotezę \ref{malyutin1} dla pewnej szerokiej klasy węzłów, obejmującej wszystkie węzły torusowe oraz wiele węzłów alternujących oraz pewne inne obiekty, których nie chcemy opisywać.

\begin{conjecture}
    Satelita ma większy (w słabszej wersji: nie mniejszy) indeks skrzyżowaniowy niż jego towarzysze.
\end{conjecture}

Lackenby pokazał w~\cite{lackenby14}, że jeśli $K$ jest satelitą z towarzyszem $L$, to $\operatorname{cr} K \ge 10^{-13} \operatorname{cr} L$.

\begin{conjecture}
    Węzeł złożony ma większy (w słabszej wersji: nie mniejszy) indeks skrzyżowaniowy niż jego faktory.
\end{conjecture}

Mówimy, że węzeł pierwszy $P$ jest $\lambda$-regularny, jeśli $cr K \ge \lambda \cdot cr P$ za każdym razem, gdy węzeł $P$ jest faktorem węzła $K$.
Zatem hipotezę można wysłowić krótko ,,węzły pierwsze są $1$-regularne''.
Na podstawie prac Murasugiego, Kauffmana i~Thistlethwaite'a z~końca lat 80. wiemy, że zachodzi dla węzłów alternujących.
Diao pokazał w \cite[tw. 3.8]{diao04}, że węzły torusowe także są $1$-regularne, natomiast Lackenby przedstawił w~\cite{lackenby09} rozumowanie, dlaczego wszystkie węzły są $1/152$-regularne.
Pisaliśmy o tym w podsekcji \ref{sub:crossing_number}.

\begin{conjecture}
    \label{malyutin4}
    Węzły pierwsze są $2/3$-regularne.
\end{conjecture}

Rozwiązanie zagadki przyniosła praca samego Malyutina \cite{malyutin19} opublikowana latem 2019 roku, przynajmniej dla splotów.
Pokazał w~niej, że jeśli oznaczymy liczbę splotów pierwszych i~nierozszczepialnych o~$n$ lub mniej skrzyżowaniach przez $P_n$, zaś liczbę hiperbolicznych splotów, także o~$n$ lub mniej skrzyżowaniach, przez $H_n$, prawdziwe będzie oszacowanie
\begin{equation}
    \liminf_{n \to \infty} \frac{H_n}{P_n} < 1 - 10^{-13}.
\end{equation}

% Every non-split, prime, alternating link that is not a~torus link is hyperbolic by a~result of William Menasco.

% https://arxiv.org/abs/math/0309466
% https://arxiv.org/abs/math/0311380

Czwarty rozdział książki \cite{purcell2020} zawiera ćwiczenie, by znaleźć dwuparametrową rodzinę zupełnych struktur hiperbolicznych na dziurawym torusie oraz czterokrotnie dziurawej sferze.
Elastyczność tego rodzaju nie występuje w przestrzeniach wyższych wymiarów.
Z twierdzenia o sztywności, w wersji algebraicznej:

\begin{theorem}[Mostow-Prasad]
    \index{twierdzenie!o sztywności}
    Niech $\Gamma_1, \Gamma_2$ będą dyskretnymi podgrupami grupy izometrii $\mathbb H^n$ dla $n \ge 3$ takimi, że ilorazy $\mathbb H^n$ mają skończone objętości.
    Załóżmy też, że istnieje izomorfizm grup $\varphi \colon \Gamma_1 \to \Gamma_2$.
    Wtedy podgrupy $\Gamma_1, \Gamma_2$ są sprzężone.
\end{theorem}

albo geometrycznej:

\begin{theorem}[Mostow-Prasad]
    Niech $M_1, M_2$ będą zupełnymi, hiperbolicznymi rozmaitościami o skończonych objętościach.
    Wtedy każdy izomorfizm grup podstawowych $\varphi \colon \pi_1(M_1) \to \pi_1(M_2)$ realizowany jest jednoznacznie przez izometrię.
\end{theorem}

wynika, że jeśli znaleźliśmy jakąś zupełną strukturę hiperboliczną na dopełnieniu splotu, to innych już nie ma. 
Dzięki temu niezmienniki mające korzenie w geometrii hiperbolicznej mają przyjemne własności.
Patrz także oryginalne prace: Mostowa \cite{mostow73}, Prasada \cite{prasad73}.

\begin{proof}
    Thurston przedstawił szkic rozumowania w sekcji 5.9 swoich notatek, na bazie których powstała później książka \cite{thurston97}.
    Inne szczegółowe rozumowanie można znaleźć w rozdziale C podręcznika \cite{benedetti92}.
\end{proof}

% luźno związane: http://www.deltami.edu.pl/temat/matematyka/topologia/2012/12/27/%William_Thurston_i_hipoteza_geometryzacyjna/

\begin{tobedone}[kryterium Thurstona]
    Niech $L$ będzie splotem z dopełnieniem $X$ i grupą podstawową $\pi = \pi_1(X)$.
    Jeżeli spełnione są następujące warunki:
    \begin{enumerate}
    \item $L$ nie rozszczepia się,
    \item $L$ nie jest niewęzłem,
    \item żadna składowa $L$ nie jest niezakłóconym węzłem satelitarnym (?),
    \item $L$ nie jest węzłem torusowym,
    \end{enumerate}
    to $L$ jest splotem hiperbolicznym.
    Warunki podane wyżej mają swoje odpowiedniki dla przestrzeni $X$ ($X$ nie zawiera właściwej 2-sfery, właściwego dysku, właściwego torusa, właściwego pierścienia) oraz grupy $\pi$ ($\pi$ nie jest wolnym produktem, nie jest cykliczna oraz nie zawiera kopii $\Z^2$).
\end{tobedone}

\begin{proposition}
    Grupa symetrii węzła hiperbolicznego jest skończona: cykliczna lub diedralna.
\end{proposition}

Uwaga -- nie chodzi tutaj ani o grupę węzła, ani grupę kolorującą, ale prawdopodobnie grupę automorfizmów zewnętrznych grupy $\pi_1(S^3 \setminus K)$.

\begin{proof}
    Pierwszy był Riley w artykule \cite{riley75}, można też zapoznać się z późniejszą pracą \cite{kodama92} Kodamy.
\end{proof}

% objętość

Twierdzenie Mostowa-Prasada pozwala nam na wprowadzenie nowych niezmienników splotów hiperbolicznych: wystarczy wziąć dowolny geometryczny niezmiennik dopełnienia węzła.
Najważniejszym z nich wydaje się być objętość.

\begin{definition}[objętość]
    Niech $L$ będzie splotem hiperbolicznym.
    Objętość dopełnienia $L$ względem zupełnej metryki hiperbolicznej nazywamy objętością splotu $L$ i oznaczamy $\volume L$.
\end{definition}

Objętość jest zawsze skończoną liczbą rzeczywistą.
Dla wygody przyjmuje się czasami, że objętość węzłów torusowych oraz satelitarnych wynosi $0$.
Komputerowy program SnapPea napisany przez J. Weeksa pozwala na wyznaczenie objętości dowolnego splotu o rozsądnej ilości skrzyżowań.

\begin{example}
    $\volume 4_1 \approx 2.0298832$.
\end{example}

Patrz też ciąg A091518 w bazie danych OEIS.

\begin{proposition}
    \label{least_volume}
    Żaden węzeł nie ma mniejszej objętości hiperbolicznej od ósemki.
\end{proposition}

\begin{proof}
    Cao, Meyerhoff w \cite{cao01} przeanalizowali pakowania horokul w uniwersalnym nakryciu związanym z rozmaitościami.
    Doszli do wniosku, że nie ma tam dostatecznieo wolnego miejsca, jeżeli ostrze (cusp) nie jest odpowiedniego rozmiaru.
    Trzykrotnie wspierają się przy tym pomocą komputera, by sprawdzić, że określone warunki są spełnione we wszystkich punktach danej przestrzeni parametrów.
\end{proof}

\begin{example}
    $\volume 5_2 \approx 2.82812$.
\end{example}

W encyklopedii Wolfram Mathworld znajduje się informacja, że $5_2$ oraz pewien węzeł o~dwunastu skrzyżowaniach mają tę samą objętość, prawdopodobnie chodzi tu o~$12n_{242}$, który znany jest także jako $(-2, 3, 7)$-precel. 

\begin{example}
    $\volume 6_1 \approx 3.16396$.
\end{example}

\begin{example}
    $\volume 6_2 \approx 4.40083$.
\end{example}

\begin{example}
    $\volume 6_3 \approx 5.69302$.
\end{example}

\begin{example}
    $\volume 7_4 \approx 5.13794$.
\end{example}

\begin{example}
    Niech $K$ będzie jednym z dwóch węzłów w parze Perko.
    Wtedy $\volume K \approx 5.63877$.
\end{example}

Praca \cite{purcell19} wspomina kilka przyjemnych ograniczeń, jakie musi spełniać objętość.
Aby je przytoczyć, musimy najpierw zdefiniować dwie stałe: $v_4$ oraz $v_8$, odpowiednio objętość idealnego czworościanu (ośmiościanu) foremnego w $\mathbb H^3$.
Mamy
\begin{align}
    v_4 & = 1.0149\ldots \\
    v_8 & = 3.6638\ldots
\end{align}

I tak najpierw Adams pokazał w swojej rozprawie doktorskiej \cite{adams83}:

\begin{proposition}
    Niech $D$ będzie diagramem hiperbolicznego splotu o $\operatorname{cr} L \ge 5$ skrzyżowaniach.
    Wtedy
    \begin{equation}
        \volume K \le 4 (\operatorname{cr} D - 4) v_4.
    \end{equation} 
\end{proposition}

A trzy dekady później poprawił wswój wynik w \cite{adams13}:

\begin{proposition}
    Niech $D$ będzie diagramem hiperbolicznego splotu o $\operatorname{cr} L \ge 5$ skrzyżowaniach.
    Wtedy
    \begin{equation}
        \volume K \le 4 (\operatorname{cr} D - 5) v_8 + 4v_4.
    \end{equation} 
\end{proposition}

Jego metoda polega na podzieleniu dopełnienia splotu na czterościany i ośmiościany oraz policzeniu ich.
To, w połączeniu ze znanymi ograniczeniam na objętość ,,cegiełek'', wystarcza.
Podział na ośmiościany zaproponował Dylan Thurston (nie mylić z Williamem!).

% Wiki

Thurston zauważył, że tylko skończenie wiele hiperbolicznych 3-rozmaitości może mieć tę samą objętość -- wynika to z prac Gromova i Jørgensena.
Następnie Wielenberg przedstawił w~\cite{wielenberg81} przykłady pokazujące, że istnieją dowolnie duże kolizje wśród węzłów hiperbolicznych: pewne podgrupy klasycznej grupy Picarda działają  na półprzestrzeni hiperbolicznej wymiaru 3 i mają przy tym fundamentalne wielościany identyczne jako zbiory, ale znacząco różne w~sposobie, w jaki zidentyfikowano ich ściany.

Na przykład mutacja węzła hiperbolicznego nigdy nie zmienia objętości \cite{ruberman87}, \cite[s. 124]{adams94}.
Praktyka pokazuje jednak, że niezmiennik dobrze wspomaga proces tablicowania węzłów.

\begin{proposition}
    Objętości hiperboliczne 3-rozmaitości tworzą dobrze uporządkowany podzbiór $\R$, typu $\omega^\omega$.
\end{proposition}

\begin{proof}
    Wikipedia twierdzi, że dowód jest gdzieś w \cite{neumann85}, ja tego nie widzę.
\end{proof}

W dowolnej rodzinie węzłów istnieje element o najmniejszej objętości.
Dla orientowalnych, niezwartych hiperbolicznych 3-rozmaitości, klasy obejmującej dopełnienia hiperbolicznych węzłów, najmniejszą objętość posiada dopełnienie ósemki oraz jej bliźniak, otrzymany przez $(5, 1)$-chirurgię jednego z ogniw splotu Whiteheada.
To jest fakt \ref{least_volume}.
Natomiast wśród orientowalnych 3-rozmaitości o dwóch ostrzach najmniejszą objętość, ma $(-2, 3, 8)$-precel oraz splot Whiteheada: Agol pokazał to w \cite{agol10}.
Ich objętość wynosi
\begin{equation}
    v_8 = 4 \sum_{n=0}^\infty \frac{(-1)^n}{(2n+1)^2} = \approx 3.6638623767. % ... 08876060218414059729536443096597497126688537065 ...
\end{equation}
Yoshida \cite{yoshida13} znalazł splot z najmniejszą objętością równą $2v_8$ wśród tych o 4 ogniwach, sploty o 3 ogniwach nie są zbyt dobrze zrozumiane.

% Koniec sekcji Węzły hiperboliczne
