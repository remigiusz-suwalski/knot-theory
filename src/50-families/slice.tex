
\section{Węzły plastrowe i taśmowe}
\label{sec:slice}
Węzły plastrowe i taśmowe oraz pojęcie kobordyzmu, które wkrótce opiszemy, należą do świata 4-wymiarowej teorii węzłów.
Nie zapoznamy się z nią bliżej oraz nie podamy naszego ulubionego odniesienia do tego tematu w~literaturze, ponieważ sami nie rozumiemy go zbyt dobrze.
Wszystko zaczęło się od artykułu \cite{fox66} Foxa, Milnora.

%%% Kawauchi 155:

\begin{definition}[płaski dysk]
    Niech $D \subseteq B^4$ będzie dyskiem posiadającym otoczenie $N$, kopię zbioru $D \times I^2$, która przecina sferę $S^3$ dokładnie w $\partial D \times I^2$.
    Mówimy wtedy, że dysk $D$ jest płaski.
\end{definition}

\begin{tobedone}
    Płaski? Lokalnie płaski?
    \cite[s. 155]{kawauchi96}
\end{tobedone}

\begin{definition}[węzeł plastrowy]
    \index{węzeł!plastrowy}
    % z \cite{gompf86}
    Niech $K$ będzie takim węzłem w $S^3 = \partial B^4$, który ogranicza gładko zanurzony 2-dysk w $B^4$.
    O węźle $K$ mówimy wtedy, że jest plastrowy.
    % stare:
    % Niech $K \subseteq S^3$ będzie takim węzłem, że w kuli $B^4$ istnieje płaski dysk $D$ taki, że $K = \partial D = D \cap S^3$.
    % Wtedy $K$ nazywamy węzłem plastrowym.
\end{definition}

Następujące węzły o~mniej niż jedenastu skrzyżowaniach są plastrowe: $6_1$, $8_{8}$, $8_{9}$, $8_{20}$, $9_{27}$, $9_{41}$, $9_{46}$, $10_{3}$, $10_{22}$, $10_{35}$, $10_{42}$, $10_{48}$, $10_{75}$, $10_{87}$, $10_{99}$, $10_{123}$, $10_{129}$, $10_{137}$, $10_{140}$, $10_{153}$ oraz $10_{155}$.
\index{węzeł!Conwaya}
Wśród pierwszych węzłów do dwunastu skrzyżowań najdłużej opierał się węzeł Conwaya, aż Piccirillo pokazała w~\cite{piccirillo20}, że nie jest plastrowy.
\index[persons]{Piccirillo, Lisa}%

\begin{proposition}
    Niech $K$ będzie węzłem.
    Wtedy $K \shrap mr K$ jest węzłem plastrowym.
\end{proposition}

\begin{proof}[Niedowód]
\index[persons]{Fox, Ralph}%
\index[persons]{Milnor, ?}%
    Pierwszy był Fox z Milnorem \cite{fox66}, patrz także lemat 12.1.2.2 w \cite{kawauchi96}.
\end{proof}

\begin{proposition}
    Albo wszystkie trzy węzły $K_1, K_2, K_1 \shrap K_2$ są plastrowe, albo co najwyżej jeden z~nich.
\end{proposition}

\begin{proof}[Niedowód]
    Lemat 12.1.2.3 w \cite{kawauchi96}.
\end{proof}

Pierwszym poważnym wynikiem z dziedziny teorii węzłów plastrowych, pochodzącym jeszcze z pracy \cite{fox66}, był:

\begin{proposition}[warunek Foxa-Milnora]
    \index{warunek!Foxa-Milnora}
    Niech $K$ będzie węzłem plastrowym.
    Wtedy jego wielomian Alexandera jest postaci $\alexander(t) = f(t) f(1/t)$ dla pewnego wielomianu Laurenta $f \in \Z[t, 1/t]$.
\end{proposition}

\begin{corollary}
    \index{wyznacznik}
    Wyznacznik węzła plastrowego jest kwadratem.
\end{corollary}

\begin{proof}
    Mamy $\det K = |\alexander(-1)| = f(-1) f(-1)$.
\end{proof}

Ten prosty test stwierdza, że 2743 spośród 2977 węzłów o mniej niż 13 skrzyżowaniach nie jest plastrowych.

\begin{proposition}
    \index{sygnatura}
%label{prp:slice_signature}
    Niech $K$ będzie węzłem plastrowym.
    Wtedy $\sigma(K) = 0$.
\end{proposition}

\begin{tobedone}[Szkic dowodu]
    Ustalmy odwzorowanie $f$, które jest nieosobliwe, symetryczne i~dwuliniowe, z~przestrzeni $V$ o~wymiarze $2n$ oraz wyznaczoną przez nie formę kwadratową.
    Jeśli znika ona na podprzestrzeni wymiaru $n$, to ma zerową sygnaturę.
    dowód znaleziony w~podręczniku Lickorisha.
    Patrz też twierdzenie 8.8 z~artykułu \cite{murasugi65}.
    Praca "Infinite Order Amphicheiral Knots". (Charles Livingston, 2001) -- chyba nie?
\end{tobedone}

Test ten eliminuje kolejne 45 węzłów poniżej 13 skrzyżowań.

\begin{proposition}
    \index{niezmiennik!Arfa}
    Niech $K$ będzie węzłem plastrowym.
    Wtedy $\operatorname{Arf} K = 0$.
\end{proposition}

\begin{proof}
    Ustalmy węzeł $K$, wiemy już, że jego wyznacznik jest kwadratem, a na mocy faktu \ref{cor:knot_determinant_odd} także tyle, że jest liczbą nieparzystą.
    Wynika stąd przystawanie $\det K \equiv 1 \mod 8$, które w~połączeniu z warunkiem Murasugiego (fakt \ref{prp:arf_murasugi}) daje $\operatorname{Arf} K = 0$.
\end{proof}

\begin{proposition}
    \label{prp:trivial_alexander_implies_slice}
    Niech $K$ będzie węzłem w kategorii TOP.
    Jeżeli jego wielomian Alexandera jest trywialny: $\alexander_K(t) \equiv 1$, to węzeł $K$ jest plastrowy.
\end{proposition}

\begin{proof}
\index[persons]{Freedman, Michael}%
    Freedman w \cite[tw. 1.13]{freedman82}.
\end{proof}

Implikacja \ref{prp:trivial_alexander_implies_slice} przestaje być prawdziwa po przejściu do kategorii PL.
Dość klarownie różnicę między kategoriami TOP i PL tłumaczy Gompf w~\cite{gompf86}, wspomniane jest tam także twierdzenie Donaldsona, kluczowy składnik w~uzasadnieniu tej różnicy.
\index[persons]{Gompf, Robert}%


%%% Kawauchi 156:
\subsection{Zgodność}
Zgodność jest relacją równoważności na zbiorze węzłów, która prowadzi do nowej definicji węzłów plastrowych (patrz fakt~\ref{prp:concordant_iff_sum_slice}).
My przytaczamy jej definicję z pracy Gompfa \cite{gompf86}:

\begin{definition}[zgodność]
\index{zgodność}%
\index{węzeł!zgodny|see {zgodność}}%
    Dwa węzły $K_0, K_1$ nazywamy (gładko) zgodnymi, jeżeli istnieje gładko zanurzony pierścień w $S^3 \times I$, którego brzegiem jest zbiór $K_0 \times \{0\} \cup K_1 \times \{1\}$.
\end{definition}

Kawauchi \cite[s. 156]{kawauchi96} pisze ,,Two knots (…) are knot cobordant (or concordant)'', więc tak jak wielu innych autorów nie odróżnia więc węzłów kobordantnych od zgodnych.
Mamy zamiar zrobić dokładnie to samo: różnica między tymi terminami jest subtelna; węzły zgodne są też kobordantne, ale implikacja w drugą stronę nie zachodzi (wiemy o~tym z~tekstu Blanlœila ,,Cobordism and Concordance of Knots'') chyba, że pracuje się z węzłami sferycznmi, a tak jest w klasycznej teorii węzłów.
\index[persons]{Blanloeil, Vincent}%
% https://www.maths.ed.ac.uk/~v1ranick/papers/blanloeil
% Concordant knots are cobordant, but the converse is not true in general.
% "Cobordism and Concordance of Knots" by Vincent Blanlœil

Dlatego my będziemy zawsze pisać o węzłach zgodnych i nigdy o kobordantnynch.

\begin{proposition}
\label{prp:concordant_iff_sum_slice}%
    Dwa węzły $K_1, K_2$ są zgodne wtedy i tylko wtedy, gdy suma $(mr K_0) \shrap K_1$ jest plastrowa.
\end{proposition}

\begin{proof}
    Ćwiczenie 12.1.3 w książce Kawauchiego \cite{kawauchi96}.
\end{proof}

\begin{definition}
    Węzeł zgodny z~niewęzłem nazywamy plastrowym.
\end{definition}

,,Bycie zgodnym'' jest relacją równoważności, słabszą od ,,bycia izotopijnym'', ale chyba mocniejszą od ,,bycia homotopijnym''.
% ale mocniejszą od homotopii?
% izotopia: https://encyclopediaofmath.org/wiki/Cobordism_of_knots
% homotopia: https://en.wikipedia.org/wiki/Link_concordance By its nature, link concordance is an equivalence relation. It is weaker than isotopy, and stronger than homotopy: isotopy implies concordance implies homotopy. A link is a slice link if it is concordant to the unlink.
Klasę abstrakcji węzła $K$ oznaczamy przez $[K]$.

\begin{definition}[grupa zgodności]
\index{grupa!zgodności}%
    Niech $C^1$ oznacza iloraz zbioru wszystkich węzłów przez relację zgodności.
    Zbiór $C^1$ wyposażony w~działanie
    \begin{equation}
        [K_1] + [K_2] = [K_1 \shrap K_2]
    \end{equation}
    staje się grupą abelową, nazywaną grupą zgodności.
    Jej elementem neutralnym jest klasa abstrakcji niewęzła.
    Elementem przeciwnym do $[K]$ jest $[mr K]$.
\end{definition}

%%% Kawauchi 157:

Niech $\Theta$ oznacza rodzinę macierzy Seiferta węzłów (czyli kwadratowych macierzy $V$ o~całkowitych wyrazach takich, że $\det (V - V^T) = 1$).
Mówimy, że macierz $V \in \Theta$ jest zerowo kobordantna, jeżeli jest postaci
\begin{equation}
    V = P \begin{pmatrix} 0 & V_{21} \\ V_{12} & V_{22} \end{pmatrix} P^{-1}
\end{equation}
dla pewnej całkowitoliczbowej macierzy $P$ o~wyznaczniku $\pm 1$; takie macierze nazywamy unimodularnie sprzężonymi.
\index{macierz!unimodularnie sprzężona}%
Każda zerowo kobordantna macierz $V \in \Theta$ stanowi macierz Seiferta pewnego plastrowego węzła $K$.
Kawauchi nazywa te węzły algebraicznie plastrowymi i~mówi, że to dokładnie węzły, które ograniczają izotropowe powierzchnie w kuli $B^4$, więc każdy węzeł plastrowy jest algebraicznie plastrowy.

Suma $(-V) \oplus V$ jest zerowo kobordantna dla każdej macierzy $V \in \Theta$.
To (chyba to) inspiruje Kawauchiego do wprowadzenia kolejnej definicji: dwie macierze $V_1, V_2 \in \Theta$ nazywa kobordantnymi, jeżeli $(-V_1) \oplus V_2$ jest zerowo kobordantna.
Kobordyzm stanowi relację równoważności na $\Theta$ -- iloraz $\Theta$ przez tę relację oznacza się $G_-$, jest grupą abelową.

\begin{proposition}
    % Kawauchi 12.2.8
    Odwzorowanie $\psi \colon C^1 \to G_-$ posyłające klasę abstrakcji węzła w klasę abstrakcji jego macierzy Seiferta jest dobrze określonym epimorfizmem.
\end{proposition}

\begin{proof}
    Nie umiem nic sam udowodnić, więc wymienię tylko trzy odsyłacze: z faktu~\ref{prp:cobordant_to_algebraic_is_algebraic} wynika, że odwzorowanie $\psi$ jest dobrze określone, dowód faktu~\ref{prp:signature_additive} pokazuje, że $\psi$ jest homomorfizmem, zaś w \cite[s. 62]{kawauchi96} można przeczytać, dlaczego jest ,,na''.
\end{proof}

Funkcję $\psi$ rozpatrywał Levine \cite{levine69} w latach sześćdziesiątych.
\index[persons]{Levine, Jerome}%
Po mniej niż dekadzie Casson, Gordon \cite{gordon78} wskazali nietrywialne elementy jądra.
\index[persons]{Gordon, Cameron}%
\index[persons]{Casson, Andrew}%
% to wyżej wiem z kawauchi98, "Supplementary notes for Chapter 12"
Potem był wynik Jianga \cite{jiang81}, że jądro nie jest skończenie generowalne, bo zawiera izomorficzną kopię $\Z^\infty$, a~jeszcze później Livingstona \cite{livingston99}, że zawiera też kopię $(\Z/2\Z)^\infty$.
% to wyżej wiem z https://mathscinet.ams.org/mathscinet-getitem?mr=2179265, pierwsze strony tekstu (nie recenzji)

\begin{proposition}
    $G_- \cong \Z^\infty \oplus (\Z/4\Z)^\infty \oplus (\Z/2\Z)^\infty$.
\end{proposition}

Kawauchi \cite[s. 161]{kawauchi96} bez uzasadnienia postanawia nie przytoczyć dowodu tego faktu, ale opowiada krótko, jaka jest idea przewodnia i odsyła wprost do pracy Levine'a.
Na dalszych stronach jego pracy przeglądowej pojawiają się jakieś formy kwadratowe oraz uogólnienia wszystkiego do zgodności splotów, ale ja wracam nocnym pociągiem, więc nie mam siły o tym pisać.




\subsection{Węzły taśmowe}
\index{węzeł!taśmowy|(}%
\begin{definition}
    Węzeł $K = f(S^1)$ będący brzegiem osobliwego dysku $f \colon D \to S^3$ posiadającego następującą własność: każda przecinająca siebie składowa jest łukiem $A \subseteq f(D^2)$, dla którego $f^{-1}(A)$ składa się z~dwóch łuków w~$D^2$ (jeden z~nich jest wewnętrzny), nazywamy taśmowym.
\end{definition}

Jak pisze Kawauchi, mamy oczywiste wynikanie:

\begin{proposition}
\index{węzeł!plastrowy}%
    Każdy węzeł taśmowy jest plastrowy.
\end{proposition}

Dawno temu Fox \cite[problem 1.33]{kirby78} zapytał, czy implikacja odwrotna jest prawdziwa:
\index[persons]{Fox, Ralph}%

\begin{conjecture}[slice-ribbon problem]
    \index{hipoteza!plastrowo-taśmowa}
    Czy każdy węzeł plastrowy jest taśmowy?
\end{conjecture}

Wprawdzie Lisca pokazał prawdziwość hipotezy dla węzłów dwumostowych \cite{lisca07},
\index[persons]{Lisca, Paolo}%
% korzystając ze słynnego tw. Donaldsona: that a definite intersection form of a compact, oriented, simply connected, smooth manifold of dimension 4 is diagonalisable
\index{węzeł!dwumostowy}%
zaś Greene oraz Jabuka zrobili to dla precli o trzech pasmach w~\cite{greene11};
\index[persons]{Greene, Joshua}%
\index[persons]{Jabuka, Stanisław}%
\index{precel}%
ale Gompf, Scharlemann i~Thompson zasugerowali w~\cite{gompf10} potencjalny kontrprzykład.
\index[persons]{Gompf, Robert}%
\index[persons]{Scharlemann, Martin}%
\index[persons]{Thompson, Abigail}%
\index{rozmaitość szwowa}%
Nie możemy przytoczyć tego kontrprzykładu, gdyż korzysta z~rozmaitości szwowych, opisanych w~\cite[s. 53-59]{kawauchi96}.

Teichner myśli\footnote{Patrz \url{https://mathoverflow.net/a/18154}.} o hipotezie plastrowo-taśmowej jako o~życzeniu, które uprościłoby pewne czterowymiarowe problemy, gdyby było prawdziwe.
\index[persons]{Teichner, Peter}%

\index{węzeł!taśmowy|)}

% koniec podsekcji węzły taśmowe




%%% Kawauchi 157:
\subsection{Węzły algebraicznie plastrowe}
Węzeł, którego macierz Seiferta jest zerowo kobordantna, nazywamy plastrowym algebraicznie.
Lokalnie płaską, zwartą, zorientowaną, właściwą powierzchnię $S$ w $B^4$ taką, że $K = \partial S$ jest węzłem w $\partial B^4 = S^3$ nazywamy izotropową, jeżeli istnieje lokalnie płaska, zwarta, zorientowana 3-podrozmaitość $M \subseteq B^4$, gdzie $S \subseteq \partial M$ oraz $F = \operatorname{cl} \partial M \setminus S$ jest powierzchnią Seiferta dla $K$ w $S^3$, zaś $S$ jest izotropowa w $M$.

\begin{proposition}
    Węzeł $K$ w~$S^3$ jest algebraicznie plastrowy dokładnie wtedy, gdy ogranicza izotropową powierzchnię $S$ w~kuli $B^4$.
\end{proposition}

\begin{corollary}
    Niech $K$ będzie węzłem plastrowym.
    Wtedy $K$ jest węzłem algebraicznie plastrowym.
\end{corollary}

\begin{proof}
    Kawauchi \cite[s. 158]{kawauchi96}.
\end{proof}

\begin{proposition}
    \label{prp:cobordant_to_algebraic_is_algebraic}
    Niech $K$ będzie węzłem zgodnym z węzłem algebraicznie plastrowym.
    Wtedy każda macierz Seiferta dowolnej powierzchni Seiferta $K$ jest zerowo kobordantna.
    W szczególności, $K$ jest węzłem algebraicznie plastrowym.
\end{proposition}

\begin{proof}
    Kawauchi \cite[s. 159]{kawauchi96}.
\end{proof}

% Theorem 1.3[Long 1984].A strongly positive amphicheiral knot is algebraicallyslice.
% Theorem 1.4[Hartley and Kawauchi 1979].If K is strongly positive amphicheiral,the Alexander polynomial1Kis the square of a symmetric polynomial.



\subsection{Węzły skręcone}
\index{węzeł!skręcony|(}%

% DICTIONARY;twist knot;węzeł skręcony
Węzły skręcone uważa się za najprostszą (po torusowych) rodzinę węzłów.

\begin{definition}
    \label{def:twist_knot}
    Węzeł powstały przez $n$-krotne półskręcanie domkniętej pętli oraz splecienie końców nazywamy węzłem skręconym.
\end{definition}

Węzły skręcone to dokładnie towarzyszące niewęzłowi w~węzłach satelitarnych, tak zwane whiteheadowskie duble niewęzła.
Wszystkie są odwracalne (ale tylko niewęzeł oraz ósemka są amfichiralne) i~mają liczbę gordyjską $1$, ponieważ wystarczy rozwiązać skrzyżowanie, które plotło końce.
\index{liczba gordyjska}%
Każdy jest dwumostowy (ćwiczenie w \cite[s. 114]{rolfsen76}) i~posiada zerową sygnaturę.
\index{węzeł!dwumostowy}%
\index{sygnatura}%
Dalsze własności węzłów skręconych zależą od $n$, ilości półskrętów.
Indeks skrzyżowaniowy wynosi $n + 2$.

\begin{proposition}
\index{wielomian!Conwaya}%
    Niech $K$ będzie węzłem $n$-skręconym.
    Wtedy
    \begin{equation}
    2 \conway (z) = \begin{cases}
        2 + (n+1) z^{2} & n \mbox{ nieparzyste} \\
        2 - nz^2 & n \mbox{ parzyste}
    \end{cases}
    \end{equation}
\end{proposition}

\begin{proposition}
\index{wielomian!Jonesa}%
    Niech $K$ będzie węzłem $n$-skręconym.
    Wtedy
    \begin{equation}
    (q+1)\jones(q) = \begin{cases}
        1+q^{-2}+q^{-n}-q^{-n-3} & n \mbox{ nieparzyste} \\
        q^3(1+q^{-2}-q^{-n}+q^{-n-3}) & n \mbox{ parzyste}
    \end{cases}
    \end{equation}
\end{proposition}

\begin{proposition}
\index{węzeł!plastrowy}%
    Niewęzeł oraz węzeł dokerski $6_1$ są jedynymi skręconymi węzłami plastrowymi.
\end{proposition}

\begin{proof}
    \cite{casson86}.
\end{proof}

\index{węzeł!skręcony|)}%

% koniec podsekcji Węzły skręcone


