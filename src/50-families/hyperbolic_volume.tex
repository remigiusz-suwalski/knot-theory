
\subsection{Objętość hiperboliczna}

\index{objętość|(}
\begin{definition}[objętość]
    Niech $L$ będzie splotem hiperbolicznym.
    Objętość dopełnienia $L$ względem zupełnej metryki hiperbolicznej nazywamy objętością splotu $L$ i~oznaczamy $\volume L$.
\end{definition}

Objętość jest zawsze skończoną liczbą rzeczywistą.
Dla wygody przyjmuje się czasami, że objętość węzłów torusowych oraz satelitarnych wynosi $0$.
Komputerowy program SnapPea napisany przez Weeksa pozwala na wyznaczenie objętości dowolnego splotu o~rozsądnej ilości skrzyżowań.
\index{SnapPea}

\begin{example}
    $\volume 4_1 = -6 \int_{0}^{\pi/3} \log |2\sin \theta| \,\mathrm{d}\theta \approx 2.0298832$.
\end{example}

Patrz też ciąg \href{https://oeis.org/A091518}{A091518} w~bazie danych OEIS.
Jak zobaczymy później, żaden węzeł nie ma mniejszej objętości.

\begin{example}
    $\volume 5_2 \approx 2.82812$.
\end{example}

W encyklopedii Wolfram Mathworld znajduje się informacja, że $5_2$ oraz pewien węzeł o~dwunastu skrzyżowaniach mają tę samą objętość, prawdopodobnie chodzi tu o~$12n_{242}$, który znany jest także jako $(-2, 3, 7)$-precel.
\index{precel!(-2, 3, 7)}

\begin{example}
    $\volume 6_1 \approx 3.16396$.
\end{example}

\begin{example}
    $\volume 6_2 \approx 4.40083$.
\end{example}

\begin{example}
    $\volume 6_3 \approx 5.69302$.
\end{example}

\begin{example}
    $\volume 7_4 \approx 5.13794$.
\end{example}

\begin{example}
    Niech $K$ będzie jednym z~dwóch węzłów w~parze Perko.
    Wtedy $\volume K \approx 5.63877$.
    \index{para Perko}
\end{example}

Praca \cite{purcell19} wspomina kilka przyjemnych ograniczeń, jakie musi spełniać objętość.
\index[persons]{Futer, David}%
\index[persons]{Kalfagianni, Efstratia}%
\index[persons]{Purcell, Jessica}%
Aby je przytoczyć, musimy najpierw zdefiniować dwie stałe: $v_4$ oraz $v_8$, odpowiednio objętość idealnego czworościanu\footnote{Albo rozmaitości Giesekinga, powstałej z czworościanu przez usunięcie  wierzchołków i sklejenie ściany 012 z 310 oraz 023 z 321. Dopełnienie ósemki jest podwójnym nakryciem tej rozmaitości.\index{rozmaitość Giesekinga}} oraz ośmiościanu foremnego w~$\mathbb H^3$.
Mamy
\begin{align}
    v_4 & = \int_{0}^{2\pi/3} \log(2 \cos(\theta/2)) \,\mathrm{d}\theta \approx 1.01494\,16064, \\
    % https://en.wikipedia.org/wiki/Gieseking_manifold
    v_8 & = 4 \sum_{n=0}^\infty \frac{(-1)^n}{(2n+1)^2} \approx 3.66386\,23767. % ... 08876060218414059729536443096597497126688537065 ... \ldots
\end{align}

I tak najpierw Adams pokazał w~swojej rozprawie doktorskiej \cite{adams83}:
\index[persons]{Adams, Colin}%

\begin{proposition}
    Niech $D$ będzie diagramem hiperbolicznego splotu o~$\crossing L \ge 5$ skrzyżowaniach.
    Wtedy
    \begin{equation}
        \volume L \le 4 (\crossing D - 4) v_4.
    \end{equation}
\end{proposition}

A trzy dekady później poprawił wswój wynik w~\cite{adams13}:

\begin{proposition}
    Niech $D$ będzie diagramem hiperbolicznego splotu o~$\crossing L \ge 5$ skrzyżowaniach.
    Wtedy
    \begin{equation}
        \volume L \le (\crossing D - 5) v_8 + 4v_4.
    \end{equation}
\end{proposition}

Jego metoda polega na podzieleniu dopełnienia splotu na czterościany i~ośmiościany oraz policzeniu ich.
To, w~połączeniu ze znanymi ograniczeniami na objętość ,,cegiełek'', wystarcza.
Podział na ośmiościany zaproponował Dylan (nie William!) Thurston.
% wiem to z purcell19

Thurston zauważył \cite[s. 365]{thurston82}, że tylko skończenie wiele hiperbolicznych 3-rozmaitości może mieć tę samą objętość -- wynika to z~prac Gromowa i~Jørgensena.
Następnie Wielenberg przedstawił w~\cite{wielenberg81} przykłady pokazujące, że istnieją dowolnie duże kolizje wśród węzłów hiperbolicznych: pewne podgrupy klasycznej grupy Picarda działają jako izometrie na górną półprzestrzestrzeń hiperboliczną wymiaru 3 mają podstawowe wielościany, które są takie same jako zbiory, ale różnią się jeśli chodzi o~utożsamienie ze sobą ścian.

Chociaż mutanty mają tę samą objętość hiperboliczną (fakt \ref{mutants_the_same_volume}), to praktyka pokazuje, że ten niezmiennik dobrze wspomaga proces tablicowania węzłów.

\begin{proposition}
    Zbiór
    \[
        \{\volume K: K \textrm{ jest hiperboliczny}\} \subseteq \R
    \]
    jest dobrze uporządkowany, typu porządkowego $\omega^\omega$.
\end{proposition}

\begin{proof}[Niedowód]
    Zdaniem angielskiej Wikipedii, dowód jest gdzieś w~\cite{neumann85} (gdzie Neumann znajduje eleganckie oszacowanie zmiany objętości po wykonaniu chirurgii Dehna), ja tego nie widzę.
    %=% wikipedia - angielski artykuł "hyperbolic volume" 
    Hodgson, Masa \cite{hodgson13} sugerują, że dowód da się znaleźć w notatkach Thurstona \cite{thurston02}.
\end{proof}

W dowolnej rodzinie węzłów istnieje element o~najmniejszej objętości.
Przytoczę teraz przykłady konkretnych rodzin i najmniejszych węzłów, za Futerem, Kalfagiannim, Purcell \cite[s. 16-17]{purcell19} oraz Hodgsonem, Masaiem\cite[s. 1-99]{hodgson13}.
\index[persons]{Futer, David}%
\index[persons]{Kalfagianni, Efstratia}%
\index[persons]{Purcell, Jessica}%
\index[persons]{Hodgson, Craig}%
\index[persons]{Masai, Hidetoshi}%

\begin{proposition}
%label{prp:eight_least_hyperbolic}
    Żaden węzeł nie ma mniejszej objętości hiperbolicznej od ósemki.
    \index{ósemka}
\end{proposition}

\begin{proof}
    Cao, Meyerhoff w~\cite{cao01} przeanalizowali pakowania horokul w~uniwersalnym nakryciu związanym z~rozmaitościami.
    Doszli do wniosku, że nie ma tam dostatecznieo wolnego miejsca, jeżeli szpic (cusp) nie jest odpowiedniego rozmiaru.
    Trzykrotnie wspierają się przy tym pomocą komputera, by sprawdzić, że określone warunki są spełnione we wszystkich punktach danej przestrzeni parametrów.
\end{proof}

\begin{proposition}
% DICTIONARY;cusped;szpiczasta;rozmaitość
% DICTIONARY;manifold;rozmaitość;-
\index{splot!Whiteheada}%
\index{ósemka}%
    Wśród orientowalnych 3-rozmaitości ze szpicem\footnote{rozmaitość szpiczasta -- niezwarte, zupełne hiperboliczne rozmaitości ze skończoną objętością Riemanna} najmniejszą objętość posiada dopełnienie ósemki oraz jego bliźniak, otrzymany przez $(5, 1)$-chirurgię jednego z~ogniw splotu Whiteheada.
% sformułowanie wygląda jak z "THE MINIMAL VOLUME ORIENTABLE HYPERBOLIC 3-MANIFOLD WITH 4 CUSPS"
\end{proposition}

Klasa rozmaitości wspomniana w fakcie obejmuje dopełnienia hiperbolicznych węzłów.
Powyższy fakt także został wzięty z~pracy \cite{cao01}.

Meyerhoff nie przestawał pracować nad rozmaitościami o~małych objętościach i~osiem lat później w~\cite{meyerhoff09} przedstawił z Gabaiem, Milleyem bez dowodu (obiecali pokazać go później):

\begin{proposition}
    Istnieje 10 orientowalnych 3-rozmaitości z~jednym szpicem o~objętości co najwyżej $2.848$: \texttt{m003}, \texttt{m004} ($\approx 2.02988$), \texttt{m006}, \texttt{m007} ($\approx 2.56897$), \texttt{m009}, \texttt{m010} ($\approx 2.66674$), \texttt{m011} ($\approx 2.78183$), \texttt{m015}, \texttt{m016} oraz \texttt{m017} ($\approx 2.82812$).
    Nazwy pochodzą ze spisu rozmaitości programu SnapPy.
\end{proposition}

Udało mi się rozszyfrować niektóre nazwy.
\texttt{m003} to siostra $4_1$, % https://hal.archives-ouvertes.fr/hal-02867890/document Michel Planat - Quantum computing thanks to Bianchi groups
\texttt{m004} to węzeł $4_1$, % SnapPy - also known as
% m006
% m007
% m009
% m010
% m011
\texttt{m015} to węzeł $5_2$,
\texttt{m016} to węzeł $12n242$, czyli znany nam już $(-2, 3, 7)$-precel,
\index{precel!(-2, 3, 7)}%
\texttt{m017} to siostra $5_2$. % https://arxiv.org/pdf/2107.03275.pdf

W tej samej pracy możemy jeszcze znaleźć informację, że:

\begin{proposition}
    Istnieje dokładnie jedna domknięta hiperboliczna 3-rozmaitość o najmniejszej objętości, rozmaitość Weeksa.
\end{proposition}

Rozmaitość Weeksa została odkryta przez Jeffreya Weeksa w jego rozprawie doktorskiej (1985) oraz niezależnie przez Matwiejewa, Fomenko (1988).
\index{rozmaitość Weeksa}
Powstaje ona przez wykonanie $(5, 2)$ oraz $(5, 1)$ chirurgii Dehna na dopełnieniu splotu Whiteheada, zaś jej objętość wynosi w~przybliżeniu $0.94270$. % https://oeis.org/A126774
\index{splot!Whiteheada}

Następna jest rozmaitość Meyerhoffa, powstała po $(5, 1)$ chirurgii na dopełnieniu ósemki.
\index{rozmaitość Meyerhoffa}
Meyerhoff sugerował w 1987, że ma najmniejszą objętość, ale okazało się potem, że ta wynosi $\approx 0.98136$.

\begin{proposition}
    Wśród orientowalnych 3-rozmaitości o~dwóch szpicach najmniejszą objętość mają splot Whiteheada oraz $(-2, 3, 8)$-precel.
\index{splot!Whiteheada}%
\index{precel!(-2, 3, 8)}%
\end{proposition}

% TODO: check cusped manifold in dictionary

Ich objętość wynosi $v_8$.

\begin{proof}
    Agol \cite{agol10} korzystając z metod topologicznych dowodzi istnienia ,,niezbędnej'' (z ang. essential) powierzchni, która zadaje dolne ograniczenie na objętość i skutecznie krępuje rozmaitości, które mogą to ograniczenie zrealizować.
\end{proof}

Przypadek trzech szpiców nie jest zbyt dobrze zrozumiany.

\begin{proposition}
    Wśród orientowalnych 3-rozmaitości o~czterech szpicach najmniejszą objętość posiada dopełnienie splotu $8_4^2$ wg numeracji Rolfsena (L8a13).
\end{proposition}

\begin{proof}
    Rozumowanie Yoshidy \cite{yoshida13} oparte o pracę Agola.
    Objętość splotu wynosi $2v_8$.
\end{proof}

\index{objętość|)}

