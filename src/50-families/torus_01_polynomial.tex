
\subsection{Niezmienniki liczbowe węzłów torusowych}
Podamy teraz wartości całkowitoliczbowych niezmienników dla węzłów torusowych przy założeniu, że $p$ lub $q$ nie jest zerem.
Nietrywialne węzły torusowe są pierwsze i~odwracalne, ale mają niezerową sygnaturę, więc nie są chiralne.
Wiedział to Schreier w 1924.
% TODO: \cite schreier24?

\begin{proposition}
\index{okres}%
    Węzeł torusowy $T_{p, q}$ ma okres $|p|$ oraz $|q|$.
\end{proposition}

\begin{proposition}
\index{sygnatura}%
    Niech $p, q > 0$ będą liczbami całkowitymi, zaś $R_2$ oznacza resztę z dzielenia przez dwa.
    Zdefiniujmy funkcję $\sigma(p, q) = - \sigma(T_{p, q})$.
    Spełnia zależność rekurencyjną
    \begin{equation}
        \sigma(p, q) = \begin{cases}
             q^2 + \sigma(p-2q, p) - R_2(p)       & \text{jeśli } 2q < p \\
             q^2 - 1                              & \text{jeśli } 2q = p \\
             q^2 - \sigma(2q - p, q) + R_2(r) - 2 & \text{jeśli } 2q > p > q \\
             q^2/2 + R_2(q)/2 - 1                 & \text{jeśli } p = q
             % czwarte stanowi algebraiczne przekształcenie trzeciego dla p >= q
        \end{cases}
    \end{equation}
    z warunkami brzegowymi: $\sigma(p, q) = \sigma(q, p)$, $\sigma(1, q) = 0$, $\sigma(2, q) = q-1$.
\end{proposition}

\begin{proof}[Niedowód]
\index[persons]{Litherland, Richard}%
\index[persons]{Gordon, Cameron}%
\index[persons]{Murasugi, Kunio}%
    Gordon, Litherland, Murasugi \cite[tw. 5.2]{litherland81} używają niezmiennika acyklicznego\footnote{Z angielskiego null-homologous, czyli o trywialnych zredukowanych grupach homologii.} splotu $L$ w zorientowanej 3-rozmaitości $M$ w połączeniu z jego $m$-krotnym rozgałęzionym nakryciem cyklicznym.
    Wspominają też, że można dowieść tego używając wzoru Hirzebrucha, ale nie robią tego.
\end{proof}

Borodzik niedawno przyjrzał się dokładniej sygnaturom węzłów torusowych.
\index[persons]{Borodzik, Maciej}%
W pracy \cite{borodzik10} napisanej z Oleszkiewiczem pokazał, że nie istnieje wymierna funkcja $R(p, q)$, która pokrywałaby się z sygnaturą węzła torusowego $T_{p, q}$ dla wszystkich względnie pierwszych, nieparzystych $p$ oraz $q$.
\index[persons]{Oleszkiewicz, Krzysztof}%

Uwaga: definicja funkcji $s$ z \cite{borodzik10} zawiera złośliwą literówkę.

\begin{proposition}
    Niech $p, q$ będą względnie pierwszymi liczbami, zaś $C \in [0, 1)$ stałą taką, że $Cpq$ nie jest liczbą całkowitą.
    Przyjmijmy $z = \exp (2 \pi i C)$ i zdefinujmy pomocnicze funkcje: niech $\{x\} = x - \lfloor x \rfloor$ oznacza część ułamkową, zaś
    \begin{equation}
        \langle x \rangle = \begin{cases}
            0 & \text{dla } x \in \Z \\
            \{x\} - 1/2 & \text{dla } x \not \in \Z
        \end{cases}
    \end{equation}
    funkcję piłę.
    Dalej, określmy sumę Dedekinda
    \begin{equation}
        s(p, q, x) = \sum_{j = 0}^{q-1} \left\langle \frac {j}{q} \right\rangle \left\langle \frac {jp}{q} + x \right\rangle.
    \end{equation}
    Przy tych oznaczeniach, sygnatura węzła $(p, q)$-torusowego wyznacza się wzorem
    \begin{align}
        \sigma(z) & = \frac{1}{3pq} \left (p^2 + q^2 + 6 \langle Cpq \rangle^2 - \frac {1}{2} \right)  + 2(C^2 - C) pq + (2-4C) \langle Cpq \rangle + {} \\
        & - 2s(p, q, Cp) - 2s(q, p, Cq) - 2s(p, q, p-pC) - 2s(q, p, q-qC). \nonumber
    \end{align}
\end{proposition}

\begin{corollary}
    Jeśli $p, q$ są nieparzyste i względnie pierwsze, to
    \begin{equation}
        \sigma(T_{p,q}) = \frac{1}{6pq} + \frac{2p}{3q} + \frac{2q}{3p} - \frac{pq}{2} - 4(s(2p, q, 0) + s(2q, p, 0)) - 1.
    \end{equation}
\end{corollary}

\begin{corollary}
    Jeśli $p$ jest nieparzyste, zaś $q > 2$ parzyste, to
    \begin{equation}
        \sigma(T_{p,q}) = - \frac{pq}{2} + 4s(2p, q, 0) - 8s(p, q, 0) + 1.
    \end{equation}
\end{corollary}

\begin{proposition}
\index{indeks skrzyżowaniowy}%
    Mamy $\crossing T_{p, q} = \min \{|pq| - |p|, |pq| - |q|\}$.
\end{proposition}

\begin{proof}
\index[persons]{Murasugi, Kunio}%
    Murasugi twierdzi, że udowodnił to w \cite{murasugi91}.
\end{proof}

Wyznaczenie indeksu rozwiązującego było dużo trudniejsze.
Murasugi pisze w~książce \cite{murasugi96}, że mamy nierówność
\begin{equation}
    u(T_{p, q}) \le \frac 12 (p-1)(q-1),
\end{equation}
z równością dla względnie pierwszych $p, q > 0$.
Hipoteza Milnora głosiła, że w~rzeczywistości równość zachodzi zawsze.
Dowód został odnaleziony w~latach 1993-1995 przy użyciu tzw. \emph{gauge theory} (działu teorii pola, gdzie lagranżjan jest niezmienniczy względem grup Liego lokalnych transformacji...).

\begin{proposition}
\index{liczba gordyjska}%
\label{prp:torus_unknotting_number}%
    Dla względnie pierwszych $p, q > 0$ mamy
    \begin{equation}
        \unknotting T_{p, q} = \frac 12 (p - 1)(q - 1),
    \end{equation}
\end{proposition}

% % Rasmussen podał nowy dowód hipotezy Milnora o plastrowym genusie węzłów torusowych, jest to pierwszy dowód który nie zależy od gauge theory.
% https://mathscinet.ams.org/mathscinet-getitem?mr=2729272

\begin{proof}
\index{hipoteza!Milnora}%
    ,,Jeśli $X$ jest jednospójną, gładką, domkniętą, zorientowaną 4-rozmaitością taką, że $b^+ < 3$ jest\footnote{wymiar maksymalnej dodatniej podprzedstrzeni dla formy przecięć (intersection form) drugiej homologii.} nieparzyste i taką, że wielomianowy niezmiennik Donaldsona jest nietrywialny, to genus każdej zorientowanej, gładko zanurzonej powierzchni $F$ (poza dwoma wyjątkami, których nie rozumiemy) spełnia nierówność $2g - 2 \ge F \cdot F$'' to stwierdzenie, jakie razem z~technicznymi lematami można znaleźć w \cite{kronheimer93}.
    Geometryczne elementy dowodu znalazły się w drugiej części, \cite{kronheimer95}.

    Wnioskiem z głównego twierdzenia jest dowód hipotezy Milnora.
    % This last result was proved by F. B. Kronheimer and T. S. Mrowka in [Kronheimer-Mrowka 1993], who determined the 4-dimensional genus of T(p, q) (defined in 12.3) by applying gauge theory to an embedded surface in a 4-manifold.
    % https://web.math.princeton.edu/~petero/GridHomologyBook.pdf strona 4
\end{proof}

Genus pokrywa się z~liczbą gordyjską dla węzłów torusowych, bo wyznacznik macierzy Seiferta jest niezerowy, więc genus to dokładnie stopień wielomianu Alexandera.

Patrz też \cite[s. 149]{murasugi96}.

\begin{proposition}
\index{liczba mostowa}%
\label{prp:torus_bridge_number}%
    $\bridge T_{p, q} = \min \{|p|, |q|\}$
\end{proposition}

Według Murasugiego dowód znalazł Schubert \cite{schubert54}.
\index[persons]{Schubert, Horst}%

\begin{corollary}
\index{indeks warkoczowy}%
\label{cor:torus_braid_number}%
    Niech $p, q \neq 0$ będą liczbami całkowitymi.
    Wtedy $\braid T_{p, q} = \min \{|p|, |q|\}$.
\end{corollary}

\begin{proof}
    Niech $K$ będzie węzłem torusowym typu $(p,q)$ z~minimalnym przedstawieniem jako warkocz $\beta$.
    Z konstrukcji domknięcia (czyli dołączenia rozłącznych półokręgów) wynika,
    że diagram $K$ ma dokładnie $b(K)$ lokalnych maksimów.
    Definicja liczby mostowej orzeka, iż $\bridge K \le \braid K$.
    Bez straty ogólności niech $p > q > 0$.
    Skoro węzeł $K$ powstaje z~$q$-warkocza $(\sigma_{q-1} \ldots \sigma_2\sigma_1)^p$,
    indeks $b(K)$ nie przekracza $q = br(K)$.
\end{proof}

