\subsection{Węzły taśmowe}
\begin{definition}
    \index{węzeł!taśmowy}
    Węzeł $K = f(S^1)$ będący brzegiem singularnego dysku $f \colon D \to S^3$ posiadającego następującą własność: każda przecinająca siebie składowa jest łukiem $A \subseteq f(D^2)$, dla którego $f^{-1}(A)$ składa się z~dwóch łuków w~$D^2$ (jeden z~nich jest wewnętrzny), nazywamy taśmowym.
\end{definition}

Jak pisze Kawauchi, mamy oczywiste wynikanie:

\begin{proposition}
    Każdy węzeł taśmowy jest plastrowy.
\end{proposition}

Dawno temu Fox zapytał, czy implikacja odwrotna także jest prawdziwa:

\begin{conjecture}[slice-ribbon problem]
    \index{hipoteza!plastrowo-taśmowa}
    Czy każdy węzeł plastrowy jest taśmowy?
\end{conjecture}

Nie wiemy do dzisiaj.
Lisca pokazał prawdziwość hipotezy dla węzłów 2-mostowych \cite{lisca07}, Greene oraz Jabuka zrobili to dla precli o trzech pasmach w \cite{greene11}.
\index{węzeł!dwumostowy}%
\index{węzeł!preclowy}%
P. Teichner myśli o niej jako o~życzeniu, które uprościłoby pewne 4-wymiarowe problemy, gdyby było prawdziwe, ale Gompf, Scharlemann i Thompson zasugerowali w~\cite{gompf10} potencjalny kontrprzykład.
