
\subsection{Wykrywanie niewęzła}
Wolfgang Haken \cite{haken61} podał przepis na wykrycie diagramu niewęzła i rozwiązał częściowo jeden z~ważniejszych problemów teorii węzłów.
\index{niewęzeł}
\index[persons]{Haken, Wolfgang}%
Długo nikt nie podjął się implementacji tego algorytmu\footnote{Moritz Epple pisze ,,this algorithm was extremely impractical'', w recenzji z MathSciNet proponuje, żeby przed przeczytaniem pełnej niepotrzebnych dygresji pracy Hakena poznać artykuł \cite{schubert61} Schuberta.}, udało się to Burtonowi, Budneyowi oraz Petterssonowi w~komputerowym programie Regina\footnote{Dostępny pod adresem \url{https://regina-normal.github.io/}.} na przełomie tysiącleci.
\index[persons]{Epple, Moritz}%
\index[persons]{Burton, Benjamin}%
\index[persons]{Budney, Ryan}%
\index[persons]{Pettersson, William}%
%=% https://mathscinet.ams.org/mathscinet-getitem?mr=141107
% DICTIONARY;incompressible;nieściśliwy;-
Burton, Rubinstein i~Tillman pokazali w~pracy \cite{burton12}, jak sprawdzać, czy powierzchnia normalna na striangulowanej 3-rozmaitości jest (nie)ściśliwa w~czasie wykładniczym.
\index[persons]{Rubinstein, Joachim}%
\index[persons]{Tillman, Stephan}%
To okazało się wystarczyć do udzielenia negatywnej odpowiedzi na pytanie Thurstona: ,,czy przestrzeń Seiferta-Webera jest rozmaitością Hakena?'', a zatem wykraczającego poza poziom tej pracy.
\index[persons]{Thurston, William}%
\index{przestrzeń!Seiferta-Webera}%
\index{rozmaitość!Hakena}%

SnapPea\footnote{Dostępny pod adresem \url{http://geometrygames.org/SnapPea/index.html}.} to inny popularny wśród niskowymiarowych topologów program pozwalający badać hiperboliczne 3-rozmaitości, patrz sekcja \ref{sec:hyperbolic}.

Wiadomo, że genus oraz zredukowana kohomologia Chowanowa wykrywa niewęzły (fakty \ref{prp:genus_detects_unknot}, \ref{khovanov_detects_unknot}) i nie wiadomo, czy wielomian Jonesa to robi (hipoteza \ref{con:jones}).
\index{genus}%
\index{homologia!Chowanowa}%
\index{wielomian!Jonesa}%
Od dawna wiadomo, że wielomian Alexandera nie wykrywa niewęzła (fakt \ref{alexander_no_detects_unknot}).
\index{wielomian!Alexandera}%
W lutym 2021 Lackenby ogłosił nowy algorytm rozpoznający niewęzły w~quasiwielomianowym czasie.
% i nie zrobił tego w THE EFFICIENT CERTIFICATION OF KNOTTEDNESS AND THURSTON NORM, bo to wyszło na arxiv w 2016
\index[persons]{Lackenby, Marc}%

Wyśmienitym punktem wyjścia do poszukiwań trudnych niewęzłów jest praca \cite{schleimer21}, dzieło Burtona, Changa, Löfflera, de Mesmaya, Marii, Schleimera, Sedgwicka oraz Spreera.
\index[persons]{Burton, Benjamin}%
\index[persons]{Chang, Hsien-Chih}%
\index[persons]{de Mesmay, Arnaud}%
\index[persons]{Löffler, Maarten}%
\index[persons]{Maria, Clément}%
\index[persons]{Schleimer, Saul}%
\index[persons]{Sedgwick, Eric}%
\index[persons]{Spreer, Jonatha}%
Cytuje ona artykuł Lackenby'a \cite{lackenby15}, gdzie poznaliśmy stary (z 1934 roku!) przykład Goeritza \cite{goeritz34} diagramu niewęzła o~11 skrzyżowaniach, który można zmienić w~zwykły diagram niewęzła tylko zwiększając po drodze liczbę skrzyżowań.
\index[persons]{Goeritz, Lebrecht}%
\index{niewęzeł!Goeritza}%
Na kolejne teksty przyszło poczekać ponad pół wieku.
Autorzy przywołują jeszcze klasyczny przykład Freedmana, He, Wanga \cite{freedman94}; ale też podstępne niewęzły Hakena, Ochiai, Thistlethwaite'a oraz mocno doświadczalną pracę Petronio, Zanellatiego \cite{zanellati16}.
% https://mickburton.co.uk/2015/06/05/how-do-you-construct-hakens-gordian-knot/
\index[persons]{Freedman, Michael}%
\index[persons]{He, Zheng-Xu}%
\index[persons]{Wang, Zhenghan}%
\index{niewęzeł!Freedmana}%
\index[persons]{Petronio, Carlo}%
\index[persons]{Zanellati, Adolfo}%

\begin{comment}
\begin{figure}[H]
    \begin{minipage}[b]{.32\linewidth}
        \centering
        \includegraphics[width=\linewidth]{../data/missing.jpg}
        \subcaption{normalny}
    \end{minipage}
    \begin{minipage}[b]{.32\linewidth}
        \centering
        \includegraphics[width=\linewidth]{../data/missing.jpg}
        \subcaption{Goritza}
    \end{minipage}
    \begin{minipage}[b]{.32\linewidth}
        \centering
        \includegraphics[width=\linewidth]{../data/missing.jpg}
        \subcaption{Freedmana}
    \end{minipage}
\end{figure}
\end{comment}
