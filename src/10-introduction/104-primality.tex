
% DICTIONARY;prime;pierwszy;węzeł
% DICTIONARY;composite;złożony;węzeł
\section{Węzły pierwsze}
\index{węzeł!pierwszy|(}%
Suma spójna jest dla węzłów tym, czym mnożenie dla liczb naturalnych.
Analogia ta nabiera sensu, gdy zdefiniujemy węzły pierwsze, odpowiedniki liczb pierwszych.
Do ich dobrego zrozumienia warto znać powierzchnie Seiferta (ale przy pierwszym czytaniu nie trzeba).

\begin{definition}[węzeł pierwszy]
\label{def:prime_knot}%
    Niech $K$ będzie węzłem różnym od niewęzła.
    Jeśli nie przedstawia się jako suma spójna $K_1 \shrap K_2$
    dwóch nietrywialnych węzłów $K_1, K_2$, nazywamy go węzłem pierwszym.
    W~przeciwnym razie mówimy, że jest złożony.
\end{definition}

Pokażemy później, że rozkład na węzły pierwsze istnieje i~wspomnimy, dlaczego jest jedyny.
Fakt \ref{prp:genus_of_sum} stanowi odpowiednik zasadniczego twierdzenia arytmetyki (i jest od niego trochę trudniejszy w~dowodzie).
Każdy węzeł jest sumą spójną siebie oraz niewęzła, dlatego byłoby miło, gdyby niewęzeł nie dał się zapisać jako suma dwóch innych węzłów.
Jest to dokładnie wniosek \ref{cor:connected_sum_no_inverses}: suma spójna nie posiada elementów odwrotnych.

% TODO: jak przedłużyć mimo niejednoznaczności na złożone?

Przesmyk to wąskie skrzyżowanie między dwiema rozłącznymi częśćmi diagramu.
\index{przesmyk}

\begin{proposition}
\index{splot!rozszczepialny}%
    Niech $L$ będzie alternującym splotem bez przesmyków.
    Jeśli diagram splotu $L$ jest spójny, to splot jest nierozszczepialny.
    Jeśli splot nie jest rozszczepialny, to jest też pierwszy jeśli dla każdego okręgu przecinającego diagram w dwóch niepodwójnych punktach, przekrój wnętrza okręgu z~diagramem jest łukiem.
\end{proposition}

Innymi słowy, jeśli alternujący splot jest złożony, widać to bezpośrednio na każdym jego alternującym diagramie.
Jako pierwszy pokazał to Menasco w~\cite{menasco84}.
Jego dowód opiera się na multiplikatywności wielomianu BLM/Ho.
\index{wielomian!BLM/Ho}

Czy węzłów pierwszych jest nieskończenie wiele?
Tak, patrz fakt \ref{prp:infinitely_many_prime_knots}, potrafimy nawet oszacować liczbę $K_n$ węzłów pierwszych oraz $L_n$ splotów pierwszych.
W roku 1987 C. Ernst, D. Sumners w~oparciu o~wyniki Thistlethwaite'a, Kauffmana, oraz Murasugiego dotyczące węzłów alternujących pokazali w~\cite{ernst87}, że $K_n \ge \frac 1 3 (2^{n- 2} - 1)$, przy czym węzły lustrzane traktowane są jako różne.
% te wyniki to hipotezy Taita po prostu XD
Dokładniej:

\begin{proposition}
\index{węzeł!dwumostowy}%
    Niech $f(n)$ oznacza liczbę węzłów dwumostowych o indeksie skrzyżowaniowym $n$.
    Wtedy
    \begin{equation}
        f(n) = \begin{cases}
        \frac 13 (2^{n-2} - 1) & \text{dla } n = 2k \ge 4 \\
        \frac 13 (2^{n-2} + 2^{(n-1)/2}) & \text{dla } n = 4k + 1 \ge 5 \\
        \frac 13 (2^{n-2} + 2^{(n-1)/2} + 2) & \text{dla } n = 4k + 3 \ge 7
        \end{cases}
    \end{equation}
\end{proposition}

Welsh rozpatruje w \cite{welsh92} węzły bez orientacji i znajduje poniższe ograniczenia:

\begin{proposition}
    Niech $K_n$ oznacza liczbę węzłów pierwszych o $n$ skrzyżowaniach.
    % TODO: n? co najwyżej n? coś innego?
    Wtedy
    \begin{equation}
        2.68 \le \liminf_{n \to \infty}  \sqrt[n]{K_n} \le \limsup_{n \to \infty} \sqrt[n]{L_n} \le \frac {27}{2}.
    \end{equation}
\end{proposition}

Pytanie, czy zwykłe granice istnieją, pozostaje otwarte.

\index{węzeł!pierwszy|)}

% koniec sekcji Węzły pierwsze

