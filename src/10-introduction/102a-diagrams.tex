Chociaż w~świetle definicji \ref{def:knot} węzły są pewnymi regularnymi podzbiorami przestrzeni $\R^3$, z~kombinatorycznego punktu widzenia wygodniej jest rysować je na płaszczyźnie.

% DICTIONARY;oriented;zorientowany;węzeł
\begin{definition}[orientacja]
\index{węzeł!zorientowany}%
\index{orientacja|see {węzeł zorientowany}}%
    Węzeł, w~którym wybrano kierunek, w~którym należy się po nim poruszać, nazywamy zorientowanym.
    Splot nazywamy zorientowanym, jeśli wszystkie jego ogniwa traktowane jako węzły są zorientowane.
\end{definition}

Orientację na diagramie zaznaczamy małą strzałką wskazującą kierunek poruszania się.

% DICTIONARY;shadow;cień;-
\begin{definition}
\index{cień}%
    Rzut węzła $K \subseteq \R^3$ na płaszczyznę nazywamy cieniem.
\end{definition}

% DICTIONARY;crossing;skrzyżowanie;-
\begin{definition}[skrzyżowanie]
\index{skrzyżowanie}%
    Podwójny punkt w cieniu nazywamy skrzyżowaniem.
\end{definition}

% DICTIONARY;diagram;diagram;-
\begin{definition}[diagram]
\index{diagram}%
    Cień razem z~informacją o~tym, jak przebiegają skrzyżowania i pozbawiony katastrof: potrójnych przecięć, stycznych czy dziobów nazywamy diagramem.
    % TODO: Narysować katastrofy
\end{definition}

\begin{definition}[włókno]
\index{włókno}%
    Fragment diagramu, który biegnie między dwoma kolejnymi tunelami, czyli podskrzyżowaniami, nazywamy włóknem.
\end{definition}

\begin{definition}[nić]
\index{nić}%
    Fragment diagramu, który biegnie między dwoma kolejnymi skrzyżowaniami, nazywamy nicią.
\end{definition}

Nici powstają z włókien przez rozcięcie ich przy każdym nadskrzyżowaniu.

\begin{proposition}
    Niech $L$ będzie splotem.
    Jego diagramy tworzą otwarty i~gęsty podzbiór wszystkich rzutów.
\end{proposition}

Kawauchi \cite[s. 7]{kawauchi96} wspomina w tym miejscu podręcznik Crowella, Foxa \cite[s. 7]{crowell63}.
To samo jest na przykład w~\cite[s. 10]{burde14}.

\begin{proof}
    Rzut splotu na równoległe płaszczyzny jest taki sam, a te można sparametryzować prostymi przechodzącymi przez początek układu współrzędnych, które tworzą przestrzeń rzutową $\R \mathbb P^2$.
    Niech $S$ będzie zbiorem prostych, które dają złe rzuty.
    Wystarczy pokazać jego nigdziegęstość.
    Okazuje się, że $S$ jest też jednowymiarowy.
\end{proof}

\begin{corollary}
    Każdy splot posiada diagram.
\end{corollary}

Każdy węzeł ma zatem wiele diagramów; mając dane dwa różne chcielibyśmy wiedzieć, czy przedstawiają ten sam węzeł.
Kurt Reidemeister podał proste kryterium rozstrzygające ten problem w~latach dwudziestych ubiegłego wieku.
Najpierw zdefiniujmy trzy lokalne operacje na diagramach.

% DICTIONARY;Reidemeister/Turaev/... move;ruch Reidemeistera/Turajewa/...;-
\begin{definition}[ruchy Reidemeistera]
\index{ruchy Reidemeistera}%
    Trzy gatunki lokalnych deformacji diagramu splotu:
    \begin{figure}[H]
    \centering
    %
    \begin{minipage}[b]{.3\linewidth}
        \[
            \LargeReidemeisterOneLeft \stackrel{R_1}{\cong} \LargeReidemeisterOneStraight
        \]
        \subcaption{ruch $R_1$}
    \end{minipage}
    %
    \begin{minipage}[b]{.3\linewidth}
        \[
            \LargeReidemeisterTwoA \stackrel{R_2}{\cong} \LargeReidemeisterTwoB
        \]
        \subcaption{ruch $R_2$}
    \end{minipage}
    %
    \begin{minipage}[b]{.35\linewidth}
        \[
            \LargeReidemeisterThreeA \stackrel{R_3}{\cong} \LargeReidemeisterThreeB
        \]
        \subcaption{ruch $R_3$}
    \end{minipage}
    \end{figure}
    skręcenie lub rozkręcenie ($R_1$), wsunięcie lub rozsunięcie ($R_2$) oraz przesunięcie łuku przez skrzyżowanie ($R_3$) nazywamy ruchami Reidemeistera.
\end{definition}

Ruch $R_i$ operuje więc na $i$ łukach diagramu.
Reidemeister w~swojej pierwszej pracy przyjął inną kolejność, jego drugi ruch jest naszym pierwszym.

\begin{theorem}[Reidemeister, 1927]
\label{thm:reidemeister}%
\index{twierdzenie!Reidemeistera}%
    Niech $D_1, D_2$ będą diagramami dwóch splotów $L_1, L_2$.
    Sploty $L_1, L_2$ są takie same wtedy i tylko wtedy, gdy diagram $D_2$ można otrzymać z $D_1$ wykonując skończony ciąg ruchów Reidemeistera oraz gładkich deformacji łuków, bez zmiany biegu skrzyżowań.
\end{theorem}

Dowód podali niezależnie Reidemeister \cite{reidemeister27} oraz Alexander, Briggs \cite{briggs27}.
Twierdzenie Reidemeistera jest prawdziwe także dla splotów zorientowanych, wtedy jednak w każdym ruchu trzeba uwzględnić wszystkie możliwe orientacje łuków.

\begin{proof}
    Szkielet dowodu można znaleźć w~książce Burdego i~Zieschanga \cite[s. 9-11]{burde14}.
    Kluczowe pomysły zawiera też \cite[s. 11-12]{prasolov97} Prasołowa i~Sosińskiego.
    Innym przystępnym źródłem jest podręcznik \cite[s. 50-56]{murasugi96} Murasugiego.
\end{proof}

Trace \cite{trace83} zauważył, że dwa diagramy jednego węzła są związane tylko II i III ruchem (ale nie I) wtedy i tylko wtedy, gdy mają ten sam spin oraz indeks punktu względem krzywej (,,winding number'').
Z prac Östlunda \cite{ostlund01}, Manturowa\footnote{Niestety nie wiem, o które strony tej książki chodzi.} \cite{manturov04} oraz Haggego \cite{hagge06} wynika, że dla każdego węzła istnieje para diagramów, do przejścia między którymi trzeba wykorzystać wszystkie trzy ruchy.
% praca Haggego nazywa się "Every Reidemeister move is needed for each knot type" ale nawet w MathSciNecie wspomnieni są Ostlund i Manturow, więc zostawiam. Tekst skopiowany z Wiki
Coward \cite{coward06} zademonstrował, że nawet jeśli wszystkie trzy ruchy są potrzebne, można je wykonywać w specjalnej kolejności: najpierw tylko I ruchy, potem tylko II ruchy, następnie tylko III ruchy i znowu II ruchy.

W praktyce twierdzenia \ref{thm:reidemeister} nie stosuje się bezpośrednio do diagramów splotów.
Mając dane dwa spójne diagramy tego samego splotu trudno znaleźć jest ciąg ruchów przekształcający jeden z nich w drugi.
Załóżmy, że widać na nich odpowiednio $n_1, n_2$ skrzyżowań.
Jak piszą Coward, Lackenby w \cite{coward11}, istnieje funkcja $f \colon \N \times \N \to \N$ taka, że między dwoma diagramami można przejść wykonując co najwyżej $f(n_1, n_2)$ ruchów.
Wynika to z oczywistego faktu, że istnieje skończenie wiele spójnych diagramów o danej liczbie skrzyżowań oraz twierdzenia Reidemeistera.
Okazuje się jednak, że od funkcji $f$ można żądać, by była obliczalna i faktycznie, główny wynik \cite{coward11} orzeka, że
\begin{equation}
    f(n_1, n_2) = 2^{2^{\ldots^{2^{n_1 + n_2}}}}
\end{equation}
jest taką funkcją.
Piętrowa potęga liczy sobie aż $10^{1000000 (n_1 + n_2)}$ warstw, ale przynajmniej jest jawnie zdefiniowana.
Natomiast jeżeli $n_2 = 0$, czyli drugi diagram przedstawia niewęzeł, wystarcza $(236n_1)^{11}$ ruchów, to świeższy wynik samego Lackenby'a \cite{lackenby15}.

\begin{tobedone}
    Przedstawić rozumowanie (piramidka z węzłami), dlaczego to nie jest takie oczywiste.
\end{tobedone}

Hayashi \cite{hayashi05} dowiódł, że liczbę ruchów Reidemeistera potrzebnych, by rozszczepić splot można ograniczyć z góry na podstawie indeksu skrzyżowaniowego.

% DICTIONARY;invariant;niezmiennik;-
Zamiast tego definiuje się niezmienniki, czyli funkcje ze zbioru wszystkich diagramów, które nie zmieniają swojej wartości podczas wykonywania ruchów Reidemeistera.
Kiedy pewien niezmiennik przyjmuje różne wartości na dwóch diagramach, te przedstawiają dwa istotnie różne sploty.
Gdy wartości są te same, nie dostajemy żadnej informacji.
Sploty mogą być równoważne albo nie.
Niezmienniki będą nam stale towarzyszyć w~wędrówce po krainie węzłów.

% koniec sekcji Ruchy Reidemeistera

Wolfgang Haken \cite{haken61} podał przepis na wykrycie diagramu niewęzła i rozwiązał częściowo jeden z~ważniejszych problemów teorii węzłów.
Długo nikt nie podjął się implementacji tego algorytmu\footnote{Moritz Epple pisze ,,this algorithm was extremely impractical'', w recenzji z MathSciNet proponuje, żeby przed przeczytaniem pełnej niepotrzebnych dygresji pracy Hakena poznać artykuł \cite{schubert61} Schuberta.}, udało się to Burtonowi, Budneyowi oraz Petterssonowi w~komputerowym programie Regina\footnote{Dostępny pod adresem \url{https://regina-normal.github.io/}.} na przełomie tysiącleci.
%=% https://mathscinet.ams.org/mathscinet-getitem?mr=141107
% DICTIONARY;incompressible;nieściśliwy;-
Burton, Rubinstein i~Tillman pokazali w~pracy \cite{burton12}, jak sprawdzać, czy powierzchnia normalna na striangulowanej 3-rozmaitości jest (nie)ściśliwa w~czasie wykładniczym.
To okazało się wystarczyć do udzielenia negatywnej odpowiedzi na pytanie Thurstona: ,,czy przestrzeń Seiferta-Webera jest rozmaitością Hakena?'', a zatem wykraczającego poza poziom tej pracy.
\index{przestrzeń Seiferta-Webera}%
\index{rozmaitość Hakena}%

SnapPea\footnote{Dostępny pod adresem \url{http://geometrygames.org/SnapPea/index.html}.} to inny popularny wśród niskowymiarowych topologów program pozwalający badać hiperboliczne 3-rozmaitości, patrz sekcja \ref{sec:hyperbolic}.

Wiadomo, że genus oraz zredukowana kohomologia Chowanowa wykrywa niewęzły (fakty \ref{prp:genus_detects_unknot}, \ref{khovanov_detects_unknot}) i nie wiadomo, czy wielomian Jonesa to robi (hipoteza \ref{con:jones}).
% TODO: wiadomo, że Alexandera nie wykrywa
% i nie zrobił tego w THE EFFICIENT CERTIFICATION OF KNOTTEDNESS AND THURSTON NORM, bo to wyszło na arxiv w 2016
\index{genus}%
\index{wielomian!Jonesa}%
W lutym 2021 Lackenby ogłosił nowy algorytm rozpoznający niewęzły w~quasiwielomianowym czasie.

Przykładami trudnych w~rozpoznaniu niewęzłów są: niewęzeł Goritza, Freedmana.
Więcej trudnych niewęzłów zawiera praca \cite{zanellati16} autorstwa Petronio oraz Zanellatiego.

\index{niewęzeł}
\index{niewęzeł!Goritza}
\index{niewęzeł!Freedmana}
\begin{comment}
\begin{figure}[H]
    \begin{minipage}[b]{.32\linewidth}
        \centering
        \includegraphics[width=\linewidth]{../data/missing.jpg}
        \subcaption{normalny}
    \end{minipage}
    \begin{minipage}[b]{.32\linewidth}
        \centering
        \includegraphics[width=\linewidth]{../data/missing.jpg}
        \subcaption{Goritza}
    \end{minipage}
    \begin{minipage}[b]{.32\linewidth}
        \centering
        \includegraphics[width=\linewidth]{../data/missing.jpg}
        \subcaption{Freedmana}
    \end{minipage}
\end{figure}
\end{comment}

Zanim opowiemy, jak dotąd przebiegała klasyfikacja węzłów o małej liczbie skrzyżowań, zdefiniujemy klasę splotów ze specjalnymi diagramami.

% DICTIONARY;alternating;alternujący;węzeł
\begin{definition}[alternacja]
\index{węzeł!alternujący}%
    Diagram splotu, gdzie podczas poruszania się wzdłuż każdego ogniwa nad- oraz podskrzyżowania mijane są naprzemiennie, nazywamy alternującym.
    Splot jest alternujący, jeśli posiada alternujący diagram.
\end{definition}

Około 1961 roku Fox zapytał ,,What is an alternating knot?''.
Szukano takiej definicji węzła alternującego, która nie odnosi się bezpośrednio do diagramów, aż w~2015 roku Joshua Greene podał geometryczną charakteryzację: nierozszczepialny splot w $S^3$ jest alternujący wtedy i tylko wtedy, gdy ogranicza dodatnią oraz ujemną określoną powierzchnię rozpinającą \cite{greene17}.
% definite spanning surface

Sundberg oraz Thistlethwaite pokazali w 1998 roku, że liczba splotów alternujących rośnie wykładniczo (\cite{sundberg98}):

\begin{proposition}
    Niech $a_n$ oznacza liczbę pierwszych, alternujących supłów o~$n$ skrzyżowaniach.
\index{supeł}%
    Wtedy
    \begin{equation}
        a_n \sim (3c_1/4\sqrt{\pi})n^{-5/2}\lambda^{n-3/2},
    \end{equation}
    gdzie zarówno $c_1$, pierwszy współczynnik rozwinięcia Taylora funkcji $\Phi(\eta)$ zdefiniowanej w \cite{sundberg98}, jak i $\lambda$ są jawnie znanymi stałymi:
    \begin{align}
        c_1 & = \sqrt{\frac{5^7 \cdot (21001 + 371 \sqrt{21001})^3}{2 \cdot 3^{10} \cdot (17 + 3\sqrt{21001})^5}} \\
        \lambda & = \frac {1}{40} (101 + \sqrt{21001})
    \end{align}
    Niech $A_n$ oznacza liczbę pierwszych, alternujących splotów o $n$ skrzyżowaniach.
    Wtedy $A_n \approx \lambda^n$, dokładniej: jeśli $n \ge 3$, to
    \begin{equation}
        \frac{a_{n-1}}{16n - 24} \le A \le \frac{a_n - 1}{2}.
    \end{equation}
\end{proposition}

Czasami będziemy używać słów przed ich zdefiniowaniem, tak jak uczyniliśmy tutaj: węzły pierwsze i~supły pojawiają się odpowiednio w definicjach \ref{def:prime_knot}, \ref{def:tangle}.
Książkę trzeba więc przeczytać co~najmniej dwa razy.

\begin{proposition}
    Niech $a_n$ oznacza liczbę pierwszych, alternujących supłów o~$n$ skrzyżowaniach.
    Wtedy funkcja tworząca $f(z) = \sum_n a_n z^n$ spełnia równanie
    \begin{equation}
    f(1+z) - f(z)^2 - (1+f(z))q(f(z)) -z - \frac{2z^2}{1-z} = 0,
    \end{equation}
    gdzie $q(z)$ jest pomocniczą funkcją
    \begin{equation}
        q(z) = \frac{2z^2 - 10z - 1 + \sqrt{(1-4z)^3}} {2(z+2)^3} - \frac{2}{1+z} -z + 2.
    \end{equation}
\end{proposition}

Powyższa ciekawostka także pochodzi z cytowanej wcześniej pracy \cite{sundberg98}.
