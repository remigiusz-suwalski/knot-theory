
\subsection{Hipotezy Taita}
\index{hipoteza!Taita|(}

\begin{conjecture}[I hipoteza Taita]
\index{indeks skrzyżowaniowy}%
\label{con:tait_1}%
    Zredukowany alternujący diagram splotu ma minimalny indeks skrzyżowaniowy.
\end{conjecture}

Najpierw znaleziono dowód korzystający z wielomianu Jonesa: dokonali tego w 1987 roku równocześnie Kauffman \cite{kauffman87}, Murasugi \cite{murasugi87} oraz Thistlethwaite \cite{thistlethwaite87}.
\index{człowiek!Kauffman, ?}%
\index{człowiek!Murasugi, Kunio}%
\index{człowiek!Thistlethwaite, ?}%
Trzydzieści lat później Greene zaprezentował geometryczne podejście do problemu w \cite{greene17}.
\index{człowiek!Greene, Joshua}%

\begin{conjecture}[II hipoteza Taita]
\index{spin}%
    Dwa zredukowane diagramy alternujące jednego węzła mają ten sam spin.
\end{conjecture}

Pierwsze dowody pochodzą znowu od Kauffmana \cite{kauffman87} oraz Thistlethwaite'a \cite{thistlethwaite87}.
\index{człowiek!Kauffman, ?}%
\index{człowiek!Thistlethwaite, ?}%
Dla niektórych II hipoteza brzmi inaczej (,,achiralny splot alternujący ma zerowy spin''), dla innych jest prostym wnioskiem z naszego sformułowania.

\begin{conjecture}[III hipoteza Taita]
\index{flype}%
    Niech $D_1, D_2$ będą zredukowanymi alternującymi diagramami zorientowanego pierwszego splotu.
    Wtedy diagram $D_2$ można otrzymać z~$D_1$ korzystając jedynie z~ruchu \emph{flype}.
\end{conjecture}

Trzecią hipotezę udowodnił Menasco wspólnie z~Thistlethwaitem, \cite{menasco93}.
\index{człowiek!Menasco, ?}%
\index{człowiek!Thistlethwaite, ?}%
Wynika z~niej, że dwa zredukowane diagramy alternujące tego samego węzła mają ten sam spin.
Nie jest prawdziwa dla niealternujących splotów, przez co w~tablicach węzłów tak długo mieliśmy duplikat -- parę Perko.
\index{para Perko}%

Czasami mówi się jeszcze o IV hipotezie: że zwierciadlane węzły mają parzysty indeks skrzyżowań.
\index{węzeł!zwierciadlany}
Ta okazała się fałszywa.

Przedstawimy ze szczegółami dowód pierwszej hipotezy w~sekcji \ref{sub:tait_conjectures} oraz wspomnimy krótko o technikach użytych w dowodach pozostałych trzech.

\index{hipoteza!Taita|)}

% koniec podsekcji Hipotezy Taita

