\subsection{Długość sznurowa} % (fold)
\label{sub:ropelength}
Długość sznurowa, z~angielskiego \emph{ropelength}, pochodzi z~fizycznej teorii węzłów, która bierze pod uwagę obiekty wykonane z~nieelastycznych materiałów

Długość sznurowa $L$ to stosunek długości węzła do jego grubości $\tau$ (mówimy, że węzeł $K$ jest grubości $\tau$, jeśli ma otoczenie rurowe bez samoprzecięć z~przekrojem poprzecznym o~promieniu $\tau$).
Przez wiele lat zastanawiano się, czy można zawiązać węzeł ze sznura o~długości jednej stopy i~promieniu jednego cala.
Nie jest to możliwe: rozumowanie oparte o~czterosieczne pokazuje, że długość sznurowa nietrywialnego węzła wynosi co najmniej $15.66$ (dla trójlistnika jest to co najmniej $16.372$).

Węzeł realizujący długość sznurową jest klasy $C^1$.

Prawdziwe są oszacowania asymptotyczne:
\[
    L = \Omega (\operatorname{cr}^{3/4}),  \quad
    L = O(\operatorname{cr} \log^5 \operatorname{cr})
\]

% Koniec podsekcji Ropelength