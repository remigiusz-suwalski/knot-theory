\subsection{Liczba patykowa}
\index{liczba patykowa|(}

% DICTIONARY;stick number;liczba patykowa;-

\begin{definition}
    Minimalną liczbę odcinków w~łamanej, która przedstawia węzeł $K$, nazywamy jego liczbą patykową i~oznaczamy $\operatorname{s}(K)$.
\end{definition}

Wielkość tę wprowadził do matematyki Randell w~\cite{randell98} i~znalazł dokładną jej wartość dla niewęzła (3), trójlistnika (6) oraz ósemki (7).
Negami trzy lata później w~\cite{negami91} pokazał przy użyciu teorii grafów, że dla nietrywialnych węzłów prawdziwe są nierówności
\begin{equation}
    \frac{5+\sqrt{9 + 8 \crossing K}}{2} \le \operatorname{s} K \le 2 \crossing K.
\end{equation}

Trójlistnik to jedyny węzeł realizujący górne ograniczenie.
% Huh, Oh 2011 z trójlistnikiem?
Z~pracy Elrifaia \cite{elrifai06} (a wiemy o~niej z~\cite[s. 1]{huh11}) wynika, że dla węzłów o~co najwyżej 26 skrzyżowaniach, dolne ograniczenie jest ostre: można pisać $<$ w~miejsce $\le$.

Jin oraz Kim w 1993 ograniczyli liczby patykowe dla węzłów torusowych korzystając z~liczby supermostowej.
Wkrótce wynik został poprawiony przez samego Jina, w pracy \cite{jin97} znalazł dokładne wartości dla niektórych węzłów.
I~tak, jeśli $2 \le p < q < 2p$, to $\operatorname{s} T_{p,q} = 2q$ oraz $\operatorname{s} T_{p, 2p} = 4p-1$.
Ten sam wynik, choć dla węższego zakresu parametrów, odkryto w~\cite{greilsheimer97}.
\index{suma spójna}%
Autorzy niezależnie od siebie znaleźli proste oszacowanie z góry dla liczby patykowej sumy spójnej:
\begin{equation}
    \operatorname{s}(K_1 \shrap K_2) \le \operatorname{s}(K_1) + \operatorname{s}(K_2) - 3.
\end{equation}

Koniec dekady przyniósł jeszcze jedną pracę McCabe'a z nierównością $\operatorname{s}(K) \le 3 + \crossing (K)$ dla węzłów dwumostowych (\cite{mccabe98}) oraz odkrycie Calvo: jeśli ograniczymy się do łamanych o co najwyżej siedmiu odcinkach, ósemka przestaje być odwracalna.

Na początku XXI wieku nierówności Negamiego poprawiono, z dołu dokonał tego Calvo w~\cite{calvo01}, z góry natomiast Huh, Oh w \cite{huh11}.
% huh11: simple and self-contained
Górne ograniczenie można zmniejszyć o $3/2$, jeżeli $K$ jest niealternującym węzłem pierwszym.
\begin{equation}
    \frac{7+\sqrt{1 + 8 \crossing K}}{2} \le \operatorname{s} K \le \frac{3}{2} (1 + \crossing K).
\end{equation}

Liczba patykowa nie pojawia się już nigdzie w następnych rozdziałach.

\index{liczba patykowa|)}

% Koniec podsekcji Liczba patykowa
