
\subsection{Niezmienniki Wasiljewa wyższych rzędów}
(Na podstawie początku pracy Czmutowa \cite{chmutov12}).
\index[persons]{Czmutow, Siergiej}%
Niech $\mathcal V_n$ będzie zbiorem niezmienników Wasiljewa rzędu co najwyżej $n$, o~wartościach w zbiorze liczb zespolonych $\C$.
Z definicji \ref{def:vassiliev_order} wynika, że $\mathcal V_n$ jest przestrzenią wektorową nad ciałem $\C$ oraz $\mathcal V_n \subseteq \mathcal V_{n+1}$ i mamy rosnącą filtrację
\begin{equation}
    \mathcal V_0 \subseteq \mathcal V_1 \subseteq \mathcal V_2 \subseteq \ldots \subseteq \mathcal V := \bigcup_{n=0}^\infty \mathcal V_n.
\end{equation}

Oznaczmy wymiar przestrzeni $\mathcal V_n / \mathcal V_{n-1}$ przez $d_n$.
Dla wyższych rzędów nie dość, że nie znamy dokładnych wartości ciągu $d_n$, to dolne i górne ograniczenia asymptotyczne są od siebie bardzo różne: górne jest niemalże silnią, dolne natomiast jest podwykładnicze.

% https://people.math.osu.edu/chmutov.1/talks/2015/talk-KinW-XL-2015.pdf ? strona 22
% Chmutov, Duzhin. Mostovoy - Introduction to Vassiliev Knot Invariants, strona 432
\begin{proposition}
    $d_n < (2n-1)!!$.
\end{proposition}

\begin{proof}
%=% chmutovduzhin94 wspomina, że:
    Na tyle sposobów można narysować $n$ cięciw wewnątrz okręgu.
\end{proof}

\begin{proposition}
    $d_n < (n-1)!$.
\end{proposition}

\begin{proof}
\index[persons]{Czmutow, Siergiej}%
\index[persons]{Dużin, Siergiej}%
\index{diagram!z kręgosłupem}%
    Czmutow, Dużin \cite{chmutovduzhin94} zauważają, że jeśli $\Delta_n$ oznacza przestrzeń generowaną przez wszystkie diagramy cięciw modulo relacje 1T i 4T, to $\dim \mathcal V_n/\mathcal V_{n-1} \le \dim \Delta_n$.
    Wprowadzają diagramy z kręgosłupem: cięciwą, która tnie wszystkie inne cięciwy, po czym pokazują, że każdy diagram jest równoważny diagramowi z kręgosłupem.
\end{proof}

\begin{proposition}
    $d_n < \frac 12 (n-2)!$.
\end{proposition}

\begin{proof}
\index[persons]{Ng, Ka}%
    Ng \cite{ng98} jako produkt uboczny: klasy abstrakcji węzłów, które nie są odróżnialne niezmiennikami Wasiljewa rzędu co najwyżej $n$ tworzą wolną grupę abelową, zawiera ona pewną podgrupę węzłów taśmowych indeksu 2.
    % https://mathscinet.ams.org/mathscinet-getitem?mr=1681654
\end{proof}

\begin{proposition}
    Ciąg $d_n$ rośnie wolniej niż $n! \cdot (11/10)^n$.
\end{proposition}

\begin{proof}
\index[persons]{Stojmenow, Aleksander}%
    Stojmenow \cite{stoimenow98}.
\end{proof}

\begin{proposition}
    $d_n \lesssim n! / (2 \log 2 + O(1))^n$.
\end{proposition}

\begin{proof}
\index[persons]{Bollobás, Béla}%
\index[persons]{Riordan, Oliver}%
    Bollobás, Riordan \cite{bollobas00}.
\end{proof}

\begin{proposition}
    Niech $a < \frac 1 6 \pi^2$ będzie stałą.
    Wtedy
    \begin{equation}
        \dim \mathcal V_n / \mathcal V_{n-1} \lesssim \frac{n!}{a^n}.
    \end{equation}
\end{proposition}

\begin{proof}
\index[persons]{Zagier, Don}%
    Zagier \cite{zagier01} znalazł to ograniczenie przy użyciu szeregów Dirichleta.
\end{proof}

Zanim przejdziemy do ograniczeń z dołu, zdefinujmy jeszcze jedną przestrzeń, $\mathcal P_n \subseteq \mathcal V_n$.
Składa się z~tych niezmienników Wasiljewa, które są jednocześnie morfizmami, to znaczy spełniają równość $v(K_1 \shrap K_2) = v(K_1) + v(K_2)$.
Każdy niezmiennik jest wielomianową kombinacją niezmienników pierwotnych (elementów $\mathcal P_n$).

% https://people.math.osu.edu/chmutov.1/talks/2015/talk-KinW-XL-2015.pdf ? strona 32
% Chmutov, Duzhin. Mostovoy - Introduction to Vassiliev Knot Invariants, strona 434
\begin{proposition}
    $\dim \mathcal P_n \ge 1$.
\end{proposition}

\begin{proof}
\index[persons]{Czmutow, Siergiej}%
\index[persons]{Dużin, Siergiej}%
\index[persons]{Lando, Siergiej}%
    Czmutow, Dużin, Lando \cite{duzhin94}.
\end{proof}

\begin{proposition}
    $\dim \mathcal P_n \ge [n/2]$.
\end{proposition}

\begin{proof}
\index[persons]{Czmutow, Siergiej}%
\index[persons]{Melvin, Paul}%
\index[persons]{Morton, Hugh}%
\index[persons]{Warczenko, Aleksander}%
    Melvin, Morton \cite{melvin95}, Czmutow, Warczenko \cite{varchenko97}.
\end{proof}

\begin{proposition}
    $\dim \mathcal P_n \gtrsim \frac{1}{96} n^2$.
\end{proposition}

\begin{proof}
\index[persons]{Czmutow, Siergiej}%
    Czmutow \cite{duzhin96}.
\end{proof}

\begin{proposition}
    $\dim \mathcal P_n \gtrsim n^{\log_b n}$ dla $b > 4$.
\end{proposition}

\begin{proof}
\index[persons]{Czmutow, Siergiej}%
\index[persons]{Dużin, Siergiej}%
    Czmutow, Dużin \cite{duzhin99}.
\end{proof}

\begin{proposition}
    $\dim \mathcal P_n \gtrsim \exp (\pi \sqrt{n/3})$.
\end{proposition}

\begin{proof}
\index[persons]{Czmutow, Siergiej}%
\index[persons]{Koncewicz, Maksim}%
    Koncewicz w faksie do Czmutowa z 1997 roku. :)
\end{proof}

\begin{proposition}
    $\dim \mathcal P_n \gtrsim \exp (c \sqrt{n})$ dla każdej stałej $c < \pi \sqrt{2/3}$.
\end{proposition}

\begin{proof}
\index[persons]{Dasbach, Oliver}%
    Dasbach \cite{dasbach00} pracuje z~algebrą skończonych grafów, których wierzchołki są stopnia 1 lub 3, modulo relacje IHX i~antysymetrii.
    Pojawiają się też układy ciężarów pochodzące od algebry Liego $\mathfrak{gl}(N)$, dzięki którym można użyć pewnych dobrze znanych wyników z teorii algebr Liego i~teorii liczb.
\end{proof}

Praca \cite{dasbach00} ma tylko 11 stron, ale jest zamknięta w sobie i (zdaniem Birman) przykładem czytelności.
Ograniczenie Dasbacha pozostaje najlepsze (stan na 2011 rok).

\begin{corollary}
    Niech $a < \frac 1 6 \pi^2$ będzie stałą.
    Wtedy
    \begin{equation}
        \exp \left(\frac {n}{\log_a n} \right) \lesssim \dim \mathcal V_n / \mathcal V_{n-1}.
    \end{equation}
\end{corollary}

\begin{proof}
\index[persons]{Dasbach, Oliver}%
    Dasbach w \cite{dasbach00}.
\end{proof}

% Chmutov, Duzhin. Mostovoy - Introduction to Vassiliev Knot Invariants, strona 432
Dokładny wymiar przestrzeni $\mathcal V_n$ jest znany tylko dla $n \le 12$.
Poniższa tabela ma dość ciekawą historię.
Wasiljew znalazł ręcznie wartości w kolumnach dla $n \le 4$ w 1990 roku.
\index[persons]{Wasiljew, Wiktor}%
Potem Bar-Natan napisał komputerowy program rozwiazujący pewne równania liniowe i~znalazł tak wymiary przestrzeni $\mathcal V_n$ dla $n \le 9$, miało to miejsce w roku 1993.
\index[persons]{Bar-Natan, Dror}%
Wreszcie Kneissler cztery lata później znalazł dolne oraz górne ograniczenia: dolne oparte o znaczone powierzchnie, górne pochodzące od algebry Vogela (\cite{kneissler97}).
\index[persons]{Kneissler, Jan}%
\index{algebra Vogela}%
Dla $n \le 12$ ograniczenia te pokrywają się!

% Chmutov, Duzhin. Mostovoy - Introduction to Vassiliev Knot Invariants, strona 432
{
\renewcommand*{\arraystretch}{1.4} % bez tego wiersze mają minimalną wysokość, by pomieścić litery = ciasno
\footnotesize
\begin{longtable}{lcccccccccccccc}
\hline
    $n$ & $0$ & $1$ & $2$ & $3$ & $4$ & $5$ & $6$ & $7$ & $8$ & $9$ & $10$ & $11$ & $12$ \\ \hline \endhead
    $\dim \mathcal V_n$ & $1$ & $1$ & $2$ & $3$ & $6$ & $10$ & $19$ & $33$ & $60$ & $104$ & $184$ & $316$ & $548$ \\
    $\dim \mathcal V_n / \mathcal V_{n-1}$ & $1$ & $0$ & $1$ & $1$ & $3$ & $4$ & $9$ & $14$ & $27$ & $44$ & $80$ & $132$ & $232$ \\
    \hline
\end{longtable}
\normalsize
}

% kneissler97 podaje inny ciąg: 0, 1, 1, 2, 3, 5, 8, 12, 18, 27, 39, 55... (rk Pm), nasz nazywając (rk Am / rk Arm)
% Am: Z<circle diagrams of degree m> / Z<STU relations>
% Arm: Am / Z<FI relations>
% Pm: podmoduł Am generowany przez spójne diagramy

