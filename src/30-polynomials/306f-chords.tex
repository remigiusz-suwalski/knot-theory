
\subsection{Diagramy cięciw, układy ciężarów, algebra chińskich znaków}

Okazuje się, że wartość niezmiennika Wasiljewa $v$ nie zależy wprost od tego, jak zaplątany jest węzeł osobliwy $K$, ale od tego, jak ułożone są wierzchołki wzdłuż węzła. Standardową metodą kodowania tej informacji jest diagram cięciw.

\begin{definition}[diagram cięciw]
% DICTIONARY;chord;cięciw;diagram
\index{diagram cięciw}%
    Zorientowany okrąg razem z~$2n$ punktami leżącymi na nim (oraz~połączonymi w pary) z~dokładnością do zachowujących orientację homeomorfizmów nazywamy diagramem cięciw rzędu $n$, albo stopnia $n$.
\end{definition}

\begin{comment}
\begin{figure}[H]
    \centering
    \begin{minipage}[b]{.18\linewidth}
        \[\LargeChordDiagramA\]
        \subcaption{}
    \end{minipage}
    \begin{minipage}[b]{.18\linewidth}
        \[\LargeChordDiagramB\]
        \subcaption{}
    \end{minipage}
    \begin{minipage}[b]{.18\linewidth}
        \[\LargeChordDiagramC\]
        \subcaption{}
    \end{minipage}
    \begin{minipage}[b]{.18\linewidth}
        \[\LargeChordDiagramD\]
        \subcaption{}
    \end{minipage}
    \begin{minipage}[b]{.18\linewidth}
        \[\LargeChordDiagramE\]
        \subcaption{}
    \end{minipage}
    \caption{Wszystkie pięć diagramów cięciw stopnia 3}
\end{figure}
\end{comment}

Jak zamienić węzeł osobliwy w diagram cięciw?
Wybierzmy dowolny punkt na węźle, różny od wierzchołka i~przemierzmy węzeł.
Mijanym skrzyżowaniom przypiszmy liczby $1, 2, \ldots, 2n$.
Następnie na okręgu zaznaczmy kolejno te same punkty $1, 2, \ldots 2n$.
Wreszcie połączmy ze sobą liczby, które występują na tych samych skrzyżowaniach.

% TODO: wstawić obrazek.

\begin{proposition}
    Niech $K_1, K_2$ będą dwoma osobliwymi węzłami o~tym samym diagramie cięciw, zaś $v$~niezmiennikiem Wasiljewa.
    Wtedy $v(K_1) = v(K_2)$.
\end{proposition}

\begin{proof}
    Umieśćmy węzły osobliwe $K_1, K_2$ w przestrzeni tak, by ich wierzchołki oraz obie gałęzie wychodzące z wierzchołków leżały tak samo. Wtedy można tak zdeformować łuki $K_1$ tak, by jedynymi osobliwościami, jakie się pojawią lub znikną, były podwójne punkty.
    Teraz kłębiasta Wasiljewa mówi, że wartość $v$ nie zmienia się podczas tego procesu, zatem $v(K_1) = v(K_2)$, co należało okazać.
    % chmutov12
    % TODO: (\cite{duzhin12}, prop. 3.4.2)
\end{proof}

\begin{definition}[symbol niezmiennika]
\index{symbol niezmiennika (osobliwego)}%
    Niech $v$ będzie niezmiennikiem Wasiljewa.
    Obcięcie $v$ do zbioru węzłów osobliwych o~dokładnie $n$ wierzchołkach traktowane jako funkcja ze zbioru diagramów cięciw nazywamy symbolem tego niezmiennika.
\end{definition}

Jeśli $v_1, v_2$ są niezmiennikami Wasiljewa rzędu co najwyżej $n$ o~tych samych symbolach, to ich różnica jest niezmiennikiem rzędu co najwyżej $n - 1$.
Oznacza to, że przestrzeń $\mathcal V_n/\mathcal V_{n-1}$ pokrywa się z przestrzenią wszystkich symboli niezmienników Wasiljewa rzędu co najwyżej $n$.
Zbiór diagramów cięciw rzędu $n$ jest skończony, więc przestrzeń funkcji na tym zbiorze też jest skończona, a zatem przestrzenie $\mathcal V_n$ są skończonego wymiaru.

Stojmenow w \cite{stoimenow001} znalazł jakościowy wynik bez praktycznego znaczenia (ponieważ już dla $k = 4$ musielibyśmy znać wszystkie węzły o 36 skrzyżowaniach, a~jeszcze ich nie znamy, stan na 2021 rok).
\index[persons]{Stojmenow, Aleksander}%
Dokładniej:

\begin{proposition}
    Niezmiennik Wasiljewa rzędu co najwyżej $k$ jest jednoznacznie określony przez wartości, jakie przyjmuje na alternujących węzłach o co najwyżej $2k^2 + k$ skrzyżowaniach.
\end{proposition}

Symbol nie jest byle jaką funkcją, spełnia dwie relacje:
\index{relacje 1T i 4T}%
\begin{comment}
\begin{figure}[H]
    \[
        \LargeOneTerm \mapsto 0
    \]
    \caption{Relacja ,,one-term'' (1T albo FI?)}
\end{figure}
oraz
\begin{figure}[H]
    \[
        \LargeFourTermA - \LargeFourTermB + \LargeFourTermC - \LargeFourTermD \mapsto 0.
    \]
    \caption{Relacja ,,four-term'' (4T)}
\end{figure}
\end{comment}

Diagramy mogą mieć więcej cięciw z końcami tam, gdzie linia jest kropkowana, natomiast wszystkie końce cięciw na czarnych, pogrubionych łukach zostały zaznaczone explicite.

Zastosowaliśmy tutaj mały skrót dla oszczędności miejsca: oczywiście nie umiemy jeszcze odejmować od siebie diagramów, dlatego powyższe relacje należy rozumieć tak, że na każdym diagramie liczymy symbol niezmiennika i porównujemy tak otrzymane liczby zespolone.

\begin{definition}[układ ciężarów]
% DICTIONARY;weight system;układ ciężarów;-
\index{układ ciężarów}%
    Funkcję określoną na zbiorze diagramów $n$ cięciw, która spełnia relacje 1T oraz 4T, nazywamy układem ciężarów.
\end{definition}

Okazuje się, że wszystkie zależności, jakie występują między niezmiennikami Wasiljewa, są konsekwencjami relacji 1T oraz 4T.
Mówi o~tym głębokie twierdzenie Koncewicza:

\begin{proposition}
    Każdy układ ciężarów jest symbolem pewnego niezmiennika Wasiljewa. % rzędu co najwyżej $n$ - nie mieści się, przenosi samo $n$ do nowej linii.
\end{proposition}

\begin{proof}
    Koncewicz w \cite{kontsevich93}. % chmutow12/chmutov11 theorem 3.4
\end{proof}

\begin{definition}[chiński znak]
    Spójny graf złożony z pojedynczego zorientowanego okręgu oraz pewnej liczby niezorientowanych, kreskowanych linii, które mogą się spotykać w~jednym z dwóch typów wierzchołków:
    \begin{itemize}
        \item wewnętrznych wierzchołkach, gdzie spotykają się trzy kreskowane linie;
        \item zewnętrznych wierzchołkach, gdzie kreskowane linie kończą się na okręgu.
    \end{itemize}
    Wierzchołki wewnętrzne są zorientowane, zgodnie lub przeciwnie do ruchu wskazówek zegara.
\end{definition}

Diagramy cięciw modulo relacja 4T jest tym samym, co algebra chińskich znaków modulo relacja STU.
% DICTIONARY;algebra of Chinese characters;algebra chińskich znaków;-
\index{algebra!chińskich znaków}%
W~tej drugiej spełnione są jeszcze relacje AS oraz IHX, nie mam siły tego rysować, ale wszystko można znaleźć w pracy Bar-Natana \cite{barnatan_95}.
\index[persons]{Bar-Natan, Dror}%

\begin{tobedone}
% DICTIONARY;actuality table;tablica rzeczywistości;-
    \index{tablica rzeczywistości?}
    Actuality tables.
\end{tobedone}

