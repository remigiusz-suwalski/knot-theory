
\section{Wielomian Kauffmana}
\index{wielomian!Kauffmana|(}%
Mniej więcej w~tym samym czasie, gdy odkryto wielomian BLM/Ho, Kauffman opisał w~\cite{kauffman90} sposób, jak uogólnić ten niezmiennik do odróżniającego lustra.
\index[persons]{Kauffman, Louis}%
Wielomianu $F$ Kauffmana nie należy mylić z~klamrą Kauffmana $\langle \ldots \rangle$!

\begin{definition}[wielomian Kauffmana]
    Niech $L$ będzie zorientowanym splotem, zaś $D$ ustalonym diagramem o~spinie $\writhe D$.
    Istnieje wielomian $\Lambda(L)$ wyznaczony przez relację kłębiastą
    \begin{equation}
\begin{comment}
        \Lambda \left(\MediumPlusCrossing\right) +
        \Lambda \left(\MediumMinusCrossing\right) =
        z \cdot \left(
        \Lambda \left(\MediumBetaSmoothing\right) +
        \Lambda \left(\MediumAlphaSmoothing\right)
        \right)
\end{comment}
        ,
    \end{equation}
    który jest niezmienniczy względem II i III ruchu Reidemeistera, spełnia równości:
\begin{comment}
    \begin{equation}
        \Lambda \left(\MediumReidemeisterOneRight\right) =
        a \Lambda \left(\MediumReidemeisterOneStraight\right),
        \quad\quad\quad
        \Lambda \left(\MediumReidemeisterOneLeft\right) =
        \frac 1 a \Lambda \left(\MediumReidemeisterOneStraight\right)
    \end{equation}
\end{comment}
    oraz warunek brzegowy $\Lambda(\SmallUnknot) = 1$.
    Wtedy wielomian dwóch zmiennych
    \begin{equation}
        F_L(a, z) = a^{-\writhe D} \Lambda(D),
    \end{equation}
    nazywamy wielomianem Kauffmana.
    Jest niezmiennikiem splotów.
\end{definition}

Jego związki z~wielomianem HOMFLY pozostają nieznane.
Wiemy natomiast, że

\begin{proposition}
\index{wielomian!BLM/Ho}%
    Wielomian Kauffmana uogólnia wielomian BLM/Ho, zgodnie z podstawieniem
    \begin{equation}
        Q(x) = F(1, x).
    \end{equation}
\end{proposition}

\begin{proposition}
\index{wielomian!Jonesa}%
    Wielomian Kauffmana uogólnia wielomian Jonesa, zgodnie z podstawieniem
    \begin{equation}
        \jones(t)=F(-t^{-3/4},t^{-1/4}+t^{1/4}).
    \end{equation}
\end{proposition}

Rozpatrzmy relację kłębiastą $\Lambda_+ - \Lambda_- = x(\Lambda_0 - \Lambda_\infty)$.
Prowadzi ona do wielomianu ,,z~Dubrownika'': Kauffman na pocztówce napisanej do Lickorisha z Dubrownika w~1985 roku opisał ten wielomian sądząc, że jest to nowy, niezależny od $F$ niezmiennik.
\index{wielomian!z Dubrownika}%
\index{wielomian!Kauffmana|)}

% Koniec sekcji Wielomian Kauffmana

