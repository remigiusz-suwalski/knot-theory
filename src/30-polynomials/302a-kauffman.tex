\subsection{Definicja kombinatoryczna -- klamra Kauffmana}
\index{klamra Kauffmana|(}
Klamra Kauffmana to wielomian Laurenta jednej zmiennej zdefiniowany w pracy \cite{kauffman87} z 1987 roku, oparty na ruchach Reidemeistera.
Dzięki swojej prostocie mógł być odkryty na początku XX wieku, nim jeszcze maszyneria teorii węzłów została rozwinięta.

Poszukujemy niezmiennika dla splotów o~kilku prostych własnościach.
Przede wszystkim żądamy, by niewęzłowi przypisany był wielomian $1$: $\bracket{\SmallUnknot} = 1$.
Po drugie chcemy wyznaczać nawiasy znając je dla prostszych splotów, co zapiszemy symbolicznie:
\begin{comment}
\begin{equation}
    \bracket{\MediumMinusCrossing} = A \bracket{\MediumAlphaSmoothing} + B \bracket{\MediumBetaSmoothing}
\end{equation}
\end{comment}
Zależy nam też na tym, by móc dodać do splotu trywialną składową: $\langle L \cup \SmallUnknot \rangle = C \langle L \rangle$.
Prosty rachunek pokazuje wpływ drugiego ruchu Reidemeistera na klamrę:
\begin{comment}
\begin{equation}
    \bracket{\MediumKauffmanReidemeisterTwoA}
    = (A^2 + ABC + B^2) \bracket{\MediumBetaSmoothing} + BA \bracket{\MediumAlphaSmoothing}
    \stackrel{?}{=} \bracket{\MediumAlphaSmoothing}.
\end{equation}
\end{comment}

Aby zachodziła ostatnia równość wystarczy przyjąć $B = A^{-1}$, co wymusza na nas wartość trzeciego parametru: $C = -A^2 - A^{-2}$.
W ten sposób odkryliśmy definicję.

\begin{definition}[klamra Kauffmana]
    \label{def:kauffman_bracket}
    Wielomian Laurenta $\bracket{D}$ dla diagramu splotu $D$ zmiennej $A$,
    który jest niezmienniczy ze względu na gładkie deformacje diagramu,
    a~przy tym spełnia trzy poniższe aksjomaty:
\begin{comment}
    \begin{align}
        \bracket{\MediumUnknot} & = 1
        \label{eqn:kauffman_axiom_1}%
        \\
        \bracket{\MediumMinusCrossing} & =
        A \bracket{\MediumAlphaSmoothing} +
        A^{-1} \bracket{\MediumBetaSmoothing}
        \label{eqn:kauffman_axiom_2}%
        \\
        \bracket{D \sqcup \MediumUnknot} & =
        (-A^{-2} - A^2) \bracket{D}
        \label{eqn:kauffman_axiom_3}%
    \end{align}
\end{comment}
    nazywamy klamrą Kauffmana.
\end{definition}

Drugi aksjomat jest wariacją na temat relacji kłębiastej.

\begin{lemma}
    Klamra Kauffmana każdego diagramu wyznacza się w~skończenie wielu krokach.
\end{lemma}

\begin{proof}
    Najprościej dowieść tego indukcyjnie, ze względu na liczbę skrzyżowań na diagramie splotu.
    Baza indukcji to przypadek zero skrzyżowań, czyli niesplotów.
    Zauważmy, że ostatni (i później pierwszy) aksjomat pozwala wyznaczyć wartość klamry Kauffmana dla każdego niesplotu w tylu krokach, ile ogniw ma niesplot.

    Pozostał krok indukcyjny.
    Załóżmy, że wyznaczyliśmy już wartości klamry dla każdego diagramu o $n$ skrzyżowaniach i chcemy ją obliczyć dla kolejnego splotu z diagramem o~$n + 1$ skrzyżowaniach.
    Pozwala na to drugi aksjomat, usuwający jedno ze skrzyżowań.
\end{proof}

Przedstawimy teraz wpływ ruchów Reidemeistera na nasz nowy wielomian.

\begin{lemma}
    Drugi i~trzeci ruch Reidemeistera nie ma wpływu na klamrę Kauffmana,
    pierwszy ruch zmienia ją zgodnie z~regułą:
\begin{comment}
    \begin{equation}
        \bracket{\MediumReidemeisterOneLeft} = -A^{-3} \bracket{\,\MediumReidemeisterOneStraight\,}.
    \end{equation}
\end{comment}
\end{lemma}

\begin{proof}
Pierwszy ruch Reidemeistera:
\begin{comment}
\begin{align}
    \bracket{\MediumReidemeisterOneLeft} & \stackrel{K2}{=} A
    \bracket{\MediumReidemeisterOneSmoothA} +
    A^{-1} \bracket{\MediumReidemeisterOneSmoothB} \\ & \stackrel{K3}{=}
    A \bracket{\MediumReidemeisterOneStraight} +
    A^{-1}(-A^{-2}-A^2) \bracket{\MediumReidemeisterOneStraight} =
    -A^{-3}\bracket{\MediumReidemeisterOneStraight}
\end{align}
\end{comment}

Dla drugiego ruchu:
\begin{comment}
\begin{align}
    \bracket{\MediumKauffmanReidemeisterTwoA} & \stackrel{K2}{=}
    A \bracket{\MediumKauffmanReidemeisterTwoB} +
    A^{-1} \bracket{\MediumKauffmanReidemeisterTwoC} \\ & \stackrel{K1}{=}
    -A^{-2} \bracket{\MediumBetaSmoothing} +
    A^{-1} \bracket{\MediumKauffmanReidemeisterTwoC} \\ & \stackrel{K2}{=}
    -A^{-2} \bracket{\MediumBetaSmoothing} +
    A^{-1}A \bracket{\MediumAlphaSmoothing} +
    A^{-1}A^{-1} \bracket{\MediumBetaSmoothing} \\ & =
    \bracket{\MediumAlphaSmoothing}
\end{align}
\end{comment}

Dla trzeciego ruchu:
\begin{comment}
\begin{align}
\bracket{\MediumKauffmanReidemeisterThreeA} & \stackrel{K2}{=}
A \bracket{\MediumKauffmanReidemeisterThreeB} +
A^{-1} \bracket{\MediumKauffmanReidemeisterThreeC} \stackrel{R2}{=}
A \bracket{\MediumKauffmanReidemeisterThreeD} +
A^{-1} \bracket{\MediumKauffmanReidemeisterThreeE} \\ & \stackrel{R2}{=}
A \bracket{\MediumKauffmanReidemeisterThreeFlippedB} +
A^{-1} \bracket{\MediumKauffmanReidemeisterThreeFlippedC} \stackrel{K2}{=}
\bracket{\MediumKauffmanReidemeisterThreeFlippedA},
\end{align}
\end{comment}
korzystaliśmy tu z~własności drugiego ruchu.
\end{proof}

\begin{corollary}
    Rozpiętość klamry Kauffmana jest niezmiennikiem węzłów.
\end{corollary}

Klamra Kauffmana nie jest niezmiennikiem węzłów ze względu na I ruch Reidemeistera.
Jeżeli przypomnimy sobie, że na mocy lematu \ref{lem:writhe_reidemeister} spin także nie jest niezmiennikiem węzłów, odkryjemy ,,trik Kauffmana'': niedoskonałości tych dwóch obiektów znoszą się wzajemnie.
\index{trik Kauffmana}%
\index{spin}%

\begin{definition}
\label{def:jones_polynomial}%
    Niech $L$ będzie zorientowanym splotem.
    Wielomian Laurenta $\jones(L) \in \Z[t^{\pm 1/2}]$ określony przez
    \begin{equation}
        \jones(L)=\left[(-A)^{-3w(D)} \bracket{D}\right]_{t^{1/2}=A^{-2}},
    \end{equation}
    gdzie $D$ to dowolny diagram dla $L$, nazywamy wielomianem Jonesa.
\end{definition}

Sama klamra odegrała ważną rolę podczas unifikacji wielomianu Jonesa oraz innych niezmienników kwantowych.
W szczególności pozwoliła na uogólnienie go do niezmiennika 3-rozmaitości.

\begin{proposition}
    Wielomian Jonesa jest niezmiennikiem zorientowanych splotów.
\end{proposition}

\begin{proof}
    %Skorzystamy z~tego, że indeks zaczepienia jest niezmiennikiem.
    Wystarczy pokazać niezmienniczość $(-A)^{-3w(D)}\langle D\rangle$ na ruchy Reidemeistera.

    Niech
\begin{comment}
    \begin{equation}
        D_1 = \LargeReidemeisterOneLeft,
        \quad\quad\quad
        D_2 = \LargeReidemeisterOneStraight
    \end{equation}
\end{comment}
    Jak zauważyliśmy już wcześniej, II i III ruch nie zmienia ani spinu, ani klamry Kauffmana.
    Pozostało sprawdzić I ruch.
    Mamy:
    \begin{equation}
        (-A)^{-3 w\left(D_1\right)} \bracket{D_1} =
        (-A)^{-3 w\left(D_2\right) + 3} (-A)^{-3}\bracket{D_2} =
        (-A)^{-3 w\left(D_2\right)} \bracket{D_2},
    \end{equation}
    co kończy dowód.
\end{proof}

Zazwyczaj, ale nie zawsze, wielomian Jonesa lepiej radzi sobie z odróżnianiem od siebie splotów.
Zaczniemy od wyznaczenia bezpośrednio z definicji, jakie są wielomiany Jonesa niesplotów.
Dla porównania, wielomian Alexandera wszystkich splotów rozszczepialnych jest taki sam (stwierdzenie \ref{prp:alexander_unlinks}).

\begin{proposition}
    \label{prp:jones_trivial_link}
    Wielomianem Jonesa splotu trywialnego o $n$ ogniwach jest
    \begin{equation}
        \jones(K_n) = \left(-\sqrt{t} - \frac{1}{\sqrt {t}}\right)^{n-1}.
    \end{equation}
\end{proposition}

Co więcej, wielomian Jonesa odróżnia od siebie dowolne dwa węzły pierwsze o~co najwyżej 9 skrzyżowaniach.
Dalej występują już kolizje, oto pełna ich lista do 10 skrzyżowań:
$5_{1}$ -- $10_{132}$,
$8_{8}$ -- $10_{129}$,
$8_{16}$ -- $10_{156}$,
$10_{22}$ -- $10_{35}$,
$10_{25}$ -- $10_{56}$,
$10_{40}$ -- $10_{103}$,
$10_{41}$ -- $10_{94}$,
$10_{43}$ -- $10_{91}$,
$10_{59}$ -- $10_{106}$,
$10_{60}$ -- $10_{86}$,
$10_{71}$ -- $10_{104}$,
$10_{73}$ -- $10_{83}$,
$10_{81}$ -- $10_{109}$,
$10_{137}$ -- $10_{155}$.
Jones wiedział, że wielomianowe niezmienniki nie radzą sobie z~odróżnianiem od siebie mutantów, dlatego zapytał w~2000 roku, czy jego wielomian wykrywa niewęzły.
Pozostaje to otwartym problemem do dziś.

\begin{conjecture}
\index{hipoteza!o wielomianie Jonesa i niewęźle}%
\label{con:jones}%
    Niech $K$ będzie węzłem.
    Jeśli $\jones_K(t) \equiv 1$, to $K$ jest niewęzłem.
\end{conjecture}

Hipotezę zweryfikowano komputerowo dla węzłów o~małej liczbie skrzyżowań.
W latach dziewięćdziesiątych Hoste, Thistlethwaite, Weeks zrobili to dla węzłów spełniających $\operatorname{cr} \le 16$.
Wynik poprawiano: Dasbach, Hougardy w~1997 do $\operatorname{cr} = 17$; Yamada w~2000 do $\operatorname{cr} = 18$; wreszcie Tuzun, Sikora w~2016 do $\operatorname{cr} \le 22$.
Patrz \cite[s. 381]{ohtsuki02}.

% TODO: Argumentem przemawiającym za prawdziwością hipotezy jest twierdzenie ,,udowodnione'' przez Jørgena Andersena.
% TODO: \textbf{NIE Pokazał on, że rodzina okablowanych wielomianów Jonesa wykrywa niewęzeł.}
% TODO: Tutaj $n$-okablowanie węzła $K$ to $n$-komponentowy splot $K^n$, który powstaje z~$K$ po zamianie pojedynczej ,,żyły'' na $n$ równoległych żył.

Istnieją sploty o~trywialnym wielomianie Jonesa, jest ich nawet nieskończenie wiele, jak Eliahou, Kauffman i~Thistlethwaite pokazali w~pracy \cite{eliahou03}.

\begin{proposition}
    Niech $k \ge 2$ będzie liczbą naturalną.
    Istnieje nieskończenie wiele splotów pierwszych z $k$ ogniwami, których wielomian Jonesa nie odróżnia od niesplotu z $k$ ogniwami.

    Co więcej, można wymagać, by wszystkie te sploty były satelitami splotu Hopfa.
\index{splot!Hopfa}%
\end{proposition}

Niech $\jones$ będzie wielomianem Jonesa splotu $L$ o~$n$ składowych spójności.
Jego wartości w~niektórych pierwiastkach jedności są związane z~innymi niezmiennikami węzłów.
I tak przyjmując oznaczenie $\omega_k = \exp(2\pi i/k)$ mamy

\begin{proposition}
    \label{prp:jones_at_roots_of_unity}
    $\jones_L(\omega_3) = 1$.
\end{proposition}

\begin{proposition}
    $\jones_L(1) = (-2)^{n-1}$.
\end{proposition}

\begin{proof}
    Jak wkrótce się przekonamy, to proste wnioski z~relacji kłębiastej.
    Explicite wskazał je Jones w \cite[twierdzenie 14, 15]{jones85}.
\end{proof}

\begin{proposition}
    Pochodna w punkcie $t = 1$ znika: $\jones'_L(1) = 0$.
\end{proposition}

\begin{proof}
    Twierdzenie 16 w \cite{jones85}.
\end{proof}

\begin{proposition}
    $V_L(\omega_6) = \pm i^{n-1} \cdot (\sqrt 3i)^r$, gdzie $r$ jest rangą pierwszej grupy homologii podwójnego rozgałęzionego nakrycia $L$ nad $\Z_3$.
\end{proposition}

\begin{proof}
    Znak $\pm$ został wyznaczony przez Lipsona w \cite{lipson86}, praca ta zawiera też odsyłacz do wyprowadzenia reszty wzoru.
\end{proof}

\begin{proposition}
    Liczba trzy-kolorowań splotu $L$ wynosi $3|\jones_L(\omega_6)|^2$.
\end{proposition}

\begin{proof}
    Patrz \cite{przytycki98}.
\end{proof}

\begin{proposition}
    Jeśli $L$ jest właściwym splotem (indeks zaczepienia każdej składowej o~resztę splotu jest parzysty), to $\jones_L(i) = (-\sqrt 2)^{n-1}(-1)^{\operatorname{Arf} L}$.
    W przeciwnym razie $\jones_L(i) = 0$.
\end{proposition}

\begin{proof}
    Równość tę pokazał Murakami w~1986 roku (\cite{murakami86}).
\end{proof}

\begin{proposition}
    Niech $G$ będzie pierwszą grupą homologii podwójnego nakrycia $S^3$ rozgałęzionego nad składowymi.
    Jeśli $G$ jest torsyjna, to $\jones_L(-1) = |G|$.
    W przeciwnym razie $\jones_L(-1) = 0$.
\end{proposition}

\begin{proof}
    ???? % TODO
\end{proof}

Nie jest znana topologiczna interpretacja wielomianu Jonesa (którą posiada wielomian Alexandera) ani charakteryzacja poza warunkami koniecznymi z~pięciu faktów powyżej.

\begin{corollary}
    Niech $K$ będzie węzłem.
    Wtedy
    \begin{align}
        \jones(1) & = 1 \\
        \jones(-1) & = \pm \det K \\
        \jones(i) & = \begin{cases}
            1 & \text{dla } \alexander(-1) \equiv \pm 1 \mod 8 \\
            -1 & \text{w przeciwnym razie.}
        \end{cases}
    \end{align}
\end{corollary}

Poza powyżej opisanymi przypadkami, wartości wielomianu Jonesa nie można znaleźć w~czasie wielomianowym od ilości skrzyżowań na diagramie (jest to problem $\#P$-trudny).

% Czemu wielomian Jonesa jest wielomianem?
% Odpowiedniki wielomianu Jonesa dla węzłów w~3-rozmaitościach innych niż sfera $S^3$ nie są wielomianami, ale funkcjami z~pierwiastków jedności w~zbiór elementów całkowitch\footnote{algebraic integers} (jak podaje J. Roberts).

Dotychczas wyznaczyliśmy wielomian Jonesa jedynie dla splotów trywialnych (fakt \ref{prp:jones_trivial_link}).
Dlaczego?
Chociaż klamra Kauffmana to użyteczne narzędzie podczas dowodzenia różnych teoretycznych własności, niezbyt nadaje się do obliczeń, szczególnie ręcznych.
Na szczęście wtedy z pomocą przychodzi:

\begin{definition}
    Niech $L$ będzie zorientowanym splotem.
    Wielomian Laurenta $\jones_L(t) \in \Z[t^{\pm 1/2}]$, który spełnia relację kłębiastą
\index{relacja kłębiasta}%
    \begin{equation}
        t^{-1} \jones(L_+) - t\jones(L_-) + (t^{-1/2} - t^{1/2}) \jones(L_0) = 0
    \end{equation}
    z warunkiem brzegowym $\jones(\SmallUnknot) = 1$, nazywamy wielomianem Jonesa.
\end{definition}

\begin{proof}
Niech
\begin{comment}
\begin{equation}
    L_+ = \MediumMinusCrossing
    \quad\quad
    L_- = \MediumPlusCrossing
    \quad\quad
    L_0 = \MediumAlphaSmoothing
    \quad\quad
    L_\infty = \MediumBetaSmoothing
\end{equation}
\end{comment}
% TODO: zdefiniować je raz, a dobrze.
% ack -l 'L_\+'
% src/30-polynomials/alexander.tex
% src/30-polynomials/jones-kauffman.tex
% src/30-polynomials/blmho.tex
% src/30-polynomials/homfly.tex
% src/00-meta-latex/diagrams.tex
(oznaczenia te są standardowe i pozwalają oszczędzić trochę miejsca).
NIE SĄ - KOLIZJA OZNACZEŃ?Wyraźmy wielomian Jonesa przez klamrę Kauffmana i~spin.
Chcemy pokazać, że
\begin{align}
    & A^{4}(-A)^{-3w(L_+)}\bracket{L_+} \\
    - & A^{-4}(-A)^{-3w(L_-)}\bracket{L_0} \\
    + & (A^2-A^{-2})(-A)^{-3w(L_0)}\bracket{L_0} = 0.
\end{align}

Ale $w(L_\pm) = w(L_0)\pm 1$, zatem to jest równoważne z

\begin{equation}
    -A \bracket{L_+} +
    A^{-1} \bracket{L_-} +
    (A^2-A^{-2}) \bracket{L_0} =0.
\end{equation}

Z definicji klamry Kauffmana wnioskujemy, że

\begin{equation}
    \begin{cases}
        \bracket{L_+} = A\bracket{L_0} + A^{-1}\bracket{L_\infty} \\
        \bracket{L_-} = A\bracket{L_\infty} + A^{-1}\bracket{L_0}
    \end{cases}
\end{equation}

Pierwsze równanie przemnóżmy przez $A$, drugie przez $A^{-1}$, a~następnie dodajmy je do siebie.
Wtedy otrzymamy
\begin{equation}
    A\bracket{L_+} - A^{-1}\bracket{L_-} =
    A^2 (\bracket{L_0} - \bracket{L_\infty}),
\end{equation}
quod erat demonstrandum.
\end{proof}

% \subsection{Odwrotności, lustra i~sumy}
Wielomian Jonesa nie wykrywa orientacji splotu.

\begin{proposition}
    Niech $L$ będzie zorientowanym splotem.
    Wtedy $\jones(rL)=\jones(L)$.
\index{rewers}%
\end{proposition}

\begin{proof}
    Aby obliczyć wielomian rewersu, wykorzystujemy te same diagramy kłębiaste,
    jak dla zwykłego, a~przy tym nie zmieniamy znaku żadnego skrzyżowania.
\end{proof}

Ale czasami potrafi odróżnić splot od jego lustra:

\begin{proposition}
    Niech $L$ będzie zorientowanym splotem.
    Wtedy $\jones(mL)(t)=\jones(L)(t^{-1})$.
\index{lustro}%
\end{proposition}

\begin{proof}
    Zauważmy, że diagramy $L_-$ oraz $L_+$ są wzajemnymi lustrami.
    Dlatego każda relacja kłębiasta dla splotu postaci
    \begin{equation}
        t^{-1} \jones(L_+)(t) - t\jones(L_-)(t) + (t^{-1/2} - t^{1/2}) \jones(L_0)(t) = 0
    \end{equation}
    odpowiada pewnej relacji dla lustra splotu:
    \begin{equation}
        -t\jones(L_+)(t) + t^{-1} \jones(L_-)(t) + (t^{-1/2} - t^{1/2}) \jones(L_0)(t) = 0,
    \end{equation}
    co po zamianie zmiennych $t \mapsto t^{-1}$ i przemnożeniu przez $-1$ daje
    \begin{equation}
        -t^{-1} \jones(L_+)(t^{-1}) + t \jones(L_-)(t^{-1}) + (t^{1/2} - t^{-1/2}) \jones(L_0)(t^{-1}) = 0.
    \end{equation}

    Patrz też: Florian Gellert, Kombinatorische Invarianten, strona 12.
\end{proof}

\begin{corollary}
    \label{cor:joines_of_amphicheiral}
    Jeśli $K$ jest węzłem zwierciadlanym, to wielomian $\jones_K$ jest symetryczny.
\index{węzeł!zwierciadlany}
\end{corollary}

Implikacja odwrotna nie zachodzi na mocy wniosku \ref{cor:acheiral_signature}: węzeł $9_{42}$ ma symetryczny wielomian Jonesa, ale niezerową sygnaturę.
\index{sygnatura}%
Poniżej trzynastu skrzyżowań taka sytuacja ma miejsce dla dokładnie czternastu węzłów pierwszych.
% 9_42, 10_125, 11n_19, 11n_24, 11n_82, 12a_0669, 12a_1171, 12a_1179, 12a_1205, 12n_0362, 12n_0506, 12n_0562, 12n_0571, 12n_0821

Równość $\jones(mL)(t)=\jones(L)(t^{-1})$ nie jest spełniona dla trójlistnika, zatem ten nie jest równoważny ze swoim lustrem.
Wcześniej pokazał to z~dużo większym wysiłkiem Dehn, patrz przykład \ref{exm:trefoil_is_chiral}.

\begin{corollary}
    Wielomian Jonesa nie zależy od orientacji węzła.
    Nie jest to prawdą dla splotów.
\end{corollary}

\begin{proof}
    Każdy węzeł ma tylko dwie orientacje, splot może mieć ich $2^n$, gdzie $n$ to liczba składowych.
\end{proof}

\begin{proposition}
    \label{prp:jones_multiplicative_1}
    Niech $L_1, L_2$ będą zorientowanymi splotami.
    Wtedy
    \begin{equation}
        \jones(L_1 \sqcup L_2) = (-t^{1/2} - t^{-1/2}) \jones(L_1) \jones(L_2).
    \end{equation}
\end{proposition}

\begin{proof}
    Wybierzmy diagramy $D_1, D_2$ dla splotów $L_1, L_2$.
    Po podstawieniu $t^{1/2} = A^{-2}$ widzimy, że chcemy pokazać
    \begin{equation}
        (-A)^{-3w(D_1 \sqcup D_2)} \langle D_1 \sqcup D_2 \rangle
        =
        (-A^2 - A^{-2})(-A)^{-3(w(D_1) + w(D_2))} \langle D_1 \rangle \langle D_2 \rangle.
    \end{equation}

    Oczywiście $w(D_1 \sqcup D_2) = w(D_1) + w(D_2)$, więc wystarczy udowodnić, że
    \begin{equation}
        \langle D_1 \sqcup D_2 \rangle = (-A^2 - A^{-2}) \langle D_1 \rangle \langle D_2 \rangle.
    \end{equation}

    Oznaczmy przez $f_1(D_1)$, $f_2(D_1)$ odpowiednio lewą i~prawą stronę ostatniego równania.
    Są to wielomiany Laurenta, które zależą tylko od $D_1$.
    Aksjomaty Kauffmana pozwalają na pokazanie, że obie funkcje mają następujące własności:
    \begin{align}
        f_i(\SmallUnknot)            & = (-A^2 - A^{-2}) \langle D_2 \rangle \\
        f_i(D_1 \sqcup \SmallUnknot) & = (-A^2 - A^{-2}) f_i(D_1) \\
        f_i(\LittleRightCrossing)     & = A f_i(\LittleRightSmoothing) + A^{-1} f_i(\LittleLeftSmoothing).
    \end{align}
    Ponieważ powyższe tożsamości wystarczają do wyznaczenia wartości funkcji $f_i$ dla dowolnego diagramu $D_1$, dochodzimy do wniosku, że $f_1 \equiv f_2$.
    To kończy dowód.
\end{proof}

\begin{proposition}
\label{prp:jones_multiplicative_2}%
\index{relacja kłębiasta}%
    Niech $K_1, K_2$ będą zorientowanymi węzłami.
    Wtedy
    \begin{equation}
        \jones(K_1 \# K_2) = \jones(K_1) \jones(K_2).
    \end{equation}
\end{proposition}

\begin{proof}
    Rozpatrzmy sploty
\begin{comment}
    \begin{figure}[H]
    \centering
        %
        \begin{minipage}[b]{.3\linewidth}
            \[
                \MediumJonesShrapA
            \]
        \end{minipage}
        %
        \begin{minipage}[b]{.3\linewidth}
            \[
                \MediumJonesShrapB
            \]
        \end{minipage}
        %
        \begin{minipage}[b]{.3\linewidth}
            \[
                \MediumJonesShrapAB
            \]
        \end{minipage}
    \end{figure}
\end{comment}
    Relacja kłębiasta orzeka w tym przypadku, że
    \begin{equation}
        t^{-1} \jones(K_1 \# K_2) - t \jones(K_1 \# K_2) + (t^{-1/2} - t^{1/2}) \jones(K_1 \sqcup K_2) = 0.
    \end{equation}
    Ostatni składnik sumy można rozwinąć na mocy faktu \ref{prp:jones_multiplicative_1}.
    Po uporządkowaniu dostaniemy:
    \begin{equation}
        (t^{-1} - t) \jones(K_1 \# K_2) - (t^{-1} - t) \jones(K_1) \jones(K_2) = 0,
    \end{equation}
    a stąd widać już prawdziwość dowodzonej tezy.
\end{proof}

% Koniec sekcji Relacja kłębiasta
% Koniec podsekcji Wielomian Jonesa

\index{klamra Kauffmana|)}

% Koniec podsekcji Nawias Kauffmana