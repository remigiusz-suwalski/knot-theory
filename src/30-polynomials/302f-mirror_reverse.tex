
\subsection{Lustra, rewersy. Sumy}
Wielomian Jonesa nie wykrywa orientacji splotu:

\begin{proposition}
\index{rewers}%
    Niech $L$ będzie zorientowanym splotem.
    Wtedy $\jones(rL)=\jones(L)$.
\end{proposition}

\begin{proof}
    Aby obliczyć wielomian rewersu, wykorzystujemy te same diagramy kłębiaste,
    jak dla zwykłego, a~przy tym nie zmieniamy znaku żadnego skrzyżowania.
\end{proof}

Ale czasami potrafi odróżnić splot od jego lustra:

\begin{proposition}
\index{lustro}%
    Niech $L$ będzie zorientowanym splotem.
    Wtedy $\jones(mL)(t)=\jones(L)(t^{-1})$.
\end{proposition}

\begin{proof}
    Zauważmy, że diagramy $L_-$ oraz $L_+$ są wzajemnymi lustrami.
    Dlatego każda relacja kłębiasta dla splotu postaci
    \begin{equation}
        t^{-1} \jones(L_+)(t) - t\jones(L_-)(t) + (t^{-1/2} - t^{1/2}) \jones(L_0)(t) = 0
    \end{equation}
    odpowiada pewnej relacji dla lustra splotu:
    \begin{equation}
        -t\jones(L_+)(t) + t^{-1} \jones(L_-)(t) + (t^{-1/2} - t^{1/2}) \jones(L_0)(t) = 0,
    \end{equation}
    co po zamianie zmiennych $t \mapsto t^{-1}$ i przemnożeniu przez $-1$ daje
    \begin{equation}
        -t^{-1} \jones(L_+)(t^{-1}) + t \jones(L_-)(t^{-1}) + (t^{1/2} - t^{-1/2}) \jones(L_0)(t^{-1}) = 0.
    \end{equation}

    Patrz też: Florian Gellert, Kombinatorische Invarianten, strona 12.
\end{proof}

\begin{corollary}
\index{węzeł!zwierciadlany}
\label{cor:joines_of_amphicheiral}%
    Jeśli $K$ jest węzłem zwierciadlanym, to wielomian $\jones_K$ jest symetryczny.
\end{corollary}

Implikacja odwrotna nie zachodzi na mocy wniosku \ref{cor:acheiral_signature}: węzeł $9_{42}$ ma symetryczny wielomian Jonesa, ale niezerową sygnaturę.
\index{sygnatura}%
Poniżej trzynastu skrzyżowań taka sytuacja ma miejsce dla dokładnie czternastu węzłów pierwszych.
% 9_42, 10_125, 11n_19, 11n_24, 11n_82, 12a_0669, 12a_1171, 12a_1179, 12a_1205, 12n_0362, 12n_0506, 12n_0562, 12n_0571, 12n_0821
% TODO: uwzględnić kod programu, który to znalazł

Równość $\jones(mL)(t)=\jones(L)(t^{-1})$ nie jest spełniona dla trójlistnika, zatem ten nie jest równoważny ze swoim lustrem.
Wcześniej pokazał to z~dużo większym wysiłkiem Dehn, patrz przykład \ref{exm:trefoil_is_chiral}.
\index[persons]{Dehn, Max}%

\begin{corollary}
    Wielomian Jonesa nie zależy od orientacji węzła.
\end{corollary}

Nie jest to prawdą dla splotów.

\begin{proof}
    Każdy węzeł ma tylko dwie orientacje, splot może mieć ich $2^n$, gdzie $n$ to liczba składowych.
\end{proof}

\begin{proposition}
\index{suma niespójna}%
\label{prp:jones_multiplicative_1}%
    Niech $L_1, L_2$ będą zorientowanymi splotami.
    Wtedy
    \begin{equation}
        \jones(L_1 \sqcup L_2) = (-t^{1/2} - t^{-1/2}) \jones(L_1) \jones(L_2).
    \end{equation}
\end{proposition}

\begin{proof}
    Wybierzmy diagramy $D_1, D_2$ dla splotów $L_1, L_2$.
    Po podstawieniu $t^{1/2} = A^{-2}$ widzimy, że chcemy pokazać
    \begin{equation}
        (-A)^{-3w(D_1 \sqcup D_2)} \langle D_1 \sqcup D_2 \rangle
        =
        (-A^2 - A^{-2})(-A)^{-3(w(D_1) + w(D_2))} \langle D_1 \rangle \langle D_2 \rangle.
    \end{equation}

    Oczywiście $w(D_1 \sqcup D_2) = w(D_1) + w(D_2)$, więc wystarczy udowodnić, że
    \begin{equation}
        \langle D_1 \sqcup D_2 \rangle = (-A^2 - A^{-2}) \langle D_1 \rangle \langle D_2 \rangle.
    \end{equation}

    Oznaczmy przez $f_1(D_1)$, $f_2(D_1)$ odpowiednio lewą i~prawą stronę ostatniego równania.
    Są to wielomiany Laurenta, które zależą tylko od $D_1$.
    Aksjomaty Kauffmana pozwalają na pokazanie, że obie funkcje mają następujące własności:
    \begin{align}
        f_i(\SmallUnknot)            & = (-A^2 - A^{-2}) \langle D_2 \rangle \\
        f_i(D_1 \sqcup \SmallUnknot) & = (-A^2 - A^{-2}) f_i(D_1) \\
        f_i(\LittleRightCrossing)     & = A f_i(\LittleRightSmoothing) + A^{-1} f_i(\LittleLeftSmoothing).
    \end{align}
    Ponieważ powyższe tożsamości wystarczają do wyznaczenia wartości funkcji $f_i$ dla dowolnego diagramu $D_1$, dochodzimy do wniosku, że $f_1 \equiv f_2$.
    To kończy dowód.
\end{proof}

\begin{proposition}
\label{prp:jones_multiplicative_2}%
\index{relacja kłębiasta}%
\index{suma spójna}%
    Niech $K_1, K_2$ będą zorientowanymi węzłami.
    Wtedy
    \begin{equation}
        \jones(K_1 \# K_2) = \jones(K_1) \jones(K_2).
    \end{equation}
\end{proposition}

\begin{proof}
    Rozpatrzmy sploty
\begin{comment}
    \begin{figure}[H]
    \centering
        %
        \begin{minipage}[b]{.3\linewidth}
            \[
                \MediumJonesShrapA
            \]
        \end{minipage}
        %
        \begin{minipage}[b]{.3\linewidth}
            \[
                \MediumJonesShrapB
            \]
        \end{minipage}
        %
        \begin{minipage}[b]{.3\linewidth}
            \[
                \MediumJonesShrapAB
            \]
        \end{minipage}
    \end{figure}
\end{comment}
    Relacja kłębiasta orzeka w tym przypadku, że
    \begin{equation}
        t^{-1} \jones(K_1 \# K_2) - t \jones(K_1 \# K_2) + (t^{-1/2} - t^{1/2}) \jones(K_1 \sqcup K_2) = 0.
    \end{equation}
    Ostatni składnik sumy można rozwinąć na mocy faktu \ref{prp:jones_multiplicative_1}.
    Po uporządkowaniu dostaniemy:
    \begin{equation}
        (t^{-1} - t) \jones(K_1 \# K_2) - (t^{-1} - t) \jones(K_1) \jones(K_2) = 0,
    \end{equation}
    a stąd widać już prawdziwość dowodzonej tezy.
\end{proof}

