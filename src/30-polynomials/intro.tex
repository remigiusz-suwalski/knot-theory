Większość poznanych dotychczas niezmienników przyjmowała rzeczywiste wartości, teraz poszerzymy nasz arsenał o trzy klasyczne wielomiany.
Każdy z nich wywodzi się z innego działu matematyki -- \emph{wielomian Alexandera} z homologii pewnej przestrzeni nakryciowej, \emph{Jonesa}: z algebr operatorowych/mechaniki statystycznej, zaś \emph{HOMFLY-PT} stanowi ich dość naturalne uogólnienie.
Atrakcyjnym wprowadzeniem do tematu jest przygotowana przez matematyków niemieckich (a przez to dostępna tylko w ich języku) praca \cite{gellert09}.
Pierwotnymi artykułami były \cite{alexander28}, \cite{jones85} oraz \cite{homfly85}, wszystkie należą do przełomowych w kombinatorycznej teorii węzłów.