\subsection{Wielomian Jonesa} % (fold)
\label{sub:jones}
\begin{definition}
	\index{wielomian!Jonesa}
	\emph{Wielomian Jonesa} zorientowanego splotu to wielomian Laurenta $V(L)\in\Z[t^{1/2},t^{-1/2}]$ określony przez
	\[
		V(L)=\left[(-A)^{-3w(D)} \bracket{D}\right]_{t^{1/2}=A^{-2}},
	\]
	gdzie $D$ to dowolny diagram dla $L$.
\end{definition}

Połączenie \emph{writhe} z nawiasem nazywamy ,,trikiem Kauffmana''.

\begin{proposition}
	Wielomian Jonesa jest niezmiennikiem zorientowanych splotów.
\end{proposition}

\begin{proof}
	%Skorzystamy z tego, że indeks zaczepienia jest niezmiennikiem.
	Wystarczy pokazać niezmienniczość $(-A)^{-3w(D)}\langle D\rangle$ na ruchy Reidemeistera.
	Ale
	\[
		(-A)^{-3 w\left(\MalyreidemeisterIa\right)} \bracket{\MalyreidemeisterIa} =
		(-A)^{-3 w\left(\ \MalyreidemeisterIb\ \right)+3} (-A)^{-3}\bracket{\ \MalyreidemeisterIb\ } =
		(-A)^{-3 w\left(\ \MalyreidemeisterIb\ \right)}	\bracket{\,\MalyreidemeisterIb\,}. \qedhere
	\]
\end{proof}

Wielomian Jonesa jest naprawdę potężnym narzędziem.
Pozwala bowiem odróżnić dowolne dwa węzły pierwsze o co najwyżej dziewięciu skrzyżowaniach.

\begin{conjecture} \label{jones_conjecture}
	Nie istnieje nietrywialny węzeł, którego wielomian Jonesa nie odróżnia od niewęzła.
\end{conjecture}

Hipoteza została zweryfikowana metodami komputerowymi dla węzłów o mniej niż dziewiętnastu skrzyżowaniach.
Istnieją natomiast sploty o trywialnym wielomianie Jonesa, jest ich nawet nieskończenie wiele, jak Eliahou, Kauffman i Thistlethwaite pokazali w pracy \cite{eliahou03}.

Wartości wielomianu Jonesa w niektórych pierwiastkach jedności są związane z innymi niezmiennikami węzłów.
I tak przyjmując oznaczenie $\omega_n = \exp(2\pi i/n)$ mamy

\begin{proposition} \label{jones_sharp_p_hard}
	Niech $V$ będzie wielomianem Jonesa splotu $K$ o $n$ składowych spójności.
	Wtedy:
	\begin{itemize}
		\item $V(1) = (-2)^{n-1}$;
		\item $V(-1)$ jest rzędem pierwszej grupy homologii podwójnego nakrycia $S^3$ rozgałęzionego nad składowymi, jeśli jest torsyjna; $V(-1) = 0$ w przeciwnym razie;
		\item $V(\omega_3) = 1$;
		\item $V(i) = (-\sqrt 2)^{n-1}(-1)^{\operatorname{Arf} K}$ jeśli $K$ jest właściwym splotem (indeks zaczepienia każdej składowej o resztę splotu jest parzysty), $V(i) = 0$ w przeciwnym razie;
		\item wielkość $3|V(\omega_6)|^2$ opisuje liczbę trzy-kolorowań splotu.
	\end{itemize}
	Dla węzłów te zależności można uprościć: $V(1) = 1$; $V(-1) = \pm \det K$; $V(i) = 1$ jeśli $\Delta(-1)$ jest postaci $8k \pm 1$, w przeciwnym razie $V(i) = -1$.
\end{proposition}

Poza powyżej opisanymi przypadkami, wartości wielomianu Jonesa nie można znaleźć w czasie wielomianowym od ilości skrzyżowań na diagramie (jest to problem $\#P$-trudny).

% Koniec podsekcji Wielomian Jonesa
