
\subsection{Relacja kłębiasta (definicja druga)}

\begin{definition}[relacja kłębiasta]
\index{relacja kłębiasta}%
    Niech $L$ będzie zorientowanym splotem z ustalonym diagramem oraz skrzyżowaniem.
    Oznaczmy przez $L_+, L_-, L_0$ trzy diagramy splotów, które różnią się jedynie na małym obszarze wokół ustalonego skrzyżowania:
\begin{comment}
    \begin{figure}[H]
        \centering
        \begin{minipage}[b]{.3\linewidth}
            \centering
            \[\LargePlusCrossingArrows\]
            \subcaption{$L_+$}
        \end{minipage}
        \begin{minipage}[b]{.3\linewidth}
            \centering
            \[\LargeMinusCrossingArrows\]
            \subcaption{$L_-$}
        \end{minipage}
        \begin{minipage}[b]{.3\linewidth}
            \centering
            \[\LargeJustSmoothing\]
            \subcaption{$L_0$}
        \end{minipage}
    \end{figure}
\end{comment}
    Mówimy, że niezmiennik zorientowanych splotów $f$ spełnia relację kłębiastą, jeżeli wartości $f(L_+)$, $f(L_-)$ i $f(L_0)$ są związane pewnym wielomianowym równaniem, niezależnie od wyboru splotu $L$.
\end{definition}

Termin ,,skein'' (kłąb) wprowadził Conway około roku 1970, kontynuując tradycję używania słów, które kojarzą się ze sznurkami.
\index[persons]{Conway, John}%
Czasami mówi się o relacji motkowej, my nie zamierzamy używać tego synonimu.

\begin{definition}
    Niech $L$ będzie zorientowanym splotem.
    Wielomian Laurenta $\alexander_L(t) \in \Z[t^{\pm 1/2}]$, który spełnia relację kłębiastą
    \begin{equation}
        \alexander_{L_+}(t) - \alexander_{L_-}(t) - (t^{1/2} - t^{-1/2}) \alexander_{L_0}(t) = 0
    \end{equation}
    z warunkiem brzegowym $\alexander_{\SmallUnknot}(t) = 1$, nazywamy wielomianem Alexandera.
\end{definition}

Wzór ten, choć znany był Alexanderowi, nie zyskał przez wiele dekad uwagi matematyków.
\index[persons]{Alexander, James}%
Mogło tak być, gdyż w pracy \cite{alexander28} znalazł się on na samym końcu, pod nagłówkiem ,,twierdzenia różne''.
Na nowo odkrył go Conway: chcąc szybko liczyć wielomian Alexandera zaproponował, by reparametryzować go wzorem $\alexander(x^2) = \conway(x - 1/x)$.
Spełnia wtedy zależność
\begin{equation}
    \conway_{L_+}(x)- \conway_{L_-}(x) = x \conway_{L_0}(x).
\end{equation}

Relacja kłębiasta wystarcza do wyznaczenia $\alexander_L$ każdego splotu na mocy lematu \ref{lem:unknotting_well_defined}.

\begin{proposition}
    \label{prp:alexander_unlinks}
    Niech $L$ będzie splotem rozszczepialnym.
    Wtedy $\alexander_L(t) \equiv 0$.
\end{proposition}

\begin{proof}
    Skorzystamy z~relacji kłębiastej.
    Niech $L_0$ będzie splotem rozsczepialnym z~dwoma ogniwami.
    Wtedy węzły $L_+$ oraz $L_-$ powstałe przez dodanie skrzyżowania między ogniwami są tego samego typu, zatem
    \begin{equation}
        \alexander_{L_0} = \frac{\alexander_{L_+} - \alexander_{L_-}}{t^{1/2} - t^{-1/2}} = 0,
    \end{equation}
    a to chcieliśmy udowodnić.
\end{proof}

Implikacja w drugą stronę jest fałszywa.
Niech $\sigma_* = \sigma_{2} \sigma_{3}^{-2} \sigma_{2}$.
Domknięcie warkocza $\sigma_{1} \sigma_* \sigma_{1} \sigma_{3} \sigma_* \sigma_{1} \sigma_{3} \sigma_* \sigma_{3}$ nie jest rozszczepialne, ale jego wielomian Alexandera jest zerem.
\index{warkocz}
Warkocze poznamy w rozdziale piątym.

\begin{corollary}
    Wielomian Alexandera nie odróżnia od siebie niesplotów.
\end{corollary}

Wady tej nie posiada wielomian Jonesa.

% koniec podsekcji Relacja kłębiasta

