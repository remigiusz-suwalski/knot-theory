\section{Wielomian BLM/Ho} % (fold)
\label{sec:blm_ho}
Na przełomie grudnia i~stycznia 1984 i~1985 roku, K. Nowiński sugerował uwzględnienie czwartego diagramu w~relacji kłębiastej, poza $L_+$, $L_-$, $L_0$.
Nie udało się uzyskać niezmiennika splotów.
Sukces odnieśli, dla niezorientowanych splotów, mniej więcej rok później Brandt, Lickorish, Millett oraz Ho (w pracy \cite{brandt86}).

\begin{definition}
    \label{def:blm_ho}
    \index{wielomian!BLM/Ho}
    Wielomian BLM/Ho to niezmiennik zdefiniowany relacją kłębiastą,
    \begin{equation}
        Q_{L_+}(x) + Q_{L_-}(x) = x (Q_{L_0}(x) + Q_{L_\infty}(x)),
    \end{equation}
    z warunkiem początkowym $Q_{\LittleUnknot} = 1$.
\end{definition}

Jest multiplikatywny.
Nie odróżnia luster ani mutantów (patrz definicja \ref{def:mutant}) i~potrafi liczyć składowe:
jeśli jest ich $c$, to najmniejszą potęgą $x$ występującą w~wielomianie $Q_L$ jest $x^{1-c}$.
Jego stopień nie przekracza indeksu skrzyżowaniowego.

Brandt w 1986 roku podał jawne wzory na wartości wielomianu BLM/Ho w~czterech punktach:

Znane są jawne wartości wielomianu BLM/Ho w~czterech punktach
w~pracy \cite{brandt86}

\begin{proposition}
    Niech $L$ będzie splotem.
    Wtedy $Q_L(1) = 1$.
\end{proposition}

\begin{proposition}
    Niech $L$ będzie splotem.
    Wtedy $Q_L(2) = (\det L)^2$.
\end{proposition}

\begin{proposition}
    Niech $L$ będzie splotem o $c$ ogniwach.
    Wtedy $Q_L(-2) = (-1)^{c-1}$.
\end{proposition}

\begin{proposition}
    Niech $L$ będzie splotem z $d$-wymiarową homologią modulo $3$ dla jego dwukrotnego nakrycia.
    Wtedy $Q_L(-1) = 3^d$.
\end{proposition}

Kanenobu i~Sumi w~pracy \cite{kanenobu93} podali prosty test potrafiący czasem wykrywać, które węzły nie są dwumostowe:

\begin{proposition}
    \label{prop:blmho_twobridge}
    Jeśli $L$ jest węzłem dwumostowym, to
    \begin{equation}
        z Q_L(z) = 2 \jones_L(t) \jones_L (1-2z^{-1}+t^{-1}),
    \end{equation}
    gdzie $z = -t - t^{-1}$.
\end{proposition}

% Koniec sekcji Wielomian BLM/Ho
