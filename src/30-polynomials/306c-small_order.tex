
\subsection{Niezmienniki Wasiljewa małych rzędów}
Pokażemy najpierw, jakie są niezmienniki Wasiljewa rzędu 0, 1, 2.

\begin{proposition}
    Każdy niezmiennik Wasiljewa rzędu 0 jest funkcją stałą.
\end{proposition}

\begin{proof}
    Niech $v$ będzie niezmiennikiem rzędu zero i~znika na każdym osobliwym węźle o~jednym wierzchołku.
    Relacja kłębiasta mówi, że $v(\SmallPlusCrossing) = v(\SmallMinusCrossing)$, to znaczy odwrócenie dowolnego skrzyżowania nie zmienia wartości niezmiennika.

    Z lematu \ref{lem:unknotting_well_defined} wiemy jednak, że każdy węzeł można zmienić w niewęzeł odwracając niektóre skrzyżowania.
    Wynika stąd, że $v(K) = v(\SmallUnknot)$ dla każdego osobliwego węzła $K$, co należało udowodnić
\end{proof}

Każda zespolona krotność niezmiennika Wasiljewa znowu jest takim niezmiennikiem.
Niezmienniki rzędu zero tworzą przestrzeń liniową wymiaru 1 nad ciałem $\C$, zatem możemy krótko (choć nie dokładnie) powiedzieć, że jest jeden niezmiennik rzędu zero.

\begin{proposition}
    Nie istnieje niezmiennik Wasiljewa rzędu 1.
\end{proposition}

\begin{proposition}
    Istnieje dokładnie jeden niezmiennik Wasiljewa rzędów 2 i 3.
\end{proposition}

\begin{proof}
    Murasugi \cite[s. 315-320]{murasugi96} używa diagramów cięciw, wprowadzimy je później.
\end{proof}

