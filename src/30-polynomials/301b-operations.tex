
\subsection{Wielomian Alexandera a operacje na węzłach}

Wielomian Alexandera nie odróżnia luster i~rewersów od wyjściowych węzłów:

\begin{proposition}
\index{lustro}%
\index{rewers}%
    Niech $L$ będzie zorientowanym splotem.
    Wtedy $\alexander_{mL}(t) = \alexander_L(1/t) = \alexander_{rL}(t)$.
\end{proposition}

\begin{proof}
    Po odbiciu diagramu względem pionowej prostej skrzyżowanie z~definicji \ref{def:colouring_equation} też się odbija.
    Równanie związane z~nim zmienia się według schematu:
    \begin{equation}
        a + tc - ta - b = 0 \rightleftharpoons a + tb - ta - c = 0
    \end{equation}
    Pierwsze równanie z~$t$ zamienionym na $1/t$ staje się drugim równaniem przemnożonym przez $-1/t$.
    Dowód drugiej równości przebiega analogicznie.
\end{proof}

\begin{proposition}
\label{prp:alexander_multiplicative}%
    Niech $K_1, K_2$ będą zorientowanymi węzłami.
    Wtedy
    \begin{equation}
        \alexander_{K_1 \shrap K_2}(t) \equiv \alexander_{K_1}(t) \alexander_{K_2}(t)
    \end{equation}
\end{proposition}

\begin{proof}
    Wybierzmy poniższe diagramy dla węzłów $K_1$ oraz $K$:
\begin{comment}
    \[\begin{tikzpicture}[baseline=-0.65ex, scale=0.07]
    %\useasboundingbox (-5, -5) rectangle (5,5);
    \begin{knot}[clip width=7, end tolerance=1pt]
        \strand[thick] (-70, -20) rectangle (-30, 20);
        \strand[thick] (30, -20) rectangle ( 70, 20);
        \strand[thick] (-10, -10) [in=right, out=left] to (-25, 10);
        \strand[thick,-latex] (-25, 10) to (-30, 10);
        \strand[thick] (-30,-10) [in=left, out=right] to (-25, -10) to (-10, 10);
        \strand[thick] (-10, 10) [in=up, out=right] to (-5, 0) [in=right, out=down] to (-10, -10);

        % prawe strzalki
        \strand[thick] (30, 10) [in=right, out=left] to (25, 10) to (10, -10);
        \strand[thick,latex-] (30, -10) [in=right, out=left] to (25, -10) to (10, 10);
        \strand[thick] (10, 10) [in=up, out=left] to (5, 0) [in=left, out=down] to (10, -10);

        \node[darkblue] at (-50,10) [below] {$x_1,\ldots,x_{m-1}$};
        \node[red] at (-50,-10) [above] {$1,\ldots,m$};

        \node[darkblue] at (50,10) [below] {$y_1,\ldots,y_{n-1}$};
        \node[red] at (50,-10) [above] {$1,\ldots,n$};

        \node[darkblue] at (-20,-10)[below] {$x_m$};
        \node[darkblue] at (-10, 10)[above] {$x_0$};
        \node[darkblue] at (25,-10)[below] {$y_n$};
        \node[darkblue] at (10,10)[above] {$y_0$};
        \node[red] at ( 20,  0)[right]{$0$};
        \node[red] at (-20,  0)[left]{$0$};
    \end{knot}
    \end{tikzpicture}\]
\end{comment}
    Niech $A$ oraz $B$ oznaczają macierze otrzymane z~wielomianowych równań kolorujących dla $K_1$ oraz $K_2$ przez skreślenie skrajnie lewej kolumny i~górnego wiersza.
    Wtedy $\alexander_{K_1}(t) = \det A$ oraz $\alexander_{K_2}(t) = \det B$.
    Poniższy diagram przedstawia sumę $K_1 \shrap K_2$:

\begin{comment}
    \[\begin{tikzpicture}[baseline=-0.65ex, scale=0.07]
        %\useasboundingbox (-5, -5) rectangle (5,5);
        \begin{knot}[clip width=5, end tolerance=1pt]
            \strand[thick] (-70, -20) rectangle (-30, 20);
            \strand[thick] (30, -20) rectangle ( 70, 20);
            \strand[thick] (-10, -10) [in=right, out=left] to (-25, 10);
            \strand[thick,-latex] (-25, 10) to (-30, 10);
            \strand[thick] (-30,-10) [in=left, out=right] to (-25, -10) to (-10, 10);
            \strand[thick] (-10, -10) to (10, -10);
            \strand[thick] (-10, 10) to (10, 10);

            % prawe strzalki
            \strand[thick] (30, 10) [in=right, out=left] to (25, 10) to (10, -10);
            \strand[thick,latex-] (30, -10) [in=right, out=left] to (25, -10) to (10, 10);

            \node[darkblue] at (-50,10) [below] {$x_1,\ldots,x_{m-1}$};
            \node[red] at (-50,-10) [above] {$1,\ldots,m$};

            \node[darkblue] at (50,10) [below] {$y_1,\ldots,y_{n-1}$};
            \node[red] at (50,-10) [above] {$1,\ldots,n$};

            \node[darkblue] at (-20,-10)[below] {$x_m$};
            \node[darkblue] at (0, 10)[above] {$z$};
            \node[darkblue] at (25,-10)[below] {$y_n$};
            \node[darkblue] at (0, -10)[below] {$x_0 = y_0$};
            \node[red] at ( 20,  0)[right]{$0$};
            \node[red] at (-20,  0)[left]{$0$};
        \end{knot}
    \end{tikzpicture}\]
\end{comment}

    Uporządkujmy łuki na diagramie jako $x_0 = y_0$, $x_1, \ldots, x_m$, $y_1, \ldots, y_n$, $z$; skrzyżowania: $0, 1, \ldots, m$ (z $K_1$), $1, \ldots, n$ (z $K_2$), $\zeta$.
    Wielomianowe równanie kolorujące dla $K_1 \shrap K_2$ nad skrzyżowaniami $1, \ldots, m$ ($1, \ldots, n$) są takie same, jak przed dodaniem do siebie węzłów.
    Nad skrzyżowaniem $\zeta$ równanie orzeka, że $(1-t)y_0+t z-y_n=0$.

    Wynika stąd, że $\alexander_{K_1 \shrap K_2}(t)$ jest wyznacznikiem macierzy
    \begin{align*}
        M &= \left(\begin{array}{cc|cc|c}
            & & & & \\
            \multicolumn{2}{c|}{\smash{\raisebox{.5\normalbaselineskip}{$A$}}} & & \\
            \hline \\[-\normalbaselineskip]
            & & & & \\
            & & \multicolumn{2}{c|}{\smash{\raisebox{.5\normalbaselineskip}{$B$}}}\\ \hline
            & & & -1 & t
    \end{array}\right)
    \end{align*}

    Skreśliliśmy lewą kolumnę oraz górny wiersz.
    Zatem $\alexander_{K_1 \shrap K_2}(t) = t^?\alexander_{K_1}(t) \alexander_{K_2}(t)$, jeśli nie pomyliliśmy się w~obliczeniach.
\end{proof}

% koniec podsekcji Wielomian Alexandera a operacje na węzłach

