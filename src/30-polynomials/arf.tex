\section{Niezmiennik Cahita Arfa} % (fold)
\label{sub:arf}
Niezmiennik Arfa dla węzłów można zdefiniować na kilka sposobów, z których żaden jest istotnie lepszy od pozostałych.
Dwa węzły nazwiemy równoważnymi przez przejścia, jeśli są związane skończenie wieloma ,,przejściami'':
\[
	\begin{tikzpicture}[baseline=-0.65ex,scale=0.35]
	\begin{knot}[clip width=7]
		\strand[-latex, thick] (-2.5,-0.5) to (2.5,-0.5);
		\strand[-latex, thick] (2.5,0.5) to (-2.5,0.5);
		\strand[-latex, thick] (-0.5,-2.5) to (-0.5,2.5);
		\strand[-latex, thick] (0.5,2.5) to (0.5,-2.5);
	\end{knot}
	\end{tikzpicture}
	\cong
	\begin{tikzpicture}[baseline=-0.65ex,scale=0.35]
	\begin{knot}[clip width=7, flip crossing/.list={1,2,3,4}]
		\strand[-latex, thick] (-2.5,-0.5) to (2.5,-0.5);
		\strand[-latex, thick] (2.5,0.5) to (-2.5,0.5);
		\strand[-latex, thick] (-0.5,-2.5) to (-0.5,2.5);
		\strand[-latex, thick] (0.5,2.5) to (0.5,-2.5);
	\end{knot}
	\end{tikzpicture}
	\quad\mbox{albo}\quad
	\begin{tikzpicture}[baseline=-0.65ex,scale=0.35]
	\begin{knot}[clip width=7]
		\strand[-latex, thick] (-2.5,-0.5) to (2.5,-0.5);
		\strand[-latex, thick] (2.5,0.5) to (-2.5,0.5);
		\strand[latex-, thick] (-0.5,-2.5) to (-0.5,2.5);
		\strand[latex-, thick] (0.5,2.5) to (0.5,-2.5);
	\end{knot}
	\end{tikzpicture}
	\cong
	\begin{tikzpicture}[baseline=-0.65ex,scale=0.35]
	\begin{knot}[clip width=7, flip crossing/.list={1,2,3,4}]
		\strand[-latex, thick] (-2.5,-0.5) to (2.5,-0.5);
		\strand[-latex, thick] (2.5,0.5) to (-2.5,0.5);
		\strand[latex-, thick] (-0.5,-2.5) to (-0.5,2.5);
		\strand[latex-, thick] (0.5,2.5) to (0.5,-2.5);
	\end{knot}
	\end{tikzpicture}
\]

\index{niezmiennik Arfa} 
Każdy węzeł jest równoważny przez przejścia z niewęzłem (powiemy, że wyznacznik Arfa wynosi $0$) albo trójlistnikiem (że niezmiennik Arfa przyjmuje wartość $1$).

Takie podejście zaproponował Louis Kauffman.

\begin{proposition}[Murasugi]
	$\operatorname{Arf}(K) = 0$ wtedy i tylko wtedy, gdy $\Delta_K(-1) \equiv \pm 1 \mod 8$.
\end{proposition}

\begin{proposition}[Jones, 1985]
	$\operatorname{Arf}(K) = V_K(i)$.
\end{proposition}

\begin{proposition}[Robertello]
	Niech $ \Delta (t)=c_{0}+c_{1}t+\cdots +c_{n}t^{n}+\cdots +c_{0}t^{2n}$ będzie wielomianem Alexandera.
	Wtedy niezmiennik Arfa to $ c_{n-1}+c_{n-3}+\cdots +c_{r}\mod 2$, gdzie $r = 0$ dla nieparzystych $n$, $r = 1$ w przeciwnym razie.
\end{proposition}

\begin{proposition}
	Niech $(v_{ij})$ będzie macierzą Seiferta powstałą z krzywych genusu $g$, które reprezentują bazę pierwszej grupy homologii powierzchni.
	To oznacza, że macierz $V$ wymiaru $2g \times 2g$ ma następującą własność: różnica $V - V^t$ jest symplektyczna.
	Niezmiennik Arfa to
	\[
		\sum^g_{i=1}v_{2i-1,2i-1}v_{2i,2i} \pmod 2.
	\]
\end{proposition}

\begin{proposition}
	Niezmiennik Arfa jest $\shrap$-addytywny.
\end{proposition}

\begin{proposition}
	Niezmiennik Arfa znika na węzłach plastrowych.
	%It is additive under connected sum, and vanishes on slice knots, so is atv invariant. https://en.wikipedia.org/wiki/Link_concordance
\end{proposition}

% OEIS http://oeis.org/A131433, ...1434
% Koniec sekcji Niezmiennik Cahita Arfa
