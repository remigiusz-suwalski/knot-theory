% strona czwarta

\thispagestyle{empty}
\begin{figure}[H]
\begin{minipage}[b]{.48\linewidth}
{\noindent Prof. Casimir Allard\\
Université Bordeaux I\\
351 Cours de la Libération\\
33400 Talence, Francja}
\end{minipage}
\begin{minipage}[b]{.48\linewidth}
{\noindent Adélaïde Gauthier\\
École polytechnique\\
Route de Saclay\\
91128 Palaiseau, Francja}
\end{minipage}
\end{figure}

{\noindent \textbf{Kategorie MSC 2020}\\57K10 (niskowymiarowa teoria węzłów),\\57K30 (topologia 3-rozmaitości)} \vspace{5mm}

{\noindent \textbf{Tytuł oryginału}\\La théorie combinatoire des næuds}
\vspace{5mm}

{\noindent \textbf{Z francuskiego tłumaczyła}\\Juliette Buis} 
\vspace{5mm}

{\noindent \textbf{Okładkę zaprojektował}\\Wulfgang Kot}
\vspace{5mm}

{\noindent \textbf{Zredagował}\\Radosław Jagodowy}
\vspace{5mm}

{\noindent \textbf{Zredagowała technicznie}\\Klara Chmiel}
\vspace{5mm}

{\noindent \textbf{Złożyli i połamali}\\Porte de Versailles, Paryż}
\vspace{5mm}

{\noindent \textbf{Korekty dokonali}\\Jerzy Maślanka, Zuzanna Szpinak}

\vfill

{\noindent Copyleft by Antykwariat Czarnoksięski, Gorzów Wielkopolski 2022.
Książka, a także każda jej część, mogą być przedrukowywane oraz w jakikolwiek inny sposób reprodukowane czy powielane mechanicznie, fotooptycznie, zapisywane elektronicznie lub magnetycznie, oraz odczytywane w środkach publicznego przekazu bez pisemnej zgody wydawcy.}

\vspace{5mm}

{\noindent Przygotowano w systemie \TeX, wydrukowano na siarczystym papierze.}

% koniec strony czwartej