\tikzset{
    ->-/.style={decoration={markings, mark=at position .5 with {\arrow{>}}},postaction={decorate}},
    -<-/.style={decoration={markings, mark=at position .5 with {\arrow{<}}},postaction={decorate}},
    TIKZ_ARCH/.style ={
        draw=black,
        line join=miter,
        line cap=butt,
        miter limit=4.00,
        line width=0.2 mm
    },
}



%%%%%%%%%% UNKNOT START %%%%%%%%%%

\newcommand{\Unknot} {\begin{tikzpicture}[baseline=-0.65ex, scale=0.1]
    \begin{knot}[clip width=5, end tolerance=1pt]
        \strand[semithick] (0, 0) circle (5);
    \end{knot}
\end{tikzpicture}}

\newcommand{\MediumUnknot} {\begin{tikzpicture}[baseline=-0.65ex, scale=0.06]
    \useasboundingbox (-7, -7) rectangle (7, 7);
    \begin{knot}[clip width=5, end tolerance=1pt]
        \strand[semithick] (0, 0) circle (5);
    \end{knot}
\end{tikzpicture}}

\newcommand{\LittleUnknot} {\begin{tikzpicture}[baseline=-0.65ex, scale=0.03]
    \begin{knot}[clip width=5, end tolerance=1pt]
        \strand[semithick] (0, 0) circle (5);
    \end{knot}
\end{tikzpicture}}

%%%%%%%%%% UNKNOT END %%%%%%%%%%



%%%%%%%%%% LEFT/RIGHT, BIG/SMALL START %%%%%%%%%%

% \LeftCrossing - for example in crossing sign definition, where nothing besides the crossing is shown
% \MediumLeftCrossing - for skein relations
% \LittleLeftCrossing - for inline usage

\newcommand{\LeftCrossing} {\begin{tikzpicture}[baseline=-0.65ex, scale=0.1]
    \begin{knot}[clip width=5, end tolerance=1pt]
        \strand[thick] (-5, -5) to (5, 5);
        \strand[thick] (-5, 5) to (5, -5);
    \end{knot}
\end{tikzpicture}}

\newcommand{\RightCrossing} {\begin{tikzpicture}[baseline=-0.65ex, scale=0.1]
    \begin{knot}[clip width=5, end tolerance=1pt, flip crossing/.list={1}]
        \strand[thick] (-5, -5) to (5, 5);
        \strand[thick] (-5, 5) to (5, -5);
    \end{knot}
\end{tikzpicture}}

\newcommand{\MediumLeftCrossing} {\begin{tikzpicture}[baseline=-0.65ex, scale=0.06]
    \useasboundingbox (-7, -7) rectangle (7, 7);
    \begin{knot}[clip width=5, end tolerance=1pt]
        \strand[thick] (-5, -5) to (5, 5);
        \strand[thick] (-5, 5) to (5, -5);
    \end{knot}
\end{tikzpicture}}

% used in proof that 2nd Reidemeister move does not change Kauffman bracket
\newcommand{\MediumLeftCrossingLow} {\begin{tikzpicture}[baseline=-0.65ex, scale=0.06]
    \useasboundingbox (-7, -7) rectangle (7, 7);
    \begin{knot}[clip width=5, end tolerance=1pt]
        \strand[thick] (5, -5) .. controls (5, -2) and (-5, -2) .. (-5, 0);
        \strand[thick] (5, 5) to (5, 0);
        \strand[thick] (-5, -5) .. controls (-5, -2) and (5, -2) .. (5, 0);
        \strand[thick] (-5, 5) to (-5, 0);
    \end{knot}
\end{tikzpicture}}

\newcommand{\MediumRightCrossing} {\begin{tikzpicture}[baseline=-0.65ex, scale=0.06]
    \useasboundingbox (-7, -7) rectangle (7, 7);
    \begin{knot}[clip width=5, end tolerance=1pt, flip crossing/.list={1}]
        \strand[thick] (-5, -5) to (5, 5);
        \strand[thick] (-5, 5) to (5, -5);
    \end{knot}
\end{tikzpicture}}

\newcommand{\LittleLeftCrossing} {\begin{tikzpicture}[baseline=-0.65ex, scale=0.03]
    \useasboundingbox (-7, -5) rectangle (7, 5);
    \begin{knot}[clip width=5, end tolerance=1pt]
        \strand[semithick] (-5, -5) to (5, 5);
        \strand[semithick] (-5, 5) to (5, -5);
    \end{knot}
    \end{tikzpicture}
}

\newcommand{\LittleRightCrossing} {\begin{tikzpicture}[baseline=-0.65ex, scale=0.03]
    \useasboundingbox (-7, -5) rectangle (7, 5);
    \begin{knot}[clip width=5, end tolerance=1pt, flip crossing/.list={1}]
        \strand[semithick] (-5, -5) to (5, 5);
        \strand[semithick] (-5, 5) to (5, -5);
    \end{knot}
\end{tikzpicture}}

%%%%%%%%%% LEFT/RIGHT, BIG/SMALL  END %%%%%%%%%%



%%%%%%%%%% LEFT/RIGHT SMOOTHING, BIG/SMALL START %%%%%%%%%%

\newcommand{\LeftCrossSmoothing} {\begin{tikzpicture}[baseline=-0.65ex, scale=0.1]
    \begin{knot}[clip width=5, end tolerance=1pt]
        \strand[thick] (-5, -5) [in=135, out=45] to (5, -5);
        \strand[thick] (-5, 5) [in=-135, out=-45] to (5, 5);
    \end{knot}
\end{tikzpicture}}

\newcommand{\RightCrossSmoothing} {\begin{tikzpicture}[baseline=-0.65ex, scale=0.1]
    \begin{knot}[clip width=5, end tolerance=1pt]
        \strand[thick] (-4, -5) to [out=45, in=-45] (-4, 5);
        \strand[thick] (4, -5) to [out=135, in=-135] (4, 5);
    \end{knot}
\end{tikzpicture}}

\newcommand{\MediumLeftSmoothing} {\begin{tikzpicture}[baseline=-0.65ex,scale=0.06]
    \useasboundingbox (-7, -7) rectangle (7, 7);
    \begin{knot}[clip width=5, end tolerance=1pt]
        \strand[thick] (-5, -5) [in=135, out=45] to (5, -5);
        \strand[thick] (-5, 5) [in=-135, out=-45] to (5, 5);
    \end{knot}
\end{tikzpicture}}

\newcommand{\MediumRightSmoothing} {\begin{tikzpicture}[baseline=-0.65ex,scale=0.06]
    \useasboundingbox (-7, -7) rectangle (7, 7);
    \begin{knot}[clip width=5, end tolerance=1pt]
        \strand[thick] (-4, -5) to [out=45, in=-45] (-4, 5);
        \strand[thick] (4, -5) to [out=135, in=-135] (4, 5);
    \end{knot}
\end{tikzpicture}}

\newcommand{\LittleLeftSmoothing} {\begin{tikzpicture}[baseline=-0.65ex,scale=0.03]
    \useasboundingbox (-7, -5) rectangle (7, 5);
    \begin{knot}[clip width=5, end tolerance=1pt]
        \strand[semithick] (-5, -5) [in=135, out=45] to (5, -5);
        \strand[semithick] (-5, 5) [in=-135, out=-45] to (5, 5);
    \end{knot}
\end{tikzpicture}}

\newcommand{\LittleRightSmoothing} {\begin{tikzpicture}[baseline=-0.65ex,scale=0.03]
    \useasboundingbox (-7, -5) rectangle (7, 5);
    \begin{knot}[clip width=5, end tolerance=1pt]
        \strand[semithick] (-4, -5) to [out=45, in=-45] (-4, 5);
        \strand[semithick] (4, -5) to [out=135, in=-135] (4, 5);
    \end{knot}
\end{tikzpicture}}

%%%%%%%%%% LEFT/RIGHT SMOOTHING, BIG/SMALL END %%%%%%%%%%



%%%%%%%%%% REIDEMEISTER-1 START %%%%%%%%%%

% for skein relation defining Kauffman F polynomial start
\newcommand{\MediumReidemeisterIaRight} {\begin{tikzpicture}[baseline=-0.65ex,scale=0.06]
    \useasboundingbox (-7, -7) rectangle (7, 7);
    \begin{knot}[clip width=5, end tolerance=1pt]
        \strand[thick] (-5, -5) [in=left, out=60] to  (3, 5)  [in=up, out=right]   to (5, 0);
        \strand[thick] (-5, 5)  [in=left, out=-60] to (3, -5) [in=down, out=right] to (5, 0);
    \end{knot}
\end{tikzpicture}}

\newcommand{\MediumReidemeisterIaLeft} {\begin{tikzpicture}[baseline=-0.65ex,scale=0.06]
    \useasboundingbox (-7, -7) rectangle (7, 7);
    \begin{knot}[clip width=5, end tolerance=1pt]
        \strand[thick] (-5, 5)  [in=left, out=-60] to (3, -5) [in=down, out=right] to (5, 0);
        \strand[thick] (-5, -5) [in=left, out=60]  to (3, 5)  [in=up, out=right]   to (5, 0);
    \end{knot}
\end{tikzpicture}}

\newcommand{\MediumReidemeisterIaLeftFirstSmoothering} {\begin{tikzpicture}[baseline=-0.65ex,scale=0.06]
    \useasboundingbox (-7, -7) rectangle (7, 7);
    \begin{knot}[clip width=5, end tolerance=1pt]
        \strand[thick] (-5, 5)  [in=left, out=-60] to (-2, 1.5)  [in=left, out=right] to (3, 5)  [in=up]   to (5, 0);
        \strand[thick] (-5, -5) [in=left, out=60] to  (-2, -1.5) [in=left, out=right] to (3, -5) [in=down] to (5, 0);
    \end{knot}
\end{tikzpicture}}

\newcommand{\MediumReidemeisterIaLeftSecondSmoothering} {\begin{tikzpicture}[baseline=-0.65ex,scale=0.06]
    \useasboundingbox (-7, -7) rectangle (7, 7);
    \begin{knot}[clip width=5, end tolerance=1pt]
        \strand[thick] (-5, -5) [in=down, out=up] to (-3.5, 0) to (-5, 5);
        \strand[thick] (-1, 0) [in=left, out=up] to (3, 5) [in=up, out=right] to (5, 0);
        \strand[thick] (-1, 0) [in=left, out=down] to (3, -5) [in=down, out=right] to (5, 0);
    \end{knot}
\end{tikzpicture}}

\newcommand{\MediumReidemeisterIb} {\begin{tikzpicture}[baseline=-0.65ex,scale=0.06]
    \useasboundingbox (-2, -7) rectangle (2, 7);
    \begin{knot}[clip width=5, end tolerance=1pt]
        \strand[thick] (0, -6.65) to (0, 6);
    \end{knot}
\end{tikzpicture}}

%%%%%%%%%% REIDEMEISTER-1 END %%%%%%%%%%



%%%%%%%%%% REIDEMEISTER-2 START %%%%%%%%%%

\newcommand{\MediumReidemeisterIIa} {\begin{tikzpicture}[baseline=-0.65ex,scale=0.06]
    \useasboundingbox (-7, -7) rectangle (7, 7);
    \begin{knot}[clip width=5, end tolerance=1pt]
        \strand[thick] (5, -5) .. controls (5, -2) and (-5, -2) .. (-5, 0);
        \strand[thick] (5, 5) .. controls (5, 2) and (-5, 2) .. (-5, 0);
        \strand[thick] (-5, -5) .. controls (-5, -2) and (5, -2) .. (5, 0);
        \strand[thick] (-5, 5) .. controls (-5, 2) and (5, 2) .. (5, 0);
    \end{knot}
\end{tikzpicture}}

\newcommand{\MediumReidemeisterIIaHorizontal} {\begin{tikzpicture}[baseline=-0.65ex,scale=0.05]
    \useasboundingbox (-9.5, -7) rectangle (9.5, 7);
    \begin{knot}[clip width=5, end tolerance=1pt]
        \strand[thick] (-7.5, -5) .. controls (-3, -5) and (-3, 5) .. (0, 5);
        \strand[thick] (7.5, -5) .. controls (3, -5) and (3, 5) .. (0, 5);
        \strand[thick] (-7.5, 5) .. controls (-3, 5) and (-3, -5) .. (0, -5);
        \strand[thick] (7.5, 5) .. controls (3, 5) and (3, -5) .. (0, -5);
    \end{knot}
\end{tikzpicture}}

% used in proof...
\newcommand{\MediumReidemeisterIIaSmoothed} {\begin{tikzpicture}[baseline=-0.65ex,scale=0.06]
    \useasboundingbox (-7, -7) rectangle (7, 7);
    \begin{knot}[clip width=5, end tolerance=1pt]
        \strand[thick] (5, -5) .. controls (5, -3) and (-5, -3) .. (-5, -1);
        \strand[thick] (-5, -5) .. controls (-5, -3) and (5, -3) .. (5, -1);
        \strand[thick] (-5, -1) [in=left, out=up] to (0, 1) to [in=up, out=right] (5, -1);
        \strand[thick] (-5, 5) [in=left, out=down] to (0, 3) to [in=down, out=right] (5, 5);
    \end{knot}
\end{tikzpicture}}

%%%%%%%%%% REIDEMEISTER-2 END %%%%%%%%%%



%%%%%%%%%% REIDEMEISTER-3 START %%%%%%%%%%

\newcommand{\MediumReidemeisterIIIa} {\begin{tikzpicture}[baseline=-0.65ex,scale=0.06]
    \useasboundingbox (-7, -7) rectangle (7, 7);
    \begin{knot}[clip width=5, flip crossing/.list={1, 2, 3}, end tolerance=1pt]
        \strand[thick] (-5, -5) -- (5, 5);
        \strand[thick] (-5, 5) -- (5, -5);
        \strand[thick] (-5, 0) .. controls (-3, 0) and (-3, 5) .. (0, 5) .. controls (3, 5) and (3, 0) .. (5, 0);
    \end{knot}
\end{tikzpicture}}

\newcommand{\MediumReidemeisterIIIaFlipped} {\begin{tikzpicture}[baseline=-0.65ex,scale=0.06]
    \useasboundingbox (-7, -7) rectangle (7, 7);
    \begin{knot}[clip width=5, flip crossing/.list={1, 2, 3}, end tolerance=1pt]
        \strand[thick] (-5, -5) -- (5, 5);
        \strand[thick] (-5, 5) -- (5, -5);
        \strand[thick] (-5, 0) .. controls (-3, 0) and (-3, -5) .. (0, -5) .. controls (3, -5) and (3, 0) .. (5, 0);
    \end{knot}
\end{tikzpicture}}

\newcommand{\MediumReidemeisterIIIb} {\begin{tikzpicture}[baseline=-0.65ex,scale=0.06]
    \useasboundingbox (-7, -7) rectangle (7, 7);
    \begin{knot}[clip width=5, flip crossing/.list={1, 2, 3}, end tolerance=1pt]
        \strand[thick] (-5, 5) [in=-120, out=-60] to (5, 5);
        \strand[thick] (-5, -5) [in=120, out=60] to (5, -5);
        \strand[thick] (-5, 0) .. controls (-3, 0) and (-3, 5) .. (0, 5) .. controls (3, 5) and (3, 0) .. (5, 0);
    \end{knot}
\end{tikzpicture}}

\newcommand{\MediumReidemeisterIIIbFlipped} {\begin{tikzpicture}[baseline=-0.65ex,scale=0.06]
    \useasboundingbox (-7, -7) rectangle (7, 7);
    \begin{knot}[clip width=5, flip crossing/.list={1, 2, 3}, end tolerance=1pt]
        \strand[thick] (-5, 5) [in=-120, out=-60] to (5, 5);
        \strand[thick] (-5, -5) [in=120, out=60] to (5, -5);
        \strand[thick] (-5, 0) .. controls (-3, 0) and (-3, -5) .. (0, -5) .. controls (3, -5) and (3, 0) .. (5, 0);
    \end{knot}
\end{tikzpicture}}

\newcommand{\MediumReidemeisterIIIc} {\begin{tikzpicture}[baseline=-0.65ex,scale=0.06]
    \useasboundingbox (-7, -7) rectangle (7, 7);
    \begin{knot}[clip width=5, flip crossing/.list={1, 2, 3}, end tolerance=1pt]
        \strand[thick] (-5, -5) to [out=30, in=-30] (-5, 5);
        \strand[thick] (5, -5) to [out=150, in=-150] (5, 5);
        \strand[thick] (-6, 0) .. controls (-3, 0) and (-3, 5) .. (0, 5) .. controls (3, 5) and (3, 0) .. (6, 0);    
    \end{knot}
\end{tikzpicture}}

\newcommand{\MediumReidemeisterIIIcFlipped} {\begin{tikzpicture}[baseline=-0.65ex,scale=0.06]
    \useasboundingbox (-7, -7) rectangle (7, 7);
    \begin{knot}[clip width=5, flip crossing/.list={1, 2, 3}, end tolerance=1pt]
        \strand[thick] (-5, -5) to [out=30, in=-30] (-5, 5);
        \strand[thick] (5, -5) to [out=150, in=-150] (5, 5);
        \strand[thick] (-6, 0) .. controls (-3, 0) and (-3, -5) .. (0, -5) .. controls (3, -5) and (3, 0) .. (6, 0);    
    \end{knot}
\end{tikzpicture}}

\newcommand{\MediumReidemeisterIIId} {\begin{tikzpicture}[baseline=-0.65ex,scale=0.06]
    \useasboundingbox (-7, -7) rectangle (7, 7);
    \begin{knot}[clip width=5, flip crossing/.list={1, 2, 3}, end tolerance=1pt]
        \strand[thick] (-5, 5) [in=-120, out=-60] to (5, 5);
        \strand[thick] (-5, -5) [in=120, out=60] to (5, -5);
        \strand[thick] (-5, 0) to (5, 0);
    \end{knot}
\end{tikzpicture}}

\newcommand{\MediumReidemeisterIIIe} {\begin{tikzpicture}[baseline=-0.65ex,scale=0.06]
    \useasboundingbox (-7, -7) rectangle (7, 7);
    \begin{knot}[clip width=5, flip crossing/.list={1, 2, 3}, end tolerance=1pt]
        \strand[thick] (-5, -5) to [out=30, in=-30] (-5, 5);
        \strand[thick] (5, -5) to [out=150, in=-150] (5, 5);
        \strand[thick] (-6, 0) to (6, 0);
    \end{knot}
\end{tikzpicture}}

%%%%%%%%%% REIDEMEISTER-3 END %%%%%%%%%%



%%%%%%%%%% SKEIN START %%%%%%%%%%

\newcommand{\skeinplus} {\begin{tikzpicture}[baseline=-0.65ex,scale=0.1]
    \useasboundingbox (-7, -7) rectangle (7, 5);
    \begin{knot}[clip width=5, end tolerance=1pt]
        \strand[semithick, -latex] (-5, -5) to (5, 5);
        \strand[semithick,latex-] (-5, 5) to (5, -5);
        \node[darkblue] at (0, -4)[below] {$L_+$};
    \end{knot}
\end{tikzpicture}}

\newcommand{\MediumSkeinPlus} {\begin{tikzpicture}[baseline=-0.65ex,scale=0.06]
    \useasboundingbox (-7, -5) rectangle (7, 5);
    \begin{knot}[clip width=5, end tolerance=1pt]
        \strand[semithick, -Latex] (-5, -5) to (5, 5);
        \strand[semithick,Latex-] (-5, 5) to (5, -5);
    \end{knot}
\end{tikzpicture}}

\newcommand{\skeinminus} {\begin{tikzpicture}[baseline=-0.65ex,scale=0.1]
    \useasboundingbox (-7, -7) rectangle (7, 5);
    \begin{knot}[clip width=5, end tolerance=1pt, flip crossing/.list={1}]
        \strand[semithick, -latex] (-5, -5) to (5, 5);
        \strand[semithick,latex-] (-5, 5) to (5, -5);
        \node[darkblue] at (0, -4)[below] {$L_-$};
    \end{knot}
\end{tikzpicture}}

\newcommand{\MediumSkeinMinus} {\begin{tikzpicture}[baseline=-0.65ex,scale=0.06]
    \useasboundingbox (-7, -5) rectangle (7, 5);
    \begin{knot}[clip width=5, end tolerance=1pt, flip crossing/.list={1}]
        \strand[semithick, -Latex] (-5, -5) to (5, 5);
        \strand[semithick,Latex-] (-5, 5) to (5, -5);
    \end{knot}
\end{tikzpicture}}

\newcommand{\skeinzero} {\begin{tikzpicture}[baseline=-0.65ex,scale=0.1]
    \useasboundingbox (-7, -7) rectangle (7, 5);
    \begin{knot}[clip width=5, end tolerance=1pt, flip crossing/.list={1}]
        \strand[semithick, -latex] (-5, -5) to [out=45, in=-45] (-5, 5);
        \strand[semithick, -Latex] (5, -5) to [out=135, in=-135] (5, 5);
        \node[darkblue] at (0, -4)[below] {$L_0$};
    \end{knot}
\end{tikzpicture}}

%%%%%%%%%% SKEIN END %%%%%%%%%%



%%% VASSILIEV START %%%
\newcommand{\SingularCrossing} {\begin{tikzpicture}[baseline=-0.65ex,scale=0.06]
    \useasboundingbox (-7, -5) rectangle (7, 5);
    \begin{knot}[clip width=5, end tolerance=1pt]
        \strand[semithick] (-0.5, -0.5) to (-5, -5);
        \strand[semithick,-Latex] (-0.5, 0.5) to (-5, 5);
        \strand[semithick] (0.5, -0.5) to (5, -5);
        \strand[semithick,-Latex] (0.5, 0.5) to (5, 5);
        \draw[black,fill=black] (0,0) circle (1);
    \end{knot}
\end{tikzpicture}}
      
%%% VASSILIEV END